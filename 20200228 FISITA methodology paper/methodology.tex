%----------------------------------------------------------------------------------------
%	PACKAGES AND OTHER DOCUMENT CONFIGURATIONS
%----------------------------------------------------------------------------------------

\documentclass[twoside,twocolumn,9pt]{article}

\usepackage{blindtext} % Package to generate dummy text throughout this template 

%\usepackage[sc]{mathpazo} % Use the Palatino font
\usepackage{mathptmx}
\usepackage[T1]{fontenc} % Use 8-bit encoding that has 256 glyphs
\linespread{1.05} % Line spacing - Palatino needs more space between lines
\usepackage{microtype} % Slightly tweak font spacing for aesthetics

\usepackage{graphicx} 

\usepackage[english]{babel} % Language hyphenation and typographical rules

\usepackage[a4paper,left=0.71in,top=0.98in,right=0.71in,bottom=0.98in,columnsep=15pt]{geometry} % Document margins
\usepackage[hang, small,labelfont=bf,up,textfont=it,up]{caption} % Custom captions under/above floats in tables or figures
\usepackage{booktabs} % Horizontal rules in tables

\usepackage{enumitem} % Customized lists
\setlist[itemize]{noitemsep} % Make itemize lists more compact

\usepackage[runin]{abstract} % Allows abstract customization
\setlength{\abstitleskip}{-\parindent}
%\newcommand{\abstitlestyle}[1]{#1}
\renewcommand{\abstractnamefont}{\normalfont\bfseries\MakeUppercase} % Set the "Abstract" text to bold
\renewcommand{\abstracttextfont}{\normalfont} % Set the abstract itself to small italic text
\abslabeldelim{:}


\usepackage{titlesec} % Allows customization of titles
%\renewcommand\thesection{\Roman{section}} % Roman numerals for the sections
%\renewcommand\thesubsection{\roman{subsection}} % roman numerals for subsections
\titleformat{\section}[block]{\large\bfseries}{\thesection.}{1em}{} % Change the look of the section titles
\titleformat{\subsection}[block]{\large}{\thesubsection.}{1em}{} % Change the look of the section titles
\titleformat{\subsubsection}[block]{\normalsize}{\thesubsubsection.}{1em}{} % Change the look of the section titles

\usepackage{fancyhdr} % Headers and footers
\pagestyle{fancy} % All pages have headers and footers
\fancyhf{}% clear default for head and foot
\fancyhead[L]{\includegraphics[scale=0.11]{figures/FISITA.png}}
\fancyhead[R]{\thepage}
\fancyfoot[L]{Proceedings of the FISITA 2020 World Congress, Prague, 14 - 18 September 2020}
\renewcommand{\headrulewidth}{0pt}
\fancypagestyle{firstpage}{
	\fancyhf{}% clear default for head and foot
	\renewcommand{\headrulewidth}{0pt}
	\pagestyle{fancy}
	\lhead{\textbf{F2020-VES-014}}
	\rhead{\includegraphics[scale=0.15]{figures/FISITA.png}}
	\fancyfoot[L]{Proceedings of the FISITA 2020 World Congress, Prague, 14 - 18 September 2020}
}
\usepackage{xpatch}
\xapptocmd{\titlepage}{\thispagestyle{firstpage}}{}{}

\usepackage{titling} % Customizing the title section

%\usepackage{hyperref} % For hyperlinks in the PDF

\usepackage{authblk}

\usepackage{caption}
\DeclareCaptionLabelFormat{nospace}{#1#2}
\captionsetup[table]{labelfont=normal, textfont=normal,name=Table ,labelsep=period, justification=raggedright, singlelinecheck=off}
\captionsetup[figure]{labelfont=normal, textfont=normal,name=Figure ,labelsep=period, justification=raggedright, singlelinecheck=off}
\usepackage[utf8]{inputenc}   				 	%% utf8 support (required for biblatex)
\usepackage{silence}  							%% For filtering warnings
\usepackage[style=ieee,doi=false,isbn=false,url=false,date=year,backend=biber,maxbibnames=15,maxcitenames=2,mincitenames=1,uniquelist=false,uniquename=false,giveninits=true]{biblatex}
% Filter warnings issued by package biblatex starting with "Patching footnotes failed"
\WarningFilter{biblatex}{Patching footnotes failed}
%\renewcommand*{\bibfont}{\footnotesize}		%% Use this for papers
\renewcommand*{\bibfont}{\small}
\setlength{\biblabelsep}{\labelsep}
\bibliography{../bib}
\usepackage[fleqn]{amsmath}

\usepackage{xcolor}

%----------------------------------------------------------------------------------------
%	TITLE SECTION
%----------------------------------------------------------------------------------------

\providecommand{\keywords}[1]{\textbf{KEY WORDS:} #1}

\pretitle{\begin{center}\huge\normalfont\MakeUppercase} % Article title formatting
\posttitle{\end{center}\vskip 1em} % Article title closing formatting
\title{type the title of the paper here} % Article title

\author[1,2]{\large\bfseries Erwin de Gelder}
\author[1]{\large\bfseries Olaf Op den Camp}
\author[1,3]{\large\bfseries Jan-Pieter Paardekooper}
\author[2]{\large\bfseries Bart De Schutter}

\affil[1]{\normalsize\textit{TNO, Integrated Vehicle Safety, Helmond, The Netherlands (E-mail: erwin.degelder@tno.nl)}}
\affil[2]{\normalsize\textit{Delft University of Technology, Delft Center for Systems and Control, Delft, The Netherlands}}
\affil[3]{\normalsize\textit{Radboud University, Donders Institute for Brain, Cognition and Behaviour, Nijmegen, The Netherlands}}
\date{} % Leave empty to omit a date

\renewcommand{\maketitlehookd}{%
\begin{abstract}	
\noindent The development of Autonomous Vehicles (AVs) has made significant progress in the last years and it is expected that AVs will soon be introduced on our roads. An important aspect in the development of AVs is the assessment of their safety. As traditional methods for safety assessment of vehicles are not feasible to be performed for AVs within reasonable time and cost, other approaches need to be worked out. Among these, real-world scenario-based assessment is widely supported by many players in the automotive field. Scenario-based assessment allows for using virtual simulation tools in addition to physical tests, such as on a test track, proving ground, or public road.

We propose a framework for real-world scenario-based road-approval assessment considering three stakeholders: the applicant, the assessor, and the (road or vehicle) authority. In this framework, the applicant applies for an AV approval, the assessor assesses the AV and advises the authority, and the authority sets the requirements for the AV and decides whether an AV is approved for road use. The challenges are as follows. Firstly, the tests need to be tailored to the operational design domain (ODD) and dynamic driving task (DDT) description of the AV. Secondly, it is assumed that the applicant does not want to disclose detailed test results because of proprietary or confidential information contained in these results. Thirdly, due to the complex ODD and DDT, many test scenarios are required to obtain sufficient confidence in the assessment of the AV. Consequently, it is assumed that due to the assessor's limited resources, it is infeasible for the assessor to conduct all tests. 

We propose a systematic approach for determining the tests that are based on the requirements set by the authority and the AV's ODD and DDT description, such that the tests are tailored to the applicable ODD and DDT. Each test comes with key performance indicators that enables the applicant to provide a performance rating of the AV for each of the tests. By only providing a performance rating for each test, the applicant does not need to disclose the details of the test results. In our proposed framework, the assessor only conducts a limited number of tests. The main purpose of these tests is to verify the fidelity of the results provided by the applicant. Still, the applicant needs to conduct many tests, but this is assumed to be feasible because the applicant can utilize virtual simulation tools to conduct (a part of) the tests whereas this is not possible for the assessor due to proprietary or confidential information. To illustrate the proposed framework, a case study is presented. 

\keywords{safety, assessment, autonomous vehicle}
\end{abstract}
}

%----------------------------------------------------------------------------------------

\begin{document}
	
% Print the title
\maketitle
\thispagestyle{firstpage}
%----------------------------------------------------------------------------------------
%	ARTICLE CONTENTS
%----------------------------------------------------------------------------------------

\section{Introduction}
\label{sec:introduction}

\cstartg
% Introduce scenario-based testing.
The development of Automated Vehicles (AVs) has made significant progress in the last years and it is expected that AVs will soon be introduced on our roads \autocite{madni2018autonomous,bimbraw2015autonomous} and become an integral part of intelligent transportation systems \autocite{eskandarian2012introduction,chanedmiston2020itsjpo}. \cendg
\cstarta An essential aspect in the development of AVs is the assessment of quality and performance aspects of the AVs, such as safety, comfort, and efficiency \autocite{bengler2014threedecades, stellet2015taxonomy}. 
Among other methods, a scenario-based approach has been proposed \autocite{elrofai2018scenario, putz2017pegasus}. 
% Explain that these scenarios may be based on real-world scenarios.
For scenario-based assessment, proper specification of scenarios is crucial since they are directly reflected in the test cases used for the assessment \autocite{stellet2015taxonomy}. 
One approach for specifying these test cases is to base them on captured scenarios from real-world data collected on the level of individual vehicles \autocite{elrofai2018scenario, putz2017pegasus, roesener2016scenariobased, deGelder2017assessment}. 

% Mention other literature that tries to extract scenarios.
Different techniques for capturing scenarios and driving maneuvers have been proposed in literature. 
\textcite{kasper2012oobayesnetworks} use object-oriented Bayesian networks for the recognition of 27 type of driving maneuvers. 
\textcite{krajewski2018highD} detect lane changes using lane crossings and \textcite{schlechtriemen2015lanechange} detect lane changes using a naive Bayes classifier and a hidden Markov model. 
%\textcite{paardekooper2019dataset6000km} present an approach for identification of scenarios and include results for scenarios labeled ``braking in front'' and ``cut in''. 
In \autocite{xie2017driving}, random forest classifiers are used for detecting accelerating, braking, and turning with features extracted using principal component analysis, stacked sparse auto-encoders, and statistical features.
In \autocite{cara2015carcyclist}, safety-critical car-cyclist scenarios are extracted from data collected by a vehicle using several machine-learning techniques, among which support vector machines and multiple instance learning.

% Contribution of this paper.
In this paper, we propose a new method for mining scenarios from real-world driving data using automated tagging and searching for combination of tags. 
Our method consists of two steps. 
First, the data is automatically tagged with relevant information. For example, a tag ``lane change'' is added to a vehicle at the time this vehicle is performing a lane change. 
Second, the scenarios are mined based on the aforementioned tags. \cenda
\cstartd To do this, we represent a scenario using a combination of tags and we search for this combination of tags in the tagged data from the previous step. \cendd

% Advantages of our method:
% 1. Tags are pretty basic --> easy.
% 2. Tagging can be very different, depending on the type of data --> scenario mining still the same!
% 3. Accuracy: by not only relying on past data, accuracy is improved.
% 4. Scalable: many more type of scenarios could be extracted.
\cstarta The proposed method brings several benefits. 
First, by tagging the data, characteristics that are shared among different type of scenarios need to be identified only once, whereas these characteristics would be identified multiple times if each type of scenarios would be identified completely independently. \cenda
\cstartf For example, a characteristic could be the presence of a lead vehicle, so if we independently identify two different types of scenarios that consider a lead vehicle, we would identify the lead vehicle two times. \cendf
\cstarta Second, by splitting the process in two parts, i.e., the tagging and the scenario mining, the scenario mining can be applied to different types of data (e.g., data from a vehicle \autocite{paardekooper2019dataset6000km} or a measurement unit above the road \autocite{kovvali2007video,krajewski2018highD}). 
It is also possible to have manually tagged data, e.g., see \autocite{fontana2018action}. 
%Thirdly, because the scenario mining is performed offline, we do not only rely on past data, which, in turn, increases the accuracy of the scenario mining. 
Third, our approach is easily scalable because additional types of scenarios can be mined by  describing them as a combination of (sequential) tags. \cenda
\cstartf Fourth, the approach reveals promising future possibilities, such as the generation of scenarios based on the mined scenarios. \cendf
\cstartg The generated scenarios can be used to define the test cases for the assessment of intelligent vehicles \autocite{elrofai2018scenario, putz2017pegasus, roesener2016scenariobased, deGelder2017assessment, stellet2015taxonomy, zhao2018evaluation}. \cendg

% Structure.
\cstarta In \cref{sec:problem}, we formulate the problem of scenario mining. \Cref{sec:tagging,sec:mining} describe the two steps of our proposed method, i.e., the tagging of the data and the scenario mining based on these tags. 
We illustrate the proposed scenario mining approach with few examples in \cref{sec:case study}. \cenda
\cstartf In \cref{sec:discussion}, we discuss the approach, results, and some possible future improvements. \cendf
We end this paper with conclusions and discuss next steps in \cref{sec:conclusions}. \cenda


\section{Conclusions}
\label{sec:conclusions}

	Systematic quantification of the risk provides additional trust in the safety analysis that depends on the availability of data rather than experts judgment. 

%----------------------------------------------------------------------------------------
%	REFERENCE LIST
%----------------------------------------------------------------------------------------

\printbibliography

%----------------------------------------------------------------------------------------

\end{document}
