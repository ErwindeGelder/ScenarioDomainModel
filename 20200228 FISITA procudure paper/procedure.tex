%----------------------------------------------------------------------------------------
%	PACKAGES AND OTHER DOCUMENT CONFIGURATIONS
%----------------------------------------------------------------------------------------

\documentclass[twoside,twocolumn,9pt]{article}

\usepackage{blindtext} % Package to generate dummy text throughout this template 

%\usepackage[sc]{mathpazo} % Use the Palatino font
\usepackage{mathptmx}
\usepackage[T1]{fontenc} % Use 8-bit encoding that has 256 glyphs
\linespread{1.05} % Line spacing - Palatino needs more space between lines
\usepackage{microtype} % Slightly tweak font spacing for aesthetics

\usepackage{graphicx} 

\usepackage[english]{babel} % Language hyphenation and typographical rules

\usepackage[a4paper,left=0.71in,top=0.98in,right=0.71in,bottom=0.98in,columnsep=15pt]{geometry} % Document margins
\usepackage[hang, small,labelfont=bf,up,textfont=it,up]{caption} % Custom captions under/above floats in tables or figures
\usepackage{booktabs} % Horizontal rules in tables

\usepackage{enumitem} % Customized lists
\setlist[itemize]{noitemsep} % Make itemize lists more compact

\usepackage[runin]{abstract} % Allows abstract customization
\setlength{\abstitleskip}{-\parindent}
%\newcommand{\abstitlestyle}[1]{#1}
\renewcommand{\abstractnamefont}{\normalfont\bfseries\MakeUppercase} % Set the "Abstract" text to bold
\renewcommand{\abstracttextfont}{\normalfont} % Set the abstract itself to small italic text
\abslabeldelim{:}


\usepackage{titlesec} % Allows customization of titles
%\renewcommand\thesection{\Roman{section}} % Roman numerals for the sections
%\renewcommand\thesubsection{\roman{subsection}} % roman numerals for subsections
\titleformat{\section}[block]{\large\bfseries}{\thesection.}{1em}{} % Change the look of the section titles
\titleformat{\subsection}[block]{\large}{\thesubsection.}{1em}{} % Change the look of the section titles
\titleformat{\subsubsection}[block]{\normalsize}{\thesubsubsection.}{1em}{} % Change the look of the section titles

\usepackage{fancyhdr} % Headers and footers
\pagestyle{fancy} % All pages have headers and footers
\fancyhf{}% clear default for head and foot
\fancyhead[L]{\includegraphics[scale=0.11]{figures/FISITA.png}}
\fancyhead[R]{\thepage}
\fancyfoot[L]{Proceedings of the FISITA 2020 World Congress, Prague, 14 - 18 September 2020}
\renewcommand{\headrulewidth}{0pt}
\fancypagestyle{firstpage}{
	\fancyhf{}% clear default for head and foot
	\renewcommand{\headrulewidth}{0pt}
	\pagestyle{fancy}
	\lhead{\textbf{F2020-VES-014}}
	\rhead{\includegraphics[scale=0.15]{figures/FISITA.png}}
	\fancyfoot[L]{Proceedings of the FISITA 2020 World Congress, Prague, 14 - 18 September 2020}
}
\usepackage{xpatch}
\xapptocmd{\titlepage}{\thispagestyle{firstpage}}{}{}

\usepackage{titling} % Customizing the title section

%\usepackage{hyperref} % For hyperlinks in the PDF

\usepackage{authblk}

\usepackage{caption}
\DeclareCaptionLabelFormat{nospace}{#1#2}
\captionsetup[table]{labelfont=normal, textfont=normal,name=Table ,labelsep=period, justification=raggedright, singlelinecheck=off}
\captionsetup[figure]{labelfont=normal, textfont=normal,name=Figure ,labelsep=period, justification=raggedright, singlelinecheck=off}
\usepackage[utf8]{inputenc}   				 	%% utf8 support (required for biblatex)
\usepackage{silence}  							%% For filtering warnings
\usepackage[style=ieee,doi=false,isbn=false,url=false,date=year,backend=biber,maxbibnames=15,maxcitenames=2,mincitenames=1,uniquelist=false,uniquename=false,giveninits=true]{biblatex}
% Filter warnings issued by package biblatex starting with "Patching footnotes failed"
\WarningFilter{biblatex}{Patching footnotes failed}
%\renewcommand*{\bibfont}{\footnotesize}		%% Use this for papers
\renewcommand*{\bibfont}{\small}
\setlength{\biblabelsep}{\labelsep}
\bibliography{../bib}
\usepackage[fleqn]{amsmath}

\usepackage{xcolor}
\usepackage[capitalize]{cleveref}
\crefname{figure}{Figure}{Figures}

% Table stuff
\usepackage{booktabs}
\usepackage{tabularx}
\setlength{\heavyrulewidth}{0.1em}
\newcommand{\otoprule}{\midrule[\heavyrulewidth]}

\usepackage{xcolor}
\usepackage{tikz}
\usepackage{pifont} % \ding{}
\definecolor{resultfail}{rgb}{1.,0,0}
\definecolor{resultacceptable}{rgb}{1,0.76.,0}
\definecolor{resultfair}{rgb}{0,0.69.,0.94}
\definecolor{resultgood}{rgb}{0,0.69.,0.31}
\newcommand{\mycircle}[2]{\tikz \path[draw=#1, fill=#2] (0, 0) circle (1ex);}
\newcommand{\mysquare}[2]{\tikz \path[draw=#1, fill=#2] (0, 0) rectangle (2ex, 2ex);}
\newcommand{\fail}{\mycircle{resultfail}{resultfail}}
\newcommand{\acceptable}{\mycircle{resultacceptable}{resultacceptable}}
\newcommand{\fair}{\mycircle{resultfair}{resultfair}}
\newcommand{\good}{\mycircle{resultgood}{resultgood}}
\newcommand{\pass}{\textcolor{green}{\ding{52}}}
\newcommand{\nopass}{\textcolor{red}{\ding{55}}}
\newcommand{\nonconformity}{\mysquare{black}{yellow}}
\newcommand{\observation}{\mysquare{black}{blue}}




%----------------------------------------------------------------------------------------
%	TITLE SECTION
%----------------------------------------------------------------------------------------

\providecommand{\keywords}[1]{\textbf{KEY WORDS:} #1}

\pretitle{\begin{center}\huge\normalfont\MakeUppercase} % Article title formatting
\posttitle{\end{center}\vskip 1em} % Article title closing formatting
\title{type the title of the paper here} % Article title

\author[1,2]{\large\bfseries Erwin de Gelder}
\author[1]{\large\bfseries Olaf Op den Camp}
%\author[1,3]{\large\bfseries Jan-Pieter Paardekooper}
%\author[2]{\large\bfseries Bart De Schutter}

\affil[1]{\normalsize\textit{TNO, Integrated Vehicle Safety, Helmond, The Netherlands (E-mail: erwin.degelder@tno.nl, olaf.opdencamp@tno.nl)}}
\affil[2]{\normalsize\textit{Delft University of Technology, Delft Center for Systems and Control, Delft, The Netherlands}}
%\affil[3]{\normalsize\textit{Radboud University, Donders Institute for Brain, Cognition and Behaviour, Nijmegen, The Netherlands}}
\date{} % Leave empty to omit a date

\renewcommand{\maketitlehookd}{%
\begin{abstract}	
\noindent The development of Autonomous Vehicles (AVs) has made significant progress in the last years and it is expected that AVs will soon be introduced on our roads. An important aspect in the development of AVs is the assessment of their safety. As traditional methods for safety assessment of vehicles are not feasible to be performed for AVs within reasonable time and cost, new approaches need to be worked out. Among these, real-world scenario-based assessment is widely supported by many players in the automotive field. Scenario-based assessment allows for using virtual simulation tools in addition to physical tests, such as on a test track, proving ground, or public road.

We propose a procedure for real-world scenario-based road-approval assessment considering three stakeholders: the applicant, the assessor, and the (road or vehicle) authority. In this procedure, the applicant applies for the approval of one specific AV, the assessor assesses this AV and advises the authority, and the authority sets the requirements for the AV and decides whether this specific AV is approved for road testing The challenges are as follows. Firstly, the tests need to be tailored to the operational design domain (ODD) and dynamic driving task (DDT) description of the AV. Secondly, it is assumed that the applicant does not want to disclose all of the detailed test results because of proprietary or confidential information contained in these results. Thirdly, due to the complex ODD and DDT, many test scenarios are required to obtain sufficient confidence in the assessment of the AV. Consequently, it is assumed that due to limited resources, it is infeasible for the assessor to conduct all (physical) tests. 

We propose a systematic approach for determining the tests that are based on the requirements set by the authority and the AV’s ODD and DDT description, such that the tests are tailored to the applicable ODD and DDT. Each test comes with metrics that enables the applicant to provide a performance rating of the AV for each of the tests. By only providing a performance rating for each test, the applicant does not need to disclose the details of the test results. In our proposed procedure, the assessor only conducts a limited number of tests. The main purpose of these tests is to verify the fidelity of the results provided by the applicant. Still, the applicant needs to conduct many tests, but this is assumed to be feasible because the applicant can utilize virtual simulation tools to conduct (a part of) the tests whereas this is not possible for the assessor due to proprietary or confidential information. To illustrate the proposed procedure, an example is presented.

\keywords{safety assessment, autonomous vehicle, scenario-based assessment}
\end{abstract}
}

%----------------------------------------------------------------------------------------

\begin{document}
	
% Print the title
\maketitle
\thispagestyle{firstpage}
%----------------------------------------------------------------------------------------
%	ARTICLE CONTENTS
%----------------------------------------------------------------------------------------

\section{Introduction}
\label{sec:introduction}

\cstartg
% Introduce scenario-based testing.
The development of Automated Vehicles (AVs) has made significant progress in the last years and it is expected that AVs will soon be introduced on our roads \autocite{madni2018autonomous,bimbraw2015autonomous} and become an integral part of intelligent transportation systems \autocite{eskandarian2012introduction,chanedmiston2020itsjpo}. \cendg
\cstarta An essential aspect in the development of AVs is the assessment of quality and performance aspects of the AVs, such as safety, comfort, and efficiency \autocite{bengler2014threedecades, stellet2015taxonomy}. 
Among other methods, a scenario-based approach has been proposed \autocite{elrofai2018scenario, putz2017pegasus}. 
% Explain that these scenarios may be based on real-world scenarios.
For scenario-based assessment, proper specification of scenarios is crucial since they are directly reflected in the test cases used for the assessment \autocite{stellet2015taxonomy}. 
One approach for specifying these test cases is to base them on captured scenarios from real-world data collected on the level of individual vehicles \autocite{elrofai2018scenario, putz2017pegasus, roesener2016scenariobased, deGelder2017assessment}. 

% Mention other literature that tries to extract scenarios.
Different techniques for capturing scenarios and driving maneuvers have been proposed in literature. 
\textcite{kasper2012oobayesnetworks} use object-oriented Bayesian networks for the recognition of 27 type of driving maneuvers. 
\textcite{krajewski2018highD} detect lane changes using lane crossings and \textcite{schlechtriemen2015lanechange} detect lane changes using a naive Bayes classifier and a hidden Markov model. 
%\textcite{paardekooper2019dataset6000km} present an approach for identification of scenarios and include results for scenarios labeled ``braking in front'' and ``cut in''. 
In \autocite{xie2017driving}, random forest classifiers are used for detecting accelerating, braking, and turning with features extracted using principal component analysis, stacked sparse auto-encoders, and statistical features.
In \autocite{cara2015carcyclist}, safety-critical car-cyclist scenarios are extracted from data collected by a vehicle using several machine-learning techniques, among which support vector machines and multiple instance learning.

% Contribution of this paper.
In this paper, we propose a new method for mining scenarios from real-world driving data using automated tagging and searching for combination of tags. 
Our method consists of two steps. 
First, the data is automatically tagged with relevant information. For example, a tag ``lane change'' is added to a vehicle at the time this vehicle is performing a lane change. 
Second, the scenarios are mined based on the aforementioned tags. \cenda
\cstartd To do this, we represent a scenario using a combination of tags and we search for this combination of tags in the tagged data from the previous step. \cendd

% Advantages of our method:
% 1. Tags are pretty basic --> easy.
% 2. Tagging can be very different, depending on the type of data --> scenario mining still the same!
% 3. Accuracy: by not only relying on past data, accuracy is improved.
% 4. Scalable: many more type of scenarios could be extracted.
\cstarta The proposed method brings several benefits. 
First, by tagging the data, characteristics that are shared among different type of scenarios need to be identified only once, whereas these characteristics would be identified multiple times if each type of scenarios would be identified completely independently. \cenda
\cstartf For example, a characteristic could be the presence of a lead vehicle, so if we independently identify two different types of scenarios that consider a lead vehicle, we would identify the lead vehicle two times. \cendf
\cstarta Second, by splitting the process in two parts, i.e., the tagging and the scenario mining, the scenario mining can be applied to different types of data (e.g., data from a vehicle \autocite{paardekooper2019dataset6000km} or a measurement unit above the road \autocite{kovvali2007video,krajewski2018highD}). 
It is also possible to have manually tagged data, e.g., see \autocite{fontana2018action}. 
%Thirdly, because the scenario mining is performed offline, we do not only rely on past data, which, in turn, increases the accuracy of the scenario mining. 
Third, our approach is easily scalable because additional types of scenarios can be mined by  describing them as a combination of (sequential) tags. \cenda
\cstartf Fourth, the approach reveals promising future possibilities, such as the generation of scenarios based on the mined scenarios. \cendf
\cstartg The generated scenarios can be used to define the test cases for the assessment of intelligent vehicles \autocite{elrofai2018scenario, putz2017pegasus, roesener2016scenariobased, deGelder2017assessment, stellet2015taxonomy, zhao2018evaluation}. \cendg

% Structure.
\cstarta In \cref{sec:problem}, we formulate the problem of scenario mining. \Cref{sec:tagging,sec:mining} describe the two steps of our proposed method, i.e., the tagging of the data and the scenario mining based on these tags. 
We illustrate the proposed scenario mining approach with few examples in \cref{sec:case study}. \cenda
\cstartf In \cref{sec:discussion}, we discuss the approach, results, and some possible future improvements. \cendf
We end this paper with conclusions and discuss next steps in \cref{sec:conclusions}. \cenda

\section{Problem definitions}
\label{sec:problem} % Arash

The need for quantification on risk calculation for automated driving. 
TODO: formalizing this sentence in the context of scenario. 

Differentiating probability of a single (hazardous) event (state of art) to what we are doing: i.e. probability of whole scenario. 

\section{Procedure for the safety assessment}
\label{sec:procedure}

This paper assumes that many of the relevant tests for the safety assessment are performed in a virtual simulation environment that is controlled by the applicant. The proposed procedure intends to consider all results, both from virtual simulation and from actually performed physical tests. Where the assessor does not have access to the required models of the AV under test, the assessor will have the capability to perform physical tests on the AV. How to balance between the different results in the assessment, considering virtual and physical test results of the applicant and physical test results of the assessor is schematically presented in \cref{fig:procedure}. Each rectangular block represents an action. The procedure distinguishes between actions for which the applicant is responsible and actions for which the assessor is responsible. The procedure consists of the following actions:
\begin{enumerate}
	\item The first action is to derive which system-level tests need to be performed with reference to the ODD and DDT of the AV under test. Here, “system-level” is mentioned explicitly, because it is assumed that also in case of a failure of any of the subsystems, the AV would fail the system-level tests. Note, however, that it is advised that the applicant ensures that each of the subsystems underwent a rigorous assessment before applying for the AV assessment. 
	\item If the derived tests are acceptable, the next action is to select the tests for the assessment. Here, a distinction is made between tests for which the applicant is fully responsible and physical tests that are conducted by the assessor. The latter will focus more on spot checking.
	\item Once the tests are selected, these tests need to be conducted. The results of these tests will be described using prescribed metrics. Note, however, that these metrics may not contain too much information as it is assumed that the applicant does not want to disclose details of sensor and system implementation or even detailed test results because of the proprietary or confidential information contained in these results. 
	\item The final step is to assess the results from the tests and to formulate an advice for the authority on whether the AV is ready for deployment and under which conditions.
\end{enumerate}

\begin{figure*}
	\centering
	\includegraphics[width=\linewidth]{figures/procedure}
	\caption{Proposed framework for the safety assessment of an autonomous vehicle (AV).}
	\label{fig:procedure}
\end{figure*}

In the following sections, each of the actions are further detailed. We end this section with a short note on monitored deployment in case of a successful completion of the assessment.



\subsection{Deriving test descriptions}
\label{sec:test descriptions}

Based on the ODD and the DDT of the AV, the tests are derived. Following the same reasoning as \textcite{stellet2015taxonomy}, a test is an evaluation of:
\begin{itemize}
	\item a statement on the system-under-test (test criteria; what are we going to evaluate using the test);
	\item under a set of specified conditions (test case; how are we going to evaluate the test criteria);
	\item using quantitative measures (metrics; how to express the outcome of the test quantitatively);
	\item and a reference of what would be the acceptable outcome (reference; when is the outcome acceptable).
\end{itemize}

Since the applicant has designed and developed the AV, it is expected that the applicant has a clear notion of the tests that are required for a complete assessment of the AV and which the AV should appropriately handle. Similarly, if the set of relevant test descriptions is not complete during the development of the AV, it is conceivable that the AV will not operate safely for all circumstances possible within the ODD. 

Although it is expected that the applicant provides all relevant test descriptions, it is important that the applicant and the assessor discuss these test descriptions, and that a check is made whether or not the test descriptions are complete and cover the ODD sufficiently. If an important test description is missing, it is conceivable that the AV is not specifically designed to pass the corresponding test. In order to assess the completeness of the test descriptions provided by the applicant, the assessor needs to define the test domain for the relevant system-level test descriptions and use these to investigate if any important test descriptions are missing. Here, the so-called test domain refers to a more high-level description of the range of tests that are expected, rather than an enumeration of the large number of relevant tests.

If it turns out that the test descriptions that are provided by the applicant are not complete, the process needs to be restarted, as indicated by the ``Not OK'' line in \cref{fig:procedure}. On the other hand, if the test descriptions are deemed to be complete enough, the assessment proceeds to the next step: selecting tests for the assessment.



\subsection{Selecting tests for the assessment}
\label{sec:selection}

In principle, the applicant is expected to provide results for all tests. In the next section, we explain how these results may look like. Based on these results, tests are selected for the physical testing performed by the assessor. This is indicated by the arrow pointing from ``results of tests'' of the applicant to ``select tests for assessment'' in \cref{fig:procedure}.  

A test is selected for physical testing by the assessor if any of the following three statements are true:
\begin{itemize}
	\item The applicant does not provide a result. Although the applicant is expected to provide results for most tests, it might be possible that there are some tests for which the applicant does not have the resources to perform the tests reliably, for example if specific tooling is required. Note, however, that if the applicant does not provide results for too many tests, the assessment automatically results in a fail.
	\item The result seems inconsistent. If there is sufficient reason for the assessor to not confide in the result provided by the applicant, the test can be performed by the assessor to check the result provided by the applicant.
	\item The test is selected for spot checking. The main reason to perform spot checking is to assess the fidelity of the results provided by the assessor.
\end{itemize}
The process of test selection is summarized in \cref{fig:selection}.

\begin{figure}
	\centering
	\includegraphics[width=.65\linewidth]{figures/test_selection}
	\caption{Decision scheme for the selection of a test for physical assessment by the assessor.}
	\label{fig:selection}
\end{figure}



\subsection{Testing}
\label{sec:testing}

As explained in the previous section, the applicant is expected to provide results for most tests. However, it is assumed that the applicant does not want to disclose detailed test results. Therefore, a rating scheme is proposed. Using a specific metric for each test, three references are defined: an acceptable result, a fair result, and a good result. If the result of the test is worse that the acceptable result, a ``fail'' is reported. If the result passes the acceptable reference but not the what is defined as a fair result, an ``acceptable'' is reported. Similarly, a ``fair'' is reported if the result is between a fair and a good result. If the result is better than what has been defined as a good result, a ``good'' is reported. This is schematically shown in \cref{fig:rating}. 

\begin{figure}
	\centering
	\includegraphics[width=\linewidth]{figures/rating}
	\caption{Different scoring options per test.}
	\label{fig:rating}
\end{figure}

In principle, the applicant is free to choose any method to derive the results. However, considering the large number of tests, the use of virtual simulations seems inevitable. In practice, it is expected that a both virtual simulations, physical tests, and a combination, such as hardware-in-the-loop testing, is used to determine the test results.

On the other hand, the tests by the assessor are performed physically. The main reason for this is that virtual simulations are ruled out as that would require the applicant to provide a model of the AV, which is expected to be impossible because of proprietary reasons.



\subsection{Assess results}
\label{sec:assess results}

The following assessment results are distinguished per test:
\begin{itemize}
	\item In case the test results show that for the specific test the AV performs acceptable (i.e., ``acceptable'', ``fair'', or ``good'', see \cref{fig:rating}), the test is passed. If this is not the case, then the specific test fails.
	\item Inspired by ISO~9001 on quality management \autocite{ISO9001}, a passed test may result in a non-conformity. An “acceptable” result automatically leads to a non-conformity (NC). This means that the response of the AV deviates substantially from  response that is qualified as “good”, but the deviation is not severe. Since the AV meets the minimum requirement for this test and consequently safety is not compromised, there is no reason to fail the AV based on this test. Nevertheless, an NC is issued to indicate that the applicant is asked to consider improvements, e.g., for a next version of the system.
	\item In case the test is also performed by the assessor and the corresponding result is worse than the reported result of the applicant, this also leads to a NC.
	\item The assessment of a test result might come with an observation (OB) that needs consideration of the applicant. 
\end{itemize}

If a test results in a fail, then either the assessment results in a negative advice of the assessor to the authority or it is advised to only allow for deployment of the AV under certain conditions. For example, if the only tests that are failed consider low-light conditions, the AV might be deployed under the condition that it operates only from sunrise till sunset. 

NCs and OBs do not lead to an immediate fail of the assessment. However, it is likely that they lead to a fail in a future assessment, e.g., when test criteria become increasingly demanding, and the applicant does not appropriately consider such NCs or OBs. NCs provide information to the applicant on how requirements might develop in the future, which, consequently, gives direction and motivation on continuous improvement of AVs regarding safety. On the other hand, many NCs – the AV barely passes the test in many cases – might mean that safety is compromised and, therefore, it might also result in a negative advice of the assessor to the authority regarding the deployment of the AV.

Note that when many NCs are observed, the AV probably will not be able to pass all tests if all tests would be performed physically by the assessor. Theoretically, this is however still possible. To minimize the risk of having an AV that passes all tests, but with many NCs, a system using demerit points is introduced. The AV starts with, e.g., 100 points, and in the assessment, 1, 2, or 3 points are subtracted for each NC, depending on the severity of the NC. Once the number of points for the AV are reduced to 0, then the AV is indicated to have failed he assessment because of an overrun of NCs. The numbers given here are merely provided as an example.



\subsection{Monitored deployment}
\label{sec:monitored deployment}

A successful completion of the proposed assessment might result in the approval for the deployment of the AV under the condition that the behavior of the AV on the road is continuously monitored. We propose that during such a deployment phase, the applicant is required to upload detailed driving data to allow for monitoring the AV behavior. This is implemented for two reasons:
\begin{itemize}
	\item After completion of the assessment pipeline, road and/or vehicle authorities may require the monitoring of safety continuously when driving on the public road.
	\item The uploaded data may be used to improve the generation of tests and the selection of relevant test cases for a particular AV, as is possible that some tests have been overlooked during the initial assessment process or that situations on the road gradually change with changes in traffic, e.g.because of the introduction of new mobility systems.
\end{itemize}

The feedback to the data acquisition element allows for ongoing learning and improvement of the standards and assessment systems, while being able to adapt to new types of transportation such as personal mobility devices. For example, additional test cases could be identified and incorporated into future safety assessment procedures. A deployment might consider new operational areas, the extension of the scenario database with scenarios that potentially differ between such areas would then be covered. Moreover, to obtain a scenario database that is `complete', i.e., statistically accurate, it is expected that operational data collection is required over an extended period, which most probably will not be realized before the deployment is operationalized. In other words, the imperfection of the assessment framework should not become a barrier for the introduction of new safe mobility solutions onto the market, in case these devices are tested to be safe for all currently known conditions. The assessment method, especially the step regarding monitored deployment, supports the continuous increase in knowledge on the state-of-the-art of road safety and herewith prepares the safety assessment method to be sustainable for the future.

\section{Case Study} % Erwin & Hala
\label{sec:example}

In this section, we present a case study to illustrate the method of quantifying the risk for a type of scenario, i.e., a scenario class \cite{elrofai2018scenario}. We will first explain the scenario class and the use case. The actual system for which the risk is computed is presented in \cref{sec:system}. Next, we will describe the conditions and activities in \cref{sec:conditions,sec:activities}, respectively. Finally, we will present the results in \cref{sec:results}.

\subsection{The scenario class and its use case}
\label{sec:scenario class}

We want to quantify the risk for scenarios that are linguistically described as follows: while the ego vehicle drives at a moderate to high speed while staying in its lane, another vehicle cuts into the lane of the ego vehicle, such that this vehicle becomes the ego vehicle's lead vehicle. Furthermore, the ego vehicle needs to brake to prevent a collision.

For the quantification of the risk, 60 hours of data (see also \cite{deGelder2017assessment}) are collected by driving a specific route in and between Eindhoven and Helmond, The Netherlands, with twenty different drivers, each driving the route twice. Therefore, it is assumed that the use case of the automation system is simply driving this route. We will use the data for the estimation of the risk. Hence, we will make use of the following assumption:
\begin{assumption}
	The recorded naturalistic driving data is representative for what a vehicle with an automation system might encounter along the same route.
\end{assumption}

\subsection{System-under-test}
\label{sec:system}

To reduce efforts for the assessment, often simulations are employed. However, even simulations can consume considerable time, as these simulations might run real-time \cite{shah2018airsim} or slower when a higher level of detail is used \cite{zofka2016testing}. For our method, we will simplify the simulations, such that the total required time on a common computer is in the order of minutes. Since we are interested in approximate results, a high level of detail is not required. 

To simplify the system-under-test, it is assumed that the system's desired acceleration is similar to the adaptive cruise control defined in \cite{deGelder2017assessment}, i.e.,
\begin{equation}
	\label{eq:desired acceleration} 
	u(t) = k_{\mathrm{d}}(v(t))(d(t) - \tau_{\mathrm{h}} v(t) - s_0) + k_{\mathrm{v}}\left(\dot{d}(t) - ha(t) \right),
\end{equation}
with
\begin{equation}
	\label{eq:gain}
	k_{\mathrm{d}}(v(t)) = k_{\mathrm{d1}} + \left( k_{\mathrm{d2}} - k_{\mathrm{d1}} \right) \exp \left\{ -\frac{v(t)^2}{2\sigma_{\mathrm{d}}} \right\}.
\end{equation}
Here, $v$ is the speed of the ego vehicle, $d$ denotes the clearance between the ego vehicle and its predecessor, i.e., the vehicle that performs the cut-in. The relative speed is denoted by $\dot{d}$ and $a$ refers to the acceleration of the ego vehicle. The ego vehicle is modeled using a first order model with a time delay, i.e.,
\begin{equation}
	\label{eq:vehicle model}
	\tau \dot{a}(t) + a(t) = u(t - \theta).
\end{equation}
Furthermore, the deceleration is limited at \unit[-6]{ms^{-2}}. A description of the constants of \cref{eq:desired acceleration,eq:gain,eq:vehicle model} are listed in \cref{tab:constants}.

\begin{table}
	\centering
	\caption{The constants used for the simple automation system of \cref{eq:desired acceleration,eq:gain,eq:vehicle model}.}
	\label{tab:constants}
	\begin{tabular}{clc}
		\toprule
		Parameter & Description & Value \\ \otoprule
		$\tau_{\mathrm{h}}$ & Desired time headway & \unit[1.0]{s} and \unit[2.0]{s} \\
		$s_0$ & Safety distance & \unit[1.5]{m} \\
		$k_{\mathrm{d1}}$ & Distance gain at high speed & $\unit[0.7]{s^{-2}}$ \\
		$k_{\mathrm{d1}}$ & Distance gain at low speed & $\unit[2.0]{s^{-2}}$ \\
		$\sigma_{\mathrm{d}}$ & Shaping coefficient of distance gain & $\unit[5]{ms^{-1}}$ \\
		$k_{\mathrm{v}}$ & Speed difference gain & $\unit[0.35]{s^{-1}}$ \\
		$\tau$ & Time constant of vehicle model & \unit[0.1]{s} \\
		$\theta$ & Delay of the vehicle response & \unit[0.2]{s} \\
		\bottomrule
	\end{tabular}
\end{table}

\subsection{Conditions}
\label{sec:conditions}

All scenarios are subject to the following conditions:
\begin{itemize}
	\item $C_1$: The speed of the ego vehicle is within the range of \unit[60]{km/h} and \unit[130]{km/h}.
	\item $C_2$: There are no restrictions on the weather conditions.
	\item $C_3$: There are no restrictions on the lighting conditions.
\end{itemize}

Obviously, because there are no restrictions to the weather and lighting conditions, we have $P(C_2,C_3)=1$. For the first condition, we can use the data to estimate the likelihood. The data, however, has been recorded during sunny weather at daylight. Therefore, we need to following assumption.

\begin{assumption} \label{asm:conditions}
	Let $C_2'$ and $C_3'$ denote the conditions of having sunny weather and daylight, respectively. Then we have $P(C_1|C_2,C_3)=P(C_1|C_2',C_3')$.
\end{assumption}

From the data, it appeared that $P(C_1|C_2',C_3')=0.20$. Using \cref{asm:conditions}, we have
\begin{dmath}
	P(C) = P(C_1,C_2,C_3)=P(C_1|C_2',C_3')\cdot P(C_2,C_3)=0.20.
\end{dmath}

\subsection{Activities}
\label{sec:activities}

\subsection{Results}
\label{sec:results}


\section{Conclusions}
\label{sec:conclusions}

	Systematic quantification of the risk provides additional trust in the safety analysis that depends on the availability of data rather than experts judgment. 

%----------------------------------------------------------------------------------------
%	REFERENCE LIST
%----------------------------------------------------------------------------------------

\printbibliography

%----------------------------------------------------------------------------------------

\end{document}
