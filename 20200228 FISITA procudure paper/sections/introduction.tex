\section{Introduction}
\label{sec:introduction}

% Automated Vehicles are introduced
% Assessment of AVs is important
The development of Autonomous Vehicles (AVs) has made significant progress. It is expected that AVs will be mainstream by 2040 \cite{madni2018autonomous} or earlier \cite{bimbraw2015autonomous}. 
An important aspect in the development of AVs is the safety assessment \cite{bengler2014threedecades, stellet2015taxonomy, Helmer2017safety, putz2017pegasus, wachenfeld2016release}. For legal and public acceptance of AVs, a clear definition of system performance is important, as are quantitative measures for the system quality. The more traditional methods \cite{ISO26262, response2006code}, used for evaluation of driver assistance systems, are no longer sufficient for the assessment of the safety of higher level AVs, as it is not feasible to complete the quantity of testing required by these methodologies \cite{wachenfeld2016release}. Therefore, the development of assessment methods is important to not delay the deployment of AVs \cite{bengler2014threedecades}.

One of the many challenges regarding the assessment of an AV \cite{koopman2016challenges} is to agree on a procedure that results in a reliable evaluation of the AV, provided that:
\begin{itemize}
	\item the assessment is sufficiently tailored to the Operational Design Domain (ODD) and Dynamic Driving Task (DDT) of the AV,
	\item the proprietary and confidential information regarding the development of the AV are respected, and
	\item the resources are limited. 
\end{itemize}
In this paper, we propose a procedure for a assessment of an AV that takes into account the aforementioned considerations. The procedure assumes a scenario-based approach for assessing the safety \cite{putz2017pegasus, winner2017pegasus, nhtsa2018framework}. In our procedure, three stakeholders are considered: the applicant that provides the AV, the authority that decides on the approval of the AV for road testing, and the assessor that performs the independent safety assessment of the AV. Based on the requirements set by the authority, the applicant and the assessor need to come to an agreement on the set of tests for the safety assessment. Based on the test results from both the applicant and the assessor, the assessor evaluates whether the AV is ready for deployment on the road and, if so, under which conditions.

To be best of the authors' knowledge, there is no literature that provides a procedure for the assessment of AVs while distinguishing between the different stakeholders that are involved. The proposed procedure could used within a legal framework for the approval of AVs for road testing. Eventually, as technology improves, the procedure might be a good starting point for developing the legal framework for the type-approval of AVs.

In \cref{sec:problem}, we provide the problem definition after elaborating more on the context. The proposed procedure is presented in \cref{sec:procedure}. Using an hypothetical example in \cref{sec:example}, the procedure is illustrated. We end this work with conclusions in \cref{sec:conclusions}.
