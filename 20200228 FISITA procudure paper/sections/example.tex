\section{Example}
\label{sec:example}

In this section, we present a hypothetical example to illustrate how the safety assessment as discussed in the previous section may look like. \Cref{tab:example} lists the results of an assessment consisting of 14 tests. Note that in reality, the number of tests is likely to be much larger, but for the sake of the example, we keep the number of tests rather limited. 

\begin{table}
	\centering
	\caption{Results from a hypothetical assessment. The result may be ``fail'' \protect\fail, ``acceptable'' \protect\acceptable, ``fair'' \protect\fair, or ``good'' \protect\good; \protect\pass{} denotes a pass and \protect\nopass{} denotes a fail; NC and OB are denoted by \protect\nonconformity{} and \protect\observation{}, respectively.}
	\label{tab:example}
	\begin{tabular}{llllll}
		\toprule
		Test            & Result      & Result      & Fidelity & P/F     & NC  \\
		ID              & applicant   & assessor    & check    &         & OB  \\
		\otoprule
		1.1             & \good       & \good       & \pass    & \pass   &     \\
		1.2             & \fair       & -           & \pass    & \pass   &       \\
		1.3             & \fair       & -           & \pass    & \pass   &        \\
		1.4             & \acceptable & -           & \pass    & \pass   &\nonconformity    \\
		2.1             & \good       & -           & \pass    & \pass   &        \\
		2.2             & \good       & \fair       & \nopass  & \pass   & \nonconformity      \\
		2.3             & \fair       & -           & \pass    & \pass   &        \\
		2.4             & \fair       & \fair       & \pass    & \pass   &     \observation \\
		3.1             & \fair       & \acceptable & \nopass  & \pass   & \nonconformity     \\
		3.2             & \fair       & \good       & \pass    & \pass   &       \\
		3.3             & \fail       & -           & \pass    & \nopass &       \\
		4.1             & \fair       & -           & \pass    & \pass   &       \\
		4.2             & \acceptable & -           & \pass    & \pass   & \nonconformity      \\
		4.3             & \acceptable & \fail       & \nopass  & \nopass &       \\ \bottomrule
	\end{tabular}
\end{table}


The applicant reports a good result for the first test (test ID 1.1). Because the assessor comes to the same conclusion, the fidelity check is passed as well as the test. The next three tests (1.2 till 1.4) are not performed by the assessor, so the reported results of the applicant are included in the assessment result. These three tests are all passed, but because the last of these tests (1.4) barely passes the minimum requirement (an “acceptable” is reported), an NC is issued.

The tests 2.1 till 2.4 are all passed. However, the applicant reports a better result for test 2.2 than the assessor. Therefore, the fidelity check failed, and an NC is issued. For test 2.4, an OB is made. This could be, e.g., because during the test that is performed by the assessor the AV showed erratic behavior even though safety has not been compromised.

For the next three tests (3.1 till 3.3), an NC is issued, and one test is failed. The NC is issued for two reasons: the assessor reports an acceptable result and the fidelity check has failed. The applicant reports a fail for test 3.3. There is no reason for the assessor to also perform this test, because regardless of that result, the applicant has failed the test. As mentioned in Section 3.4, this might lead to a failed assessment or to some restrictions for the deployment of the AV. Note that for test 3.2, the result of the assessor is better than the reported result of the applicant. This might be caused by the applicant reporting the worst-case outcome, e.g., the outcome with a relatively long detection delay of an object while the detection delay is shorter during the test performed by the assessor. Because the result of the assessor is better, this does not lead to an NC.

For the last three tests, again an NC is issued, and a test is failed. The NC is issued because the applicant reports an acceptable result. Test 4.3 is failed even though the applicant has not reported to fail this test. This might lead to additional measures that need to be taken before the applicant can proceed with the deployment of the AV.
