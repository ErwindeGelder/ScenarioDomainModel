\section{Problem definition}
\label{sec:problem}

In this section, we first explain why many players in the automotive field support scenario-based testing for the assessment of performance aspects of the automated and autonomous vehicles, such as the safety assessment. Next, in \cref{sec:stakeholders}, we elaborate on the different (type of) stakeholders that are involved in the assessment. Given an assessment with these stakeholders, in \cref{sec:challenges}, we describe the problem and the corresponding practical challenges that our procedure should address.



\subsection{Scenario-based testing}
\label{sec:scenario-based testing}

Perhaps the most basic way of assessing the performance of an AV is to drive with the AV in its intended area of operation. While this might provide useful data for further development of the AV, \textcite{kalra2016driving,wachenfeld2016release} show that the number of hours of driving that are required to demonstrate with enough certainty that the AV performs safely is infeasible. So, when it comes to demonstrating the reliability of an AV, another approach is necessary.

An advantage of scenario-based testing is that it allows for selecting those scenarios that are relevant for the safety evaluation. Therefore, a large repetition of scenarios that are relatively straightforward to deal with can be prevented. Furthermore, because virtual simulations can potentially be used for performing scenario-based tests, the number of physical tests can be reduced \autocite{ploeg2018cetran}, ultimately resulting in a less expensive assessment. However, one of the main challenges of scenario-based testing is the selection of the scenarios itself \autocite{riedmaier2020survey}. 

Although there are challenges to be resolved for scenario-based testing, it is used in large research projects and by many players in the automotive field. For example, in European projects such as AdaptIVe \autocite{roesener2017comprehensive}, ENABLE-S3 \autocite{leitner2019validation}, and HEADSTART \autocite{wagener2020headstart}, scenario-based testing is proposed for assessing several performance aspects, such as safety and emissions. In Germany, a large project named PEGASUS was fully dedicated to scenario-based assessment of automated driving functions \autocite{pegasus2019}. Also, in Japan \autocite{jacobo2019development} and in Singapore \autocite{cetran2020}, a scenario-based approach is adopted for the assessment of AVs. To support the scenario-based assessment, different initiatives are started to create a database of scenarios and test cases, see, e.g., \autocite{elrofai2018scenario,myers2020pass}.

Considering the adoption of and the many resources dedicated to scenario-based assessment, it seems convincing that scenario-based assessment is a promising method for a framework for the safety assessment of AVs.



\subsection{Stakeholders for the assessment}
\label{sec:stakeholders}

We consider three different types of stakeholder for the assessment of an AV for its readiness to be deployed in a certain operational area:
\begin{enumerate}
	\item The applicant;
	\item The assessor; and
	\item The authority.
\end{enumerate}

The applicant is applying for the approval for the deployment of an AV for testing on the road. In practice, the applicant can be, e.g., the operator of the vehicles or the developer of the vehicle. The authority is the stakeholder that eventually decides whether the AV can be deployed, so this can be the local vehicle authority. It is a prerequisite for a proper process that the assessor is independent of both the applicant and the authority.
Each stakeholder has different responsibilities, see \cref{tab:stakeholders}. The applicant provides the AV to be assessed. The applicant is also responsible for providing an AV that meets the applicable safety requirements. The assessor is responsible for performing the tests in the assessment, not for deciding whether the AV is approved or not. Instead, the assessor advises the authority whether the AV can be approved and, if necessary, under which conditions. It is then up to the authority to make the final decision. Another responsibility of the authority is to organize a legal framework for the safety assessment of AVs and to set and communicate the requirements for the AV.

\begin{table}
	\centering
	\caption{The stakeholders and their responsibilities.}
	\label{tab:stakeholders}
	\begin{tabularx}{\linewidth}{lX}
		\toprule
		Stakeholder & Responsibilities \\ \otoprule
		Applicant & Apply for approval to deploy the AV. \\
		& Prepare the AV to meet the applicable safety requirements. \\
		& Provide the AV to be assessed. \\ \otoprule
		Assessor & Perform an independent safety assessment of the AV. \\
		& Report results and advise the authority for AV approval. \\ \otoprule
		Authority & Organize a legal framework for safety assessment of AVs. \\
		& Specify needs and set realistic AV safety requirements. \\
		& Decide on the approval of the AV. \\
		\bottomrule
	\end{tabularx}
\end{table}

\begin{remark}
	Within the terminology of the United Nations Economic Commission for Europe (UNECE), the assessor is called ``technical service''.
\end{remark}

\begin{remark}
	The presented stakeholders are particularly applicable for non-US approach. Currently, the role of the applicant and the assessor is often represented by one stakeholder in the US. Nevertheless, the presented procedure might still be applicable if a distinction is made between these two roles within the organization of the applicable stakeholder.
\end{remark}



\subsection{Challenges}
\label{sec:challenges}

Scenario-based assessment comes with many challenges. For example, questions like ``How to generate the test cases?'' and ``How to validate the fidelity of the virtual simulations?'' are discussed extensively in literature. In this paper, however, we want to specifically address the challenges that arise when considering the different stakeholders and each of their responsibilities and capabilities. Therefore, the problem we address in this paper can be formulated as follows:

\emph{What procedure could be used for the safety assessment of an Autonomous Vehicle (AV) by an independent assessor?}

In the framework presented in this paper, the following challenges are addressed:
\begin{itemize}
	\item The tests need to be tailored to the operational design domain (ODD) and dynamic driving task (DDT) description of the AV. However, it is expected to be infeasible for the assessor to go through a rigorous analysis to define all relevant tests for each applicant.
	\item It is assumed that the applicant does not want to disclose details of sensor and system implementation or even detailed test results because of proprietary or confidential information contained in these results. As a result, the assessor does not have access to these detailed test results. The challenge is that the assessor still needs enough confidence in the safe operation of the AV without having access to the detailed test results carried out by the applicant.
	\item Due to the complex ODD and DDT, it is expected that many tests are required to obtain enough confidence in the assessment of the AV. Also, it is assumed that the assessor's resources are too limited to conduct all tests physically. The challenge is that the procedure should still allow the assessor to have enough evidence that the AV is safe or not.
\end{itemize}
