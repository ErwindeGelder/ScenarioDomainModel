\documentclass[10pt,final,a4paper,oneside,onecolumn]{article}

%%==========================================================================
%% Packages
%%==========================================================================
\usepackage[a4paper,left=3.5cm,right=3.5cm,top=3cm,bottom=3cm]{geometry} %% change page layout; remove for IEEE paper format
\usepackage[T1]{fontenc}                        %% output font encoding for international characters (e.g., accented)
\usepackage[cmex10]{amsmath}                    %% math typesetting; consider using the [cmex10] option
\usepackage{amssymb}                            %% special (symbol) fonts for math typesetting
\usepackage{amsthm}                             %% theorem styles
\usepackage{dsfont}                             %% double stroke roman fonts: the real numbers R: $\mathds{R}$
\usepackage{mathrsfs}                           %% formal script fonts: the Laplace transform L: $\mathscr{L}$
\usepackage[pdftex]{graphicx}                   %% graphics control; use dvips for TeXify; use pdftex for PDFTeXify
\usepackage{array}                              %% array functionality (array, tabular)
\usepackage{upgreek}                            %% upright Greek letters; add the prefix 'up', e.g. \upphi
\usepackage{stfloats}                           %% improved handling of floats
\usepackage{multirow}                           %% cells spanning multiple rows in tables
%\usepackage{subfigure}                         %% subfigures and corresponding captions (for use with IEEEconf.cls)
\usepackage{subfig}                             %% subfigures (IEEEtran.cls: set caption=false)
\usepackage{fancyhdr}                           %% page headers and footers
\usepackage[official,left]{eurosym}             %% the euro symbol; command: \euro
\usepackage{appendix}                           %% appendix layout
\usepackage{xspace}                             %% add space after macro depending on context
\usepackage{verbatim}                           %% provides the comment environment
\usepackage[dutch,USenglish]{babel}             %% language support
\usepackage{wrapfig}                            %% wrapping text around figures
\usepackage{longtable}                          %% tables spanning multiple pages
\usepackage{pgfplots}                           %% support for TikZ figures (Matlab/Python)
\pgfplotsset{compat=1.14}						%% Run in backwards compatibility mode
\usepackage[breaklinks=true,hidelinks,          %% implement hyperlinks (dvips yields minor problems with breaklinks;
bookmarksnumbered=true]{hyperref}   %% IEEEtran: set bookmarks=false)
%\usepackage[hyphenbreaks]{breakurl}            %% allow line breaks in URLs (don't use with PDFTeX)
\usepackage[final]{pdfpages}                    %% Include other pdfs
\usepackage[capitalize]{cleveref}				%% Referensing to figures, equations, etc.
\usepackage{units}								%% Appropriate behavior of units
\usepackage[utf8]{inputenc}   				 	%% utf8 support (required for biblatex)
\usepackage{csquotes}							%% Quoted texts are typeset according to rules of main language
\usepackage[style=ieee,doi=false,isbn=false,url=false,date=year,minbibnames=15,maxbibnames=15,backend=biber]{biblatex}
%\renewcommand*{\bibfont}{\footnotesize}		%% Use this for papers
\setlength{\biblabelsep}{\labelsep}
\bibliography{../../bib}

%%==========================================================================
%% Define reference stuff
%%==========================================================================
\crefname{figure}{Figure}{Figures}
\crefname{equation}{}{}

%%==========================================================================
%% Define header/title stuff
%%==========================================================================
\newcommand{\progressreportnumber}{19}
\renewcommand{\author}{Erwin de Gelder}
\renewcommand{\date}{June 19, 2019}
\renewcommand{\title}{Performance assessment of automated vehicles using real-world driving scenarios}

%%==========================================================================
%% Fancy headers and footers
%%==========================================================================
\pagestyle{fancy}                                       %% set page style
\fancyhf{}                                              %% clear all header & footer fields
\fancyhead[L]{Progress report \progressreportnumber}    %% define headers (LE: left field/even pages, etc.)
\fancyhead[R]{\author, \date}                           %% similar
\fancyfoot[C]{\thepage}                                 %% define footer

\begin{document}
	
\begin{center}
	\begin{tabular}{c}
		\title \\ \\
		\textbf{\huge Progress report \progressreportnumber} \\ \\
		\author \\ 
		\date
	\end{tabular}
\end{center}

\section{Previous meeting minutes}

\begin{itemize}
	\item We discussed the comments of Bart on the ontology paper.
	\item We discussed the potential for a new paper on a high-level assessment procedure for a highly automated vehicle.
	\item I need a co-promotor. This co-promotor needs to be appointed by a university.
	\item Bart suggested to organize a colloquium at the TU Delft.
\end{itemize}

\section{Summary of work}

\begin{itemize}
	\item I finished a first draft of the ontology paper.
	\item I received (very useful!) feedback from Mark van den Brand (TU Eindhoven) and Ludwig Friedmann (BMW) on the ontology paper. 
	\begin{itemize}
		\item Both Mark and Ludwig liked the paper.
		\item Mark thinks that the introduction and conclusion need to be improved. The motivation is not strong enough and the applications of the ontology are not clear.
		\item Ludwig proposes to change the definition of scenario to the following definition: ``the scenario describes the dynamic parts of the virtual world while it references the static parts of the virtual world''. 
		
		Apparently, it is not clear that the definition of scenario should be broader than describing only the \emph{virtual} world. 
		
		Furthermore, Ludwig makes a difference between ``describing'' and ``referring'', whereas in the original definition of scenario states that a scenario ``describes'' the static and the dynamic environment. I propose to keep the original definition, but to make clear that the static environment can be ``described'' by simply ``referring'' to an external source. 
		\item Ludwig has some other comments where he asks for clarifications.
	\end{itemize}
	\item I continued with the work on the high-level assessment procedure. I also discussed a potential paper on this subject with Niels de Boer (head of CETRAN, the project in Singapore). Unfortunately, he does not want us to write the paper, because he is afraid of problems with the Land Transport Authority (LTA) (who pays for the project). I am not fully aware of the current relationship between CETRAN and the LTA, but my assumption is that first a few things need to be settled before we can discuss this again. I will still continue the work, because that is part of the job in Singapore. I hope that later we can turn it in a paper anyway, so with that in mind, I continue writing the report.
	\item I did an exam on Applied Statistics, worth 5 graduate school credits. I expect the result in the coming weeks.
\end{itemize}

\section{Future plans}

\begin{itemize}
	\item Process the feedback on the ontology paper. I want to send the paper on June 28 for feedback to Jan-Pieter, Olaf, Hala, and Bart.
	\item Continue with the high-level assessment procedure.
\end{itemize}

\section{Questions}

\begin{itemize}
	\item Previous meeting, we discussed the need for a co-promotor. I thought of the following possibilities:
	\begin{itemize}
		\item Jeroen Ploeg (TU Eindhoven): He knows the subject of my PhD. I am pretty sure that he will help with giving useful feedback on my work too. The only remark is that he might be more of an ``engineer'' than a ``researcher''.
		\item Justin Dauwels (NTU, Singapore): He knows the CETRAN project very well and as such, he is partly aware of my subject. It might be an advantage that he is from Singapore as it gives more of an international status (would this be a good thing?) I am afraid, however, that he will not have much time to review my work.
		\item Joris Sijs (TU Delft, TNO): It might be an advantage that he did his PhD while working for TNO.
		\item Any other (associate/assistant) professor from the TU Delft?
		\item Any other (associate/assistant) professor that appears frequently in my literature list? E.g., Markus Maurer \cite{menzel2018scenarios, schuldt2013effiziente, stolte2017hara, ulbrich2015, bagschik2017ontology, wachenfeld2016usecases, bengler2014threedecades}, Hermann Winner \cite{bengler2014threedecades, geyer2014, wachenfeld2016release, wachenfeld2016usecases, winner2017pegasus}, Philip Koopman \cite{koopman2018toward, koopman2017interdisciplinary, wagner2015philosophy, koopman2016challenges}.
	\end{itemize}
	What would be the three best options?
	\item It might be a bit early, but could we plan a ``progress meeting'' after my second year? That would be somewhere in October. If it is up to me, we do it in a similar fashion as the Go/No Go meeting. Are there other people who can join that meeting, next to Jan-Pieter, Olaf, and Bart?
	\item Can we plan a colloquium around the same time, so that I can reuse the material of the ``yearly progress meeting'' to present during a colloquium?
\end{itemize}


\printbibliography

\end{document}