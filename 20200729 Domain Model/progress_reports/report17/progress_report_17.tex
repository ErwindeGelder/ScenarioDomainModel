\documentclass[10pt,final,a4paper,oneside,onecolumn]{article}

%%==========================================================================
%% Packages
%%==========================================================================
\usepackage[a4paper,left=3.5cm,right=3.5cm,top=3cm,bottom=3cm]{geometry} %% change page layout; remove for IEEE paper format
\usepackage[T1]{fontenc}                        %% output font encoding for international characters (e.g., accented)
\usepackage[cmex10]{amsmath}                    %% math typesetting; consider using the [cmex10] option
\usepackage{amssymb}                            %% special (symbol) fonts for math typesetting
\usepackage{amsthm}                             %% theorem styles
\usepackage{dsfont}                             %% double stroke roman fonts: the real numbers R: $\mathds{R}$
\usepackage{mathrsfs}                           %% formal script fonts: the Laplace transform L: $\mathscr{L}$
\usepackage[pdftex]{graphicx}                   %% graphics control; use dvips for TeXify; use pdftex for PDFTeXify
\usepackage{array}                              %% array functionality (array, tabular)
\usepackage{upgreek}                            %% upright Greek letters; add the prefix 'up', e.g. \upphi
\usepackage{stfloats}                           %% improved handling of floats
\usepackage{multirow}                           %% cells spanning multiple rows in tables
%\usepackage{subfigure}                         %% subfigures and corresponding captions (for use with IEEEconf.cls)
\usepackage{subfig}                             %% subfigures (IEEEtran.cls: set caption=false)
\usepackage{fancyhdr}                           %% page headers and footers
\usepackage[official,left]{eurosym}             %% the euro symbol; command: \euro
\usepackage{appendix}                           %% appendix layout
\usepackage{xspace}                             %% add space after macro depending on context
\usepackage{verbatim}                           %% provides the comment environment
\usepackage[dutch,USenglish]{babel}             %% language support
\usepackage{wrapfig}                            %% wrapping text around figures
\usepackage{longtable}                          %% tables spanning multiple pages
\usepackage{pgfplots}                           %% support for TikZ figures (Matlab/Python)
\pgfplotsset{compat=1.14}						%% Run in backwards compatibility mode
\usepackage[breaklinks=true,hidelinks,          %% implement hyperlinks (dvips yields minor problems with breaklinks;
bookmarksnumbered=true]{hyperref}   %% IEEEtran: set bookmarks=false)
%\usepackage[hyphenbreaks]{breakurl}            %% allow line breaks in URLs (don't use with PDFTeX)
\usepackage[final]{pdfpages}                    %% Include other pdfs
\usepackage[capitalize]{cleveref}				%% Referensing to figures, equations, etc.
\usepackage{units}								%% Appropriate behavior of units
\usepackage[utf8]{inputenc}   				 	%% utf8 support (required for biblatex)
\usepackage{csquotes}							%% Quoted texts are typeset according to rules of main language
\usepackage[style=ieee,doi=false,isbn=false,url=false,date=year,minbibnames=15,maxbibnames=15,backend=biber]{biblatex}
%\renewcommand*{\bibfont}{\footnotesize}		%% Use this for papers
\setlength{\biblabelsep}{\labelsep}
\bibliography{../../bib}

%%==========================================================================
%% Define reference stuff
%%==========================================================================
\crefname{figure}{Figure}{Figures}
\crefname{equation}{}{}

%%==========================================================================
%% Define header/title stuff
%%==========================================================================
\newcommand{\progressreportnumber}{17}
\renewcommand{\author}{Erwin de Gelder}
\renewcommand{\date}{April 15, 2019}
\renewcommand{\title}{Performance assessment of automated vehicles using real-world driving scenarios}

%%==========================================================================
%% Fancy headers and footers
%%==========================================================================
\pagestyle{fancy}                                       %% set page style
\fancyhf{}                                              %% clear all header & footer fields
\fancyhead[L]{Progress report \progressreportnumber}    %% define headers (LE: left field/even pages, etc.)
\fancyhead[R]{\author, \date}                           %% similar
\fancyfoot[C]{\thepage}                                 %% define footer

\begin{document}
	
\begin{center}
	\begin{tabular}{c}
		\title \\ \\
		\textbf{\huge Progress report \progressreportnumber} \\ \\
		\author \\ 
		\date
	\end{tabular}
\end{center}

\section{Previous meeting minutes}

\begin{itemize}
	\item We shortly discussed another potential direction for research, i.e., the quantification of the ``scenario risk'', because I wrote a conference paper on that. It is OK to explore this further for a possible journal paper.
	\item We discussed the progress of the ontology paper.
\end{itemize}

\section{Summary of work}

\begin{itemize}
	\item Some changes are made to the completeness paper. The paper is now accepted and in production.
	\item I discussed with Mascha Toppenburg about whether I could use TNO courses from 2017 for by doctoral education. This is possible. This gives me 7.5 Graduate School Credits (GCS) for the research skills. To actually get these credits, I need to fill in some forms that Bart has to sign. I will bring those forms once I go to Delft again. Furthermore, an online course that I did just before starting the PhD can be used (5 GSC) for discipline-related skills.
	\item I continued the work on the ontology paper. The latest version is attached. Changes are highlighted by the blue text.
\end{itemize}

\section{Future plans}

\begin{itemize}
	\item Finish the ontology paper. The to-dos are mentioned in the paper.
	\item I will be in NL from April 19 till May 10, so I also want to discuss with colleagues from NL on some future plans for my PhD. Although not limited to, this can be on the quantification of the scenario risk and the generation of the test cases.
\end{itemize}

\section{Questions}

\begin{itemize}
	\item I am involved in an ISO working group. The goal of the ISO working group is to define a standard set of test cases for autonomous vehicle (very ambitious). From the last meeting, it appeared that there was no consensus on the terminology. After some hours of non-conclusive discussing, I proposed to take the lead on drafting some kind of a glossary. To minimize any further discussions, I also want to provide the argumentation as to why I use certain definitions. For this, the ontology paper can be perfectly used. However, is it wise to share a preliminary version of the paper with the working group members?
\end{itemize}


\newpage
\includepdf[pages=-,pagecommand={},width=\paperwidth]{../../"20180629 Journal paper ontology"/journal_ontology.pdf}

\end{document}