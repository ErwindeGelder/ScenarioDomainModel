%%==========================================================================
%% This is the very first version of a Streetwise LaTeX template. By
%% chosing the document class, it is easily convertible to a two-column IEEE
%% paper format, provided any conflicting packages and (new)commmands are
%% removed.
%%
%% Erwin de Gelder, January 6, 2018
%%==========================================================================

%%==========================================================================
%% Title:
%% Author(s):
%% To be published in:
%% Date of submission:
%% Date of submission R1:
%% Date of submission R2:
%% Date of final submission:
%%==========================================================================


%% Generic
\documentclass[10pt,final,a4paper,oneside,onecolumn]{article}

%% IEEE
%% Document class options: replace "draftcls" by "final" for final document.
%% All other options may just as well be omitted because they are the default values.
%\documentclass[10pt,final,journal,letterpaper,twoside,twocolumn]{IEEEtran}


%%==========================================================================
%% Document automation
%%==========================================================================

\def\reptitle{Go/No Go report}
\def\repauthor{Erwin de Gelder}


%%==========================================================================
%% Packages
%%==========================================================================

\usepackage[a4paper,left=3.5cm,right=3.5cm,top=3cm,bottom=3cm]{geometry} %% change page layout; remove for IEEE paper format
\usepackage[T1]{fontenc}                        %% output font encoding for international characters (e.g., accented)
\usepackage[cmex10]{amsmath}                    %% math typesetting; consider using the [cmex10] option
\usepackage{amssymb}                            %% special (symbol) fonts for math typesetting
\usepackage{amsthm}                             %% theorem styles
\usepackage{dsfont}                             %% double stroke roman fonts: the real numbers R: $\mathds{R}$
\usepackage{mathrsfs}                           %% formal script fonts: the Laplace transform L: $\mathscr{L}$
\usepackage[pdftex]{graphicx}                   %% graphics control; use dvips for TeXify; use pdftex for PDFTeXify
\usepackage{array}                              %% array functionality (array, tabular)
\usepackage{upgreek}                            %% upright Greek letters; add the prefix 'up', e.g. \upphi

\usepackage[utf8]{inputenc}   				 	%% utf8 support (required for biblatex)
\usepackage[style=ieee,doi=false,isbn=false,url=false,date=year,minbibnames=15,maxbibnames=15,backend=biber]{biblatex}
%\renewcommand*{\bibfont}{\footnotesize}		%% Use this for papers
\setlength{\biblabelsep}{\labelsep}
\bibliography{../bib}

\usepackage{stfloats}                           %% improved handling of floats
\usepackage{multirow}                           %% cells spanning multiple rows in tables
%\usepackage{subfigure}                         %% subfigures and corresponding captions (for use with IEEEconf.cls)
%\usepackage{subfig}                             %% subfigures (IEEEtran.cls: set caption=false)
\usepackage{fancyhdr}                           %% page headers and footers
\usepackage[official,left]{eurosym}             %% the euro symbol; command: \euro
\usepackage{appendix}                           %% appendix layout
\usepackage{xspace}                             %% add space after macro depending on context
\usepackage{verbatim}                           %% provides the comment environment
\usepackage[dutch,USenglish]{babel}             %% language support
\usepackage{wrapfig}                            %% wrapping text around figures
\usepackage{longtable}                          %% tables spanning multiple pages
\usepackage{pgfplots}                           %% support for TikZ figures (Matlab)
\usepackage[breaklinks=true,hidelinks,          %% implement hyperlinks (dvips yields minor problems with breaklinks;
            bookmarksnumbered=true]{hyperref}   %% IEEEtran: set bookmarks=false)
%\usepackage[hyphenbreaks]{breakurl}            %% allow line breaks in URLs (don't use with PDFTeX)
\usepackage{lmodern} 
\usepackage{etoolbox}							%% Needed for apptocmd later
\usepackage[capitalize]{cleveref}
\usepackage{units}
\usepackage{subcaption}
\usepackage{csquotes}							%% Quoted texts are typeset according to rules of main language
\usepackage{xparse}
\usepackage[framemethod=TikZ]{mdframed}
\usepackage[final]{pdfpages}                    %% Include other pdfs

%%==========================================================================
%% Fancy headers and footers
%%==========================================================================

\newtoggle{standalone}
\toggletrue{standalone}
\pagestyle{fancy}                                       %% set page style
\fancyhf{}                                              %% clear all header & footer fields
\iftoggle{standalone}{%
	\fancyhead[L]{Go/No Go} %% define headers (LE: left field/even pages, etc.)
	\fancyhead[R]{Erwin de Gelder, October 2, 2018}                   %% similar
	\fancyfoot[C]{\thepage}                                 %% define footer
	\setlength{\headheight}{18pt}
	\renewcommand*{\headrulewidth}{0.25pt}                  %% header line width
	%% Redefine the default "plain" page style (automatically activated by \maketitle, \section, ...)
	\fancypagestyle{plain}{%
		\fancyhf{}
		\renewcommand{\headrulewidth}{0pt}
		\renewcommand{\footrulewidth}{0pt}
		\setlength{\headheight}{36pt}
	}
}{%
	\renewcommand*{\headrulewidth}{0pt}                  	%% No line in this case
	%% Redefine the default "plain" page style (automatically activated by \maketitle, \section, ...)
	\fancypagestyle{plain}{%
		\fancyhf{}
	}
}
\renewcommand*{\footrulewidth}{0pt}                     %% footer line width


%%==========================================================================
%% TikZ figures
%%==========================================================================

\newlength\figurewidth
\setlength\figurewidth{0.35\textwidth}              %% set figure width
\newlength\figureheight
\setlength\figureheight{0.3\textwidth}              %% set figure height
\pgfplotsset{every axis/.append style={
    scaled y ticks=false,
    scaled x ticks=false,
    y tick label style={/pgf/number format/fixed},
    x tick label style={/pgf/number format/fixed},
    legend style={font=\small}},
    compat=1.9}                                     %% PGFPlots package options
\usetikzlibrary{shapes.geometric, arrows, arrows.meta}
\usepgfplotslibrary{groupplots}
%\usetikzlibrary{external}                           %% Create pdf figures from TikZ. Use PDFTeXify ...
%\tikzexternalize[prefix=./tikz/]                    %% ... with --tex-option=--shell-escape switch.
%\tikzset{external/force remake}                    %% force pdf figure update


%%==========================================================================
%% User-defined commands
%%==========================================================================

\newcommand*{\mat}[1]{\mathbf{#1}}                              %% matrix/vector notation
\newcommand*{\matsym}[1]{\boldsymbol{#1}}                       %% matrix/vector notation for Greek letters
\newcommand*{\T}{^{\scriptscriptstyle\mathsf{T}}}               %% transpose operator
\newcommand*{\Hr}{^{\scriptscriptstyle\mathsf{H}}}              %% conjugate transpose operator
\newcommand*{\ud}{\mathrm{\,d}}                                 %% differential operator (upright d)
\newcommand*{\defeq}{\mathrel{\mathop:}=}                       %% definition sign :=
\newcommand*{\eqdef}{=\mathrel{\mathop:}}                       %% definition sign =:
\newcommand*{\ip}[2]{\left\langle#1\,{,}\,#2\right\rangle}      %% inner product
\newcommand*{\real}[1]{\mathrm{Re}(#1)}                         %% real part
\newcommand*{\imag}[1]{\mathrm{Im}(#1)}                         %% imaginary part
\newcommand*{\lsup}[1]{{}^{#1}\!}                               %% left superscript
\newcommand*{\hi}[1]{$^\text{#1}$}                              %% superscript in normal text
\newcommand*{\lo}[1]{$_\text{#1}$}                              %% subscript in normal text
\newcommand*{\w}[1]{\mathrm{#1}}                                %% multiple character super-/subscript in math mode
\newcommand*{\capskip}{\vspace{-12pt}}                          %% caption skip for figures with subfloats
\newcommand*{\etal}{et al.}                                     %% may be required for Natbib bibliography styles
\renewcommand*{\qedsymbol}{$\blacksquare$}                      %% redefine the end-of-proof symbol
\renewcommand*{\labelitemi}{$\bullet$}                          %% first level item list bullet
\renewcommand*{\labelitemii}{$-$}                               %% second level item list bullet
%\renewcommand*{\theenumi}{\textit{\roman{enumi}}}               %% first level enumerator
\renewcommand*{\labelenumi}{\theenumi.}
\renewcommand*{\theenumii}{\textit{\alph{enumii}}}              %% second level enumerator
\renewcommand*{\labelenumii}{\theenumii.}
\DeclareMathOperator{\tr}{tr}                                   %% trace of a matrix
\DeclareMathOperator{\sgn}{sgn}                                 %% signum function
\DeclareMathOperator{\atan}{atan}                               %% arc tangent

\newcommand{\class}[1]{\emph{#1}}

\crefname{figure}{Figure}{Figures}
\crefname{equation}{}{}
\Crefname{equation}{Equation}{Equations}

%%==========================================================================
%% User-defined environments
%%==========================================================================

\theoremstyle{plain}\newtheorem{definition}{Definition}[section]    %% definition
                    \newtheorem{theorem}{Theorem}[section]          %% theorem
                    \newtheorem{lemma}[theorem]{Lemma}              %% lemma
                    \newtheorem{corollary}[theorem]{Corollary}      %% corollary
                    \newtheorem{assumption}{Assumption}[section]    %% assumption
                    \newtheorem{condition}{Condition}[section]      %% condition
\theoremstyle{definition}\newtheorem{example}{Example}[section]     %% examples
\theoremstyle{remark}\newtheorem{remarkenv}{Remark}[section]        %% remarks
\newenvironment{remark}{\begin{remarkenv}}%
                       {\hfill$\blacklozenge$\end{remarkenv}}       %% end remark with a lozenge
\newenvironment{reviewer}{\itshape}{\upshape}                       %% environment for reviewer's comments


%%==========================================================================
%% Miscellaneous
%%==========================================================================

\graphicspath{{./../}{./figures/}{./more_figures/}} %% (graphicx) directory path for figures
%\setlength{\parindent}{0pt}                        %% no paragraph indentation
%\setlength{\parskip}{2ex}                          %% create empty line between paragraphs
%\interdisplaylinepenalty=2500                      %% (amsmath) allow for page breaks within multiline equations
%\numberwithin{equation}{section}                   %% (amsmath) include section number in equation numbering
%\numberwithin{figure}{section}                     %% (amsmath) include section number in figure numbering
%\numberwithin{table}{section}                      %% (amsmath) include section number in table numbering
\addtolength{\arraycolsep}{-0.5mm}                  %% squeeze matrix columns a little
\fnbelowfloat                                       %% (stfloats) put footnote below a float at the page bottom
\urlstyle{same}                                     %% (hyperref) use current font for URLs
\hypersetup{pdftitle={\reptitle},
            pdfauthor={\repauthor}}                 %% (hyperref) pdf properties title and author
%\raggedbottom                                      %% don't add inter-paragraph spacing to achieve \textheight
%\setlength\subfigcapskip{-3pt}                     %% (subfigure) distance between subfloat and subcaption
%\setlength\subfigbottomskip{-3pt}                  %% (subfigure) distance between subcaption and caption
%\renewcommand*{\subcapsize}{\small}                %% (subfigure) subcaption font size
\renewcommand*{\thesubfigure}{(\alph{subfigure})}   %% (subfig) implement 1(a) instead of 1a ...as subfigure reference
%\captionsetup[subfloat]{labelformat=simple}        %% ... as subfigure reference
%\captionsetup[subfloat]{
%    farskip=0pt,
%    nearskip=8pt,
%    captionskip=1.5pt,
%    labelfont={small,bf},
%    textfont=small}                                %% (subfig) subfloat caption format
%\captionsetup[figure]{
%    labelfont={small,bf},
%    textfont=small}                                %% (subfig/caption) figure caption format
%\captionsetup[table]{
%    aboveskip=2pt,
%    labelfont={small,bf},
%    textfont={small,sc}}                           %% (subfig/caption) table caption format
\apptocmd{\thebibliography}{\raggedright}{}{}		%% Suppress badness warnings in bibliography


% Table stuff
\usepackage{booktabs}
\usepackage{tabularx}
\setlength{\heavyrulewidth}{0.1em}
\newcommand{\otoprule}{\midrule[\heavyrulewidth]}

% Define box for main problem
\mdfdefinestyle{MyFrame}{%
	linecolor=black,
	outerlinewidth=2pt,
	roundcorner=5pt,
	innertopmargin=-2pt,
	innerbottommargin=10pt,
	innerrightmargin=10pt,
	innerleftmargin=10pt}

%\usepackage{changebar}  %% When including the package earlier, it does not work for some reason...
\newlength{\venncircle}
\newcommand{\expectation}[1]{\textup{E} \left[ #1 \right]}
\newcommand{\mise}[1]{\textup{MISE} \left( #1 \right)}

%%==========================================================================
%% Begin document
%%==========================================================================

\begin{document}

\selectlanguage{USenglish}

\title{\textbf{\reptitle}}
\author{\repauthor}
\date{October 2, 2018}
\maketitle

\tableofcontents

\newpage

\section{Biography}
\label{sec:bio}

My academic career started when I began my bachelor in Mechanical Engineering at the Delft University of Technology (TUD) in 2008. I received my B.Sc.\ in 2011 (cum laude). After my bachelor, I immediately started with a master in Systems and Control at TUD. A break of one year allowed me as a member of the board of the Formula Student team of TUD to work on the most awesome race car. In 2014, I received my M.Sc.\ degree (cum laude) after graduating under the supervision of Michel Verhaegen \cite{deGelder2015sabre}.

After graduating from the TUD, I wanted to continue my academic career and, therefore, I started working at the Netherlands Organisation for Applied Scientific Research (TNO). This gave me the opportunity to continue with research while working on various projects, among which applying machine learning for the classification of car-cyclist scenarios \cite{cara2015carcyclist}, working on cooperative driving (e.g., with the Grand Cooperative Driving Challenge), researching personalized Advanced Driving Assistance Systems using the study of naturalistic driving data \cite{gelder2016pacc}, and the assessment of Automated Driving (AD) functions using real-world data \cite{deGelder2017assessment}.

When I was working on various projects related to the assessment of AD functions, my enthusiasm indicated that this was the topic I wanted to focus on for a longer period. For me, having this focus was a prerequisite for allowing me to pursue a PhD degree. As a result, I started as a PhD candidate in October 2017. I chose Bart De Schutter as my supervisor from the TUD as he also supervised Olaf Gietelink with his PhD on a similar topic \cite{gietelink2007phd}. Given their experience in the field of research, Jan-Pieter Paardekooper and Olaf Op den Camp were the logical choices as supervisors from TNO.

Having started my pursue to a PhD degree, TNO started a collaboration with the Centre of Excellence of Testing and Research at the Nanyang Technological University (CETRAN) to work on a certification procedure for automated vehicles (AVs). As a result, I am now working on a scenario-based safety assessment framework for AVs \cite{ploeg2018cetran} in Singapore from September 2017 till September 2019. 

\section{Research}
\label{sec:research}

An introduction of the research, including the main research question, is presented in \cref{sec:introduction}. \Cref{sec:focus} describes more details of the research and the sub-problems. In \cref{sec:paper}, a list of resulted papers and reports is shown. \Cref{sec:outlook} provides an outlook for the coming years of my PhD.


\subsection{Introduction}
\label{sec:introduction}

% Automated Vehicles are introduced (in Singapore)
The development of automated vehicles (AVs) has made significant progress. It is expected that before 2020, automated and autonomous vehicles will be introduced in controlled environments, whereas autonomous vehicles will be mainstream by 2040 \cite{madni2018autonomous} or earlier \cite{bimbraw2015autonomous}. Especially in densely populated cities such as Singapore, there is a need for automated vehicles to increase traffic safety and traffic efficiency by enabling flexible, automated, mobility-on-demand systems \cite{spieser2014toward}, scheduled services for public transport needs, and automated freight and service vehicles to support 24 hours operations and labor shortage needs.

% Assessment of AVs is important
An important aspect in the development of AVs is the safety assessment of the AVs \cite{bengler2014threedecades, stellet2015taxonomy, putz2017pegasus, wachenfeld2016release}. For legal and public acceptance of AVs, a clear definition of system performance is important, as are quantitative measures for the system quality. The more traditional methods \cite{ISO26262, response2006code}, used for evaluation of driver assistance systems, are no longer sufficient for assessment of the safety of higher level AVs, as it is not feasible to complete the quantity of testing required by these methodologies \cite{wachenfeld2016release}. Therefore, the development of assessment methods is important to not delay the deployment of AVs \cite{bengler2014threedecades}.

% Scenario-based approach
One proposed way to assess safety is to test drive AVs in real traffic, observe their performance, and make statistical comparisons to human driver performance. However, this requires AVs to drive hundreds of millions of kilometers and sometimes hundreds of billions of kilometers to demonstrate their reliability in terms of fatalities and injuries \cite{kalra2016driving}. It does not seem to be feasible to drive these millions of kilometers with the increased speed of development of automated driving (AD) functions and the high level of safety requirements that are expected from these functions. As an alternative, a scenario-based assessment is adopted \cite{putz2017pegasus, stellet2015taxonomy, deGelder2017assessment, ploeg2018cetran, elrofai2018scenario}. 
% Scenarios are obtained with data
These test-scenarios can be knowledge-driven or data-driven \cite{stellet2015taxonomy}. A drawback of knowledge-based test scenarios is that it does not allow to generalize the results to the performance of the system-under-test when operating in traffic, i.e., the test cases may not be valid or representative for real-world traffic. A data-driven approach is adopted, as it allows to generalize the results \cite{deGelder2017assessment}. The challenge is, however, to extract the interesting, e.g., performance critical, scenarios from the data, such that the number of simulations is still limited.

In summary, the ultimate goal of this research is a methodology for generating test case that assess the performance of an AV using a data-driven approach. Hence, the main research question is formulated as follows:

\begin{mdframed}[style=MyFrame]
	\paragraph{Main research question:} How can test cases be generated that assess the performance of automated vehicles in real traffic using real-world driving data?
\end{mdframed}

\subsection{Focus of research}
\label{sec:focus}

\Cref{fig:scheme} presents a schematic overview of the process of the assessment of an automated vehicle using real-world driving data. The primary goal of the research is to develop a methodology for the generation of the test cases using the preprocessed data, represented by the green blocks in \cref{fig:scheme}. This is achieved by detecting and parameterizing so-called activities. Using the activities and the static environment, the scenarios can be obtained, after which test cases can be generated for the assessment of the AV.

\tikzstyle{block}=[node distance=5.2em, text width=4.5em, minimum height=7em, align=center, rounded corners=8pt, font=\small]
\tikzstyle{my brace}=[decorate, decoration={brace, amplitude=10pt}]
\tikzstyle{horz}=[node distance=2.7em]
\tikzstyle{vert}=[node distance=4em]
\begin{figure}[b]
	\centering
	\begin{tikzpicture}[scale=0.95, every node/.style={transform shape}]
		% Draw the blocks
		\node[block, fill=blue!30](dc){\includegraphics[width=4em]{data_collection.png} \\ Data collection};
		\node[block, fill=blue!30, right of=dc](dp){\includegraphics[width=2.5em]{data_processing.png} \\ Data preprocessing};
		\node[block, fill=green!30, right of=dp](ed){\includegraphics[width=4.2em]{event_detection.png} \\ Activity detection};
		\node[block, fill=green!30, right of=ed](s){\includegraphics[width=3em]{scenario_mining.png} \\ Scenario mining};
		\node[block, fill=green!30, right of=s](p){\includegraphics[width=3em]{parametrisation.png} \\ Parame-trization};
		\node[block, fill=green!30, right of=p](tc){\includegraphics[width=3.2em]{scenarios.png} \\ Test case genera-tion};
		\node[block, fill=red!30, right of=tc](sim){\includegraphics[width=4em]{simulation.png} \\ Simulation};
		\node[block, fill=red!30, right of=sim](eval){\includegraphics[width=3em]{evaluation.png} \\ Evaluation};
		
		% Show completeness part
		\node[horz, left of=ed](b1){};
		\node[vert, below of=b1](b2){};
		\node[horz, right of=tc](b3){};
		\node[vert, below of=b3](b4){};
		\draw[my brace](b4) -- (b2) node[midway, yshift=-2em, align=center]{Completeness};
	\end{tikzpicture}
	\vspace{-1em}
	\caption{Schematic overview of the process of the assessment of an automated vehicle using real-world data. The green blocks are the focus of the PhD research.}
	\label{fig:scheme}
\end{figure}


% Definition of scenario
First of all, a clear definition of scenario is required to avoid the ambiguity that often arises when talking about \emph{scenarios}. Next to that, related notions need to be defined, e.g., the components that constitute a scenario, such as the activities and the static environment. The goal is to formally define scenario and the related notions, such that these definitions can be directly reflected to, e.g., code and database structure.

% Activity detection
An activity is considered the smallest building block of the dynamic part of the scenario (maneuver of the ego vehicle and the dynamic environment). An activity refers to the behavior of an actor, such as ``braking'' and ``changing lane''. The first step is to detect the activities from the preprocessed data. 

% Scenario mining
In scenario mining, the events and activities that are independently identified for the ego vehicle, the other traffic participants, the static environment, and the conditions are combined to construct a scenario. 

% Parametrization
The recorded scenarios (and its associated activities) are to be stored in a database \cite{elrofai2018scenario}. The scenario database does not contain raw sensor data, but a parametrized model of the real world based on the sensor signals. Therefore, dependency of the stored scenarios on the sensor set with which the scenario was recorded is avoided. The fact that the stored scenarios do not contain the original sensor data brings several benefits: 
\begin{itemize}
	\item The original sensor data might be sensitive as it reveals the sensor setup and processing capabilities of a car. This is much less of an issue when only parameters of the models of the scenarios are stored. Whereas the original sensor data is unlikely to be shared among different parties, the resulting scenarios might be shared, such that the involved parties can benefit from each other.
	\item By condensing the information of all sensors into the least number of parameters that describe the scenario within the error bounds of sensors, substantial reduction in storage volume is achieved.
	\item All the parameters of scenarios of a specific scenario class can be used to construct probability density functions. These probability density functions can be used to generate test cases that lead to probabilistic results \cite{deGelder2017assessment}. Furthermore, it is possible, using the parametrized scenarios, to emphasize scenarios in which the system-under-test shows performance-critical behavior without a-priori knowledge of what scenarios might be critical \cite{deGelder2017assessment}. 
	\item The parametrized models of the scenarios allow for time interpolation. Hence, the stored scenarios can be used for any given sample time, regardless of the sample time of the original sensor data. 
\end{itemize}

% Test case generation
To generate the test cases, the parametrized scenarios are used. More specifically, a probability density function (pdf) of the parameters is estimated. By drawing samples from the pdf, new test cases are generated. The parametrization influences the estimation of the pdf, e.g., the number of parameters and the correlation among the parameters influence the uncertainty of the estimation. In general, the test case generation will be easier when fewer parameters are used to describe a scenario. On the other hand, using too few parameters might result in a loss of important information of the scenario. It needs to be investigated what the balance is between having enough parameters to retain sufficient information while enabling reliable generation of test cases.

% Completeness
On the one hand, the number of scenarios should be limited such that the test load remains feasible and, on the other hand, the scenarios should cover a large part of the variety of the actual scenarios that can be found in real-world traffic. For this reason, it is important to quantify to what extent the collected scenarios and the test cases are \emph{complete}. In \cref{fig:scheme}, this is indicated by the notion of \emph{completeness}. A goal of this research is to quantify the \emph{completeness} of the data.

% Research question
Based on the main research question, many sub-questions can be formulated. To limit the scope of the research, only three sub-questions are formulated. 

\begin{mdframed}[style=MyFrame]
	\paragraph{Sub-question 1:} How to formally define scenario in the context of the assessment of automated vehicles?
	
	\paragraph{Sub-question 2:} How to parametrize scenarios while retaining sufficient information and enabling reliable estimation of the probability density function of the parameters?
	
	\paragraph{Sub-question 3:} How to quantify the completeness of the driving data?
\end{mdframed}

Some preliminary results regarding sub-questions 1 and 3 are presented in \cref{sec:results}.

\subsection{Papers and reports}
\label{sec:paper}

During last year, several papers and reports are written, among which are the following:
\begin{enumerate}
	\item Paper on ontology for scenarios, see \cref{sec:attached papers}, submitted to the Intelligent Vehicles Symposium (IV). Unfortunately, the paper has been rejected. The main criticism was that the paper presents a set of definitions, rather than an ontology.
	\item When working at CETRAN, I worked on a report that describes the framework for the scenario-based assessment of AVs. Part of this report has been turned into a paper for the Intelligent Transportation Systems (ITS) Asia-Pacific Forum \cite{ploeg2018cetran} of which I am the second author, see \cref{sec:attached papers}.
	\item Together with Olaf Op den Camp, I wrote a report describing different scenario classes relevant for Singaporean traffic. To do this, use is made of the ontology defined earlier, so this report might be a good use case of the ontology. We aim to have this document publicly available as soon as it is finalized. \label{item:scenario classes}
	\item The methodology of generating test cases for the assessment of AVs using real-world driving data is called ``StreetWise'' within TNO. TNO wrote a position paper on this topic \cite{elrofai2018scenario}.
\end{enumerate}

\subsection{Outlook}
\label{sec:outlook}

My goal is to write at least four journal papers on the following topics:
\begin{itemize}
	\item \emph{Ontology for scenarios for the assessment of AVs.} This paper extends the work done earlier for the conference paper that is rejected, see \cref{sec:ontology}. UML will be used as the ontology modeling language. Part of the work on the scenario classes (see item \ref{item:scenario classes} of \cref{sec:paper}) will be used as well.
	\item \emph{Quantification of the completeness of traffic data.} This work extends the work partially described in \cref{sec:completeness}. The plan is to first write a conference paper on this topic with a simple case study involving artificial data (of which the true distributions are known). For the journal paper, real-world data will be used as well.
	\item \emph{Framework for the safety assessment of AVs.} Work out the whole pipeline for at least one scenario class, i.e., including completeness, simulation, and evaluation, see \cref{fig:scheme}. This work should extend the work of \cite{ploeg2018cetran}. The goal is to finish this near the end of my period in Singapore, as this paper will be in cooperation with CETRAN.
	\item \emph{Test case generation for the assessment of AVs.} This work includes the parametrization of the scenarios and the estimation of the probability density functions. Additionally, this work should describe how critical test cases (i.e., test cases for which the performance of the AV might be critical) can be generated.
\end{itemize}

\Cref{fig:planning} shows the planning of the aforementioned journal papers together with some preliminary conference papers. To ensure the feasibility of the planning, the work will be done in cooperation with colleagues. Arash Khabbaz Saberi helps with the ontology, Jan-Pieter Paardekooper helps with the completeness, and Olaf Op den Camp will help with the framework (``overall methodology'' in \cref{fig:planning}). 

\begin{figure}
	\centering
	\includegraphics[width=\linewidth]{planning}
	\caption{Planning for the four main topics for which I aim to write a journal paper and the final dissertation. The red line indicates the time of writing this report.}
	\label{fig:planning}
\end{figure}

Next to the aforementioned topics, my goal is to also work on the following topics, which might result in a conference paper.
\begin{itemize}
	\item The whole methodology assumes that real-world data is being collected. Although TNO will depend on the industry to come with data, we will also collect some data ourselves. My aim would be to make part of the data we collect publicly available. I want to write a conference paper to describe the way the data is collected, the details of the data, and the potential use and limitations of the data.
	\item The traditional ISO26262 standard is used to design a system that is functionally safe. This standard is used nowadays for the development for Advanced Driver Assistance Systems. For higher levels of automation, however, this standard is not enough for designing a safe system. One issue with this standard for designing and assessing higher levels of automation is that it only allows to compute the risk of an hazardous event while the risk of a particular scenario is also very relevant. Therefore, I want to work on a paper that quantifies the risk of a scenario (which can turn into a hazardous event).
	\item The detection and classification of activities is the first step to go from the preprocessed data to the generation of test cases. I will work on this, but I do not expect to write a journal paper for this topic.
\end{itemize}

Note that scenario mining is also important, as mentioned in \cref{fig:scheme}. Because scenario mining is already addressed by other people within TNO, I did not list it above.

\section{Personal comments}
\label{sec:personal}

My pursue for a PhD degree is now almost one year underway, so this is a good time to reflect how I personally experienced this. When looking back, I realize that there are some areas I can improve upon. Few points for improvements are mentioned in \cref{sec:evaluation}, but first some other personal comments are listed:
\begin{itemize}
	\item I am very satisfied with my choice for Bart De Schutter as a supervisor from the DUT and I am grateful that he accepted to supervise me. I am amazed by the comprehensiveness and accurateness of his comments and his patience to explain his remarks (again and again).
	\item I am also very satisfied with my supervisors from TNO, Jan-Pieter Paardekooper and Olaf Op den Camp. The discussions I have with them are a huge inspiration for me.
	\item Most importantly, I am still enjoying my pursue for a PhD degree every day. I still feel enthusiasm when talking about my PhD and I am very eager to continue every day with my PhD and work at TNO. 
\end{itemize}

\subsection{Self-evaluation}
\label{sec:evaluation}

I am aware that there are many points for improvement. To limit the following list, I selected two points that probably deserve most attention.

\begin{itemize}
	\item \emph{Time management}:
	\item \emph{Stakeholder management}:
\end{itemize}

\begin{figure}[b]
	\centering
	\includegraphics[width=\linewidth]{figures/stakeholders}
	\caption{Overview of the different stakeholders that are involved in either my work or personal life.}
	\label{fig:stakeholders} 
\end{figure}

%\subsection{Graduate school}
%\label{sec:graduate school}

\section{Doctoral Education}
\label{sec:de}

The Doctoral Education (DE) program requires a total 45 GS credits. The total program is divided into three categories, i.e., research skills (\cref{sec:r skills}), discipline-related skills \cref{sec:d skills}), and transferable skills (\cref{sec:t skills}). A minimum of 15 GS credits is required for each category.

\subsection{Research skills}
\label{sec:r skills}

Research skills will help to reach full proficiency in conducting research. GS credits can be obtained by completing Learning-on-the-job activities. \Cref{tab:research skills} lists the Learning-on-the-job activities that are either completed or planned. All the activities up to Q3 2018 are completed, i.e., a total of 5 GS credits (out of 15 GC credits) are obtained at the time of writing this report.

\begin{table}[b]
	\centering
	\caption{Learning-on-the-job activities for obtaining the 15 GS credits for the research skills.}
	\label{tab:research skills}
	\begin{tabular}{lll}
		\toprule
		Learning-on-the-job activity & When & GS credits \\ \otoprule
		Paper review & Q1 2018 & 1 \\
		Writing first conference paper & Q1 2018 & 1 \\
		Paper review & Q1 2018 & 1 \\
		Addressing a small audience with presentation & Q2 2018 & 0.5 \\
		Addressing a major international audience with presentation & Q2 2018 & 1 \\
		Participation in work consultation with research partners & Q3 2018 & 0.5 \\
		Writing first journal paper & Q1 2019 & 3 \\
		Supervising MSc student & Q2 2019 & 4 \\
		Addressing a small audience with presentation & Q2 2019 & 0.5 \\
		Addressing a major international audience with presentation & Q2 2019 & 1 \\
		Poster presentation for small audience & Q2 2019 & 0.5 \\
		Addressing a major international audience with presentation & Q2 2020 & 1 \\
		\bottomrule
	\end{tabular}
\end{table}

\subsection{Discipline-related skills}
\label{sec:d skills}

Discipline-related skills will improve the breadth of knowledge and, therefore, add to the quality of the doctoral research. No GC credits are earned yet. I will follow the MSc.\ course ``Applied statistics'' (6 ECTS) in the beginning of 2019. Other courses are not planned yet. The following courses might be useful regarding my PhD:
\begin{itemize}
	\item Uncertainty and sensitivity analysis (6 ECTS).
	\item Statistical learning for engineers (4 ECTS).
	\item Machine Learning (5 ECTS).
	\item Models in Software Engineering from the Institute for Programming research and Algorithmics (IPA).
	\item Introduction to R (winter course of the university of Leiden).
	\item Multilevel and Longitudinal Data Analysis (winter course of the university of Leiden).
\end{itemize}

\subsection{Transferable skills}
\label{sec:t skills}

Transferable skills helps to improve on a personal level. The PhD Start-up (2 GS credits) and Career Development (1 GS credit) are mandatory. The remaining 12 GC credits will be obtained by attending the following courses:
\begin{itemize}
	\item 3 GS credits, Q4 2018, Stakeholder management;
	\item 2.5 GS credits, Q2 2019, Effective negotiation: win-win communication;
	\item 2 GS credits, Q4 2019, Self-presentation: Confidence, focus and persuasion;
	\item 2 GS credits, Q2 2020, Communication, coping-strategies \& Awareness;
	\item 3 GS credits, Q4 2020, Writing a dissertation.
\end{itemize}



\printbibliography

%%==========================================================================
%% Appendices
%%==========================================================================

\clearpage
\appendix                                               %% Adapt section numbering
\appendixpage                                           %% Add "Appendices" chapter head
\addappheadtotoc                                        %% Add "Appendices" chapter head to ToC
\renewcommand{\thefigure}{\thesection.\arabic{figure}}  %% Adapt figure numbering
\renewcommand{\thetable}{\thesection.\arabic{table}}    %% Adapt table numbering

% Preliminary results
\section{Preliminary results}
\label{sec:results}

I mainly worked on defining an appropriate ontology regarding scenarios and quantifying completeness. 

\subsection{Ontology of scenario for the assessment of automated vehicles}
\label{sec:ontology}

The notion of scenario is frequently used in the context of automated driving \cite{putz2017pegasus, roesener2017comprehensive, gietelink2006development, hulshof2013autonomous, karaduman2013interactivebehavior, englund2016grand, xu2002effects, ebner2011identifying, ploeg2017GCDC, zofka2015datadrivetrafficscenarios}, despite the fact that an explicit definition is often not provided. However, as mentioned by various authors \cite{stellet2015taxonomy, alvarez2017prospective, zofka2015datadrivetrafficscenarios, aparicio2013pre, lesemann2011test, putz2017pegasus, geyer2014, ulbrich2015}, using a scenario in the context of the development or assessment of AVs requires a clear definition of a scenario. To this end, few definitions of a scenario in the context of (automated) driving have been proposed \cite{geyer2014, ulbrich2015, elrofai2016scenario}. For the context of the scenario database that we aim for \cite{elrofai2018scenario}, however, a more concrete definition of a scenario is required to minimize any ambiguity regarding the scenarios. Ultimately, an ontology is desired to directly use the definitions for the scenario database.

Part of the ontology are the definitions of the different concepts that are adopted. For defining the notion of a scenario, the following characteristics are considered:
\begin{enumerate}
	\item A scenario corresponds to a time interval \cite{go2004blind, geyer2014, ulbrich2015, elrofai2016scenario}.
	\item A scenario consists of one or several events \cite{vannotten2003updated, go2004blind, geyer2014, ulbrich2015, kahn1962, englund2016grand, schoemaker1993multiple, cuppens2002alert, bach2016modelbased}.
	\item Real-world traffic scenarios are quantitative scenarios.
	\item The time interval of a scenario contains all relevant events \cite{geyer2014}.
	\item A scenario can contain goal(s) of one or multiple actors \cite{geyer2014, ulbrich2015, elrofai2016scenario}.
	\item A scenario includes the description of the environment \cite{geyer2014, ulbrich2015, elrofai2016scenario, ebner2011identifying, schuldt2013effiziente, althoff2017CommonRoad}.
\end{enumerate}

Hence, a scenario is defined as follows.
\begin{definition}[Scenario]\label{def:scenario}
	A scenario is a quantitative description of the ego vehicle, its activities and/or goals, its dynamic environment (consisting of traffic environment and conditions) and its static environment. From the perspective of the ego vehicle, a scenario contains all relevant events.
\end{definition}

For the ontology, the other notions, such as ego vehicle, activity, dynamic environment, static environment, conditions, and events, need to be defined as well. However, this is out of scope for this report.

Although a scenario is a quantitative description, there also exists a qualitative description of a scenario. We refer to the qualitative description of a scenario as a \emph{scenario class}. The qualitative description can be regarded as an abstraction of the quantitative scenario.

Scenarios are instances of scenario classes. A scenario class can contain multiple scenarios. On the other hand, a scenario may belong to one or multiple scenario classes. As an example, consider the scenario class ``Day'', which contains all scenarios that occur during the day, and the scenario class ``Rain'', which contains all scenarios with rain, see \cref{fig:venn diagram scenario class}. A scenario that occurs during the night without rain does not belong to any of the previously defined scenario classes. Likewise, a scenario that occurs during the day with rain belongs to both scenario classes ``Day'' and ``Rain''. 

\setlength{\venncircle}{10em}
\begin{figure}
	\centering
	\begin{tikzpicture}
		\fill[red, fill opacity=0.5] (-\venncircle/2, 0) circle (\venncircle);
		\fill[green, fill opacity=0.5] (\venncircle/2, 0) circle (\venncircle);
		\draw (-\venncircle/2, 0) circle (\venncircle);
		\draw (\venncircle/2, 0) circle (\venncircle);
		
		\node[anchor=east](daylight) at (-4/3*\venncircle, 3/4*\venncircle) {Day};
		\draw (daylight) -- ({(-sqrt(3)/2-1/2)*\venncircle}, \venncircle/2);
		\node[anchor=west](rain) at (4/3*\venncircle, 3/4*\venncircle) {Rain};
		\draw (rain) -- ({(sqrt(3)/2+1/2)*\venncircle}, \venncircle/2);
		
		\node[text width=\venncircle, align=center] at (-\venncircle, 0) {Scenarios without rain during day};
		\node[text width=\venncircle, align=center] at (0, 0) {Scenarios with rain during day};
		\node[text width=\venncircle, align=center] at (\venncircle, 0) {Scenarios with rain during night};
	\end{tikzpicture}
	\caption{The two circles correspond to the two scenario classes ``Day'' and ``Rain'', respectively. Scenarios that occur during the day with rain belong to both scenario classes ``Day'' and ``Rain''. The new scenario class ``Day and rain'' can be defined as the class that contains all scenarios that occur during the day with rain. The scenario class ``Day and rain'' is a subclass of the scenario classes ``Day'' and ``Rain''.}
	\label{fig:venn diagram scenario class}
\end{figure}

A scenario class can be a subclass of another scenario class. For example, when we continue our previous example and consider the scenario class ``Day and rain'', this scenario class is a subclass of the scenario classes ``Day'' and ``Rain''. Also, a scenario that occurs during the day with rain now belongs to three scenario classes: ``Day'', ``Rain'', and ``Day and rain''.

\subsection{Quantification of completeness}
\label{sec:completeness}

To draw conclusions on how an AV would perform in real-world traffic, it is necessary to know how representative the scenario database, that is used for the scenario-based assessment of the AV, is. Therefore, it is important to quantify how complete the scenario database is \cite{geyer2014, alvarez2017prospective, stellet2015taxonomy}. To quantify how complete a scenario database is, it is assumed that a scenario is defined according to \cref{def:scenario} and that a scenario class refers to a qualitative description of a scenario, such as described in \cref{sec:ontology}. The problem of quantifying the completeness can now be divided into two subproblems.

\begin{itemize}
	\item The first subproblem deals with the quantification of the completeness regarding the number of scenario classes. 
	\item The second subproblem deals with the quantification of the completeness regarding the scenarios of a specific scenario class.
\end{itemize}

During my research, I only focused on the second subproblem. To address this problem, it is assumed that each scenario of a specific scenario class can be parametrized using similar variables. Therefore, a probability density function (pdf) can be estimated based on the recorded scenarios. This pdf can be used to generate new instances that serve as test cases for the assessment of AVs. To obtain test cases that reflect the real-world traffic, the estimated pdf needs to resemble the true underlying pdf. Therefore, the second subproblem is addressed by quantifying the uncertainty of the estimated pdf.

One way to estimate the uncertainty of an estimated pdf employs the so-called Mean Integrated Squared Error. Let $x \in \mathbb{R}^d$ be the vector of parameters and $f(x)$ the true underlying distribution. The pdf $f(x)$ is estimated using $n$ datapoints. Let $\hat{f}(x;n)$ denote the estimated pdf. This the MISE is defined as follows:
\begin{equation} \label{eq:mise}
	\mise{n} = \expectation{ \int_{\mathbb{R}^d} \left(\hat{f}(x;n) - f(x)\right)^2 \, \textup{d}x} = \int_{\mathbb{R}^d} \expectation{\left(\hat{f}(x;n) - f(x)\right)^2} \, \textup{d}x,
\end{equation}

For calculating the MISE, the true pdf $f(x)$ is required. However, the MISE can be estimated using $\hat{f}(x, n)$. When $f(x)$ is estimated using a parametric statistics, e.g., a Gaussian distribution, the posterior distributions of the parameters of the pdf can be used \cite{bishop2006pattern}, whereas for nonparametric statistics, e.g., Kernel Density Estimation \cite{rosenblatt1956remarks, parzen1962estimation}, the Asymptotic MISE (AMISE) can be employed \cite{chen2017tutorial}. As this report is meant to give an overview of the research, more details on quantifying the completeness are omitted in this report.


% Papers that are written
\section{Papers}
\label{sec:attached papers}
\includepdf[pages=-,pagecommand={},width=\paperwidth]{papers/ontology.pdf}
\includepdf[pages=-,pagecommand={},width=\paperwidth]{papers/its_ap_2018.pdf}

\end{document} 