\section{Biography}
\label{sec:bio}

My academic career started when I began my bachelor in Mechanical Engineering at the Delft University of Technology (TUD) in 2008. I received my B.Sc.\ in 2011 (cum laude). After my bachelor, I immediately started with a master in Systems and Control at TUD. A break of one year allowed me as a member of the board of the Formula Student team of TUD to work on the most awesome race car. In 2014, I received my M.Sc.\ degree (cum laude) after graduating under the supervision of Michel Verhaegen \cite{deGelder2015sabre}.

After graduating from the TUD, I wanted to continue my academic career and, therefore, I started working at the Netherlands Organisation for Applied Scientific Research (TNO). This gave me the opportunity to continue with research while working on various projects, among which applying machine learning for the classification of car-cyclist scenarios \cite{cara2015carcyclist}, working on cooperative driving (e.g., with the Grand Cooperative Driving Challenge), researching personalized Advanced Driving Assistance Systems using the study of naturalistic driving data \cite{gelder2016pacc}, and the assessment of Automated Driving (AD) functions using real-world data \cite{deGelder2017assessment}.

When I was working on various projects related to the assessment of AD functions, my enthusiasm indicated that this was the topic I wanted to focus on for a longer period. For me, having this focus was a prerequisite for allowing me to pursue a PhD degree. As a result, I started as a PhD candidate in October 2017. I chose Bart De Schutter as my supervisor from the TUD as he also supervised Olaf Gietelink with his PhD on a similar topic \cite{gietelink2007phd}. Given their experience in the field of research, Jan-Pieter Paardekooper and Olaf Op den Camp were the logical choices as supervisors from TNO.

Having started my pursue to a PhD degree, TNO started a collaboration with the Centre of Excellence of Testing and Research at the Nanyang Technological University (CETRAN) to work on a certification procedure for automated vehicles (AVs). As a result, I am now working on a scenario-based safety assessment framework for AVs \cite{ploeg2018cetran} in Singapore from September 2017 till September 2019. 
