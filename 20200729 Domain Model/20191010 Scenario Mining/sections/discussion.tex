\cstartf\section{Discussion}\cendf
\label{sec:discussion}

%The case study illustrates that the performance of the scenario mining is limited by the quality of the data rather than the proposed approach.
\cstartd The false detections are a result of inaccurate or missing data.
For example, in case of the four false negatives, the other vehicle is not detected at the time of the cut in or overtaking.
For one cut in, this is because another vehicle obstructs the view toward the vehicle at the moment of the cut in. 
For the other three false negatives, the other vehicles appear from the sensor's blind spot (dotted area in \cref{fig:sensors}).
The three false positives of the cut-in scenario are a result of inaccurate measurements of the lane line distances. \cendd
\cstartf On the one hand, it might be interpreted as that the false detections are due to limitations of the data.
On the other hand, for future work, we can expand our work to deal with these limitations of the data.
For example, using techniques used for correcting the interpretation of natural language \autocite{hull1982experiments}, we might be able to correct wrong tags or to add missing tags. \cendf

% Elaborate on advantages of this work.
\cstartf To mine scenarios from a scenario category, the scenario category needs to be represented by a a certain combination of tags, such as shown in \cref{fig:cutin formulation tags,fig:overtaking template}.
Provided that there are no new tags required, there are no new algorithms required for mining scenarios from new scenario categories. 
As a result, it is relatively straightforward to apply the proposed approach for mining scenarios from other scenario categories than the ones presented in our case study. \cendf
\cstarte Future work includes more tags, e.g., ``turning left'' or ``turning right'', and to consider more actors, e.g., pedestrians and cyclists. This will enable the mining of many more scenarios. \cende

% Elaborate on NLP work.
% Use n-grams for mining scenarios. New possibilities: correcting mistakes, graphical representation, test case generation.
\cstartf For future research, the analogy between the proposed scenario mining and natural language processing (NLP) could be explored. 
In NLP, natural language is analyzed by searching for certain combination of words or syllables. 
Similarly, we are searching for certain combinations of tags. \cendf
\cstarte In NLP, n-gram models are successfully used to correct \autocite{hull1982experiments} and predict \autocite{brown1992class} words and to generate text \autocite{oh2002stochastic}; so n-gram models might be used to correct and predict tags and to generate new scenarios for the assessment of automated vehicles. \cende
