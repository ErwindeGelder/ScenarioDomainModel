\section{Related work}
\label{sec:related}

The question of how much data are enough have been asked in other works. For example, \textcite{guest2006many} ask how many interviews are enough. They use the degree of data saturation and variability to make evidence-based recommendations regarding sample sizes for interviews. In a different application, \textcite{marks2018howmuch} use simulations to determine the necessary sampling size to accurately estimate the perioperative mortality rate (an indicator for surgical quality).

The question of how much data are enough regarding traffic-related applications is less frequently answered. \textcite{wang2017much} even claim to be the first in literature to point out and discuss issues concerning the amount of data needed to understand and model driver behaviors. They propose a statistical approach to determine how much naturalistic driving data are enough for understanding driving behaviors. \textcite{kalra2016driving} estimate the number of miles of driving to demonstrate autonomous vehicle reliability by comparing fatality rates of autonomous vehicles and manually driven vehicles.

To determine the amount of data, \textcite{wang2017much} compare two pdfs: one pdf computed with a bit more data than the other pdf. 

\color{red}


Describe other works that deal with uncertainty of probability distributions:
\begin{itemize}
	\item Separating the Contributions of Variability and Parameter Uncertainty in Probability Distributions \cite{sankararaman2013separating}.
	\item Confidence bands and confidence intervals for Kernel Density Estimation \cite{chen2017tutorial}.
\end{itemize}

\color{black}
