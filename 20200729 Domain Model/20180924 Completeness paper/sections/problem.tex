\section{Problem definition}
\label{sec:problem}

% Explain that we want to quantify the amount of data
The required amount of data depends on the use of the data \cite{wang2017much}. For example, when investigating (near)-accident scenarios from naturalistic driving data, more data might be required compared to studying nominal driving behavior, because of the low probability of having a (near)-accident scenario in naturalistic driving data. Therefore, in this paper, the goal is to define a quantitative measure for the completeness of the data that can be used to determine whether the data are enough.

% Explain assumptions
To define the problem of quantifying the completeness of the data, few assumptions are made:
\begin{enumerate}
	\item The data are interpreted as an endless sequence of scenarios, where scenarios might overlap in time \cite{elrofai2018scenario}. Several definitions of the term scenario in the context of traffic data have been proposed in literature, e.g., by \textcite{geyer2014, ulbrich2015, elrofai2016scenario, elrofai2018scenario}. Because we want to differentiate between quantitative and qualitative descriptions, the definition of the term scenario is adopted from \textcite{elrofai2018scenario} as it explicitly defines a scenario as a quantitative description: ``A scenario is a quantitative description of the ego vehicle, its activities and/or goals, its dynamic environment (consisting of traffic environment and conditions) and its static environment. From the perspective of the ego vehicle, a scenario contains all relevant events.'' Extracting scenarios from data received significant attention and the applied methods are very diverse. For example, \textcite{elrofai2016scenario} use a model-based approach to detect scenarios in which the ego vehicle is changing lane, whereas \textcite{kasper2012oobayesnetworks} use Bayesian networks to detect scenarios with lane changes of other vehicles around the ego vehicle. \textcite{xie2017driving} use a random forest classifier for extracting various scenarios and \textcite{paardekooper2019dataset6000km} employ rule-based algorithms for scenario extraction.
	
	\item Just as \textcite{elrofai2018scenario}, we assume that a scenario consists of activities: ``An activity is considered [to be] the smallest building block of the dynamic part of the scenario (maneuver of the ego vehicle and the dynamic environment).'' An activity describes the time evolution of state variables. For example, an activity can be ``braking'', where the activity describes the evolution of the speed over time. Furthermore, ``the end of an activity marks the start of the next activity'' \cite{elrofai2018scenario}.
	
	\item Though a scenario refers to a quantitative description, these scenarios can be abstracted by means of a qualitative description, referred to as scenario class; see also \textcite{ploeg2018cetran, elrofai2018scenario}. An example of a scenario class could have the name ``ego vehicle braking''; that is, this scenario qualitatively describes a scenario in which the ego vehicle brakes. An actual (real-world) scenario in which the ego vehicle is braking would fall into this scenario class. It is assumed that all scenarios can be categorized into these scenario classes. This assumption does not limit the applicability of this paper, though it might require a large number of scenario classes to describe all scenarios that are in the data.
	
	\item It is assumed that all scenarios that fall into a specific scenario class can be parametrized similarly. As a result, the specific activities that constitute the scenario are also parametrized similarly. As with the previous assumption, this does not limit the applicability of this article. However, it might constrain the variety of scenarios that fall into a scenario class. 
\end{enumerate}

% Define subproblems
Using these assumptions, we can describe the problem of quantifying the completeness of a dataset into three subproblems:
\begin{enumerate}
	\item How to quantify the completeness regarding the scenario classes?
	\item How to quantify the completeness regarding all scenarios that fall into a specific scenario class?
	\item How to quantify the completeness regarding the activities?
\end{enumerate}

% Why splitting problem?
% Comparing different scenario types directly is like comparing apples and oranges
%The first subproblem can be regarded as the so-called unseen species problem \cite{bunge1993estimating, gandolfi2004nonparametric} or species estimation problem \cite{yang2012estimating}. In case of the unseen species problem, the entire population is partitioned into $C$ classes and the objective is to estimate $C$ given only a part of the entire population. To continue the analogy, the second subproblem relates to quantifying whether we have a complete view on the variety among one species, given the number of individuals that we have seen. Both these problems require a different approach, which is why the main problem is divided into these subproblems. The third subproblem addresses only a part of the scenarios, i.e., the activities. In line with the previous analogy, this can be seen as quantifying whether we have a complete view of the parts of the species, e.g., its limbs or organs.

% Mention that we only address sub-problem 3 (and why)
The first step towards quantification of the completeness of the data is to assess the completeness of the activities. The next step is to quantify the completeness of the scenarios, i.e., the combinations of activities. The final step is to quantify the completeness of the scenario classes. In this article,  the first step, i.e., the third subproblem, is addressed. Because of the different approach required for answering the first and second subproblem, those will be addressed in a forthcoming paper. 
%In summary, a method is proposed to answer the following question: \emph{How to quantify the completeness regarding the activities of traffic participants?}
