\section{Introduction}
\label{sec:introduction}

An essential aspect in the development of autonomous vehicles (AVs) is the assessment of AVs \autocite{bengler2014threedecades, stellet2015taxonomy, Helmer2017safety, putz2017pegasus, roesener2017comprehensive, gietelink2006development, wachenfeld2016release}.
For legal and public acceptance of AVs, a clear definition of system performance is important, as well as quantitative measures for system quality. 
Traditional methods, such as \autocite{response2006code, ISO26262}, used for the evaluation of driver assistance systems, are no longer sufficient for the assessment of quality and performance aspects of an AV, because they would require too many resources \autocite{wachenfeld2016release}. 
Therefore, among other methods, a scenario-based approach has been proposed \autocite{elrofai2018scenario, putz2017pegasus}. 

It is important that the collection of scenarios used for the scenario-based assessment of AVs cover the variety of what an AV can encounter during real operation in traffic. 
\Cref{fig:scenario schematic} provides a schematic overview of the required components to describe the test cases that an AV should be subjected to \autocite{elrofai2018scenario}. 
Two different abstraction levels can be considered where the first level qualitatively describes the scenarios and the second level quantitatively describes the scenarios using parameters and models \autocite{degelder2020ontology}. In a similar manner, the AV can be described, such that the relevant test cases can be deduced. 

Because the scenarios need to cover the variety of what an AV can encounter in real traffic, many different scenarios need to be considered. To handle a large number of scenarios, this paper focuses on the qualitative description of the scenarios. Therefore, the scenarios are categorized into so-called scenario categories, where a scenario category can be regarded as an abstraction of a quantitative scenario \autocite{degelder2020ontology}. For example, ``Lead vehicle braking'' is a scenario category referring to the scenarios in which a car in front of the ego vehicle brakes. 

In this paper, we propose a method for defining the scenario categories using a system of tags. Next to proposing the method, we introduce an appropriate system of tags to describe different scenario categories that cover a large portion of the possible varieties that are found in real-world traffic.
It is the objective to get a fair indication of the safe operation of the AV when deployed in real-world traffic by subjecting the AV to test cases (both in physical tests, e.g., on a test track, and virtual simulations using appropriate AV models) based on the scenario categories that are defined using the system of tags proposed in this paper.

This paper does not provide an overview of required test cases. The test cases, however, might be based on the scenario categories that are defined using the presented tags. Before selecting and generating test cases, a match needs to be made of the AV's Operational Design Domain (ODD) \autocite{sae2018j3016} onto the requirements for deployment of the AV in a certain area. % of Singapore. 
Based on the ODD, test cases can be selected for the safety assessment of the AV according to, e.g., ISO~34502 \autocite{ISO34502}. Describing a process for selecting test cases is out of the scope of the current paper. 

\begin{figure*}[t]
	\centering
	\includegraphics[width=\linewidth]{figures/"Scenario schematic"}	
	\caption{A schematic overview of the required components to describe a scenario, and the relation to AV specifications and test cases, based on \cite{elrofai2018scenario}. The red box indicates the focus of the current report.}
	\label{fig:scenario schematic}
\end{figure*}

This paper is organized as follows. In \cref{sec:scenario category}, we explain what we mean with a scenario category and how this relates to scenarios and test cases. Next, in \cref{sec:tags}, we propose a selection of tags, structured in trees, to define scenario categories. Few examples of scenario categories defined using the tags that are proposed in \cref{sec:tags} are presented in \cref{sec:examples}. We conclude this paper in \cref{sec:conclusions}.
