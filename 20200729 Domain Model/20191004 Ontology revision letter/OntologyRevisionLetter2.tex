\documentclass[10pt,final,a4paper,oneside,onecolumn]{article}

%%==========================================================================
%% Packages
%%==========================================================================
\usepackage[a4paper,left=3.5cm,right=3.5cm,top=3cm,bottom=3cm]{geometry} %% change page layout; remove for IEEE paper format
\usepackage[T1]{fontenc}                        %% output font encoding for international characters (e.g., accented)
\usepackage[cmex10]{amsmath}                    %% math typesetting; consider using the [cmex10] option
\usepackage{amssymb}                            %% special (symbol) fonts for math typesetting
\usepackage{amsthm}                             %% theorem styles
\usepackage{dsfont}                             %% double stroke roman fonts: the real numbers R: $\mathds{R}$
\usepackage{mathrsfs}                           %% formal script fonts: the Laplace transform L: $\mathscr{L}$
\usepackage[pdftex]{graphicx}                   %% graphics control; use dvips for TeXify; use pdftex for PDFTeXify
\usepackage{array}                              %% array functionality (array, tabular)
\usepackage{upgreek}                            %% upright Greek letters; add the prefix 'up', e.g. \upphi
\usepackage{stfloats}                           %% improved handling of floats
\usepackage{multirow}                           %% cells spanning multiple rows in tables
%\usepackage{subfigure}                         %% subfigures and corresponding captions (for use with IEEEconf.cls)
\usepackage{subfig}                             %% subfigures (IEEEtran.cls: set caption=false)
\usepackage{fancyhdr}                           %% page headers and footers
\usepackage[official,left]{eurosym}             %% the euro symbol; command: \euro
\usepackage{appendix}                           %% appendix layout
\usepackage{xspace}                             %% add space after macro depending on context
\usepackage{verbatim}                           %% provides the comment environment
\usepackage[dutch,USenglish]{babel}             %% language support
\usepackage{wrapfig}                            %% wrapping text around figures
\usepackage{longtable}                          %% tables spanning multiple pages
\usepackage{pgfplots}                           %% support for TikZ figures (Matlab/Python)
\pgfplotsset{compat=1.14}						%% Run in backwards compatibility mode
\usepackage[breaklinks=true,hidelinks,          %% implement hyperlinks (dvips yields minor problems with breaklinks;
bookmarksnumbered=true]{hyperref}   %% IEEEtran: set bookmarks=false)
%\usepackage[hyphenbreaks]{breakurl}            %% allow line breaks in URLs (don't use with PDFTeX)
\usepackage[final]{pdfpages}                    %% Include other pdfs
\usepackage[capitalize]{cleveref}				%% Referensing to figures, equations, etc.
\usepackage{units}								%% Appropriate behavior of units


\usepackage{silence}  							%% For filtering warnings
\usepackage{csquotes}
% Add doi=false if no DOI
\usepackage[style=authoryear-comp,isbn=false,date=year,backend=biber,maxbibnames=15,maxcitenames=2,uniquelist=false,uniquename=false,giveninits=true]{biblatex}
% Filter warnings issued by package biblatex starting with "Patching footnotes failed"
\WarningFilter{biblatex}{Patching footnotes failed}
%\renewcommand*{\bibfont}{\footnotesize}		%% Use this for papers
\setlength{\biblabelsep}{\labelsep}
\bibliography{../bib}


\pagestyle{fancy}                                       %% set page style
\fancyhf{}                                              %% clear all header & footer fields
%\renewcommand*{\headrulewidth}{1pt}                  	%% No line in this case
\fancyhead[l]{Ontology for Scenarios for the Assessment of Automated Vehicles
}
\fancyhead[r]{TRC\_2019\_1058}
\fancyfoot[c]{\thepage}

\newcommand{\expectation}[1]{\textup{E} \left[ #1 \right]}
\newcommand{\mise}[2]{\textup{MISE}_{#1}\left( #2 \right)}
\newcommand{\amise}[2]{\textup{AMISE}_{#1} \left( #2 \right)}
\newcommand{\measure}[2]{J_{#1} \left( #2 \right)}
\newcommand*{\ud}{\mathrm{\,d}}

\usepackage{titlesec}
\titlespacing*{\paragraph}{0ex}{1ex}{1ex}
\newcommand{\toauthor}{\paragraph*{Comment to authors:} \itshape}
\newcommand{\fromauthor}{\paragraph*{Reply:} \normalfont}
\newcommand{\toauthornew}{\paragraph*{Comment to authors:} \itshape}
\newcommand{\fromauthornew}{\paragraph*{Reply:} \normalfont}
\newcommand{\additionend}[1]{\color{black}[#1]\color{red}}
\newcommand{\addition}[1]{\additionend{#1}\ }

\usepackage{changebar}
\newcommand{\cstart}{\cbstart\color{red}}
\newcommand{\cend}{\cbend\color{black}}


% Notations
\newcommand{\amplitude}{\Delta v}
\newcommand{\attrtstart}{start time}
\newcommand{\attrtend}{end time}
\newcommand{\distancecondition}{d_{\mathrm{v,p}}}
\newcommand{\duration}{T}
\newcommand{\east}{x}
\newcommand{\north}{y}
\newcommand{\head}{\phi}
\newcommand{\egosub}{ego}
\newcommand{\egoeast}{\east_{\mathrm{\egosub}}}
\newcommand{\egonorth}{\north_{\mathrm{\egosub}}}
\newcommand{\egospeed}{\dot{\east}_{\mathrm{\egosub}}}
\newcommand{\egospeedinitsymbol}{v}
\newcommand{\egospeedinit}{\egospeedinitsymbol_{0}}
\newcommand{\egospeedinitb}{\egospeedinitsymbol_{0}}
\newcommand{\egospeedinitc}{\egospeedinitsymbol_{0}}
\newcommand{\egoacceleration}{\ddot{\east}_{\mathrm{\egosub}}}
\newcommand{\egoheading}{\head_{\mathrm{\egosub}}}
\newcommand{\function}{f}
\newcommand{\inputsystem}{u}
\newcommand{\dimensionstate}{n}
\newcommand{\parameter}{\theta}
\newcommand{\parametera}{a}
\newcommand{\parameterb}{b}
\newcommand{\pedsub}{ped}
\newcommand{\pedeast}{\east_{\mathrm{\pedsub}}}
\newcommand{\pednorth}{\north_{\mathrm{\pedsub}}}
\newcommand{\pednorthinit}{\north_{0}}
\newcommand{\pedheading}{\head_{\mathrm{\pedsub}}}
\newcommand{\pedspeed}{\dot{\north}_{\mathrm{\pedsub}}}
\newcommand{\origin}{O}
\newcommand{\scenario}{S}
\newcommand{\scenarioa}{\scenario_{1}}
\newcommand{\scenariob}{\scenario_{2}}
\newcommand{\scenarioc}{\scenario_{3}}
\newcommand{\comprises}{\ni}
\newcommand{\scenariocategory}{\mathcal{C}}
\newcommand{\scenariocategorya}{\scenariocategory_{1}}
\newcommand{\scenariocategoryb}{\scenariocategory_{2}}
\newcommand{\scenariocategoryc}{\scenariocategory_{3}}
\newcommand{\includes}{\supseteq}
\newcommand{\slopeego}{a_{\mathrm{\egosub}}}
\newcommand{\slopepedestrian}{v_{\mathrm{\pedsub}}}
\newcommand{\state}{x}
\newcommand{\statedot}{\dot{\state}}
\renewcommand{\time}{t}
\newcommand{\inittime}{\time_{0}}
\newcommand{\inittimeb}{\time_{1}}
\newcommand{\inittimec}{\time_{2}}
\newcommand{\hasone}{$1$}
\newcommand{\hasn}{$N$}

\begin{document}
	
\section*{Response letter}

Dear Editor and Reviewer,

We thank you for your time and efforts to review the paper and for your constructive comments that were helpful in further improving the overall quality of the paper. We have carefully examined the suggestions raised by the reviewer. In response to these comments, we have prepared a revised version of the paper. A detailed reply to the review comments can be found below.  Additions to the paper are indicated in \cstart red \cend font. Furthermore, a gray bar is shown next to the updated parts of the paper.

Yours sincerely,

Erwin de Gelder, Jan-Pieter Paardekooper, Arash Khabbaz Saberi, Hala Elrofai, Olaf Op den Camp, Jeroen Ploeg, Ludwig Friedmann, Bart De Schutter


	
\section*{Reviewer 1}

\subsection*{A}

%\toauthor Novelty: Similar work has been published before, see e.g.\ \autocite{provine2004ontology, morignot2013ontology, schlenoff2003using, zhao2015core}. Thus, it is unclear what the novelty of the presented ontology is.
%
%\fromauthor Thank you for the mentioned references. In the introduction, we have added the following sentences to explain the new addition of our work in the automotive domain and how our work differs from previous published works: 
%\cstart Ontologies have been widely used in the field of automated driving \autocite{provine2004ontology, morignot2013ontology, schlenoff2003using, zhao2015core, maiti2017conceptualization, benvenuti2017ontologybased, bagschik2017ontology}. However, to the best of our knowledge, we are the first to propose an ontology for scenarios for the assessment of AVs. 
%From the implementation side, there are several file formats and methods for scenario ontologies, e.g., OpenSCENARIO \autocite{openscenario} and CommonRoad \autocite{althoff2017CommonRoad}. Our proposed ontology differs from these implementations as our ontology also allows for qualitative descriptions of scenarios, which is useful because it enables to group scenarios and to interpret the scenarios more easily.
%Furthermore, our ontology is supported with the definitions and justifications of each of the terms.\cend \ldots

\toauthornew It is unclear why qualitative descriptions are not possible for previous ontologies. Being able to precise describe the novelty is crucial for every scientific paper. 

\fromauthornew The previous ontologies are used to describe specific tests, such that these tests can be executed in a virtual environment. In order to be used for testing in a virtual environment, these implementation describe scenarios on a quantitative level and, consequently, they do not provide concepts for a qualitative description of a scenario, because of their focus on scenarios that can be executed in a simulation. We have updated the text to clarify this:

``From the implementation side, there are several file formats and methods for scenario ontologies, e.g., OpenSCENARIO \autocite{openscenario} and CommonRoad \autocite{althoff2017CommonRoad}. 
\cstart These implementations are used to describe specific tests for AVs that can be executed in a virtual environment.
Because of the focus on scenarios that can be simulated, these implementations describe scenarios at a quantitative level and, consequently, they do not provide concepts for a qualitative description of a scenario. \cend
Our proposed ontology differs from these implementations as our ontology also allows for qualitative descriptions of scenarios, which is useful because it enables to group scenarios and to interpret the scenarios more easily.
Furthermore, our ontology is supported with the definitions and justifications of each of the terms.''



\subsection*{B}

%\toauthor It is unclear whether an executable implementation of the proposed ontology exists. This should be clarified in the paper.
%
%\fromauthor To clarify that we implemented the proposed ontology in a coding language, we have added to following line to the introduction (including the footnote): \cstart The implementation code of our ontology is publicly available\footnote{\cstart As a coding language, Python is used. The code is publicly available at \url{https://github.com/ErwindeGelder/ScenarioDomainModel}.\cend}\cend.

\toauthornew I have looked at the code and could not figure out how to execute it. It would be helpful if the authors could provide information how to execute the code.

\fromauthornew We have added a description to the ``readme'' file on the repository that explains how to execute the code.



\subsection*{C}

%\toauthor The paper has several mistakes: 1. event: It is written that an event requires a time step. This, however, is not required, see e.g.\ finite state machines. Only in timed automata and in hybrid systems timing information is required.
%
%\fromauthor To address this comment, we have clarified that we use a hybrid-systems setting to define events and that, therefore, an event happens at a time instant rather than a time span. However, the time does not need to be fully defined in advance: ``\cstart We considered the definitions of event in hybrid systems and control and event-based control. Therefore, an event happens at some time instant rather than taking a time span. Note, however, that the time does not need to be fully defined in advance.\cend''

\toauthornew The authors should have provided one reference that clarifies what kind of hybrid system their model is based on, e.g., is urgent semantics used, what time model is used, etc. Proper definitions could be found here: \url{https://www.cis.upenn.edu/~alur/TCS95.pdf}

\fromauthornew 
We have now added a reference to \autocite{alur1994theory}. We have updated the text as follows:
``\cstart For the definition of event, we consider a hybrid-systems setting with a linear time model \autocite{alur1994theory}. Therefore, an event happens at some time instant.\cend''

%linear time model
%discrete punten in tijd
%left accumulation point, geen right accumulation point
%bouncing ball -> dat tijdsmodel
%hybride automaton -> niet autonoom
%timed content -> wel

\subsection*{D}

%\toauthor 4. ``an event marks a mode transition or the moment a system reaches a threshold'': This is not correct, since an event can be an external event that is broadcast or a user input.
%
%\fromauthor To address this comment, we have clarified that there are two options for defining an event and we provide an example of an event based on a user input:
%\cstart
%Definition~2 indicates that the moment of an event can be defined in two different ways: (1) by the system reaching a threshold or (2) by a mode transition. \cend [...] \cstart The second type of event, i.e., a mode transition, can, e.g., occur at the moment of a driver input.\cend

\toauthornew I do not think that defining terms by examples is a good idea. Please provide a proper definition.

\fromauthornew We have updated the text preceding the definition of an event to further clarify the changes: ``\emph{An event marks a mode transition or the moment a system reaches a threshold:} A mode transition may be \cstart induced \cend by either an abrupt change of an input signal, a change of a parameter, a change in the model, \cstart or an external cause. \cend It is also possible that the event marks the moment that a system reaches a threshold.'' We have removed the examples in this part of the manuscript.

We have updated the definition of the term \emph{event}: ``An event marks \cstart the moment a mode transition occurs or the moment a system reaches a specified threshold, where the former can be induced by both internal and external causes. \cend Before and after an event, the state vector of the system corresponds to two different modes.''

We have changed the text after the definition to further clarify the meaning of the definition of an event: ``Definition~2 indicates that the moment of an event can be defined in two different ways: \cstart(1) by a mode transition or (2) by the system reaching a threshold. The first type of event, i.e., a mode transition can occur at the moment of a sudden driver input. Furthermore, an event might also be induced by an external cause, such as an environmental change. The second type of event, i.e., related to the system reaching a threshold, \cend is especially useful when describing test cases. For example, consider the ego vehicle approaching a pedestrian that is about to cross the road \autocite{seiniger2015test}. 
Here, the event marks the moment that the distance between the vehicle and pedestrian is less than $\distancecondition$ meters. 
At the moment of this event, the pedestrian starts to cross the road such that the vehicle impacts the pedestrian if it does not change its speed or direction \autocite{seiniger2015test}.
By using a variable threshold $\distancecondition$, the value is flexible and can be set differently to define multiple scenarios.''



\subsection*{E}

%\toauthor Also, Fig.~2 is not required to understand the difference between categories and instances. 
%
%\fromauthor The figure is a visualization of the text to improve the understanding of the reader. Therefore, we have opted to keep the figure. However, to shorten the paper, the figure is made smaller.

\toauthornew Since the figure just consists of two overlapping circles, I cannot see how this clarifies anything.

\fromauthornew We have now removed the figure with the overlapping circles.



\subsection*{F}

%\toauthor Renaming of existing concepts. To me, an activity is a state trajectory. Why do we need a new name?
%
%\fromauthor We have chosen the name ``activity'' because --- as explained in Section III-C --- ``activity'' \autocite{elrofai2018scenario,catapult2018musicc} or ``action'' \autocite{ulbrich2015} are part of the existing literature on the subject. In the manuscript, we provide the reason why we opt for ``activity'' over ``action''. To the best of our knowledge, ``state (variable) trajectory'' is less common in the field of safety assessment of (automated) vehicles.

\toauthornew It seems that an activity is nothing else than a state trajectory. Since the authors previously used therms from system theory (event, state, etc.) it seems odd why one would deviate from this domain for one term. This makes the paper quite inconsistent. 

\fromauthornew The reviewer correctly points out that we use terms that are used in system theory, such as \emph{event} and \emph{state}. These terms are, however, also commonly used in the field of safety assessment of automated vehicles. Actually, the terms that we use, i.e., \emph{ego vehicle}, \emph{actor}, \emph{state}, \emph{model}, \emph{mode}, \emph{act}, \emph{static environment}, \emph{dynamic environment}, \emph{scenario}, \emph{event}, \emph{activity}, and \emph{scenario category}, are commonly used in the field of the safety assessment of automated vehicles \autocite{catapult2018musicc,catapult2018regulating,sigsim2019glossary,openscenario,ulbrich2015,geyer2014}\footnote{Note that the term \emph{scenario category} will be used in a future release of OpenSCENARIO.}. To clarify this, we have added the following to the introduction: ``\cstart To define the terms \emph{scenario} and \emph{scenario category}, we use terms that are commonly used in the field of the safety assessment of AVs \autocite{catapult2018musicc,catapult2018regulating,sigsim2019glossary,openscenario,ulbrich2015,geyer2014}. For an unambiguous formulation, some definitions from the field of control theory are adopted.\cend''
% NOTE: Add this at the end of the paragraph starting with "We aim for a definition..."



\subsection*{G}

\toauthornew Fig.\ 4., now Fig.\ 3 is sill way too large. 

\fromauthornew We have now made Fig.\ 3 smaller.




\printbibliography

\end{document}