\section{Introduction}
\label{sec:introduction}

% Automated Vehicles are introduced (in Singapore)
The development of automated vehicles (AVs) has made significant progress. It is expected that before 2020, automated and autonomous vehicles will be introduced in controlled environments, whereas autonomous vehicles will be mainstream by 2040 \cite{madni2018autonomous} or earlier \cite{bimbraw2015autonomous}. Especially in densely populated cities such as Singapore, there is a need for automated vehicles to increase traffic safety and traffic efficiency by enabling flexible, automated, mobility-on-demand systems \cite{spieser2014toward}, scheduled services for public transport needs, and automated freight and service vehicles to support 24 hours operations and labour shortage needs.

% Assessment of AVs is important
An important aspect in the development of autonomous vehicles (AVs) is the safety assessment of the AVs \cite{bengler2014threedecades, stellet2015taxonomy, Helmer2017safety, putz2017pegasus, wachenfeld2016release}. For legal and public acceptance of AVs, a clear definition of system performance is important, as are quantitative measures for the system quality. The more traditional methods \cite{ISO26262, response2006code}, used for evaluation of driver assistance systems, are no longer sufficient for assessment of the safety of higher level AVs, as it is not feasible to complete the quantity of testing required by these methodologies \cite{wachenfeld2016release}. Therefore, the development of assessment methods is important to not delay the deployment of AVs \cite{bengler2014threedecades}.

% Operational safety for AVs
The ultimate goal of this report is to describe a methodology for an independent assessor for assessing the operational safety of an AV. We distinguish four aspects of operational safety: functional safety \cite{ISO26262}, cybersecurity \cite{tr68cybersecurity}, safety of the intended functionality \cite{ISO21448}, and behavioural safety \cite{Waymo2017}. Here, an AV refers to a vehicle equipped with a level 4 automated driving system according to SAE J3016 \cite{sae2018j3016}.

% Literature review
Assessment methodologies for automated driving systems are not new \cite{alvarez2017prospective}. With the development of advanced driver assistance systems and active safety systems, validation methods have been proposed, mainly using simulations of specific scenarios \cite{gietelink2006development, deGelder2017assessment, lesemann2011test}. Assessing higher levels of automation are, however, much more challenging \cite{koopman2017interdisciplinary, koopman2016challenges}. For assessing higher levels of automation, it is proposed to develop a database approach to collect all relevant scenarios \cite{putz2017pegasus, winner2017pegasus, nhtsa2018framework}. These scenarios can be used to identify the capabilities of the AV together with the operating conditions \cite{nhtsa2018framework}. 

% What is good about this assessment
% - Level 4 - One of the first attempts
% - Complete safety assessment -> several aspects
% - Point of view is independent assessor
We propose an assessment strategy for assessing the safety of AVs. As far as we are aware of, there are only few proposals for assessment strategies for AVs; see for example \cite{nhtsa2018framework, zhou2019methodology}. Furthermore, as far as we know, our proposed assessment strategy is the first to consider different aspects of operational safety, being functional safety, cybersecurity, safety of the intended functionality, and behavioural safety. Because our viewpoint is that of an independent assessor that is not involved in the development of the AV, our proposed assessment strategy can be applied by road authorities to assess the road worthiness of AVs.

% Mention limitations of this report
% - Not all details (will be addressed in seperate documents)
% - "only" safety --> not other aspects such as comfort and economical driving
% - Not assessing subsystems
Although we aim for an assessment strategy that addresses many aspects of an AV, there are some limitations. Firstly, because many aspects of the operational safety are addressed, not all the details are provided, because this would result in a very lengthy report. Details of specific parts of the assessment will be addressed in separate documents, however. Secondly, other aspects than safety, such as comfort and economical driving, are not assessed. Note, however, that if very uncomfortable driving leads to potential harm of the AV passengers, it is addressed, since it directly influences the safety of the AV. Thirdly, although the sub-systems of the AVs are addressed, the sub-systems itself are not assessed on its own. This is because our viewpoint is that of an independent assessor where we assume that we only have the availability of the AV as a whole.

% Structure of the report
\todo{Structure of paper.}