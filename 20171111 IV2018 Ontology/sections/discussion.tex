\section{Discussion}
\label{sec:discussion}

% Explain about completeness
% Problem divided into a combinatorical problem and statistical problem
In order to draw conclusions on how an automated vehicle would perform in real-life traffic, it is necessary to know how representative the scenario catalog, which is used for the scenario-based assessment of the automated vehicle, is. Therefore it is important to quantify how complete the scenario catalog is \cite{geyer2014, alvarez2017prospective, stellet2015taxonomy}. Here, definition of scenario and the concept of the scenario classes are helpful. Regarding the variety among the scenario classes, this is a combinatorial problem \cite{geyer2014}. The estimation of the number of scenario classes is very similar to the so-called species estimation problem, wherein the number of species of a population, based on a finite sample, needs to be estimated \cite{yang2012estimating, bunge1993estimating}. The variety within a scenario class, however, cannot be treated as a combinatorial problem, as there are infinite scenarios that satisfy a given qualitative scenario description. For example, when referring to the overtaking example of Section~\ref{sec:example}, the ego vehicle's speed could be any number larger than the speed of the pickup truck. As a result, other statistical measures are required to quantify the variety of the scenarios that belong to a specific scenario class (cf.~\cite{wang2017much}).

An important step towards the quantification of the completeness of the scenario catalog is the definition of the tags. The definition of tags remains future work.

According to Definition~\ref{def:scenario}, the time span of a scenario is determined by the relevant events. It should be noted, however, that this might be subjective. For instance, the relevance of events might depend on the application that is running on the ego vehicle. This poses a challenge for scenario mining for which future research is needed.
