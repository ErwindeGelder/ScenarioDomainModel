\section{Discussion}
\label{sec:discussion}

% Explain about completeness
% Problem divided into a combinatorical problem and statistical problem
In order to draw conclusions on how an automated vehicle would perform in real-life traffic, it is necessary to know how representative the scenario catalogue, which is used for the scenario-based assessment of the automated vehicle, is. Therefore it is important to quantify how complete the scenario catalogue is \cite{geyer2014, alvarez2017prospective, stellet2015taxonomy}. Here, definition of scenario and the concept of the scenario classes are helpful. Regarding the variety among the scenario classes, this is a combinatorial problem \cite{geyer2014}. The estimation of the number of scenario classes is very similar to the so-called species estimation problem, wherein the number of species of a population, based on a finite sample, needs to be estimated \cite{yang2012estimating, bunge1993estimating}. The variety within a scenario class, however, cannot be treated as a combinatorial problem, as there are infinite scenarios that satisfy a given qualitative scenario description. For example, when referring to the overtaking example of Section~\ref{sec:example}, the ego vehicle's speed could be any number larger than the speed of the pickup truck. As a result, other statistical measures are required to quantify the variety of the scenarios that belong to a specific scenario class (cf. \cite{wang2017much}).

\color{red}

Activity vs. goals

Start \& end of scenario still ambiguous

\color{black}