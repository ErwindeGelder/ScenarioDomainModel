\subsection{Definition of events}
\label{sec:events}
% Introduction of this section
A scenario, for which the definition is proposed in Section~\ref{sec:scenario definition}, consists of events. Events can be seen as the `building blocks' of a scenario. The notion of event is extensively used in literature. In this section, a selected number of descriptions is presented. Next, a definition of event is given such that it suits our context.

% Literature review
The term event is used in many different fields, e.g.:
\begin{itemize}
	\item In computing, an event is an action or occurrence recognized by software. A common source of events are inputs by the software users. An event may trigger a state transition \cite{breu1997towards}.
	\item Jeagwon Kim, a philosopher, writes: ``The term event ordinarily implies change'' \cite{kim1993supervenience}. Kim states that an event is composed of three elements: Objects, a property and a time or a temporal interval. 
	\item In probability theory, an event is an outcome or a set of outcomes of an experiment \cite{pfeiffer2013concepts}. For example, a thrown coin landing on its tail is an event.
	%\item ``In relativity, an event is any occurrence with which a definite time and a definite location are associated; it is an idealization in the sense that any actual event is bound to have a finite extent both in time and in space'' \cite{sartori1996understanding}.
	\item In the field of hybrid theory, ``the continuous and discrete dynamics interact at `event' or `trigger' times when the continuous state hits certain prescribed sets in the continuous state space'' \cite{branicky1998hybridcontrol}. ``A hybrid system can be in one of several modes, [...], and the system switches from one mode to another due to the occurrence of events.'' \cite{deschutter2000optimal}.
	\item In event-based control, a control action is computed when an event is triggered, as opposed to the more traditional approach where a control action is periodically computed \cite{heemels2012eventcontrol}. 
\end{itemize}

Before providing the definition of an event, the following is concluded about an event:

% It is a time instant
\subsubsection{An event corresponds to a time instant}
Whether it is regarding computing, philosophy, % relativity \cite{sartori1996understanding}, 
hybrid or event-based control, an event is happening at a time instant.

% Event should mark transition of a state from one set to another - mention relation with hybrid control
\deleted{%
\subsubsection{An event marks the transition of a state}
During a traffic scenario, the states (see Section~\ref{sec:state} for a description of state) are continuously evolving. For example, when a vehicle moves from A to B, the position changes. For the assessment methodology, it is required to parametrize the way the state evolves over time (e.g., see \cite{deGelder2017assessment}). Therefore, specific models will be employed to describe this (see Section~\ref{sec:model} for a description of model). These models, each with a fixed number of parameters, will only be valid for a certain time interval. Therefore, events mark the transition from one model that is describing the state to another model describing the same state.
}

\added{%
\subsubsection{An event marks a mode transition}
A mode transition may be caused by a change of either an input signal, a parameter, or a state. For example, pushing the brake pedal may cause a mode transition and therefore, this may be regarded as an event. 
}

\deleted{%
\subsubsection{The inter-event time interval corresponds to an activity}
The inter-event time interval represents the time in between two events in which a certain activity occurs, e.g., `braking' or `cruising'.
}

%\subsubsection{An event marks (a cause of) a mode transition}
%Events mark the transition of mode, which is either a change of input, parameter or state. This is analogous to the way event is described in hybrid control \cite{boel1999hybridcontrol}.

% Give definition
Hence, we define an event as follows.
\begin{definition}[Event] \label{def:event}
	\deleted{%
	An event marks the time instant at which a transition of a state occurs, such that before and after an event, the state corresponds to two different activities.  %Furthermore, it marks the start and end of a time interval that can be qualitatively described.
	}

	\added{An event marks the time instant at which mode transition occurs, such that before and after an event, the state corresponds to two different modes. }
\end{definition}

% Compare definition with literature and other remarks
% - Mention that it is basically similar to the definition adopted with hybrid control theory
% - Mention that before and after, different qualitative description
An event according to Definition~\ref{def:event} is related to an event described in hybrid control \cite{deschutter2000optimal}, where an event describes the transition from one mode of operation to another mode of operation. In our application, i.e. events in traffic scenarios, ``mode of operation'' is described by a certain model with parameters assigned to it.

\added{The inter-event time interval represents the time in between two events in which a certain activity occurs. For example, when the longitudinal acceleration is negative during such an inter-event time interval, the activity can be described by the word `braking'. Another example is when at the event, the head lights are turned on. In that case, the activities before and after the event can be described as `lights off' and `lights on', respectively.}

\deleted{During the time interval between two events, or, in short, the inter-event time interval, a state is quantitatively described by a model. It is, however, possible to describe the way the state evolves during the inter-event time interval, i.e., the activity, in a qualitative manner. For example, when the longitudinal acceleration is negative during such an inter-event time interval, the activity can be described by the word `braking'. Another example is when at the event, the head lights are turned on. In that case, the activities before and after the event can be described as `lights off' and `lights on', respectively.}
