\subsection{Definition of scenario}
\label{sec:scenario definition}
% Scenario typology according to Van Notten et al.
Van Notten et al.\ \cite{vannotten2003updated} present an overview of the diversity in the scenarios. Van Notten et al.\ state that ``in view of the observed variety in scenario approaches'', it is assumed ``that there is no `correct' scenario definition or approach. However, the typology uses the following broad working definition: scenarios are descriptions of possible futures that reflect different perspectives on the past, the present and the future.''

% Techniques for scenario development
Where the contribution of Van Notten et al.\ \cite{vannotten2003updated} relate more to the overall scenario project, Bishop et al.\ \cite{bishop2007scentechniques} focus more on the scenario techniques, i.e.\ the process of creating scenarios. These scenario development techniques vary from genius forecasting, event sequences with probability trees and sensitivity analysis \cite{bishop2007scentechniques}.

% How to apply to real-life traffic scenarios
If it was not clear yet, the contributions of Van Notten et al.\ \cite{vannotten2003updated} and Bishop et al.\ \cite{bishop2007scentechniques} point out that the notion of \emph{scenario} is very broad. We are interested in scenarios that can be used in the context of road traffic, which limits the scope of what a scenario is. The description of a scenario by Go and Carroll \cite{go2004blind} is more suited to our needs.
% Definition according to Go and Carroll
Go and Carroll \cite{go2004blind} describe a scenario within the field of system design as ``a description that contains (1) actors, (2) background information on the actors and assumptions about their environment, (3) actors' goals or objectives, and (4) sequences of actions and events. Some applications may omit one of the elements or they may simply or implicitly express it. Although, in general, the elements of scenarios are the same in any field, the use of scenarios is quite different.'' 

% Definition according to Geyer et al.
Several definitions of a scenario can be directly applied to traffic scenarios. For example, Geyer et al.\ \cite{geyer2014} uses the metaphor of a movie or a storybook for describing a scenario. Geyer et al.\ states that ``a scenario includes at least one situation within a scene including the scenery and dynamic elements. However, [a] scenario further includes the ongoing activity of one or both actors.'' For a further explanation of the terms situation, scene and scenery, see \cite{geyer2014}. It is mentioned that the action of the driver and/or automation might be predefined. Here, the meaning of action is not detailed. In an example of a so-called crossway scenario, they mention that the course of events might be different. For example, when a car keeps constant speed and then turns right, the scenario consists of one situation. The car might also first decelerate, accelerate and decelerate before turning right. In this case, the scenario consists of four situations.

% Definition according to Ulbrich et al.
Ulbrich et al.\ \cite{ulbrich2015} define a scenario in the context of automated driving. They define a scenario as ``the temporal development between several scenes in a sequence of scenes. Every scenario starts with an initial scene. Actions \& events as well as goals \& values may be specified to characterize this temporal development in a scenario. Other than a scene, a scenario spans a certain amount of time.'' They state that actions and events link the different scenes. A further description of actions and events is not given.

% Definition according to Elrofai et al.
Another definition of a scenario in the context of automated driving is given by Elrofai et al.\ \cite{elrofai2016scenario}. They define scenario as ``the combination of actions and manoeuvres of the host vehicle in the passive [i.e., static] environment, and the ongoing activities and manoeuvres of the immediate surrounding active [i.e., dynamic] environment for a certain period of time.'' They further mention that the duration of a scenario typically is in the order of seconds.

% "Requirements"
Based on the aforementioned references, the following is concluded about a scenario.

% Order of seconds
\subsubsection{A scenario corresponds to a time interval}
Van Notten et al.\ \cite{vannotten2003updated} call such a scenario a chain scenario (``like movies''), as opposed to a snapshot scenario, i.e. a scenario that describes the state at a time instant (``like photos''). The definitions mentioned above mention that a scenario corresponds to a time interval \cite{go2004blind, geyer2014, ulbrich2015, elrofai2016scenario}. The duration of a scenario is in the order of seconds, as explicitly mentioned by Elrofai et al.\ \cite{elrofai2016scenario}. Though the duration is not mentioned by Ulbrich et al.\ \cite{ulbrich2015}, the presented example is in the order of seconds. Furthermore, other scenarios regarding (automated) driving are also in the order of seconds, e.g., see \cite{gietelink2006development, zofka2015datadrivetrafficscenarios, roesener2017comprehensive, karaduman2013interactivebehavior, hulshof2013autonomous, englund2016grand}.

% Scenarios consists of one or several events
\subsubsection{A scenario consists of one or several events \cite{vannotten2003updated, go2004blind, geyer2014, ulbrich2015, kahn1962, englund2016grand, schoemaker1993multiple, cuppens2002alert, bach2016modelbased}}
As already stated by Bishop et al.\ \cite{bishop2007scentechniques}, it can be helpful to develop scenarios using events. Thus, a scenario could be defined as a particular sequence of events.  Kahn writes that ``a scenario results from an attempt to describe in more or less detail some hypothetical sequence of events'' \cite{kahn1962}. Geyer et al.\ \cite{geyer2014} and Ulbrich et al.\ \cite{ulbrich2015} use the term event (``course of events'' and ``events \& actions'', respectively), although they do not provide a definition of the term \emph{event}. In the next section, we will elaborate on the notion of \emph{event}.

% Semantically described
\subsubsection{Real-life traffic scenarios are quantitative scenarios}
Regarding the nature of the data, a scenario can be either qualitative or quantitative \cite{vannotten2003updated}. Real-life traffic scenarios are quantitative scenarios, such that it is, e.g., suitable for simulation purposes. A scenario, however, can be qualitatively described, such that it is readable and understandable for human experts. This has become known as a Story-and-Simulation approach \cite{alcamo2001scenarios}. Note that several quantitative scenarios might have the same qualitative description, thus a qualitative description of a scenario does not uniquely define a quantitative scenario. A qualitative description can be regarded as an abstraction of the quantitative scenario.
	
% Some relevance between events
\subsubsection{The time interval of a scenario contains all relevant events}
According to Geyer et al.\ \cite{geyer2014}, ``the end of a scenario is defined by the first irrelevant situation with respect to the scenario''. In a similar manner, we require that the time interval of a scenario should contain all relevant events. For practical reasons, the term `relevant' must be detailed. First of all, it is important to note that `relevant' is subjective. Therefore, an event is considered to be relevant for the scenario, if it is relevant for the ego vehicle (see section \ref{sec:ego vehicle} for a description of the ego vehicle). Next to that, an event is regarded as irrelevant, if it is independent of the relevant events.
	
% Description of static environment
\subsubsection{A scenario includes the description of the static environment}
A scenario should include the description of the static environment, e.g., road layout and weather information. It is assumed that the static environment does not change during a scenario. Although this is not a general prerequisite of a scenario, the description of the static environment is often included when speaking about traffic scenarios \cite{geyer2014, ulbrich2015, elrofai2016scenario, hulshof2013autonomous, ebner2011identifying, schuldt2013effiziente}.

% Definition
A scenario is defined as follows: \emph{A scenario is a quantitative description of the ego vehicle, its activity, its static environment and its dynamic environment. From the perspective of the ego vehicle, it contains the relevant events.}
