\section{Introduction}
\label{sec:introduction}

% For validating all traffic scenarios
An important aspect in the development of automated vehicles is the assessment of the safety, comfort and efficiency of the automated vehicles (AVs) \cite{bengler2014threedecades, stellet2015taxonomy, Helmer2017safety, putz2017pegasus, roesener2017comprehensive, gietelink2006development}. For legal and public acceptance, it is important that there is a clear definition of system performance and quantitative measures for the system performance. For the assessment, a crucial aspect is the test scenarios \cite{stellet2015taxonomy}. This paper proposes an ontology for these \emph{scenarios}.

% Explain importance/relevance
The notion of scenario is frequently used in the context of automated driving, although an explicit definition is often not provided \cite{putz2017pegasus, roesener2017comprehensive, gietelink2006development, hulshof2013autonomous, karaduman2013interactivebehavior, englund2016grand, xu2002effects, ebner2011identifying, ploeg2017GCDC, zofka2015datadrivetrafficscenarios}. When using the notion of a scenario in the context of the development or assessment of automated vehicles, however, a clear definition is required \cite{stellet2015taxonomy, Helmer2017safety, alvarez2017prospective, zofka2015datadrivetrafficscenarios, aparicio2013pre, lesemann2011test, putz2017pegasus, geyer2014, ulbrich2015}. To this end, several definitions of a scenario in the context of (automated) driving have been proposed \cite{geyer2014, ulbrich2015, elrofai2016scenario}.

% What is the added value of this paper w.r.t. other work
% - Easier to apply event/scenario mining, because more concrete
% - Important step towards quantification of completeness
% - More complete (also event defined etc.)
% - Backed up with literature (should we mention this???)
% - Real example given
We aim for a definition of a scenario that is, on the one hand, broadly consistent with existing definitions \cite{geyer2014, ulbrich2015, elrofai2016scenario} while, on the other hand, more concrete, such that it is more applicable for scenario mining \cite{elrofai2016scenario} and scenario-based assessment \cite{stellet2015taxonomy, deGelder2017assessment}. We intend to provide a more concrete definition by providing more details of a scenario and by defining the attributes of a scenario in detail. For example, the notion of event - which is part of a scenario - will be detailed. A concrete real-life example is provided which demonstrates the use of the ontology with a real-life scenario.

% Outline
The outline of the paper is as follows. Section~\ref{sec:nomenclature} specifies notions that are adopted to define a scenario. Section~\ref{sec:ontology} describes the ontology of real-life scenarios in the context of automated driving. First, a definition of a \emph{scenario} is provided, after which the more abstract and generic concept of \emph{scenario classes} is explained. In Section~\ref{sec:example}, an application example is provided to demonstrate the use of the ontology with a real-life scenario. The paper is concluded with a discussion in Section~\ref{sec:discussion} and a conclusion in Section~\ref{sec:conclusion}.
