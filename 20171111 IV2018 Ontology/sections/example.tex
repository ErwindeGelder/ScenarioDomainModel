\section{Application example}
\label{sec:example}

To illustrate the ontology, a real-world example is presented. A schematic overview of the example is shown in \cref{fig:example schematic}. Two vehicles are considered: a pickup truck and a sedan. The example can be described in words as follows. The ego vehicle (i.e., the sedan) accelerates towards its predecessor (i.e., the pickup truck), such that the distance between the ego vehicle and its predecessor becomes smaller. At some point, the ego vehicle brakes, such that it reaches approximately the same speed as the pickup truck's speed. From then on, the ego vehicle cruises, such that the distance between the two vehicles remains approximately constant. The pickup truck drives at a constant speed.

\begin{figure}
	\centering
	\setlength\figureheight{121pt}
	\setlength\figurewidth{260pt}
	% This file was created by matplotlib2tikz v0.6.14.
\begin{tikzpicture}

\begin{axis}[
xmin=-15, xmax=15,
ymin=-5, ymax=5.3,
width=\figurewidth,
height=\figureheight,
tick align=outside,
tick pos=left,
x grid style={white!69.019607843137251!black},
y grid style={white!69.019607843137251!black},
axis background/.style={fill=white!90.0!black},
ticks=none,
hide axis
]
\path [draw=white, fill=white] (axis cs:-20,3.5)
--(axis cs:20,3.5)
--(axis cs:20,-3.5)
--(axis cs:-20,-3.5)
--cycle;

\addplot [semithick, black, forget plot]
table {%
-20 3.5
20 3.5
};
\addplot [semithick, black, forget plot]
table {%
-20 -3.5
20 -3.5
};
\addplot [semithick, black, forget plot]
table {%
-20 0
-15.5555555555556 0
};
\addplot [semithick, black, forget plot]
table {%
-6.66666666666666 0
-2.22222222222222 0
};
\addplot [semithick, black, forget plot]
table {%
6.66666666666667 0
11.1111111111111 0
};
\addplot [red, forget plot]
table {%
12.25 1.77571428571429
12.25 1.36428571428571
12.1323529411765 1.08142857142857
11.4852941176471 0.927142857142857
9.33823529411765 0.927142857142857
9.48529411764706 0.798571428571428
9.33823529411765 0.927142857142857
8.30882352941176 0.927142857142857
7.89705882352941 1.08142857142857
7.75 1.44142857142857
7.75 1.75
7.75 1.46714285714286
8.72058823529412 1.13285714285714
};
\addplot [red, forget plot]
table {%
12.25 1.72428571428571
12.25 2.13571428571429
12.1323529411765 2.41857142857143
11.4852941176471 2.57285714285714
9.33823529411765 2.57285714285714
9.48529411764706 2.70142857142857
9.33823529411765 2.57285714285714
8.30882352941176 2.57285714285714
7.89705882352941 2.41857142857143
7.75 2.05857142857143
7.75 1.75
7.75 2.03285714285714
8.72058823529412 2.36714285714286
};
\addplot [red, forget plot]
table {%
11.1617647058824 1.03
10.7794117647059 1.13285714285714
9.89705882352941 1.13285714285714
9.48529411764706 1.03
};
\addplot [red, forget plot]
table {%
11.1617647058824 2.47
10.7794117647059 2.36714285714286
9.89705882352941 2.36714285714286
9.48529411764706 2.47
};
\addplot [red, forget plot]
table {%
11.1617647058824 1.75
11.1617647058824 1.57
11.0441176470588 1.21
11.3970588235294 1.21
11.6323529411765 1.28714285714286
11.75 1.39
11.8382352941176 1.59571428571429
11.8382352941176 1.75
};
\addplot [red, forget plot]
table {%
11.1617647058824 1.75
11.1617647058824 1.93
11.0441176470588 2.29
11.3970588235294 2.29
11.6323529411765 2.21285714285714
11.75 2.11
11.8382352941176 1.90428571428571
11.8382352941176 1.75
};
\addplot [red, forget plot]
table {%
9.69117647058824 1.75
9.69117647058824 1.41571428571429
9.72058823529412 1.21
9.01470588235294 1.03
8.86764705882353 1.62142857142857
8.86764705882353 1.75
};
\addplot [red, forget plot]
table {%
9.69117647058824 1.75
9.69117647058824 2.08428571428571
9.72058823529412 2.29
9.01470588235294 2.47
8.86764705882353 1.87857142857143
8.86764705882353 1.75
};
\addplot [blue, forget plot]
table {%
-3.22093023255814 1.75
-3.22093023255814 1.73767123287671
-4.50290697674419 1.73767123287671
-4.50290697674419 1.75
};
\addplot [blue, forget plot]
table {%
-3.22093023255814 1.75
-3.22093023255814 1.76232876712329
-4.50290697674419 1.76232876712329
-4.50290697674419 1.75
};
\addplot [blue, forget plot]
table {%
-3.22093023255814 1.49109589041096
-4.50290697674419 1.49109589041096
-4.50290697674419 1.51575342465753
-3.22093023255814 1.51575342465753
-3.22093023255814 1.49109589041096
};
\addplot [blue, forget plot]
table {%
-3.22093023255814 2.00890410958904
-4.50290697674419 2.00890410958904
-4.50290697674419 1.98424657534247
-3.22093023255814 1.98424657534247
-3.22093023255814 2.00890410958904
};
\addplot [blue, forget plot]
table {%
-3.22093023255814 1.21986301369863
-4.50290697674419 1.21986301369863
-4.50290697674419 1.24452054794521
-3.22093023255814 1.24452054794521
-3.22093023255814 1.21986301369863
};
\addplot [blue, forget plot]
table {%
-3.22093023255814 2.28013698630137
-4.50290697674419 2.28013698630137
-4.50290697674419 2.25547945205479
-3.22093023255814 2.25547945205479
-3.22093023255814 2.28013698630137
};
\addplot [blue, forget plot]
table {%
-2.75 1.75
-2.75 1.07191780821918
-2.82848837209302 0.997945205479452
-2.9593023255814 0.997945205479452
-2.9593023255814 1.75
-2.9593023255814 0.923972602739726
-3.03779069767442 0.85
-5.99418604651163 0.874657534246575
-5.83720930232558 0.726712328767123
-5.99418604651163 0.874657534246575
-6.75290697674419 0.899315068493151
-7.09302325581395 1.02260273972603
-7.25 1.63904109589041
-7.25 1.75
};
\addplot [blue, forget plot]
table {%
-2.75 1.75
-2.75 2.42808219178082
-2.82848837209302 2.50205479452055
-2.9593023255814 2.50205479452055
-2.9593023255814 1.75
-2.9593023255814 2.57602739726027
-3.03779069767442 2.65
-5.99418604651163 2.62534246575343
-5.83720930232558 2.77328767123288
-5.99418604651163 2.62534246575343
-6.75290697674419 2.60068493150685
-7.09302325581395 2.47739726027397
-7.25 1.86095890410959
-7.25 1.75
};
\addplot [blue, forget plot]
table {%
-4.65988372093023 1.75
-4.65988372093023 0.997945205479452
-4.8953488372093 1.09657534246575
-4.8953488372093 1.75
};
\addplot [blue, forget plot]
table {%
-4.65988372093023 1.75
-4.65988372093023 2.50205479452055
-4.8953488372093 2.40342465753425
-4.8953488372093 1.75
};
\addplot [blue, forget plot]
table {%
-5.83720930232558 1.75
-5.83720930232558 1.44178082191781
-5.75872093023256 1.07191780821918
-6.28197674418605 0.973287671232877
-6.36046511627907 1.46643835616438
-6.36046511627907 1.75
};
\addplot [blue, forget plot]
table {%
-5.83720930232558 1.75
-5.83720930232558 2.05821917808219
-5.75872093023256 2.42808219178082
-6.28197674418605 2.52671232876712
-6.36046511627907 2.03356164383562
-6.36046511627907 1.75
};
\addplot [blue, forget plot]
table {%
-3.11627906976744 1.75
-3.11627906976744 1.02260273972603
-4.52906976744186 1.02260273972603
};
\addplot [blue, forget plot]
table {%
-3.11627906976744 1.75
-3.11627906976744 2.47739726027397
-4.52906976744186 2.47739726027397
};
\addplot [semithick, red, dashed, forget plot]
table {%
7.75 1.75
1.75 1.75
};
\addplot [semithick, red, forget plot]
table {%
2.5 2.5
1.75 1.75
2.5 1
};
\addplot [semithick, blue, dashed, forget plot]
table {%
-7.25 1.75
-11.25 1.75
};
\addplot [semithick, blue, forget plot]
table {%
-10.5 2.5
-11.25 1.75
-10.5 1
};
\node at (axis cs:10,5.3)[
  scale=0.8,
  anchor=north,
  text=black,
  rotate=0.0
]{ Sedan};
\node at (axis cs:-5,5.3)[
  scale=0.8,
  anchor=north,
  text=black,
  rotate=0.0
]{ Pickup truck};
\end{axis}

\end{tikzpicture}%
	\caption{Schematic overview of the traffic scenarios. The sedan (red, right vehicle) is defined as the ego vehicle. Initially, the ego vehicle accelerates towards the pickup truck (blue, left vehicle).}
	\label{fig:example schematic}
	\spaceaftercaption
\end{figure}

In the presented example, two scenarios are identified. The scenarios can be qualitatively described as `gap-closing on rural road with clear weather' and `cruising behind target vehicle on rural road with clear weather', respectively. The first scenario ends when the ego vehicle starts cruising. 
% Note that, when testing an AV, the first scenario might end later in case it takes longer for the AV to start the cruising activity.

The activities of the ego vehicle are shown in \cref{fig:example ego states}, in which the speed is shown with respect to time $t$. Regarding the speed, two events are identified. At the first event, the ego vehicle's longitudinal acceleration transits from being positive to negative, i.e., the ego vehicle starts braking. As there are two events identified regarding the speed, three different activities are to be modeled, qualitatively described as `accelerating', `braking', and `cruising', respectively. 

Whereas the definitions from Geyer~et~al.~\cite{geyer2014}, Ulbrich~et~al.~\cite{ulbrich2015}, and Elrofai~et~al.~\cite{elrofai2016scenario} do not state when a scenario ends, \cref{def:scenario} states that a scenario contains all the relevant events. Therefore, it follows from \cref{def:scenario} that the first scenario ends at the second event.

For a scenario-based assessment, test cases have to be defined. These test cases can be derived from real-world scenarios \cite{zofka2015datadrivetrafficscenarios,stellet2015taxonomy,deGelder2017assessment}. For example, based on the first scenario, a test case can be defined where the ego vehicle's objective is to follow the pickup truck while maintaining a safe distance and where the ego vehicle's initial speed is higher than the speed of the pickup truck. Other conditions can be applied for the test case, e.g., bad weather conditions that might cause a late detection, to assess the driving capabilities of the AV in different scenarios.

\begin{figure}
	\centering
	\setlength\figureheight{130pt}
	\setlength\figurewidth{248pt}
	% This file was created by matplotlib2tikz v0.6.14.
\begin{tikzpicture}

\definecolor{color0}{rgb}{0.12156862745098,0.466666666666667,0.705882352941177}

\begin{axis}[
xlabel={$t$ [s]},
ylabel={Speed [km/h]},
xmin=0, xmax=17.5,
ymin=35, ymax=70,
width=\figurewidth,
height=\figureheight,
tick align=outside,
tick pos=left,
xmajorgrids,
x grid style={white!69.019607843137251!black},
ymajorgrids,
y grid style={white!69.019607843137251!black},
clip marker paths
]
\addplot [line width=2.4000000000000004pt, color0, forget plot]
table {%
0 37.6080419972848
0.025 37.7152844154791
0.05 37.822339777017
0.075 37.9292080818984
0.1 38.0358893301232
0.125 38.1423835216916
0.15 38.2486906566035
0.175 38.3548107348589
0.2 38.4607437564578
0.225 38.5664897214002
0.25 38.6720486296862
0.275 38.7774204813156
0.3 38.8826052762886
0.325 38.987603014605
0.35 39.092413696265
0.375 39.1970373212685
0.4 39.3014738896155
0.425 39.405723401306
0.45 39.50978585634
0.475 39.6136612547175
0.5 39.7173495964385
0.525 39.8208508815031
0.55 39.9241651099112
0.575 40.0272922816627
0.6 40.1302323967578
0.625 40.2329854551964
0.65 40.3355514569785
0.675 40.4379304021041
0.7 40.5401222905732
0.725 40.6421271223858
0.75 40.743944897542
0.775 40.8455756160416
0.8 40.9470192778848
0.825 41.0482758830714
0.85 41.1493454316016
0.875 41.2502279234753
0.9 41.3509233586925
0.925 41.4514317372532
0.95 41.5517530591574
0.975 41.6518873244051
1 41.7518345329964
1.025 41.8515946849311
1.05 41.9511677802094
1.075 42.0505538188312
1.1 42.1497528007964
1.125 42.2487647261052
1.15 42.3475895947575
1.175 42.4462274067534
1.2 42.5446781620927
1.225 42.6429418607755
1.25 42.7410185028019
1.275 42.8389080881717
1.3 42.9366106168851
1.325 43.0341260889419
1.35 43.1314545043423
1.375 43.2285958630862
1.4 43.3255501651736
1.425 43.4223174106045
1.45 43.518897599379
1.475 43.6152907314969
1.5 43.7114968069583
1.525 43.8075158257633
1.55 43.9033477879118
1.575 43.9989926934037
1.6 44.0944505422392
1.625 44.1897213344182
1.65 44.2848050699407
1.675 44.3797017488067
1.7 44.4744113710163
1.725 44.5689339365693
1.75 44.6632694454659
1.775 44.7574178977059
1.8 44.8513792932895
1.825 44.9451536322166
1.85 45.0387409144872
1.875 45.1321411401013
1.9 45.2253543090589
1.925 45.31838042136
1.95 45.4112194770046
1.975 45.5038714759928
2 45.5963364183244
2.025 45.6886143039996
2.05 45.7807051330182
2.075 45.8726089053804
2.1 45.9643256210861
2.125 46.0558552801353
2.15 46.147197882528
2.175 46.2383534282642
2.2 46.329321917344
2.225 46.4201033497672
2.25 46.510697725534
2.275 46.6011050446442
2.3 46.691325307098
2.325 46.7813585128953
2.35 46.8712046620361
2.375 46.9608637545204
2.4 47.0503357903482
2.425 47.1396207695195
2.45 47.2287186920344
2.475 47.3176295578927
2.5 47.4063533670946
2.525 47.4948901196399
2.55 47.5832398155288
2.575 47.6714024547612
2.6 47.7593780373371
2.625 47.8471665632565
2.65 47.9347680325194
2.675 48.0221824451259
2.7 48.1094098010758
2.725 48.1964501003692
2.75 48.2833033430062
2.775 48.3699695289867
2.8 48.4564486583106
2.825 48.5427407309781
2.85 48.6288457469891
2.875 48.7147637063436
2.9 48.8004946090417
2.925 48.8860384550832
2.95 48.9713952444682
2.975 49.0565649771968
3 49.1415476532689
3.025 49.2263432726844
3.05 49.3109518354435
3.075 49.3953733415461
3.1 49.4796077909922
3.125 49.5636551837818
3.15 49.6475155199149
3.175 49.7311887993916
3.2 49.8146750222117
3.225 49.8979741883754
3.25 49.9810862978825
3.275 50.0640113507332
3.3 50.1467493469274
3.325 50.2293002864651
3.35 50.3116641693463
3.375 50.393840995571
3.4 50.4758307651392
3.425 50.557633478051
3.45 50.6392491343062
3.475 50.720677733905
3.5 50.8019192768472
3.525 50.882973763133
3.55 50.9638411927623
3.575 51.0445215657351
3.6 51.1250148820514
3.625 51.2053211417112
3.65 51.2854403447146
3.675 51.3653724910614
3.7 51.4451175807517
3.725 51.5246756137856
3.75 51.604046590163
3.775 51.6832305098839
3.8 51.7622273729483
3.825 51.8410371793561
3.85 51.9196599291076
3.875 51.9980956222025
3.9 52.0763442586409
3.925 52.1544058384228
3.95 52.2322803615483
3.975 52.3099678280173
4 52.3874682378297
4.025 52.4647815909857
4.05 52.5419078874852
4.075 52.6188471273282
4.1 52.6955993105147
4.125 52.7721644370448
4.15 52.8485425069183
4.175 52.9247335201353
4.2 53.0007374766959
4.225 53.0765543766
4.25 53.1521842198475
4.275 53.2276270064386
4.3 53.3028827363732
4.325 53.3779514096513
4.35 53.4528330262729
4.375 53.5275275862381
4.4 53.6020350895467
4.425 53.6763555361988
4.45 53.7504889261945
4.475 53.8244352595337
4.5 53.8982111571136
4.525 53.9719163443169
4.55 54.0455674420409
4.575 54.1191644502856
4.6 54.1927073690509
4.625 54.2661961983369
4.65 54.3396309381435
4.675 54.4130115884708
4.7 54.4863381493188
4.725 54.5596106206874
4.75 54.6328290025767
4.775 54.7059932949866
4.8 54.7791034979172
4.825 54.8521596113684
4.85 54.9251616353404
4.875 54.9981095698329
4.9 55.0710034148462
4.925 55.1438431703801
4.95 55.2166288364346
4.975 55.2893604130098
5 55.3620379001057
5.025 55.4346612977223
5.05 55.5072306058594
5.075 55.5797458245173
5.1 55.6522069536958
5.125 55.724613993395
5.15 55.7969669436148
5.175 55.8692658043553
5.2 55.9415105756165
5.225 56.0137012573983
5.25 56.0858378497007
5.275 56.1579203525239
5.3 56.2299487658677
5.325 56.3019230897321
5.35 56.3738433241172
5.375 56.445709469023
5.4 56.5175215244494
5.425 56.5892794903965
5.45 56.6609833668643
5.475 56.7326331538527
5.5 56.8042288513618
5.525 56.8757704593915
5.55 56.9472579779419
5.575 57.0186914070129
5.6 57.0900707466046
5.625 57.161395996717
5.65 57.23266715735
5.675 57.3038842285037
5.7 57.3750472101781
5.725 57.4461561023731
5.75 57.5172109050887
5.775 57.5882116183251
5.8 57.6591582420821
5.825 57.7300507763597
5.85 57.800889221158
5.875 57.871673576477
5.9 57.9424038423166
5.925 58.0130800186769
5.95 58.0837021055578
5.975 58.1542701029594
6 58.2247840108817
6.025 58.2952438293246
6.05 58.3656495582882
6.075 58.4360011977725
6.1 58.5062987477774
6.125 58.5765422083029
6.15 58.6467315793492
6.175 58.716866860916
6.2 58.7869480530036
6.225 58.8569751556118
6.25 58.9269481687407
6.275 58.9968670923902
6.3 59.0667319265604
6.325 59.1365426712512
6.35 59.2062993264627
6.375 59.2760018921949
6.4 59.3456503684477
6.425 59.4152447552212
6.45 59.4847850525153
6.475 59.5542712603301
6.5 59.6237033786656
6.525 59.6930814075217
6.55 59.7624053468985
6.575 59.8316751967959
6.6 59.900890957214
6.625 59.9700526281528
6.65 60.0391602096122
6.675 60.1082137015923
6.7 60.177213104093
6.725 60.2461584171144
6.75 60.3150496406565
6.775 60.3838867747192
6.8 60.4526698193026
6.825 60.5213987744066
6.85 60.5900736400313
6.875 60.6586944161767
6.9 60.7272611028427
6.925 60.7957737000294
6.95 60.8642322077367
6.975 60.9326366259647
7 61.0009869547134
7.025 61.0692831939827
7.05 61.1375253437727
7.075 61.2057134040833
7.1 61.2738473749146
7.125 61.3419272562666
7.15 61.4099530481392
7.175 61.4779247505325
7.2 61.5458423634464
7.225 61.613705886881
7.25 61.6815153208363
7.275 61.7492706653122
7.3 61.8169719203087
7.325 61.884619085826
7.35 61.9522121618639
7.375 62.0197511484224
7.4 62.0872360455017
7.425 62.1546668531015
7.45 62.2220435712221
7.475 62.2893661998633
7.5 62.3566347390251
7.525 62.4238491887076
7.55 62.4910095489108
7.575 62.5581158196346
7.6 62.6251680008791
7.625 62.6921660926443
7.65 62.7591100949301
7.675 62.8260000077366
7.7 62.8928358310637
7.725 62.9596175649115
7.75 63.02634520928
7.775 63.0930187641691
7.8 63.1596382295788
7.825 63.2262036055093
7.85 63.2927148919603
7.875 63.3591720889321
7.9 63.4255751964245
7.925 63.4919242144376
7.95 63.5582191429713
7.975 63.6244599820257
8 63.6906467316008
8.025 63.7567793916965
8.05 63.8228579623129
8.075 63.8888824434499
8.1 63.9548528351076
8.125 64.0207691372859
8.15 64.0866313499849
8.175 64.1524394732046
8.2 64.2181935069449
8.225 64.2838934512059
8.25 64.3495393059876
8.275 64.4151310712899
8.3 64.4806687471128
8.325 64.5461523334565
8.35 64.6115818303207
8.375 64.6769572377057
8.4 64.7422785556113
8.425 64.8075457840376
8.45 64.8727589229845
8.475 64.9379179724521
8.5 65.0030229324403
8.525 65.0680738029492
8.55 65.1330705839788
8.575 65.198013275529
8.6 65.2629018775999
8.625 65.3277363901914
8.65 65.3925168133037
8.675 65.4572431469365
8.7 65.52191539109
8.725 65.5865335457642
8.75 65.6510976109591
8.775 65.7156075866746
8.8 65.7800634729108
8.825 65.8444652696676
8.85 65.9088129769451
8.875 65.9731065947432
8.9 66.037346123062
8.925 66.1015315619015
8.95 66.1656629112616
8.975 66.2297401711424
9 65.9163525027803
9.025 65.816737139492
9.05 65.7180682141708
9.075 65.6203457268167
9.1 65.5235696774297
9.125 65.4277400660097
9.15 65.3328568925569
9.175 65.2389201570712
9.2 65.1459298595526
9.225 65.053886000001
9.25 64.9627885784166
9.275 64.8726375947992
9.3 64.783433049149
9.325 64.6951749414658
9.35 64.6078632717497
9.375 64.5214980400008
9.4 64.4360792462189
9.425 64.3516068904041
9.45 64.2680809725564
9.475 64.1855014926758
9.5 64.1038684507623
9.525 64.0231818468159
9.55 63.9434416808365
9.575 63.8646479528243
9.6 63.7868006627792
9.625 63.7098998107011
9.65 63.6339453965902
9.675 63.5589374204463
9.7 63.4848758822695
9.725 63.4117607820599
9.75 63.3395921198173
9.775 63.2683698955418
9.8 63.1980941092334
9.825 63.1287647608922
9.85 63.0603818505179
9.875 62.9929453781108
9.9 62.9264553436708
9.925 62.8609117471979
9.95 62.7963145886921
9.975 62.7326638681533
10 62.6700920870255
10.025 62.6093942539702
10.05 62.5507028704314
10.075 62.4940179364089
10.1 62.4393394519028
10.125 62.3866674169131
10.15 62.3360018314397
10.175 62.2873426954828
10.2 62.2406900090422
10.225 62.196043772118
10.25 62.1534039847103
10.275 62.1127706468189
10.3 62.0741437584438
10.325 62.0375233195852
10.35 62.002909330243
10.375 61.9703017904171
10.4 61.9397007001076
10.425 61.9111060593145
10.45 61.8845178680378
10.475 61.8599361262775
10.5 61.8373608340335
10.525 61.816791991306
10.55 61.7982295980948
10.575 61.7816736544
10.6 61.7671241602216
10.625 61.7545811155596
10.65 61.744044520414
10.675 61.7355143747847
10.7 61.7289906786719
10.725 61.7244734320754
10.75 61.7219626349953
10.775 61.7214582874316
10.8 61.7229603893843
10.825 61.7264689408534
10.85 61.7319839418388
10.875 61.7395053923406
10.9 61.7490332923588
10.925 61.7605676418935
10.95 61.7741084409445
10.975 61.7896556895118
11 61.392702927607
11.025 61.392702927607
11.05 61.392702927607
11.075 61.392702927607
11.1 61.392702927607
11.125 61.392702927607
11.15 61.392702927607
11.175 61.392702927607
11.2 61.392702927607
11.225 61.392702927607
11.25 61.392702927607
11.275 61.392702927607
11.3 61.392702927607
11.325 61.392702927607
11.35 61.392702927607
11.375 61.392702927607
11.4 61.392702927607
11.425 61.392702927607
11.45 61.392702927607
11.475 61.392702927607
11.5 61.392702927607
11.525 61.392702927607
11.55 61.392702927607
11.575 61.392702927607
11.6 61.392702927607
11.625 61.392702927607
11.65 61.392702927607
11.675 61.392702927607
11.7 61.392702927607
11.725 61.392702927607
11.75 61.392702927607
11.775 61.392702927607
11.8 61.392702927607
11.825 61.392702927607
11.85 61.392702927607
11.875 61.392702927607
11.9 61.392702927607
11.925 61.392702927607
11.95 61.392702927607
11.975 61.392702927607
12 61.392702927607
12.025 61.392702927607
12.05 61.392702927607
12.075 61.392702927607
12.1 61.392702927607
12.125 61.392702927607
12.15 61.392702927607
12.175 61.392702927607
12.2 61.392702927607
12.225 61.392702927607
12.25 61.392702927607
12.275 61.392702927607
12.3 61.392702927607
12.325 61.392702927607
12.35 61.392702927607
12.375 61.392702927607
12.4 61.392702927607
12.425 61.392702927607
12.45 61.392702927607
12.475 61.392702927607
12.5 61.392702927607
12.525 61.392702927607
12.55 61.392702927607
12.575 61.392702927607
12.6 61.392702927607
12.625 61.392702927607
12.65 61.392702927607
12.675 61.392702927607
12.7 61.392702927607
12.725 61.392702927607
12.75 61.392702927607
12.775 61.392702927607
12.8 61.392702927607
12.825 61.392702927607
12.85 61.392702927607
12.875 61.392702927607
12.9 61.392702927607
12.925 61.392702927607
12.95 61.392702927607
12.975 61.392702927607
13 61.392702927607
13.025 61.392702927607
13.05 61.392702927607
13.075 61.392702927607
13.1 61.392702927607
13.125 61.392702927607
13.15 61.392702927607
13.175 61.392702927607
13.2 61.392702927607
13.225 61.392702927607
13.25 61.392702927607
13.275 61.392702927607
13.3 61.392702927607
13.325 61.392702927607
13.35 61.392702927607
13.375 61.392702927607
13.4 61.392702927607
13.425 61.392702927607
13.45 61.392702927607
13.475 61.392702927607
13.5 61.392702927607
13.525 61.392702927607
13.55 61.392702927607
13.575 61.392702927607
13.6 61.392702927607
13.625 61.392702927607
13.65 61.392702927607
13.675 61.392702927607
13.7 61.392702927607
13.725 61.392702927607
13.75 61.392702927607
13.775 61.392702927607
13.8 61.392702927607
13.825 61.392702927607
13.85 61.392702927607
13.875 61.392702927607
13.9 61.392702927607
13.925 61.392702927607
13.95 61.392702927607
13.975 61.392702927607
14 61.392702927607
14.025 61.392702927607
14.05 61.392702927607
14.075 61.392702927607
14.1 61.392702927607
14.125 61.392702927607
14.15 61.392702927607
14.175 61.392702927607
14.2 61.392702927607
14.225 61.392702927607
14.25 61.392702927607
14.275 61.392702927607
14.3 61.392702927607
14.325 61.392702927607
14.35 61.392702927607
14.375 61.392702927607
14.4 61.392702927607
14.425 61.392702927607
14.45 61.392702927607
14.475 61.392702927607
14.5 61.392702927607
14.525 61.392702927607
14.55 61.392702927607
14.575 61.392702927607
14.6 61.392702927607
14.625 61.392702927607
14.65 61.392702927607
14.675 61.392702927607
14.7 61.392702927607
14.725 61.392702927607
14.75 61.392702927607
14.775 61.392702927607
14.8 61.392702927607
14.825 61.392702927607
14.85 61.392702927607
14.875 61.392702927607
14.9 61.392702927607
14.925 61.392702927607
14.95 61.392702927607
14.975 61.392702927607
15 61.392702927607
15.025 61.392702927607
15.05 61.392702927607
15.075 61.392702927607
15.1 61.392702927607
15.125 61.392702927607
15.15 61.392702927607
15.175 61.392702927607
15.2 61.392702927607
15.225 61.392702927607
15.25 61.392702927607
15.275 61.392702927607
15.3 61.392702927607
15.325 61.392702927607
15.35 61.392702927607
15.375 61.392702927607
15.4 61.392702927607
15.425 61.392702927607
15.45 61.392702927607
15.475 61.392702927607
15.5 61.392702927607
15.525 61.392702927607
15.55 61.392702927607
15.575 61.392702927607
15.6 61.392702927607
15.625 61.392702927607
15.65 61.392702927607
15.675 61.392702927607
15.7 61.392702927607
15.725 61.392702927607
15.75 61.392702927607
15.775 61.392702927607
15.8 61.392702927607
15.825 61.392702927607
15.85 61.392702927607
15.875 61.392702927607
15.9 61.392702927607
15.925 61.392702927607
15.95 61.392702927607
15.975 61.392702927607
16 61.392702927607
16.025 61.392702927607
16.05 61.392702927607
16.075 61.392702927607
16.1 61.392702927607
16.125 61.392702927607
16.15 61.392702927607
16.175 61.392702927607
16.2 61.392702927607
16.225 61.392702927607
16.25 61.392702927607
16.275 61.392702927607
16.3 61.392702927607
16.325 61.392702927607
16.35 61.392702927607
16.375 61.392702927607
16.4 61.392702927607
16.425 61.392702927607
16.45 61.392702927607
16.475 61.392702927607
16.5 61.392702927607
16.525 61.392702927607
16.55 61.392702927607
16.575 61.392702927607
16.6 61.392702927607
16.625 61.392702927607
16.65 61.392702927607
16.675 61.392702927607
16.7 61.392702927607
16.725 61.392702927607
16.75 61.392702927607
16.775 61.392702927607
16.8 61.392702927607
16.825 61.392702927607
16.85 61.392702927607
16.875 61.392702927607
16.9 61.392702927607
16.925 61.392702927607
16.95 61.392702927607
16.975 61.392702927607
17 61.392702927607
17.025 61.392702927607
17.05 61.392702927607
17.075 61.392702927607
17.1 61.392702927607
17.125 61.392702927607
17.15 61.392702927607
17.175 61.392702927607
17.2 61.392702927607
17.225 61.392702927607
17.25 61.392702927607
17.275 61.392702927607
17.3 61.392702927607
17.325 61.392702927607
17.35 61.392702927607
17.375 61.392702927607
17.4 61.392702927607
17.425 61.392702927607
17.45 61.392702927607
17.475 61.392702927607
17.5 61.392702927607
17.525 61.392702927607
17.55 61.392702927607
17.575 61.392702927607
17.6 61.392702927607
17.625 61.392702927607
17.65 61.392702927607
17.675 61.392702927607
17.7 61.392702927607
17.725 61.392702927607
17.75 61.392702927607
17.775 61.392702927607
17.8 61.392702927607
17.825 61.392702927607
17.85 61.392702927607
17.875 61.392702927607
17.9 61.392702927607
17.925 61.392702927607
17.95 61.392702927607
17.975 61.392702927607
18 61.7326653372238
18.025 61.8311539114564
18.05 61.9289491671942
18.075 62.026051104437
18.1 62.1224597231851
18.125 62.2181750234382
18.15 62.3131970051965
18.175 62.4075256684599
18.2 62.5011610132285
18.225 62.5941030395022
18.25 62.686351747281
18.275 62.7779071365649
18.3 62.868769207354
18.325 62.9589379596482
18.35 63.0484133934476
18.375 63.1371955087521
18.4 63.2252843055617
18.425 63.3126797838764
18.45 63.3993819436963
18.475 63.4853907850213
18.5 63.5707063078515
18.525 63.6553285121868
18.55 63.7392573980272
18.575 63.8224929653727
18.6 63.9050352142234
18.625 63.9868841445793
18.65 64.0680397564402
18.675 64.1485020498063
18.7 64.2282710246775
18.725 64.3073466810539
18.75 64.3857290189354
18.775 64.463418038322
18.8 64.5404137392137
18.825 64.6167161216106
18.85 64.6923251855127
18.875 64.7672409309198
18.9 64.8414633578321
18.925 64.9149924662495
18.95 64.9878282561721
18.975 65.0599707275998
19 65.1314198805326
19.025 65.2021757149706
19.05 65.2722382309137
19.075 65.3416074283619
19.1 65.4102833073152
19.125 65.4782658677737
19.15 65.5455551097374
19.175 65.6121510332061
19.2 65.67805363818
19.225 65.743262924659
19.25 65.8077788926432
19.275 65.8716015421325
19.3 65.9347308731269
19.325 65.9971668856265
19.35 66.0589095796312
19.375 66.119958955141
19.4 66.180315012156
19.425 66.239977750676
19.45 66.2989471707013
19.475 66.3572232722316
19.5 66.4148060552671
19.525 66.4716955198078
19.55 66.5278916658535
19.575 66.5833944934044
19.6 66.6382040024605
19.625 66.6923201930216
19.65 66.7457430650879
19.675 66.7984726186594
19.7 66.8505088537359
19.725 66.9018517703176
19.75 66.9525013684045
19.775 67.0024576479964
19.8 67.0517206090935
19.825 67.1002902516958
19.85 67.1481665758031
19.875 67.1953495814156
19.9 67.2418392685333
19.925 67.287635637156
19.95 67.3327386872839
19.975 67.3771484189169
20 67.4208648320551
20.025 67.4638879266984
20.05 67.5062177028469
20.075 67.5478541605004
20.1 67.5887972996591
20.125 67.629047120323
20.15 67.6686036224919
20.175 67.707466806166
20.2 67.7456366713453
20.225 67.7831132180296
20.25 67.8198964462192
20.275 67.8559863559138
20.3 67.8913829471136
20.325 67.9260862198185
20.35 67.9600961740285
20.375 67.9934128097437
20.4 68.026036126964
20.425 68.0579661256894
20.45 68.08920280592
20.475 68.1197461676557
20.5 68.1496431270512
20.525 68.1791751810337
20.55 68.2083892457577
20.575 68.2372853212234
20.6 68.2658634074307
20.625 68.2941235043795
20.65 68.32206561207
20.675 68.3496897305021
20.7 68.3769958596758
20.725 68.4039839995912
20.75 68.4306541502481
20.775 68.4570063116466
20.8 68.4830404837868
20.825 68.5087566666685
20.85 68.5341548602919
20.875 68.5592350646568
20.9 68.5839972797634
20.925 68.6084415056116
20.95 68.6325677422014
20.975 68.6563759895328
21 68.6798662476058
21.025 68.7030385164204
21.05 68.7258927959766
21.075 68.7484290862745
21.1 68.7706473873139
21.125 68.792547699095
21.15 68.8141300216176
21.175 68.8353943548819
21.2 68.8563406988878
21.225 68.8769690536352
21.25 68.8972794191243
21.275 68.917271795355
21.3 68.9369461823274
21.325 68.9563025800413
21.35 68.9753409884968
21.375 68.9940614076939
21.4 69.0124638376327
21.425 69.030548278313
21.45 69.048314729735
21.475 69.0657631918986
21.5 69.0828936648037
21.525 69.0997061484505
21.55 69.1162006428389
21.575 69.1323771479689
21.6 69.1482356638405
21.625 69.1637761904538
21.65 69.1789987278086
21.675 69.193903275905
21.7 69.2084898347431
21.725 69.2227584043227
21.75 69.236708984644
21.775 69.2503415757069
21.8 69.2636561775114
21.825 69.2766527900574
21.85 69.2893314133451
21.875 69.3016920473744
21.9 69.3137346921453
21.925 69.3254593476579
21.95 69.336866013912
21.975 69.3479546909077
22 69.3587253786451
22.025 69.369178077124
22.05 69.3793127863446
22.075 69.3891295063068
22.1 69.3986282370105
22.125 69.4078089784559
22.15 69.4166717306429
22.175 69.4252164935716
22.2 69.4334432672418
22.225 69.4413520516536
22.25 69.448942846807
22.275 69.4562156527021
22.3 69.4631704693387
22.325 69.469807296717
22.35 69.4761261348368
22.375 69.4821269836983
22.4 69.4878098433014
22.425 69.4931747136461
22.45 69.4982215947324
22.475 69.5029504865603
22.5 69.5073613891298
22.525 69.5114543024409
22.55 69.5152292264937
22.575 69.518686161288
22.6 69.521825106824
22.625 69.5246460631015
22.65 69.5271490301207
22.675 69.5293340078815
22.7 69.5312009963838
22.725 69.5327499956278
22.75 69.5339810056134
22.775 69.5348940263407
22.8 69.5354890578095
22.825 69.5357661000199
22.85 69.5357251529719
22.875 69.5353662166656
22.9 69.5346892911008
22.925 69.5336943762777
22.95 69.5323814721961
22.975 69.5307505788562
23 69.935051266673
23.025 69.935051266673
23.05 69.935051266673
23.075 69.935051266673
23.1 69.935051266673
23.125 69.935051266673
23.15 69.935051266673
23.175 69.935051266673
23.2 69.935051266673
23.225 69.935051266673
23.25 69.935051266673
23.275 69.935051266673
23.3 69.935051266673
23.325 69.935051266673
23.35 69.935051266673
23.375 69.935051266673
23.4 69.935051266673
23.425 69.935051266673
23.45 69.935051266673
23.475 69.935051266673
23.5 69.935051266673
23.525 69.935051266673
23.55 69.935051266673
23.575 69.935051266673
23.6 69.935051266673
23.625 69.935051266673
23.65 69.935051266673
23.675 69.935051266673
23.7 69.935051266673
23.725 69.935051266673
23.75 69.935051266673
23.775 69.935051266673
23.8 69.935051266673
23.825 69.935051266673
23.85 69.935051266673
23.875 69.935051266673
23.9 69.935051266673
23.925 69.935051266673
23.95 69.935051266673
23.975 69.935051266673
24 69.935051266673
24.025 69.935051266673
24.05 69.935051266673
24.075 69.935051266673
24.1 69.935051266673
24.125 69.935051266673
24.15 69.935051266673
24.175 69.935051266673
24.2 69.935051266673
24.225 69.935051266673
24.25 69.935051266673
24.275 69.935051266673
24.3 69.935051266673
24.325 69.935051266673
24.35 69.935051266673
24.375 69.935051266673
24.4 69.935051266673
24.425 69.935051266673
24.45 69.935051266673
24.475 69.935051266673
24.5 69.935051266673
24.525 69.935051266673
24.55 69.935051266673
24.575 69.935051266673
24.6 69.935051266673
24.625 69.935051266673
24.65 69.935051266673
24.675 69.935051266673
24.7 69.935051266673
24.725 69.935051266673
24.75 69.935051266673
24.775 69.935051266673
24.8 69.935051266673
24.825 69.935051266673
24.85 69.935051266673
24.875 69.935051266673
24.9 69.935051266673
24.925 69.935051266673
24.95 69.935051266673
24.975 69.935051266673
25 69.935051266673
25.025 69.935051266673
25.05 69.935051266673
25.075 69.935051266673
25.1 69.935051266673
25.125 69.935051266673
25.15 69.935051266673
25.175 69.935051266673
25.2 69.935051266673
25.225 69.935051266673
25.25 69.935051266673
25.275 69.935051266673
25.3 69.935051266673
25.325 69.935051266673
25.35 69.935051266673
25.375 69.935051266673
25.4 69.935051266673
25.425 69.935051266673
25.45 69.935051266673
25.475 69.935051266673
25.5 69.935051266673
25.525 69.935051266673
25.55 69.935051266673
25.575 69.935051266673
25.6 69.935051266673
25.625 69.935051266673
25.65 69.935051266673
25.675 69.935051266673
25.7 69.935051266673
25.725 69.935051266673
25.75 69.935051266673
25.775 69.935051266673
25.8 69.935051266673
25.825 69.935051266673
25.85 69.935051266673
25.875 69.935051266673
25.9 69.935051266673
25.925 69.935051266673
25.95 69.935051266673
25.975 69.935051266673
26 69.935051266673
26.025 69.935051266673
26.05 69.935051266673
26.075 69.935051266673
26.1 69.935051266673
26.125 69.935051266673
26.15 69.935051266673
26.175 69.935051266673
26.2 69.935051266673
26.225 69.935051266673
26.25 69.935051266673
26.275 69.935051266673
26.3 69.935051266673
26.325 69.935051266673
26.35 69.935051266673
26.375 69.935051266673
26.4 69.935051266673
26.425 69.935051266673
26.45 69.935051266673
26.475 69.935051266673
26.5 69.935051266673
26.525 69.935051266673
26.55 69.935051266673
26.575 69.935051266673
26.6 69.935051266673
26.625 69.935051266673
26.65 69.935051266673
26.675 69.935051266673
26.7 69.935051266673
26.725 69.935051266673
26.75 69.935051266673
26.775 69.935051266673
26.8 69.935051266673
26.825 69.935051266673
26.85 69.935051266673
26.875 69.935051266673
26.9 69.935051266673
26.925 69.935051266673
26.95 69.935051266673
26.975 69.935051266673
27 69.935051266673
27.025 69.935051266673
27.05 69.935051266673
27.075 69.935051266673
27.1 69.935051266673
27.125 69.935051266673
27.15 69.935051266673
27.175 69.935051266673
27.2 69.935051266673
27.225 69.935051266673
27.25 69.935051266673
27.275 69.935051266673
27.3 69.935051266673
27.325 69.935051266673
27.35 69.935051266673
27.375 69.935051266673
27.4 69.935051266673
27.425 69.935051266673
27.45 69.935051266673
27.475 69.935051266673
27.5 69.935051266673
27.525 69.935051266673
27.55 69.935051266673
27.575 69.935051266673
27.6 69.935051266673
27.625 69.935051266673
27.65 69.935051266673
27.675 69.935051266673
27.7 69.935051266673
27.725 69.935051266673
27.75 69.935051266673
27.775 69.935051266673
27.8 69.935051266673
27.825 69.935051266673
27.85 69.935051266673
27.875 69.935051266673
27.9 69.935051266673
27.925 69.935051266673
27.95 69.935051266673
27.975 69.935051266673
28 69.935051266673
28.025 69.935051266673
28.05 69.935051266673
28.075 69.935051266673
28.1 69.935051266673
28.125 69.935051266673
28.15 69.935051266673
28.175 69.935051266673
28.2 69.935051266673
28.225 69.935051266673
28.25 69.935051266673
28.275 69.935051266673
28.3 69.935051266673
28.325 69.935051266673
28.35 69.935051266673
28.375 69.935051266673
28.4 69.935051266673
28.425 69.935051266673
28.45 69.935051266673
28.475 69.935051266673
28.5 69.935051266673
28.525 69.935051266673
28.55 69.935051266673
28.575 69.935051266673
28.6 69.935051266673
28.625 69.935051266673
28.65 69.935051266673
28.675 69.935051266673
28.7 69.935051266673
28.725 69.935051266673
28.75 69.935051266673
28.775 69.935051266673
28.8 69.935051266673
28.825 69.935051266673
28.85 69.935051266673
28.875 69.935051266673
28.9 69.935051266673
28.925 69.935051266673
28.95 69.935051266673
28.975 69.935051266673
29 69.935051266673
29.025 69.935051266673
29.05 69.935051266673
29.075 69.935051266673
29.1 69.935051266673
29.125 69.935051266673
29.15 69.935051266673
29.175 69.935051266673
29.2 69.935051266673
29.225 69.935051266673
29.25 69.935051266673
29.275 69.935051266673
29.3 69.935051266673
29.325 69.935051266673
29.35 69.935051266673
29.375 69.935051266673
29.4 69.935051266673
29.425 69.935051266673
29.45 69.935051266673
29.475 69.935051266673
29.5 69.935051266673
29.525 69.935051266673
29.55 69.935051266673
29.575 69.935051266673
29.6 69.935051266673
29.625 69.935051266673
29.65 69.935051266673
29.675 69.935051266673
29.7 69.935051266673
29.725 69.935051266673
29.75 69.935051266673
29.775 69.935051266673
29.8 69.935051266673
29.825 69.935051266673
29.85 69.935051266673
29.875 69.935051266673
29.9 69.935051266673
29.925 69.935051266673
29.95 69.935051266673
29.975 69.935051266673
30 69.935051266673
30.025 69.935051266673
30.05 69.935051266673
30.075 69.935051266673
30.1 69.935051266673
30.125 69.935051266673
30.15 69.935051266673
30.175 69.935051266673
30.2 69.935051266673
30.225 69.935051266673
30.25 69.935051266673
30.275 69.935051266673
30.3 69.935051266673
30.325 69.935051266673
30.35 69.935051266673
30.375 69.935051266673
30.4 69.935051266673
30.425 69.935051266673
30.45 69.935051266673
30.475 69.935051266673
30.5 69.935051266673
30.525 69.935051266673
30.55 69.935051266673
30.575 69.935051266673
30.6 69.935051266673
30.625 69.935051266673
30.65 69.935051266673
30.675 69.935051266673
30.7 69.935051266673
30.725 69.935051266673
30.75 69.935051266673
30.775 69.935051266673
30.8 69.935051266673
30.825 69.935051266673
30.85 69.935051266673
30.875 69.935051266673
30.9 69.935051266673
30.925 69.935051266673
30.95 69.935051266673
30.975 69.935051266673
31 69.935051266673
31.025 69.935051266673
31.05 69.935051266673
31.075 69.935051266673
31.1 69.935051266673
31.125 69.935051266673
31.15 69.935051266673
31.175 69.935051266673
31.2 69.935051266673
31.225 69.935051266673
31.25 69.935051266673
31.275 69.935051266673
31.3 69.935051266673
31.325 69.935051266673
31.35 69.935051266673
31.375 69.935051266673
31.4 69.935051266673
31.425 69.935051266673
31.45 69.935051266673
31.475 69.935051266673
31.5 69.935051266673
31.525 69.935051266673
31.55 69.935051266673
31.575 69.935051266673
31.6 69.935051266673
31.625 69.935051266673
31.65 69.935051266673
31.675 69.935051266673
31.7 69.935051266673
31.725 69.935051266673
31.75 69.935051266673
31.775 69.935051266673
31.8 69.935051266673
31.825 69.935051266673
31.85 69.935051266673
31.875 69.935051266673
31.9 69.935051266673
31.925 69.935051266673
31.95 69.935051266673
31.975 69.935051266673
32 69.935051266673
32.025 69.935051266673
32.05 69.935051266673
32.075 69.935051266673
32.1 69.935051266673
32.125 69.935051266673
32.15 69.935051266673
32.175 69.935051266673
32.2 69.935051266673
32.225 69.935051266673
32.25 69.935051266673
32.275 69.935051266673
32.3 69.935051266673
32.325 69.935051266673
32.35 69.935051266673
32.375 69.935051266673
32.4 69.935051266673
32.425 69.935051266673
32.45 69.935051266673
32.475 69.935051266673
32.5 69.935051266673
32.525 69.935051266673
32.55 69.935051266673
32.575 69.935051266673
32.6 69.935051266673
32.625 69.935051266673
32.65 69.935051266673
32.675 69.935051266673
32.7 69.935051266673
32.725 69.935051266673
32.75 69.935051266673
32.775 69.935051266673
32.8 69.935051266673
32.825 69.935051266673
32.85 69.935051266673
32.875 69.935051266673
32.9 69.935051266673
32.925 69.935051266673
32.95 69.935051266673
32.975 69.935051266673
33 69.935051266673
33.025 69.935051266673
33.05 69.935051266673
33.075 69.935051266673
33.1 69.935051266673
33.125 69.935051266673
33.15 69.935051266673
33.175 69.935051266673
33.2 69.935051266673
33.225 69.935051266673
33.25 69.935051266673
33.275 69.935051266673
33.3 69.935051266673
33.325 69.935051266673
33.35 69.935051266673
33.375 69.935051266673
33.4 69.935051266673
33.425 69.935051266673
33.45 69.935051266673
33.475 69.935051266673
33.5 69.935051266673
33.525 69.935051266673
33.55 69.935051266673
33.575 69.935051266673
33.6 69.935051266673
33.625 69.935051266673
33.65 69.935051266673
33.675 69.935051266673
33.7 69.935051266673
33.725 69.935051266673
33.75 69.935051266673
33.775 69.935051266673
33.8 69.935051266673
33.825 69.935051266673
33.85 69.935051266673
33.875 69.935051266673
33.9 69.935051266673
33.925 69.935051266673
33.95 69.935051266673
33.975 69.935051266673
34 69.935051266673
34.025 69.935051266673
34.05 69.935051266673
34.075 69.935051266673
34.1 69.935051266673
34.125 69.935051266673
34.15 69.935051266673
34.175 69.935051266673
34.2 69.935051266673
34.225 69.935051266673
34.25 69.935051266673
34.275 69.935051266673
34.3 69.935051266673
34.325 69.935051266673
34.35 69.935051266673
34.375 69.935051266673
34.4 69.935051266673
34.425 69.935051266673
34.45 69.935051266673
34.475 69.935051266673
34.5 69.935051266673
34.525 69.935051266673
34.55 69.935051266673
34.575 69.935051266673
34.6 69.935051266673
34.625 69.935051266673
34.65 69.935051266673
34.675 69.935051266673
34.7 69.935051266673
34.725 69.935051266673
34.75 69.935051266673
34.775 69.935051266673
34.8 69.935051266673
34.825 69.935051266673
34.85 69.935051266673
34.875 69.935051266673
34.9 69.935051266673
34.925 69.935051266673
34.95 69.935051266673
34.975 69.935051266673
35 69.935051266673
35.025 69.935051266673
35.05 69.935051266673
35.075 69.935051266673
35.1 69.935051266673
35.125 69.935051266673
35.15 69.935051266673
35.175 69.935051266673
35.2 69.935051266673
35.225 69.935051266673
35.25 69.935051266673
35.275 69.935051266673
35.3 69.935051266673
35.325 69.935051266673
35.35 69.935051266673
35.375 69.935051266673
35.4 69.935051266673
35.425 69.935051266673
35.45 69.935051266673
35.475 69.935051266673
35.5 69.935051266673
35.525 69.935051266673
35.55 69.935051266673
35.575 69.935051266673
35.6 69.935051266673
35.625 69.935051266673
35.65 69.935051266673
35.675 69.935051266673
35.7 69.935051266673
35.725 69.935051266673
35.75 69.935051266673
35.775 69.935051266673
35.8 69.935051266673
35.825 69.935051266673
35.85 69.935051266673
35.875 69.935051266673
35.9 69.935051266673
35.925 69.935051266673
35.95 69.935051266673
35.975 69.935051266673
36 69.935051266673
36.025 69.935051266673
36.05 69.935051266673
36.075 69.935051266673
36.1 69.935051266673
36.125 69.935051266673
36.15 69.935051266673
36.175 69.935051266673
36.2 69.935051266673
36.225 69.935051266673
36.25 69.935051266673
36.275 69.935051266673
36.3 69.935051266673
36.325 69.935051266673
36.35 69.935051266673
36.375 69.935051266673
36.4 69.935051266673
36.425 69.935051266673
36.45 69.935051266673
36.475 69.935051266673
36.5 69.935051266673
36.525 69.935051266673
36.55 69.935051266673
36.575 69.935051266673
36.6 69.935051266673
36.625 69.935051266673
36.65 69.935051266673
36.675 69.935051266673
36.7 69.935051266673
36.725 69.935051266673
36.75 69.935051266673
36.775 69.935051266673
36.8 69.935051266673
36.825 69.935051266673
36.85 69.935051266673
36.875 69.935051266673
36.9 69.935051266673
36.925 69.935051266673
36.95 69.935051266673
36.975 69.935051266673
37 69.935051266673
37.025 69.935051266673
37.05 69.935051266673
37.075 69.935051266673
37.1 69.935051266673
37.125 69.935051266673
37.15 69.935051266673
37.175 69.935051266673
37.2 69.935051266673
37.225 69.935051266673
37.25 69.935051266673
37.275 69.935051266673
37.3 69.935051266673
37.325 69.935051266673
37.35 69.935051266673
37.375 69.935051266673
37.4 69.935051266673
37.425 69.935051266673
37.45 69.935051266673
37.475 69.935051266673
37.5 69.935051266673
37.525 69.935051266673
37.55 69.935051266673
37.575 69.935051266673
37.6 69.935051266673
37.625 69.935051266673
37.65 69.935051266673
37.675 69.935051266673
37.7 69.935051266673
37.725 69.935051266673
37.75 69.935051266673
37.775 69.935051266673
37.8 69.935051266673
37.825 69.935051266673
37.85 69.935051266673
37.875 69.935051266673
37.9 69.935051266673
37.925 69.935051266673
37.95 69.935051266673
37.975 69.935051266673
38 69.935051266673
38.025 69.935051266673
38.05 69.935051266673
38.075 69.935051266673
38.1 69.935051266673
38.125 69.935051266673
38.15 69.935051266673
38.175 69.935051266673
38.2 69.935051266673
38.225 69.935051266673
38.25 69.935051266673
38.275 69.935051266673
38.3 69.935051266673
38.325 69.935051266673
38.35 69.935051266673
38.375 69.935051266673
38.4 69.935051266673
38.425 69.935051266673
38.45 69.935051266673
38.475 69.935051266673
38.5 69.935051266673
38.525 69.935051266673
38.55 69.935051266673
38.575 69.935051266673
38.6 69.935051266673
38.625 69.935051266673
38.65 69.935051266673
38.675 69.935051266673
38.7 69.935051266673
38.725 69.935051266673
38.75 69.935051266673
38.775 69.935051266673
38.8 69.935051266673
38.825 69.935051266673
38.85 69.935051266673
38.875 69.935051266673
38.9 69.935051266673
38.925 69.935051266673
38.95 69.935051266673
38.975 69.935051266673
39 69.935051266673
39.025 69.935051266673
39.05 69.935051266673
39.075 69.935051266673
39.1 69.935051266673
39.125 69.935051266673
39.15 69.935051266673
39.175 69.935051266673
39.2 69.935051266673
39.225 69.935051266673
39.25 69.935051266673
39.275 69.935051266673
39.3 69.935051266673
39.325 69.935051266673
39.35 69.935051266673
39.375 69.935051266673
39.4 69.935051266673
39.425 69.935051266673
39.45 69.935051266673
39.475 69.935051266673
39.5 69.935051266673
39.525 69.935051266673
39.55 69.935051266673
39.575 69.935051266673
39.6 69.935051266673
39.625 69.935051266673
39.65 69.935051266673
39.675 69.935051266673
39.7 69.935051266673
39.725 69.935051266673
39.75 69.935051266673
39.775 69.935051266673
39.8 69.935051266673
39.825 69.935051266673
39.85 69.935051266673
39.875 69.935051266673
39.9 69.935051266673
39.925 69.935051266673
39.95 69.935051266673
39.975 69.935051266673
40 69.935051266673
40.025 69.935051266673
40.05 69.935051266673
40.075 69.935051266673
40.1 69.935051266673
40.125 69.935051266673
40.15 69.935051266673
40.175 69.935051266673
40.2 69.935051266673
40.225 69.935051266673
40.25 69.935051266673
40.275 69.935051266673
40.3 69.935051266673
40.325 69.935051266673
40.35 69.935051266673
40.375 69.935051266673
40.4 69.935051266673
40.425 69.935051266673
40.45 69.935051266673
40.475 69.935051266673
40.5 69.935051266673
40.525 69.935051266673
40.55 69.935051266673
40.575 69.935051266673
40.6 69.935051266673
40.625 69.935051266673
40.65 69.935051266673
40.675 69.935051266673
40.7 69.935051266673
40.725 69.935051266673
40.75 69.935051266673
40.775 69.935051266673
40.8 69.935051266673
40.825 69.935051266673
40.85 69.935051266673
40.875 69.935051266673
40.9 69.935051266673
40.925 69.935051266673
40.95 69.935051266673
40.975 69.935051266673
41 69.935051266673
41.025 69.935051266673
41.05 69.935051266673
41.075 69.935051266673
41.1 69.935051266673
41.125 69.935051266673
41.15 69.935051266673
41.175 69.935051266673
41.2 69.935051266673
41.225 69.935051266673
41.25 69.935051266673
41.275 69.935051266673
41.3 69.935051266673
41.325 69.935051266673
41.35 69.935051266673
41.375 69.935051266673
41.4 69.935051266673
41.425 69.935051266673
41.45 69.935051266673
41.475 69.935051266673
41.5 69.935051266673
41.525 69.935051266673
41.55 69.935051266673
41.575 69.935051266673
41.6 69.935051266673
41.625 69.935051266673
41.65 69.935051266673
41.675 69.935051266673
41.7 69.935051266673
41.725 69.935051266673
41.75 69.935051266673
41.775 69.935051266673
41.8 69.935051266673
41.825 69.935051266673
41.85 69.935051266673
41.875 69.935051266673
41.9 69.935051266673
41.925 69.935051266673
41.95 69.935051266673
41.975 69.935051266673
42 69.935051266673
42.025 69.935051266673
42.05 69.935051266673
42.075 69.935051266673
42.1 69.935051266673
42.125 69.935051266673
42.15 69.935051266673
42.175 69.935051266673
42.2 69.935051266673
42.225 69.935051266673
42.25 69.935051266673
42.275 69.935051266673
42.3 69.935051266673
42.325 69.935051266673
};
\addplot [ultra thick, black, forget plot]
table {%
9 35
9 70
};
\addplot [ultra thick, black, forget plot]
table {%
11 35
11 70
};
\addplot [ultra thick, black, forget plot]
table {%
18 35
18 70
};
\node at (axis cs:4.5,42)[
  anchor=base,
  text=black,
  rotate=0.0
]{ Accelerating};
\node at (axis cs:10,42)[
  anchor=base,
  text=black,
  rotate=0.0,
  align=center
]{ Bra-\\
king};
\node at (axis cs:14.5,42)[
  anchor=base,
  text=black,
  rotate=0.0
]{ Cruising};
\end{axis}

\end{tikzpicture}%
	\caption{Activities of the ego vehicle considering its speed. The black vertical lines indicate the events. The three different activities that can be qualitatively described as `accelerating', `braking', and `cruising', respectively.}
	\label{fig:example ego states}
	\spaceaftercaption
\end{figure}

%Figure~\ref{fig:example dynamic environment} shows the longitudinal and lateral position of the pickup truck and the station wagon with respect to the ego vehicle. The station wagon accelerates and performs a lane change at approximately $t\approx10$ s. At $t\approx12$ s, the station wagon overtakes the ego vehicle. The ego vehicle overtakes the pickup truck at $t\approx25$ s, as can be seen by the relative longitudinal distance of the pickup truck in Figure~\ref{fig:example dynamic environment}a, which becomes negative at $t\approx25$ s. The events related to the pickup truck and the station wagon are shown in Figure~\ref{fig:example events}. Here, it can be seen that the pickup truck is in the field of view of the ego vehicle's sensors from $t=1$ s until $t=34$ s. The station wagon is in the field of view of the ego vehicle's sensors from $t=0$ s until $t=18$ s.

%\begin{figure}
%	\centering
%	\setlength\figureheight{150pt}
%	\setlength\figurewidth{248pt}
%	\subfloat[Longitudinal distance of other vehicles with respect to ego vehicle.]{% This file was created by matplotlib2tikz v0.6.14.
\begin{tikzpicture}

\definecolor{color0}{rgb}{0.172549019607843,0.627450980392157,0.172549019607843}
\definecolor{color3}{rgb}{0.549019607843137,0.337254901960784,0.294117647058824}
\definecolor{color1}{rgb}{0.83921568627451,0.152941176470588,0.156862745098039}
\definecolor{color2}{rgb}{0.580392156862745,0.403921568627451,0.741176470588235}

\begin{axis}[
xlabel={$t$ [s]},
ylabel={Relative longitudinal distance [m]},
xmin=0, xmax=34.1600000858307,
ymin=-61.3809092368509, ymax=111.328903581739,
width=\figurewidth,
height=\figureheight,
tick align=outside,
tick pos=left,
xmajorgrids,
x grid style={lightgray!92.026143790849673!black},
ymajorgrids,
y grid style={lightgray!92.026143790849673!black},
clip marker paths
]
\addplot [semithick, color0, mark=*, mark size=1, mark options={solid}, only marks, forget plot]
table {%
0 -29.7074271603815
0.0400002002716064 -29.7193473728894
0.0800001621246338 -29.7383384251862
0.120000123977661 -29.7460490026369
0.160000085830688 -29.7664243002673
0.200000047683716 -29.779038823357
0.240000009536743 -29.7981872784185
0.28000020980835 -29.8295946814196
0.320000171661377 -29.8472175355928
0.360000133514404 -29.8765110027307
0.400000095367432 -29.8976779480472
0.440000057220459 -29.9161635570545
0.480000019073486 -29.9500390967969
0.519999980926514 -29.9626746751619
0.56000018119812 -29.9931910108844
0.600000143051147 -30.0158738229457
0.640000104904175 -30.0470681455827
0.680000066757202 -30.0739041547095
0.720000028610229 -30.1103165653785
0.759999990463257 -30.1499582959623
0.800000190734863 -30.1749450570478
0.840000152587891 -30.2110858703618
0.880000114440918 -30.2457953815992
0.920000076293945 -30.263344395089
0.960000038146973 -30.2944718244144
1 -30.3275244008983
1.04000020027161 -30.3461533459049
1.08000016212463 -30.404412105634
1.12000012397766 -30.4299916279488
1.16000008583069 -30.469466645427
1.20000004768372 -30.5036813379666
1.24000000953674 -30.5424287676051
1.28000020980835 -30.5992346562234
1.32000017166138 -30.6254737172967
1.3600001335144 -30.6676030564413
1.40000009536743 -30.7052177632922
1.44000005722046 -30.7366472687481
1.48000001907349 -30.7958631867077
1.51999998092651 -30.8290239459493
1.56000018119812 -30.8624309720108
1.60000014305115 -30.9005839226102
1.64000010490417 -30.936000716034
1.6800000667572 -30.9846167561773
1.72000002861023 -31.0065385651487
1.75999999046326 -31.0374855482733
1.80000019073486 -31.0730068300436
1.84000015258789 -31.1073708202912
1.88000011444092 -31.1585357137046
1.92000007629395 -31.1704042716519
1.96000003814697 -31.1984069697191
2 -31.2331737318946
2.04000020027161 -31.2569818936063
2.08000016212463 -31.3134767408137
2.12000012397766 -31.33457098291
2.16000008583069 -31.3772719417411
2.20000004768372 -31.4094436982869
2.24000000953674 -31.4405307212692
2.28000020980835 -31.4909659151326
2.32000017166138 -31.5102945930848
2.3600001335144 -31.5454310228088
2.40000009536743 -31.5738279901016
2.44000005722046 -31.6058762154462
2.48000001907349 -31.6507218298466
2.51999998092651 -31.6632214247602
2.56000018119812 -31.6973162838203
2.60000014305115 -31.7268398443266
2.64000010490417 -31.7571284448313
2.6800000667572 -31.808816608831
2.72000002861023 -31.8266725322683
2.75999999046326 -31.8658204385556
2.80000019073486 -31.8940627079883
2.84000015258789 -31.924219830471
2.88000011444092 -31.9840174570636
2.92000007629395 -32.0051623854943
2.96000003814697 -32.0588416641222
3 -32.0921224670674
3.04000020027161 -32.1367741825361
3.08000016212463 -32.196525812773
3.12000012397766 -32.2202876176016
3.16000008583069 -32.2727025826334
3.20000004768372 -32.3074194994151
3.24000000953674 -32.3521726259696
3.28000020980835 -32.4058262929539
3.32000017166138 -32.4296208923552
3.3600001335144 -32.4786247010088
3.40000009536743 -32.5130321161469
3.44000005722046 -32.5584644286209
3.48000001907349 -32.6141678829208
3.51999998092651 -32.6406701762025
3.56000018119812 -32.6823762718086
3.60000014305115 -32.7264484987882
3.64000010490417 -32.7713429856267
3.6800000667572 -32.8255580331006
3.72000002861023 -32.856951402975
3.75999999046326 -32.910074862717
3.80000019073486 -32.9485975998832
3.84000015258789 -32.9895311112195
3.88000011444092 -33.0398992823993
3.92000007629395 -33.0691890743856
3.96000003814697 -33.0977602159146
4 -33.159594566323
4.04000020027161 -33.1731866178852
4.08000016212463 -33.2148813720869
4.12000012397766 -33.2496221080928
4.16000008583069 -33.2803923345346
4.20000004768372 -33.3199595746482
4.24000000953674 -33.3390741809362
4.28000020980835 -33.3788283571612
4.32000017166138 -33.4023925829551
4.3600001335144 -33.4338549808817
4.40000009536743 -33.471955628771
4.44000005722046 -33.4928417260071
4.48000001907349 -33.5457550454139
4.51999998092651 -33.567108358624
4.56000018119812 -33.6140201087146
4.60000014305115 -33.6598147006916
4.64000010490417 -33.6878860990491
4.6800000667572 -33.7355324311775
4.72000002861023 -33.7683442772741
4.75999999046326 -33.8261440587339
4.80000019073486 -33.8687390755385
4.84000015258789 -33.9099162060938
4.88000011444092 -33.9738719872839
4.92000007629395 -33.9980793437735
4.96000003814697 -34.0767215251435
5 -34.1119637704469
5.04000020027161 -34.1593912526369
5.08000016212463 -34.214012101389
5.12000012397766 -34.2728830271117
5.16000008583069 -34.3172987296166
5.20000004768372 -34.383580664924
5.24000000953674 -34.4472430902279
5.28000020980835 -34.4886011693106
5.32000017166138 -34.5505265284828
5.3600001335144 -34.5957848910039
5.40000009536743 -34.6659730994397
5.44000005722046 -34.714066504248
5.48000001907349 -34.7630324400961
5.51999998092651 -34.8323117799755
5.56000018119812 -34.8743756620715
5.60000014305115 -34.9418214232555
5.64000010490417 -34.9858726325892
5.6800000667572 -35.0257822271687
5.72000002861023 -35.0993768488479
5.75999999046326 -35.1364252221247
5.80000019073486 -35.2288924718887
5.84000015258789 -35.2767301638487
5.88000011444092 -35.320289510335
5.92000007629395 -35.3877188390488
5.96000003814697 -35.4358495067754
6 -35.5199657559551
6.04000020027161 -35.60914191743
6.08000016212463 -35.6483753038101
6.12000012397766 -35.7333624830098
6.16000008583069 -35.7971334856502
6.20000004768372 -35.8847336545605
6.24000000953674 -35.9162921739353
6.28000020980835 -35.9859252078058
6.32000017166138 -36.0500405582061
6.3600001335144 -36.0750500777667
6.40000009536743 -36.1402647152117
6.44000005722046 -36.2106582522938
6.48000001907349 -36.2388524432699
6.51999998092651 -36.3051267675819
6.56000018119812 -36.3660438160696
6.60000014305115 -36.4123215717282
6.64000010490417 -36.4695835284092
6.6800000667572 -36.5204593584458
6.72000002861023 -36.5878880129167
6.75999999046326 -36.6522965958156
6.80000019073486 -36.7036051560499
6.84000015258789 -36.7577059540854
6.88000011444092 -36.8058453285303
6.92000007629395 -36.8648800318606
6.96000003814697 -36.9153492368514
7 -36.9676276877672
7.04000020027161 -37.0200439129221
7.08000016212463 -37.0696535416864
7.12000012397766 -37.1215291295975
7.16000008583069 -37.1879415296498
7.20000004768372 -37.2439704962035
7.24000000953674 -37.3016181066505
7.28000020980835 -37.3486455192688
7.32000017166138 -37.4160809300884
7.3600001335144 -37.4559836321223
7.40000009536743 -37.5210997451741
7.44000005722046 -37.5828120242459
7.48000001907349 -37.6020625552355
7.51999998092651 -37.6591875650829
7.56000018119812 -37.6927522353089
7.60000014305115 -37.7192409635682
7.64000010490417 -37.787252095015
7.6800000667572 -37.8113144948238
7.72000002861023 -37.848399161061
7.75999999046326 -37.8717833188257
7.80000019073486 -37.8933181017856
7.84000015258789 -37.9485304696154
7.88000011444092 -37.9624663773448
7.92000007629395 -38.0146831032653
7.96000003814697 -38.0409629800179
8 -38.0730814660183
8.04000020027161 -38.0862488593975
8.08000016212463 -38.099714781576
8.12000012397766 -38.1352397092269
8.16000008583069 -38.1369820298005
8.20000004768372 -38.147280783267
8.24000000953674 -38.1719449717675
8.28000020980835 -38.1563860035712
8.32000017166138 -38.1864213254939
8.3600001335144 -38.1782663117428
8.40000009536743 -38.1817153648408
8.44000005722046 -38.1742719525719
8.48000001907349 -38.161119165954
8.51999998092651 -38.1688152827337
8.56000018119812 -38.1701919391071
8.60000014305115 -38.1656855714955
8.64000010490417 -38.1476529812517
8.6800000667572 -38.1142517326825
8.72000002861023 -38.0963731100655
8.75999999046326 -38.0762722647596
8.80000019073486 -38.0255003242382
8.84000015258789 -38.0151752920829
8.88000011444092 -37.9693337524677
8.92000007629395 -37.9217376187516
8.96000003814697 -37.8897895912669
9 -37.812877766115
9.04000020027161 -37.7924157517209
9.08000016212463 -37.7151076064274
9.12000012397766 -37.640522485126
9.16000008583069 -37.5990262113537
9.20000004768372 -37.4940963246008
9.24000000953674 -37.4606192931769
9.28000020980835 -37.3371298010206
9.32000017166138 -37.23826584788
9.3600001335144 -37.1515125689275
9.40000009536743 -36.9918119513386
9.44000005722046 -36.9285336874655
9.48000001907349 -36.7962827252013
9.51999998092651 -36.6931405823179
9.56000018119812 -36.5694319435843
9.60000014305115 -36.4851381752378
9.64000010490417 -36.2599097684451
9.6800000667572 -36.1117586445798
9.72000002861023 -35.9809530615439
9.75999999046326 -35.8077820722556
9.80000019073486 -35.7035651437564
9.84000015258789 -35.4261585182394
9.88000011444092 -35.2410680008943
9.92000007629395 -35.0992599954934
9.96000003814697 -34.8726392893768
10 -34.7454597982378
10.0400002002716 -34.4411446770082
10.0800001621246 -34.233127927575
10.1200001239777 -34.0313373131121
10.1600000858307 -33.8354171908941
10.2000000476837 -33.5654751741386
10.2400000095367 -33.3117328715616
10.2800002098083 -33.0746374579539
10.3200001716614 -32.7310973859676
10.3600001335144 -32.5877009884753
10.4000000953674 -32.2912747910868
10.4400000572205 -32.0059770649514
10.4800000190735 -31.74191626227
10.5199999809265 -31.4421755285694
10.5600001811981 -31.2038473013163
10.6000001430511 -30.8992015238127
10.6400001049042 -30.5676398735959
10.6800000667572 -30.2787464374487
10.7200000286102 -29.9570306091809
10.7599999904633 -29.6933969814163
10.8000001907349 -29.3992908669352
10.8400001525879 -29.0010395113823
10.8800001144409 -28.6816084341845
10.9200000762939 -28.2293537422502
10.960000038147 -28.0533175649962
11 -27.7481268427546
11.0400002002716 -27.1939342429196
11.0800001621246 -26.9676291924116
11.1200001239777 -26.4778014131789
11.1600000858307 -26.2981569862113
11.2000000476837 -25.7823998891727
11.2400000095367 -25.4005123981951
11.2800002098083 -25.1704446725089
11.3200001716614 -24.6509466389034
11.3600001335144 -24.4665139782737
11.4000000953674 -23.9171570582475
11.4400000572205 -23.5083246894446
11.4800000190735 -23.2748806240961
11.5199999809265 -22.9254589927186
11.5600001811981 -22.5327539030077
11.6000001430511 -22.1384555607128
11.6400001049042 -21.7594986876302
11.6800000667572 -21.3516546520277
11.7200000286102 -20.9537587907071
11.7599999904633 -20.5428801713952
11.8000001907349 -20.0988092930475
11.8400001525879 -19.7071988682437
11.8800001144409 -19.2228427529953
11.9200000762939 -18.6044093580276
11.960000038147 -18.4035304260287
12 -17.9787021192078
12.0400002002716 -17.5429201174975
12.0800001621246 -17.079700822349
12.1200001239777 -16.6419597842887
12.1600000858307 -16.1375779392056
12.2000000476837 -15.7486996354583
12.2400000095367 -15.4589798666839
12.2800002098083 -14.8264947746738
12.3200001716614 -14.1703211419299
12.3600001335144 -13.8968332323839
12.4000000953674 -13.7114346185263
12.4400000572205 -12.9626582077162
12.4800000190735 -12.4685804669971
12.5199999809265 -12.2724841421514
12.5600001811981 -11.509945696087
12.6000001430511 -11.0268636638066
12.6400001049042 -10.7084681175693
12.6800000667572 -10.0204804755667
12.7200000286102 -9.3287742265511
12.7599999904633 -8.91321427856565
12.8000001907349 -8.21998669377172
12.8400001525879 -7.80847770170476
12.8800001144409 -7.268529564657
12.9200000762939 -6.72207962568973
12.960000038147 -6.23143474607969
13 -5.53360740659106
13.0400002002716 -4.8357759077234
13.0800001621246 -4.13794856823478
13.1200001239777 -3.44012122874616
13.1600000858307 -2.74229388925754
13.2000000476837 -2.04446654976891
13.2400000095367 -1.34663921028029
13.2800002098083 -0.648807711412632
13.3200001716614 0.0490196280759907
13.3600001335144 0.746846967564613
13.4000000953674 1.44467430705324
13.4400000572205 2.14250164654186
13.4800000190735 2.84032898603048
13.5199999809265 3.5381563255191
13.5600001811981 4.23598782438676
13.6000001430511 4.93381516387539
13.6400001049042 5.63164250336401
13.6800000667572 6.32946984285263
13.7200000286102 6.59663515279499
13.7599999904633 6.91507421624192
13.8000001907349 7.36358850169745
13.8400001525879 7.51360365835171
13.8800001144409 7.94167528904836
13.9200000762939 8.37009874303658
13.960000038147 8.68058540081438
14 8.91829565265107
14.0400002002716 9.67761481379603
14.0800001621246 10.3013219123295
14.1200001239777 10.6365294901971
14.1600000858307 11.1594149825978
14.2000000476837 11.7254737591356
14.2400000095367 12.1136385812315
14.2800002098083 12.5756518553208
14.3200001716614 13.0625135830778
14.3600001335144 13.4497263736266
14.4000000953674 13.9732587770159
14.4400000572205 14.3082878775567
14.4800000190735 14.9950598698833
14.5199999809265 15.652072886156
14.5600001811981 15.9875154633537
14.6000001430511 16.3510255465499
14.6400001049042 16.9892150072556
14.6800000667572 17.459709168912
14.7200000286102 18.0131879948458
14.7599999904633 18.4210395850077
14.8000001907349 18.8464314271132
14.8400001525879 19.3309100337337
14.8800001144409 20.0135110500887
14.9200000762939 20.2123697165607
14.960000038147 21.0479691289111
15 21.3648939073973
15.2000000476837 24.3686649939609
15.2400000095367 24.4334943564572
15.2800002098083 24.5685119330574
15.3200001716614 24.6704316239793
15.3600001335144 25.498243723534
15.4000000953674 25.5499004733701
15.4400000572205 26.2889026424637
15.4800000190735 26.4228098014428
15.5199999809265 26.5489864090632
15.5600001811981 26.6389803841303
15.6000001430511 26.9724396863803
15.6400001049042 27.1662413898084
15.6800000667572 27.2506874450501
15.7200000286102 28.6347522694105
15.7599999904633 29.8581646424063
15.8000001907349 30.2339136086321
15.8400001525879 30.3084825929363
15.8800001144409 31.4282980904154
15.9200000762939 32.3824495788267
15.960000038147 32.700553968125
16 33.6328035879069
16.0400002002716 33.8673750668913
16.0800001621246 34.381195706681
16.1200001239777 35.4193410273729
16.1600000858307 36.0757201501856
16.2000000476837 36.8633581596714
16.2400000095367 37.1473289805945
16.2800002098083 38.0630464370497
16.3200001716614 38.761957506611
16.3600001335144 39.1961462417075
16.4000000953674 39.937083341023
16.4400000572205 40.2649042567045
16.4800000190735 40.9478108444055
16.5199999809265 41.6263547744275
16.5600001811981 41.8948139790791
16.6000001430511 42.1654710781986
16.6400001049042 42.7387004257816
16.6800000667572 43.8411866273163
16.7200000286102 44.6673733483331
16.7599999904633 45.1244529573905
16.8000001907349 45.9967583665875
16.8400001525879 46.3654057657714
16.8800001144409 47.1395076419012
16.9200000762939 47.8641584779743
16.960000038147 48.2118127102403
17 48.9268941407809
17.0400002002716 49.2571265880215
17.0800001621246 49.8908053678169
17.1200001239777 50.1287872850517
17.1600000858307 51.0050614973097
17.2000000476837 51.6235918735074
17.2400000095367 52.3617076193077
17.2800002098083 52.7698168235875
17.3200001716614 53.5343237969464
17.3600001335144 53.9851337655946
17.4000000953674 54.1722581298818
17.4400000572205 55.2373630401016
17.4800000190735 55.8044304237519
17.5199999809265 56.0278847427908
17.5600001811981 57.0066121870695
17.6000001430511 57.6472276676432
17.6400001049042 58.1377461491975
17.6800000667572 58.8690445852444
17.7200000286102 59.737045744434
};
\addplot [semithick, color2, mark=*, mark size=1, mark options={solid}, only marks, forget plot]
table {%
0.800000190734863 101.481304680266
0.840000152587891 101.526985366871
0.880000114440918 101.565141825275
0.920000076293945 101.603370900992
0.960000038146973 101.646926877722
1 101.667385767189
1.04000020027161 101.69353465608
1.08000016212463 101.769457757408
1.12000012397766 101.813049728076
1.16000008583069 101.881585933119
1.20000004768372 101.943780187674
1.24000000953674 102.000734001884
1.28000020980835 102.077222044298
1.32000017166138 102.126680899333
1.3600001335144 102.110322336566
1.40000009536743 102.079564758151
1.44000005722046 102.055635798928
1.48000001907349 102.096734356823
1.51999998092651 102.126532141361
1.56000018119812 102.161250392581
1.60000014305115 102.18813955109
1.64000010490417 102.216541846394
1.6800000667572 102.252364719265
1.72000002861023 102.273045256319
1.75999999046326 102.336399366579
1.80000019073486 102.387245738459
1.84000015258789 102.435107366582
1.88000011444092 102.456775613236
1.92000007629395 102.55990562101
1.96000003814697 102.597046040619
2 102.62334959074
2.04000020027161 102.63779822287
2.08000016212463 102.726229786051
2.12000012397766 102.787255714891
2.16000008583069 102.805661899247
2.20000004768372 102.99257542323
2.24000000953674 103.01288461376
2.28000020980835 103.111777668471
2.32000017166138 103.194413163299
2.3600001335144 103.278176584017
2.40000009536743 103.38958907356
2.44000005722046 103.412386269027
2.48000001907349 103.47845754453
2.51999998092651 103.412045813178
2.56000018119812 103.347810031864
2.60000014305115 103.362673922806
2.64000010490417 103.289602193405
2.6800000667572 103.138496493411
2.72000002861023 103.065028071531
2.75999999046326 102.987044654885
2.80000019073486 102.909754278122
2.84000015258789 102.832580607483
2.88000011444092 102.815989859731
2.92000007629395 102.730714452638
2.96000003814697 102.581560204068
3 102.511840364392
3.04000020027161 102.436903193262
3.08000016212463 102.355229003415
3.12000012397766 102.332721774999
3.16000008583069 102.2535749293
3.20000004768372 102.157414712025
3.24000000953674 102.086958287555
3.28000020980835 101.999710693784
3.32000017166138 101.940805974398
3.3600001335144 101.832651298677
3.40000009536743 101.748329591424
3.44000005722046 101.656420665249
3.48000001907349 101.58479085585
3.51999998092651 101.549968598059
3.56000018119812 101.489744052611
3.60000014305115 101.432703987195
3.64000010490417 101.370124337356
3.6800000667572 101.296042011641
3.72000002861023 101.249227451573
3.75999999046326 101.167825882119
3.80000019073486 101.107207611258
3.84000015258789 101.035188781065
3.88000011444092 100.932492798865
3.92000007629395 100.884001395445
3.96000003814697 100.805232805578
4 100.701300281637
4.04000020027161 100.646773203041
4.08000016212463 100.537578849735
4.12000012397766 100.479427699776
4.16000008583069 100.378905875108
4.20000004768372 100.311561022285
4.24000000953674 100.23184603062
4.28000020980835 100.107443910925
4.32000017166138 100.086792307966
4.3600001335144 99.9634554600161
4.40000009536743 99.9118715765653
4.44000005722046 99.8326502830478
4.48000001907349 99.6991523945235
4.51999998092651 99.6752130427885
4.56000018119812 99.5582558588048
4.60000014305115 99.4875620480634
4.64000010490417 99.4227320332084
4.6800000667572 99.2846585254902
4.72000002861023 99.2659606480502
4.75999999046326 99.1239863613027
4.80000019073486 99.0389680428561
4.84000015258789 98.9748164943285
4.88000011444092 98.8400444662293
4.92000007629395 98.8021551418897
4.96000003814697 98.6503618025472
5 98.5737063134002
5.04000020027161 98.4724011675244
5.08000016212463 98.3350817488554
5.12000012397766 98.2544304569456
5.16000008583069 98.1617105926671
5.20000004768372 98.0490401065435
5.24000000953674 97.9434214909088
5.28000020980835 97.7951467589428
5.32000017166138 97.7160388562133
5.3600001335144 97.5758704656073
5.40000009536743 97.4677594501027
5.44000005722046 97.3651035750117
5.48000001907349 97.2280500001016
5.51999998092651 97.1267791856626
5.56000018119812 96.9988390636681
5.60000014305115 96.8766536203057
5.64000010490417 96.7502318796742
5.6800000667572 96.595016244979
5.72000002861023 96.4909765345074
5.75999999046326 96.3506675396111
5.80000019073486 96.2231530993176
5.84000015258789 96.0940445282467
5.88000011444092 95.9318985765749
5.92000007629395 95.8201664087719
5.96000003814697 95.6880559596902
6 95.5617145112883
6.04000020027161 95.4319133641111
6.08000016212463 95.290892333087
6.12000012397766 95.1832825687161
6.16000008583069 95.0511965021797
6.20000004768372 94.9054138146948
6.24000000953674 94.7717042434579
6.28000020980835 94.6209399415311
6.32000017166138 94.4793964985602
6.3600001335144 94.3239479667263
6.40000009536743 94.1716351935465
6.44000005722046 94.0197089062658
6.48000001907349 93.8496162846404
6.51999998092651 93.7080691100982
6.56000018119812 93.5053054968048
6.60000014305115 93.3972468432639
6.64000010490417 93.2392201299881
6.6800000667572 93.081127059726
6.72000002861023 92.9370776744418
6.75999999046326 92.7424178325964
6.80000019073486 92.6112349464693
6.84000015258789 92.4501744763838
6.88000011444092 92.2602529025735
6.92000007629395 92.1146002937967
6.96000003814697 91.901477039919
7 91.7567898048219
7.04000020027161 91.5933802124764
7.08000016212463 91.3856459812141
7.12000012397766 91.2318763106305
7.16000008583069 91.05425331474
7.20000004768372 90.8780010400769
7.24000000953674 90.6965430504169
7.28000020980835 90.4899699575835
7.32000017166138 90.3190719911581
7.3600001335144 90.1314329415472
7.40000009536743 89.8678534456667
7.44000005722046 89.7105932357827
7.48000001907349 89.5598101428404
7.51999998092651 89.385180428204
7.56000018119812 89.1950403277624
7.60000014305115 89.0904584861419
7.64000010490417 88.7740559649865
7.6800000667572 88.6523217211015
7.72000002861023 88.483337743306
7.75999999046326 88.2812435431842
7.80000019073486 88.179917384301
7.84000015258789 87.8431384234918
7.88000011444092 87.6960592417145
7.92000007629395 87.5249195184424
7.96000003814697 87.3316586626315
8 87.2234866174949
8.04000020027161 86.7690075841983
8.08000016212463 86.5583631163954
8.12000012397766 86.3758981975843
8.16000008583069 85.9997479503018
8.20000004768372 85.8797037677396
8.24000000953674 85.4146080198789
8.28000020980835 85.19328984811
8.32000017166138 84.9937010739686
8.3600001335144 84.7025286749194
8.40000009536743 84.5753696146949
8.44000005722046 84.1079965556528
8.48000001907349 83.8759391888416
8.51999998092651 83.6741330908808
8.56000018119812 83.3917022214828
8.60000014305115 83.2648741314861
8.64000010490417 82.9434520948016
8.6800000667572 82.6705405555949
8.72000002861023 82.3275562403887
8.75999999046326 82.1909267317787
8.80000019073486 81.8554675869018
8.84000015258789 81.7712238912536
8.88000011444092 81.4894668793258
8.92000007629395 81.252011492903
8.96000003814697 81.111811736153
9 80.7889219056851
9.04000020027161 80.6952749607754
9.08000016212463 80.4256673427735
9.12000012397766 80.1731179189155
9.16000008583069 80.023768318968
9.20000004768372 79.8262967206283
9.24000000953674 79.8892582716871
9.28000020980835 79.7639551922948
9.32000017166138 79.6147651942138
9.3600001335144 79.4572805592179
9.40000009536743 79.2871550613963
9.44000005722046 79.1624307891125
9.48000001907349 78.9950910817261
9.51999998092651 78.7732583720481
9.56000018119812 78.6711034929067
9.60000014305115 78.6076642039334
9.64000010490417 78.2965774110562
9.6800000667572 78.1482090560094
9.72000002861023 77.968358724529
9.75999999046326 77.8510652732966
9.80000019073486 77.7550080436522
9.84000015258789 77.483606921287
9.88000011444092 77.3648147486856
9.92000007629395 77.199954486081
9.96000003814697 77.0297278903963
10 76.9939979722676
10.0400002002716 76.6693078498174
10.0800001621246 76.5444326317156
10.1200001239777 76.3546367141134
10.1600000858307 76.1859152552315
10.2000000476837 75.9753319519295
10.2400000095367 75.8666503380427
10.2800002098083 75.7161943208976
10.3200001716614 75.4708107303068
10.3600001335144 75.3935057592043
10.4000000953674 75.163933548456
10.4400000572205 75.0476571020008
10.4800000190735 74.8952086077334
10.5199999809265 74.7158731643012
10.5600001811981 74.5807673227282
10.6000001430511 74.3446630600574
10.6400001049042 74.200188338189
10.6800000667572 74.0300311362025
10.7200000286102 73.8134511400185
10.7599999904633 73.6937341530866
10.8000001907349 73.5295471981026
10.8400001525879 73.3416382460964
10.8800001144409 73.1947327940961
10.9200000762939 72.9524152560025
10.960000038147 72.8766942431157
11 72.7442821386794
11.0400002002716 72.5081499953412
11.0800001621246 72.4115475297112
11.1200001239777 72.2131571529972
11.1600000858307 72.1400693315518
11.2000000476837 71.8960517075357
11.2400000095367 71.7618605381394
11.2800002098083 71.6672934896087
11.3200001716614 71.4594231316205
11.3600001335144 71.387142577576
11.4000000953674 71.1243966418333
11.4400000572205 71.0271575902225
11.4800000190735 70.8976694302601
11.5199999809265 70.8004433525639
11.5600001811981 70.6441063854381
11.6000001430511 70.4869642190315
11.6400001049042 70.3567480318907
11.6800000667572 70.1536921317584
11.7200000286102 70.0321304683785
11.7599999904633 69.9182711179819
11.8000001907349 69.7433549425223
11.8400001525879 69.596418219513
11.8800001144409 69.3925740738487
11.9200000762939 69.2369178161025
11.960000038147 69.1652093230332
12 68.9793267816203
12.0400002002716 68.8227919031087
12.0800001621246 68.6743533082499
12.1200001239777 68.5240155722659
12.1600000858307 68.3239246499761
12.2000000476837 68.1867124200489
12.2400000095367 68.1207254970323
12.2800002098083 67.8795839918312
12.3200001716614 67.7031552670123
12.3600001335144 67.5720209652227
12.4000000953674 67.5090173814588
12.4400000572205 67.2942849645933
12.4800000190735 67.0767863790734
12.5199999809265 67.0145269256827
12.5600001811981 66.7290996267184
12.6000001430511 66.6042031637789
12.6400001049042 66.5182594476209
12.6800000667572 66.2929317666913
12.7200000286102 66.0934819637423
12.7599999904633 65.9537550670393
12.8000001907349 65.7573855118571
12.8400001525879 65.6800711266769
12.8800001144409 65.5067331991486
12.9200000762939 65.3642591699263
12.960000038147 65.1874737130911
13 65.0314839285384
13.0400002002716 64.8260016394543
13.0800001621246 64.7241622982256
13.1200001239777 64.5517780405971
13.1600000858307 64.4496662423462
13.2000000476837 64.2750228351524
13.2400000095367 64.1197544686565
13.2800002098083 63.9956006658267
13.3200001716614 63.8239235176934
13.3600001335144 63.684230734556
13.4000000953674 63.646412660466
13.4400000572205 63.4113680738192
13.4800000190735 63.2575593062375
13.5199999809265 63.2219683607582
13.5600001811981 62.9787905826124
13.6000001430511 62.8097530966952
13.6400001049042 62.6886928021486
13.6800000667572 62.5056263512561
13.7200000286102 62.2962698809915
13.7599999904633 62.1990800204276
13.8000001907349 61.9741356745308
13.8400001525879 61.8384067664811
13.8800001144409 61.6900228698796
13.9200000762939 61.520578837406
13.960000038147 61.3712835800306
14 61.2882388707039
14.0400002002716 61.0777614707949
14.0800001621246 60.8721208536554
14.1200001239777 60.7426215103533
14.1600000858307 60.5652529110394
14.2000000476837 60.3577575218915
14.2400000095367 60.2275598712367
14.2800002098083 60.0633195360733
14.3200001716614 59.8775263221596
14.3600001335144 59.7498937537566
14.4000000953674 59.5479537440933
14.4400000572205 59.4550682828612
14.4800000190735 59.2441864623197
14.5199999809265 59.0239782900699
14.5600001811981 58.9227898398112
14.6000001430511 58.8075831533715
14.6400001049042 58.6272747536641
14.6800000667572 58.4883576822285
14.7200000286102 58.3151069097894
14.7599999904633 58.1817803648228
14.8000001907349 58.0446374164621
14.8400001525879 57.90368694676
14.8800001144409 57.7011295642951
14.9200000762939 57.6491269185244
14.960000038147 57.3918662189517
15 57.2738482121367
15.0400002002716 57.140767299783
15.0800001621246 56.9352796307194
15.1200001239777 56.8749033971799
15.1600000858307 56.6815295200704
15.2000000476837 56.4715277780888
15.2400000095367 56.3701543543903
15.2800002098083 56.1547464667055
15.3200001716614 55.9108216813875
15.3600001335144 55.8163490902298
15.4000000953674 55.701743822161
15.4400000572205 55.4955314710987
15.4800000190735 55.3445586924627
15.5199999809265 55.1796220770011
15.5600001811981 55.0720333679255
15.6000001430511 54.8234931573988
15.6400001049042 54.6227902485371
15.6800000667572 54.5294824312205
15.7200000286102 54.4552960028795
15.7599999904633 54.1995251637345
15.8000001907349 53.9896440220637
15.8400001525879 53.9449902595388
15.8800001144409 53.7395372224728
15.9200000762939 53.557330572432
15.960000038147 53.4161600219286
16 52.8933454087928
16.0400002002716 52.8287269836746
16.0800001621246 52.5958988120601
16.1200001239777 52.4423079533371
16.1600000858307 52.3535474912387
16.2000000476837 52.2108871194068
16.2400000095367 52.1207778991502
16.2800002098083 51.903704573846
16.3200001716614 51.760798637948
16.3600001335144 51.6290850177247
16.4000000953674 51.4615979602077
16.4400000572205 51.3711298913495
16.4800000190735 51.1557696345062
16.5199999809265 50.9565505869596
16.5600001811981 50.8678627867193
16.6000001430511 50.7931877469491
16.6400001049042 50.60271399456
16.6800000667572 50.400335798693
16.7200000286102 50.2500810495967
16.7599999904633 50.1143226950062
16.8000001907349 49.6605169290233
16.8400001525879 49.5427721819597
16.8800001144409 49.1769244162297
16.9200000762939 48.9661154405767
16.960000038147 48.7757367114227
17 48.5644383420695
17.0400002002716 48.3650144847143
17.0800001621246 48.2285070788039
17.1200001239777 48.1587131239721
17.1600000858307 47.8532749736496
17.2000000476837 47.5918577888442
17.2400000095367 47.440558965025
17.2800002098083 47.2987053716188
17.3200001716614 47.0434672121464
17.3600001335144 46.9339045326087
17.4000000953674 46.8686612129131
17.4400000572205 46.6488654722471
17.4800000190735 46.4288592162411
17.5199999809265 46.3696913238691
17.5600001811981 46.1644681866837
17.6000001430511 45.9707507779003
17.6400001049042 45.9121507764248
17.6800000667572 45.7037861452136
17.7200000286102 45.5738659826839
17.7599999904633 45.4974162556209
17.8000001907349 45.2500980906207
17.8400001525879 45.2068670938843
17.8800001144409 45.0974219083437
17.9200000762939 44.962168946462
17.960000038147 44.902468921613
18 44.8578663595617
18.0400002002716 44.7483312594868
18.0800001621246 44.5303715610207
18.1200001239777 44.4072990742334
18.1600000858307 44.2425721938253
18.2000000476837 44.0736742230511
18.2400000095367 43.9250716992774
18.2800002098083 43.4125830179619
18.3200001716614 43.2627285149829
18.3600001335144 43.1015450170635
18.4000000953674 42.9462275140231
18.4400000572205 42.7644965609688
18.4800000190735 42.5838978472748
18.5199999809265 42.5514266770879
18.5600001811981 42.268589014504
18.6000001430511 42.0533921655169
18.6400001049042 41.9679917284866
18.6800000667572 41.8392801879272
18.7200000286102 41.6548477970991
18.7599999904633 41.108161887707
18.8000001907349 40.7470343574023
18.8400001525879 40.4287551458619
18.8800001144409 40.1496807864478
18.9200000762939 39.8330183470425
18.960000038147 39.6826766453596
19 39.5896465481746
19.0400002002716 39.180948770947
19.0800001621246 38.9435419878337
19.1200001239777 38.6127631311138
19.1600000858307 38.5069438402097
19.2000000476837 38.2187979798509
19.2400000095367 37.9880523154225
19.2800002098083 37.8191972757377
19.3200001716614 37.4393978520202
19.3600001335144 37.3447040925294
19.4000000953674 37.2440134860899
19.4400000572205 36.8250454999161
19.4800000190735 36.6382115582674
19.5199999809265 36.3718098585359
19.5600001811981 36.2261817778926
19.6000001430511 35.9622111300469
19.6400001049042 35.8186304817591
19.6800000667572 35.502190601821
19.7200000286102 35.2647618670781
19.7599999904633 35.0647019009266
19.8000001907349 34.8113134998785
19.8400001525879 34.6043074554782
19.8800001144409 34.3549088083109
19.9200000762939 34.1001584362384
19.960000038147 33.9066141441508
20 33.6516875993439
20.0400002002716 33.4624807351775
20.0800001621246 33.2125688848901
20.1200001239777 32.9692411026317
20.1600000858307 32.7548883619074
20.2000000476837 32.5032443090495
20.2400000095367 32.3004009337401
20.2800002098083 32.0532929702767
20.3200001716614 31.8401740098016
20.3600001335144 31.6151487420575
20.4000000953674 31.3665451064917
20.4400000572205 31.1662384056654
20.4800000190735 30.9163665707802
20.5199999809265 30.7153229648211
20.5600001811981 30.4622102289341
20.6000001430511 30.2183949631981
20.6400001049042 30.0250388625391
20.6800000667572 29.7687860190053
20.7200000286102 29.5547277751211
20.7599999904633 29.298235489634
20.8000001907349 29.0485994120736
20.8400001525879 28.8450235084147
20.8800001144409 28.569980872051
20.9200000762939 28.4427271924342
20.960000038147 28.1593901164015
21 28.0129054721838
21.0400002002716 27.7393458578481
21.0800001621246 27.3870792166035
21.1200001239777 27.2283612260144
21.1600000858307 26.9626020073993
21.2000000476837 26.8179353694704
21.2400000095367 26.56275790158
21.2800002098083 26.1927447305952
21.3200001716614 26.012915401594
21.3600001335144 25.7670369710286
21.4000000953674 25.604121147735
21.4400000572205 25.3481422794357
21.4800000190735 24.985332189417
21.5199999809265 24.7180189203482
21.5600001811981 24.6019234601226
21.6000001430511 24.2369073490081
21.6400001049042 24.0194671260888
21.6800000667572 23.7517484831496
21.7200000286102 23.4756911986788
21.7599999904633 23.3234344469056
21.8000001907349 23.1603400349213
21.8400001525879 22.8276357945306
21.8800001144409 22.4754804450949
21.9200000762939 22.2051117929677
21.960000038147 21.9938474312949
22 21.721095581921
22.0400002002716 21.4512012303585
22.0800001621246 21.4938177093354
22.1200001239777 20.9611231369909
22.1600000858307 20.754933347127
22.2000000476837 20.4812174621875
22.2400000095367 20.214181801628
22.2800002098083 20.009337681513
22.3200001716614 19.7405844179157
22.3600001335144 19.5288031747223
22.4000000953674 19.2550208639459
22.4400000572205 18.9829085847778
22.4800000190735 18.8021288543441
22.5199999809265 18.4949580946304
22.5600001811981 18.2908743615117
22.6000001430511 18.0139553841254
22.6400001049042 17.7405577121663
22.6800000667572 17.5543233667195
22.7200000286102 17.2531851706717
22.7599999904633 16.996866964746
22.8000001907349 16.783870561294
22.8400001525879 16.5042519236868
22.8800001144409 16.3530786450374
22.9200000762939 16.0192735586043
22.960000038147 15.8489053919311
23 15.5407619015896
23.0400002002716 15.2617572443505
23.0800001621246 15.1525235334029
23.1200001239777 14.9577532499134
23.1600000858307 14.4972395834266
23.2000000476837 14.377653780386
23.2400000095367 14.1342848793338
23.2800002098083 13.764351130525
23.3200001716614 13.70190527052
23.3600001335144 13.3203086557933
23.4000000953674 13.0949248431934
23.4400000572205 12.9470117639921
23.4800000190735 12.5233736908904
23.5199999809265 12.4360788038102
23.5600001811981 12.1227882688472
23.6000001430511 11.8367790409102
23.6400001049042 11.5486776019152
23.6800000667572 11.4516330398765
23.7200000286102 11.2115306646938
23.7599999904633 10.9231616360375
23.8000001907349 10.5325445839007
23.8400001525879 10.3937985609482
23.8800001144409 10.0400192288544
23.9200000762939 9.85277089686679
23.960000038147 9.54638944603357
24 9.41215320500305
24.0400002002716 9.01416069670813
24.0800001621246 8.91639514231247
24.1200001239777 8.56138548014133
24.1600000858307 8.30973078305942
24.2000000476837 8.19010770008754
24.2400000095367 7.83452820983257
24.2800002098083 7.73887402655782
24.3200001716614 7.44058631563712
24.3600001335144 7.25586748671594
24.4000000953674 7.19955539676812
24.4400000572205 7.04521880163156
24.4800000190735 6.95760035088824
24.5199999809265 6.72991129099319
24.5600001811981 6.57797092521832
24.6000001430511 6.54692020296898
24.6400001049042 6.05203488708139
24.6800000667572 5.72135322686042
24.7200000286102 5.39067156663944
24.7599999904633 5.05998990641847
24.8000001907349 4.72930627517933
24.8400001525879 4.39862461495835
24.8800001144409 4.06794295473738
24.9200000762939 3.7372612945164
24.960000038147 3.40657963429543
25 3.07589797407446
25.0400002002716 2.74521434283531
25.0800001621246 2.41453268261434
25.1200001239777 2.08385102239337
25.1600000858307 1.75316936217239
25.2000000476837 1.42248770195142
25.2400000095367 1.09180604173044
25.2800002098083 0.761122410491302
25.3200001716614 0.430440750270328
25.3600001335144 0.0997590900493535
25.4000000953674 -0.230922570171621
25.4400000572205 -0.561604230392594
25.4800000190735 -0.892285890613568
25.5199999809265 -1.22296755083454
25.5600001811981 -1.55365118207368
25.6000001430511 -1.88433284229466
25.6400001049042 -2.21501450251563
25.6800000667572 -2.54569616273661
25.7200000286102 -2.87637782295758
25.7599999904633 -3.20705948317855
25.8000001907349 -3.5377431144177
25.8400001525879 -3.86842477463867
25.8800001144409 -4.19910643485965
25.9200000762939 -4.52978809508062
25.960000038147 -4.86046975530159
26 -5.19115141552257
26.0400002002716 -5.52183504676171
26.0800001621246 -5.85251670698268
26.1200001239777 -6.18319836720366
26.1600000858307 -6.34269787828816
26.2000000476837 -6.4551393305901
26.2400000095367 -6.75037590070315
26.2800002098083 -6.90046827996957
26.3200001716614 -7.19344752401412
26.3600001335144 -7.34442746254354
26.4000000953674 -7.41793076970134
26.4400000572205 -7.62233715359798
26.4800000190735 -7.84928645275249
26.5199999809265 -8.04018979420289
26.5600001811981 -8.18588997182269
26.6000001430511 -8.29410446614202
26.6400001049042 -8.65567044169256
26.6800000667572 -8.81020059913317
26.7200000286102 -9.44169753260576
26.7599999904633 -9.57321300304284
26.8000001907349 -9.66905491330908
26.8400001525879 -9.92303286529568
26.8800001144409 -10.1837382522117
26.9200000762939 -10.437642173385
26.960000038147 -10.5901330325814
27 -10.714304641655
27.0400002002716 -11.115113244512
27.0800001621246 -11.1807890889995
27.1200001239777 -11.5929128578919
27.1600000858307 -11.7531271840307
27.2000000476837 -11.8920918047334
27.2400000095367 -12.3448769610441
27.2800002098083 -12.4047798967331
27.3200001716614 -12.8432828591212
27.3600001335144 -13.0121988515566
27.4000000953674 -13.1575324731275
27.4400000572205 -13.5198671809085
27.4800000190735 -13.6593364139644
27.5199999809265 -14.1228665389553
27.5600001811981 -14.3009511535074
27.6000001430511 -14.4450129154975
27.6400001049042 -14.8259638117888
27.6800000667572 -14.9741990941238
27.7200000286102 -15.4201923588407
27.7599999904633 -15.6063091848428
27.8000001907349 -15.744839267667
27.8400001525879 -16.1226214781709
27.8800001144409 -16.26811134932
27.9200000762939 -16.7081152942865
27.960000038147 -16.9013549478186
28 -17.031722869433
28.0400002002716 -17.4857008491854
28.0800001621246 -17.5482300134117
28.1200001239777 -17.989597254742
28.1600000858307 -18.1812828324637
28.2000000476837 -18.3161401467132
28.2400000095367 -18.7027211175937
28.2800002098083 -18.8243539524337
28.3200001716614 -19.2614023603674
28.3600001335144 -19.4496491320879
28.4000000953674 -19.5795754933843
28.4400000572205 -19.9885586535365
28.4800000190735 -20.0871460088183
28.5199999809265 -20.5337730135634
28.5600001811981 -20.7258404268814
28.6000001430511 -20.8391770099352
28.6400001049042 -21.2792300030942
28.6800000667572 -21.3372227602304
28.7200000286102 -21.7875850943765
28.7599999904633 -21.966646818948
28.8000001907349 -22.0789943004893
28.8400001525879 -22.5296290901115
28.8800001144409 -22.5779580321941
28.9200000762939 -23.0307706867843
28.960000038147 -23.2094415304346
29 -23.329274779735
29.0400002002716 -23.7829151760325
29.0800001621246 -23.8514600270719
29.1200001239777 -24.29606071323
29.1600000858307 -24.4764859031711
29.2000000476837 -24.6006154368442
29.2400000095367 -25.04397066196
29.2800002098083 -25.1032559630439
29.3200001716614 -25.5465036126388
29.3600001335144 -25.7271658477785
29.4000000953674 -25.8584042771254
29.4400000572205 -26.3298960640313
29.4800000190735 -26.3803117962343
29.5199999809265 -26.8196697242493
29.5600001811981 -27.000737984521
29.6000001430511 -27.120171080629
29.6400001049042 -27.5609671461189
29.6800000667572 -27.613976538194
29.7200000286102 -28.0602445297445
29.7599999904633 -28.2402723911637
29.8000001907349 -28.350380686883
29.8400001525879 -28.7974325652121
29.8800001144409 -28.8449194284676
29.9200000762939 -29.2919462002501
29.960000038147 -29.4713486230939
30 -29.5891572474447
30.0400002002716 -30.0384627457734
30.0800001621246 -30.085822151912
30.1200001239777 -30.5276554665434
30.1600000858307 -30.7051624945307
30.2000000476837 -30.8220865840685
30.2400000095367 -31.2604995585916
30.2800002098083 -31.4663039977349
30.3200001716614 -31.50723830469
30.3600001335144 -31.8893539981473
30.4000000953674 -32.2303377878834
30.4400000572205 -32.4776080970605
30.4800000190735 -32.5241237622886
30.5199999809265 -32.9659232401609
30.5600001811981 -33.1411500857121
30.6000001430511 -33.2644369610498
30.6400001049042 -33.696888092114
30.6800000667572 -33.9012092473549
30.7200000286102 -34.179165681313
30.7599999904633 -34.3537292364963
30.8000001907349 -34.6670679562812
30.8400001525879 -34.7327897953492
30.8800001144409 -35.1076711571714
30.9200000762939 -35.4016810404628
30.960000038147 -35.5708997145193
31 -35.68662248896
31.0400002002716 -36.1101107898867
31.0800001621246 -36.3245547379538
31.1200001239777 -36.3940119977815
31.1600000858307 -36.7792782806591
31.2000000476837 -36.8941622725961
31.2400000095367 -37.2641448524082
31.2800002098083 -37.5294697985701
31.3200001716614 -37.6197988614749
31.3600001335144 -37.9410825236919
31.4000000953674 -38.3351562655953
31.4400000572205 -38.460242380108
31.4800000190735 -38.7578077735398
31.5199999809265 -39.0553079623751
31.5600001811981 -39.2124785794003
31.6000001430511 -39.339861883207
31.6400001049042 -39.6392979916636
31.6800000667572 -39.9714506718101
31.7200000286102 -40.2120387440227
31.7599999904633 -40.3800448817892
31.8000001907349 -40.7808287118187
31.8400001525879 -40.8529843019987
31.8800001144409 -41.1867113863664
31.9200000762939 -41.3927126456638
31.960000038147 -41.6663102513758
32 -41.9992190241483
32.0400002002716 -42.2862690839338
32.0800001621246 -42.2896692464965
32.1200001239777 -42.5738616783146
32.1600000858307 -42.9095866755324
32.2000000476837 -43.1674840022533
32.2400000095367 -43.2791464588245
32.2800002098083 -43.6763818520012
32.3200001716614 -44.0221587463711
32.3600001335144 -44.0924282124433
32.4000000953674 -44.364746152427
32.4400000572205 -44.5912162997502
32.4800000190735 -44.8292777579081
32.5199999809265 -45.0631080417843
32.5600001811981 -45.3207879184338
32.6000001430511 -45.5526869969035
32.6400001049042 -45.8052508141391
32.6800000667572 -46.083107164588
32.7200000286102 -46.1749218554123
32.7599999904633 -46.5542458971668
32.8000001907349 -46.6748024754779
32.8400001525879 -46.9905359527656
32.8800001144409 -47.3234813646541
32.9200000762939 -47.5254904218091
32.960000038147 -47.8058956548521
33 -48.0321085357282
33.0400002002716 -48.2845930197091
33.0800001621246 -48.5443714751727
33.1200001239777 -48.7856099574819
33.1600000858307 -49.0503223747673
33.2000000476837 -49.2860321341523
33.2400000095367 -49.5399599587145
33.2800002098083 -49.8401721027712
33.3200001716614 -49.9311127161327
33.3600001335144 -50.3093517807902
33.4000000953674 -50.4274854639716
33.4400000572205 -50.7881927215003
33.4800000190735 -51.0582099315816
33.5199999809265 -51.3752998567343
33.5600001811981 -51.5594607958938
33.6000001430511 -51.9085602364794
33.6400001049042 -52.0470561753627
33.6800000667572 -52.2931560794677
33.7200000286102 -52.5886182672257
33.7599999904633 -52.7895713949219
33.8000001907349 -53.1437683703589
33.8400001525879 -53.2811793235214
33.8800001144409 -53.5304631996423
33.9200000762939 -53.5304631996423
33.960000038147 -53.5304631996423
34 -53.5304631996423
34.0400002002716 -53.5304631996423
34.0800001621246 -53.5304631996423
34.1200001239777 -53.5304631996423
34.1600000858307 -53.5304631996423
};
\path [draw=black, fill=black] (axis cs:6,90)
--(axis cs:6.5,85)
--(axis cs:6.125,85)
--(axis cs:6.125,55)
--(axis cs:5.875,55)
--(axis cs:5.875,85)
--(axis cs:5.5,85)
--cycle;

\path [draw=black, fill=black] (axis cs:6,-30)
--(axis cs:5.5,-25)
--(axis cs:5.875,-25)
--(axis cs:5.875,0)
--(axis cs:6.125,0)
--(axis cs:6.125,-25)
--(axis cs:6.5,-25)
--cycle;

\addplot [line width=2.4000000000000004pt, color1, dashed, forget plot]
table {%
0 -29.7188931389674
0.0400002002716064 -29.7299563409853
0.0800001621246338 -29.7411997067783
0.120000123977661 -29.7533279461824
0.160000085830688 -29.7667942653549
0.200000047683716 -29.78089545525
0.240000009536743 -29.7954320841488
0.28000020980835 -29.8107226668905
0.320000171661377 -29.8266145371051
0.360000133514404 -29.8435831895172
0.400000095367432 -29.8612173987661
0.440000057220459 -29.8792910559265
0.480000019073486 -29.8984000859081
0.519999980926514 -29.9181969884245
0.56000018119812 -29.9378888004887
0.600000143051147 -29.9586271579341
0.640000104904175 -29.9809722186025
0.680000066757202 -30.0036080929521
0.720000028610229 -30.0261793658401
0.759999990463257 -30.0491229499174
0.800000190734863 -30.0727676251873
0.840000152587891 -30.0967012373611
0.880000114440918 -30.1204339951422
0.920000076293945 -30.1449705067457
0.960000038146973 -30.1705194735039
1 -30.1960332089379
1.04000020027161 -30.2226529458099
1.08000016212463 -30.2506743427523
1.12000012397766 -30.2790858547478
1.16000008583069 -30.3081599824256
1.20000004768372 -30.3374458924741
1.24000000953674 -30.3667706765184
1.28000020980835 -30.3964421972041
1.32000017166138 -30.4276625305402
1.3600001335144 -30.4589606067176
1.40000009536743 -30.4908940866163
1.44000005722046 -30.5229066418293
1.48000001907349 -30.5548583033759
1.51999998092651 -30.5874071023747
1.56000018119812 -30.6214014755213
1.60000014305115 -30.6558450356474
1.64000010490417 -30.6900410166422
1.6800000667572 -30.7241311668848
1.72000002861023 -30.7587892330752
1.75999999046326 -30.7945972090148
1.80000019073486 -30.8305036649767
1.84000015258789 -30.8662867778163
1.88000011444092 -30.9023954351012
1.92000007629395 -30.939712919949
1.96000003814697 -30.9773674603892
2 -31.0151675563924
2.04000020027161 -31.0531163806854
2.08000016212463 -31.0913630290855
2.12000012397766 -31.1303082083296
2.16000008583069 -31.1690068189263
2.20000004768372 -31.20774495773
2.24000000953674 -31.2474772824826
2.28000020980835 -31.2881010864492
2.32000017166138 -31.328181017626
2.3600001335144 -31.369283837434
2.40000009536743 -31.4106404982149
2.44000005722046 -31.4517258743984
2.48000001907349 -31.4934874444162
2.51999998092651 -31.5357277066921
2.56000018119812 -31.5773942107844
2.60000014305115 -31.6196359297934
2.64000010490417 -31.6627700405243
2.6800000667572 -31.7053379542075
2.72000002861023 -31.748335028955
2.75999999046326 -31.7917505977884
2.80000019073486 -31.8348820314408
2.84000015258789 -31.8789320151247
2.88000011444092 -31.923359919059
2.92000007629395 -31.9675297909875
2.96000003814697 -32.0123794028405
3 -32.0574352176028
3.04000020027161 -32.1024283934086
3.08000016212463 -32.1476527069657
3.12000012397766 -32.1924483062808
3.16000008583069 -32.2381474228078
3.20000004768372 -32.2837373659379
3.24000000953674 -32.3293289259164
3.28000020980835 -32.3753128285893
3.32000017166138 -32.4212036462442
3.3600001335144 -32.4592673242483
3.40000009536743 -32.5050920371814
3.44000005722046 -32.551517298032
3.48000001907349 -32.5979449910911
3.51999998092651 -32.6436784814627
3.56000018119812 -32.6893456115955
3.60000014305115 -32.7342927432693
3.64000010490417 -32.7794284517455
3.6800000667572 -32.8247678869039
3.72000002861023 -32.8708062970172
3.75999999046326 -32.9173511741383
3.80000019073486 -32.9628429776992
3.84000015258789 -33.0217852501104
3.88000011444092 -33.0730370543073
3.92000007629395 -33.1215302165186
3.96000003814697 -33.1702130788799
4 -33.2187705760904
4.04000020027161 -33.2672521660474
4.08000016212463 -33.3158798314681
4.12000012397766 -33.3640687446004
4.16000008583069 -33.413080970087
4.20000004768372 -33.4623609274466
4.24000000953674 -33.5111815383242
4.28000020980835 -33.5601704987456
4.32000017166138 -33.6087291434336
4.3600001335144 -33.6582035889056
4.40000009536743 -33.7086284591084
4.44000005722046 -33.7587357636379
4.48000001907349 -33.8092909008177
4.51999998092651 -33.8602163767029
4.56000018119812 -33.9104551436356
4.60000014305115 -33.9602149857272
4.64000010490417 -34.0102712813008
4.6800000667572 -34.0595618402416
4.72000002861023 -34.1094765000282
4.75999999046326 -34.1589279279602
4.80000019073486 -34.2087467840773
4.84000015258789 -34.2588738511894
4.88000011444092 -34.3096253149394
4.92000007629395 -34.3595248525727
4.96000003814697 -34.4091738029831
5 -34.4596382304195
5.04000020027161 -34.510051647409
5.08000016212463 -34.5596717998596
5.12000012397766 -34.6100142418731
5.16000008583069 -34.6598356758821
5.20000004768372 -34.7095505231636
5.24000000953674 -34.7594245983285
5.28000020980835 -34.8091479736693
5.32000017166138 -34.8597812923927
5.3600001335144 -34.9108997405343
5.40000009536743 -34.9619114460077
5.44000005722046 -35.012712703222
5.48000001907349 -35.0635139202775
5.51999998092651 -35.1140386895103
5.56000018119812 -35.1638787886297
5.60000014305115 -35.2136727264368
5.64000010490417 -35.2638796462667
5.6800000667572 -35.3131849811394
5.72000002861023 -35.3624117420911
5.75999999046326 -35.41246656023
5.80000019073486 -35.4624775136113
5.84000015258789 -35.5131054113043
5.88000011444092 -35.564313966377
5.92000007629395 -35.6155791128782
5.96000003814697 -35.6663992122161
6 -35.7170631068987
6.04000020027161 -35.7678505871746
6.08000016212463 -35.818263324465
6.12000012397766 -35.8701123070442
6.16000008583069 -35.9196865313809
6.20000004768372 -35.969084559418
6.24000000953674 -36.018369180037
6.28000020980835 -36.0677739171766
6.32000017166138 -36.1173218876912
6.3600001335144 -36.1673519508237
6.40000009536743 -36.2172556112757
6.44000005722046 -36.2666630826898
6.48000001907349 -36.3157289909648
6.51999998092651 -36.3647926215127
6.56000018119812 -36.4139701526667
6.60000014305115 -36.4632179974224
6.64000010490417 -36.5121893692333
6.6800000667572 -36.5605265477663
6.72000002861023 -36.6082225151977
6.75999999046326 -36.6552661830385
6.80000019073486 -36.702604210177
6.84000015258789 -36.7496355455635
6.88000011444092 -36.7967507853433
6.92000007629395 -36.8442942416474
6.96000003814697 -36.8917431624572
7 -36.9381884650527
7.04000020027161 -36.9840257807805
7.08000016212463 -37.0295941149252
7.12000012397766 -37.0742821491065
7.16000008583069 -37.1177326955777
7.20000004768372 -37.160925224589
7.24000000953674 -37.2033953462056
7.28000020980835 -37.244819201771
7.32000017166138 -37.2857257201783
7.3600001335144 -37.325516581038
7.40000009536743 -37.3654158583043
7.44000005722046 -37.4052747150782
7.48000001907349 -37.4438445141569
7.51999998092651 -37.4813501532385
7.56000018119812 -37.5179983706579
7.60000014305115 -37.5532756891371
7.64000010490417 -37.5868489676343
7.6800000667572 -37.6191734689775
7.72000002861023 -37.6508982455054
7.75999999046326 -37.6810749318967
7.80000019073486 -37.7096938476282
7.84000015258789 -37.7373698531326
7.88000011444092 -37.7642174209035
7.92000007629395 -37.7900940968963
7.96000003814697 -37.8147540802152
8 -37.8383397184766
8.04000020027161 -37.8603338082544
8.08000016212463 -37.8802792019204
8.12000012397766 -37.8985402810716
8.16000008583069 -37.9158260004466
8.20000004768372 -37.9315331652459
8.24000000953674 -37.9442846229136
8.28000020980835 -37.9553907017183
8.32000017166138 -37.9648760816569
8.3600001335144 -37.9705257162266
8.40000009536743 -37.9716108073267
8.44000005722046 -37.9690704723479
8.48000001907349 -37.9617984915756
8.51999998092651 -37.9491842414429
8.56000018119812 -37.9334883182979
8.60000014305115 -37.9146903349072
8.64000010490417 -37.8917562236944
8.6800000667572 -37.8650619844981
8.72000002861023 -37.8351236944606
8.75999999046326 -37.8002980688337
8.80000019073486 -37.761130469867
8.84000015258789 -37.7167650033463
8.88000011444092 -37.6671386003961
8.92000007629395 -37.6123557220074
8.96000003814697 -37.5505932142281
9 -37.4838478676474
9.04000020027161 -37.4117540899989
9.08000016212463 -37.3345321074775
9.12000012397766 -37.2527519864792
9.16000008583069 -37.1664856947391
9.20000004768372 -37.0749866664422
9.24000000953674 -36.9792316795759
9.28000020980835 -36.8782455112196
9.32000017166138 -36.7719483427981
9.3600001335144 -36.660439164239
9.40000009536743 -36.5451999201506
9.44000005722046 -36.4198624717382
9.48000001907349 -36.2916153775127
9.51999998092651 -36.1625755319952
9.56000018119812 -36.0287222928359
9.60000014305115 -35.890168287735
9.64000010490417 -35.7468003044487
9.6800000667572 -35.5988831835993
9.72000002861023 -35.447273049114
9.75999999046326 -35.2905741975939
9.80000019073486 -35.1297702756238
9.84000015258789 -34.9648442664693
9.88000011444092 -34.7946991999288
9.92000007629395 -34.6193052207576
9.96000003814697 -34.439131312072
10 -34.2541185019251
10.0400002002716 -34.0640385965227
10.0800001621246 -33.8698366940526
10.1200001239777 -33.6553297692756
10.1600000858307 -33.4474959585272
10.2000000476837 -33.237471066401
10.2400000095367 -33.0225184184428
10.2800002098083 -32.8025488747119
10.3200001716614 -32.5779124666641
10.3600001335144 -32.3481764405966
10.4000000953674 -32.1142898460657
10.4400000572205 -31.8743998250167
10.4800000190735 -31.6292110225473
10.5199999809265 -31.3785994291177
10.5600001811981 -31.1231872712592
10.6000001430511 -30.8792270835557
10.6400001049042 -30.6331333950744
10.6800000667572 -30.3649801025207
10.7200000286102 -30.0917225137079
10.7599999904633 -29.8129063037376
10.8000001907349 -29.5297489690965
10.8400001525879 -29.2411685225155
10.8800001144409 -28.9470928298366
10.9200000762939 -28.6478473399682
10.960000038147 -28.3429305891538
11 -28.0329702322223
11.0400002002716 -27.7171549238476
11.0800001621246 -27.395326604608
11.1200001239777 -27.0690414917607
11.1600000858307 -26.7383646979205
11.2000000476837 -26.4021243944844
11.2400000095367 -26.0605019317856
11.2800002098083 -25.7133163891195
11.3200001716614 -25.360445619182
11.3600001335144 -25.002200562557
11.4000000953674 -24.638283264491
11.4400000572205 -24.2688907406558
11.4800000190735 -23.8946273118927
11.5199999809265 -23.5147698131029
11.5600001811981 -23.1288586413156
11.6000001430511 -22.7364755500646
11.6400001049042 -22.3380911250798
11.6800000667572 -21.9344135477036
11.7200000286102 -21.5254500353819
11.7599999904633 -21.1106676904657
11.8000001907349 -20.689129993824
11.8400001525879 -20.29706282257
11.8800001144409 -19.8640017509812
11.9200000762939 -19.4243193013899
11.960000038147 -18.9791681990435
12 -18.5287150879937
12.0400002002716 -18.0725029798952
12.0800001621246 -17.6104655563942
12.1200001239777 -17.1429110009244
12.1600000858307 -16.6699749198931
12.2000000476837 -16.1912963054992
12.2400000095367 -15.7054675027534
12.2800002098083 -15.2131750129306
12.3200001716614 -14.7138768417118
12.3600001335144 -14.2095792255251
12.4000000953674 -13.6978115715972
12.4400000572205 -13.179520582162
12.4800000190735 -12.6543907475131
12.5199999809265 -12.1225540712096
12.5600001811981 -11.584035230646
12.6000001430511 -11.0397221572369
12.6400001049042 -10.4895868694578
12.6800000667572 -9.933095934613
12.7200000286102 -9.37020839303646
12.7599999904633 -8.80109403413127
12.8000001907349 -8.22483974919149
12.8400001525879 -7.64114440126028
12.8800001144409 -7.05104279400427
12.9200000762939 -6.45465243026399
12.960000038147 -5.85060706217353
13 -5.23925471207057
13.0400002002716 -4.62149392747352
13.0800001621246 -3.99563893770535
13.1200001239777 -3.36218795751825
13.1600000858307 -2.72167172545472
13.2000000476837 -2.02333188773975
13.2400000095367 -1.47429107590696
13.2800002098083 -0.925342423273158
13.3200001716614 -0.376433323753442
13.3600001335144 0.171123961537185
13.4000000953674 0.718733085198463
13.4400000572205 1.26526814659516
13.4800000190735 1.81135029260447
13.5199999809265 2.35725936841064
13.5600001811981 2.90216660249036
13.6000001430511 3.44770807447346
13.6400001049042 3.99348754269006
13.6800000667572 4.53927865494391
13.7200000286102 5.08476626977801
13.7599999904633 5.63016415363298
13.8000001907349 6.1757693812579
13.8400001525879 6.72150995431999
13.8800001144409 7.26667416943683
13.9200000762939 7.81166418847609
13.960000038147 8.35744923760418
14 8.9025868554062
14.0400002002716 9.44769869378443
14.0800001621246 9.99285118839083
14.1200001239777 10.5377816516648
14.1600000858307 11.0836967334246
14.2000000476837 11.6285997311623
14.2400000095367 12.1738728108323
14.2800002098083 12.7198306486966
14.3200001716614 13.2654431479518
14.3600001335144 13.8105008968487
14.4000000953674 14.3552822839232
14.4400000572205 14.9005741769356
14.4800000190735 15.4461069554744
14.5199999809265 15.9921543134437
14.5600001811981 16.5379013998782
14.6000001430511 17.0833472963177
14.6400001049042 17.6290455360846
14.6800000667572 18.1757509891891
14.7200000286102 18.7226558340471
14.7599999904633 19.2692106163673
14.8000001907349 19.8160475314199
14.8400001525879 20.3629392388697
14.8800001144409 20.9102117508727
14.9200000762939 21.4578481938898
14.960000038147 22.0053472193294
15 22.5528425907148
15.2000000476837 25.2899095826299
15.2400000095367 25.8377236079452
15.2800002098083 26.3855831254082
15.3200001716614 26.9330683431346
15.3600001335144 27.4800058034118
15.4000000953674 28.0274731093809
15.4400000572205 28.5748163464334
15.4800000190735 29.1219529713711
15.5199999809265 29.6684568284541
15.5600001811981 30.2158317122956
15.6000001430511 30.7633607296686
15.6400001049042 31.3102488882495
15.6800000667572 31.8569314564365
15.7200000286102 32.4038060012691
15.7599999904633 32.951076337087
15.8000001907349 33.4984552516116
15.8400001525879 34.0458504458165
15.8800001144409 34.5934387558902
15.9200000762939 35.1403920420385
15.960000038147 35.6871319056818
16 36.2348834702941
16.0400002002716 36.7833425578203
16.0800001621246 37.3312715881442
16.1200001239777 37.8806158064033
16.1600000858307 38.4297826294229
16.2000000476837 38.9794293522664
16.2400000095367 39.5301103399228
16.2800002098083 40.082624553994
16.3200001716614 40.6350632341094
16.3600001335144 41.1870866521758
16.4000000953674 41.7397438571715
16.4400000572205 42.2921193874646
16.4800000190735 42.8451485132065
16.5199999809265 43.39727776784
16.5600001811981 43.9503692262001
16.6000001430511 44.502402537165
16.6400001049042 45.056495029347
16.6800000667572 45.6096962532241
16.7200000286102 46.1620915321461
16.7599999904633 46.7140196689211
16.8000001907349 47.2668847594364
16.8400001525879 47.8197911967044
16.8800001144409 48.3717706872225
16.9200000762939 48.9230596652657
16.960000038147 49.4758804491005
17 50.0271016965841
17.0400002002716 50.5774130986268
17.0800001621246 51.1242836598685
17.1200001239777 51.672545309317
17.1600000858307 52.2201959140254
17.2000000476837 52.7670063488809
17.2400000095367 53.3131671285719
17.2800002098083 53.8577651885671
17.3200001716614 54.4016310368785
17.3600001335144 54.9443054556316
17.4000000953674 55.4838825124497
17.4400000572205 56.022849208036
17.4800000190735 56.5609616948444
17.5199999809265 57.0969222566835
17.5600001811981 57.6309200385135
17.6000001430511 58.1645269295756
17.6400001049042 58.6974319173026
17.6800000667572 59.2278827586706
17.7200000286102 59.7572335684629
};
\addplot [line width=2.4000000000000004pt, color3, dashed, forget plot]
table {%
0.800000190734863 101.465466608788
0.840000152587891 101.508171380934
0.880000114440918 101.550091975127
0.920000076293945 101.590320374476
0.960000038146973 101.628320310422
1 101.665256728601
1.04000020027161 101.700516717164
1.08000016212463 101.733437676589
1.12000012397766 101.764888781866
1.16000008583069 101.79428611963
1.20000004768372 101.822255750803
1.24000000953674 101.849203872278
1.28000020980835 101.874684713105
1.32000017166138 101.897256684029
1.3600001335144 101.918547127497
1.40000009536743 101.937959382701
1.44000005722046 101.955878449043
1.48000001907349 101.972811025616
1.51999998092651 101.987941380819
1.56000018119812 102.000239689067
1.60000014305115 102.010907037484
1.64000010490417 102.020632319241
1.6800000667572 102.02944605594
1.72000002861023 102.036526461145
1.75999999046326 102.041009300919
1.80000019073486 102.0443810469
1.84000015258789 102.04666503889
1.88000011444092 102.047492395032
1.92000007629395 102.046190770439
1.96000003814697 102.043525903541
2 102.039479665938
2.04000020027161 102.034052242092
2.08000016212463 102.027257814753
2.12000012397766 102.018745044625
2.16000008583069 102.009337300196
2.20000004768372 101.998655608808
2.24000000953674 101.985919152105
2.28000020980835 101.971142302881
2.32000017166138 101.955478324773
2.3600001335144 101.936062535231
2.40000009536743 101.909116730549
2.44000005722046 101.891533257563
2.48000001907349 101.871982693068
2.51999998092651 101.850689447015
2.56000018119812 101.827798227201
2.60000014305115 101.801651186686
2.64000010490417 101.773416958882
2.6800000667572 101.744575313137
2.72000002861023 101.714197669936
2.75999999046326 101.682207923448
2.80000019073486 101.649584403147
2.84000015258789 101.614986458969
2.88000011444092 101.583862175153
2.92000007629395 101.547840617267
2.96000003814697 101.507149993773
3 101.464923174395
3.04000020027161 101.421887481755
3.08000016212463 101.378541212529
3.12000012397766 101.334831993266
3.16000008583069 101.288921905912
3.20000004768372 101.241839616412
3.24000000953674 101.193494494269
3.28000020980835 101.143474145806
3.32000017166138 101.097779504246
3.3600001335144 101.057485521361
3.40000009536743 101.002884391101
3.44000005722046 100.946411491978
3.48000001907349 100.888623304287
3.51999998092651 100.830221972261
3.56000018119812 100.770595516136
3.60000014305115 100.710319752151
3.64000010490417 100.648519359272
3.6800000667572 100.585147828171
3.72000002861023 100.51971097126
3.75999999046326 100.452440118411
3.80000019073486 100.384786781866
3.84000015258789 100.320395143145
3.88000011444092 100.247478852722
3.92000007629395 100.17519098234
3.96000003814697 100.101481527236
4 100.026666396513
4.04000020027161 99.950694439651
4.08000016212463 99.8733430170946
4.12000012397766 99.7951874182703
4.16000008583069 99.7149741684241
4.20000004768372 99.6332550944455
4.24000000953674 99.5507631001514
4.28000020980835 99.4668732589905
4.32000017166138 99.3821839403125
4.3600001335144 99.2953405750387
4.40000009536743 99.2063115574046
4.44000005722046 99.1163581801867
4.48000001907349 99.0247170600869
4.51999998092651 98.9314682965128
4.56000018119812 98.8376659727511
4.60000014305115 98.7431052516281
4.64000010490417 98.647009366826
4.6800000667572 98.5504308456814
4.72000002861023 98.4519766159655
4.75999999046326 98.3527347877912
4.80000019073486 98.2518569398588
4.84000015258789 98.1494159738837
4.88000011444092 98.0450890777364
4.92000007629395 97.9403433066317
4.96000003814697 97.8345927441907
5 97.717402050284
5.04000020027161 97.6082556560407
5.08000016212463 97.4986318688928
5.12000012397766 97.3870094696572
5.16000008583069 97.2746308166943
5.20000004768372 97.1610885992189
5.24000000953674 97.0461313492432
5.28000020980835 96.9301000968196
5.32000017166138 96.8118969700399
5.3600001335144 96.6919414709391
5.40000009536743 96.5708736394519
5.44000005722046 96.4487900451215
5.48000001907349 96.3254685197967
5.51999998092651 96.2012153358897
5.56000018119812 96.0762507496213
5.60000014305115 95.9500476973486
5.64000010490417 95.8221755093968
5.6800000667572 95.6939564506665
5.72000002861023 95.5645201887134
5.75999999046326 95.4329666181412
5.80000019073486 95.3001604243693
5.84000015258789 95.1654702918253
5.88000011444092 95.0289222761752
5.92000007629395 94.8910501482853
5.96000003814697 94.7523857089783
6 94.6127048628777
6.04000020027161 94.4716812654369
6.08000016212463 94.3299192366003
6.12000012397766 94.1880417136584
6.16000008583069 94.0445106193647
6.20000004768372 93.8999096838418
6.24000000953674 93.7541062121897
6.28000020980835 93.6069221272582
6.32000017166138 93.4582837621784
6.3600001335144 93.3078527423384
6.40000009536743 93.1561946346525
6.44000005722046 93.0037151282287
6.48000001907349 92.8502430071796
6.51999998092651 92.6954435381169
6.56000018119812 92.5392398938169
6.60000014305115 92.3816342915312
6.64000010490417 92.2229757259688
6.6800000667572 92.0635606458248
6.72000002861023 91.9034085441897
6.75999999046326 91.7425144331308
6.80000019073486 91.5799079518556
6.84000015258789 91.4161285351568
6.88000011444092 91.2508014745571
6.92000007629395 91.0834981580226
6.96000003814697 90.9146793316134
7 90.7452076992629
7.04000020027161 90.5746560652683
7.08000016212463 90.4026669216031
7.12000012397766 90.2298164431686
7.16000008583069 90.0564471836372
7.20000004768372 89.8815189298149
7.24000000953674 89.705462727994
7.28000020980835 89.5285091828133
7.32000017166138 89.3501681226308
7.3600001335144 89.1709799281471
7.40000009536743 88.989699075209
7.44000005722046 88.8064419076188
7.48000001907349 88.622470498655
7.51999998092651 88.4374692915834
7.56000018119812 88.251221015129
7.60000014305115 88.0641743190748
7.64000010490417 87.8765914336536
7.6800000667572 87.688003159597
7.72000002861023 87.4977572326571
7.75999999046326 87.3067446896694
7.80000019073486 87.1149455768267
7.84000015258789 86.9216942904709
7.88000011444092 86.7268355173789
7.92000007629395 86.5304957870994
7.96000003814697 86.3327945979764
8 86.1336612265625
8.04000020027161 85.9335711445442
8.08000016212463 85.7329440909613
8.12000012397766 85.5314228909331
8.16000008583069 85.3282581009228
8.20000004768372 85.1239921698082
8.24000000953674 84.9198888653146
8.28000020980835 84.7146857756439
8.32000017166138 84.5082790681873
8.3600001335144 84.3028765433739
8.40000009536743 84.0991606789504
8.44000005722046 83.8961530200729
8.48000001907349 83.6950185302418
8.51999998092651 83.4962802985132
8.56000018119812 83.2975518952508
8.60000014305115 83.098905324023
8.64000010490417 82.9012753914575
8.6800000667572 82.7043213768006
8.72000002861023 82.5075169093375
8.75999999046326 82.3125709171429
8.80000019073486 82.1188022875684
8.84000015258789 81.9271504063619
8.88000011444092 81.7374803191095
8.92000007629395 81.5464787071878
8.96000003814697 81.3409456003816
9 81.1612145760248
9.04000020027161 80.9834388807422
9.08000016212463 80.8071908803188
9.12000012397766 80.6318435882067
9.16000008583069 80.4573231624023
9.20000004768372 80.2844263820603
9.24000000953674 80.111818829007
9.28000020980835 79.9405930539112
9.32000017166138 79.7707260404285
9.3600001335144 79.6022106134816
9.40000009536743 79.4335290944813
9.44000005722046 79.2762151472035
9.48000001907349 79.1112427750271
9.51999998092651 78.943618714578
9.56000018119812 78.7768507143628
9.60000014305115 78.6107847821658
9.64000010490417 78.4455010029867
9.6800000667572 78.2807005433926
9.72000002861023 78.115506524454
9.75999999046326 77.9512765798501
9.80000019073486 77.7869901798834
9.84000015258789 77.6226243553022
9.88000011444092 77.4592511865325
9.92000007629395 77.2968628358098
9.96000003814697 77.1352009782151
10 76.9744147855574
10.0400002002716 76.8144860810798
10.0800001621246 76.6543216333883
10.1200001239777 76.5077079895385
10.1600000858307 76.3488649113857
10.2000000476837 76.18947483516
10.2400000095367 76.0303798628399
10.2800002098083 75.8716093818074
10.3200001716614 75.7127949072004
10.3600001335144 75.5542893313577
10.4000000953674 75.3951832455615
10.4400000572205 75.2372158086712
10.4800000190735 75.0795757644566
10.5199999809265 74.9223155463665
10.5600001811981 74.7648132029937
10.6000001430511 74.6077911152214
10.6400001049042 74.4477513542248
10.6800000667572 74.2915626057954
10.7200000286102 74.1355438423443
10.7599999904633 73.9801253219129
10.8000001907349 73.823994981082
10.8400001525879 73.6682645151115
10.8800001144409 73.5129208039836
10.9200000762939 73.3576417647607
10.960000038147 73.202929888692
11 73.0481572151907
11.0400002002716 72.8940880218246
11.0800001621246 72.7408464375047
11.1200001239777 72.5869587887586
11.1600000858307 72.4321769877472
11.2000000476837 72.2776578458731
11.2400000095367 72.1232083661357
11.2800002098083 71.9690830841409
11.3200001716614 71.8153201639372
11.3600001335144 71.6615515864351
11.4000000953674 71.5081057483543
11.4400000572205 71.354702894441
11.4800000190735 71.2007368542867
11.5199999809265 71.0468006114144
11.5600001811981 70.8932914813136
11.6000001430511 70.7404882260844
11.6400001049042 70.5879813362569
11.6800000667572 70.4351017591217
11.7200000286102 70.2818067638911
11.7599999904633 70.1286980746389
11.8000001907349 69.9761166552294
11.8400001525879 69.8231873655727
11.8800001144409 69.6702712183523
11.9200000762939 69.5179826019143
11.960000038147 69.365064028374
12 69.2112867654504
12.0400002002716 69.0572446822591
12.0800001621246 68.9029657257043
12.1200001239777 68.7481064561243
12.1600000858307 68.5924261540286
12.2000000476837 68.4362901577738
12.2400000095367 68.2810399433327
12.2800002098083 68.1258743941194
12.3200001716614 67.9714420653347
12.3600001335144 67.8155036959318
12.4000000953674 67.6605814965305
12.4400000572205 67.5058002994636
12.4800000190735 67.3515307252465
12.5199999809265 67.1975048776493
12.5600001811981 67.0435096645865
12.6000001430511 66.8888138334459
12.6400001049042 66.7333994231666
12.6800000667572 66.5777597124291
12.7200000286102 66.4218781809627
12.7599999904633 66.2654992446979
12.8000001907349 66.1094914094813
12.8400001525879 65.9539124669883
12.8800001144409 65.7979407905959
12.9200000762939 65.6412822948327
12.960000038147 65.4851791260443
13 65.3295228182616
13.0400002002716 65.1733966381897
13.0800001621246 65.0181669106104
13.1200001239777 64.8636319622695
13.1600000858307 64.7090828087404
13.2000000476837 64.5541453454782
13.2400000095367 64.3989154496912
13.2800002098083 64.2436707280104
13.3200001716614 64.0883749838987
13.3600001335144 63.9322583409428
13.4000000953674 63.7761475246937
13.4400000572205 63.6192935353966
13.4800000190735 63.4620249582441
13.5199999809265 63.3046554887386
13.5600001811981 63.1466607588448
13.6000001430511 62.9892958745526
13.6400001049042 62.8322530036825
13.6800000667572 62.6752072157405
13.7200000286102 62.5177905331673
13.7599999904633 62.3601198963861
13.8000001907349 62.2027852075524
13.8400001525879 62.0455947859602
13.8800001144409 61.8877562595208
13.9200000762939 61.729884853707
13.960000038147 61.5727759976336
14 61.4150052804567
14.0400002002716 61.2571397285544
14.0800001621246 61.0992793260796
14.1200001239777 60.9411479851152
14.1600000858307 60.7840246211574
14.2000000476837 60.6262087558196
14.2400000095367 60.4685730460752
14.2800002098083 60.3114901395628
14.3200001716614 60.1540972748559
14.3600001335144 59.9961041533824
14.4000000953674 59.8378721859086
14.4400000572205 59.6800288809867
14.4800000190735 59.5225227092641
14.5199999809265 59.3654275396787
14.5600001811981 59.2080042050245
14.6000001430511 59.0502348920216
14.6400001049042 58.8926696121296
14.6800000667572 58.7356518129402
14.7200000286102 58.5785679158959
14.7599999904633 58.4211323402833
14.8000001907349 58.2636333404352
14.8400001525879 58.1061449368038
14.8800001144409 57.9488288674554
14.9200000762939 57.7918089239485
14.960000038147 57.6347275010522
15 57.4773470840657
15.0400002002716 57.3195471916877
15.0800001621246 57.1621018832065
15.1200001239777 57.004722407619
15.1600000858307 56.8470292178756
15.2000000476837 56.6897481587284
15.2400000095367 56.5329115037562
15.2800002098083 56.3763729267182
15.3200001716614 56.2194413248137
15.3600001335144 56.0621310635828
15.4000000953674 55.9053281749818
15.4400000572205 55.7484374195664
15.4800000190735 55.5912148346689
15.5199999809265 55.4335795052116
15.5600001811981 55.2767308966675
15.6000001430511 55.1202215562291
15.6400001049042 54.9631376879994
15.6800000667572 54.8058658470374
15.7200000286102 54.6486175817357
15.7599999904633 54.4916688866989
15.8000001907349 54.3348273846328
15.8400001525879 54.1781925174416
15.8800001144409 54.0215532697723
15.9200000762939 53.8645283368053
15.960000038147 53.7073452924015
16 53.5506587706211
16.0400002002716 53.3945779251125
16.0800001621246 53.2377651999576
16.1200001239777 53.0821161765203
16.1600000858307 52.9261648536891
16.2000000476837 52.7704641841648
16.2400000095367 52.6154037578739
16.2800002098083 52.4605870670757
16.3200001716614 52.3062258060836
16.3600001335144 52.1520533777766
16.4000000953674 51.9977982253957
16.4400000572205 51.8436445541677
16.4800000190735 51.6896252070198
16.5199999809265 51.5355181585452
16.5600001811981 51.3808141778506
16.6000001430511 51.2256426458942
16.6400001049042 51.0718430881181
16.6800000667572 50.9171879543837
16.7200000286102 50.7620637554124
16.7599999904633 50.6074441353987
16.8000001907349 50.4542410423314
16.8400001525879 50.3008424082106
16.8800001144409 50.1463480567084
16.9200000762939 49.991889410051
16.960000038147 49.8364584874565
17 49.680755345217
17.0400002002716 49.5241363723502
17.0800001621246 49.3658149151723
17.1200001239777 49.2075458681001
17.1600000858307 49.0502429314958
17.2000000476837 48.8908844969691
17.2400000095367 48.7297844372532
17.2800002098083 48.5680727321286
17.3200001716614 48.4055478419073
17.3600001335144 48.2417579380372
17.4000000953674 48.0760834446151
17.4400000572205 47.9099712774263
17.4800000190735 47.7445262442941
17.5199999809265 47.5758185565421
17.5600001811981 47.405764809612
17.6000001430511 47.2338839207132
17.6400001049042 47.0615859915252
17.6800000667572 46.8858760394434
17.7200000286102 46.7114654614059
17.7599999904633 46.5357732718061
17.8000001907349 46.3577969681155
17.8400001525879 46.1770224997191
17.8800001144409 45.9939779319207
17.9200000762939 45.8088067516682
17.960000038147 45.6247661465719
18 45.4389638195626
18.0400002002716 45.2532075783406
18.0800001621246 45.0652248616047
18.1200001239777 44.8788711213539
18.1600000858307 44.6866347105773
18.2000000476837 44.4922387757162
18.2400000095367 44.2971713221496
18.2800002098083 44.0997178207817
18.3200001716614 43.9005299901639
18.3600001335144 43.7001070461897
18.4000000953674 43.5000557146376
18.4400000572205 43.3001815317167
18.4800000190735 43.0989442319801
18.5199999809265 42.8980331793653
18.5600001811981 42.6947782746714
18.6000001430511 42.4921151442457
18.6400001049042 42.2890667176372
18.6800000667572 42.0882581862528
18.7200000286102 41.885270231771
18.7599999904633 41.6800936391941
18.8000001907349 41.4753134734698
18.8400001525879 41.2683281656195
18.8800001144409 41.0596030721356
18.9200000762939 40.8491342421977
18.960000038147 40.63333577016
19 40.4169038567798
19.0400002002716 40.2018712976426
19.0800001621246 39.9839735906491
19.1200001239777 39.7655748841976
19.1600000858307 39.5450075121882
19.2000000476837 39.3265391825446
19.2400000095367 39.1038292206461
19.2800002098083 38.8819368553974
19.3200001716614 38.6587068100507
19.3600001335144 38.434416495177
19.4000000953674 38.2102950502687
19.4400000572205 37.9860085354236
19.4800000190735 37.7621601464763
19.5199999809265 37.5372004794173
19.5600001811981 37.3115851677587
19.6000001430511 37.0854205358464
19.6400001049042 36.8565161641218
19.6800000667572 36.628258094558
19.7200000286102 36.3991636975544
19.7599999904633 36.1728613590257
19.8000001907349 35.9450418932745
19.8400001525879 35.7174673439664
19.8800001144409 35.4891479261291
19.9200000762939 35.2607498453736
19.960000038147 35.0302667526709
20 34.797891505878
20.0400002002716 34.5647762542711
20.0800001621246 34.3326455584301
20.1200001239777 34.1008253248913
20.1600000858307 33.8696069970241
20.2000000476837 33.6387757384778
20.2400000095367 33.4063957793103
20.2800002098083 33.1723540265775
20.3200001716614 32.9385698180619
20.3600001335144 32.7046653930613
20.4000000953674 32.4691006619885
20.4400000572205 32.2351494695508
20.4800000190735 32.0012811365015
20.5199999809265 31.7682605112782
20.5600001811981 31.5340312545496
20.6000001430511 31.2998609444236
20.6400001049042 31.0653043568497
20.6800000667572 30.8293653942445
20.7200000286102 30.5936716234625
20.7599999904633 30.3586078081221
20.8000001907349 30.1227285543418
20.8400001525879 29.8861403761844
20.8800001144409 29.6496261817034
20.9200000762939 29.413439977471
20.960000038147 29.1762909206114
21 28.9393289275502
21.0400002002716 28.7022825813321
21.0800001621246 28.4643979863655
21.1200001239777 28.2261197332291
21.1600000858307 27.987112663702
21.2000000476837 27.7469072457258
21.2400000095367 27.5063743349271
21.2800002098083 27.2657489787819
21.3200001716614 27.0242966352622
21.3600001335144 26.7832853552066
21.4000000953674 26.5416657448216
21.4400000572205 26.300141845967
21.4800000190735 26.0578793760505
21.5199999809265 25.817045363988
21.5600001811981 25.5762341878491
21.6000001430511 25.3345923240945
21.6400001049042 25.0926570579492
21.6800000667572 24.8501573432208
21.7200000286102 24.6069673409354
21.7599999904633 24.3636732593147
21.8000001907349 24.1193555215649
21.8400001525879 23.8746458946486
21.8800001144409 23.6298754814919
21.9200000762939 23.3849374971739
21.960000038147 23.1399097407925
22 22.8949193445956
22.0400002002716 22.6491203505718
22.0800001621246 22.4038365642698
22.1200001239777 22.1587283130219
22.1600000858307 21.9133153150657
22.2000000476837 21.6683673328099
22.2400000095367 21.4222541488439
22.2800002098083 21.1757613280753
22.3200001716614 20.9286143751897
22.3600001335144 20.6809377793
22.4000000953674 20.4331468412434
22.4400000572205 20.1854313947352
22.4800000190735 19.9381302143106
22.5199999809265 19.6897192409059
22.5600001811981 19.441552433052
22.6000001430511 19.1935446112685
22.6400001049042 18.9452564227879
22.6800000667572 18.6961902983461
22.7200000286102 18.4474586984681
22.7599999904633 18.1986775220064
22.8000001907349 17.949887873525
22.8400001525879 17.70098297255
22.8800001144409 17.4517825332636
22.9200000762939 17.202289534821
22.960000038147 16.9535230777206
23 16.7039720263089
23.0400002002716 16.4545381575863
23.0800001621246 16.2050122028874
23.1200001239777 15.9548622987454
23.1600000858307 15.7048234752558
23.2000000476837 15.4549346808338
23.2400000095367 15.2051046483577
23.2800002098083 14.955120579584
23.3200001716614 14.7050584746453
23.3600001335144 14.4552137385411
23.4000000953674 14.2048617999041
23.4400000572205 13.9540893575258
23.4800000190735 13.7039293383133
23.5199999809265 13.4516092889239
23.5600001811981 13.1993863861105
23.6000001430511 12.9465123234725
23.6400001049042 12.69350372124
23.6800000667572 12.4405551971526
23.7200000286102 12.1872920285993
23.7599999904633 11.9334940906324
23.8000001907349 11.6799448338252
23.8400001525879 11.426291113364
23.8800001144409 11.1731234817774
23.9200000762939 10.9198860101382
23.960000038147 10.6668188715893
24 10.4146965541029
24.0400002002716 10.1636406218949
24.0800001621246 9.91344282755913
24.1200001239777 9.6631160583321
24.1600000858307 9.41340511002221
24.2000000476837 9.16296972336683
24.2400000095367 8.91224697782263
24.2800002098083 8.6613473053971
24.3200001716614 8.41084024104772
24.3600001335144 8.16044470829353
24.4000000953674 7.90966275926215
24.4400000572205 7.65870030456552
24.4800000190735 7.40827563899264
24.5199999809265 7.15797719064307
24.5600001811981 6.90801385414846
24.6000001430511 6.65765349164203
24.6400001049042 6.40764621441656
24.6800000667572 6.15797722523948
24.7200000286102 5.90872236761659
24.7599999904633 5.65970831361437
24.8000001907349 5.41036750494398
24.8400001525879 5.16131861787894
24.8800001144409 4.91192559128296
24.9200000762939 4.66250596482899
24.960000038147 4.41329970590703
25 4.1636741208305
25.0400002002716 3.91396756286075
25.0800001621246 3.6644408763731
25.1200001239777 3.41046217096077
25.1600000858307 3.16057783074613
25.2000000476837 2.91072241549958
25.2400000095367 2.66037380271954
25.2800002098083 2.4103527264704
25.3200001716614 2.15995148349976
25.3600001335144 1.91001296444877
25.4000000953674 1.65990240585027
25.4400000572205 1.41032819933659
25.4800000190735 1.16061433611834
25.5199999809265 0.910811568931973
25.5600001811981 0.66148801482618
25.6000001430511 0.411402739427066
25.6400001049042 0.160781639393686
25.6800000667572 -0.0898112608159694
25.7200000286102 -0.340664168378237
25.7599999904633 -0.59153821833448
25.8000001907349 -0.842725440563792
25.8400001525879 -1.09356091729119
25.8800001144409 -1.34471368372387
25.9200000762939 -1.59587085991672
25.960000038147 -1.84697025984159
26 -2.09830544098833
26.0400002002716 -2.34945707323683
26.0800001621246 -2.60022234182376
26.1200001239777 -2.85054307501551
26.1600000858307 -3.10067257407531
26.2000000476837 -3.35152228025431
26.2400000095367 -3.60290706315831
26.2800002098083 -3.85485035732425
26.3200001716614 -4.10689346841187
26.3600001335144 -4.35916383306359
26.4000000953674 -4.61140687276691
26.4400000572205 -4.86368395901019
26.4800000190735 -5.11621882441978
26.5199999809265 -5.36867396369211
26.5600001811981 -5.62075975996893
26.6000001430511 -5.87264712197534
26.6400001049042 -6.12431815446832
26.6800000667572 -6.37618390975538
26.7200000286102 -6.62801394705092
26.7599999904633 -6.87935049609781
26.8000001907349 -7.13063141663258
26.8400001525879 -7.38185382191955
26.8800001144409 -7.63310604086204
26.9200000762939 -7.88434526008133
26.960000038147 -8.13629612969886
27 -8.38799030963377
27.0400002002716 -8.63961120406166
27.0800001621246 -8.89193691961668
27.1200001239777 -9.14404363468748
27.1600000858307 -9.39566893000909
27.2000000476837 -9.64729622211332
27.2400000095367 -9.89908071316457
27.2800002098083 -10.1504419896779
27.3200001716614 -10.4020868747503
27.3600001335144 -10.6526775786703
27.4000000953674 -10.9031656901268
27.4400000572205 -11.1530968028841
27.4800000190735 -11.4031116811071
27.5199999809265 -11.6531185646257
27.5600001811981 -11.903024547133
27.6000001430511 -12.1531702473933
27.6400001049042 -12.4035431429459
27.6800000667572 -12.6536518891218
27.7200000286102 -12.9037818024708
27.7599999904633 -13.1537432814631
27.8000001907349 -13.4041263577938
27.8400001525879 -13.6546924414542
27.8800001144409 -13.9051902490042
27.9200000762939 -14.1570200899525
27.960000038147 -14.4079210852829
28 -14.6581620410596
28.0400002002716 -14.9077834504288
28.0800001621246 -15.1572916172504
28.1200001239777 -15.4068213184548
28.1600000858307 -15.6562591984014
28.2000000476837 -15.9062961191131
28.2400000095367 -16.1564870935656
28.2800002098083 -16.4064019286268
28.3200001716614 -16.6567447449977
28.3600001335144 -16.9068841790189
28.4000000953674 -17.1573678577616
28.4400000572205 -17.4081428097771
28.4800000190735 -17.6590326828916
28.5199999809265 -17.9099635948383
28.5600001811981 -18.1607080649002
28.6000001430511 -18.4112955939156
28.6400001049042 -18.6619079006554
28.6800000667572 -18.9127360815621
28.7200000286102 -19.1634117427168
28.7599999904633 -19.4134454544847
28.8000001907349 -19.6638689645441
28.8400001525879 -19.9145966654544
28.8800001144409 -20.1643008876558
28.9200000762939 -20.4148265806106
28.960000038147 -20.6651612582664
29 -20.9161327302933
29.0400002002716 -21.1678452767611
29.0800001621246 -21.4196657600986
29.1200001239777 -21.6710868265815
29.1600000858307 -21.9227981843048
29.2000000476837 -22.174428535416
29.2400000095367 -22.4255641885559
29.2800002098083 -22.6764721829951
29.3200001716614 -22.9279361742118
29.3600001335144 -23.1792142937129
29.4000000953674 -23.4304529264594
29.4400000572205 -23.6816277972663
29.4800000190735 -23.9323202454243
29.5199999809265 -24.1829121901347
29.5600001811981 -24.4334753146694
29.6000001430511 -24.6840727549135
29.6400001049042 -24.9349498332542
29.6800000667572 -25.1852717997397
29.7200000286102 -25.4353035832072
29.7599999904633 -25.6851541800636
29.8000001907349 -25.9356122078494
29.8400001525879 -26.1861004903034
29.8800001144409 -26.435725346515
29.9200000762939 -26.6862659981062
29.960000038147 -26.9355077142518
30 -27.1851403836143
30.0400002002716 -27.4343687275215
30.0800001621246 -27.6837120231568
30.1200001239777 -27.9331002692602
30.1600000858307 -28.1830084620029
30.2000000476837 -28.4326941087759
30.2400000095367 -28.6825677363671
30.2800002098083 -28.9311672565118
30.3200001716614 -29.1794634032031
30.3600001335144 -29.4269262364131
30.4000000953674 -29.6757638811973
30.4400000572205 -29.9236873212456
30.4800000190735 -30.1731422541724
30.5199999809265 -30.4215908067139
30.5600001811981 -30.6689437779959
30.6000001430511 -30.9169328361235
30.6400001049042 -31.1646489972985
30.6800000667572 -31.4120994050819
30.7200000286102 -31.6591438336593
30.7599999904633 -31.9068416640432
30.8000001907349 -32.155454239994
30.8400001525879 -32.4011850624117
30.8800001144409 -32.6480014672964
30.9200000762939 -32.895720033516
30.960000038147 -33.142910812059
31 -33.3904483715059
31.0400002002716 -33.6397445152825
31.0800001621246 -33.8885898949994
31.1200001239777 -34.135976004499
31.1600000858307 -34.3823958939774
31.2000000476837 -34.6288573277577
31.2400000095367 -34.873695470996
31.2800002098083 -35.1200161690857
31.3200001716614 -35.3694530481699
31.3600001335144 -35.6193213597913
31.4000000953674 -35.8693258612159
31.4400000572205 -36.1176556133975
31.4800000190735 -36.368372687031
31.5199999809265 -36.616107428335
31.5600001811981 -36.864436065903
31.6000001430511 -37.1223447582784
31.6400001049042 -37.3829906944785
31.6800000667572 -37.6323460354048
31.7200000286102 -37.8829340169007
31.7599999904633 -38.1334249245977
31.8000001907349 -38.3811259864001
31.8400001525879 -38.6255686624233
31.8800001144409 -38.8721771605604
31.9200000762939 -39.1184131717691
31.960000038147 -39.3657804213787
32 -39.6133783333926
32.0400002002716 -39.8653647030356
32.0800001621246 -40.1152547104709
32.1200001239777 -40.3468182635226
32.1600000858307 -40.5982964491093
32.2000000476837 -40.8508694061365
32.2400000095367 -41.1018683634634
32.2800002098083 -41.3481569348728
32.3200001716614 -41.5988502368218
32.3600001335144 -41.8497175864068
32.4000000953674 -42.1016204946384
32.4400000572205 -42.3563765135033
32.4800000190735 -42.6067553166379
32.5199999809265 -42.8605561609665
32.5600001811981 -43.1123823217141
32.6000001430511 -43.3644422338657
32.6400001049042 -43.6148984073859
32.6800000667572 -43.8705664539516
32.7200000286102 -44.124004625709
32.7599999904633 -44.3830507666645
32.8000001907349 -44.6429121899369
32.8400001525879 -44.891883672455
32.8800001144409 -45.1132104586108
32.9200000762939 -45.3658799832957
32.960000038147 -45.6183144355328
33 -45.8728416185282
33.0400002002716 -46.1244282063641
33.0800001621246 -46.3780250628728
33.1200001239777 -46.629640021988
33.1600000858307 -46.8811776996769
33.2000000476837 -47.1317261861055
33.2400000095367 -47.3838468187955
33.2800002098083 -47.6370728531147
33.3200001716614 -47.8895552130781
33.3600001335144 -48.1767597663419
33.4000000953674 -48.445251124502
33.4400000572205 -48.700369210031
33.4800000190735 -48.9560792606711
33.5199999809265 -49.2091274871409
33.5600001811981 -49.4630959422897
33.6000001430511 -49.7160740555733
33.6400001049042 -49.9723448197502
33.6800000667572 -50.2273517707698
33.7200000286102 -50.4833576371101
33.7599999904633 -50.7373973388056
33.8000001907349 -50.9909787714524
33.8400001525879 -51.2448373823363
33.8800001144409 -51.498799091827
33.9200000762939 -51.7519371681672
33.960000038147 -52.0052327324858
34 -52.2602762255826
34.0400002002716 -52.5148682300343
34.0800001621246 -52.7700201273581
34.1200001239777 -53.0245602068897
34.1600000858307 -53.2779622756575
};
\node at (axis cs:6,55)[
  anchor=north,
  text=black,
  rotate=0.0
]{ Pickup truck};
\node at (axis cs:6,0)[
  anchor=south,
  text=black,
  rotate=0.0
]{ Station wagon};
\end{axis}

\end{tikzpicture}}\\
%	\subfloat[Lateral distance of other vehicles with respect to ego vehicle.]{% This file was created by matplotlib2tikz v0.6.14.
\begin{tikzpicture}

\definecolor{color3}{rgb}{0.549019607843137,0.337254901960784,0.294117647058824}
\definecolor{color0}{rgb}{0.172549019607843,0.627450980392157,0.172549019607843}
\definecolor{color2}{rgb}{0.580392156862745,0.403921568627451,0.741176470588235}
\definecolor{color1}{rgb}{0.83921568627451,0.152941176470588,0.156862745098039}

\begin{axis}[
xlabel={$t$ [s]},
ylabel={Relative lateral distance [m]},
xmin=0, xmax=34.1600000858307,
ymin=-3.6944648464502, ymax=3.79876173773052,
width=\figurewidth,
height=\figureheight,
tick align=outside,
tick pos=left,
xmajorgrids,
x grid style={lightgray!92.026143790849673!black},
ymajorgrids,
y grid style={lightgray!92.026143790849673!black},
clip marker paths
]
\addplot [semithick, color0, mark=*, mark size=1, mark options={solid}, only marks, forget plot]
table {%
0 -0.0417787396782297
0.0400002002716064 -0.0434833324633892
0.0800001621246338 -0.00793896066897679
0.120000123977661 -0.0290788596134219
0.160000085830688 -0.00511650671826534
0.200000047683716 0.026785123248344
0.240000009536743 0.0205590435513228
0.28000020980835 -0.0226561519444399
0.320000171661377 -0.00188009407999191
0.360000133514404 -0.00377719282237902
0.400000095367432 -0.0155240147251251
0.440000057220459 -0.0306548872097677
0.480000019073486 -0.0606875543387061
0.519999980926514 -0.0349109763945282
0.56000018119812 -0.0274273183054372
0.600000143051147 -0.0255048803860961
0.640000104904175 -0.0142116114614845
0.680000066757202 -0.0144869500468184
0.720000028610229 -0.00634319719655396
0.759999990463257 0.00264264047317925
0.800000190734863 0.0105393193251953
0.840000152587891 0.0138097292343392
0.880000114440918 0.0291535571020424
0.920000076293945 0.0366716656258401
0.960000038146973 0.0297297342137315
1 0.0217351281834865
1.04000020027161 0.0103010571788247
1.08000016212463 -0.012837636547751
1.12000012397766 -0.000666284051467028
1.16000008583069 -0.0196001067388685
1.20000004768372 -0.00508295675099915
1.24000000953674 -0.0302094671738396
1.28000020980835 -0.0652997387717573
1.32000017166138 -0.0518677885327045
1.3600001335144 -0.062191293287375
1.40000009536743 -0.0583230235330692
1.44000005722046 -0.0637485272072151
1.48000001907349 -0.0848910782785489
1.51999998092651 -0.0887539614874596
1.56000018119812 -0.0915527800488931
1.60000014305115 -0.094033839429671
1.64000010490417 -0.0986166093511816
1.6800000667572 -0.101533324659034
1.72000002861023 -0.097418266876103
1.75999999046326 -0.0951203112193729
1.80000019073486 -0.0768142142537029
1.84000015258789 -0.0758488769996015
1.88000011444092 -0.0657215850608496
1.92000007629395 -0.0596173475464898
1.96000003814697 -0.0774694105485675
2 -0.0819576342397531
2.04000020027161 -0.0737450374304574
2.08000016212463 -0.0632697980881775
2.12000012397766 -0.065146546111649
2.16000008583069 -0.0379296996072994
2.20000004768372 -0.0298124907636027
2.24000000953674 -0.0104287031121793
2.28000020980835 -0.000783934746713261
2.32000017166138 -0.0146727622936489
2.3600001335144 -0.0116230187023605
2.40000009536743 -0.00800980925905722
2.44000005722046 -0.0569500180955829
2.48000001907349 -0.0194943770969402
2.51999998092651 0.0151885988311344
2.56000018119812 0.0656892286827005
2.60000014305115 0.072081029845608
2.64000010490417 0.0845924477338064
2.6800000667572 0.105742618639308
2.72000002861023 0.119406130685862
2.75999999046326 0.12824628236362
2.80000019073486 0.139938514157885
2.84000015258789 0.148019094859861
2.88000011444092 0.25583690341537
2.92000007629395 0.306554447711603
2.96000003814697 0.278564315163056
3 0.244764365743769
3.04000020027161 0.234652405155853
3.08000016212463 0.237902782307596
3.12000012397766 0.235483204363214
3.16000008583069 0.239052413177167
3.20000004768372 0.232617816658026
3.24000000953674 0.260339993436241
3.28000020980835 0.258706864517433
3.32000017166138 0.3574681748221
3.3600001335144 0.901479815125867
3.40000009536743 0.918641901317467
3.44000005722046 0.952841176995706
3.48000001907349 0.986183120187692
3.51999998092651 0.988627684322505
3.56000018119812 1.00186064011514
3.60000014305115 1.01180667626192
3.64000010490417 1.01591165870319
3.6800000667572 1.04432666445691
3.72000002861023 1.06150147302882
3.75999999046326 1.09942249772943
3.80000019073486 1.11890528969187
3.84000015258789 0.636547295423912
3.88000011444092 0.405911449278643
3.92000007629395 0.389694720289353
3.96000003814697 0.41780990901604
4 0.414559403397376
4.04000020027161 0.420269918803582
4.08000016212463 0.449963792167126
4.12000012397766 0.418271949134033
4.16000008583069 0.429857636237069
4.20000004768372 0.407949681378507
4.24000000953674 0.415058137103386
4.28000020980835 0.431460320618351
4.32000017166138 0.413362124001608
4.3600001335144 0.415399411612331
4.40000009536743 0.395302178564078
4.44000005722046 0.415159251091263
4.48000001907349 0.40905699661753
4.51999998092651 0.392595507550862
4.56000018119812 0.253553725662647
4.60000014305115 0.250807804710817
4.64000010490417 0.263448388963936
4.6800000667572 0.283164182167046
4.72000002861023 0.260610309470746
4.75999999046326 0.264162076704724
4.80000019073486 0.249604925137301
4.84000015258789 0.225354232271675
4.88000011444092 0.22634417982975
4.92000007629395 0.198150775817807
4.96000003814697 0.197521672397681
5 0.182313265544689
5.04000020027161 0.194446681398852
5.08000016212463 0.221427069799978
5.12000012397766 0.189127232670235
5.16000008583069 0.209346408108924
5.20000004768372 0.163559505577604
5.24000000953674 0.139405583163256
5.28000020980835 0.16605103007943
5.32000017166138 0.119682918201199
5.3600001335144 0.131886410955644
5.40000009536743 0.112349399868415
5.44000005722046 0.11047751927303
5.48000001907349 0.144443713438279
5.51999998092651 0.117395918805469
5.56000018119812 0.1614418734515
5.60000014305115 0.12118195280291
5.64000010490417 0.168949677401619
5.6800000667572 0.213932440600319
5.72000002861023 0.18063949003915
5.75999999046326 0.220967184431904
5.80000019073486 0.191980929452678
5.84000015258789 0.205817819563089
5.88000011444092 0.234895748189027
5.92000007629395 0.1917900253487
5.96000003814697 0.201873463051521
6 0.191549293366018
6.04000020027161 0.165585265412575
6.08000016212463 0.214685869194678
6.12000012397766 0.329367520623184
6.16000008583069 0.337115141512145
6.20000004768372 0.325935692683958
6.24000000953674 0.372339743198021
6.28000020980835 0.388521744711257
6.32000017166138 0.393964485258088
6.3600001335144 0.456375310335519
6.40000009536743 0.469509591666124
6.44000005722046 0.45917285117698
6.48000001907349 0.527237683952156
6.51999998092651 0.502872396074831
6.56000018119812 0.524079771777374
6.60000014305115 0.523329068632831
6.64000010490417 0.527918935909131
6.6800000667572 0.545962045217118
6.72000002861023 0.502727470836427
6.75999999046326 0.482184477662969
6.80000019073486 0.477095554564596
6.84000015258789 0.444204777600845
6.88000011444092 0.44850776699175
6.92000007629395 0.416270276152863
6.96000003814697 0.420725740179746
7 0.414874046023105
7.04000020027161 0.38603477883713
7.08000016212463 0.399460643621401
7.12000012397766 0.392432457259056
7.16000008583069 0.383393069805033
7.20000004768372 0.37759507821646
7.24000000953674 0.360152991313975
7.28000020980835 0.369336407770393
7.32000017166138 0.318961385583633
7.3600001335144 0.338294565931529
7.40000009536743 0.351662758020797
7.44000005722046 0.334317194704018
7.48000001907349 0.351272703445039
7.51999998092651 0.356151038601924
7.56000018119812 0.370647339093179
7.60000014305115 0.361366313709705
7.64000010490417 0.358361932543915
7.6800000667572 0.377377270780715
7.72000002861023 0.367017494780768
7.75999999046326 0.385762241170437
7.80000019073486 0.377993316734781
7.84000015258789 0.385367301479262
7.88000011444092 0.42649928924518
7.92000007629395 0.376952841485367
7.96000003814697 0.32511495293492
8 0.313859984558447
8.04000020027161 0.303630483685739
8.08000016212463 0.354531774151929
8.12000012397766 0.298820160892349
8.16000008583069 0.284601621841473
8.20000004768372 0.27365181123016
8.24000000953674 0.255467651706338
8.28000020980835 0.298369025216015
8.32000017166138 0.225502205215161
8.3600001335144 0.23583690821237
8.40000009536743 0.224470184450044
8.44000005722046 0.220576961146684
8.48000001907349 0.284973198074928
8.51999998092651 0.210160351948411
8.56000018119812 0.232056477010676
8.60000014305115 0.22463754242515
8.64000010490417 0.246830359009954
8.6800000667572 0.320988924900878
8.72000002861023 0.235447281501163
8.75999999046326 0.242505953399173
8.80000019073486 0.308828695113704
8.84000015258789 0.226868064053963
8.88000011444092 0.298516583395571
8.92000007629395 0.112239341909362
8.96000003814697 0.233319506270085
9 0.222930825871416
9.04000020027161 0.173799730495064
9.08000016212463 0.133034145241127
9.12000012397766 0.271600448852642
9.16000008583069 0.240262057773767
9.20000004768372 0.272490445382458
9.24000000953674 0.290509172202269
9.28000020980835 0.203546656494682
9.32000017166138 0.240886270162535
9.3600001335144 0.292646349281046
9.40000009536743 0.24357226985176
9.44000005722046 0.0457886916948231
9.48000001907349 0.113493234988786
9.51999998092651 -0.000772053611664832
9.56000018119812 0.112654417957984
9.60000014305115 0.19868405246894
9.64000010490417 0.289832259579926
9.6800000667572 0.336780099632129
9.72000002861023 0.546307860143487
9.75999999046326 0.605127321445179
9.80000019073486 0.709152583080343
9.84000015258789 0.86787252442825
9.88000011444092 0.855295164046075
9.92000007629395 1.01077547057974
9.96000003814697 1.03499509359713
10 1.09273377718478
10.0400002002716 1.1245075426224
10.0800001621246 1.04513123852704
10.1200001239777 0.590629446494773
10.1600000858307 0.490265626919553
10.2000000476837 0.525338806170876
10.2400000095367 0.582509165031396
10.2800002098083 0.550542282891613
10.3200001716614 0.569727304410648
10.3600001335144 0.58622729699181
10.4000000953674 0.600210523025847
10.4400000572205 0.745514249594854
10.4800000190735 0.744749887615746
10.5199999809265 0.891990517277949
10.5600001811981 0.899691473428397
10.6000001430511 1.20455610658641
10.6400001049042 1.70099379394201
10.6800000667572 1.76319378604241
10.7200000286102 1.8848609444214
10.7599999904633 1.942461616096
10.8000001907349 2.00791692021982
10.8400001525879 2.08900244115099
10.8800001144409 2.15631064645281
10.9200000762939 2.24277276393423
10.960000038147 2.26034880712028
11 2.2847860362985
11.0400002002716 2.38502518267754
11.0800001621246 2.41149425484073
11.1200001239777 2.46287144341076
11.1600000858307 2.45554859483347
11.2000000476837 2.50325144504693
11.2400000095367 2.4748176006217
11.2800002098083 2.53893538470829
11.3200001716614 2.6149336120236
11.3600001335144 2.58903833083674
11.4000000953674 2.6709464333538
11.4400000572205 2.71302412942567
11.4800000190735 2.71737212973987
11.5199999809265 2.73644748425465
11.5600001811981 2.78274488624952
11.6000001430511 2.81908712925223
11.6400001049042 2.84631088708817
11.6800000667572 2.87858715226997
11.7200000286102 2.8526665423972
11.7599999904633 2.82163607458537
11.8000001907349 2.81946235112765
11.8400001525879 2.77045666703232
11.8800001144409 2.81113021465049
11.9200000762939 2.83063097523508
11.960000038147 2.8547743756347
12 2.85652138845991
12.0400002002716 2.83735527463517
12.0800001621246 2.84826293840178
12.1200001239777 2.94650899192099
12.1600000858307 2.96834716525122
12.2000000476837 2.9459678874127
12.2400000095367 2.95552809754005
12.2800002098083 2.98052639741588
12.3200001716614 3.03308143254575
12.3600001335144 3.03516685628485
12.4000000953674 3.03830755409455
12.4400000572205 3.05198260959917
12.4800000190735 3.0678785048125
12.5199999809265 3.07061290230589
12.5600001811981 3.08839716534139
12.6000001430511 3.0832046156668
12.6400001049042 3.09135883909449
12.6800000667572 3.10858451990355
12.7200000286102 3.03941106173655
12.7599999904633 3.0286950647477
12.8000001907349 3.0340884180457
12.8400001525879 2.99966137050916
12.8800001144409 3.00007675451334
12.9200000762939 2.92460981991743
12.960000038147 2.91652296508919
13 2.84387097629045
13.0400002002716 2.82922891089638
13.0800001621246 2.784352756705
13.1200001239777 2.7617945740745
13.1600000858307 2.75657701446277
13.2000000476837 2.63835897994763
13.2400000095367 2.61485943274389
13.2800002098083 2.59233925987106
13.3200001716614 2.54632233644069
13.3600001335144 2.53502060367273
13.4000000953674 2.53048062053379
13.4400000572205 2.4933923496161
13.4800000190735 2.49556359346183
13.5199999809265 2.49158551094374
13.5600001811981 2.52205181480324
13.6000001430511 2.55991862959851
13.6400001049042 2.55413321029144
13.6800000667572 2.54124768509059
13.7200000286102 2.60696714951397
13.7599999904633 2.60402189107454
13.8000001907349 2.66285799233406
13.8400001525879 2.71306910117497
13.8800001144409 2.71683005828622
13.9200000762939 2.72031786075228
13.960000038147 2.73824721379009
14 2.74174426664553
14.0400002002716 2.78187433127749
14.0800001621246 2.86799681689796
14.1200001239777 2.87925312794838
14.1600000858307 2.89610280642867
14.2000000476837 2.91435251281274
14.2400000095367 2.92266929287403
14.2800002098083 2.93546697106999
14.3200001716614 3.00334255019609
14.3600001335144 3.01768822302289
14.4000000953674 3.03556724485131
14.4400000572205 3.01297996432891
14.4800000190735 3.01002518731099
14.5199999809265 3.0251960717066
14.5600001811981 3.00817321327189
14.6000001430511 3.01049629972815
14.6400001049042 2.99215120632159
14.6800000667572 2.99225186649551
14.7200000286102 2.95555593367695
14.7599999904633 2.9528534374746
14.8000001907349 2.95507923768087
14.8400001525879 2.99711772569004
14.8800001144409 2.99898927695971
14.9200000762939 3.00566460827698
14.960000038147 3.01015581961001
15 3.01644627112053
15.2000000476837 3.3620217705862
15.2400000095367 3.36236510006473
15.2800002098083 3.35264191549625
15.3200001716614 3.45816052935867
15.3600001335144 3.28908788760309
15.4000000953674 3.38583134435336
15.4400000572205 3.25156133938931
15.4800000190735 3.23405095981175
15.5199999809265 3.32269457553571
15.5600001811981 3.31887655673157
15.6000001430511 3.40343196799003
15.6400001049042 3.40716664924312
15.6800000667572 3.40956343709971
15.7200000286102 3.31672020465732
15.7599999904633 3.24866368285574
15.8000001907349 3.32397850709351
15.8400001525879 3.32650928617088
15.8800001144409 3.20824210039003
15.9200000762939 3.18228634701947
15.960000038147 3.15800436307625
16 3.2586327824088
16.0400002002716 3.25299428130253
16.0800001621246 3.22557641523254
16.1200001239777 3.18870393876074
16.1600000858307 3.21108888850197
16.2000000476837 3.16667358794859
16.2400000095367 3.16149233014769
16.2800002098083 3.12760682312541
16.3200001716614 3.02746670675659
16.3600001335144 2.99932713581388
16.4000000953674 3.0794154043405
16.4400000572205 3.07811237480725
16.4800000190735 3.17230242186235
16.5199999809265 3.13040485931683
16.5600001811981 3.16553648913305
16.6000001430511 3.32523983633196
16.6400001049042 3.33552316814808
16.6800000667572 3.20596796864897
16.7200000286102 3.15341218549578
16.7599999904633 3.14067369762768
16.8000001907349 3.12112796439612
16.8400001525879 3.11060822932094
16.8800001144409 3.15898096982907
16.9200000762939 3.09511129442034
16.960000038147 3.23231214923136
17 3.19656782795671
17.0400002002716 3.23766812528487
17.0800001621246 3.20983969689052
17.1200001239777 3.25479426494558
17.1600000858307 3.18412109014656
17.2000000476837 3.18202625105208
17.2400000095367 3.16884716106373
17.2800002098083 3.20181181156275
17.3200001716614 3.19369118608081
17.3600001335144 3.20997527150776
17.4000000953674 3.24944315752324
17.4400000572205 3.12758829404063
17.4800000190735 3.10249432758436
17.5199999809265 3.13525716009609
17.5600001811981 3.02941396537664
17.6000001430511 3.04261737822625
17.6400001049042 3.05885295114235
17.6800000667572 3.08026320160039
17.7200000286102 2.98320854652992
};
\addplot [semithick, color2, mark=*, mark size=1, mark options={solid}, only marks, forget plot]
table {%
0.800000190734863 0.566310207694819
0.840000152587891 0.610843449352141
0.880000114440918 0.646629711214613
0.920000076293945 0.675662457956674
0.960000038146973 0.677383853894884
1 0.652233842837334
1.04000020027161 0.617705696642899
1.08000016212463 0.59707644187842
1.12000012397766 0.582773435789414
1.16000008583069 0.57319667237221
1.20000004768372 0.544899207724968
1.24000000953674 0.520278560671475
1.28000020980835 0.486819697340015
1.32000017166138 0.471235098019191
1.3600001335144 0.482251080844038
1.40000009536743 0.498766569171103
1.44000005722046 0.525970282627713
1.48000001907349 0.546432838878246
1.51999998092651 0.570399635333319
1.56000018119812 0.603940263122975
1.60000014305115 0.637042374110034
1.64000010490417 0.671570818616942
1.6800000667572 0.768573049015447
1.72000002861023 0.854839414125309
1.75999999046326 0.921061127208674
1.80000019073486 0.972855269388712
1.84000015258789 1.02464311800119
1.88000011444092 1.06445564214345
1.92000007629395 0.997485319848947
1.96000003814697 0.905785809928621
2 0.826862849189619
2.04000020027161 0.833231959205346
2.08000016212463 0.651094119210421
2.12000012397766 0.258213124322324
2.16000008583069 0.236399503264688
2.20000004768372 0.0448681660358898
2.24000000953674 0.0231399156933656
2.28000020980835 -0.0463956041838169
2.32000017166138 0.0291771100656393
2.3600001335144 0.043865510420214
2.40000009536743 0.0832582982314672
2.44000005722046 -0.0523143689140556
2.48000001907349 0.0410986709152589
2.51999998092651 0.0119072781358935
2.56000018119812 -0.0342830780656945
2.60000014305115 -0.0409483506911916
2.64000010490417 0.00288127999770402
2.6800000667572 0.0280583743270773
2.72000002861023 -0.00967951797540012
2.75999999046326 -0.0235366700101965
2.80000019073486 0.0118969778108895
2.84000015258789 0.0404327908569302
2.88000011444092 -0.0160557699592126
2.92000007629395 0.00551272847477687
2.96000003814697 0.0458368679025721
3 0.00325816038120176
3.04000020027161 0.0808821824303544
3.08000016212463 0.104864896536889
3.12000012397766 0.0942851146853216
3.16000008583069 0.12244417347231
3.20000004768372 0.142543943031819
3.24000000953674 0.158265555786577
3.28000020980835 0.178455439635281
3.32000017166138 0.0464456730218337
3.3600001335144 1.71684116383779
3.40000009536743 1.80984270687744
3.44000005722046 1.8959152879875
3.48000001907349 1.96452916292364
3.51999998092651 2.01682161590802
3.56000018119812 2.07905052524954
3.60000014305115 2.14780912078468
3.64000010490417 2.2216339958221
3.6800000667572 2.27966431102876
3.72000002861023 2.35061349917311
3.75999999046326 2.42485896977757
3.80000019073486 2.48595951961801
3.84000015258789 1.35608615843019
3.88000011444092 0.800633217619586
3.92000007629395 0.806348912787322
3.96000003814697 0.793682823714538
4 0.781876202946178
4.04000020027161 0.779983413198837
4.08000016212463 0.743868245406269
4.12000012397766 0.714927957546375
4.16000008583069 0.673641919155063
4.20000004768372 0.640113584474457
4.24000000953674 0.609827272549596
4.28000020980835 0.619925098724548
4.32000017166138 0.593513720024116
4.3600001335144 0.585118610765496
4.40000009536743 0.542491445026684
4.44000005722046 0.514798994022574
4.48000001907349 0.495125997341028
4.51999998092651 0.455079345551271
4.56000018119812 0.411902997749079
4.60000014305115 0.414135237819253
4.64000010490417 0.392015050219518
4.6800000667572 0.381675412418476
4.72000002861023 0.373460900595211
4.75999999046326 0.356430183844504
4.80000019073486 0.353605795312068
4.84000015258789 0.338985483920712
4.88000011444092 0.297862774116112
4.92000007629395 0.270229610660648
4.96000003814697 0.234806377942342
5 0.197459938713907
5.04000020027161 0.160225007734228
5.08000016212463 0.129040816772171
5.12000012397766 0.10975342962631
5.16000008583069 0.102104780014038
5.20000004768372 0.070453461075896
5.24000000953674 0.0478691878701982
5.28000020980835 0.0317514027692696
5.32000017166138 0.0180137507933136
5.3600001335144 0.0146181817805523
5.40000009536743 -0.0146888413641624
5.44000005722046 -0.0213805046728804
5.48000001907349 -0.00648853465014628
5.51999998092651 0.00530702145781812
5.56000018119812 0.00225931900450016
5.60000014305115 -0.00310089537004378
5.64000010490417 -0.00748847930732433
5.6800000667572 0.030737689679129
5.72000002861023 0.0600183831049996
5.75999999046326 0.0508558302315239
5.80000019073486 0.0936188338125774
5.84000015258789 0.102305691474107
5.88000011444092 0.105565447949982
5.92000007629395 0.0765239492002127
5.96000003814697 0.112394011334227
6 0.120328411551254
6.04000020027161 0.110537871326804
6.08000016212463 0.138945929325808
6.12000012397766 0.166163563309374
6.16000008583069 0.183933603125409
6.20000004768372 0.1953600528994
6.24000000953674 0.197118693448992
6.28000020980835 0.195238790931309
6.32000017166138 0.199424579114837
6.3600001335144 0.197564247717759
6.40000009536743 0.190046991038282
6.44000005722046 0.183229822318331
6.48000001907349 0.152979201536525
6.51999998092651 0.125625637089855
6.56000018119812 0.0841914873662651
6.60000014305115 0.0510819772389325
6.64000010490417 0.0504747655169298
6.6800000667572 0.00932592477742102
6.72000002861023 -0.0129447702698416
6.75999999046326 -0.0164690498439564
6.80000019073486 -0.0262103437160532
6.84000015258789 -0.0329670587294698
6.88000011444092 -0.0384263744283208
6.92000007629395 -0.0262605696872765
6.96000003814697 -0.0351780229515891
7 -0.0432309925837881
7.04000020027161 -0.0151981113250474
7.08000016212463 0.0338100476016889
7.12000012397766 0.0357679562669317
7.16000008583069 0.0592956410198714
7.20000004768372 0.0931031224978316
7.24000000953674 0.109539192761189
7.28000020980835 0.146419504304959
7.32000017166138 0.177827589876486
7.3600001335144 0.208639028888138
7.40000009536743 0.241338746802459
7.44000005722046 0.242672236542932
7.48000001907349 0.289665332091754
7.51999998092651 0.321895144934148
7.56000018119812 0.328060234949125
7.60000014305115 0.34673592934652
7.64000010490417 0.389457517074822
7.6800000667572 0.427037136242893
7.72000002861023 0.457571806512278
7.75999999046326 0.469860417641607
7.80000019073486 0.477150907636018
7.84000015258789 0.50638483287477
7.88000011444092 0.517117167656501
7.92000007629395 0.481972901194229
7.96000003814697 0.520853194232856
8 0.555521718227795
8.04000020027161 0.524745312629409
8.08000016212463 0.534539829863825
8.12000012397766 0.570375525037023
8.16000008583069 0.516393241180434
8.20000004768372 0.537096404126594
8.24000000953674 0.482310099671144
8.28000020980835 0.501713631853749
8.32000017166138 0.523627180729363
8.3600001335144 0.487648713697725
8.40000009536743 0.510649260771908
8.44000005722046 0.470145698223579
8.48000001907349 0.481780408790086
8.51999998092651 0.499816324526014
8.56000018119812 0.489557610179938
8.60000014305115 0.513996777222232
8.64000010490417 0.513186447871466
8.6800000667572 0.535450528900637
8.72000002861023 0.53446231575872
8.75999999046326 0.546404819940854
8.80000019073486 0.564148460926211
8.84000015258789 0.551930000078432
8.88000011444092 0.58428951203691
8.92000007629395 0.764741188347422
8.96000003814697 1.55375805788906
9 1.49972280383137
9.04000020027161 1.42908013881005
9.08000016212463 1.34565933230997
9.12000012397766 1.28410794596156
9.16000008583069 1.20190537542625
9.20000004768372 1.05106507492809
9.24000000953674 0.97219118376518
9.28000020980835 0.85186529657332
9.32000017166138 0.793467436620649
9.3600001335144 0.711929168654996
9.40000009536743 0.636589809799067
9.44000005722046 0.0111924741370204
9.48000001907349 0.184432560387042
9.51999998092651 0.0928070148279042
9.56000018119812 -0.0035604359525262
9.60000014305115 0.00764646137104876
9.64000010490417 0.109509650398619
9.6800000667572 0.183795393463004
9.72000002861023 0.28361730148163
9.75999999046326 0.369584729844607
9.80000019073486 0.436554177496068
9.84000015258789 0.53993090817934
9.88000011444092 0.60936556681791
9.92000007629395 0.701576671484731
9.96000003814697 0.766588155874757
10 0.795320818150348
10.0400002002716 0.822163739214917
10.0800001621246 0.819465444767499
10.1200001239777 0.160630832338604
10.1600000858307 0.368721974828637
10.2000000476837 0.429991782721702
10.2400000095367 0.491752522703567
10.2800002098083 0.55212876541671
10.3200001716614 0.57109841778446
10.3600001335144 0.623836037549689
10.4000000953674 0.67635406648016
10.4400000572205 0.702848258560234
10.4800000190735 0.738574248868264
10.5199999809265 0.787722261072312
10.5600001811981 0.829154987026429
10.6000001430511 0.390461506995752
10.6400001049042 0.088210622555838
10.6800000667572 0.118442428606877
10.7200000286102 0.0862487188791029
10.7599999904633 0.109056080858271
10.8000001907349 0.0999966986860275
10.8400001525879 0.112223766509256
10.8800001144409 0.125940889207166
10.9200000762939 0.0939719427990252
10.960000038147 0.0954343909285349
11 0.0495611679202714
11.0400002002716 0.0499953368987809
11.0800001621246 0.0522129403243237
11.1200001239777 0.000892564743278264
11.1600000858307 0.00416056760696292
11.2000000476837 -0.0114628756722205
11.2400000095367 -0.0287236270662935
11.2800002098083 -0.0303494237736359
11.3200001716614 0.00908528571427872
11.3600001335144 0.00101057931345877
11.4000000953674 0.031681532710757
11.4400000572205 0.0330757205978438
11.4800000190735 0.0441293978755079
11.5199999809265 0.0546038547184303
11.5600001811981 0.0586091412894697
11.6000001430511 0.0628140335914317
11.6400001049042 0.0870766563838677
11.6800000667572 0.0898512604110598
11.7200000286102 0.0952771450046767
11.7599999904633 0.0934841837491885
11.8000001907349 0.0864083913513956
11.8400001525879 0.0849029112771754
11.8800001144409 0.0901387050616518
11.9200000762939 0.109423038511248
11.960000038147 0.117090565579414
12 0.128960814528231
12.0400002002716 0.146061296707827
12.0800001621246 0.140733140537947
12.1200001239777 0.154080838830084
12.1600000858307 0.133019293641508
12.2000000476837 0.126440674501128
12.2400000095367 0.11672144334535
12.2800002098083 0.0787771936971545
12.3200001716614 0.0256632256961259
12.3600001335144 0.0261733034384334
12.4000000953674 0.0303289529064699
12.4400000572205 -0.000944586122840076
12.4800000190735 -0.00968484800864877
12.5199999809265 -0.000273297049830694
12.5600001811981 -0.0052414452892844
12.6000001430511 0.0147797617976586
12.6400001049042 0.0146422796429012
12.6800000667572 0.0333753775917582
12.7200000286102 0.0594680698933437
12.7599999904633 0.0433029877477595
12.8000001907349 0.0758864451535606
12.8400001525879 0.132975106045697
12.8800001144409 0.152004155587209
12.9200000762939 0.193625398094346
12.960000038147 0.193292114022355
13 0.234876265889778
13.0400002002716 0.244397281672815
13.0800001621246 0.292911429532265
13.1200001239777 0.345180930434846
13.1600000858307 0.341011107825739
13.2000000476837 0.35596996360535
13.2400000095367 0.353265448015626
13.2800002098083 0.384979407064744
13.3200001716614 0.435744258130932
13.3600001335144 0.411237814555455
13.4000000953674 0.436464773778761
13.4400000572205 0.443470334011553
13.4800000190735 0.454673119389233
13.5199999809265 0.463302464134554
13.5600001811981 0.425825911697083
13.6000001430511 0.44239121962162
13.6400001049042 0.397786357296248
13.6800000667572 0.386133315776477
13.7200000286102 0.346914837932545
13.7599999904633 0.33891940100642
13.8000001907349 0.341350930766195
13.8400001525879 0.309790035946524
13.8800001144409 0.29750845610216
13.9200000762939 0.291639044370678
13.960000038147 0.288427938829084
14 0.274808572329144
14.0400002002716 0.25634174722573
14.0800001621246 0.257866649496806
14.1200001239777 0.251856026769563
14.1600000858307 0.244557503850695
14.2000000476837 0.23708568710016
14.2400000095367 0.233878568140892
14.2800002098083 0.23902251210239
14.3200001716614 0.197539551329336
14.3600001335144 0.208320735808366
14.4000000953674 0.196750895956934
14.4400000572205 0.204102732860377
14.4800000190735 0.212694612034852
14.5199999809265 0.218067670237961
14.5600001811981 0.211442809596015
14.6000001430511 0.175364470214746
14.6400001049042 0.173946624046427
14.6800000667572 0.168364948373697
14.7200000286102 0.168050266272658
14.7599999904633 0.17601600013847
14.8000001907349 0.181933060884221
14.8400001525879 0.177373849603686
14.8800001144409 0.197899952538733
14.9200000762939 0.213707804393029
14.960000038147 0.271238492358582
15 0.251444165503166
15.0400002002716 0.301896236658166
15.0800001621246 0.371651485747634
15.1200001239777 0.346971511720756
15.1600000858307 0.39226866970485
15.2000000476837 0.380482063803515
15.2400000095367 0.399623485315372
15.2800002098083 0.437694281824329
15.3200001716614 0.429529490736138
15.3600001335144 0.444422087994565
15.4000000953674 0.424034117137449
15.4400000572205 0.466133375252341
15.4800000190735 0.518911727566767
15.5199999809265 0.481066425741159
15.5600001811981 0.509581586610643
15.6000001430511 0.466325235525048
15.6400001049042 0.488396030737598
15.6800000667572 0.494905862370299
15.7200000286102 0.473714082063661
15.7599999904633 0.486474816098836
15.8000001907349 0.533609134022427
15.8400001525879 0.541887694552721
15.8800001144409 0.570135719502637
15.9200000762939 0.515307093370158
15.960000038147 0.516284628827609
16 0.330152038730749
16.0400002002716 0.367435154613507
16.0800001621246 0.327668306125615
16.1200001239777 0.34750467826179
16.1600000858307 0.352935054294176
16.2000000476837 0.391503327390787
16.2400000095367 0.432523304960908
16.2800002098083 0.406801371463973
16.3200001716614 0.45341647350826
16.3600001335144 0.505841648368469
16.4000000953674 0.467727895771039
16.4400000572205 0.506102889100949
16.4800000190735 0.473141046614069
16.5199999809265 0.479638922135009
16.5600001811981 0.525803814491428
16.6000001430511 0.515817534573067
16.6400001049042 0.514998743522826
16.6800000667572 0.480735315134913
16.7200000286102 0.478211176518587
16.7599999904633 0.477970710889447
16.8000001907349 0.274700255046721
16.8400001525879 0.260458180035826
16.8800001144409 0.0784179343925006
16.9200000762939 0.0813456511950165
16.960000038147 0.0119598054522513
17 0.00689240438001953
17.0400002002716 0.0203330215755822
17.0800001621246 -0.0136044825773348
17.1200001239777 0.0546710789614252
17.1600000858307 0.0640013333894688
17.2000000476837 0.193876301265352
17.2400000095367 0.12437566054101
17.2800002098083 0.0964831749665264
17.3200001716614 0.0776572298173204
17.3600001335144 0.137382696135798
17.4000000953674 0.0842097948356671
17.4400000572205 0.0489896024131397
17.4800000190735 0.116478444927321
17.5199999809265 0.0354244869242144
17.5600001811981 0.0598586379706848
17.6000001430511 0.0075663268172757
17.6400001049042 -0.0277578415043747
17.6800000667572 -0.11066865723301
17.7200000286102 -0.0933330142122363
17.7599999904633 -0.132659020156006
17.8000001907349 -0.1190174236814
17.8400001525879 -0.135165294037518
17.8800001144409 -0.176077792953201
17.9200000762939 -0.233190318349368
17.960000038147 -0.224460237723129
18 -0.276026780436551
18.0400002002716 -0.27100310888099
18.0800001621246 -0.281742996626321
18.1200001239777 -0.261303698134667
18.1600000858307 -0.305963607402613
18.2000000476837 -0.366999641402713
18.2400000095367 -0.415860436203306
18.2800002098083 -0.381925922617564
18.3200001716614 -0.449515713244959
18.3600001335144 -0.479008649191778
18.4000000953674 -0.526065153924264
18.4400000572205 -0.555062775086706
18.4800000190735 -0.563840935419121
18.5199999809265 -0.650808802236333
18.5600001811981 -0.645287875088315
18.6000001430511 -0.674056840978301
18.6400001049042 -0.696454080488008
18.6800000667572 -0.705973488418374
18.7200000286102 -0.687588581678176
18.7599999904633 -0.651796222040873
18.8000001907349 -0.669318613447536
18.8400001525879 -0.668670633539845
18.8800001144409 -0.699659836064121
18.9200000762939 -0.684550010908187
18.960000038147 -0.774139808445955
19 -0.847483846314773
19.0400002002716 -0.789853684681743
19.0800001621246 -0.882705806092499
19.1200001239777 -0.887991217056227
19.1600000858307 -1.00670539696924
19.2000000476837 -1.02696917033419
19.2400000095367 -1.10683648923957
19.2800002098083 -1.17721374071009
19.3200001716614 -1.14842622928867
19.3600001335144 -1.26307464292016
19.4000000953674 -1.33711854363836
19.4400000572205 -1.37062088606972
19.4800000190735 -1.45597983487223
19.5199999809265 -1.51829022184806
19.5600001811981 -1.59490033597692
19.6000001430511 -1.71039401653401
19.6400001049042 -1.79332094032618
19.6800000667572 -1.90688166443455
19.7200000286102 -1.93277090809491
19.7599999904633 -1.98157763298714
19.8000001907349 -2.01415135951048
19.8400001525879 -2.03603167152783
19.8800001144409 -2.08145048839823
19.9200000762939 -2.13427038389052
19.960000038147 -2.17986351736825
20 -2.2215365356194
20.0400002002716 -2.25179126968644
20.0800001621246 -2.27575958698248
20.1200001239777 -2.29509360091733
20.1600000858307 -2.35356574807719
20.2000000476837 -2.37988602596902
20.2400000095367 -2.39586358905932
20.2800002098083 -2.39791396949491
20.3200001716614 -2.41858756009156
20.3600001335144 -2.43708588184386
20.4000000953674 -2.44596525147163
20.4400000572205 -2.47621739818731
20.4800000190735 -2.49832537918992
20.5199999809265 -2.5289021242192
20.5600001811981 -2.54101956334867
20.6000001430511 -2.51667965594035
20.6400001049042 -2.52324005282374
20.6800000667572 -2.53152789311519
20.7200000286102 -2.56837064707082
20.7599999904633 -2.58717134270261
20.8000001907349 -2.60497636494775
20.8400001525879 -2.65142507249725
20.8800001144409 -2.66978526305454
20.9200000762939 -2.72750643483382
20.960000038147 -2.75167688992973
21 -2.77789235085584
21.0400002002716 -2.81564277616127
21.0800001621246 -2.82378367018273
21.1200001239777 -2.88394419058188
21.1600000858307 -2.92664984135118
21.2000000476837 -2.96763101280468
21.2400000095367 -3.00754755581002
21.2800002098083 -3.01767951747502
21.3200001716614 -3.04959447355103
21.3600001335144 -3.06483539860865
21.4000000953674 -3.08108535471511
21.4400000572205 -3.09988038177406
21.4800000190735 -3.07938192962744
21.5199999809265 -3.07450179160106
21.5600001811981 -3.11126867081998
21.6000001430511 -3.0772164112275
21.6400001049042 -3.09095531866039
21.6800000667572 -3.10269302185509
21.7200000286102 -3.10620439626484
21.7599999904633 -3.14846644360885
21.8000001907349 -3.15143292877913
21.8400001525879 -3.17539500684569
21.8800001144409 -3.16336295867154
21.9200000762939 -3.16684817754278
21.960000038147 -3.19276571448159
22 -3.16221074430849
22.0400002002716 -3.14962689474149
22.0800001621246 -3.19963011182661
22.1200001239777 -3.15126987289506
22.1600000858307 -3.17046951271812
22.2000000476837 -3.17009791109698
22.2400000095367 -3.18408447135549
22.2800002098083 -3.20265986033878
22.3200001716614 -3.18135070467814
22.3600001335144 -3.23107140953676
22.4000000953674 -3.20049992058705
22.4400000572205 -3.205351533558
22.4800000190735 -3.24689020341992
22.5199999809265 -3.24618794564699
22.5600001811981 -3.31296933309177
22.6000001430511 -3.32238731323191
22.6400001049042 -3.32085445451615
22.6800000667572 -3.35386363807835
22.7200000286102 -3.33440467245594
22.7599999904633 -3.27004764585176
22.8000001907349 -3.31705378775831
22.8400001525879 -3.32534795773888
22.8800001144409 -3.35238068703109
22.9200000762939 -3.30895890740336
22.960000038147 -3.30847977971329
23 -3.31484166481725
23.0400002002716 -3.31039253919606
23.0800001621246 -3.30921887967748
23.1200001239777 -3.31496718737862
23.1600000858307 -3.30344205879509
23.2000000476837 -3.3262146420195
23.2400000095367 -3.27484787419676
23.2800002098083 -3.26676867533237
23.3200001716614 -3.27966046112407
23.3600001335144 -3.25184571148178
23.4000000953674 -3.25812513888116
23.4400000572205 -3.23622625108198
23.4800000190735 -3.20261363734703
23.5199999809265 -3.2068762297282
23.5600001811981 -3.17399343737053
23.6000001430511 -3.17085537683501
23.6400001049042 -3.09102151167569
23.6800000667572 -3.07990074763016
23.7200000286102 -3.09147149598831
23.7599999904633 -3.04601365509262
23.8000001907349 -3.00431183095612
23.8400001525879 -2.96059977383607
23.8800001144409 -2.92070586405343
23.9200000762939 -2.88615294547282
23.960000038147 -2.84515825858977
24 -2.81476280561682
24.0400002002716 -2.76147381586448
24.0800001621246 -2.74818468608631
24.1200001239777 -2.71746692050952
24.1600000858307 -2.6698666124587
24.2000000476837 -2.63814799391607
24.2400000095367 -2.5860266896065
24.2800002098083 -2.57211902024349
24.3200001716614 -2.51950719128724
24.3600001335144 -2.58940700535638
24.4000000953674 -2.59120146933747
24.4400000572205 -2.56571001634112
24.4800000190735 -2.55655367996439
24.5199999809265 -2.53298952777565
24.5600001811981 -2.50554421944333
24.6000001430511 -2.4825804451454
24.6400001049042 -2.43864632084909
24.6800000667572 -2.43142804747703
24.7200000286102 -2.41183840668535
24.7599999904633 -2.39964292712566
24.8000001907349 -2.39825341311332
24.8400001525879 -2.46718946023636
24.8800001144409 -2.46491657907907
24.9200000762939 -2.42284496173631
24.960000038147 -2.40576869199577
25 -2.39544185452559
25.0400002002716 -2.35186107948322
25.0800001621246 -2.34811620717847
25.1200001239777 -2.32657782026529
25.1600000858307 -2.31207744852484
25.2000000476837 -2.31167722606531
25.2400000095367 -2.29977998642911
25.2800002098083 -2.29588080764236
25.3200001716614 -2.27847447474836
25.3600001335144 -2.2741632278136
25.4000000953674 -2.27419167934333
25.4400000572205 -2.26377251310878
25.4800000190735 -2.2628438239207
25.5199999809265 -2.2697114643276
25.5600001811981 -2.24988932495678
25.6000001430511 -2.23938165513604
25.6400001049042 -2.23462903161494
25.6800000667572 -2.22760710015989
25.7200000286102 -2.22802254104448
25.7599999904633 -2.2445938752599
25.8000001907349 -2.24200788333704
25.8400001525879 -2.23557840532604
25.8800001144409 -2.22249489206524
25.9200000762939 -2.21677273716648
25.960000038147 -2.22533793896382
26 -2.23404312104643
26.0400002002716 -2.2593390499648
26.0800001621246 -2.27615922686597
26.1200001239777 -2.28648302737267
26.1600000858307 -2.30341907296632
26.2000000476837 -2.34453838697522
26.2400000095367 -2.4281880745939
26.2800002098083 -2.43865109641466
26.3200001716614 -2.48650210017652
26.3600001335144 -2.44970634277806
26.4000000953674 -2.45475809371296
26.4400000572205 -2.46755026655577
26.4800000190735 -2.47681788892842
26.5199999809265 -2.48396338592078
26.5600001811981 -2.49725018920557
26.6000001430511 -2.50490773581862
26.6400001049042 -2.49390751057243
26.6800000667572 -2.50288433766719
26.7200000286102 -2.4513887598724
26.7599999904633 -2.45362101184234
26.8000001907349 -2.47598063118207
26.8400001525879 -2.50348544767923
26.8800001144409 -2.52222594099045
26.9200000762939 -2.53797841460131
26.960000038147 -2.54510518718761
27 -2.57157083690572
27.0400002002716 -2.62029681031889
27.0800001621246 -2.62464812958259
27.1200001239777 -2.64189720240698
27.1600000858307 -2.6526236577082
27.2000000476837 -2.67282090842203
27.2400000095367 -2.65491264456947
27.2800002098083 -2.6574144228119
27.3200001716614 -2.70728465858581
27.3600001335144 -2.71422010483035
27.4000000953674 -2.73367314242464
27.4400000572205 -2.75768024577716
27.4800000190735 -2.77047956984109
27.5199999809265 -2.74682595589388
27.5600001811981 -2.75181138442918
27.6000001430511 -2.76049805157132
27.6400001049042 -2.73587135978063
27.6800000667572 -2.71002586255824
27.7200000286102 -2.73210615906794
27.7599999904633 -2.73611180861448
27.8000001907349 -2.71526808339848
27.8400001525879 -2.71963398659206
27.8800001144409 -2.74113967170981
27.9200000762939 -2.80385364035568
27.960000038147 -2.81192461418436
28 -2.82151117847386
28.0400002002716 -2.8662003204457
28.0800001621246 -2.86789730066448
28.1200001239777 -2.89002520328185
28.1600000858307 -2.8973569880417
28.2000000476837 -2.88877312670165
28.2400000095367 -2.92543677157336
28.2800002098083 -2.87852247512787
28.3200001716614 -2.87398743011126
28.3600001335144 -2.88075960259602
28.4000000953674 -2.83739239807695
28.4400000572205 -2.81173675389736
28.4800000190735 -2.81765386803174
28.5199999809265 -2.78147043421476
28.5600001811981 -2.78150852027185
28.6000001430511 -2.75318300056105
28.6400001049042 -2.76884247961643
28.6800000667572 -2.75838117614253
28.7200000286102 -2.76060105664844
28.7599999904633 -2.75837731917145
28.8000001907349 -2.76579487820908
28.8400001525879 -2.87378495935205
28.8800001144409 -2.8760473519881
28.9200000762939 -2.90582607072331
28.960000038147 -2.91425939096074
29 -2.93657455293848
29.0400002002716 -2.94575688980919
29.0800001621246 -2.97916976518797
29.1200001239777 -3.01724646396958
29.1600000858307 -3.02416281957772
29.2000000476837 -2.99568116818603
29.2400000095367 -3.02261579426044
29.2800002098083 -3.00669704736426
29.3200001716614 -3.05927785604151
29.3600001335144 -3.06683659375732
29.4000000953674 -3.0684727393434
29.4400000572205 -3.0380287586231
29.4800000190735 -3.0375686355351
29.5199999809265 -3.04340467587728
29.5600001811981 -3.04687883838274
29.6000001430511 -2.98890304163213
29.6400001049042 -2.9990769212831
29.6800000667572 -2.95547896174324
29.7200000286102 -2.98736340999783
29.7599999904633 -2.98522332973768
29.8000001907349 -2.99763381204247
29.8400001525879 -3.01471429135446
29.8800001144409 -3.0108347675208
29.9200000762939 -3.03038114623164
29.960000038147 -3.02784021447056
30 -3.00634249675779
30.0400002002716 -3.04254666053674
30.0800001621246 -3.03064264155103
30.1200001239777 -3.07279285756794
30.1600000858307 -3.07773873836244
30.2000000476837 -3.06462669168446
30.2400000095367 -3.12608503955543
30.2800002098083 -3.12523114745277
30.3200001716614 -3.09766465266305
30.3600001335144 -3.14790265819908
30.4000000953674 -3.26171812992882
30.4400000572205 -3.23586516421113
30.4800000190735 -3.22879795246832
30.5199999809265 -3.24008360942349
30.5600001811981 -3.23119952952039
30.6000001430511 -3.22169658525557
30.6400001049042 -3.23104687855408
30.6800000667572 -3.21673438359682
30.7200000286102 -3.20422933458226
30.7599999904633 -3.19849504325427
30.8000001907349 -3.15316621136524
30.8400001525879 -3.11502467137735
30.8800001144409 -3.09366118605176
30.9200000762939 -3.12221302091524
30.960000038147 -3.10746115952173
31 -3.11950373313499
31.0400002002716 -3.11420236875809
31.0800001621246 -3.1462762997198
31.1200001239777 -3.12190182512929
31.1600000858307 -3.1413995377537
31.2000000476837 -3.07391465867788
31.2400000095367 -3.07354459741406
31.2800002098083 -3.01698151800646
31.3200001716614 -2.99326740638999
31.3600001335144 -2.97976676209942
31.4000000953674 -2.9505046102769
31.4400000572205 -2.92490161649248
31.4800000190735 -2.90618219454163
31.5199999809265 -2.86002318940931
31.5600001811981 -2.90033390738575
31.6000001430511 -2.96704108478009
31.6400001049042 -3.07472301259415
31.6800000667572 -3.06883054768239
31.7200000286102 -3.00946945013467
31.7599999904633 -2.97002807907616
31.8000001907349 -2.95274972630053
31.8400001525879 -2.85992727628906
31.8800001144409 -2.78052903604534
31.9200000762939 -2.70911716382429
31.960000038147 -2.64946456537842
32 -2.51962526279854
32.0400002002716 -2.46365620557724
32.0800001621246 -2.47920235855583
32.1200001239777 -2.26170462784135
32.1600000858307 -2.20704535742009
32.2000000476837 -2.24299002479363
32.2400000095367 -2.20396445839646
32.2800002098083 -2.09618641617809
32.3200001716614 -1.9770392294933
32.3600001335144 -1.97561668216405
32.4000000953674 -1.9571304304276
32.4400000572205 -1.92008592337765
32.4800000190735 -1.84891699494613
32.5199999809265 -1.82512187252416
32.5600001811981 -1.71903130640654
32.6000001430511 -1.62936754554125
32.6400001049042 -1.59266188297517
32.6800000667572 -1.62128089127122
32.7200000286102 -1.60972519106358
32.7599999904633 -1.58332909982517
32.8000001907349 -1.64517557763296
32.8400001525879 -1.55431943700714
32.8800001144409 -1.09901995845896
32.9200000762939 -1.01675410008137
32.960000038147 -0.933309716740302
33 -0.898074741824657
33.0400002002716 -0.829436605363087
33.0800001621246 -0.783736095627514
33.1200001239777 -0.742654390037021
33.1600000858307 -0.70801887724927
33.2000000476837 -0.601050299094892
33.2400000095367 -0.522436921926209
33.2800002098083 -0.381034988096312
33.3200001716614 -0.350820764074171
33.3600001335144 -0.683945456779577
33.4000000953674 -0.879792398929308
33.4400000572205 -0.792223194796082
33.4800000190735 -0.760605237564636
33.5199999809265 -0.691684602925099
33.5600001811981 -0.668958795737696
33.6000001430511 -0.575949471987297
33.6400001049042 -0.661034602890749
33.6800000667572 -0.70473204934717
33.7200000286102 -0.685390736011053
33.7599999904633 -0.675678718792236
33.8000001907349 -0.615669240678282
33.8400001525879 -0.602849580566696
33.8800001144409 -0.565048967036272
33.9200000762939 -0.577432030661821
33.960000038147 -0.555647910091302
34 -0.634970574233768
34.0400002002716 -0.680970577002872
34.0800001621246 -0.736431421807733
34.1200001239777 -0.775654138014481
34.1600000858307 -0.800359213922397
};
\path [draw=black, fill=black] (axis cs:15,0.12)
--(axis cs:15.5,-0.1)
--(axis cs:15.125,-0.1)
--(axis cs:15.125,-2)
--(axis cs:14.875,-2)
--(axis cs:14.875,-0.1)
--(axis cs:14.5,-0.1)
--cycle;

\path [draw=black, fill=black] (axis cs:17,2.97)
--(axis cs:17.5,2.75)
--(axis cs:17.125,2.75)
--(axis cs:17.125,2.2)
--(axis cs:16.875,2.2)
--(axis cs:16.875,2.75)
--(axis cs:16.5,2.75)
--cycle;

\addplot [line width=2.4000000000000004pt, color1, dashed, forget plot]
table {%
0 -0.0417787396782297
0.0400002002716064 -0.0234894145660781
0.0800001621246338 -0.00316769069747583
0.120000123977661 0.0171410027993888
0.160000085830688 0.0353008948896494
0.200000047683716 0.0519132876052167
0.240000009536743 0.047620933155939
0.28000020980835 0.0354367334666362
0.320000171661377 0.0243375702749589
0.360000133514404 0.0147874162005871
0.400000095367432 0.00626683768990029
0.440000057220459 -0.00117534729387254
0.480000019073486 -0.00759592921282741
0.519999980926514 -0.0134284232414302
0.56000018119812 -0.0182270835009956
0.600000143051147 -0.022438744343384
0.640000104904175 -0.026070568590947
0.680000066757202 -0.0291511821203337
0.720000028610229 -0.0319400999575896
0.759999990463257 -0.0338169274746991
0.800000190734863 -0.0352436446804821
0.840000152587891 -0.0363389452063536
0.880000114440918 -0.0371832000288373
0.920000076293945 -0.0378755526499222
0.960000038146973 -0.0379489570409571
1 -0.0377584947677796
1.04000020027161 -0.0376390233325493
1.08000016212463 -0.0374952551139339
1.12000012397766 -0.0372917578207389
1.16000008583069 -0.0369938784342557
1.20000004768372 -0.0366344301219262
1.24000000953674 -0.0360929457393068
1.28000020980835 -0.0354371027776929
1.32000017166138 -0.0345109548601747
1.3600001335144 -0.0334581229483179
1.40000009536743 -0.0324192791695158
1.44000005722046 -0.030980956112695
1.48000001907349 -0.0293790907626932
1.51999998092651 -0.0272088269687246
1.56000018119812 -0.024882385056818
1.60000014305115 -0.0226886086533483
1.64000010490417 -0.0190998139744281
1.6800000667572 -0.0151483238709784
1.72000002861023 -0.0110182941941803
1.75999999046326 -0.00714158725049169
1.80000019073486 -0.00356854737946884
1.84000015258789 0.00048544445960752
1.88000011444092 0.00451394310810775
1.92000007629395 0.00841177825825727
1.96000003814697 0.0119801034553173
2 0.01527436559098
2.04000020027161 0.0187924828262535
2.08000016212463 0.0222216665186688
2.12000012397766 0.0254353989836296
2.16000008583069 0.0283560527972835
2.20000004768372 0.0310911488127918
2.24000000953674 0.0341602267221054
2.28000020980835 0.037264868811495
2.32000017166138 0.0403389177592532
2.3600001335144 0.0432238920390264
2.40000009536743 0.0463886045314703
2.44000005722046 0.0496490639713337
2.48000001907349 0.0526877993519101
2.51999998092651 0.0564129153244892
2.56000018119812 0.0602846008974836
2.60000014305115 0.0638569429020855
2.64000010490417 0.0677044802291149
2.6800000667572 0.0715338351926627
2.72000002861023 0.0756254784338014
2.75999999046326 0.079603058963355
2.80000019073486 0.0832873003763861
2.84000015258789 0.0874524012518538
2.88000011444092 0.0916925884473081
2.92000007629395 0.0960618042740521
2.96000003814697 0.100200412245092
3 0.104271661953581
3.04000020027161 0.108167817374537
3.08000016212463 0.111783227709938
3.12000012397766 0.115684704289017
3.16000008583069 0.119522929461469
3.20000004768372 0.12315851979343
3.24000000953674 0.125141227311931
3.28000020980835 0.124029761602394
3.32000017166138 0.122598006925633
3.3600001335144 0.121279407271869
3.40000009536743 0.119340532654102
3.44000005722046 0.117189462631772
3.48000001907349 0.127221066263764
3.51999998092651 0.113301125155132
3.56000018119812 0.11144265952714
3.60000014305115 0.109868516394461
3.64000010490417 0.107719769339863
3.6800000667572 0.106500928672304
3.72000002861023 0.105042468454708
3.75999999046326 0.103877521716361
3.80000019073486 0.102937771611001
3.84000015258789 0.102171915711101
3.88000011444092 0.101579580501347
3.92000007629395 0.101226654302596
3.96000003814697 0.101113989199034
4 0.101616293426719
4.04000020027161 0.102418837022119
4.08000016212463 0.103231067324208
4.12000012397766 0.104120570963666
4.16000008583069 0.105118444100313
4.20000004768372 0.1061130722062
4.24000000953674 0.107099502125028
4.28000020980835 0.10808039216406
4.32000017166138 0.10905367990172
4.3600001335144 0.110000619620419
4.40000009536743 0.110710271456434
4.44000005722046 0.121753809426045
4.48000001907349 0.111943376392363
4.51999998092651 0.112505432724912
4.56000018119812 0.113037328530031
4.60000014305115 0.113606122874255
4.64000010490417 0.114212016721308
4.6800000667572 0.114937984492269
4.72000002861023 0.115942378212514
4.75999999046326 0.117000394895368
4.80000019073486 0.118307752468225
4.84000015258789 0.119605674340697
4.88000011444092 0.120902193469763
4.92000007629395 0.122538104963462
4.96000003814697 0.124289500251855
5 0.126108929838909
5.04000020027161 0.127843713065531
5.08000016212463 0.129489474225519
5.12000012397766 0.131066693075334
5.16000008583069 0.132579986193214
5.20000004768372 0.13395889045711
5.24000000953674 0.135024695265243
5.28000020980835 0.136097275109658
5.32000017166138 0.137301891448054
5.3600001335144 0.138522282896608
5.40000009536743 0.140322133246502
5.44000005722046 0.142360937875605
5.48000001907349 0.144572170213079
5.51999998092651 0.146722021325224
5.56000018119812 0.148727839038772
5.60000014305115 0.150742032117834
5.64000010490417 0.152816424101772
5.6800000667572 0.154949766866354
5.72000002861023 0.157208720762795
5.75999999046326 0.159367421435761
5.80000019073486 0.161433522444049
5.84000015258789 0.163369160557194
5.88000011444092 0.164821837764834
5.92000007629395 0.166124342488959
5.96000003814697 0.167369968847014
6 0.168116567065514
6.04000020027161 0.168625163666823
6.08000016212463 0.168751701543232
6.12000012397766 0.168478845217233
6.16000008583069 0.168307856437836
6.20000004768372 0.168231728600646
6.24000000953674 0.168188884458664
6.28000020980835 0.167959200938365
6.32000017166138 0.167825517347502
6.3600001335144 0.167778793190581
6.40000009536743 0.167686063493615
6.44000005722046 0.167577660315265
6.48000001907349 0.167414043164538
6.51999998092651 0.167352124229287
6.56000018119812 0.16730211208751
6.60000014305115 0.16726784797416
6.64000010490417 0.166864112795472
6.6800000667572 0.165765939475376
6.72000002861023 0.164868805072815
6.75999999046326 0.164104203344044
6.80000019073486 0.163175634034676
6.84000015258789 0.162404237764292
6.88000011444092 0.161376542242743
6.92000007629395 0.160366243110981
6.96000003814697 0.159332653747468
7 0.158396458797799
7.04000020027161 0.157743728819583
7.08000016212463 0.157670882961741
7.12000012397766 0.157998181690912
7.16000008583069 0.158570906013077
7.20000004768372 0.159516228425704
7.24000000953674 0.160643503265796
7.28000020980835 0.162220036107218
7.32000017166138 0.163808533736528
7.3600001335144 0.165476506757328
7.40000009536743 0.167105373620904
7.44000005722046 0.168665782559703
7.48000001907349 0.169805000798356
7.51999998092651 0.170751800959562
7.56000018119812 0.171827140200371
7.60000014305115 0.172935017713235
7.64000010490417 0.174222904089218
7.6800000667572 0.175631408882665
7.72000002861023 0.176995424582933
7.75999999046326 0.178355224513547
7.80000019073486 0.179779153121134
7.84000015258789 0.181321531430517
7.88000011444092 0.183416695002654
7.92000007629395 0.185557119325337
7.96000003814697 0.188003976170535
8 0.190525844739645
8.04000020027161 0.193066122560267
8.08000016212463 0.195395145382109
8.12000012397766 0.197570964148006
8.16000008583069 0.199726411527884
8.20000004768372 0.201854983180803
8.24000000953674 0.203986200810348
8.28000020980835 0.206255365030284
8.32000017166138 0.208366027970262
8.3600001335144 0.210387484551534
8.40000009536743 0.212325539830309
8.44000005722046 0.214381526209379
8.48000001907349 0.216517628578634
8.51999998092651 0.218616770703851
8.56000018119812 0.221095809107547
8.60000014305115 0.22355554916839
8.64000010490417 0.226048185659467
8.6800000667572 0.228557681507201
8.72000002861023 0.230702647590885
8.75999999046326 0.23243481748732
8.80000019073486 0.234081004254391
8.84000015258789 0.23558011349421
8.88000011444092 0.236788258690557
8.92000007629395 0.237924488952279
8.96000003814697 0.238731923484593
9 0.239499077388328
9.04000020027161 0.239734043719992
9.08000016212463 0.239799866402045
9.12000012397766 0.239988545359118
9.16000008583069 0.240262057773767
9.20000004768372 0.27249044538246
9.24000000953674 0.289991062240143
9.28000020980835 0.306487324447838
9.32000017166138 0.322004096110039
9.3600001335144 0.336267088026604
9.40000009536743 0.350096351134862
9.44000005722046 0.363051527791928
9.48000001907349 0.375882951802808
9.51999998092651 0.388226280791758
9.56000018119812 0.400754764070436
9.60000014305115 0.414008879288613
9.64000010490417 0.427690903121654
9.6800000667572 0.441974399828963
9.72000002861023 0.457064922618433
9.75999999046326 0.473005982408575
9.80000019073486 0.490280442656957
9.84000015258789 0.509064261184364
9.88000011444092 0.529485993180569
9.92000007629395 0.552115731425192
9.96000003814697 0.576684401166912
10 0.60325270940883
10.0400002002716 0.649831130856617
10.0800001621246 0.663265913770234
10.1200001239777 0.697282600223065
10.1600000858307 0.734081533815605
10.2000000476837 0.773597182126251
10.2400000095367 0.815988606103267
10.2800002098083 0.861317943809147
10.3200001716614 0.909598423165704
10.3600001335144 0.961256891333492
10.4000000953674 1.01645285377695
10.4400000572205 1.07552925798337
10.4800000190735 1.13846920296723
10.5199999809265 1.20554806746357
10.5600001811981 1.27517714672044
10.6000001430511 1.35045689811077
10.6400001049042 1.42734502594035
10.6800000667572 1.5065466494216
10.7200000286102 1.58753967702282
10.7599999904633 1.66973949662613
10.8000001907349 1.75282800954276
10.8400001525879 1.8364926698402
10.8800001144409 1.92004994454084
10.9200000762939 2.00314700538797
10.960000038147 2.08522734424601
11 2.16606889393365
11.0400002002716 2.2451298367563
11.0800001621246 2.32178394087486
11.1200001239777 2.39562664107728
11.1600000858307 2.46621406025363
11.2000000476837 2.55660417158619
11.2400000095367 2.59551971083539
11.2800002098083 2.65369595965046
11.3200001716614 2.70647836614852
11.3600001335144 2.75369088275477
11.4000000953674 2.79472028564417
11.4400000572205 2.82932207898036
11.4800000190735 2.85686386213842
11.5199999809265 2.87708908193203
11.5600001811981 2.88963753824534
11.6000001430511 2.89418151611971
11.6400001049042 2.89020334658292
11.6800000667572 2.87707720121421
11.7200000286102 2.85438721110032
11.7599999904633 2.82163607458536
11.8000001907349 2.81946235112765
11.8400001525879 2.82116591494254
11.8800001144409 2.82292899929882
11.9200000762939 2.82487931454895
11.960000038147 2.82687799052953
12 2.82901084490146
12.0400002002716 2.83118772363196
12.0800001621246 2.83336431404865
12.1200001239777 2.83554041684541
12.1600000858307 2.83770421958513
12.2000000476837 2.83941622049006
12.2400000095367 2.84091721275971
12.2800002098083 2.84239674481811
12.3200001716614 2.84384764503737
12.3600001335144 2.84521260282666
12.4000000953674 2.84626790164339
12.4400000572205 2.84747487840261
12.4800000190735 2.84850920790293
12.5199999809265 2.8494915962145
12.5600001811981 2.85025217788525
12.6000001430511 2.85099510747219
12.6400001049042 2.85197948850158
12.6800000667572 2.85318728562048
12.7200000286102 2.8545628921675
12.7599999904633 2.85601886621953
12.8000001907349 2.8575184955966
12.8400001525879 2.85890121794457
12.8800001144409 2.85989187934172
12.9200000762939 2.86090450652759
12.960000038147 2.86173581345884
13 2.86263661943759
13.0400002002716 2.86365100062622
13.0800001621246 2.86510819226026
13.1200001239777 2.86678465209858
13.1600000858307 2.86855735242891
13.2000000476837 2.87037406523006
13.2400000095367 2.87229955612768
13.2800002098083 2.87434111080212
13.3200001716614 2.87639503129837
13.3600001335144 2.8784233679331
13.4000000953674 2.88032740506776
13.4400000572205 2.88234655692363
13.4800000190735 2.88488259176671
13.5199999809265 2.88750131518647
13.5600001811981 2.89008614787922
13.6000001430511 2.89265031655973
13.6400001049042 2.89572143181652
13.6800000667572 2.89886507095544
13.7200000286102 2.90189527304969
13.7599999904633 2.90485947248228
13.8000001907349 2.90775999261128
13.8400001525879 2.91064154555121
13.8800001144409 2.9135789669666
13.9200000762939 2.91669274001231
13.960000038147 2.92104549711275
14 2.92549185329583
14.0400002002716 2.9310390710285
14.0800001621246 2.93652863347197
14.1200001239777 2.94194056627251
14.1600000858307 2.9481758322133
14.2000000476837 2.95432151438774
14.2400000095367 2.96028160901867
14.2800002098083 2.96602889185049
14.3200001716614 2.97117268010714
14.3600001335144 2.97577392363073
14.4000000953674 2.98027180157868
14.4400000572205 2.98479270070921
14.4800000190735 2.9891602958126
14.5199999809265 2.99374931884239
14.5600001811981 2.99818193068313
14.6000001430511 3.00241745373083
14.6400001049042 3.0061846823938
14.6800000667572 3.00978859099482
14.7200000286102 3.01305770668934
14.7599999904633 3.01603644197872
14.8000001907349 3.01867789969351
14.8400001525879 3.02114224926212
14.8800001144409 3.02332188552257
14.9200000762939 3.02522324269404
14.960000038147 3.02714463113932
15 3.02902239115438
15.2000000476837 3.03581285876219
15.2400000095367 3.03682919669528
15.2800002098083 3.03647918557522
15.3200001716614 3.03588694612308
15.3600001335144 3.03560266712244
15.4000000953674 3.03519843673356
15.4400000572205 3.03466744154129
15.4800000190735 3.03419394789522
15.5199999809265 3.03361529292566
15.5600001811981 3.03314342764902
15.6000001430511 3.03218547747298
15.6400001049042 3.03154820456549
15.6800000667572 3.03083075783232
15.7200000286102 3.03019726795694
15.7599999904633 3.02973953767429
15.8000001907349 3.02993220761341
15.8400001525879 3.03059178639963
15.8800001144409 3.032015592186
15.9200000762939 3.03396743956902
15.960000038147 3.0362630955797
16 3.03912061617393
16.0400002002716 3.0419153660061
16.0800001621246 3.04472339185263
16.1200001239777 3.04807617225551
16.1600000858307 3.05143227117412
16.2000000476837 3.05530167094487
16.2400000095367 3.05923011438077
16.2800002098083 3.06321889489764
16.3200001716614 3.06716322145259
16.3600001335144 3.07085821029738
16.4000000953674 3.07493434208533
16.4400000572205 3.07912470784107
16.4800000190735 3.08352294577925
16.5199999809265 3.08816542401901
16.5600001811981 3.0928191337742
16.6000001430511 3.09779896709111
16.6400001049042 3.10257366390395
16.6800000667572 3.10763294493057
16.7200000286102 3.11255837134667
16.7599999904633 3.11724023618146
16.8000001907349 3.12234074492009
16.8400001525879 3.12758011258413
16.8800001144409 3.13321495302348
16.9200000762939 3.13822630652863
16.960000038147 3.13753400168899
17 3.13608046020745
17.0400002002716 3.13498617028145
17.0800001621246 3.13327969580245
17.1200001239777 3.12992708185703
17.1600000858307 3.12667921847227
17.2000000476837 3.12170348773769
17.2400000095367 3.11752597903024
17.2800002098083 3.11183076637403
17.3200001716614 3.10599601329874
17.3600001335144 3.09492581492654
17.4000000953674 3.08307202517713
17.4400000572205 3.07334654246205
17.4800000190735 3.0628729101452
17.5199999809265 3.05207030549858
17.5600001811981 3.04111577457144
17.6000001430511 3.02891162055312
17.6400001049042 3.01603343285441
17.6800000667572 2.99881223059587
17.7200000286102 2.98320854652992
};
\addplot [line width=2.4000000000000004pt, color3, dashed, forget plot]
table {%
0.800000190734863 0.566310207694819
0.840000152587891 0.563927639889975
0.880000114440918 0.561796117788519
0.920000076293945 0.559816497888462
0.960000038146973 0.558455826218455
1 0.55735902121266
1.04000020027161 0.5561912176962
1.08000016212463 0.555047718635843
1.12000012397766 0.553963948650066
1.16000008583069 0.552974560757577
1.20000004768372 0.552046741790935
1.24000000953674 0.551300958894582
1.28000020980835 0.550669526904506
1.32000017166138 0.550308407543052
1.3600001335144 0.550073972175936
1.40000009536743 0.549825548675766
1.44000005722046 0.549976604453615
1.48000001907349 0.550291202524645
1.51999998092651 0.551174199039641
1.56000018119812 0.552213365999857
1.60000014305115 0.553119875124355
1.64000010490417 0.555421402524303
1.6800000667572 0.558085625348781
1.72000002861023 0.560928387746607
1.75999999046326 0.563517827411323
1.80000019073486 0.565803592330656
1.84000015258789 0.56857031689076
1.88000011444092 0.571311548260288
1.92000007629395 0.573922116131465
1.96000003814697 0.576203174049553
2 0.578210168906244
2.04000020027161 0.580441011189827
2.08000016212463 0.58258292760327
2.12000012397766 0.584509392789259
2.16000008583069 0.58614277932394
2.20000004768372 0.587590608060477
2.24000000953674 0.589372418690818
2.28000020980835 0.591189785828517
2.32000017166138 0.592976567497303
2.3600001335144 0.594574274498105
2.40000009536743 0.596451719711576
2.44000005722046 0.598424911872467
2.48000001907349 0.600176379974072
2.51999998092651 0.602614228667679
2.56000018119812 0.605198639288983
2.60000014305115 0.607483714014613
2.64000010490417 0.61004398406267
2.6800000667572 0.612586071747246
2.72000002861023 0.615390447709412
2.75999999046326 0.618080760959994
2.80000019073486 0.620477727421334
2.84000015258789 0.62335556101783
2.88000011444092 0.626308480934312
2.92000007629395 0.629390429482084
2.96000003814697 0.632241770174151
3 0.635025752603668
3.04000020027161 0.637634633072934
3.08000016212463 0.639962776129363
3.12000012397766 0.64257698542947
3.16000008583069 0.64512794332295
3.20000004768372 0.647476266375938
3.24000000953674 0.648171706615468
3.28000020980835 0.64577296595424
3.32000017166138 0.643053943998507
3.3600001335144 0.640448077065771
3.40000009536743 0.637221935169032
3.44000005722046 0.633783597867729
3.48000001907349 0.642527934220749
3.51999998092651 0.627320725833145
3.56000018119812 0.624174985253463
3.60000014305115 0.621313574841812
3.64000010490417 0.617877560508242
3.6800000667572 0.615371452561711
3.72000002861023 0.612625725065143
3.75999999046326 0.610173511047823
3.80000019073486 0.607946485990773
3.84000015258789 0.605893362811901
3.88000011444092 0.604013760323175
3.92000007629395 0.602373566845452
3.96000003814697 0.600973634462917
4 0.600188671411631
4.04000020027161 0.59970394005534
4.08000016212463 0.599228903078457
4.12000012397766 0.598831139438943
4.16000008583069 0.598541745296617
4.20000004768372 0.598249106123532
4.24000000953674 0.597948268763388
4.28000020980835 0.59764188385073
4.32000017166138 0.597327904309418
4.3600001335144 0.596987576749144
4.40000009536743 0.596409961306188
4.44000005722046 0.606166231996826
4.48000001907349 0.595068531684172
4.51999998092651 0.594343320737749
4.56000018119812 0.593587941591178
4.60000014305115 0.592869468656429
4.64000010490417 0.59218809522451
4.6800000667572 0.591626795716499
4.72000002861023 0.591343922157773
4.75999999046326 0.591114671561654
4.80000019073486 0.591134754182821
4.84000015258789 0.591145408776321
4.88000011444092 0.591154660626415
4.92000007629395 0.591503304841141
4.96000003814697 0.591967432850562
5 0.592499595158644
5.04000020027161 0.592947103433576
5.08000016212463 0.593305597314591
5.12000012397766 0.593595548885434
5.16000008583069 0.593821574724342
5.20000004768372 0.593913211709266
5.24000000953674 0.593691749238427
5.28000020980835 0.593477054131152
5.32000017166138 0.593394403190575
5.3600001335144 0.593327527360158
5.40000009536743 0.593840110431079
5.44000005722046 0.59459164778121
5.48000001907349 0.595515612839712
5.51999998092651 0.596378196672885
5.56000018119812 0.597096739434742
5.60000014305115 0.597823665234832
5.64000010490417 0.598610789939798
5.6800000667572 0.599456865425409
5.72000002861023 0.600428552042877
5.75999999046326 0.601299985436871
5.80000019073486 0.602078811493469
5.84000015258789 0.602727182327641
5.88000011444092 0.602892592256309
5.92000007629395 0.602907829701462
5.96000003814697 0.602866188780546
6 0.602325519720073
6.04000020027161 0.601546841369692
6.08000016212463 0.600386111967129
6.12000012397766 0.598825988362157
6.16000008583069 0.597367732303788
6.20000004768372 0.596004337187626
6.24000000953674 0.594674225766672
6.28000020980835 0.593157267294683
6.32000017166138 0.591736316424847
6.3600001335144 0.590402324988954
6.40000009536743 0.589022328013016
6.44000005722046 0.587626657555694
6.48000001907349 0.586175773125995
6.51999998092651 0.584826586911772
6.56000018119812 0.583489299818304
6.60000014305115 0.582167768425982
6.64000010490417 0.580476765968322
6.6800000667572 0.578091325369254
6.72000002861023 0.575906923687721
6.75999999046326 0.573855054679978
6.80000019073486 0.571639210418919
6.84000015258789 0.569580546869564
6.88000011444092 0.567265584069042
6.92000007629395 0.564968017658308
6.96000003814697 0.562647161015823
7 0.560423698787181
7.04000020027161 0.558483693857275
7.08000016212463 0.557123580720461
7.12000012397766 0.55616361217066
7.16000008583069 0.555449069213853
7.20000004768372 0.555107124347508
7.24000000953674 0.554947131908628
7.28000020980835 0.555236389798359
7.32000017166138 0.555537620148697
7.3600001335144 0.555918325890525
7.40000009536743 0.556259925475129
7.44000005722046 0.556533067134955
7.48000001907349 0.556385018094637
7.51999998092651 0.556044550976871
7.56000018119812 0.555832615265989
7.60000014305115 0.555653225499881
7.64000010490417 0.555653844596892
7.6800000667572 0.555775082111367
7.72000002861023 0.555851830532663
7.75999999046326 0.555924363184304
7.80000019073486 0.556061016840202
7.84000015258789 0.556316127870613
7.88000011444092 0.557124024163776
7.92000007629395 0.557977181207488
7.96000003814697 0.559136770773713
8 0.560371372063851
8.04000020027161 0.561624374932784
8.08000016212463 0.562666130475653
8.12000012397766 0.563554681962578
8.16000008583069 0.564422862063484
8.20000004768372 0.565264166437431
8.24000000953674 0.566108116788004
8.28000020980835 0.56709000605625
8.32000017166138 0.567913401717255
8.3600001335144 0.568647591019555
8.40000009536743 0.569298379019357
8.44000005722046 0.570067098119456
8.48000001907349 0.570915933209738
8.51999998092651 0.571727808055983
8.56000018119812 0.572919571507989
8.60000014305115 0.57409204428986
8.64000010490417 0.575297413501965
8.6800000667572 0.576519642070727
8.72000002861023 0.577377340875438
8.75999999046326 0.577822243492902
8.80000019073486 0.578181155308282
8.84000015258789 0.578392997269129
8.88000011444092 0.578313875186503
8.92000007629395 0.578162838169254
8.96000003814697 0.577683005422596
9 0.577162892047358
9.04000020027161 0.576110583427332
9.08000016212463 0.574889138830413
9.12000012397766 0.573790550508514
9.16000008583069 0.572776795644191
9.20000004768372 0.57182208536616
9.24000000953674 0.570855994111366
9.28000020980835 0.569824442798667
9.32000017166138 0.568593387227091
9.3600001335144 0.566729314620851
9.40000009536743 0.564893159455378
9.44000005722046 0.56248544762589
9.48000001907349 0.560097396475496
9.51999998092651 0.557205547166552
9.56000018119812 0.554323939860843
9.60000014305115 0.551834118657573
9.64000010490417 0.549279152598448
9.6800000667572 0.546673489480972
9.72000002861023 0.544063566051141
9.75999999046326 0.54133377676557
9.80000019073486 0.53880774691803
9.84000015258789 0.536502553116121
9.88000011444092 0.534387519592495
9.92000007629395 0.532873622664875
9.96000003814697 0.53153267112004
10 0.530266255499196
10.0400002002716 0.546925549187508
10.0800001621246 0.528198039854712
10.1200001239777 0.527649981150819
10.1600000858307 0.527322600214427
10.2000000476837 0.526991248162036
10.2400000095367 0.526655869480012
10.2800002098083 0.526219203610849
10.3200001716614 0.525535906794667
10.3600001335144 0.524873446159609
10.4000000953674 0.524232210708219
10.4400000572205 0.52379603146589
10.4800000190735 0.523388890985208
10.5199999809265 0.523228539236435
10.5600001811981 0.522074161923519
10.6000001430511 0.523476872174834
10.6400001049042 0.523843772231425
10.6800000667572 0.524329791026581
10.7200000286102 0.524862646749479
10.7599999904633 0.525307537003112
10.8000001907349 0.525795678861039
10.8400001525879 0.526465323014089
10.8800001144409 0.5270822559427
10.9200000762939 0.527743459111048
10.960000038147 0.528342234104429
11 0.529106323462411
11.0400002002716 0.529943259380987
11.0800001621246 0.530677554474344
11.1200001239777 0.531353981030273
11.1600000858307 0.531978471659719
11.2000000476837 0.556058809265848
11.2400000095367 0.532767539329965
11.2800002098083 0.533289424039148
11.3200001716614 0.533420410755758
11.3600001335144 0.53343390396562
11.4000000953674 0.533166489564569
11.4400000572205 0.53282348143712
11.4800000190735 0.532222288679252
11.5199999809265 0.531556167825498
11.5600001811981 0.530914673708877
11.6000001430511 0.530420060561538
11.6400001049042 0.53000436711532
11.6800000667572 0.529491574670342
11.7200000286102 0.528915624034234
11.7599999904633 0.528229023271985
11.8000001907349 0.527339178792165
11.8400001525879 0.526501261177906
11.8800001144409 0.525722864105031
11.9200000762939 0.525131697926006
11.960000038147 0.524588892477429
12 0.524180265420195
12.0400002002716 0.523815647573118
12.0800001621246 0.523450756560656
12.1200001239777 0.52308537792826
12.1600000858307 0.522707699238823
12.2000000476837 0.521878218714596
12.2400000095367 0.52083772955509
12.2800002098083 0.519775765035912
12.3200001716614 0.518685183826015
12.3600001335144 0.517508660186145
12.4000000953674 0.516022477573725
12.4400000572205 0.514687972903787
12.4800000190735 0.513180820974952
12.5199999809265 0.511621727857366
12.5600001811981 0.50984081295053
12.6000001430511 0.508042261108313
12.6400001049042 0.506485160708548
12.6800000667572 0.505151476398292
12.7200000286102 0.50398560151616
12.7599999904633 0.502900094139036
12.8000001907349 0.501858226938522
12.8400001525879 0.500699467857341
12.8800001144409 0.499148647825328
12.9200000762939 0.497619793582047
12.960000038147 0.495909619084141
13 0.494268943633728
13.0400002002716 0.492741828244786
13.0800001621246 0.49165753844967
13.1200001239777 0.490792516858827
13.1600000858307 0.490023735760005
13.2000000476837 0.489298967132004
13.2400000095367 0.48868297660046
13.2800002098083 0.488183034697325
13.3200001716614 0.487695473764414
13.3600001335144 0.487182328969994
13.4000000953674 0.48654488467549
13.4400000572205 0.486022555102209
13.4800000190735 0.486017108516137
13.5199999809265 0.486094350506732
13.5600001811981 0.486137686621905
13.6000001430511 0.486160373873264
13.6400001049042 0.486690007700896
13.6800000667572 0.487292165410659
13.7200000286102 0.487780886075753
13.7599999904633 0.488203604079186
13.8000001907349 0.488562627630611
13.8400001525879 0.488902699141377
13.8800001144409 0.489298639127612
13.9200000762939 0.489870930744165
13.960000038147 0.491682206415451
14 0.493587081169375
14.0400002002716 0.496592802324463
14.0800001621246 0.49954088333878
14.1200001239777 0.502411334710166
14.1600000858307 0.506105119221799
14.2000000476837 0.509709319967081
14.2400000095367 0.513127933168859
14.2800002098083 0.516333719423099
14.3200001716614 0.518936026250591
14.3600001335144 0.520995788345029
14.4000000953674 0.522952184863815
14.4400000572205 0.524931602565191
14.4800000190735 0.526757716239426
14.5199999809265 0.528805257840063
14.5600001811981 0.530696373103217
14.6000001430511 0.532390414721763
14.6400001049042 0.533616161955578
14.6800000667572 0.534678589127438
14.7200000286102 0.535406223392801
14.7599999904633 0.53584347725303
14.8000001907349 0.535943438390238
14.8400001525879 0.535866306529692
14.8800001144409 0.535504461360984
14.9200000762939 0.534864337103298
14.960000038147 0.534244244119427
15 0.53358052270533
15.0400002002716 0.532662227247694
15.0800001621246 0.531738019324988
15.1200001239777 0.530434322229276
15.1600000858307 0.529142164816211
15.2000000476837 0.527663568018939
15.2400000095367 0.526138424522874
15.2800002098083 0.523246916825231
15.3200001716614 0.520113195943936
15.3600001335144 0.517287435514134
15.4000000953674 0.514341723696098
15.4400000572205 0.51126924707468
15.4800000190735 0.508254271999453
15.5199999809265 0.505134135600738
15.5600001811981 0.502120773746518
15.6000001430511 0.49862134214132
15.6400001049042 0.495442587804676
15.6800000667572 0.492183659642341
15.7200000286102 0.489008688337813
15.7599999904633 0.486009476626005
15.8000001907349 0.483660649987539
15.8400001525879 0.481778747344605
15.8800001144409 0.48066107170182
15.9200000762939 0.480071437655686
15.960000038147 0.479825612237212
16 0.480141651402282
16.0400002002716 0.480394904656874
16.0800001621246 0.480661449074243
16.1200001239777 0.481472748047967
16.1600000858307 0.482287365537429
16.2000000476837 0.483615283879024
16.2400000095367 0.485002245885759
16.2800002098083 0.486449529825051
16.3200001716614 0.487852374950846
16.3600001335144 0.489005882366482
16.4000000953674 0.490540532725273
16.4400000572205 0.492189417051856
16.4800000190735 0.494046173560887
16.5199999809265 0.496147170371491
16.5600001811981 0.498259383549096
16.6000001430511 0.500697735436852
16.6400001049042 0.502930950820535
16.6800000667572 0.505448750417994
16.7200000286102 0.507832695404942
16.7599999904633 0.509973078810582
16.8000001907349 0.512532090971624
16.8400001525879 0.515229977206515
16.8800001144409 0.518323336216702
16.9200000762939 0.520793208292698
16.960000038147 0.517559422023902
17 0.513564399113204
17.0400002002716 0.509928612609625
17.0800001621246 0.505680656701469
17.1200001239777 0.499786561326895
17.1600000858307 0.493997216512978
17.2000000476837 0.486480004349246
17.2400000095367 0.479761014212639
17.2800002098083 0.471524304978844
17.3200001716614 0.463148070474397
17.3600001335144 0.449536390673046
17.4000000953674 0.435141119494483
17.4400000572205 0.422874155350239
17.4800000190735 0.409859041604239
17.5199999809265 0.396514955528463
17.5600001811981 0.383018928023737
17.6000001430511 0.368273292576262
17.6400001049042 0.352853623448394
17.6800000667572 0.333090939760707
17.7200000286102 0.314945774265597
17.7599999904633 0.292931715366997
17.8000001907349 0.272382267142471
17.8400001525879 0.249909796719576
17.8800001144409 0.227007533258429
17.9200000762939 0.205556134603106
17.960000038147 0.183123015003599
18 0.161866119933623
18.0400002002716 0.136477825554885
18.0800001621246 0.10859364002154
18.1200001239777 0.0820231452406671
18.1600000858307 0.0551840037930995
18.2000000476837 0.0281780294691293
18.2400000095367 -0.00166156217087476
18.2800002098083 -0.0294411269390502
18.3200001716614 -0.0647262780977721
18.3600001335144 -0.100047232038578
18.4000000953674 -0.133356194565386
18.4400000572205 -0.168643587611325
18.4800000190735 -0.20022827116965
18.5199999809265 -0.231791290968456
18.5600001811981 -0.263616250127143
18.6000001430511 -0.295076251382716
18.6400001049042 -0.32779320112527
18.6800000667572 -0.360207325313561
18.7200000286102 -0.40085453669958
18.7599999904633 -0.44070215252467
18.8000001907349 -0.481750767354482
18.8400001525879 -0.520361587622894
18.8800001144409 -0.555477954414314
18.9200000762939 -0.594487755733066
18.960000038147 -0.63305777529849
19 -0.676109740801417
19.0400002002716 -0.717576484920324
19.0800001621246 -0.757689262558619
19.1200001239777 -0.796505524920395
19.1600000858307 -0.834101763723671
19.2000000476837 -0.872172195942563
19.2400000095367 -0.913016155435836
19.2800002098083 -0.968049491227328
19.3200001716614 -1.01991097991935
19.3600001335144 -1.06765599933218
19.4000000953674 -1.1180306563882
19.4400000572205 -1.16629438798686
19.4800000190735 -1.21530488378435
19.5199999809265 -1.26282637580694
19.5600001811981 -1.30827650525853
19.6000001430511 -1.3483711088362
19.6400001049042 -1.3877826525578
19.6800000667572 -1.42670953515843
19.7200000286102 -1.46196574961523
19.7599999904633 -1.49793501433718
19.8000001907349 -1.54051604284473
19.8400001525879 -1.5799879611256
19.8800001144409 -1.62026129007031
19.9200000762939 -1.65852213254711
19.960000038147 -1.69807372700624
20 -1.73579085751822
20.0400002002716 -1.78144333065173
20.0800001621246 -1.82696101900612
20.1200001239777 -1.86554507952709
20.1600000858307 -1.90319968209045
20.2000000476837 -1.93972804944744
20.2400000095367 -1.97513656869397
20.2800002098083 -2.01177486442198
20.3200001716614 -2.04732632861467
20.3600001335144 -2.08936807891758
20.4000000953674 -2.12920894768779
20.4400000572205 -2.17014513101829
20.4800000190735 -2.20857703921015
20.5199999809265 -2.24874836858047
20.5600001811981 -2.28833880398407
20.6000001430511 -2.32398723712702
20.6400001049042 -2.35779262868662
20.6800000667572 -2.39064231061415
20.7200000286102 -2.42490908481529
20.7599999904633 -2.46052811872392
20.8000001907349 -2.49493405280961
20.8400001525879 -2.52879016170607
20.8800001144409 -2.56051054590224
20.9200000762939 -2.59280653394196
20.960000038147 -2.6222437538717
21 -2.6493792230323
21.0400002002716 -2.68040269635519
21.0800001621246 -2.71216477842703
21.1200001239777 -2.7442752944313
21.1600000858307 -2.77340037722636
21.2000000476837 -2.80056889123675
21.2400000095367 -2.82618478350346
21.2800002098083 -2.84921539358482
21.3200001716614 -2.87201483468714
21.3600001335144 -2.89643900154388
21.4000000953674 -2.91935013798375
21.4400000572205 -2.94074032305327
21.4800000190735 -2.96085979618567
21.5199999809265 -2.98073681292527
21.5600001811981 -2.99899311492534
21.6000001430511 -3.01748447248947
21.6400001049042 -3.03606762567086
21.6800000667572 -3.05296805890715
21.7200000286102 -3.06780476079598
21.7599999904633 -3.0814757779189
21.8000001907349 -3.09411753194364
21.8400001525879 -3.10574577956912
21.8800001144409 -3.11647873874857
21.9200000762939 -3.12586667343871
21.960000038147 -3.1343722840652
22 -3.14219651637647
22.0400002002716 -3.14867125157525
22.0800001621246 -3.15464672327943
22.1200001239777 -3.16005534907116
22.1600000858307 -3.16506436122899
22.2000000476837 -3.16960233300277
22.2400000095367 -3.17346310079766
22.2800002098083 -3.17698262871556
22.3200001716614 -3.18022028826922
22.3600001335144 -3.18318574715157
22.4000000953674 -3.18569926685406
22.4400000572205 -3.18734445343163
22.4800000190735 -3.18889711749647
22.5199999809265 -3.19029333838841
22.5600001811981 -3.19115479067629
22.6000001430511 -3.19191818456762
22.6400001049042 -3.19096453474222
22.6800000667572 -3.18954611464741
22.7200000286102 -3.18824063325985
22.7599999904633 -3.18625309557559
22.8000001907349 -3.18421658952843
22.8400001525879 -3.18178541444866
22.8800001144409 -3.17917107832383
22.9200000762939 -3.17612119904749
22.960000038147 -3.17219201616058
23 -3.16857081015958
23.0400002002716 -3.16515725619533
23.0800001621246 -3.16115810188355
23.1200001239777 -3.15739879649366
23.1600000858307 -3.1527030006046
23.2000000476837 -3.14811909138904
23.2400000095367 -3.14373820703735
23.2800002098083 -3.13677307923692
23.3200001716614 -3.12980436707365
23.3600001335144 -3.1230111220286
23.4000000953674 -3.11638853139072
23.4400000572205 -3.11044626452857
23.4800000190735 -3.10465819940002
23.5199999809265 -3.09872369727516
23.5600001811981 -3.09313368842693
23.6000001430511 -3.08761979197185
23.6400001049042 -3.08266926256685
23.6800000667572 -3.07768587405984
23.7200000286102 -3.07279755586332
23.7599999904633 -3.06819058357103
23.8000001907349 -3.06385226643817
23.8400001525879 -3.06017879579008
23.8800001144409 -3.05650434515481
23.9200000762939 -3.05283166234428
23.960000038147 -3.04953571408226
24 -3.04625645159661
24.0400002002716 -3.04241159189465
24.0800001621246 -3.03884764818621
24.1200001239777 -3.03541160452394
24.1600000858307 -3.03227214109481
24.2000000476837 -3.02930256644047
24.2400000095367 -3.02656384632114
24.2800002098083 -3.02398035216542
24.3200001716614 -3.02168990780746
24.3600001335144 -3.01999081570264
24.4000000953674 -3.01861795719642
24.4400000572205 -3.01744409467171
24.4800000190735 -3.01725621068551
24.5199999809265 -3.01859008113163
24.5600001811981 -3.01993128343023
24.6000001430511 -3.02102215142922
24.6400001049042 -3.02229521646069
24.6800000667572 -3.0235127758207
24.7200000286102 -3.0250542217152
24.7599999904633 -3.02663353472275
24.8000001907349 -3.02813112776014
24.8400001525879 -3.03131457513377
24.8800001144409 -3.03476407440966
24.9200000762939 -3.03863570358227
24.960000038147 -3.04284474806492
25 -3.04722629528103
25.0400002002716 -3.05212834414081
25.0800001621246 -3.05671280009892
25.1200001239777 -3.06101180433907
25.1600000858307 -3.06494335731499
25.2000000476837 -3.06861409720777
25.2400000095367 -3.07206865959385
25.2800002098083 -3.07538054795567
25.3200001716614 -3.07922833770367
25.3600001335144 -3.0830691049182
25.4000000953674 -3.08645949521743
25.4400000572205 -3.08964065330216
25.4800000190735 -3.09268463359324
25.5199999809265 -3.09573090250841
25.5600001811981 -3.09879765826611
25.6000001430511 -3.10155757056799
25.6400001049042 -3.10380737601632
25.6800000667572 -3.10588661248871
25.7200000286102 -3.1076445877
25.7599999904633 -3.10926472539818
25.8000001907349 -3.1107860064564
25.8400001525879 -3.11151635221304
25.8800001144409 -3.11193606990889
25.9200000762939 -3.11123573900057
25.960000038147 -3.11012607009645
26 -3.10921695826847
26.0400002002716 -3.10813323274568
26.0800001621246 -3.10695589521532
26.1200001239777 -3.10585963500949
26.1600000858307 -3.1047741822136
26.2000000476837 -3.10367425916209
26.2400000095367 -3.10199112543039
26.2800002098083 -3.10040470990576
26.3200001716614 -3.09842523185473
26.3600001335144 -3.09660275143389
26.4000000953674 -3.09491983696476
26.4400000572205 -3.09210404374657
26.4800000190735 -3.08904975209152
26.5199999809265 -3.08598222497896
26.5600001811981 -3.08295351500312
26.6000001430511 -3.07989391487829
26.6400001049042 -3.0768073492051
26.6800000667572 -3.07398176848043
26.7200000286102 -3.07142041446374
26.7599999904633 -3.06884085519035
26.8000001907349 -3.0664312114736
26.8400001525879 -3.06443046921649
26.8800001144409 -3.06259046814647
26.9200000762939 -3.06088428597908
26.960000038147 -3.05927089701911
27 -3.057863201863
27.0400002002716 -3.05614577467182
27.0800001621246 -3.05434368041142
27.1200001239777 -3.05255039938368
27.1600000858307 -3.05076923499717
27.2000000476837 -3.04912681307888
27.2400000095367 -3.04752941532641
27.2800002098083 -3.0461495349667
27.3200001716614 -3.04502254463527
27.3600001335144 -3.04519891389882
27.4000000953674 -3.04554464894153
27.4400000572205 -3.0467402093181
27.4800000190735 -3.04819348534165
27.5199999809265 -3.05020248015308
27.5600001811981 -3.05228497304395
27.6000001430511 -3.05413729221519
27.6400001049042 -3.05603369881925
27.6800000667572 -3.05795171264271
27.7200000286102 -3.05980518403021
27.7599999904633 -3.0616272723732
27.8000001907349 -3.06332656566938
27.8400001525879 -3.06506498571582
27.8800001144409 -3.06670845008976
27.9200000762939 -3.06819324484233
27.960000038147 -3.07142120880214
28 -3.0747425711339
28.0400002002716 -3.07892630048381
28.0800001621246 -3.08315042161306
28.1200001239777 -3.08676719839462
28.1600000858307 -3.09083348408661
28.2000000476837 -3.09518268124913
28.2400000095367 -3.09929184238516
28.2800002098083 -3.10331192038176
28.3200001716614 -3.10690328534861
28.3600001335144 -3.11011354009615
28.4000000953674 -3.1131590884315
28.4400000572205 -3.11524478376099
28.4800000190735 -3.11721636773006
28.5199999809265 -3.11916297325803
28.5600001811981 -3.12098541192642
28.6000001430511 -3.12266619658216
28.6400001049042 -3.12413676233627
28.6800000667572 -3.1254747602067
28.7200000286102 -3.12691185707381
28.7599999904633 -3.12846578872467
28.8000001907349 -3.12989501329388
28.8400001525879 -3.13126186887142
28.8800001144409 -3.13267266867584
28.9200000762939 -3.13395580594061
28.960000038147 -3.13471372949924
29 -3.13538757328855
29.0400002002716 -3.13490932103054
29.0800001621246 -3.1341252739791
29.1200001239777 -3.13341849006276
29.1600000858307 -3.13232679834441
29.2000000476837 -3.13122018636376
29.2400000095367 -3.12997079926449
29.2800002098083 -3.12863161921327
29.3200001716614 -3.12766667516988
29.3600001335144 -3.12713027711098
29.4000000953674 -3.1266910232181
29.4400000572205 -3.12629975284805
29.4800000190735 -3.12583140628429
29.5199999809265 -3.12544686715797
29.5600001811981 -3.12390693799057
29.6000001430511 -3.12227947671975
29.6400001049042 -3.11978546323652
29.6800000667572 -3.11714347716993
29.7200000286102 -3.11485986613387
29.7599999904633 -3.11271153060996
29.8000001907349 -3.11064119611469
29.8400001525879 -3.10852454494824
29.8800001144409 -3.10626817302645
29.9200000762939 -3.10426862792158
29.960000038147 -3.102470316615
30 -3.10080797207185
30.0400002002716 -3.09947724489471
30.0800001621246 -3.09852574292678
30.1200001239777 -3.09769334645661
30.1600000858307 -3.09665787860721
30.2000000476837 -3.09564611681417
30.2400000095367 -3.0944041163228
30.2800002098083 -3.09322931651988
30.3200001716614 -3.09221760720683
30.3600001335144 -3.09091134842154
30.4000000953674 -3.08927227027325
30.4400000572205 -3.08761952013195
30.4800000190735 -3.08567962711249
30.5199999809265 -3.08391170246022
30.5600001811981 -3.08215708845827
30.6000001430511 -3.08046221756638
30.6400001049042 -3.07844508410792
30.6800000667572 -3.07642796688965
30.7200000286102 -3.07455139759923
30.7599999904633 -3.07154034670631
30.8000001907349 -3.06842096207526
30.8400001525879 -3.06495137941348
30.8800001144409 -3.06127082902975
30.9200000762939 -3.05795891665044
30.960000038147 -3.05451936904564
31 -3.05032685485218
31.0400002002716 -3.0458846782397
31.0800001621246 -3.03942147273481
31.1200001239777 -3.03337294953614
31.1600000858307 -3.02621415403253
31.2000000476837 -3.01921699007315
31.2400000095367 -3.01257162831714
31.2800002098083 -3.00489113274947
31.3200001716614 -2.99745877256903
31.3600001335144 -2.98760299835945
31.4000000953674 -2.97698141416926
31.4400000572205 -2.96584412151992
31.4800000190735 -2.95404659434919
31.5199999809265 -2.94226596984242
31.5600001811981 -2.92936315032436
31.6000001430511 -2.91695392775296
31.6400001049042 -2.89905599374622
31.6800000667572 -2.88168217302807
31.7200000286102 -2.86509627952803
31.7599999904633 -2.84596053977222
31.8000001907349 -2.82661375479393
31.8400001525879 -2.80698480522511
31.8800001144409 -2.78839463494095
31.9200000762939 -2.76614228427842
31.960000038147 -2.74142163191376
32 -2.71894206978041
32.0400002002716 -2.69695768531981
32.0800001621246 -2.67326320789268
32.1200001239777 -2.65080879597698
32.1600000858307 -2.62572252576284
32.2000000476837 -2.6020264114522
32.2400000095367 -2.56920242720012
32.2800002098083 -2.53735767543013
32.3200001716614 -2.50207906979168
32.3600001335144 -2.46802778932719
32.4000000953674 -2.43880097392883
32.4400000572205 -2.40988958326224
32.4800000190735 -2.38184194627937
32.5199999809265 -2.35352341977618
32.5600001811981 -2.32471803490088
32.6000001430511 -2.29536513174775
32.6400001049042 -2.260115290012
32.6800000667572 -2.23197929838065
32.7200000286102 -2.19604967731882
32.7599999904633 -2.15758113755654
32.8000001907349 -2.12269741627009
32.8400001525879 -2.08866960478418
32.8800001144409 -2.05553259784881
32.9200000762939 -2.01525945631235
32.960000038147 -1.97671398049917
33 -1.93924104769494
33.0400002002716 -1.90277044243617
33.0800001621246 -1.86712145972105
33.1200001239777 -1.83278821936665
33.1600000858307 -1.79745270023536
33.2000000476837 -1.76215683394709
33.2400000095367 -1.7274161989365
33.2800002098083 -1.68559335896546
33.3200001716614 -1.64719181793382
33.3600001335144 -1.59536739468671
33.4000000953674 -1.54347700936637
33.4400000572205 -1.49439716594584
33.4800000190735 -1.44719190199054
33.5199999809265 -1.40686581037722
33.5600001811981 -1.3645624196192
33.6000001430511 -1.32289170731313
33.6400001049042 -1.28580426174661
33.6800000667572 -1.24778905473918
33.7200000286102 -1.20982473069319
33.7599999904633 -1.17062406807163
33.8000001907349 -1.1339732081369
33.8400001525879 -1.09528476817552
33.8800001144409 -1.05709751020293
33.9200000762939 -1.02131671136402
33.960000038147 -0.951625583804719
34 -0.943213589384272
34.0400002002716 -0.905293540085858
34.0800001621246 -0.869052156940205
34.1200001239777 -0.834685359899977
34.1600000858307 -0.800359213922397
};
\node at (axis cs:13.5,-2)[
  anchor=north,
  text=black,
  rotate=0.0
]{ Pickup truck};
\node at (axis cs:17,2.2)[
  anchor=north,
  text=black,
  rotate=0.0
]{ Station wagon};
\end{axis}

\end{tikzpicture}}
%	\caption{States of the dynamic environment.}
%	\label{fig:example dynamic environment}
%\end{figure}

%\begin{figure*}
%	\centering
%	\setlength\figureheight{170pt}
%	\setlength\figurewidth{520pt}
%	% This file was created by matplotlib2tikz v0.6.14.
\begin{tikzpicture}

\begin{axis}[
xlabel={$t$ [s]},
xmin=-7, xmax=42.325,
ymin=0, ymax=6,
width=\figurewidth,
height=\figureheight,
tick align=outside,
tick pos=left,
x grid style={white!69.019607843137251!black},
clip marker paths,
ytick=\empty,
xtick={0,5,10,15,20,25,30,35,40},
y axis line style={draw opacity=0}
]
\path [draw=white!80.0!black, fill=white!80.0!black] (axis cs:0,0)
--(axis cs:42.325,0)
--(axis cs:42.325,1)
--(axis cs:0,1)
--cycle;

\path [draw=white!70.0!black, fill=white!70.0!black] (axis cs:0,1)
--(axis cs:42.325,1)
--(axis cs:42.325,2)
--(axis cs:0,2)
--cycle;

\path [draw=white!80.0!black, fill=white!80.0!black] (axis cs:0,2)
--(axis cs:17.7200000286102,2)
--(axis cs:17.7200000286102,3)
--(axis cs:0,3)
--cycle;

\path [draw=white!70.0!black, fill=white!70.0!black] (axis cs:0,3)
--(axis cs:17.7200000286102,3)
--(axis cs:17.7200000286102,4)
--(axis cs:0,4)
--cycle;

\path [draw=white!80.0!black, fill=white!80.0!black] (axis cs:0.800000190734863,4)
--(axis cs:34.1600000858307,4)
--(axis cs:34.1600000858307,5)
--(axis cs:0.800000190734863,5)
--cycle;

\path [draw=white!70.0!black, fill=white!70.0!black] (axis cs:0.800000190734863,5)
--(axis cs:34.1600000858307,5)
--(axis cs:34.1600000858307,6)
--(axis cs:0.800000190734863,6)
--cycle;

\addplot [semithick, black, forget plot]
table {%
0 0
0 1
};
\addplot [semithick, black, forget plot]
table {%
9 0
9 1
};
\addplot [semithick, black, forget plot]
table {%
11 0
11 1
};
\addplot [semithick, black, forget plot]
table {%
18 0
18 1
};
\addplot [semithick, black, forget plot]
table {%
23 0
23 1
};
\addplot [semithick, black, forget plot]
table {%
0 1
0 2
};
\addplot [semithick, black, forget plot]
table {%
17 1
17 2
};
\addplot [semithick, black, forget plot]
table {%
23 1
23 2
};
\addplot [semithick, black, forget plot]
table {%
31.25 1
31.25 2
};
\addplot [semithick, black, forget plot]
table {%
36.25 1
36.25 2
};
\addplot [semithick, black, forget plot]
table {%
42.325 0
42.325 2
};
\addplot [semithick, black, forget plot]
table {%
0 2
0 3
};
\addplot [semithick, black, forget plot]
table {%
0 3
0 4
};
\addplot [semithick, black, forget plot]
table {%
17.7200000286102 2
17.7200000286102 3
};
\addplot [semithick, black, forget plot]
table {%
17.7200000286102 3
17.7200000286102 4
};
\addplot [semithick, black, forget plot]
table {%
13.2000000476837 2
13.2000000476837 3
};
\addplot [semithick, black, forget plot]
table {%
9.20000004768372 3
9.20000004768372 4
};
\addplot [semithick, black, forget plot]
table {%
11.8000001907349 3
11.8000001907349 4
};
\addplot [semithick, black, forget plot]
table {%
0.800000190734863 4
0.800000190734863 5
};
\addplot [semithick, black, forget plot]
table {%
0.800000190734863 5
0.800000190734863 6
};
\addplot [semithick, black, forget plot]
table {%
34.1600000858307 4
34.1600000858307 5
};
\addplot [semithick, black, forget plot]
table {%
34.1600000858307 5
34.1600000858307 6
};
\addplot [semithick, black, forget plot]
table {%
0 1
42.325 1
};
\addplot [semithick, black, forget plot]
table {%
0 2
42.325 2
};
\addplot [semithick, black, forget plot]
table {%
0 3
42.325 3
};
\addplot [semithick, black, forget plot]
table {%
0 4
42.325 4
};
\addplot [semithick, black, forget plot]
table {%
0 5
42.325 5
};
\node at (axis cs:4.4875,0.5)[
  scale=0.75,
  text=black,
  rotate=0.0
]{ Accelerating};
\node at (axis cs:9.9875,0.5)[
  scale=0.75,
  text=black,
  rotate=0.0,
  align=center
]{ Bra-\\
king};
\node at (axis cs:14.4875,0.5)[
  scale=0.75,
  text=black,
  rotate=0.0
]{ Cruising};
\node at (axis cs:20.4875,0.5)[
  scale=0.75,
  text=black,
  rotate=0.0
]{ Accelerating};
\node at (axis cs:32.6625,0.5)[
  scale=0.75,
  text=black,
  rotate=0.0
]{ Cruising};
\node at (axis cs:8.4875,1.5)[
  scale=0.75,
  text=black,
  rotate=0.0
]{ Straight};
\node at (axis cs:19.9875,1.5)[
  scale=0.75,
  text=black,
  rotate=0.0,
  align=center
]{ Lane\\
Change};
\node at (axis cs:27.1125,1.5)[
  scale=0.75,
  text=black,
  rotate=0.0
]{ Straight};
\node at (axis cs:33.7375,1.5)[
  scale=0.75,
  text=black,
  rotate=0.0,
  align=center
]{ Lane\\
Change};
\node at (axis cs:39.2875,1.5)[
  scale=0.75,
  text=black,
  rotate=0.0
]{ Straight};
\node at (axis cs:6.60000002384186,2.5)[
  scale=0.75,
  text=black,
  rotate=0.0
]{ Accelerating};
\node at (axis cs:15.460000038147,2.5)[
  scale=0.75,
  text=black,
  rotate=0.0
]{ Cruising};
\node at (axis cs:4.60000002384186,3.5)[
  scale=0.75,
  text=black,
  rotate=0.0
]{ Straight};
\node at (axis cs:10.5000001192093,3.5)[
  scale=0.75,
  text=black,
  rotate=0.0,
  align=center
]{ Lane\\
Change};
\node at (axis cs:14.7600001096725,3.5)[
  scale=0.75,
  text=black,
  rotate=0.0
]{ Straight};
\node at (axis cs:17.4800001382828,4.5)[
  scale=0.75,
  text=black,
  rotate=0.0
]{ Cruising};
\node at (axis cs:17.4800001382828,5.5)[
  scale=0.75,
  text=black,
  rotate=0.0
]{ Straight};
\node at (axis cs:-3.5,0.5)[
  scale=0.75,
  text=black,
  rotate=0.0,
  align=center
]{ Ego vehicle,\\
Longitudinal state};
\node at (axis cs:-3.5,1.5)[
  scale=0.75,
  text=black,
  rotate=0.0,
  align=center
]{ Ego vehicle,\\
Lateral state};
\node at (axis cs:-3.5,2.5)[
  scale=0.75,
  text=black,
  rotate=0.0,
  align=center
]{ Station wagon,\\
Longitudinal state};
\node at (axis cs:-3.5,3.5)[
  scale=0.75,
  text=black,
  rotate=0.0,
  align=center
]{ Station wagon,\\
Lateral state};
\node at (axis cs:-3.5,4.5)[
  scale=0.75,
  text=black,
  rotate=0.0,
  align=center
]{ Pickup truck,\\
Longitudinal state};
\node at (axis cs:-3.5,5.5)[
  scale=0.75,
  text=black,
  rotate=0.0,
  align=center
]{ Pickup truck,\\
Lateral state};
\end{axis}

\end{tikzpicture}
%	\caption{Complete overview of the events and activities of the example. Every vertical black line represents an event. The end of the first scenario and the start of the second scenario is at the moment the ego vehicle starts the first lane change, i.e., at $t=17\ \textup{s}$.}
%	\label{fig:example events}
%\end{figure*}
