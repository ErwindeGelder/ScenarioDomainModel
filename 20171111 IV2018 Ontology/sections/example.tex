\section{Application example}
\label{sec:example}

To illustrate the ontology, an real-life example is presented. A schematic overview of the example is shown in Figure~\ref{fig:example schematic}. Three vehicles are present in the example, a pickup truck, a sedan, and a station wagon, see Figure~\ref{fig:example schematic}a from left to right, respectively. The ego vehicle, i.e., the sedan, is displayed in red. The example can be described in words as follows. The ego vehicle accelerate towards its predecessor (i.e., the pickup truck), such that it ends up following it with an approximately constant distance. In the meantime, the car behind the ego vehicle (i.e., the station wagon) accelerates and takes over both other vehicles (i.e., the ego vehicle and the pickup truck). When the station wagon passed both vehicles, the ego vehicle accelerates and performs a lane change after which it overtakes the pickup truck.

\begin{figure}
	\centering
	\setlength\figureheight{100pt}
	\setlength\figurewidth{260pt}
	\subfloat[Last vehicle overtakes other vehicles.]{% This file was created by matplotlib2tikz v0.6.14.
\begin{tikzpicture}

\begin{axis}[
xmin=-20, xmax=20,
ymin=-5, ymax=5,
width=\figurewidth,
height=\figureheight,
tick align=outside,
tick pos=left,
x grid style={lightgray!92.026143790849673!black},
y grid style={lightgray!92.026143790849673!black},
axis background/.style={fill=white!90.0!black},
ticks=none,
hide axis
]
\path [draw=white, fill=white] (axis cs:-20,3.5)
--(axis cs:20,3.5)
--(axis cs:20,-3.5)
--(axis cs:-20,-3.5)
--cycle;

\addplot [semithick, black, forget plot]
table {%
-20 3.5
20 3.5
};
\addplot [semithick, black, forget plot]
table {%
-20 -3.5
20 -3.5
};
\addplot [semithick, black, forget plot]
table {%
-20 0
-15.5555555555556 0
};
\addplot [semithick, black, forget plot]
table {%
-6.66666666666666 0
-2.22222222222222 0
};
\addplot [semithick, black, forget plot]
table {%
6.66666666666667 0
11.1111111111111 0
};
\addplot [red, forget plot]
table {%
7.25 1.75
7.25 1.318
7.12614678899083 1.318
7.0848623853211 1.354
7.0848623853211 1.534
7.12614678899083 1.57
7.25 1.57
7.25 1.282
7.0848623853211 1.066
6.79587155963303 1.21
6.79587155963303 1.75
6.79587155963303 1.21
6.67201834862385 1.138
4.81422018348624 1.138
4.73165137614679 1.282
4.73165137614679 1.75
};
\addplot [red, forget plot]
table {%
7.25 1.75
7.25 2.182
7.12614678899083 2.182
7.0848623853211 2.146
7.0848623853211 1.966
7.12614678899083 1.93
7.25 1.93
7.25 2.218
7.0848623853211 2.434
6.79587155963303 2.29
6.79587155963303 1.75
6.79587155963303 2.29
6.67201834862385 2.362
4.81422018348624 2.362
4.73165137614679 2.218
4.73165137614679 1.75
};
\addplot [red, forget plot]
table {%
4.81422018348624 1.138
4.56651376146789 0.958
4.19495412844037 1.246
4.02981651376147 1.534
4.02981651376147 1.75
};
\addplot [red, forget plot]
table {%
4.81422018348624 2.362
4.56651376146789 2.542
4.19495412844037 2.254
4.02981651376147 1.966
4.02981651376147 1.75
};
\addplot [red, forget plot]
table {%
4.56651376146789 0.958
6.7545871559633 0.886
7.0848623853211 1.066
6.7545871559633 0.886
4.56651376146789 0.958
4.69036697247706 0.85
4.56651376146789 0.958
3.4105504587156 0.922
3.16284403669725 0.994
2.9151376146789 1.246
2.99770642201835 1.282
3.2454128440367 1.138
3.16284403669725 0.994
2.9151376146789 1.246
2.83256880733945 1.39
2.75 1.642
2.75 1.75
};
\addplot [red, forget plot]
table {%
4.56651376146789 2.542
6.7545871559633 2.614
7.0848623853211 2.434
6.7545871559633 2.614
4.56651376146789 2.542
4.69036697247706 2.65
4.56651376146789 2.542
3.4105504587156 2.578
3.16284403669725 2.506
2.9151376146789 2.254
2.99770642201835 2.218
3.2454128440367 2.362
3.16284403669725 2.506
2.9151376146789 2.254
2.83256880733945 2.11
2.75 1.858
2.75 1.75
};
\addplot [blue, forget plot]
table {%
-3.22093023255814 1.75
-3.22093023255814 1.73767123287671
-4.50290697674419 1.73767123287671
-4.50290697674419 1.75
};
\addplot [blue, forget plot]
table {%
-3.22093023255814 1.75
-3.22093023255814 1.76232876712329
-4.50290697674419 1.76232876712329
-4.50290697674419 1.75
};
\addplot [blue, forget plot]
table {%
-3.22093023255814 1.49109589041096
-4.50290697674419 1.49109589041096
-4.50290697674419 1.51575342465753
-3.22093023255814 1.51575342465753
-3.22093023255814 1.49109589041096
};
\addplot [blue, forget plot]
table {%
-3.22093023255814 2.00890410958904
-4.50290697674419 2.00890410958904
-4.50290697674419 1.98424657534247
-3.22093023255814 1.98424657534247
-3.22093023255814 2.00890410958904
};
\addplot [blue, forget plot]
table {%
-3.22093023255814 1.21986301369863
-4.50290697674419 1.21986301369863
-4.50290697674419 1.24452054794521
-3.22093023255814 1.24452054794521
-3.22093023255814 1.21986301369863
};
\addplot [blue, forget plot]
table {%
-3.22093023255814 2.28013698630137
-4.50290697674419 2.28013698630137
-4.50290697674419 2.25547945205479
-3.22093023255814 2.25547945205479
-3.22093023255814 2.28013698630137
};
\addplot [blue, forget plot]
table {%
-2.75 1.75
-2.75 1.07191780821918
-2.82848837209302 0.997945205479452
-2.9593023255814 0.997945205479452
-2.9593023255814 1.75
-2.9593023255814 0.923972602739726
-3.03779069767442 0.85
-5.99418604651163 0.874657534246575
-5.83720930232558 0.726712328767123
-5.99418604651163 0.874657534246575
-6.75290697674419 0.899315068493151
-7.09302325581395 1.02260273972603
-7.25 1.63904109589041
-7.25 1.75
};
\addplot [blue, forget plot]
table {%
-2.75 1.75
-2.75 2.42808219178082
-2.82848837209302 2.50205479452055
-2.9593023255814 2.50205479452055
-2.9593023255814 1.75
-2.9593023255814 2.57602739726027
-3.03779069767442 2.65
-5.99418604651163 2.62534246575343
-5.83720930232558 2.77328767123288
-5.99418604651163 2.62534246575343
-6.75290697674419 2.60068493150685
-7.09302325581395 2.47739726027397
-7.25 1.86095890410959
-7.25 1.75
};
\addplot [blue, forget plot]
table {%
-4.65988372093023 1.75
-4.65988372093023 0.997945205479452
-4.8953488372093 1.09657534246575
-4.8953488372093 1.75
};
\addplot [blue, forget plot]
table {%
-4.65988372093023 1.75
-4.65988372093023 2.50205479452055
-4.8953488372093 2.40342465753425
-4.8953488372093 1.75
};
\addplot [blue, forget plot]
table {%
-5.83720930232558 1.75
-5.83720930232558 1.44178082191781
-5.75872093023256 1.07191780821918
-6.28197674418605 0.973287671232877
-6.36046511627907 1.46643835616438
-6.36046511627907 1.75
};
\addplot [blue, forget plot]
table {%
-5.83720930232558 1.75
-5.83720930232558 2.05821917808219
-5.75872093023256 2.42808219178082
-6.28197674418605 2.52671232876712
-6.36046511627907 2.03356164383562
-6.36046511627907 1.75
};
\addplot [blue, forget plot]
table {%
-3.11627906976744 1.75
-3.11627906976744 1.02260273972603
-4.52906976744186 1.02260273972603
};
\addplot [blue, forget plot]
table {%
-3.11627906976744 1.75
-3.11627906976744 2.47739726027397
-4.52906976744186 2.47739726027397
};
\addplot [blue, forget plot]
table {%
17.25 1.75
17.25 1.40714285714286
17.1590909090909 1.10714285714286
14.5227272727273 1.10714285714286
14.3863636363636 1.40714285714286
14.3409090909091 1.53571428571429
14.3409090909091 1.75
};
\addplot [blue, forget plot]
table {%
17.25 1.75
17.25 2.09285714285714
17.1590909090909 2.39285714285714
14.5227272727273 2.39285714285714
14.3863636363636 2.09285714285714
14.3409090909091 1.96428571428571
14.3409090909091 1.75
};
\addplot [blue, forget plot]
table {%
14.5227272727273 1.10714285714286
14.2954545454545 0.935714285714286
14.0227272727273 0.935714285714286
13.7954545454545 1.15
13.6590909090909 1.49285714285714
13.6590909090909 1.75
};
\addplot [blue, forget plot]
table {%
14.5227272727273 2.39285714285714
14.2954545454545 2.56428571428571
14.0227272727273 2.56428571428571
13.7954545454545 2.35
13.6590909090909 2.00714285714286
13.6590909090909 1.75
};
\addplot [blue, forget plot]
table {%
14.2954545454545 0.935714285714286
15.7954545454545 0.978571428571429
15.7954545454545 1.10714285714286
15.7954545454545 0.978571428571429
17.1136363636364 0.978571428571429
17.1590909090909 1.10714285714286
17.1136363636364 0.978571428571429
16.9772727272727 0.85
13.1590909090909 0.85
12.8409090909091 1.15
13.0227272727273 1.15
13.2045454545455 0.978571428571429
13.2045454545455 0.85
};
\addplot [blue, forget plot]
table {%
14.2954545454545 2.56428571428571
15.7954545454545 2.52142857142857
15.7954545454545 2.39285714285714
15.7954545454545 2.52142857142857
17.1136363636364 2.52142857142857
17.1590909090909 2.39285714285714
17.1136363636364 2.52142857142857
16.9772727272727 2.65
13.1590909090909 2.65
12.8409090909091 2.35
13.0227272727273 2.35
13.2045454545455 2.52142857142857
13.2045454545455 2.65
};
\addplot [blue, forget plot]
table {%
12.8409090909091 1.15
12.75 1.45
12.75 1.75
};
\addplot [blue, forget plot]
table {%
12.8409090909091 2.35
12.75 2.05
12.75 1.75
};
\addplot [semithick, blue, dashed, forget plot]
table {%
12.75 1.75
12.6717171717172 1.74996447297214
12.5934343434343 1.74972010306519
12.5151515151515 1.74906980619819
12.4368686868687 1.74782908517264
12.3585858585859 1.74582576468553
12.280303030303 1.74289972634234
12.2020202020202 1.73890264367003
12.1237373737374 1.73369771713003
12.0454545454545 1.72715940913127
11.9671717171717 1.71917317904316
11.8888888888889 1.70963521820861
11.8106060606061 1.69845218495698
11.7323232323232 1.68554093961714
11.6540404040404 1.67082827953046
11.5757575757576 1.65425067406376
11.4974747474747 1.63575399962237
11.4191919191919 1.6152932746631
11.3409090909091 1.59283239470727
11.2626262626263 1.56834386735364
11.1843434343434 1.54180854729151
11.1060606060606 1.51321537131363
11.0277777777778 1.48256109332927
10.9494949494949 1.44985001937716
10.8712121212121 1.41509374263854
10.7929292929293 1.37831087845013
10.7146464646465 1.33952679931715
10.6363636363636 1.2987733699263
10.5580808080808 1.25608868215878
10.479797979798 1.21151679010327
10.4015151515152 1.16510744506897
10.3232323232323 1.11691583059853
10.2449494949495 1.06700229748114
10.1666666666667 1.01543209876543
10.0883838383838 0.962275124772575
10.010101010101 0.907605638109211
9.93181818181818 0.85150200868048
9.85353535353535 0.794046448703017
9.77525252525253 0.735324747717952
9.6969696969697 0.67542600760391
9.61868686868687 0.614442377590011
9.54040404040404 0.552468789268875
9.46212121212121 0.489602691609613
9.38383838383838 0.425943785970835
9.30555555555556 0.361593761113651
9.22727272727273 0.296656028214665
9.1489898989899 0.231235455878979
9.07070707070707 0.165438105153197
8.99242424242424 0.0993709645384162
8.91414141414141 0.033141685003238
8.83585858585858 -0.0331416850032387
8.75757575757576 -0.0993709645384178
8.67929292929293 -0.165438105153197
8.6010101010101 -0.23123545587898
8.52272727272727 -0.296656028214666
8.44444444444444 -0.361593761113652
8.36616161616162 -0.425943785970836
8.28787878787879 -0.489602691609613
8.20959595959596 -0.552468789268874
8.13131313131313 -0.614442377590012
8.0530303030303 -0.67542600760391
7.97474747474747 -0.735324747717952
7.89646464646465 -0.794046448703018
7.81818181818182 -0.851502008680481
7.73989898989899 -0.907605638109211
7.66161616161616 -0.962275124772574
7.58333333333333 -1.01543209876543
7.50505050505051 -1.06700229748113
7.42676767676768 -1.11691583059853
7.34848484848485 -1.16510744506897
7.27020202020202 -1.21151679010327
7.19191919191919 -1.25608868215878
7.11363636363636 -1.2987733699263
7.03535353535353 -1.33952679931715
6.95707070707071 -1.37831087845013
6.87878787878788 -1.41509374263854
6.80050505050505 -1.44985001937716
6.72222222222222 -1.48256109332927
6.64393939393939 -1.51321537131363
6.56565656565657 -1.54180854729151
6.48737373737374 -1.56834386735365
6.40909090909091 -1.59283239470727
6.33080808080808 -1.6152932746631
6.25252525252525 -1.63575399962237
6.17424242424242 -1.65425067406375
6.0959595959596 -1.67082827953045
6.01767676767677 -1.68554093961715
5.93939393939394 -1.69845218495698
5.86111111111111 -1.7096352182086
5.78282828282828 -1.71917317904316
5.70454545454545 -1.72715940913127
5.62626262626263 -1.73369771713003
5.5479797979798 -1.73890264367003
5.46969696969697 -1.74289972634234
5.39141414141414 -1.74582576468553
5.31313131313131 -1.74782908517264
5.23484848484848 -1.7490698061982
5.15656565656566 -1.74972010306519
5.07828282828283 -1.74996447297215
5 -1.75
-10 -1.75
};
\addplot [semithick, blue, forget plot]
table {%
-10 -1.75
-9.25 -1
};
\addplot [semithick, blue, forget plot]
table {%
-10 -1.75
-9.25 -2.5
};
\addplot [semithick, red, dashed, forget plot]
table {%
2.75 1.75
0.75 1.75
};
\addplot [semithick, red, forget plot]
table {%
1.5 2.5
0.75 1.75
1.5 1
};
\addplot [semithick, blue, dashed, forget plot]
table {%
-7.25 1.75
-9.25 1.75
};
\addplot [semithick, blue, forget plot]
table {%
-8.5 2.5
-9.25 1.75
-8.5 1
};
\end{axis}

\end{tikzpicture}}\\
	\subfloat[Another overtaking.]{% This file was created by matplotlib2tikz v0.6.14.
\begin{tikzpicture}

\begin{axis}[
xmin=-20, xmax=20,
ymin=-5, ymax=5,
width=\figurewidth,
height=\figureheight,
tick align=outside,
tick pos=left,
x grid style={lightgray!92.026143790849673!black},
y grid style={lightgray!92.026143790849673!black},
axis background/.style={fill=white!90.0!black},
ticks=none,
hide axis
]
\path [draw=white, fill=white] (axis cs:-20,3.5)
--(axis cs:20,3.5)
--(axis cs:20,-3.5)
--(axis cs:-20,-3.5)
--cycle;

\addplot [semithick, black, forget plot]
table {%
-20 3.5
20 3.5
};
\addplot [semithick, black, forget plot]
table {%
-20 -3.5
20 -3.5
};
\addplot [semithick, black, forget plot]
table {%
-20 0
-15.5555555555556 0
};
\addplot [semithick, black, forget plot]
table {%
-6.66666666666666 0
-2.22222222222222 0
};
\addplot [semithick, black, forget plot]
table {%
6.66666666666667 0
11.1111111111111 0
};
\addplot [red, forget plot]
table {%
7.25 1.75
7.25 1.318
7.12614678899083 1.318
7.0848623853211 1.354
7.0848623853211 1.534
7.12614678899083 1.57
7.25 1.57
7.25 1.282
7.0848623853211 1.066
6.79587155963303 1.21
6.79587155963303 1.75
6.79587155963303 1.21
6.67201834862385 1.138
4.81422018348624 1.138
4.73165137614679 1.282
4.73165137614679 1.75
};
\addplot [red, forget plot]
table {%
7.25 1.75
7.25 2.182
7.12614678899083 2.182
7.0848623853211 2.146
7.0848623853211 1.966
7.12614678899083 1.93
7.25 1.93
7.25 2.218
7.0848623853211 2.434
6.79587155963303 2.29
6.79587155963303 1.75
6.79587155963303 2.29
6.67201834862385 2.362
4.81422018348624 2.362
4.73165137614679 2.218
4.73165137614679 1.75
};
\addplot [red, forget plot]
table {%
4.81422018348624 1.138
4.56651376146789 0.958
4.19495412844037 1.246
4.02981651376147 1.534
4.02981651376147 1.75
};
\addplot [red, forget plot]
table {%
4.81422018348624 2.362
4.56651376146789 2.542
4.19495412844037 2.254
4.02981651376147 1.966
4.02981651376147 1.75
};
\addplot [red, forget plot]
table {%
4.56651376146789 0.958
6.7545871559633 0.886
7.0848623853211 1.066
6.7545871559633 0.886
4.56651376146789 0.958
4.69036697247706 0.85
4.56651376146789 0.958
3.4105504587156 0.922
3.16284403669725 0.994
2.9151376146789 1.246
2.99770642201835 1.282
3.2454128440367 1.138
3.16284403669725 0.994
2.9151376146789 1.246
2.83256880733945 1.39
2.75 1.642
2.75 1.75
};
\addplot [red, forget plot]
table {%
4.56651376146789 2.542
6.7545871559633 2.614
7.0848623853211 2.434
6.7545871559633 2.614
4.56651376146789 2.542
4.69036697247706 2.65
4.56651376146789 2.542
3.4105504587156 2.578
3.16284403669725 2.506
2.9151376146789 2.254
2.99770642201835 2.218
3.2454128440367 2.362
3.16284403669725 2.506
2.9151376146789 2.254
2.83256880733945 2.11
2.75 1.858
2.75 1.75
};
\addplot [blue, forget plot]
table {%
-3.22093023255814 1.75
-3.22093023255814 1.73767123287671
-4.50290697674419 1.73767123287671
-4.50290697674419 1.75
};
\addplot [blue, forget plot]
table {%
-3.22093023255814 1.75
-3.22093023255814 1.76232876712329
-4.50290697674419 1.76232876712329
-4.50290697674419 1.75
};
\addplot [blue, forget plot]
table {%
-3.22093023255814 1.49109589041096
-4.50290697674419 1.49109589041096
-4.50290697674419 1.51575342465753
-3.22093023255814 1.51575342465753
-3.22093023255814 1.49109589041096
};
\addplot [blue, forget plot]
table {%
-3.22093023255814 2.00890410958904
-4.50290697674419 2.00890410958904
-4.50290697674419 1.98424657534247
-3.22093023255814 1.98424657534247
-3.22093023255814 2.00890410958904
};
\addplot [blue, forget plot]
table {%
-3.22093023255814 1.21986301369863
-4.50290697674419 1.21986301369863
-4.50290697674419 1.24452054794521
-3.22093023255814 1.24452054794521
-3.22093023255814 1.21986301369863
};
\addplot [blue, forget plot]
table {%
-3.22093023255814 2.28013698630137
-4.50290697674419 2.28013698630137
-4.50290697674419 2.25547945205479
-3.22093023255814 2.25547945205479
-3.22093023255814 2.28013698630137
};
\addplot [blue, forget plot]
table {%
-2.75 1.75
-2.75 1.07191780821918
-2.82848837209302 0.997945205479452
-2.9593023255814 0.997945205479452
-2.9593023255814 1.75
-2.9593023255814 0.923972602739726
-3.03779069767442 0.85
-5.99418604651163 0.874657534246575
-5.83720930232558 0.726712328767123
-5.99418604651163 0.874657534246575
-6.75290697674419 0.899315068493151
-7.09302325581395 1.02260273972603
-7.25 1.63904109589041
-7.25 1.75
};
\addplot [blue, forget plot]
table {%
-2.75 1.75
-2.75 2.42808219178082
-2.82848837209302 2.50205479452055
-2.9593023255814 2.50205479452055
-2.9593023255814 1.75
-2.9593023255814 2.57602739726027
-3.03779069767442 2.65
-5.99418604651163 2.62534246575343
-5.83720930232558 2.77328767123288
-5.99418604651163 2.62534246575343
-6.75290697674419 2.60068493150685
-7.09302325581395 2.47739726027397
-7.25 1.86095890410959
-7.25 1.75
};
\addplot [blue, forget plot]
table {%
-4.65988372093023 1.75
-4.65988372093023 0.997945205479452
-4.8953488372093 1.09657534246575
-4.8953488372093 1.75
};
\addplot [blue, forget plot]
table {%
-4.65988372093023 1.75
-4.65988372093023 2.50205479452055
-4.8953488372093 2.40342465753425
-4.8953488372093 1.75
};
\addplot [blue, forget plot]
table {%
-5.83720930232558 1.75
-5.83720930232558 1.44178082191781
-5.75872093023256 1.07191780821918
-6.28197674418605 0.973287671232877
-6.36046511627907 1.46643835616438
-6.36046511627907 1.75
};
\addplot [blue, forget plot]
table {%
-5.83720930232558 1.75
-5.83720930232558 2.05821917808219
-5.75872093023256 2.42808219178082
-6.28197674418605 2.52671232876712
-6.36046511627907 2.03356164383562
-6.36046511627907 1.75
};
\addplot [blue, forget plot]
table {%
-3.11627906976744 1.75
-3.11627906976744 1.02260273972603
-4.52906976744186 1.02260273972603
};
\addplot [blue, forget plot]
table {%
-3.11627906976744 1.75
-3.11627906976744 2.47739726027397
-4.52906976744186 2.47739726027397
};
\addplot [blue, forget plot]
table {%
-12.75 -1.75
-12.75 -2.09285714285714
-12.8409090909091 -2.39285714285714
-15.4772727272727 -2.39285714285714
-15.6136363636364 -2.09285714285714
-15.6590909090909 -1.96428571428571
-15.6590909090909 -1.75
};
\addplot [blue, forget plot]
table {%
-12.75 -1.75
-12.75 -1.40714285714286
-12.8409090909091 -1.10714285714286
-15.4772727272727 -1.10714285714286
-15.6136363636364 -1.40714285714286
-15.6590909090909 -1.53571428571429
-15.6590909090909 -1.75
};
\addplot [blue, forget plot]
table {%
-15.4772727272727 -2.39285714285714
-15.7045454545455 -2.56428571428571
-15.9772727272727 -2.56428571428571
-16.2045454545455 -2.35
-16.3409090909091 -2.00714285714286
-16.3409090909091 -1.75
};
\addplot [blue, forget plot]
table {%
-15.4772727272727 -1.10714285714286
-15.7045454545455 -0.935714285714286
-15.9772727272727 -0.935714285714286
-16.2045454545455 -1.15
-16.3409090909091 -1.49285714285714
-16.3409090909091 -1.75
};
\addplot [blue, forget plot]
table {%
-15.7045454545455 -2.56428571428571
-14.2045454545455 -2.52142857142857
-14.2045454545455 -2.39285714285714
-14.2045454545455 -2.52142857142857
-12.8863636363636 -2.52142857142857
-12.8409090909091 -2.39285714285714
-12.8863636363636 -2.52142857142857
-13.0227272727273 -2.65
-16.8409090909091 -2.65
-17.1590909090909 -2.35
-16.9772727272727 -2.35
-16.7954545454545 -2.52142857142857
-16.7954545454545 -2.65
};
\addplot [blue, forget plot]
table {%
-15.7045454545455 -0.935714285714286
-14.2045454545455 -0.978571428571429
-14.2045454545455 -1.10714285714286
-14.2045454545455 -0.978571428571429
-12.8863636363636 -0.978571428571429
-12.8409090909091 -1.10714285714286
-12.8863636363636 -0.978571428571429
-13.0227272727273 -0.85
-16.8409090909091 -0.85
-17.1590909090909 -1.15
-16.9772727272727 -1.15
-16.7954545454545 -0.978571428571429
-16.7954545454545 -0.85
};
\addplot [blue, forget plot]
table {%
-17.1590909090909 -2.35
-17.25 -2.05
-17.25 -1.75
};
\addplot [blue, forget plot]
table {%
-17.1590909090909 -1.15
-17.25 -1.45
-17.25 -1.75
};
\addplot [semithick, red, dashed, forget plot]
table {%
2.75 1.75
2.67171717171717 1.74996447297214
2.59343434343434 1.74972010306519
2.51515151515152 1.74906980619819
2.43686868686869 1.74782908517264
2.35858585858586 1.74582576468553
2.28030303030303 1.74289972634234
2.2020202020202 1.73890264367003
2.12373737373737 1.73369771713003
2.04545454545455 1.72715940913127
1.96717171717172 1.71917317904316
1.88888888888889 1.70963521820861
1.81060606060606 1.69845218495698
1.73232323232323 1.68554093961714
1.6540404040404 1.67082827953046
1.57575757575758 1.65425067406376
1.49747474747475 1.63575399962237
1.41919191919192 1.6152932746631
1.34090909090909 1.59283239470727
1.26262626262626 1.56834386735364
1.18434343434343 1.54180854729151
1.10606060606061 1.51321537131363
1.02777777777778 1.48256109332927
0.949494949494949 1.44985001937716
0.871212121212121 1.41509374263854
0.792929292929293 1.37831087845013
0.714646464646465 1.33952679931715
0.636363636363636 1.2987733699263
0.558080808080808 1.25608868215878
0.47979797979798 1.21151679010327
0.401515151515151 1.16510744506897
0.323232323232323 1.11691583059853
0.244949494949495 1.06700229748114
0.166666666666667 1.01543209876543
0.0883838383838382 0.962275124772575
0.0101010101010099 0.907605638109211
-0.0681818181818183 0.85150200868048
-0.146464646464647 0.794046448703017
-0.224747474747475 0.735324747717952
-0.303030303030303 0.67542600760391
-0.381313131313131 0.614442377590011
-0.45959595959596 0.552468789268875
-0.537878787878788 0.489602691609613
-0.616161616161616 0.425943785970835
-0.694444444444445 0.361593761113651
-0.772727272727273 0.296656028214665
-0.851010101010101 0.231235455878979
-0.92929292929293 0.165438105153197
-1.00757575757576 0.0993709645384162
-1.08585858585859 0.033141685003238
-1.16414141414141 -0.0331416850032387
-1.24242424242424 -0.0993709645384178
-1.32070707070707 -0.165438105153197
-1.3989898989899 -0.23123545587898
-1.47727272727273 -0.296656028214666
-1.55555555555556 -0.361593761113652
-1.63383838383838 -0.425943785970836
-1.71212121212121 -0.489602691609613
-1.79040404040404 -0.552468789268874
-1.86868686868687 -0.614442377590012
-1.9469696969697 -0.67542600760391
-2.02525252525253 -0.735324747717952
-2.10353535353535 -0.794046448703018
-2.18181818181818 -0.851502008680481
-2.26010101010101 -0.907605638109211
-2.33838383838384 -0.962275124772574
-2.41666666666667 -1.01543209876543
-2.49494949494949 -1.06700229748113
-2.57323232323232 -1.11691583059853
-2.65151515151515 -1.16510744506897
-2.72979797979798 -1.21151679010327
-2.80808080808081 -1.25608868215878
-2.88636363636364 -1.2987733699263
-2.96464646464647 -1.33952679931715
-3.04292929292929 -1.37831087845013
-3.12121212121212 -1.41509374263854
-3.19949494949495 -1.44985001937716
-3.27777777777778 -1.48256109332927
-3.35606060606061 -1.51321537131363
-3.43434343434343 -1.54180854729151
-3.51262626262626 -1.56834386735365
-3.59090909090909 -1.59283239470727
-3.66919191919192 -1.6152932746631
-3.74747474747475 -1.63575399962237
-3.82575757575758 -1.65425067406375
-3.9040404040404 -1.67082827953045
-3.98232323232323 -1.68554093961715
-4.06060606060606 -1.69845218495698
-4.13888888888889 -1.7096352182086
-4.21717171717172 -1.71917317904316
-4.29545454545455 -1.72715940913127
-4.37373737373737 -1.73369771713003
-4.4520202020202 -1.73890264367003
-4.53030303030303 -1.74289972634234
-4.60858585858586 -1.74582576468553
-4.68686868686869 -1.74782908517264
-4.76515151515152 -1.7490698061982
-4.84343434343434 -1.74972010306519
-4.92171717171717 -1.74996447297215
-5 -1.75
-11 -1.75
};
\addplot [semithick, red, forget plot]
table {%
-10.25 -1
-11 -1.75
-10.25 -2.5
};
\addplot [semithick, blue, dashed, forget plot]
table {%
-7.25 1.75
-9.25 1.75
};
\addplot [semithick, blue, forget plot]
table {%
-8.5 2.5
-9.25 1.75
-8.5 1
};
\addplot [semithick, blue, dashed, forget plot]
table {%
-17.25 -1.75
-19.25 -1.75
};
\addplot [semithick, blue, forget plot]
table {%
-18.5 -1
-19.25 -1.75
-18.5 -2.5
};
\end{axis}

\end{tikzpicture}}
	\caption{Schematic overview of the traffic scenarios. The red car (the middle car in (a)) is the defined as the ego vehicle. First, the last car overtakes both vehicle (a). Next, the ego vehicle overtakes its predecessor (b).}
	\label{fig:example schematic}
\end{figure}

Two different scenarios can be identified. The first scenario starts at the beginning of the example and ends when the lane change start. When the first scenario ends, the second scenario starts until. The second scenario ends when the ego vehicle completes the overtaking of the pickup truck.

\begin{figure}
	\centering
	\setlength\figureheight{150pt}
	\setlength\figurewidth{248pt}
	\subfloat[Longitudinal velocity ego vehicle.]{% This file was created by matplotlib2tikz v0.6.14.
\begin{tikzpicture}

\definecolor{color0}{rgb}{0.12156862745098,0.466666666666667,0.705882352941177}
\definecolor{color1}{rgb}{1,0.498039215686275,0.0549019607843137}

\begin{axis}[
xlabel={$t$ [s]},
ylabel={Longitudinal speed [km/h]},
xmin=0, xmax=40,
ymin=35, ymax=75,
width=\figurewidth,
height=\figureheight,
tick align=outside,
tick pos=left,
xmajorgrids,
x grid style={white!69.019607843137251!black},
ymajorgrids,
y grid style={white!69.019607843137251!black},
clip marker paths
]
\addplot [semithick, color0, mark=*, mark size=1, mark options={solid}, only marks, forget plot]
table {%
0 37.4401428403431
0.025 37.43996431043
0.05 37.5405880740954
0.075 37.6778728877673
0.1 37.7752498731806
0.125 37.947297130267
0.15 38.0982448203184
0.175 38.1850927617531
0.2 38.2780279942145
0.225 38.3680283893234
0.25 38.5280185061743
0.275 38.6715175936735
0.3 38.735646326606
0.325 38.8407305620651
0.35 38.9451844399066
0.375 39.0746025315269
0.4 39.2238734143283
0.425 39.3318738159309
0.45 39.4515929033386
0.475 39.5713397583653
0.5 39.7198979366809
0.525 39.857654656494
0.55 39.9252401503896
0.575 40.045546607177
0.6 40.1770879125208
0.625 40.2492796093048
0.65 40.334136515428
0.675 40.4223605060126
0.7 40.5341846525105
0.725 40.6913299918757
0.75 40.8088933015721
0.775 40.9053299048099
0.8 41.00402973893
0.825 41.0760292998492
0.85 41.1940432205534
0.875 41.3077606771305
0.9 41.3874812448841
0.925 41.5179665642182
0.95 41.6360342159925
0.975 41.734028838642
1 41.8337834953435
1.025 41.9055837346934
1.05 42.0404560371994
1.075 42.1720032536764
1.1 42.2515078594697
1.125 42.3279265718808
1.15 42.3800800635402
1.175 42.5143154611608
1.2 42.6876253021909
1.225 42.8136257771312
1.25 42.9094030407382
1.275 42.9621593843763
1.3 43.0189202582297
1.325 43.1112414661543
1.35 43.1947722340595
1.375 43.3016956459513
1.4 43.4073985153599
1.425 43.4613987161613
1.45 43.521240558735
1.475 43.5922678559857
1.5 43.7031937252783
1.525 43.8445395749978
1.55 43.9390395170369
1.575 43.9964574853556
1.6 44.0933483123004
1.625 44.21955873256
1.65 44.3801994442414
1.675 44.5047833899699
1.7 44.566540786837
1.725 44.6737601059195
1.75 44.770701919897
1.775 44.8446857578458
1.8 44.9188918714645
1.825 44.9728915608496
1.85 45.0667189688517
1.875 45.1857465641861
1.9 45.3347688823756
1.925 45.445780538885
1.95 45.477767418483
1.975 45.5899440289695
2 45.7020716822103
2.025 45.7378536399392
2.05 45.805842881925
2.075 45.8890953661698
2.1 45.983209401329
2.125 46.1398926475155
2.15 46.272538144231
2.175 46.3568218528836
2.2 46.4229133579704
2.225 46.4589134751927
2.25 46.5240883402058
2.275 46.6031512862001
2.3 46.6992407606259
2.325 46.8422144501083
2.35 46.9534191280526
2.375 47.019383559678
2.4 47.0767304568906
2.425 47.1307306950836
2.45 47.2240863239652
2.475 47.3287553076495
2.5 47.4628496312203
2.525 47.5816148892767
2.55 47.6235723903986
2.575 47.7059115184451
2.6 47.7978169178488
2.625 47.8520449210575
2.65 47.9432891415744
2.675 48.0403459980348
2.7 48.1390102552415
2.725 48.217004921237
2.75 48.2429142853792
2.775 48.319633010916
2.8 48.4398553847838
2.825 48.5478547794312
2.85 48.6792105868379
2.875 48.7694426431567
2.9 48.7799534558937
2.925 48.8851228457686
2.95 49.0344217297474
2.975 49.127253118404
3 49.1761875914911
3.025 49.1759531068561
3.05 49.2879029298287
3.075 49.4289577573252
3.1 49.4970086474854
3.125 49.5492388424953
3.15 49.5803611927873
3.175 49.693311827111
3.2 49.8327080832055
3.225 49.9227084091675
3.25 49.968190603683
3.275 49.9983920299798
3.3 50.1063924213416
3.325 50.222701578948
3.35 50.2801120327383
3.375 50.331602189975
3.4 50.4227356701119
3.425 50.5487361903824
3.45 50.7474882860562
3.475 50.7971960584297
3.5 50.7559012183913
3.525 50.8611725748315
3.55 50.9568341250207
3.575 51.035612269255
3.6 51.1095128809523
3.625 51.1861793929761
3.65 51.2779398889559
3.675 51.3325291083389
3.7 51.3589626015638
3.725 51.4641136298429
3.75 51.5923669874695
3.775 51.6439586083864
3.8 51.6950575553225
3.825 51.7670571241807
3.85 51.867743376261
3.875 51.9652841357822
3.9 52.0632433991985
3.925 52.1669860591707
3.95 52.2250627151082
3.975 52.2760272860995
4 52.3444875486562
4.025 52.4702374315002
4.05 52.5599499079724
4.075 52.5599498841554
4.1 52.6135418839279
4.125 52.6861733479854
4.15 52.7223301749962
4.175 52.759398916919
4.2 52.8313985231674
4.225 52.9212591403127
4.25 52.9935115758849
4.275 53.0981185746318
4.3 53.1881189594989
4.325 53.2967376877398
4.35 53.386738008067
4.375 53.3882036589361
4.4 53.3931299272678
4.425 53.3980561899644
4.45 53.5787034911732
4.475 53.7930221112129
4.5 53.877449579409
4.525 53.96412173262
4.55 54.0361219910955
4.575 54.1006440357069
4.6 54.1366441800421
4.625 54.2336541261238
4.65 54.327590349064
4.675 54.3461431601438
4.7 54.4150195186551
4.725 54.5230199688979
4.75 54.6275274151875
4.775 54.6992667384909
4.8 54.7544987786583
4.825 54.8046345517998
4.85 54.8780840126093
4.875 54.951674205129
4.9 54.9874120185631
4.925 55.0989517542321
4.95 55.2327075666356
4.975 55.35770416905
5 55.4490195855873
5.025 55.4667551387826
5.05 55.5630784796288
5.075 55.6890777123185
5.1 55.7624466656812
5.125 55.8014714954698
5.15 55.8207055905182
5.175 55.8730126836845
5.2 55.9630122169316
5.225 56.0281261441424
5.25 56.0463933003361
5.275 56.1342324248767
5.3 56.2296002534621
5.325 56.2733136563573
5.35 56.3246117695189
5.375 56.4148805827708
5.4 56.5190536512721
5.425 56.5910539176536
5.45 56.7070878889653
5.475 56.7920996150776
5.5 56.7852279877083
5.525 56.8183304726681
5.55 56.9083308223607
5.575 57.0374876434223
5.6 57.1454881080465
5.625 57.168218132892
5.65 57.1802680137966
5.675 57.2882684166073
5.7 57.3952114205372
5.725 57.4312115425834
5.75 57.4858461984894
5.775 57.5395719276308
5.8 57.6210190772231
5.825 57.6943916269583
5.85 57.7633181069556
5.875 57.8519822427408
5.9 57.9777058442677
5.925 58.1074768405993
5.95 58.1663019839616
5.975 58.2347701310876
6 58.3007316705235
6.025 58.3544534518366
6.05 58.4315280282154
6.075 58.5035276427362
6.1 58.5501881207208
6.125 58.5696327132425
6.15 58.6271092278341
6.175 58.7044366856197
6.2 58.7944361714351
6.225 58.8690968769898
6.25 58.8873775914665
6.275 58.938695228133
6.3 58.9926954493027
6.325 59.1116650036734
6.35 59.2324200384071
6.375 59.2869807263217
6.4 59.3799537490738
6.425 59.4879541375027
6.45 59.6073981876099
6.475 59.6950828194742
6.5 59.7579122326932
6.525 59.8142050058052
6.55 59.8502051580456
6.575 59.9240459709409
6.6 59.9960462373215
6.625 60.0829877388286
6.65 60.150375457012
6.675 60.1336168635415
6.7 60.1659033073796
6.725 60.25590364126
6.75 60.3382956052396
6.775 60.3920077710487
6.8 60.4600340615284
6.825 60.5279451801729
6.85 60.6018540023132
6.875 60.6909828221255
6.9 60.7626931566213
6.925 60.8687441987167
6.95 60.962399361411
6.975 61.0235976604566
7 61.0736101986657
7.025 61.1093188414708
7.05 61.1661220863327
7.075 61.2201217742212
7.1 61.3050352855201
7.125 61.3895883021609
7.15 61.4642018070261
7.175 61.5472603510805
7.2 61.6012600231587
7.225 61.6575853814513
7.25 61.6938793673994
7.275 61.7366731995669
7.3 61.7781778316458
7.325 61.8174453545316
7.35 61.8831590743799
7.375 61.9734543480539
7.4 62.0622331317266
7.425 62.1162333133687
7.45 62.2158662794525
7.475 62.3314664809303
7.5 62.43183793168
7.525 62.535691027315
7.55 62.5910555794157
7.575 62.6233045487439
7.6 62.6627691559728
7.625 62.7350680914819
7.65 62.8232890129898
7.675 62.8952893095645
7.7 62.9279399838726
7.725 62.9309234812944
7.75 63.0212238893649
7.775 63.1140557513263
7.8 63.1680559408736
7.825 63.2299950222877
7.85 63.2659951443339
7.875 63.3309878540051
7.9 63.4026856522013
7.925 63.4619791945439
7.95 63.5093139994851
7.975 63.6412748885102
8 63.791192107757
8.025 63.8628876923541
8.05 63.9159634757538
8.075 63.9407124258254
8.1 64.0146883710464
8.125 64.0995574127227
8.15 64.1532515168544
8.175 64.1987625360974
8.2 64.2167624390157
8.225 64.2540498544336
8.25 64.3002062039178
8.275 64.3957881249178
8.3 64.4870878974871
8.325 64.5230876889385
8.35 64.5676430838157
8.375 64.6039510111216
8.4 64.6768756980325
8.425 64.748875986701
8.45 64.8436346004948
8.475 64.9497771245957
8.5 65.0143901258389
8.525 65.091593793126
8.55 65.1524020977151
8.575 65.2387455329103
8.6 65.3149934098203
8.625 65.3513049464533
8.65 65.3859606204344
8.675 65.4039606972677
8.7 65.4828294912576
8.725 65.574318451085
8.75 65.6657391091642
8.775 65.731017503914
8.8 65.7130174587021
8.825 65.727564188074
8.85 65.7815643934314
8.875 65.849081060225
8.9 65.9027669287695
8.925 65.8626460339765
8.95 65.787569775521
8.975 65.6968603144749
9 65.6194437557189
9.025 65.6011309354335
9.05 65.4941357753864
9.075 65.3552878098024
9.1 65.2370886784668
9.125 65.1879335943838
9.15 65.24162256412
9.175 65.2264866544567
9.2 65.1724869665696
9.225 65.1064351316943
9.25 65.057034232462
9.275 65.0742264865203
9.3 65.0422511216257
9.325 64.9522516279064
9.35 64.8699075703824
9.375 64.8162168887036
9.4 64.7148903310581
9.425 64.6172661335099
9.45 64.5043151939642
9.475 64.3716315670094
9.5 64.2639382414043
9.525 64.0937045770665
9.55 63.9559696859885
9.575 63.8162052494388
9.6 63.6917696168598
9.625 63.6020730850762
9.65 63.500460693574
9.675 63.4284604350976
9.7 63.3817278563165
9.725 63.360822223028
9.75 63.2493357244163
9.775 63.1391080241197
9.8 63.1211079551924
9.825 63.0716505847499
9.85 62.9996503341829
9.875 62.9094369636823
9.9 62.8371370880973
9.925 62.7491119155905
9.95 62.6812749625699
9.975 62.698102932251
10 62.7122392812123
10.025 62.5836394668384
10.05 62.4735825406772
10.075 62.4717265447862
10.1 62.4017827364773
10.125 62.3341910795011
10.15 62.2978939631617
10.175 62.2522001290863
10.2 62.2162003586809
10.225 62.1545352627713
10.25 62.1099400481191
10.275 62.1450489158731
10.3 62.1477503634083
10.325 62.0937506926162
10.35 62.0639408379295
10.375 62.0642367642481
10.4 62.0639407991323
10.425 62.0534686226306
10.45 61.9994689518365
10.475 61.9347946619586
10.5 61.8810898740016
10.525 61.8464478014171
10.55 61.8941015318815
10.575 61.8775948671253
10.6 61.7940381358435
10.625 61.7583327271913
10.65 61.7654942943462
10.675 61.8014944242996
10.7 61.7779586415034
10.725 61.7419584892652
10.75 61.7402354941969
10.775 61.7345339458243
10.8 61.716533868992
10.825 61.6946985384773
10.85 61.6766984867888
10.875 61.6682352594377
10.9 61.667941189995
10.925 61.6500880616359
10.95 61.6403928046989
10.975 61.676245992656
11 61.6604016472154
11.025 61.5701079510887
11.05 61.5402610076713
11.075 61.5477174607125
11.1 61.5133977104494
11.125 61.5065673092131
11.15 61.5422738950947
11.175 61.4924336661739
11.2 61.425219624241
11.225 61.4210619561799
11.25 61.4165693077322
11.275 61.4159414780552
11.3 61.3974343003866
11.325 61.3614344760022
11.35 61.3692673955308
11.375 61.4055600461513
11.4 61.3986235384701
11.425 61.3772027920661
11.45 61.358520883654
11.475 61.3439414835249
11.5 61.3442340112233
11.525 61.3174026622738
11.55 61.2835729760352
11.575 61.2921119356945
11.6 61.280600528785
11.625 61.2268926480571
11.65 61.1741642713912
11.675 61.1381641479084
11.7 61.2142178646218
11.725 61.3010113825532
11.75 61.2870269891186
11.775 61.2719415744526
11.8 61.2719415744526
11.825 61.2597980610531
11.85 61.24179800325
11.875 61.2199758951098
11.9 61.2016839769579
11.925 61.1999416496397
11.95 61.1929230943468
11.975 61.174923029232
12 61.1642333482673
12.025 61.1639416678347
12.05 61.1958682834782
12.075 61.2264161056628
12.1 61.2124895478268
12.125 61.1869811981589
12.15 61.1664362616025
12.175 61.1256883072383
12.2 61.0919417283986
12.225 61.0919417514551
12.25 61.112225098732
12.275 61.1479335496527
12.3 61.1639416757071
12.325 61.1629271604948
12.35 61.1269270060915
12.375 61.0974640615395
12.4 61.1151726739227
12.425 61.1279417253855
12.45 61.1279416865905
12.475 61.1124135560862
12.5 61.0767049246905
12.525 61.0954832904215
12.55 61.1531762732287
12.575 61.2036342481922
12.6 61.2359416083904
12.625 61.2362335867999
12.65 61.248646593167
12.675 61.2666466740246
12.7 61.3075489527894
12.725 61.352601489209
12.75 61.4072865975735
12.775 61.4288165546996
12.8 61.3748163503592
12.825 61.3439414913931
12.85 61.3439415144496
12.875 61.3113910721387
12.9 61.2860404612302
12.925 61.3548835232369
12.95 61.3946324003974
12.975 61.3406322039295
13 61.3082338798701
13.025 61.3079415719981
13.05 61.2756568394762
13.075 61.2396570297151
13.1 61.2359415769071
13.125 61.2362335710572
13.15 61.2359415847773
13.175 61.2851869569079
13.2 61.3418966457189
13.225 61.3646515750917
13.25 61.3802341577584
13.275 61.37994148838
13.3 61.4040561243881
13.325 61.4400558633102
13.35 61.4366662699946
13.375 61.4169679781143
13.4 61.3959502587339
13.425 61.3861911467457
13.45 61.4041910628751
13.475 61.4374296740249
13.5 61.4737226467941
13.525 61.4578536027885
13.55 61.4193799907219
13.575 61.3954680575994
13.6 61.3987864878963
13.625 61.435079288065
13.65 61.3868295326684
13.675 61.3127894162413
13.7 61.2879018015893
13.725 61.2777986838921
13.75 61.2960909084479
13.775 61.3186451364314
13.8 61.3366452094206
13.825 61.3439414913931
13.85 61.3439414756526
13.875 61.3442339797379
13.9 61.3548282699174
13.925 61.3908284164525
13.95 61.4159414156452
13.975 61.4159414471328
14 61.4449312557874
14.025 61.4812436375244
14.05 61.5065468982412
14.075 61.5239413217044
14.1 61.5239412986543
14.125 61.5508679433721
14.15 61.5785334337761
14.175 61.5339001146483
14.2 61.4879413792551
14.225 61.4879413792573
14.25 61.4882345743094
14.275 61.4879413247175
14.3 61.5223354249973
14.325 61.5551515928396
14.35 61.5387575582376
14.375 61.5164857717254
14.4 61.4981924958116
14.425 61.5126348468798
14.45 61.5486346403416
14.475 61.5599412883275
14.5 61.5557733713392
14.525 61.5194800376221
14.55 61.5021418879076
14.575 61.538141673495
14.6 61.5599412883254
14.625 61.5602348600659
14.65 61.5599413192499
14.675 61.5655606516869
14.7 61.5835605526322
14.725 61.5426706039213
14.75 61.4756792441307
14.775 61.5286561896096
14.8 61.5797851767
14.825 61.5257854749832
14.85 61.487941356203
14.875 61.4882346209767
14.9 61.5164986431935
14.925 61.5528513119446
14.95 61.5782938200032
14.975 61.5907356457848
15 61.5730292462997
15.025 61.5506406905713
15.05 61.5326405939712
15.075 61.5107094786318
15.1 61.4927094365714
15.125 61.5061024558208
15.15 61.5279659489898
15.175 61.5460981516056
15.2 61.5504131875803
15.225 61.5324131612605
15.25 61.5077162095512
15.275 61.4948733745501
15.3 61.5113918701007
15.325 61.533141341067
15.35 61.5511414303578
15.375 61.5602348443253
15.4 61.5599412725826
15.425 61.55994125684
15.45 61.5553256483591
15.475 61.5373255826857
15.5 61.5242346505642
15.525 61.5239413143907
15.55 61.5369112156693
15.575 61.5549111160561
15.6 61.5775661343914
15.625 61.5925813603528
15.65 61.5746629813412
15.675 61.5599413035073
15.7 61.559941280455
15.725 61.5733097346137
15.75 61.5858959874688
15.775 61.5542340454422
15.8 61.5152794477645
15.825 61.4972796097878
15.85 61.475129090807
15.875 61.4574222044409
15.9 61.5034233059405
15.925 61.5507661252816
15.95 61.5172841250504
15.975 61.4879413635167
16 61.4882345506942
16.025 61.5007967247787
16.05 61.5187967904606
16.075 61.5403899119083
16.1 61.5528621813556
16.125 61.5187068936914
16.15 61.4809192133878
16.175 61.4629191634552
16.2 61.4746765957021
16.225 61.5106767034422
16.25 61.5242347287126
16.275 61.5239413531877
16.3 61.5239413222674
16.325 61.5104616165218
16.35 61.4924615508462
16.375 61.4882345664369
16.4 61.4819670530806
16.425 61.463966987966
16.45 61.462319046736
16.475 61.4803190809303
16.5 61.5023145541818
16.525 61.519736682174
16.55 61.5056566477682
16.575 61.4776111268221
16.6 61.4416109960276
16.625 61.4346683269805
16.65 61.4703752554092
16.675 61.4482901530103
16.7 61.3942904586029
16.725 61.3977901104736
16.75 61.4118313578684
16.775 61.3936900276076
16.8 61.3616094333337
16.825 61.3256095932002
16.85 61.3079415405128
16.875 61.2974442884243
16.9 61.2611521010718
16.925 61.2359416083926
16.95 61.2359415926476
16.975 61.2359415926476
17 61.2362335946701
17.025 61.2180434867383
17.05 61.2000434289374
17.075 61.199941681125
17.1 61.2084698017393
17.125 61.2267616653494
17.15 61.247268866242
17.175 61.2652689004406
17.2 61.2009506404698
17.225 61.135875545454
17.25 61.1711584990936
17.275 61.2076765342568
17.3 61.2256765847568
17.325 61.2087871705726
17.35 61.1617849504501
17.375 61.0812309506828
17.4 61.0376215735046
17.425 61.0736216964331
17.45 61.1300727077019
17.475 61.1840729277829
17.5 61.1655152689196
17.525 61.1360667728696
17.55 61.1707849520893
17.575 61.2077633854788
17.6 61.225763466895
17.625 61.321264505426
17.65 61.4469717762991
17.675 61.4550304413163
17.7 61.4176989369699
17.725 61.3966096154723
17.75 61.4081187931825
17.775 61.4798257112885
17.8 61.5466926253169
17.825 61.5826924423832
17.85 61.6448091389039
17.875 61.7049340065587
17.9 61.7277719961021
17.925 61.7805100187814
17.95 61.8445411039443
17.975 62.0199730447762
18 62.1494487779542
18.025 62.0771527154252
18.05 62.0878005323529
18.075 62.1957999127317
18.1 62.3438285405205
18.125 62.470125789022
18.15 62.511444515013
18.175 62.5314829034927
18.2 62.5519785676202
18.225 62.6101803822453
18.25 62.7184791263603
18.275 62.805337385641
18.3 62.8413375243103
18.325 62.9471787671305
18.35 63.070580302448
18.375 63.1956421163545
18.4 63.3244884023529
18.425 63.3964886723734
18.45 63.5136614661615
18.475 63.601625761677
18.5 63.5922065197024
18.525 63.59499996671
18.55 63.6490001395696
18.575 63.7269397313644
18.6 63.7989400637903
18.625 63.8880880645761
18.65 63.9650670670867
18.675 64.0492230624955
18.7 64.1296008961608
18.725 64.1836010538359
18.75 64.2543426897982
18.775 64.3080361445177
18.8 64.3888134643589
18.825 64.466024059443
18.85 64.5531497477809
18.875 64.6597152146328
18.9 64.7854063728359
18.925 64.9124627322581
18.95 64.9692265516305
18.975 65.0007022182763
19 65.0248096410059
19.025 65.0424995039048
19.05 65.1410267514407
19.075 65.2670260636873
19.1 65.3380832159678
19.125 65.3811771141881
19.15 65.4187198047675
19.175 65.4943005424434
19.2 65.6022999228225
19.225 65.6880949432076
19.25 65.7244082121265
19.275 65.7685457440657
19.3 65.8098773881943
19.325 65.8492685167357
19.35 65.9008048312782
19.375 65.955119209354
19.4 66.0446299653376
19.425 66.1346303083387
19.45 66.2497275609189
19.475 66.3509270677661
19.5 66.3954524153032
19.525 66.4743416235428
19.55 66.5502966458634
19.575 66.6038910688594
19.6 66.652503478808
19.625 66.7068214123619
19.65 66.7939593121259
19.675 66.8839596866164
19.7 66.9525452719872
19.725 66.9911882933904
19.75 67.0708987255093
19.775 67.1544312158321
19.8 67.2084314432227
19.825 67.2296031344849
19.85 67.193602995829
19.875 67.2022921503848
19.9 67.2379716092909
19.925 67.3373405179759
19.95 67.4190878915224
19.975 67.3836834505298
20 67.4283437680351
20.025 67.5720217207008
20.05 67.6274774373864
20.075 67.615011917393
20.1 67.6314699489665
20.125 67.6626275248827
20.15 67.6983047630224
20.175 67.7444518298366
20.2 67.780451591811
20.225 67.8238159719519
20.25 67.8601393810745
20.275 67.8599352973706
20.3 67.8854068924524
20.325 67.9394065947299
20.35 67.9806590951164
20.375 67.9989832549067
20.4 68.0382818498646
20.425 68.0810691170918
20.45 68.1187220163269
20.475 68.1715048063586
20.5 68.2258299762534
20.525 68.2559349021225
20.55 68.2617215867276
20.575 68.2977213959279
20.6 68.3279348421181
20.625 68.3282606549364
20.65 68.3405375449751
20.675 68.3585376027804
20.7 68.3998638557736
20.725 68.4421931490727
20.75 68.4785903338611
20.775 68.507934665115
20.8 68.5079346881652
20.825 68.5441128223883
20.85 68.5981130036833
20.875 68.6488430606418
20.9 68.6818470106122
20.925 68.6672657658382
20.95 68.6730916298897
20.975 68.7270918184902
21 68.7602623132743
21.025 68.7599344438522
21.05 68.7928872300922
21.075 68.8057402283493
21.1 68.7367872640937
21.125 68.7280908805091
21.15 68.7805066537134
21.175 68.7766779084581
21.2 68.7768603927834
21.225 68.830860605003
21.25 68.8789589477332
21.275 68.8966304142208
21.3 68.9039342693062
21.325 68.9050003176257
21.35 68.9410004253667
21.375 68.9874715784171
21.4 69.0231424662829
21.425 69.0386802349343
21.45 69.0206803424179
21.475 69.0119341753642
21.5 69.0122632456854
21.525 69.0474838682066
21.55 69.0834840299257
21.575 69.0839341310988
21.6 69.0929451365117
21.625 69.1112746098392
21.65 69.1365706426522
21.675 69.1489714674007
21.7 69.1323348994119
21.725 69.1470217454601
21.75 69.2013515379294
21.775 69.2563555921569
21.8 69.2923557460016
21.825 69.2641452741372
21.85 69.2430364025593
21.875 69.2971555738249
21.9 69.3190946164658
21.925 69.2830944929851
21.95 69.2767120786706
21.975 69.2947121443511
22 69.2832818832407
22.025 69.2713668696464
22.05 69.3063494087591
22.075 69.2977795421661
22.1 69.2077792064795
22.125 69.1968780334272
22.15 69.2535479232287
22.175 69.2849339916433
22.2 69.2999339207153
22.225 69.2999338971093
22.25 69.338976757745
22.275 69.3926460010056
22.3 69.4079338194628
22.325 69.4146680959552
22.35 69.4506682110099
22.375 69.4958731805108
22.4 69.5315417059282
22.425 69.5519336994566
22.45 69.5519336685278
22.475 69.5519336915871
22.5 69.5522653593693
22.525 69.5519336837119
22.55 69.5519336606616
22.575 69.5519336685319
22.6 69.5633457210136
22.625 69.5816774387308
22.65 69.5549070333326
22.675 69.5189068631825
22.7 69.5159336940387
22.725 69.5296405187758
22.75 69.5492011408978
22.775 69.5711626611059
22.8 69.598777842922
22.825 69.6347776369425
22.85 69.6704783882654
22.875 69.6888106237861
22.9 69.6959335406472
22.925 69.6961525833471
22.95 69.7141524528092
22.975 69.7319334679195
23 69.7322660301956
23.025 69.7140278580572
23.05 69.6780276805934
23.075 69.700084293051
23.1 69.7540845361954
23.125 69.7326725255611
23.15 69.6963400124018
23.175 69.6959335249066
23.2 69.7050492831944
23.225 69.7230493331343
23.25 69.7157355340536
23.275 69.6914466389989
23.3 69.6719769576811
23.325 69.6599335824593
23.35 69.6599335976442
23.375 69.6744598960626
23.4 69.6921276108587
23.425 69.6959335485216
23.45 69.7055617092896
23.475 69.7415618322146
23.5 69.7516816124306
23.525 69.715349138627
23.55 69.7212985696896
23.575 69.757298355279
23.6 69.767933480647
23.625 69.7682661536764
23.65 69.7679334654621
23.675 69.7825864686546
23.7 69.8185862621143
23.725 69.8265036704214
23.75 69.8088366072356
23.775 69.8039334079093
23.8 69.803933400039
23.825 69.8039334157797
23.85 69.8039334393955
23.875 69.8042662923371
23.9 69.8039334545746
23.925 69.8004981291328
23.95 69.7824982051351
23.975 69.7755907754966
24 69.7939235215112
24.025 69.7800199674788
24.05 69.7440198130734
24.075 69.7813111647678
24.1 69.8301948561137
24.125 69.7991497636012
24.15 69.7821652262554
24.175 69.8181653418659
24.2 69.8399334054569
24.225 69.8399333897172
24.25 69.8402664152613
24.275 69.8457177118729
24.3 69.8637177544942
24.325 69.8759333327202
24.35 69.8759333636498
24.375 69.8762665932817
24.4 69.8731976105577
24.425 69.8551975291374
24.45 69.8468256084892
24.475 69.8648256662995
24.5 69.876266538742
24.525 69.8759333484649
24.55 69.8450080465404
24.575 69.8144159197049
24.6 69.845341214202
24.625 69.8624914138262
24.65 69.8261584530086
24.675 69.8143594459079
24.7 69.8323593478416
24.725 69.8100419827714
24.75 69.7827325450416
24.775 69.8302908551099
24.8 69.8851093578905
24.825 69.9031092755362
24.85 69.9284585181284
24.875 69.9446794717191
24.9 69.9278209679232
24.925 69.918242065476
24.95 69.9362419438684
24.975 69.969136881762
25 70.005470580309
25.025 69.9904713435298
25.05 69.9544712021965
25.075 69.9479332890654
25.1 69.9423477575195
25.125 69.9246812401479
25.15 69.921728825865
25.175 69.9397289004627
25.2 69.9336017945888
25.225 69.9151094246621
25.25 69.8937743036403
25.275 69.875933351735
25.3 69.8759333438851
25.325 69.8759333360394
25.35 69.8788594980223
25.375 69.8971927609645
25.4 69.9119333204043
25.425 69.9119333125422
25.45 69.924433872079
25.475 69.9424339466817
25.5 69.9308996679087
25.525 69.915747626268
25.55 69.9331147889601
25.575 69.9479332890654
25.6 69.947933304766
25.625 69.9361501275352
25.65 69.9178166840742
25.675 69.944690113908
25.7 69.9780748043117
25.725 69.9633180775362
25.75 69.9482668227387
25.775 69.9479333283147
25.8 69.9591901567645
25.825 69.9771900351447
25.85 69.9506292470504
25.875 69.9209228999365
25.9 69.9358932945051
25.925 69.9343945276595
25.95 69.9176711231375
25.975 69.9312098688364
26 69.9317662213461
26.025 69.8774330054333
26.05 69.8704405390494
26.075 69.9244401977434
26.1 69.9055402932794
26.125 69.8531175147799
26.15 69.8771770121611
26.175 69.9168508977354
26.2 69.9348508232146
26.225 69.9479332890663
26.25 69.948266861992
26.275 69.9479332969112
26.3 69.9479332890614
26.325 69.9479332812156
26.35 69.9522210388521
26.375 69.9705546628137
26.4 69.9554226969239
26.425 69.9014225045507
26.45 69.9092192016437
26.475 69.9394915074898
26.5 69.9245389597603
26.525 69.9211141218351
26.55 69.9391141885748
26.575 69.9479333126068
26.6 69.9479333126118
26.625 69.9304604187013
26.65 69.9075899424237
26.675 69.8893962522269
26.7 69.8673734936664
26.725 69.8493733955446
26.75 69.8660471993066
26.775 69.9017140031596
26.8 69.9297777255269
26.825 69.9513608895644
26.85 69.9695166336177
26.875 69.984266964072
26.9 69.9839332734304
26.925 69.9839332577348
26.95 69.9839332734304
26.975 69.9839332734354
27 69.9753544775935
27.025 69.9570208928803
27.05 69.9730789566624
27.075 70.0090787291141
27.1 69.9859755138785
27.125 69.9503092907907
27.15 69.947933304757
27.175 69.9403791569283
27.2 69.9223792550069
27.225 69.9247873010107
27.25 69.9431207837037
27.275 69.9479333283024
27.3 69.9479333126068
27.325 69.9479333126068
27.35 69.947933304766
27.375 69.9482668384507
27.4 69.9479332812156
27.425 69.9479332969112
27.45 69.9311949415172
27.475 69.9190548507653
27.5 69.9361267869307
27.525 69.9615960131181
27.55 69.9783067270745
27.575 69.9646440265632
27.6 69.9535157127278
27.625 69.9718493288437
27.65 69.9938060053882
27.675 70.0118060878316
27.7 69.9911744499126
27.725 69.9551742928887
27.75 69.9482668306008
27.775 69.9527118430332
27.8 69.9707119019189
27.825 69.9747392477342
27.85 69.9567391888535
27.875 69.9625336086487
27.9 69.9801999768928
27.925 69.9661335168234
27.95 69.9481335834984
27.975 69.9479332812206
28 69.9393074824037
28.025 69.9209740546425
28.05 69.9286162438088
28.075 69.9408761141107
28.1 69.9241932220901
28.125 69.9215532376237
28.15 69.9392197235992
28.175 69.9618107090318
28.2 69.9798106580523
28.225 69.966110882567
28.25 69.9484444986223
28.275 69.9479332969112
28.3 69.9395682395192
28.325 69.9215683062065
28.35 69.9244302697647
28.375 69.9427637132043
28.4 69.9479333126027
28.425 69.9479333440103
28.45 69.9479332969112
28.475 69.9811248583009
28.5 70.0714587570709
28.525 70.0731505038292
28.55 69.9986115022961
28.575 69.963394636918
28.6 69.9295325863617
28.625 69.8938659787131
28.65 69.8887113144746
28.675 69.9067113733644
28.7 69.8957518276547
28.725 69.8846134210011
28.75 69.9191283414988
28.775 69.9479333047611
28.8 69.9479333047611
28.825 69.9359331242325
28.85 69.9179330418064
28.875 69.9285832891409
28.9 69.9435755017543
28.925 69.9272588783396
28.95 69.9185896710304
28.975 69.9365897770243
29 69.9482668227387
29.025 69.947933304757
29.05 69.9645496203242
29.075 69.9825494987043
29.1 69.9839332734354
29.125 69.984266964072
29.15 69.9826016893816
29.175 69.9646016226297
29.2 69.9479333126068
29.225 69.9479333047783
29.25 69.9582065235776
29.275 69.9758728839637
29.3 69.9551679593737
29.325 69.9197681284476
29.35 69.9305333600659
29.375 69.9532079290624
29.4 69.9708742973066
29.425 69.9567111785417
29.45 69.9027114963064
29.475 69.9043783485062
29.5 69.9407116546426
29.525 69.9123808259377
29.55 69.8844038185702
29.575 69.9199562659982
29.6 69.9572879756773
29.625 69.9756215917932
29.65 69.9839332498849
29.675 69.9778845869768
29.7 69.9598846614976
29.725 69.9718882560195
29.75 70.0082219624163
29.775 69.9913227577398
29.8 69.9553226242687
29.825 69.930009773875
29.85 69.9204664395805
29.875 69.9567235473413
29.9 69.9839332577348
29.925 69.9839332655846
29.95 69.9839332498849
29.975 69.9839332420401
30 69.984266964072
30.025 69.9805515614816
30.05 69.9625514476428
30.075 69.9479332812115
30.1 69.9479333126241
30.125 69.9482668070431
30.15 69.9479332969112
30.175 69.9479333204689
30.2 69.9576129234119
30.225 69.9756129901565
30.25 69.968533848013
30.275 69.9483719104118
30.3 69.9281048772918
30.325 69.9180302263138
30.35 69.9360303009001
30.375 69.9589218574874
30.4 69.9765882100563
30.425 69.9390925667311
30.45 69.8879047057443
30.475 69.9327453810614
30.5 69.9741492342045
30.525 69.9378157396995
30.55 69.9259704281048
30.575 69.9619702084227
30.6 69.9617965037654
30.625 69.9261301001679
30.65 69.9454303031443
30.675 69.9756456152035
30.7 69.9601485500806
30.725 69.9617740186062
30.75 69.9801076190305
30.775 69.9839332420433
30.8 69.9839332734345
30.825 69.9839332969758
30.85 69.9740249576383
30.875 69.9563585972391
30.9 69.9192407829577
30.925 69.8862852568135
30.95 69.9329776569929
30.975 69.9789209052718
31 69.9612545762801
31.025 69.9572464585998
31.05 69.9752465253313
31.075 69.9981617757138
31.1 70.0161618189171
31.125 69.984520328181
31.15 69.9564652074677
31.175 69.9922119532966
31.2 70.0104523717891
31.225 69.9924523364456
31.25 70.0177986124071
31.275 70.0353748438419
31.3 69.9838431523166
31.325 69.9600237319763
31.35 69.9780237987128
31.375 69.9842669797635
31.4 69.9811026085953
31.425 69.963102534014
31.45 69.9550575898492
31.475 69.9730576408941
31.5 69.9726219626263
31.525 69.9542883700468
31.55 69.963649217504
31.575 69.9816491272876
31.6 69.9839332577397
31.625 69.9842669326808
31.65 69.9839332498849
31.675 69.9839332655846
31.7 69.9839332577348
31.725 69.9968446088382
31.75 70.0109582209961
31.775 69.9797133087176
31.8 69.9579303603849
31.825 69.9759302230695
31.85 69.9839332577397
31.875 69.9822638705708
31.9 69.9639302701416
31.925 69.9542176340282
31.95 69.9722175281203
31.975 69.9732179721972
32 69.9555516196446
32.025 69.9630010486676
32.05 69.983592864713
32.075 70.0225247873472
32.1 70.0390700294178
32.125 69.985403880872
32.15 69.9479332812065
32.175 69.9479333125978
32.2 69.9621968022524
32.225 69.9801968611422
32.25 69.9842670190259
32.275 69.9839332812712
32.3 69.9839332420137
32.325 69.9699464728086
32.35 69.9553765572449
32.375 69.9696970720546
32.4 69.9754080495526
32.425 69.9574079671215
32.45 69.9732118035012
32.475 70.0092119448221
32.5 69.9861765617455
32.525 69.9566406544395
32.55 69.9907312137997
32.575 70.0044929745585
32.6 69.9684927939719
32.625 69.9601639796878
32.65 69.9778303400803
32.675 69.9866049725446
32.7 69.9896104643634
32.725 69.9869387273333
32.75 69.9842669671629
32.775 69.9839332541773
32.8 69.9951418871873
32.825 70.0131417866836
32.85 70.0199332179085
32.875 70.0014027050705
32.9 69.9470691048122
32.925 69.9400040322057
32.95 69.9760038783937
32.975 69.9839332851938
33 69.9842669828953
33.025 69.9839332463111
33.05 69.9839332851838
33.075 69.9839332851938
33.1 69.9835995328871
33.125 69.9676553913654
33.15 69.9856552907599
33.175 70.0150273115082
33.2 69.997027386043
33.225 69.9839332184905
33.25 69.9842669640769
33.275 69.983933289121
33.3 69.983933289121
33.325 69.9839332577397
33.35 69.9880138302036
33.375 70.0063476111221
33.4 70.0102441427911
33.425 69.9922440839013
33.45 70.0175727406262
33.475 70.0413691264324
33.5 70.0080633420333
33.525 69.9750015038712
33.55 69.9570014606811
33.575 69.9621444781727
33.6 69.9801445292127
33.625 69.9842669719268
33.65 69.9839332969709
33.675 69.9839332734295
33.7 69.9788383498529
33.725 69.9428382320733
33.75 69.9192658737314
33.775 69.9369323753803
33.8 69.9479332890646
33.825 69.9479332969162
33.85 69.9479332733666
33.875 69.9544301481595
33.9 69.9720965242436
33.925 69.9787525687911
33.95 69.9739055783579
33.975 69.9790863000629
34 69.9842669750291
34.025 69.9839332305787
34.05 69.9945870927654
34.075 70.0125869769876
34.1 70.0056271116968
34.125 69.9891557232544
34.15 70.0031279438658
34.175 70.0199332414972
34.2 70.0199332183669
34.225 70.0018023367214
34.25 69.9661361169823
34.275 69.9750310715748
34.3 70.0110311847706
34.325 69.9659417599736
34.35 69.9201963993067
34.375 69.9565216785971
34.4 69.9934798746739
34.425 70.0114799236348
34.45 70.0199331947682
34.475 70.0137075843753
34.5 69.9960414057308
34.525 69.9959397927354
34.55 70.0139398648366
34.575 70.0034830677845
34.6 69.9854830109574
34.625 69.9842669750291
34.65 69.9979971516681
34.675 70.0339973116028
34.7 70.0318220076464
34.725 69.9958218629857
34.75 69.999885738723
34.775 70.0155266431306
34.8 69.9999079183094
34.825 69.990939118716
34.85 70.0089391672185
34.875 70.0095624699883
34.9 69.9912286770354
34.925 69.9996078430981
34.95 70.0206782335657
34.975 70.041003780981
35 70.0465592568484
35.025 70.0282253222871
35.05 70.0199332100323
35.075 70.016444008825
35.1 69.9804439034955
35.125 69.967149751133
35.15 70.0208158939553
35.175 70.0453930032922
35.2 70.0273931038076
35.225 70.0199332105007
35.25 70.0189132109657
35.275 70.000579472618
35.3 69.9895989685733
35.325 70.0075988602115
35.35 70.0298235272943
35.375 70.0481574618374
35.4 70.0271857573275
35.425 69.9911855973928
35.45 69.9839332541873
35.475 69.9839332773275
35.5 69.9842669671629
35.525 69.9839332541773
35.55 69.9873043608863
35.575 70.0053042446501
35.6 70.0199332179085
35.625 70.0202670960911
35.65 70.032748053944
35.675 70.0507481339113
35.7 70.0204545703294
35.725 69.9914176973929
35.75 70.0272303594905
35.775 70.0328129517542
35.8 69.9788127664573
35.825 69.9601020235735
35.85 69.9781021035409
35.875 69.9842669592884
35.9 69.9817955437097
35.925 69.9637955183557
35.95 69.9620576043138
35.975 69.998057733242
36 70.0089635681092
36.025 69.9906297825641
36.05 69.9839332773176
36.075 69.9778695009885
36.1 69.9598694052887
36.125 69.962394190154
36.15 69.9787173147765
36.175 69.9645899471247
36.2 69.9533318533571
36.225 69.9713319411907
36.25 69.9945333708332
36.275 70.0121995494678
36.3 69.9915091859052
36.325 69.9566344521849
36.35 69.9850584999107
36.375 70.0105297906697
36.4 69.9741960977639
36.425 69.9479332899895
36.45 69.9479333057319
36.475 69.9608607811646
36.5 69.9791943862268
36.525 69.966075873222
36.55 69.9522748021986
36.575 69.9701321595652
36.6 69.9738386984321
36.625 69.9561723707715
36.65 69.9988312536302
36.675 70.0403324538906
36.7 70.0074343818882
36.725 69.9839332615951
36.75 69.9842669750291
36.775 69.9839332541773
36.8 69.978299325538
36.825 69.9422995497156
36.85 69.9260922987554
36.875 69.96242573295
36.9 69.9717614126443
36.925 69.9537613715497
36.95 69.9479332742588
36.975 69.9459823864676
37 69.928315846861
37.025 69.9237813826007
37.05 69.9597815504098
37.075 69.9839332773275
37.1 69.983933246321
37.125 69.9397141286648
37.15 69.8916124009897
37.175 69.9181651285057
37.2 69.9479332821332
37.225 69.9479332821332
37.25 69.9313638440211
37.275 69.9141066208944
37.3 70.0381342253126
37.325 70.0581638727703
37.35 69.9690393013356
37.375 69.9842670020165
37.4 69.9839332711621
37.425 69.9690635767333
37.45 69.9523943933858
37.475 69.9672640480088
37.5 69.975909825885
37.525 69.9395763036541
37.55 69.9119333031608
37.575 69.9119333188131
37.6 69.8976938374357
37.625 69.881409691436
37.65 69.8953157952187
37.675 69.9057820024322
37.7 69.8877821191958
37.725 69.8901381697364
37.75 69.9139851651875
37.775 69.9534468526167
37.8 69.9839332789792
37.825 69.9839332711721
37.85 69.9711312302492
37.875 69.9534649442115
37.9 69.9126102225341
37.925 69.8834239501216
37.95 69.9187470666506
37.975 69.9479332758059
38 69.9482668253917
38.025 69.9479332750756
38.05 69.9479332594414
38.075 69.9314824637571
38.1 69.913482391435
38.125 69.9122666807556
38.15 69.9119333507919
38.175 69.911933358609
38.2 69.9119333266302
38.225 69.9119333429766
38.25 69.9122667205614
38.275 69.9056788321399
38.3 69.887678768347
38.325 69.9040717673664
38.35 69.9374904484051
38.375 69.909685249841
38.4 69.8873546835507
38.425 69.9233548040332
38.45 69.9379764114077
38.475 69.9199763234514
38.5 69.8980347578476
38.525 69.8829567226862
38.55 69.9331887667789
38.575 69.9790431556652
38.6 69.9610430911601
38.625 69.9482668331988
38.65 69.9479332594414
38.675 69.9479332516343
38.7 69.9479332992572
38.725 69.9479333070743
38.75 69.9523319339382
38.775 69.9699983074047
38.8 69.9740194724685
38.825 69.9560195643384
38.85 69.9479332829109
38.875 69.9359219592829
38.9 69.8995886254174
38.925 69.8878262276024
38.95 69.9058261591921
38.975 69.8954924755157
39 69.8833254503911
39.025 69.917432839868
39.05 69.940937597827
39.075 69.9229376904171
39.1 69.8878186534888
39.125 69.8521519340424
39.15 69.8555204480176
39.175 69.871523880562
39.2 69.8559368394984
39.225 69.8461390983124
39.25 69.8644721949417
39.275 69.8759333788689
39.3 69.8759333788788
39.325 69.8759333468901
39.35 69.8792222523483
39.375 69.8975555060646
39.4 69.9217458314974
39.425 69.9397458881953
39.45 69.9305863798312
39.475 69.9125863785915
39.5 69.9122667205515
39.525 69.9211577195997
39.55 69.9391577841146
39.575 69.905221007671
39.6 69.851220806347
39.625 69.8580328738812
39.65 69.8756997413397
39.675 69.8759333312559
39.7 69.8592334285262
39.725 69.8232332838901
39.75 69.8299657194561
39.775 69.8656326748628
39.8 69.8759333312559
39.825 69.8826284783967
39.85 69.9186286543012
39.875 69.9405846772356
39.9 69.9222512195197
39.925 69.9396054607363
39.95 69.9692699407873
39.975 69.9415977910388
40 69.9219796417478
40.025 69.9396461794062
40.05 69.947933290728
40.075 69.9479333148914
40.1 69.9479333227085
40.125 69.9545036369768
40.15 69.9721699862716
40.175 69.962907127799
40.2 69.9269073364225
40.225 69.9119333188049
40.25 69.9134393022791
40.275 69.9311058328326
40.3 69.936933598215
40.325 69.9009337826668
40.35 69.885746903737
40.375 69.904080196547
40.4 69.8976814996408
40.425 69.880412164422
40.45 69.8946639992186
40.475 69.8995294469809
40.5 69.8638626642855
40.525 69.8684511415608
40.55 69.9044512783815
40.575 69.9119333351595
40.6 69.920123304339
40.625 69.938456714422
40.65 69.9479332992473
40.675 69.9479332992572
40.7 69.9479332672685
40.725 69.9479332516343
40.75 69.9482668019304
40.775 69.9479333148815
40.8 69.9479333305256
40.825 69.960043146057
40.85 69.9780431871026
40.875 69.9842669622007
40.9 69.9811626487603
40.925 69.9631625608139
40.95 69.9409240275503
40.975 69.9229239786877
41 69.9122666885628
41.025 69.9119333429666
41.05 69.9119333507937
41.075 69.9058307423169
41.1 69.8878306621858
41.125 69.8903008944803
41.15 69.9053051133064
41.175 69.873270698978
41.2 69.851126749017
41.225 69.8871268851238
41.25 69.8916330074075
41.275 69.8552998548007
41.3 69.8399334147892
41.325 69.8435573688761
41.35 69.8975575467406
41.375 69.9387588232025
41.4 69.9024255206136
41.425 69.8759333867041
41.45 69.8759333710617
41.475 69.9044957251257
41.5 69.9408290589813
41.525 69.9121552937899
41.55 69.8803462754598
41.575 69.8981243806041
41.6 69.9218229046541
41.625 69.9401563154574
41.65 69.9133222247887
41.675 69.871144871356
41.7 69.8517562548453
41.725 69.8520396678644
41.75 69.8703727246879
41.775 69.8759333234288
41.8 69.8785832966608
41.825 69.8965832282505
41.85 69.89792919436
41.875 69.8622623953019
41.9 69.8641561860681
41.925 69.9001563548858
41.95 69.9423410940866
41.975 69.9713072196431
42 69.9052327100012
42.025 69.8469519714576
42.05 69.882952091932
42.075 69.9197104674989
42.1 69.937710532004
42.125 69.9623864742748
42.15 69.9772952774605
42.175 69.9631756051261
42.2 69.9283130617035
42.225 69.8923129405188
42.25 69.8919249839568
42.275 69.9074936356241
42.3 69.8918351877868
42.325 69.8762665204853
};
\addplot [line width=2.4000000000000004pt, color1, dashed, forget plot]
table {%
0 37.6080419972848
0.025 37.7152844154791
0.05 37.822339777017
0.075 37.9292080818984
0.1 38.0358893301232
0.125 38.1423835216916
0.15 38.2486906566035
0.175 38.3548107348589
0.2 38.4607437564578
0.225 38.5664897214002
0.25 38.6720486296862
0.275 38.7774204813156
0.3 38.8826052762886
0.325 38.987603014605
0.35 39.092413696265
0.375 39.1970373212685
0.4 39.3014738896155
0.425 39.405723401306
0.45 39.50978585634
0.475 39.6136612547175
0.5 39.7173495964385
0.525 39.8208508815031
0.55 39.9241651099112
0.575 40.0272922816627
0.6 40.1302323967578
0.625 40.2329854551964
0.65 40.3355514569785
0.675 40.4379304021041
0.7 40.5401222905732
0.725 40.6421271223858
0.75 40.743944897542
0.775 40.8455756160416
0.8 40.9470192778848
0.825 41.0482758830714
0.85 41.1493454316016
0.875 41.2502279234753
0.9 41.3509233586925
0.925 41.4514317372532
0.95 41.5517530591574
0.975 41.6518873244051
1 41.7518345329964
1.025 41.8515946849311
1.05 41.9511677802094
1.075 42.0505538188312
1.1 42.1497528007964
1.125 42.2487647261052
1.15 42.3475895947575
1.175 42.4462274067534
1.2 42.5446781620927
1.225 42.6429418607755
1.25 42.7410185028019
1.275 42.8389080881717
1.3 42.9366106168851
1.325 43.0341260889419
1.35 43.1314545043423
1.375 43.2285958630862
1.4 43.3255501651736
1.425 43.4223174106045
1.45 43.518897599379
1.475 43.6152907314969
1.5 43.7114968069583
1.525 43.8075158257633
1.55 43.9033477879118
1.575 43.9989926934037
1.6 44.0944505422392
1.625 44.1897213344182
1.65 44.2848050699407
1.675 44.3797017488067
1.7 44.4744113710163
1.725 44.5689339365693
1.75 44.6632694454659
1.775 44.7574178977059
1.8 44.8513792932895
1.825 44.9451536322166
1.85 45.0387409144872
1.875 45.1321411401013
1.9 45.2253543090589
1.925 45.31838042136
1.95 45.4112194770046
1.975 45.5038714759928
2 45.5963364183244
2.025 45.6886143039996
2.05 45.7807051330182
2.075 45.8726089053804
2.1 45.9643256210861
2.125 46.0558552801353
2.15 46.147197882528
2.175 46.2383534282642
2.2 46.329321917344
2.225 46.4201033497672
2.25 46.510697725534
2.275 46.6011050446442
2.3 46.691325307098
2.325 46.7813585128953
2.35 46.8712046620361
2.375 46.9608637545204
2.4 47.0503357903482
2.425 47.1396207695195
2.45 47.2287186920344
2.475 47.3176295578927
2.5 47.4063533670946
2.525 47.4948901196399
2.55 47.5832398155288
2.575 47.6714024547612
2.6 47.7593780373371
2.625 47.8471665632565
2.65 47.9347680325194
2.675 48.0221824451259
2.7 48.1094098010758
2.725 48.1964501003692
2.75 48.2833033430062
2.775 48.3699695289867
2.8 48.4564486583106
2.825 48.5427407309781
2.85 48.6288457469891
2.875 48.7147637063436
2.9 48.8004946090417
2.925 48.8860384550832
2.95 48.9713952444682
2.975 49.0565649771968
3 49.1415476532689
3.025 49.2263432726844
3.05 49.3109518354435
3.075 49.3953733415461
3.1 49.4796077909922
3.125 49.5636551837818
3.15 49.6475155199149
3.175 49.7311887993916
3.2 49.8146750222117
3.225 49.8979741883754
3.25 49.9810862978825
3.275 50.0640113507332
3.3 50.1467493469274
3.325 50.2293002864651
3.35 50.3116641693463
3.375 50.393840995571
3.4 50.4758307651392
3.425 50.557633478051
3.45 50.6392491343062
3.475 50.720677733905
3.5 50.8019192768472
3.525 50.882973763133
3.55 50.9638411927623
3.575 51.0445215657351
3.6 51.1250148820514
3.625 51.2053211417112
3.65 51.2854403447146
3.675 51.3653724910614
3.7 51.4451175807517
3.725 51.5246756137856
3.75 51.604046590163
3.775 51.6832305098839
3.8 51.7622273729483
3.825 51.8410371793561
3.85 51.9196599291076
3.875 51.9980956222025
3.9 52.0763442586409
3.925 52.1544058384228
3.95 52.2322803615483
3.975 52.3099678280173
4 52.3874682378297
4.025 52.4647815909857
4.05 52.5419078874852
4.075 52.6188471273282
4.1 52.6955993105147
4.125 52.7721644370448
4.15 52.8485425069183
4.175 52.9247335201353
4.2 53.0007374766959
4.225 53.0765543766
4.25 53.1521842198475
4.275 53.2276270064386
4.3 53.3028827363732
4.325 53.3779514096513
4.35 53.4528330262729
4.375 53.5275275862381
4.4 53.6020350895467
4.425 53.6763555361988
4.45 53.7504889261945
4.475 53.8244352595337
4.5 53.8982111571136
4.525 53.9719163443169
4.55 54.0455674420409
4.575 54.1191644502856
4.6 54.1927073690509
4.625 54.2661961983369
4.65 54.3396309381435
4.675 54.4130115884708
4.7 54.4863381493188
4.725 54.5596106206874
4.75 54.6328290025767
4.775 54.7059932949866
4.8 54.7791034979172
4.825 54.8521596113684
4.85 54.9251616353404
4.875 54.9981095698329
4.9 55.0710034148462
4.925 55.1438431703801
4.95 55.2166288364346
4.975 55.2893604130098
5 55.3620379001057
5.025 55.4346612977223
5.05 55.5072306058594
5.075 55.5797458245173
5.1 55.6522069536958
5.125 55.724613993395
5.15 55.7969669436148
5.175 55.8692658043553
5.2 55.9415105756165
5.225 56.0137012573983
5.25 56.0858378497007
5.275 56.1579203525239
5.3 56.2299487658677
5.325 56.3019230897321
5.35 56.3738433241172
5.375 56.445709469023
5.4 56.5175215244494
5.425 56.5892794903965
5.45 56.6609833668643
5.475 56.7326331538527
5.5 56.8042288513618
5.525 56.8757704593915
5.55 56.9472579779419
5.575 57.0186914070129
5.6 57.0900707466046
5.625 57.161395996717
5.65 57.23266715735
5.675 57.3038842285037
5.7 57.3750472101781
5.725 57.4461561023731
5.75 57.5172109050887
5.775 57.5882116183251
5.8 57.6591582420821
5.825 57.7300507763597
5.85 57.800889221158
5.875 57.871673576477
5.9 57.9424038423166
5.925 58.0130800186769
5.95 58.0837021055578
5.975 58.1542701029594
6 58.2247840108817
6.025 58.2952438293246
6.05 58.3656495582882
6.075 58.4360011977725
6.1 58.5062987477774
6.125 58.5765422083029
6.15 58.6467315793492
6.175 58.716866860916
6.2 58.7869480530036
6.225 58.8569751556118
6.25 58.9269481687407
6.275 58.9968670923902
6.3 59.0667319265604
6.325 59.1365426712512
6.35 59.2062993264627
6.375 59.2760018921949
6.4 59.3456503684477
6.425 59.4152447552212
6.45 59.4847850525153
6.475 59.5542712603301
6.5 59.6237033786656
6.525 59.6930814075217
6.55 59.7624053468985
6.575 59.8316751967959
6.6 59.900890957214
6.625 59.9700526281528
6.65 60.0391602096122
6.675 60.1082137015923
6.7 60.177213104093
6.725 60.2461584171144
6.75 60.3150496406565
6.775 60.3838867747192
6.8 60.4526698193026
6.825 60.5213987744066
6.85 60.5900736400313
6.875 60.6586944161767
6.9 60.7272611028427
6.925 60.7957737000294
6.95 60.8642322077367
6.975 60.9326366259647
7 61.0009869547134
7.025 61.0692831939827
7.05 61.1375253437727
7.075 61.2057134040833
7.1 61.2738473749146
7.125 61.3419272562666
7.15 61.4099530481392
7.175 61.4779247505325
7.2 61.5458423634464
7.225 61.613705886881
7.25 61.6815153208363
7.275 61.7492706653122
7.3 61.8169719203087
7.325 61.884619085826
7.35 61.9522121618639
7.375 62.0197511484224
7.4 62.0872360455017
7.425 62.1546668531015
7.45 62.2220435712221
7.475 62.2893661998633
7.5 62.3566347390251
7.525 62.4238491887076
7.55 62.4910095489108
7.575 62.5581158196346
7.6 62.6251680008791
7.625 62.6921660926443
7.65 62.7591100949301
7.675 62.8260000077366
7.7 62.8928358310637
7.725 62.9596175649115
7.75 63.02634520928
7.775 63.0930187641691
7.8 63.1596382295788
7.825 63.2262036055093
7.85 63.2927148919603
7.875 63.3591720889321
7.9 63.4255751964245
7.925 63.4919242144376
7.95 63.5582191429713
7.975 63.6244599820257
8 63.6906467316008
8.025 63.7567793916965
8.05 63.8228579623129
8.075 63.8888824434499
8.1 63.9548528351076
8.125 64.0207691372859
8.15 64.0866313499849
8.175 64.1524394732046
8.2 64.2181935069449
8.225 64.2838934512059
8.25 64.3495393059876
8.275 64.4151310712899
8.3 64.4806687471128
8.325 64.5461523334565
8.35 64.6115818303207
8.375 64.6769572377057
8.4 64.7422785556113
8.425 64.8075457840376
8.45 64.8727589229845
8.475 64.9379179724521
8.5 65.0030229324403
8.525 65.0680738029492
8.55 65.1330705839788
8.575 65.198013275529
8.6 65.2629018775999
8.625 65.3277363901914
8.65 65.3925168133037
8.675 65.4572431469365
8.7 65.52191539109
8.725 65.5865335457642
8.75 65.6510976109591
8.775 65.7156075866746
8.8 65.7800634729108
8.825 65.8444652696676
8.85 65.9088129769451
8.875 65.9731065947432
8.9 66.037346123062
8.925 66.1015315619015
8.95 66.1656629112616
8.975 66.2297401711424
9 65.9163525027803
9.025 65.816737139492
9.05 65.7180682141708
9.075 65.6203457268167
9.1 65.5235696774297
9.125 65.4277400660097
9.15 65.3328568925569
9.175 65.2389201570712
9.2 65.1459298595526
9.225 65.053886000001
9.25 64.9627885784166
9.275 64.8726375947992
9.3 64.783433049149
9.325 64.6951749414658
9.35 64.6078632717497
9.375 64.5214980400008
9.4 64.4360792462189
9.425 64.3516068904041
9.45 64.2680809725564
9.475 64.1855014926758
9.5 64.1038684507623
9.525 64.0231818468159
9.55 63.9434416808365
9.575 63.8646479528243
9.6 63.7868006627792
9.625 63.7098998107011
9.65 63.6339453965902
9.675 63.5589374204463
9.7 63.4848758822695
9.725 63.4117607820599
9.75 63.3395921198173
9.775 63.2683698955418
9.8 63.1980941092334
9.825 63.1287647608922
9.85 63.0603818505179
9.875 62.9929453781108
9.9 62.9264553436708
9.925 62.8609117471979
9.95 62.7963145886921
9.975 62.7326638681533
10 62.6700920870255
10.025 62.6093942539702
10.05 62.5507028704314
10.075 62.4940179364089
10.1 62.4393394519028
10.125 62.3866674169131
10.15 62.3360018314397
10.175 62.2873426954828
10.2 62.2406900090422
10.225 62.196043772118
10.25 62.1534039847103
10.275 62.1127706468189
10.3 62.0741437584438
10.325 62.0375233195852
10.35 62.002909330243
10.375 61.9703017904171
10.4 61.9397007001076
10.425 61.9111060593145
10.45 61.8845178680378
10.475 61.8599361262775
10.5 61.8373608340335
10.525 61.816791991306
10.55 61.7982295980948
10.575 61.7816736544
10.6 61.7671241602216
10.625 61.7545811155596
10.65 61.744044520414
10.675 61.7355143747847
10.7 61.7289906786719
10.725 61.7244734320754
10.75 61.7219626349953
10.775 61.7214582874316
10.8 61.7229603893843
10.825 61.7264689408534
10.85 61.7319839418388
10.875 61.7395053923406
10.9 61.7490332923588
10.925 61.7605676418935
10.95 61.7741084409445
10.975 61.7896556895118
11 61.392702927607
11.025 61.392702927607
11.05 61.392702927607
11.075 61.392702927607
11.1 61.392702927607
11.125 61.392702927607
11.15 61.392702927607
11.175 61.392702927607
11.2 61.392702927607
11.225 61.392702927607
11.25 61.392702927607
11.275 61.392702927607
11.3 61.392702927607
11.325 61.392702927607
11.35 61.392702927607
11.375 61.392702927607
11.4 61.392702927607
11.425 61.392702927607
11.45 61.392702927607
11.475 61.392702927607
11.5 61.392702927607
11.525 61.392702927607
11.55 61.392702927607
11.575 61.392702927607
11.6 61.392702927607
11.625 61.392702927607
11.65 61.392702927607
11.675 61.392702927607
11.7 61.392702927607
11.725 61.392702927607
11.75 61.392702927607
11.775 61.392702927607
11.8 61.392702927607
11.825 61.392702927607
11.85 61.392702927607
11.875 61.392702927607
11.9 61.392702927607
11.925 61.392702927607
11.95 61.392702927607
11.975 61.392702927607
12 61.392702927607
12.025 61.392702927607
12.05 61.392702927607
12.075 61.392702927607
12.1 61.392702927607
12.125 61.392702927607
12.15 61.392702927607
12.175 61.392702927607
12.2 61.392702927607
12.225 61.392702927607
12.25 61.392702927607
12.275 61.392702927607
12.3 61.392702927607
12.325 61.392702927607
12.35 61.392702927607
12.375 61.392702927607
12.4 61.392702927607
12.425 61.392702927607
12.45 61.392702927607
12.475 61.392702927607
12.5 61.392702927607
12.525 61.392702927607
12.55 61.392702927607
12.575 61.392702927607
12.6 61.392702927607
12.625 61.392702927607
12.65 61.392702927607
12.675 61.392702927607
12.7 61.392702927607
12.725 61.392702927607
12.75 61.392702927607
12.775 61.392702927607
12.8 61.392702927607
12.825 61.392702927607
12.85 61.392702927607
12.875 61.392702927607
12.9 61.392702927607
12.925 61.392702927607
12.95 61.392702927607
12.975 61.392702927607
13 61.392702927607
13.025 61.392702927607
13.05 61.392702927607
13.075 61.392702927607
13.1 61.392702927607
13.125 61.392702927607
13.15 61.392702927607
13.175 61.392702927607
13.2 61.392702927607
13.225 61.392702927607
13.25 61.392702927607
13.275 61.392702927607
13.3 61.392702927607
13.325 61.392702927607
13.35 61.392702927607
13.375 61.392702927607
13.4 61.392702927607
13.425 61.392702927607
13.45 61.392702927607
13.475 61.392702927607
13.5 61.392702927607
13.525 61.392702927607
13.55 61.392702927607
13.575 61.392702927607
13.6 61.392702927607
13.625 61.392702927607
13.65 61.392702927607
13.675 61.392702927607
13.7 61.392702927607
13.725 61.392702927607
13.75 61.392702927607
13.775 61.392702927607
13.8 61.392702927607
13.825 61.392702927607
13.85 61.392702927607
13.875 61.392702927607
13.9 61.392702927607
13.925 61.392702927607
13.95 61.392702927607
13.975 61.392702927607
14 61.392702927607
14.025 61.392702927607
14.05 61.392702927607
14.075 61.392702927607
14.1 61.392702927607
14.125 61.392702927607
14.15 61.392702927607
14.175 61.392702927607
14.2 61.392702927607
14.225 61.392702927607
14.25 61.392702927607
14.275 61.392702927607
14.3 61.392702927607
14.325 61.392702927607
14.35 61.392702927607
14.375 61.392702927607
14.4 61.392702927607
14.425 61.392702927607
14.45 61.392702927607
14.475 61.392702927607
14.5 61.392702927607
14.525 61.392702927607
14.55 61.392702927607
14.575 61.392702927607
14.6 61.392702927607
14.625 61.392702927607
14.65 61.392702927607
14.675 61.392702927607
14.7 61.392702927607
14.725 61.392702927607
14.75 61.392702927607
14.775 61.392702927607
14.8 61.392702927607
14.825 61.392702927607
14.85 61.392702927607
14.875 61.392702927607
14.9 61.392702927607
14.925 61.392702927607
14.95 61.392702927607
14.975 61.392702927607
15 61.392702927607
15.025 61.392702927607
15.05 61.392702927607
15.075 61.392702927607
15.1 61.392702927607
15.125 61.392702927607
15.15 61.392702927607
15.175 61.392702927607
15.2 61.392702927607
15.225 61.392702927607
15.25 61.392702927607
15.275 61.392702927607
15.3 61.392702927607
15.325 61.392702927607
15.35 61.392702927607
15.375 61.392702927607
15.4 61.392702927607
15.425 61.392702927607
15.45 61.392702927607
15.475 61.392702927607
15.5 61.392702927607
15.525 61.392702927607
15.55 61.392702927607
15.575 61.392702927607
15.6 61.392702927607
15.625 61.392702927607
15.65 61.392702927607
15.675 61.392702927607
15.7 61.392702927607
15.725 61.392702927607
15.75 61.392702927607
15.775 61.392702927607
15.8 61.392702927607
15.825 61.392702927607
15.85 61.392702927607
15.875 61.392702927607
15.9 61.392702927607
15.925 61.392702927607
15.95 61.392702927607
15.975 61.392702927607
16 61.392702927607
16.025 61.392702927607
16.05 61.392702927607
16.075 61.392702927607
16.1 61.392702927607
16.125 61.392702927607
16.15 61.392702927607
16.175 61.392702927607
16.2 61.392702927607
16.225 61.392702927607
16.25 61.392702927607
16.275 61.392702927607
16.3 61.392702927607
16.325 61.392702927607
16.35 61.392702927607
16.375 61.392702927607
16.4 61.392702927607
16.425 61.392702927607
16.45 61.392702927607
16.475 61.392702927607
16.5 61.392702927607
16.525 61.392702927607
16.55 61.392702927607
16.575 61.392702927607
16.6 61.392702927607
16.625 61.392702927607
16.65 61.392702927607
16.675 61.392702927607
16.7 61.392702927607
16.725 61.392702927607
16.75 61.392702927607
16.775 61.392702927607
16.8 61.392702927607
16.825 61.392702927607
16.85 61.392702927607
16.875 61.392702927607
16.9 61.392702927607
16.925 61.392702927607
16.95 61.392702927607
16.975 61.392702927607
17 61.392702927607
17.025 61.392702927607
17.05 61.392702927607
17.075 61.392702927607
17.1 61.392702927607
17.125 61.392702927607
17.15 61.392702927607
17.175 61.392702927607
17.2 61.392702927607
17.225 61.392702927607
17.25 61.392702927607
17.275 61.392702927607
17.3 61.392702927607
17.325 61.392702927607
17.35 61.392702927607
17.375 61.392702927607
17.4 61.392702927607
17.425 61.392702927607
17.45 61.392702927607
17.475 61.392702927607
17.5 61.392702927607
17.525 61.392702927607
17.55 61.392702927607
17.575 61.392702927607
17.6 61.392702927607
17.625 61.392702927607
17.65 61.392702927607
17.675 61.392702927607
17.7 61.392702927607
17.725 61.392702927607
17.75 61.392702927607
17.775 61.392702927607
17.8 61.392702927607
17.825 61.392702927607
17.85 61.392702927607
17.875 61.392702927607
17.9 61.392702927607
17.925 61.392702927607
17.95 61.392702927607
17.975 61.392702927607
18 61.7326653372238
18.025 61.8311539114564
18.05 61.9289491671942
18.075 62.026051104437
18.1 62.1224597231851
18.125 62.2181750234382
18.15 62.3131970051965
18.175 62.4075256684599
18.2 62.5011610132285
18.225 62.5941030395022
18.25 62.686351747281
18.275 62.7779071365649
18.3 62.868769207354
18.325 62.9589379596482
18.35 63.0484133934476
18.375 63.1371955087521
18.4 63.2252843055617
18.425 63.3126797838764
18.45 63.3993819436963
18.475 63.4853907850213
18.5 63.5707063078515
18.525 63.6553285121868
18.55 63.7392573980272
18.575 63.8224929653727
18.6 63.9050352142234
18.625 63.9868841445793
18.65 64.0680397564402
18.675 64.1485020498063
18.7 64.2282710246775
18.725 64.3073466810539
18.75 64.3857290189354
18.775 64.463418038322
18.8 64.5404137392137
18.825 64.6167161216106
18.85 64.6923251855127
18.875 64.7672409309198
18.9 64.8414633578321
18.925 64.9149924662495
18.95 64.9878282561721
18.975 65.0599707275998
19 65.1314198805326
19.025 65.2021757149706
19.05 65.2722382309137
19.075 65.3416074283619
19.1 65.4102833073152
19.125 65.4782658677737
19.15 65.5455551097374
19.175 65.6121510332061
19.2 65.67805363818
19.225 65.743262924659
19.25 65.8077788926432
19.275 65.8716015421325
19.3 65.9347308731269
19.325 65.9971668856265
19.35 66.0589095796312
19.375 66.119958955141
19.4 66.180315012156
19.425 66.239977750676
19.45 66.2989471707013
19.475 66.3572232722316
19.5 66.4148060552671
19.525 66.4716955198078
19.55 66.5278916658535
19.575 66.5833944934044
19.6 66.6382040024605
19.625 66.6923201930216
19.65 66.7457430650879
19.675 66.7984726186594
19.7 66.8505088537359
19.725 66.9018517703176
19.75 66.9525013684045
19.775 67.0024576479964
19.8 67.0517206090935
19.825 67.1002902516958
19.85 67.1481665758031
19.875 67.1953495814156
19.9 67.2418392685333
19.925 67.287635637156
19.95 67.3327386872839
19.975 67.3771484189169
20 67.4208648320551
20.025 67.4638879266984
20.05 67.5062177028469
20.075 67.5478541605004
20.1 67.5887972996591
20.125 67.629047120323
20.15 67.6686036224919
20.175 67.707466806166
20.2 67.7456366713453
20.225 67.7831132180296
20.25 67.8198964462192
20.275 67.8559863559138
20.3 67.8913829471136
20.325 67.9260862198185
20.35 67.9600961740285
20.375 67.9934128097437
20.4 68.026036126964
20.425 68.0579661256894
20.45 68.08920280592
20.475 68.1197461676557
20.5 68.1496431270512
20.525 68.1791751810337
20.55 68.2083892457577
20.575 68.2372853212234
20.6 68.2658634074307
20.625 68.2941235043795
20.65 68.32206561207
20.675 68.3496897305021
20.7 68.3769958596758
20.725 68.4039839995912
20.75 68.4306541502481
20.775 68.4570063116466
20.8 68.4830404837868
20.825 68.5087566666685
20.85 68.5341548602919
20.875 68.5592350646568
20.9 68.5839972797634
20.925 68.6084415056116
20.95 68.6325677422014
20.975 68.6563759895328
21 68.6798662476058
21.025 68.7030385164204
21.05 68.7258927959766
21.075 68.7484290862745
21.1 68.7706473873139
21.125 68.792547699095
21.15 68.8141300216176
21.175 68.8353943548819
21.2 68.8563406988878
21.225 68.8769690536352
21.25 68.8972794191243
21.275 68.917271795355
21.3 68.9369461823274
21.325 68.9563025800413
21.35 68.9753409884968
21.375 68.9940614076939
21.4 69.0124638376327
21.425 69.030548278313
21.45 69.048314729735
21.475 69.0657631918986
21.5 69.0828936648037
21.525 69.0997061484505
21.55 69.1162006428389
21.575 69.1323771479689
21.6 69.1482356638405
21.625 69.1637761904538
21.65 69.1789987278086
21.675 69.193903275905
21.7 69.2084898347431
21.725 69.2227584043227
21.75 69.236708984644
21.775 69.2503415757069
21.8 69.2636561775114
21.825 69.2766527900574
21.85 69.2893314133451
21.875 69.3016920473744
21.9 69.3137346921453
21.925 69.3254593476579
21.95 69.336866013912
21.975 69.3479546909077
22 69.3587253786451
22.025 69.369178077124
22.05 69.3793127863446
22.075 69.3891295063068
22.1 69.3986282370105
22.125 69.4078089784559
22.15 69.4166717306429
22.175 69.4252164935716
22.2 69.4334432672418
22.225 69.4413520516536
22.25 69.448942846807
22.275 69.4562156527021
22.3 69.4631704693387
22.325 69.469807296717
22.35 69.4761261348368
22.375 69.4821269836983
22.4 69.4878098433014
22.425 69.4931747136461
22.45 69.4982215947324
22.475 69.5029504865603
22.5 69.5073613891298
22.525 69.5114543024409
22.55 69.5152292264937
22.575 69.518686161288
22.6 69.521825106824
22.625 69.5246460631015
22.65 69.5271490301207
22.675 69.5293340078815
22.7 69.5312009963838
22.725 69.5327499956278
22.75 69.5339810056134
22.775 69.5348940263407
22.8 69.5354890578095
22.825 69.5357661000199
22.85 69.5357251529719
22.875 69.5353662166656
22.9 69.5346892911008
22.925 69.5336943762777
22.95 69.5323814721961
22.975 69.5307505788562
23 69.935051266673
23.025 69.935051266673
23.05 69.935051266673
23.075 69.935051266673
23.1 69.935051266673
23.125 69.935051266673
23.15 69.935051266673
23.175 69.935051266673
23.2 69.935051266673
23.225 69.935051266673
23.25 69.935051266673
23.275 69.935051266673
23.3 69.935051266673
23.325 69.935051266673
23.35 69.935051266673
23.375 69.935051266673
23.4 69.935051266673
23.425 69.935051266673
23.45 69.935051266673
23.475 69.935051266673
23.5 69.935051266673
23.525 69.935051266673
23.55 69.935051266673
23.575 69.935051266673
23.6 69.935051266673
23.625 69.935051266673
23.65 69.935051266673
23.675 69.935051266673
23.7 69.935051266673
23.725 69.935051266673
23.75 69.935051266673
23.775 69.935051266673
23.8 69.935051266673
23.825 69.935051266673
23.85 69.935051266673
23.875 69.935051266673
23.9 69.935051266673
23.925 69.935051266673
23.95 69.935051266673
23.975 69.935051266673
24 69.935051266673
24.025 69.935051266673
24.05 69.935051266673
24.075 69.935051266673
24.1 69.935051266673
24.125 69.935051266673
24.15 69.935051266673
24.175 69.935051266673
24.2 69.935051266673
24.225 69.935051266673
24.25 69.935051266673
24.275 69.935051266673
24.3 69.935051266673
24.325 69.935051266673
24.35 69.935051266673
24.375 69.935051266673
24.4 69.935051266673
24.425 69.935051266673
24.45 69.935051266673
24.475 69.935051266673
24.5 69.935051266673
24.525 69.935051266673
24.55 69.935051266673
24.575 69.935051266673
24.6 69.935051266673
24.625 69.935051266673
24.65 69.935051266673
24.675 69.935051266673
24.7 69.935051266673
24.725 69.935051266673
24.75 69.935051266673
24.775 69.935051266673
24.8 69.935051266673
24.825 69.935051266673
24.85 69.935051266673
24.875 69.935051266673
24.9 69.935051266673
24.925 69.935051266673
24.95 69.935051266673
24.975 69.935051266673
25 69.935051266673
25.025 69.935051266673
25.05 69.935051266673
25.075 69.935051266673
25.1 69.935051266673
25.125 69.935051266673
25.15 69.935051266673
25.175 69.935051266673
25.2 69.935051266673
25.225 69.935051266673
25.25 69.935051266673
25.275 69.935051266673
25.3 69.935051266673
25.325 69.935051266673
25.35 69.935051266673
25.375 69.935051266673
25.4 69.935051266673
25.425 69.935051266673
25.45 69.935051266673
25.475 69.935051266673
25.5 69.935051266673
25.525 69.935051266673
25.55 69.935051266673
25.575 69.935051266673
25.6 69.935051266673
25.625 69.935051266673
25.65 69.935051266673
25.675 69.935051266673
25.7 69.935051266673
25.725 69.935051266673
25.75 69.935051266673
25.775 69.935051266673
25.8 69.935051266673
25.825 69.935051266673
25.85 69.935051266673
25.875 69.935051266673
25.9 69.935051266673
25.925 69.935051266673
25.95 69.935051266673
25.975 69.935051266673
26 69.935051266673
26.025 69.935051266673
26.05 69.935051266673
26.075 69.935051266673
26.1 69.935051266673
26.125 69.935051266673
26.15 69.935051266673
26.175 69.935051266673
26.2 69.935051266673
26.225 69.935051266673
26.25 69.935051266673
26.275 69.935051266673
26.3 69.935051266673
26.325 69.935051266673
26.35 69.935051266673
26.375 69.935051266673
26.4 69.935051266673
26.425 69.935051266673
26.45 69.935051266673
26.475 69.935051266673
26.5 69.935051266673
26.525 69.935051266673
26.55 69.935051266673
26.575 69.935051266673
26.6 69.935051266673
26.625 69.935051266673
26.65 69.935051266673
26.675 69.935051266673
26.7 69.935051266673
26.725 69.935051266673
26.75 69.935051266673
26.775 69.935051266673
26.8 69.935051266673
26.825 69.935051266673
26.85 69.935051266673
26.875 69.935051266673
26.9 69.935051266673
26.925 69.935051266673
26.95 69.935051266673
26.975 69.935051266673
27 69.935051266673
27.025 69.935051266673
27.05 69.935051266673
27.075 69.935051266673
27.1 69.935051266673
27.125 69.935051266673
27.15 69.935051266673
27.175 69.935051266673
27.2 69.935051266673
27.225 69.935051266673
27.25 69.935051266673
27.275 69.935051266673
27.3 69.935051266673
27.325 69.935051266673
27.35 69.935051266673
27.375 69.935051266673
27.4 69.935051266673
27.425 69.935051266673
27.45 69.935051266673
27.475 69.935051266673
27.5 69.935051266673
27.525 69.935051266673
27.55 69.935051266673
27.575 69.935051266673
27.6 69.935051266673
27.625 69.935051266673
27.65 69.935051266673
27.675 69.935051266673
27.7 69.935051266673
27.725 69.935051266673
27.75 69.935051266673
27.775 69.935051266673
27.8 69.935051266673
27.825 69.935051266673
27.85 69.935051266673
27.875 69.935051266673
27.9 69.935051266673
27.925 69.935051266673
27.95 69.935051266673
27.975 69.935051266673
28 69.935051266673
28.025 69.935051266673
28.05 69.935051266673
28.075 69.935051266673
28.1 69.935051266673
28.125 69.935051266673
28.15 69.935051266673
28.175 69.935051266673
28.2 69.935051266673
28.225 69.935051266673
28.25 69.935051266673
28.275 69.935051266673
28.3 69.935051266673
28.325 69.935051266673
28.35 69.935051266673
28.375 69.935051266673
28.4 69.935051266673
28.425 69.935051266673
28.45 69.935051266673
28.475 69.935051266673
28.5 69.935051266673
28.525 69.935051266673
28.55 69.935051266673
28.575 69.935051266673
28.6 69.935051266673
28.625 69.935051266673
28.65 69.935051266673
28.675 69.935051266673
28.7 69.935051266673
28.725 69.935051266673
28.75 69.935051266673
28.775 69.935051266673
28.8 69.935051266673
28.825 69.935051266673
28.85 69.935051266673
28.875 69.935051266673
28.9 69.935051266673
28.925 69.935051266673
28.95 69.935051266673
28.975 69.935051266673
29 69.935051266673
29.025 69.935051266673
29.05 69.935051266673
29.075 69.935051266673
29.1 69.935051266673
29.125 69.935051266673
29.15 69.935051266673
29.175 69.935051266673
29.2 69.935051266673
29.225 69.935051266673
29.25 69.935051266673
29.275 69.935051266673
29.3 69.935051266673
29.325 69.935051266673
29.35 69.935051266673
29.375 69.935051266673
29.4 69.935051266673
29.425 69.935051266673
29.45 69.935051266673
29.475 69.935051266673
29.5 69.935051266673
29.525 69.935051266673
29.55 69.935051266673
29.575 69.935051266673
29.6 69.935051266673
29.625 69.935051266673
29.65 69.935051266673
29.675 69.935051266673
29.7 69.935051266673
29.725 69.935051266673
29.75 69.935051266673
29.775 69.935051266673
29.8 69.935051266673
29.825 69.935051266673
29.85 69.935051266673
29.875 69.935051266673
29.9 69.935051266673
29.925 69.935051266673
29.95 69.935051266673
29.975 69.935051266673
30 69.935051266673
30.025 69.935051266673
30.05 69.935051266673
30.075 69.935051266673
30.1 69.935051266673
30.125 69.935051266673
30.15 69.935051266673
30.175 69.935051266673
30.2 69.935051266673
30.225 69.935051266673
30.25 69.935051266673
30.275 69.935051266673
30.3 69.935051266673
30.325 69.935051266673
30.35 69.935051266673
30.375 69.935051266673
30.4 69.935051266673
30.425 69.935051266673
30.45 69.935051266673
30.475 69.935051266673
30.5 69.935051266673
30.525 69.935051266673
30.55 69.935051266673
30.575 69.935051266673
30.6 69.935051266673
30.625 69.935051266673
30.65 69.935051266673
30.675 69.935051266673
30.7 69.935051266673
30.725 69.935051266673
30.75 69.935051266673
30.775 69.935051266673
30.8 69.935051266673
30.825 69.935051266673
30.85 69.935051266673
30.875 69.935051266673
30.9 69.935051266673
30.925 69.935051266673
30.95 69.935051266673
30.975 69.935051266673
31 69.935051266673
31.025 69.935051266673
31.05 69.935051266673
31.075 69.935051266673
31.1 69.935051266673
31.125 69.935051266673
31.15 69.935051266673
31.175 69.935051266673
31.2 69.935051266673
31.225 69.935051266673
31.25 69.935051266673
31.275 69.935051266673
31.3 69.935051266673
31.325 69.935051266673
31.35 69.935051266673
31.375 69.935051266673
31.4 69.935051266673
31.425 69.935051266673
31.45 69.935051266673
31.475 69.935051266673
31.5 69.935051266673
31.525 69.935051266673
31.55 69.935051266673
31.575 69.935051266673
31.6 69.935051266673
31.625 69.935051266673
31.65 69.935051266673
31.675 69.935051266673
31.7 69.935051266673
31.725 69.935051266673
31.75 69.935051266673
31.775 69.935051266673
31.8 69.935051266673
31.825 69.935051266673
31.85 69.935051266673
31.875 69.935051266673
31.9 69.935051266673
31.925 69.935051266673
31.95 69.935051266673
31.975 69.935051266673
32 69.935051266673
32.025 69.935051266673
32.05 69.935051266673
32.075 69.935051266673
32.1 69.935051266673
32.125 69.935051266673
32.15 69.935051266673
32.175 69.935051266673
32.2 69.935051266673
32.225 69.935051266673
32.25 69.935051266673
32.275 69.935051266673
32.3 69.935051266673
32.325 69.935051266673
32.35 69.935051266673
32.375 69.935051266673
32.4 69.935051266673
32.425 69.935051266673
32.45 69.935051266673
32.475 69.935051266673
32.5 69.935051266673
32.525 69.935051266673
32.55 69.935051266673
32.575 69.935051266673
32.6 69.935051266673
32.625 69.935051266673
32.65 69.935051266673
32.675 69.935051266673
32.7 69.935051266673
32.725 69.935051266673
32.75 69.935051266673
32.775 69.935051266673
32.8 69.935051266673
32.825 69.935051266673
32.85 69.935051266673
32.875 69.935051266673
32.9 69.935051266673
32.925 69.935051266673
32.95 69.935051266673
32.975 69.935051266673
33 69.935051266673
33.025 69.935051266673
33.05 69.935051266673
33.075 69.935051266673
33.1 69.935051266673
33.125 69.935051266673
33.15 69.935051266673
33.175 69.935051266673
33.2 69.935051266673
33.225 69.935051266673
33.25 69.935051266673
33.275 69.935051266673
33.3 69.935051266673
33.325 69.935051266673
33.35 69.935051266673
33.375 69.935051266673
33.4 69.935051266673
33.425 69.935051266673
33.45 69.935051266673
33.475 69.935051266673
33.5 69.935051266673
33.525 69.935051266673
33.55 69.935051266673
33.575 69.935051266673
33.6 69.935051266673
33.625 69.935051266673
33.65 69.935051266673
33.675 69.935051266673
33.7 69.935051266673
33.725 69.935051266673
33.75 69.935051266673
33.775 69.935051266673
33.8 69.935051266673
33.825 69.935051266673
33.85 69.935051266673
33.875 69.935051266673
33.9 69.935051266673
33.925 69.935051266673
33.95 69.935051266673
33.975 69.935051266673
34 69.935051266673
34.025 69.935051266673
34.05 69.935051266673
34.075 69.935051266673
34.1 69.935051266673
34.125 69.935051266673
34.15 69.935051266673
34.175 69.935051266673
34.2 69.935051266673
34.225 69.935051266673
34.25 69.935051266673
34.275 69.935051266673
34.3 69.935051266673
34.325 69.935051266673
34.35 69.935051266673
34.375 69.935051266673
34.4 69.935051266673
34.425 69.935051266673
34.45 69.935051266673
34.475 69.935051266673
34.5 69.935051266673
34.525 69.935051266673
34.55 69.935051266673
34.575 69.935051266673
34.6 69.935051266673
34.625 69.935051266673
34.65 69.935051266673
34.675 69.935051266673
34.7 69.935051266673
34.725 69.935051266673
34.75 69.935051266673
34.775 69.935051266673
34.8 69.935051266673
34.825 69.935051266673
34.85 69.935051266673
34.875 69.935051266673
34.9 69.935051266673
34.925 69.935051266673
34.95 69.935051266673
34.975 69.935051266673
35 69.935051266673
35.025 69.935051266673
35.05 69.935051266673
35.075 69.935051266673
35.1 69.935051266673
35.125 69.935051266673
35.15 69.935051266673
35.175 69.935051266673
35.2 69.935051266673
35.225 69.935051266673
35.25 69.935051266673
35.275 69.935051266673
35.3 69.935051266673
35.325 69.935051266673
35.35 69.935051266673
35.375 69.935051266673
35.4 69.935051266673
35.425 69.935051266673
35.45 69.935051266673
35.475 69.935051266673
35.5 69.935051266673
35.525 69.935051266673
35.55 69.935051266673
35.575 69.935051266673
35.6 69.935051266673
35.625 69.935051266673
35.65 69.935051266673
35.675 69.935051266673
35.7 69.935051266673
35.725 69.935051266673
35.75 69.935051266673
35.775 69.935051266673
35.8 69.935051266673
35.825 69.935051266673
35.85 69.935051266673
35.875 69.935051266673
35.9 69.935051266673
35.925 69.935051266673
35.95 69.935051266673
35.975 69.935051266673
36 69.935051266673
36.025 69.935051266673
36.05 69.935051266673
36.075 69.935051266673
36.1 69.935051266673
36.125 69.935051266673
36.15 69.935051266673
36.175 69.935051266673
36.2 69.935051266673
36.225 69.935051266673
36.25 69.935051266673
36.275 69.935051266673
36.3 69.935051266673
36.325 69.935051266673
36.35 69.935051266673
36.375 69.935051266673
36.4 69.935051266673
36.425 69.935051266673
36.45 69.935051266673
36.475 69.935051266673
36.5 69.935051266673
36.525 69.935051266673
36.55 69.935051266673
36.575 69.935051266673
36.6 69.935051266673
36.625 69.935051266673
36.65 69.935051266673
36.675 69.935051266673
36.7 69.935051266673
36.725 69.935051266673
36.75 69.935051266673
36.775 69.935051266673
36.8 69.935051266673
36.825 69.935051266673
36.85 69.935051266673
36.875 69.935051266673
36.9 69.935051266673
36.925 69.935051266673
36.95 69.935051266673
36.975 69.935051266673
37 69.935051266673
37.025 69.935051266673
37.05 69.935051266673
37.075 69.935051266673
37.1 69.935051266673
37.125 69.935051266673
37.15 69.935051266673
37.175 69.935051266673
37.2 69.935051266673
37.225 69.935051266673
37.25 69.935051266673
37.275 69.935051266673
37.3 69.935051266673
37.325 69.935051266673
37.35 69.935051266673
37.375 69.935051266673
37.4 69.935051266673
37.425 69.935051266673
37.45 69.935051266673
37.475 69.935051266673
37.5 69.935051266673
37.525 69.935051266673
37.55 69.935051266673
37.575 69.935051266673
37.6 69.935051266673
37.625 69.935051266673
37.65 69.935051266673
37.675 69.935051266673
37.7 69.935051266673
37.725 69.935051266673
37.75 69.935051266673
37.775 69.935051266673
37.8 69.935051266673
37.825 69.935051266673
37.85 69.935051266673
37.875 69.935051266673
37.9 69.935051266673
37.925 69.935051266673
37.95 69.935051266673
37.975 69.935051266673
38 69.935051266673
38.025 69.935051266673
38.05 69.935051266673
38.075 69.935051266673
38.1 69.935051266673
38.125 69.935051266673
38.15 69.935051266673
38.175 69.935051266673
38.2 69.935051266673
38.225 69.935051266673
38.25 69.935051266673
38.275 69.935051266673
38.3 69.935051266673
38.325 69.935051266673
38.35 69.935051266673
38.375 69.935051266673
38.4 69.935051266673
38.425 69.935051266673
38.45 69.935051266673
38.475 69.935051266673
38.5 69.935051266673
38.525 69.935051266673
38.55 69.935051266673
38.575 69.935051266673
38.6 69.935051266673
38.625 69.935051266673
38.65 69.935051266673
38.675 69.935051266673
38.7 69.935051266673
38.725 69.935051266673
38.75 69.935051266673
38.775 69.935051266673
38.8 69.935051266673
38.825 69.935051266673
38.85 69.935051266673
38.875 69.935051266673
38.9 69.935051266673
38.925 69.935051266673
38.95 69.935051266673
38.975 69.935051266673
39 69.935051266673
39.025 69.935051266673
39.05 69.935051266673
39.075 69.935051266673
39.1 69.935051266673
39.125 69.935051266673
39.15 69.935051266673
39.175 69.935051266673
39.2 69.935051266673
39.225 69.935051266673
39.25 69.935051266673
39.275 69.935051266673
39.3 69.935051266673
39.325 69.935051266673
39.35 69.935051266673
39.375 69.935051266673
39.4 69.935051266673
39.425 69.935051266673
39.45 69.935051266673
39.475 69.935051266673
39.5 69.935051266673
39.525 69.935051266673
39.55 69.935051266673
39.575 69.935051266673
39.6 69.935051266673
39.625 69.935051266673
39.65 69.935051266673
39.675 69.935051266673
39.7 69.935051266673
39.725 69.935051266673
39.75 69.935051266673
39.775 69.935051266673
39.8 69.935051266673
39.825 69.935051266673
39.85 69.935051266673
39.875 69.935051266673
39.9 69.935051266673
39.925 69.935051266673
39.95 69.935051266673
39.975 69.935051266673
40 69.935051266673
40.025 69.935051266673
40.05 69.935051266673
40.075 69.935051266673
40.1 69.935051266673
40.125 69.935051266673
40.15 69.935051266673
40.175 69.935051266673
40.2 69.935051266673
40.225 69.935051266673
40.25 69.935051266673
40.275 69.935051266673
40.3 69.935051266673
40.325 69.935051266673
40.35 69.935051266673
40.375 69.935051266673
40.4 69.935051266673
40.425 69.935051266673
40.45 69.935051266673
40.475 69.935051266673
40.5 69.935051266673
40.525 69.935051266673
40.55 69.935051266673
40.575 69.935051266673
40.6 69.935051266673
40.625 69.935051266673
40.65 69.935051266673
40.675 69.935051266673
40.7 69.935051266673
40.725 69.935051266673
40.75 69.935051266673
40.775 69.935051266673
40.8 69.935051266673
40.825 69.935051266673
40.85 69.935051266673
40.875 69.935051266673
40.9 69.935051266673
40.925 69.935051266673
40.95 69.935051266673
40.975 69.935051266673
41 69.935051266673
41.025 69.935051266673
41.05 69.935051266673
41.075 69.935051266673
41.1 69.935051266673
41.125 69.935051266673
41.15 69.935051266673
41.175 69.935051266673
41.2 69.935051266673
41.225 69.935051266673
41.25 69.935051266673
41.275 69.935051266673
41.3 69.935051266673
41.325 69.935051266673
41.35 69.935051266673
41.375 69.935051266673
41.4 69.935051266673
41.425 69.935051266673
41.45 69.935051266673
41.475 69.935051266673
41.5 69.935051266673
41.525 69.935051266673
41.55 69.935051266673
41.575 69.935051266673
41.6 69.935051266673
41.625 69.935051266673
41.65 69.935051266673
41.675 69.935051266673
41.7 69.935051266673
41.725 69.935051266673
41.75 69.935051266673
41.775 69.935051266673
41.8 69.935051266673
41.825 69.935051266673
41.85 69.935051266673
41.875 69.935051266673
41.9 69.935051266673
41.925 69.935051266673
41.95 69.935051266673
41.975 69.935051266673
42 69.935051266673
42.025 69.935051266673
42.05 69.935051266673
42.075 69.935051266673
42.1 69.935051266673
42.125 69.935051266673
42.15 69.935051266673
42.175 69.935051266673
42.2 69.935051266673
42.225 69.935051266673
42.25 69.935051266673
42.275 69.935051266673
42.3 69.935051266673
42.325 69.935051266673
};
\addplot [ultra thick, black, forget plot]
table {%
9 35
9 75
};
\addplot [ultra thick, black, forget plot]
table {%
11 35
11 75
};
\addplot [ultra thick, black, forget plot]
table {%
18 35
18 75
};
\addplot [ultra thick, black, forget plot]
table {%
23 35
23 75
};
\node at (axis cs:4.5,40)[
  anchor=base,
  text=black,
  rotate=0.0
]{ A};
\node at (axis cs:10,40)[
  anchor=base,
  text=black,
  rotate=0.0
]{ B};
\node at (axis cs:14.5,40)[
  anchor=base,
  text=black,
  rotate=0.0
]{ C};
\node at (axis cs:20.5,40)[
  anchor=base,
  text=black,
  rotate=0.0
]{ D};
\node at (axis cs:31.5,40)[
  anchor=base,
  text=black,
  rotate=0.0
]{ E};
\end{axis}

\end{tikzpicture}}\\
	\subfloat[Lateral position ego vehicle.]{% This file was created by matplotlib2tikz v0.6.14.
\begin{tikzpicture}

\definecolor{color0}{rgb}{0.12156862745098,0.466666666666667,0.705882352941177}
\definecolor{color1}{rgb}{1,0.498039215686275,0.0549019607843137}

\begin{axis}[
xlabel={$t$ [s]},
ylabel={Lateral distance [m]},
xmin=0, xmax=40,
ymin=-0.2, ymax=3.8,
width=\figurewidth,
height=\figureheight,
tick align=outside,
tick pos=left,
xmajorgrids,
x grid style={white!69.019607843137251!black},
ymajorgrids,
y grid style={white!69.019607843137251!black},
clip marker paths
]
\addplot [semithick, color0, mark=*, mark size=1, mark options={solid}, only marks, forget plot]
table {%
0 0.0937500001330113
0.025 0.0828903467114039
0.05 0.0720307975106381
0.075 0.0594505826611748
0.1 0.0470313523314523
0.125 0.0352070005756295
0.15 0.0244521743136864
0.175 0.0141362518893688
0.2 0.00462722711661243
0.225 0.00453474629515783
0.25 0.0133658820522649
0.275 0.0213851986889686
0.3 0.0291163107499829
0.325 0.0364014834129398
0.35 0.0429244132663706
0.375 0.0491345539530386
0.4 0.0548429105440976
0.425 0.0601911966537087
0.45 0.0652041258725517
0.475 0.0696687390350636
0.5 0.0739920138997101
0.525 0.0781016767740133
0.55 0.0816518148104859
0.575 0.0850028583115035
0.6 0.088117750405292
0.625 0.0910301416922158
0.65 0.0937522882449332
0.675 0.0961850404614553
0.7 0.0985493281102856
0.725 0.100813504571676
0.75 0.102534635330981
0.775 0.10407642126486
0.8 0.105491911585923
0.825 0.106791091813252
0.85 0.107974379674953
0.875 0.109051950579307
0.9 0.110088055486755
0.925 0.111059705738672
0.95 0.111661493011074
0.975 0.112139340593796
1 0.112576020228859
1.025 0.113057246471211
1.05 0.113579168666344
1.075 0.11404941537109
1.1 0.114504751148708
1.125 0.114937356924901
1.15 0.11532093219849
1.175 0.115685619215265
1.2 0.116021209177081
1.225 0.116281559693602
1.25 0.116468251785197
1.275 0.116635078240359
1.3 0.116717619744347
1.325 0.116619707210667
1.35 0.116533168531571
1.375 0.11645101545443
1.4 0.116375313806244
1.425 0.116114653609613
1.45 0.115674965077235
1.475 0.115279710096107
1.5 0.114695281157758
1.525 0.113709191016803
1.55 0.112829119821645
1.575 0.111990893108854
1.6 0.111213904632034
1.625 0.109772873949879
1.65 0.107716350809616
1.675 0.105882793814197
1.7 0.103975405517058
1.725 0.101862565534966
1.75 0.100020295504264
1.775 0.0982752398255858
1.8 0.0966631034448312
1.825 0.0947964198609719
1.85 0.0926739868378996
1.875 0.0907821927001157
1.9 0.0889127628526968
1.925 0.0870522073238416
1.95 0.0853984032767185
1.975 0.0838269942876814
2 0.0823894248763972
2.025 0.0808140384456236
2.05 0.079099268625621
2.075 0.0775656048233277
2.1 0.076086772923478
2.125 0.0746907480690876
2.15 0.0734400192582771
2.175 0.0722478364252017
2.2 0.0711418981985308
2.225 0.0698556671427727
2.25 0.0684073354334131
2.275 0.0670636932730139
2.3 0.0657246863498245
2.325 0.0643634418861496
2.35 0.0631438887690473
2.375 0.0619455750479519
2.4 0.0604136997882718
2.425 0.0589270664471708
2.45 0.0574937685129643
2.475 0.0562327671067012
2.5 0.0547799308865657
2.525 0.0527186809506036
2.55 0.0508721580700069
2.575 0.0491245498441236
2.6 0.0475146199581872
2.625 0.045745858812044
2.65 0.0438043114353348
2.675 0.0420354549408974
2.7 0.0401852899595268
2.725 0.038063209910462
2.75 0.0361506247736728
2.775 0.0343337928341486
2.8 0.0326535212618332
2.825 0.0307125776379243
2.85 0.0285287293014327
2.875 0.026491970559177
2.9 0.0244117545764509
2.925 0.0221727595746638
2.95 0.0201668845164022
2.975 0.0182391643286841
3 0.0162383935200944
3.025 0.0143364172765091
3.05 0.0125359013132222
3.075 0.0108996168460826
3.1 0.00917425190141665
3.125 0.00716508175155698
3.15 0.00534044137446139
3.175 0.00359567884903606
3.2 0.00192079282468207
3.225 0.000319731292477796
3.25 0.00120673187629871
3.275 0.00259468278899574
3.3 0.00400747389285228
3.325 0.00552669007709789
3.35 0.00690145186290519
3.375 0.00828607035493733
3.4 0.0103080235717949
3.425 0.0122555140476516
3.45 0.0141179014503684
3.475 0.000930918439993354
3.5 0.0175522470781492
3.525 0.0194731746699773
3.55 0.0212042644105164
3.575 0.0228468448453827
3.6 0.0243492893767263
3.625 0.0264419763743761
3.65 0.0280585013359203
3.675 0.0292408108612145
3.7 0.0307596789300958
3.725 0.0322062012946317
3.75 0.0335040589346945
3.775 0.0347331501306652
3.8 0.0358492920727369
3.825 0.0369178637475248
3.85 0.0379364235635357
3.875 0.0388630287450428
3.9 0.0397238192799848
3.925 0.0404465008230731
3.95 0.0411019096593025
3.975 0.0416683372308041
4 0.0417400270532535
4.025 0.0418106174927982
4.05 0.04187848515461
4.075 0.0419412357176481
4.1 0.0419998455300424
4.125 0.0419716689790136
4.15 0.0419138240906501
4.175 0.0418624661587798
4.2 0.0418125024525442
4.225 0.0417663039192073
4.25 0.0417223343723764
4.275 0.041680434890548
4.3 0.0416426985763516
4.325 0.0416061372973319
4.35 0.0415723637368928
4.375 0.0415923524081005
4.4 0.041784710314458
4.425 0.00164794447865664
4.45 0.0516592945093573
4.475 0.0423404451808666
4.5 0.0425339771702229
4.525 0.0427801873454714
4.55 0.0430197272993023
4.575 0.0432526660631627
4.6 0.0434579587432346
4.625 0.0436562176801357
4.65 0.0438390465335492
4.675 0.0439655257806944
4.7 0.0439068862458425
4.725 0.0438523456735977
4.75 0.043801742134144
4.775 0.0435951035304992
4.8 0.0433255862228156
4.825 0.0430833061191727
4.85 0.0428469853755793
4.875 0.0426303743477673
4.9 0.042272724946372
4.925 0.0417277971738127
4.95 0.0412067956072244
4.975 0.0406878973328025
5 0.040093655647362
5.025 0.0395577483770125
5.05 0.0390827200850529
5.075 0.0386295127953756
5.1 0.038186044023931
5.125 0.0378003001903472
5.15 0.037421008544834
5.175 0.0370634993887123
5.2 0.0368129493063491
5.225 0.0367151518971414
5.25 0.0366248888599905
5.275 0.0365382279796054
5.3 0.0363584424842424
5.325 0.036174769407985
5.35 0.0360187432551956
5.375 0.0357342915206352
5.4 0.0350189631163555
5.425 0.0343124121647861
5.45 0.0336150704181417
5.475 0.0327625881217262
5.5 0.0319327652384821
5.525 0.031217584919035
5.55 0.030531870579018
5.575 0.029873853531433
5.6 0.0291683197489171
5.625 0.0284503043149913
5.65 0.0277127579234485
5.675 0.0269585857146695
5.7 0.0261070702570375
5.725 0.0252772038499333
5.75 0.0244948365871197
5.775 0.023753557505925
5.8 0.0230460843499689
5.825 0.0223979162322381
5.85 0.0217752138803479
5.875 0.021533871892373
5.9 0.021291835677909
5.925 0.0210480526931526
5.95 0.0208310101839832
5.975 0.0206660651452854
6 0.020932283427054
6.025 0.021191517262773
6.05 0.0214348885778857
6.075 0.0219700335073666
6.1 0.0227441101400666
6.125 0.0234534179537148
6.15 0.024144304683699
6.175 0.024774533606781
6.2 0.0253863749094266
6.225 0.0259490148158042
6.25 0.0266057727940669
6.275 0.027353499013669
6.3 0.0280190187655241
6.325 0.028662917118124
6.35 0.0292534321193369
6.375 0.0298566331537929
6.4 0.0305012927367248
6.425 0.0311418514455273
6.45 0.0317780100884759
6.475 0.0324692739917272
6.5 0.0331279763543994
6.525 0.033688985984918
6.55 0.0342870333305839
6.575 0.0349210658862872
6.6 0.0354887616495046
6.625 0.036037667951111
6.65 0.037318795235879
6.675 0.0385837310907457
6.7 0.0397569237063454
6.725 0.0408474636107899
6.75 0.0418588276710913
6.775 0.0429812804023961
6.8 0.0441502248987133
6.825 0.0451885506834772
6.85 0.0462667532756874
6.875 0.0475315083447857
6.9 0.0487590588740332
6.925 0.0499366969521752
6.95 0.0511582219185159
6.975 0.0523841925021744
7 0.0534986567010675
7.025 0.0545165725381367
7.05 0.0554310091905185
7.075 0.05597409335529
7.1 0.0563633011666637
7.125 0.0567072831274513
7.15 0.0569876783514218
7.175 0.0569675172651814
7.2 0.056948141198807
7.225 0.0569297961077755
7.25 0.0566046628889646
7.275 0.0561556385875553
7.3 0.055737655122808
7.325 0.0553123279776291
7.35 0.0548291758802594
7.375 0.0543644103701772
7.4 0.0539282516224672
7.425 0.053518469143687
7.45 0.0531238401582796
7.475 0.0530605457293891
7.5 0.0530394288594526
7.525 0.053019357649522
7.55 0.0529150676665243
7.575 0.0527821205134967
7.6 0.0526678610896973
7.625 0.0524874880338097
7.65 0.0521647159187152
7.675 0.0518578892393511
7.7 0.0515642886907682
7.725 0.0512951811798999
7.75 0.0510305438768691
7.775 0.0507118143112995
7.8 0.0503929811792123
7.825 0.0500733145311238
7.85 0.0495585458152529
7.875 0.0487391536104326
7.9 0.047959084994135
7.925 0.0472059337774543
7.95 0.0462305519051314
7.975 0.0452130625384504
8 0.0442155357655194
8.025 0.0432044234726499
8.05 0.0421789199111946
8.075 0.0413367922618765
8.1 0.0405225574915033
8.125 0.0397593290962135
8.15 0.0389864532469252
8.175 0.0381962602598893
8.2 0.0374556527648801
8.225 0.0367479346479021
8.25 0.0358985166317877
8.275 0.0350344800540938
8.3 0.0342769839193291
8.325 0.0335384619323765
8.35 0.0328422537331374
8.375 0.0321830142175985
8.4 0.0315543515498977
8.425 0.0308482055002559
8.45 0.0301215527815056
8.475 0.0293429299096416
8.5 0.0285780811056558
8.525 0.0278613130010608
8.55 0.0268690573284208
8.575 0.0257951433186002
8.6 0.0248935990315108
8.625 0.0239459006066366
8.65 0.0228940850288927
8.675 0.0219064704440019
8.7 0.0209699343650884
8.725 0.0203675727359765
8.75 0.0198562266564151
8.775 0.0193899700825987
8.8 0.0189373957888283
8.825 0.0185211848341238
8.85 0.0182394348138724
8.875 0.0180860362942818
8.9 0.0179450462233644
8.925 0.0178080603438566
8.95 0.0179033891437041
8.975 0.0179995966627722
9 0.0180885681231342
9.025 0.0184185301379545
9.05 0.0190000670889748
9.075 0.0195183403231304
9.1 0.0200040430279172
9.125 0.0204248051166734
9.15 0.0208270977417118
9.175 0.0212118303342786
9.2 0.0215622841787689
9.225 0.0219023576751054
9.25 0.0223233588355407
9.275 0.0227362104508156
9.3 0.0231205924108513
9.325 0.0238082561075532
9.35 0.0247900713506881
9.375 0.0257183384453364
9.4 0.0266241159348821
9.425 0.0278893233302005
9.45 0.0291711379576596
9.475 0.0303959234370244
9.5 0.0317815441695558
9.525 0.0335439661376161
9.55 0.0351108821304218
9.575 0.036582819955911
9.6 0.0378160641167113
9.625 0.0391207433788545
9.65 0.0405821943450011
9.675 0.0419659180787217
9.7 0.0432856383563183
9.725 0.0447615527219445
9.75 0.0462751249480967
9.775 0.0476441720664905
9.8 0.0489753439759153
9.825 0.0502033378038996
9.85 0.0513763121926828
9.875 0.0524951675847219
9.9 0.0532630399350445
9.925 0.053920768691217
9.95 0.0545220964003146
9.975 0.0551087505161445
10 0.0556497570124183
10.025 0.0225843739299107
10.05 0.049305133199644
10.075 0.0569215149804316
10.1 0.0571696086872812
10.125 0.0571398186182733
10.15 0.0571109339581273
10.175 0.0570829489792512
10.2 0.0570576744193212
10.225 0.0570332072008517
10.25 0.0570105877212156
10.275 0.0570435108377658
10.3 0.0572403563770903
10.325 0.0574308603500816
10.35 0.0576135432224254
10.375 0.0577841927589238
10.4 0.0579496216908954
10.425 0.057991210965166
10.45 0.058026498458009
10.475 0.0580583137429669
10.5 0.0579972776213882
10.525 0.0577919812371441
10.55 0.0594468382743136
10.575 0.0573647758340038
10.6 0.0568378731545828
10.625 0.0563575245425989
10.65 0.0559242413337808
10.675 0.0553613071305664
10.7 0.0547453619704262
10.725 0.0542284642226387
10.75 0.053717473670496
10.775 0.0532076274499891
10.8 0.0526519787960712
10.825 0.0519896464705851
10.85 0.0513550955405798
10.875 0.0507468565494754
10.9 0.0501053962876794
10.925 0.0494535794672638
10.95 0.0488633557408286
10.975 0.0482343395346191
11 0.0474742413903244
11.025 0.0467167423495823
11.05 0.045961991336905
11.075 0.0452884000988201
11.1 0.0446272863021616
11.125 0.0439760936436389
11.15 0.0433495420689695
11.175 0.0427467439862248
11.2 0.0186546207986583
11.225 0.0421486505671494
11.25 0.0411884302250756
11.275 0.0407425874392129
11.3 0.0404157315108041
11.325 0.0401120812830512
11.35 0.0398303693104289
11.375 0.0397191937274682
11.4 0.0396798954675589
11.425 0.0396449669874875
11.45 0.0396524984988282
11.475 0.0398402642034397
11.5 0.0400252480763952
11.525 0.0402061436817667
11.55 0.0403774934606725
11.575 0.0405288539746703
11.6 0.0405592342031647
11.625 0.0405867173438491
11.65 0.0406113720256855
11.675 0.0407141363084883
11.7 0.0408478787361333
11.725 0.0409673029414032
11.75 0.0411240022199016
11.775 0.0414555006088702
11.8 0.041773023874273
11.825 0.0420716844440943
11.85 0.0423480845870751
11.875 0.0426137535714801
11.9 0.0427575111214031
11.925 0.0428859396901374
11.95 0.0430021999745392
11.975 0.0430708054431429
12 0.0430648517736617
12.025 0.043059271001645
12.05 0.0430539024785132
12.075 0.0430485617247672
12.1 0.0430433698042012
12.125 0.0430385130654874
12.15 0.0430342195914248
12.175 0.0430580342293255
12.2 0.0434998076854582
12.225 0.0439280870091652
12.25 0.0443260752601994
12.275 0.0447621182766088
12.3 0.0452286315919089
12.325 0.0456585740548785
12.35 0.0460873165065422
12.375 0.0468082521249068
12.4 0.0474884574507906
12.425 0.0480692356331978
12.45 0.0487030851365653
12.475 0.0494357376866815
12.5 0.0501734522620378
12.525 0.0509178335593052
12.55 0.0518041107829693
12.575 0.0527349466159507
12.6 0.0536015817320545
12.625 0.0543745074144306
12.65 0.055059105276401
12.675 0.0556384395708129
12.7 0.0561739133889976
12.725 0.0566290123719952
12.75 0.0570788498908901
12.775 0.0575069328796744
12.8 0.0579185273753245
12.825 0.0583081143924991
12.85 0.0589677034929508
12.875 0.0597386059486222
12.9 0.0604519316805775
12.925 0.0611829031081757
12.95 0.0620432750730388
12.975 0.0628687542441736
13 0.0636407259012504
13.025 0.0643709137382594
13.05 0.0650767506075755
13.075 0.0654418862401948
13.1 0.0657589135550939
13.125 0.0660563920168168
13.15 0.0663012881457246
13.175 0.0665287192000699
13.2 0.0667436120564862
13.225 0.0669373701808377
13.25 0.0670187288399794
13.275 0.0670976784848815
13.3 0.0671728205799505
13.325 0.0672403579218194
13.35 0.0673041471910935
13.375 0.0674602225329597
13.4 0.0676306043218253
13.425 0.0677836479075624
13.45 0.0677767629690307
13.475 0.0674691872315906
13.5 0.0671809819956593
13.525 0.0669058622683222
13.55 0.0666460893053094
13.575 0.0663911969919777
13.6 0.066148028492708
13.625 0.065665374836244
13.65 0.0649647122644543
13.675 0.0643821447457815
13.7 0.063818424008728
13.725 0.0633044708302146
13.75 0.0628022633074374
13.775 0.0623244178933219
13.8 0.0618786852304042
13.825 0.0614548332145521
13.85 0.0609721043269834
13.875 0.0604830874226082
13.9 0.060046837290523
13.925 0.0593009513132076
13.95 0.0578407876879546
13.975 0.05640240648227
14 0.0549871407067252
14.025 0.0529337578939867
14.05 0.0507241905310118
14.075 0.0486924571786
14.1 0.0466627431544503
14.125 0.0446376171267591
14.15 0.0420594290732335
14.175 0.0393494552190748
14.2 0.036997816441832
14.225 0.0346334153623516
14.25 0.0322540265713424
14.275 0.0300525501259409
14.3 0.0279227646148946
14.325 0.02633288595748
14.35 0.0247913250028745
14.375 0.0233572094618425
14.4 0.0218878628412901
14.425 0.0203719225292919
14.45 0.0189770933126975
14.475 0.0176377354396754
14.5 0.0161265169179041
14.525 0.0146115336150731
14.55 0.0131918523870971
14.575 0.0118372369455962
14.6 0.0105825432032133
14.625 0.00953752378201326
14.65 0.00861394467033343
14.675 0.00770781694864297
14.7 0.00690638149283064
14.725 0.00633150032643828
14.75 0.00582248429097655
14.775 0.00535578434683633
14.8 0.00516242502016178
14.825 0.00497341955936286
14.85 0.00479461859340614
14.875 0.00481994467359351
14.9 0.00499304487841015
14.925 0.00515332843695988
14.95 0.00531003467781676
14.975 0.00545481655105457
15 0.00565825151189213
15.025 0.0060058434947606
15.05 0.00633464526942826
15.075 0.00664587968349768
15.1 0.00718605158106193
15.125 0.00780872963419373
15.15 0.00836351137270913
15.175 0.00896207023926997
15.2 0.00970811483589581
15.225 0.0104344962777902
15.25 0.0111434056427396
15.275 0.013006982735649
15.3 0.0148616416770716
15.325 0.0164573676593931
15.35 0.0179965315752547
15.375 0.0194816235576886
15.4 0.0211628639409793
15.425 0.0228999076657065
15.45 0.024503271503943
15.475 0.0261492302020307
15.5 0.0279205082522297
15.525 0.0295826320369752
15.55 0.0311395924308808
15.575 0.033015947579894
15.6 0.0350161521327148
15.625 0.0367659482534773
15.65 0.0385251870801428
15.675 0.0403507541256717
15.7 0.0421320074331953
15.725 0.0438526998026403
15.75 0.0455151994286943
15.775 0.0470630959818971
15.8 0.0481097569008054
15.825 0.0490824657096764
15.85 0.0499754294676562
15.875 0.0503340540938549
15.9 0.0504763179045789
15.925 0.0506043250394593
15.95 0.0506268533340045
15.975 0.0501875513532176
16 0.0497616751560622
16.025 0.0493541042375007
16.05 0.0489889389262982
16.075 0.0486407399153824
16.1 0.047912264748648
16.125 0.0471598430194179
16.15 0.0464731654373724
16.175 0.0455959808584437
16.2 0.0444209548265299
16.225 0.0433072962796525
16.25 0.0422294250745643
16.275 0.041068218450726
16.3 0.039926521255374
16.325 0.0388478873867961
16.35 0.0379059822695427
16.375 0.0369827433928762
16.4 0.0356286188487179
16.425 0.0343258951636832
16.45 0.033126594469852
16.475 0.0316961781448336
16.5 0.0300959832406023
16.525 0.0286031572440493
16.55 0.0271349531868337
16.575 0.0253378293550722
16.6 0.0236043302055387
16.625 0.0220288761150735
16.65 0.0203102417040775
16.675 0.0184551841650977
16.7 0.0167116421639243
16.725 0.0150084776103158
16.75 0.0134851901464039
16.775 0.0118604822710618
16.8 0.00990289040125165
16.825 0.00797321822543921
16.85 0.0060704936950458
16.875 0.00379998437945094
16.9 0.00162408895184712
16.925 0.000245865153594583
16.95 0.00240660724797677
16.975 0.00484476589169738
17 0.00700347978626173
17.025 0.00909097880452181
17.05 0.0110490655101411
17.075 0.0134289769863933
17.1 0.0169859010803222
17.125 0.0203298323714284
17.15 0.0234371628231518
17.175 0.0275366954781984
17.2 0.0322207771845197
17.225 0.0360946233634679
17.25 0.040214177071323
17.275 0.0454134974249623
17.3 0.0504940301999144
17.325 0.0554170296601922
17.35 0.0638945718094495
17.375 0.0733348308017881
17.4 0.0816925501081944
17.425 0.0893846255933985
17.45 0.0963870996339837
17.475 0.104509966617721
17.5 0.113099211968271
17.525 0.120723317736903
17.55 0.128625448197616
17.575 0.137864406533366
17.6 0.146693268925681
17.625 0.154988684624959
17.65 0.16624007817892
17.675 0.17891559817192
17.7 0.18998163010242
17.725 0.201130304091572
17.75 0.215205779736734
17.775 0.228544401617479
17.8 0.240717148713732
17.825 0.254235333349373
17.85 0.268536806821565
17.875 0.282619816047361
17.9 0.296246177092236
17.925 0.308967321872066
17.95 0.322831380807163
17.975 0.336959959361066
18 0.349366311306964
18.025 0.363556222374015
18.05 0.38159758738865
18.075 0.398643013982791
18.1 0.414887186816871
18.125 0.43138938990489
18.15 0.447959006167044
18.175 0.464448236773233
18.2 0.481187275945912
18.225 0.499982277460327
18.25 0.517767613526006
18.275 0.533980797074274
18.3 0.554374110347912
18.325 0.577620478494071
18.35 0.599394526666116
18.375 0.620705177661819
18.4 0.640854364312325
18.425 0.662229533663766
18.45 0.684794260392932
18.475 0.703303281326228
18.5 0.721685241353057
18.525 0.74229032131788
18.55 0.762184291334003
18.575 0.780775298601419
18.6 0.800707293464191
18.625 0.821543956046575
18.65 0.840722136652283
18.675 0.860035719166599
18.7 0.885315043432724
18.725 0.910378009987315
18.75 0.934803116271472
18.775 0.959894458605611
18.8 0.985514654623191
18.825 1.00964031491779
18.85 1.03315995314325
18.875 1.05401039049577
18.9 1.07643387801769
18.925 1.10230589982071
18.95 1.12586339217794
18.975 1.1495252625463
19 1.17800672438313
19.025 1.20450845108865
19.05 1.22882745579058
19.075 1.25379271018141
19.1 1.27902538517729
19.125 1.30184632927338
19.15 1.32482214297243
19.175 1.34902926496669
19.2 1.37220204797256
19.225 1.39399921058012
19.25 1.42512155897684
19.275 1.46081521962101
19.3 1.49340049171301
19.325 1.52517537765532
19.35 1.55414597141253
19.375 1.58426119207533
19.4 1.61619334445924
19.425 1.64648587358356
19.45 1.67581559583443
19.475 1.70666592771266
19.5 1.73694043521999
19.525 1.76560107700804
19.55 1.79389576885965
19.575 1.82151993731505
19.6 1.84466668616142
19.625 1.86849133225063
19.65 1.89384718226811
19.675 1.91772054760153
19.7 1.94040957520715
19.725 1.96132408954908
19.75 1.98201360974035
19.775 2.00882225542413
19.8 2.03494448148737
19.825 2.05919691305911
19.85 2.08394041833894
19.875 2.10915762546669
19.9 2.13308434371337
19.925 2.15651695414636
19.95 2.18128365422603
19.975 2.20559612711451
20 2.22835239344105
20.025 2.25514264469042
20.05 2.28595690345595
20.075 2.31362891343604
20.1 2.33936209987282
20.125 2.36139228918606
20.15 2.38456210882462
20.175 2.40882549247525
20.2 2.4304224555356
20.225 2.45131939249065
20.25 2.47488305184092
20.275 2.49762930602905
20.3 2.51809428234759
20.325 2.54160187725095
20.35 2.56851677140092
20.375 2.59364663847068
20.4 2.61803620638292
20.425 2.64353596369807
20.45 2.6686410441614
20.475 2.69181605965899
20.5 2.7160234694134
20.525 2.74156349687695
20.55 2.76616529269913
20.575 2.78993284910355
20.6 2.81094738337209
20.625 2.83187468575636
20.65 2.85271585895476
20.675 2.87288061758859
20.7 2.89275597180972
20.725 2.9152473475692
20.75 2.93751568501055
20.775 2.95871327898736
20.8 2.98002706505988
20.825 3.00149205290701
20.85 3.0215216279289
20.875 3.04084439709682
20.9 3.06090638651476
20.925 3.08074764931588
20.95 3.09887650015776
20.975 3.11629333253814
21 3.13260531129625
21.025 3.15125704631818
21.05 3.17125397687198
21.075 3.19071981627751
21.1 3.21034024552285
21.125 3.23039122409911
21.15 3.24829414060972
21.175 3.26539062500419
21.2 3.28192785877827
21.225 3.29804735680378
21.25 3.31325235953081
21.275 3.32720722587874
21.3 3.34030831946944
21.325 3.35523976459716
21.35 3.37059949780006
21.375 3.38486138139377
21.4 3.39884199533578
21.425 3.4122003497547
21.45 3.42496440782774
21.475 3.43716654030764
21.5 3.44935813923867
21.525 3.46154606254447
21.55 3.47255119517105
21.575 3.483651383132
21.6 3.49510922557977
21.625 3.5065112447136
21.65 3.51785746826286
21.675 3.52793942753779
21.7 3.53747237326957
21.725 3.54601855255526
21.75 3.55435540690651
21.775 3.56248421252031
21.8 3.56987520541131
21.825 3.57706956123351
21.85 3.58383704096053
21.875 3.59033227131328
21.9 3.59611894232224
21.925 3.60160042993954
21.95 3.60673477772107
21.975 3.61148867083501
22 3.61608715698155
22.025 3.61994122212596
22.05 3.62368658925056
22.075 3.62713659240528
22.1 3.63040615715488
22.125 3.63343061046277
22.15 3.63633913763346
22.175 3.63914459298881
22.2 3.64162587932398
22.225 3.64388395849529
22.25 3.64593274949318
22.275 3.6478783576382
22.3 3.64978319960428
22.325 3.65145866574443
22.35 3.65310263893143
22.375 3.65463508452507
22.4 3.65585572440297
22.425 3.6566689789565
22.45 3.65743317200153
22.475 3.65817214511083
22.5 3.65884513222553
22.525 3.65945065050725
22.55 3.65971589308469
22.575 3.65997010301879
22.6 3.6602075589015
22.625 3.65956800076006
22.65 3.65842215395071
22.675 3.65730509655788
22.7 3.65622287932382
22.725 3.65520647072733
22.75 3.65370430315619
22.775 3.65206553382523
22.8 3.65063888244137
22.825 3.64903776406497
22.85 3.64703197368698
22.875 3.64519131175082
22.9 3.64346741205732
22.925 3.64091219060076
22.95 3.63813811807561
22.975 3.63559439923262
23 3.63312600310578
23.025 3.63086557141359
23.05 3.62832136863562
23.075 3.6254903599518
23.1 3.62287093264515
23.125 3.62032445082831
23.15 3.61705321593707
23.175 3.61383150655683
23.2 3.61080720005285
23.225 3.60782162452561
23.25 3.60487373961777
23.275 3.59970978127363
23.3 3.59473280814744
23.325 3.59053210297617
23.35 3.58608223601134
23.375 3.58139059005117
23.4 3.57720955981001
23.425 3.57316547956068
23.45 3.56937946135319
23.475 3.56553454928535
23.5 3.5615237046182
23.525 3.55764964217185
23.55 3.55393574619875
23.575 3.55024901998613
23.6 3.54657373404605
23.625 3.54324296958245
23.65 3.53992099023105
23.675 3.53655704476945
23.7 3.53323667512785
23.725 3.52997986866198
23.75 3.52686586464606
23.775 3.52382829109783
23.8 3.52093911950931
23.825 3.51831340136208
23.85 3.51594477531742
23.875 3.51337588416051
23.9 3.51071823965078
23.925 3.50831825265762
23.95 3.5059816600345
23.975 3.50384971038052
24 3.50147619688393
24.025 3.49877642652999
24.05 3.49624561178001
24.075 3.49380015257154
24.1 3.49140224820227
24.125 3.48903832127165
24.15 3.48685276695079
24.175 3.48471640279898
24.2 3.48265522550026
24.225 3.48068945749553
24.25 3.47877883635321
24.275 3.47693677504693
24.3 3.475183824117
24.325 3.47360694280698
24.35 3.47232607900201
24.375 3.47113539386022
24.4 3.47010352634718
24.425 3.46910284528941
24.45 3.46819184404049
24.475 3.46786961278951
24.5 3.46849625104405
24.525 3.46907018005431
24.55 3.46961518183578
24.575 3.4703851541371
24.6 3.47064062350245
24.625 3.47121442386473
24.65 3.47175750407042
24.675 3.47225101306139
24.7 3.47291801923278
24.725 3.47371104958367
24.75 3.47446418817823
24.775 3.47519956812986
24.8 3.47588250191083
24.825 3.47735430006816
24.85 3.47958468220453
24.875 3.48139643026348
24.9 3.48325735979681
24.925 3.48576919585787
24.95 3.48819543855676
24.975 3.49046299162382
25 3.49311059290947
25.025 3.49602338563937
25.05 3.49871641027344
25.075 3.50135127818903
25.1 3.50384610835494
25.125 3.50625826682239
25.15 3.50848662451288
25.175 3.51060500134037
25.2 3.51263130116714
25.225 3.51456705098991
25.25 3.51647604761824
25.275 3.5182570417433
25.3 3.52022634179652
25.325 3.52260000727595
25.35 3.5247714895654
25.375 3.52682433860096
25.4 3.52860960616397
25.425 3.53037332293543
25.45 3.53211336711615
25.475 3.53376528156973
25.5 3.53537844840413
25.525 3.53710634896218
25.55 3.5388147688877
25.575 3.54041307649245
25.6 3.54184059137809
25.625 3.54304428077412
25.65 3.54416544858249
25.675 3.54524653788518
25.7 3.54612786212184
25.725 3.54697723668819
25.75 3.54775507340155
25.775 3.54850259784782
25.8 3.54920194180864
25.825 3.5495123861067
25.85 3.54958986244664
25.875 3.54966447294611
25.9 3.54936797764758
25.925 3.54832228966534
25.95 3.54739830459951
25.975 3.54652339047888
26 3.54576580823803
26.025 3.5449255504019
26.05 3.54389741835788
26.075 3.54293889437659
26.1 3.54203399112837
26.125 3.54110179562472
26.15 3.54016555657064
26.175 3.53932507174916
26.2 3.53835602165534
26.225 3.5370449782747
26.25 3.53580246172816
26.275 3.53462525299166
26.3 3.53319721081728
26.325 3.53168413149619
26.35 3.53032453749652
26.375 3.52899030906867
26.4 3.52773451387777
26.425 3.52592642833538
26.45 3.52362455024413
26.475 3.52153914082137
26.5 3.51943137978876
26.525 3.51723795589095
26.55 3.51510471490144
26.575 3.51302924673528
26.6 3.51084150760613
26.625 3.50862393471973
26.65 3.50655324415787
26.675 3.50454671154637
26.7 3.50272574522079
26.725 3.50087824365245
26.75 3.4990014423737
26.775 3.49723474732509
26.8 3.49551171293634
26.825 3.49401737621531
26.85 3.49255100032006
26.875 3.49118902101999
26.9 3.48986456177379
26.925 3.48858951865825
26.95 3.48734146442899
26.975 3.48613210016993
27 3.48507660417368
27.025 3.4838393327183
27.05 3.48241672585918
27.075 3.48107236706797
27.1 3.47976180120253
27.125 3.4783639918866
27.15 3.47698960538083
27.175 3.47573801487414
27.2 3.47447312786023
27.225 3.47318988023389
27.25 3.47204393075587
27.275 3.47095295390784
27.3 3.46993327702716
27.325 3.46907744070463
27.35 3.46905874974472
27.375 3.46904076013782
27.4 3.46902386970421
27.425 3.46944176940057
27.45 3.47011550803733
27.475 3.47076007921161
27.5 3.47158904298136
27.525 3.47280455273786
27.55 3.47386829502034
27.575 3.47487384668433
27.6 3.47574941934699
27.625 3.47666659257971
27.65 3.47767628730032
27.675 3.47863401448282
27.7 3.47954897946335
27.725 3.48048408349574
27.75 3.48141728073758
27.775 3.48223841605713
27.8 3.48307160146108
27.825 3.48393815594595
27.85 3.4847689051995
27.875 3.48556537427003
27.9 3.48627177371304
27.925 3.48695459721023
27.95 3.48890305362923
27.975 3.49082686472783
28 3.4926205257857
28.025 3.49484060544286
28.05 3.49749095325152
28.075 3.49984477990067
28.1 3.50202850428343
28.125 3.50389898504672
28.15 3.50616382246591
28.175 3.50879865946545
28.2 3.51119354149216
28.225 3.51351710954457
28.25 3.51587074132355
28.275 3.51815921938188
28.3 3.52024277369514
28.325 3.52218165886842
28.35 3.52395387158399
28.375 3.5256460008925
28.4 3.5273028590165
28.425 3.52838104472281
28.45 3.52943786766786
28.475 3.53041703320523
28.5 3.53139840007313
28.525 3.5323835200528
28.55 3.53328871386237
28.575 3.53415569899106
28.6 3.53494287722621
28.625 3.5356553813859
28.65 3.53629645710441
28.675 3.53688983072464
28.7 3.53746371857548
28.725 3.53821943070524
28.75 3.53897711129404
28.775 3.53965634181858
28.8 3.54030460340061
28.825 3.54091243341605
28.85 3.54155511462708
28.875 3.54221522852116
28.9 3.54281621375726
28.925 3.54335237955789
28.95 3.54355208490384
28.975 3.5437460070105
29 3.54393008028148
29.025 3.54352574353661
29.05 3.54278019467555
29.075 3.54206031757581
29.1 3.54136347052101
29.125 3.54071006414418
29.15 3.53978076324413
29.175 3.5387679461035
29.2 3.53789560519088
29.225 3.53694605371515
29.25 3.53582396278707
29.275 3.53476262084749
29.3 3.53375055359337
29.325 3.5330896639109
29.35 3.53251895115579
29.375 3.53200163442398
29.4 3.53149935202163
29.425 3.53103681189138
29.45 3.53053323122158
29.475 3.52998928964645
29.5 3.52950733436717
29.525 3.52904184289721
29.55 3.52773904186508
29.575 3.52644544214957
29.6 3.52522072320101
29.625 3.52357461403269
29.65 3.52153907612803
29.675 3.5196875240058
29.7 3.51793932297582
29.725 3.516366234688
29.75 3.51479013741149
29.775 3.51321253605616
29.8 3.51171535446417
29.825 3.5102389258294
29.85 3.5085495241046
29.875 3.50689936239307
29.9 3.50538001025895
29.925 3.5039331558997
29.95 3.50256743045042
29.975 3.50127584299958
30 3.5000150314059
30.025 3.49892623197581
30.05 3.49790066646353
30.075 3.49713751769503
30.1 3.4963797784626
30.125 3.49563024867335
30.15 3.49473942269398
30.175 3.49381903764517
30.2 3.49298608801487
30.225 3.49201952562981
30.25 3.49093809715048
30.275 3.49000340076219
30.3 3.48909867766884
30.325 3.48827198978768
30.35 3.48728162567061
30.375 3.48597169236353
30.4 3.48474515411297
30.425 3.48357611606908
30.45 3.4821475646733
30.475 3.48067344877016
30.5 3.47933456698784
30.525 3.47799676846436
30.55 3.47666175322514
30.575 3.47534813249428
30.6 3.47406801528608
30.625 3.4726146135487
30.65 3.47105269515924
30.675 3.469574881804
30.7 3.46813509796942
30.725 3.46676239541105
30.75 3.46460183729979
30.775 3.46222840655845
30.8 3.46015967865129
30.825 3.45791183519143
30.85 3.45525323906634
30.875 3.45274117817281
30.9 3.45034880884389
30.925 3.44813451051851
30.95 3.44592432851682
30.975 3.44302451396429
31 3.44019846652735
31.025 3.43756363446017
31.05 3.43392907578462
31.075 3.42936196444322
31.1 3.42528350988953
31.125 3.42133452992544
31.15 3.41649852993306
31.175 3.41173247921026
31.2 3.40722152019811
31.225 3.40280349462376
31.25 3.39846889378193
31.275 3.393173178719
31.3 3.38805158564211
31.325 3.38341594261604
31.35 3.37701777644293
31.375 3.36975805134091
31.4 3.36311887176154
31.425 3.35633695295517
31.45 3.34845562091313
31.475 3.34086606317875
31.5 3.33372172938866
31.525 3.32567349611873
31.55 3.31713205615053
31.575 3.30931933138679
31.6 3.30122433501184
31.625 3.28977484928827
31.65 3.27840504322446
31.675 3.26736221887234
31.7 3.25657966683319
31.725 3.24616305896993
31.75 3.23389099563917
31.775 3.22100652254282
31.8 3.20901710103864
31.825 3.19663378041891
31.85 3.18393534129983
31.875 3.17242361154262
31.9 3.160561083381
31.925 3.14414139603487
31.95 3.12827035239722
31.975 3.11342256174471
32 3.0994782337967
32.025 3.08614822053823
32.05 3.07110210312872
32.075 3.05582953318763
32.1 3.04194499036854
32.125 3.02729472461917
32.15 3.01095842831005
32.175 2.99547503381106
32.2 2.98069552414978
32.225 2.96042748743337
32.25 2.93887849489365
32.275 2.91945239237163
32.3 2.89859086661435
32.325 2.87488734798859
32.35 2.85331748217955
32.375 2.83303202492041
32.4 2.8156030268306
32.425 2.79755720943338
32.45 2.7788255147909
32.475 2.76129995655423
32.5 2.74428571489395
32.525 2.72543499228396
32.55 2.70684277523795
32.575 2.68980201226006
32.6 2.67030015617786
32.625 2.64752078665386
32.65 2.62611423618532
32.675 2.60541210054091
32.7 2.60918833424665
32.725 2.56003338519708
32.75 2.53953486639948
32.775 2.51825358370813
32.8 2.49576538581225
32.825 2.47408167702896
32.85 2.45288574983436
32.875 2.43244999679766
32.9 2.40946837310165
32.925 2.38164177258743
32.95 2.35753690403403
32.975 2.33441501816898
33 2.31044176405235
33.025 2.28697913213889
33.05 2.26467710803421
33.075 2.24212126028671
33.1 2.21939238959634
33.125 2.1987378816953
33.15 2.17621323208626
33.175 2.15357970655093
33.2 2.13149049968689
33.225 2.11085488541089
33.25 2.08672410467926
33.275 2.05910833937282
33.3 2.03446841653969
33.325 2.01063958718687
33.35 1.97664419353156
33.375 1.94305252980103
33.4 1.91094366478981
33.425 1.87974332094757
33.45 1.84932171008278
33.475 1.81943472927451
33.5 1.79181920471497
33.525 1.76856029430243
33.55 1.74227714094924
33.575 1.71292302945731
33.6 1.68849128603188
33.625 1.66537156377182
33.65 1.64146960901813
33.675 1.6174180951244
33.7 1.5935362686859
33.725 1.56949587280498
33.75 1.5443779369111
33.775 1.5202576823738
33.8 1.49770574830296
33.825 1.4736256787401
33.85 1.44865597996983
33.875 1.42472630617282
33.9 1.40151046443805
33.925 1.37953348667978
33.95 1.30435020088538
33.975 1.32813528307277
34 1.3050788580828
34.025 1.28154783903337
34.05 1.256944084154
34.075 1.23439657967048
34.1 1.21326730799578
34.125 1.19097127836411
34.15 1.16857266703384
34.175 1.14896819320139
34.2 1.12799564993756
34.225 1.10338708930238
34.25 1.0798979254598
34.275 1.05751224178821
34.3 1.03448030162168
34.325 1.01123576577745
34.35 0.99052429399778
34.375 0.970107762668252
34.4 0.950681911271828
34.425 0.929780607048498
34.45 0.90757244633141
34.475 0.887440868019283
34.5 0.867333588262886
34.525 0.847012386892439
34.55 0.828563816364588
34.575 0.8110497585659
34.6 0.789883127878166
34.625 0.768676720083586
34.65 0.749223143781047
34.675 0.729276373026001
34.7 0.708559863349167
34.725 0.689793369604557
34.75 0.671988106353826
34.775 0.655356560980265
34.8 0.638765109398681
34.825 0.621077004727509
34.85 0.60432341572922
34.875 0.589035921315935
34.9 0.571155561847501
34.925 0.551508294175912
34.95 0.533357381054118
34.975 0.515189236516231
35 0.495856741415309
35.025 0.4774021684172
35.05 0.459697646160834
35.075 0.445565954674604
35.1 0.431542793647069
35.125 0.416261067372524
35.15 0.401559708000822
35.175 0.387937812197724
35.2 0.372944267921576
35.225 0.356980176402963
35.25 0.342477646594658
35.275 0.328243159131682
35.3 0.313863266332165
35.325 0.299947039296751
35.35 0.287040652395106
35.375 0.274239253511002
35.4 0.261524043606119
35.425 0.249913034247832
35.45 0.23857048691983
35.475 0.226910870945729
35.5 0.215310078378534
35.525 0.203827297605311
35.55 0.192025902204609
35.575 0.180504892295186
35.6 0.170587049783392
35.625 0.161213236187488
35.65 0.152461151413508
35.675 0.143998485931415
35.7 0.135652808474393
35.725 0.128400980859076
35.75 0.121285965311757
35.775 0.114674750060279
35.8 0.108327522527182
35.825 0.102332737174582
35.85 0.096908612009724
35.875 0.0917578658289242
35.9 0.0871698810824815
35.925 0.0827341584724207
35.95 0.0790478584480782
35.975 0.0755342231077944
36 0.072287924536184
36.025 0.0695275688784588
36.05 0.0670551213032032
36.075 0.0646988023297341
36.1 0.0624067662943787
36.125 0.0602260957938459
36.15 0.0586279563068764
36.175 0.0572037616470194
36.2 0.055956188984885
36.225 0.054940470509046
36.25 0.0544659123367125
36.275 0.0540254285371651
36.3 0.0536106146211718
36.325 0.0536632995743997
36.35 0.0538400287111955
36.375 0.0539993668183689
36.4 0.054154347859754
36.425 0.054297577983898
36.45 0.0545987799409128
36.475 0.0550532282187999
36.5 0.0554858103614232
36.525 0.0559098119648317
36.55 0.0568355519594502
36.575 0.0577536918217105
36.6 0.0586336551025853
36.625 0.059749903198182
36.65 0.0610511889225896
36.675 0.062245645010947
36.7 0.0634141583346353
36.725 0.0645460946881252
36.75 0.0655711447516819
36.775 0.0665430106064401
36.8 0.0674310442299998
36.825 0.0683479000912137
36.85 0.0694661923889504
36.875 0.0705511394001264
36.9 0.0715846328239481
36.925 0.0728103839485714
36.95 0.0741282073676508
36.975 0.0753556572452934
37 0.076591239092667
37.025 0.0780033018606415
37.05 0.079378330177419
37.075 0.0806779052132153
37.1 0.082217604297365
37.125 0.0839247723030692
37.15 0.0855771665958319
37.175 0.0871933995846432
37.2 0.0886924649618426
37.225 0.0901854606757357
37.25 0.0916734331708779
37.275 0.092873027373224
37.3 0.0947894788077575
37.325 0.096158064709031
37.35 0.096193548954935
37.375 0.0971880834702321
37.4 0.0979895384690288
37.425 0.0986571355731403
37.45 0.0992730818351006
37.475 0.0997854767967812
37.5 0.0991370415547667
37.525 0.0984874551709728
37.55 0.0978340656561057
37.575 0.0969926725329499
37.6 0.0960190750824526
37.625 0.095089433608319
37.65 0.0941792126476118
37.675 0.0933628930928086
37.7 0.0923299998323924
37.725 0.0908799690167937
37.75 0.0895968678126332
37.775 0.0883120393145822
37.8 0.0867710532876744
37.825 0.0853238190984908
37.85 0.0839854317278287
37.875 0.0828118334402272
37.9 0.0817051502384574
37.925 0.0807408251696595
37.95 0.0797732812046076
37.975 0.0787803183215203
38 0.0778307561493616
38.025 0.0769385538238412
38.05 0.0761042664845148
38.075 0.0752985702970274
38.1 0.0727521948422332
38.125 0.0700803793887392
38.15 0.0676316894056146
38.175 0.0648017944105832
38.2 0.0613683585179966
38.225 0.0580755007505
38.25 0.0548645544579951
38.275 0.0507561958995987
38.3 0.0467134971751228
38.325 0.0428971249983246
38.35 0.0396052846761258
38.375 0.0363679146983763
38.4 0.0319390847478411
38.425 0.0276577249798974
38.45 0.0236935858805228
38.475 0.0196404875869742
38.5 0.0155154592219613
38.525 0.0116620781651776
38.55 0.00789038267005615
38.575 0.00427027508948957
38.6 0.000734468133665922
38.625 0.00257546078165048
38.65 0.00571741458153111
38.675 0.00869864891170328
38.7 0.0117750497198474
38.725 0.014882821104266
38.75 0.0177080984344937
38.775 0.0204920258160864
38.8 0.0231523092463983
38.825 0.0257415324296114
38.85 0.02827274878742
38.875 0.0306227817731826
38.9 0.0328608595710501
38.925 0.0348964874960468
38.95 0.03675251259068
38.975 0.0385162835014938
39 0.0396014037561721
39.025 0.0406062890210453
39.05 0.0415231396193089
39.075 0.0421920027061725
39.1 0.0424707628024162
39.125 0.0427229227070186
39.15 0.0429619720779232
39.175 0.0428964010574425
39.2 0.0427849951254011
39.225 0.0426841805235613
39.25 0.0424325079156703
39.275 0.0418941217783383
39.3 0.0413911269691768
39.325 0.0409121125673162
39.35 0.0404457003479271
39.375 0.0399246998334258
39.4 0.039147760447454
39.425 0.0383996467843651
39.45 0.037675573372633
39.475 0.0368473336350497
39.5 0.0360403783011563
39.525 0.0353067329517679
39.55 0.0342575272535809
39.575 0.0329080973134943
39.6 0.0317093226357303
39.625 0.0305503465021417
39.65 0.0294566067701922
39.675 0.0282604530392112
39.7 0.0267782257533704
39.725 0.0253466089045804
39.75 0.0239734502882566
39.775 0.0226232823989178
39.8 0.0212823421760754
39.825 0.0200765974840904
39.85 0.0189396187124743
39.875 0.018075924651015
39.9 0.0172288693117017
39.925 0.0164041670077174
39.95 0.0156194109108004
39.975 0.0148845564389049
40 0.014327025389697
40.025 0.0138056953552514
40.05 0.0133152549456281
40.075 0.0130768535061336
40.1 0.0128802283054872
40.125 0.0127031076265278
40.15 0.0124550872512488
40.175 0.0120733440500074
40.2 0.0117107605957766
40.225 0.0113618248699746
40.25 0.0109567742677314
40.275 0.0105440946113783
40.3 0.010165459401873
40.325 0.00979843201522314
40.35 0.00945780522813607
40.375 0.00919581477650779
40.4 0.00899942634591127
40.425 0.00881840966010503
40.45 0.00866531721333953
40.475 0.00907664704651874
40.5 0.00948561246915795
40.525 0.00989007009494812
40.55 0.0105398722362369
40.575 0.0112335105398985
40.6 0.0118338742306321
40.625 0.0124032078476351
40.65 0.012935368798304
40.675 0.0138355015552595
40.7 0.0148910871491464
40.725 0.0158025103120952
40.75 0.016862523966703
40.775 0.0185874570023137
40.8 0.0202404749274741
40.825 0.0217969076720808
40.85 0.0230964151424298
40.875 0.0242709778256596
40.9 0.025337959547241
40.925 0.0263780894300366
40.95 0.0273266052714329
40.975 0.028196803062957
41 0.0289442168654491
41.025 0.029663598717992
41.05 0.0303668867062307
41.075 0.0306782415881494
41.1 0.0309657289564923
41.125 0.0312404373842086
41.15 0.0313879076926793
41.175 0.0314983581805248
41.2 0.0315954615324617
41.225 0.0316899391324363
41.25 0.03177860088574
41.275 0.0318611801422544
41.3 0.0319392294269222
41.325 0.032012592251298
41.35 0.0320913654348882
41.375 0.0322619893879679
41.4 0.0324293614969951
41.425 0.0325876758193751
41.45 0.032738986017293
41.475 0.0328837026771028
41.5 0.0330124773656732
41.525 0.0331370995950733
41.55 0.033251776394515
41.575 0.0333648307950029
41.6 0.0334727372981813
41.625 0.033577924053141
41.65 0.0336808773200302
41.675 0.033424153669018
41.7 0.0331649268960064
41.725 0.0329285371775062
41.75 0.0325669812404885
41.775 0.0321444920077049
41.8 0.0317590695936748
41.825 0.0313666830491258
41.85 0.0309118984607449
41.875 0.0304708982810854
41.9 0.0300515541760494
41.925 0.0296759932256767
41.95 0.02932171572635
41.975 0.0288904093714641
42 0.0284524904131624
42.025 0.0280843836640323
42.05 0.0276751326518466
42.075 0.0270959745393589
42.1 0.0265260917923815
42.125 0.0259683889999963
42.15 0.0252626248699349
42.175 0.0245261389217737
42.2 0.0238447085055249
42.225 0.0232137737273139
42.25 0.0226249827965509
42.275 0.0214296636592117
42.3 0.0201595660206856
42.325 0.019010067001732
};
\addplot [line width=2.4000000000000004pt, color1, dashed, forget plot]
table {%
0 0.0937500001330113
0.025 0.0936190645892096
0.05 0.093488129045408
0.075 0.0933571935016064
0.1 0.0932262579578047
0.125 0.0930953224140031
0.15 0.0929643868702014
0.175 0.0928334513263998
0.2 0.0927025157825982
0.225 0.0925715802387965
0.25 0.0924406446949949
0.275 0.0923097091511932
0.3 0.0921787736073916
0.325 0.09204783806359
0.35 0.0919169025197883
0.375 0.0917859669759867
0.4 0.091655031432185
0.425 0.0915240958883834
0.45 0.0913931603445818
0.475 0.0912622248007801
0.5 0.0911312892569785
0.525 0.0910003537131768
0.55 0.0908694181693752
0.575 0.0907384826255735
0.6 0.0906075470817719
0.625 0.0904766115379703
0.65 0.0903456759941686
0.675 0.090214740450367
0.7 0.0900838049065654
0.725 0.0899528693627637
0.75 0.0898219338189621
0.775 0.0896909982751604
0.8 0.0895600627313588
0.825 0.0894291271875571
0.85 0.0892981916437555
0.875 0.0891672560999539
0.9 0.0890363205561522
0.925 0.0889053850123506
0.95 0.0887744494685489
0.975 0.0886435139247473
1 0.0885125783809457
1.025 0.088381642837144
1.05 0.0882507072933424
1.075 0.0881197717495407
1.1 0.0879888362057391
1.125 0.0878579006619375
1.15 0.0877269651181358
1.175 0.0875960295743342
1.2 0.0874650940305325
1.225 0.0873341584867309
1.25 0.0872032229429293
1.275 0.0870722873991276
1.3 0.086941351855326
1.325 0.0868104163115243
1.35 0.0866794807677227
1.375 0.0865485452239211
1.4 0.0864176096801194
1.425 0.0862866741363178
1.45 0.0861557385925161
1.475 0.0860248030487145
1.5 0.0858938675049128
1.525 0.0857629319611112
1.55 0.0856319964173096
1.575 0.0855010608735079
1.6 0.0853701253297063
1.625 0.0852391897859047
1.65 0.085108254242103
1.675 0.0849773186983014
1.7 0.0848463831544997
1.725 0.0847154476106981
1.75 0.0845845120668964
1.775 0.0844535765230948
1.8 0.0843226409792932
1.825 0.0841917054354915
1.85 0.0840607698916899
1.875 0.0839298343478882
1.9 0.0837988988040866
1.925 0.083667963260285
1.95 0.0835370277164833
1.975 0.0834060921726817
2 0.08327515662888
2.025 0.0831442210850784
2.05 0.0830132855412768
2.075 0.0828823499974751
2.1 0.0827514144536735
2.125 0.0826204789098718
2.15 0.0824895433660702
2.175 0.0823586078222686
2.2 0.0822276722784669
2.225 0.0820967367346653
2.25 0.0819658011908636
2.275 0.081834865647062
2.3 0.0817039301032604
2.325 0.0815729945594587
2.35 0.0814420590156571
2.375 0.0813111234718554
2.4 0.0811801879280538
2.425 0.0810492523842521
2.45 0.0809183168404505
2.475 0.0807873812966489
2.5 0.0806564457528472
2.525 0.0805255102090456
2.55 0.080394574665244
2.575 0.0802636391214423
2.6 0.0801327035776407
2.625 0.080001768033839
2.65 0.0798708324900374
2.675 0.0797398969462357
2.7 0.0796089614024341
2.725 0.0794780258586325
2.75 0.0793470903148308
2.775 0.0792161547710292
2.8 0.0790852192272275
2.825 0.0789542836834259
2.85 0.0788233481396243
2.875 0.0786924125958226
2.9 0.078561477052021
2.925 0.0784305415082193
2.95 0.0782996059644177
2.975 0.0781686704206161
3 0.0780377348768144
3.025 0.0779067993330128
3.05 0.0777758637892111
3.075 0.0776449282454095
3.1 0.0775139927016079
3.125 0.0773830571578062
3.15 0.0772521216140046
3.175 0.0771211860702029
3.2 0.0769902505264013
3.225 0.0768593149825997
3.25 0.076728379438798
3.275 0.0765974438949964
3.3 0.0764665083511947
3.325 0.0763355728073931
3.35 0.0762046372635915
3.375 0.0760737017197898
3.4 0.0759427661759882
3.425 0.0758118306321865
3.45 0.0756808950883849
3.475 0.0755499595445833
3.5 0.0754190240007816
3.525 0.07528808845698
3.55 0.0751571529131783
3.575 0.0750262173693767
3.6 0.074895281825575
3.625 0.0747643462817734
3.65 0.0746334107379718
3.675 0.0745024751941701
3.7 0.0743715396503685
3.725 0.0742406041065669
3.75 0.0741096685627652
3.775 0.0739787330189636
3.8 0.0738477974751619
3.825 0.0737168619313603
3.85 0.0735859263875586
3.875 0.073454990843757
3.9 0.0733240552999554
3.925 0.0731931197561537
3.95 0.0730621842123521
3.975 0.0729312486685504
4 0.0728003131247488
4.025 0.0726693775809472
4.05 0.0725384420371455
4.075 0.0724075064933439
4.1 0.0722765709495422
4.125 0.0721456354057406
4.15 0.072014699861939
4.175 0.0718837643181373
4.2 0.0717528287743357
4.225 0.071621893230534
4.25 0.0714909576867324
4.275 0.0713600221429308
4.3 0.0712290865991291
4.325 0.0710981510553275
4.35 0.0709672155115258
4.375 0.0708362799677242
4.4 0.0707053444239226
4.425 0.0705744088801209
4.45 0.0704434733363193
4.475 0.0703125377925176
4.5 0.070181602248716
4.525 0.0700506667049143
4.55 0.0699197311611127
4.575 0.0697887956173111
4.6 0.0696578600735094
4.625 0.0695269245297078
4.65 0.0693959889859062
4.675 0.0692650534421045
4.7 0.0691341178983029
4.725 0.0690031823545012
4.75 0.0688722468106996
4.775 0.0687413112668979
4.8 0.0686103757230963
4.825 0.0684794401792947
4.85 0.068348504635493
4.875 0.0682175690916914
4.9 0.0680866335478897
4.925 0.0679556980040881
4.95 0.0678247624602865
4.975 0.0676938269164848
5 0.0675628913726832
5.025 0.0674319558288815
5.05 0.0673010202850799
5.075 0.0671700847412783
5.1 0.0670391491974766
5.125 0.066908213653675
5.15 0.0667772781098733
5.175 0.0666463425660717
5.2 0.0665154070222701
5.225 0.0663844714784684
5.25 0.0662535359346668
5.275 0.0661226003908651
5.3 0.0659916648470635
5.325 0.0658607293032619
5.35 0.0657297937594602
5.375 0.0655988582156586
5.4 0.0654679226718569
5.425 0.0653369871280553
5.45 0.0652060515842537
5.475 0.065075116040452
5.5 0.0649441804966504
5.525 0.0648132449528487
5.55 0.0646823094090471
5.575 0.0645513738652455
5.6 0.0644204383214438
5.625 0.0642895027776422
5.65 0.0641585672338405
5.675 0.0640276316900389
5.7 0.0638966961462372
5.725 0.0637657606024356
5.75 0.063634825058634
5.775 0.0635038895148323
5.8 0.0633729539710307
5.825 0.063242018427229
5.85 0.0631110828834274
5.875 0.0629801473396258
5.9 0.0628492117958241
5.925 0.0627182762520225
5.95 0.0625873407082208
5.975 0.0624564051644192
6 0.0623254696206176
6.025 0.0621945340768159
6.05 0.0620635985330143
6.075 0.0619326629892126
6.1 0.061801727445411
6.125 0.0616707919016094
6.15 0.0615398563578077
6.175 0.0614089208140061
6.2 0.0612779852702044
6.225 0.0611470497264028
6.25 0.0610161141826012
6.275 0.0608851786387995
6.3 0.0607542430949979
6.325 0.0606233075511962
6.35 0.0604923720073946
6.375 0.060361436463593
6.4 0.0602305009197913
6.425 0.0600995653759897
6.45 0.059968629832188
6.475 0.0598376942883864
6.5 0.0597067587445848
6.525 0.0595758232007831
6.55 0.0594448876569815
6.575 0.0593139521131798
6.6 0.0591830165693782
6.625 0.0590520810255766
6.65 0.0589211454817749
6.675 0.0587902099379733
6.7 0.0586592743941716
6.725 0.05852833885037
6.75 0.0583974033065683
6.775 0.0582664677627667
6.8 0.0581355322189651
6.825 0.0580045966751634
6.85 0.0578736611313618
6.875 0.0577427255875601
6.9 0.0576117900437585
6.925 0.0574808544999569
6.95 0.0573499189561552
6.975 0.0572189834123536
7 0.0570880478685519
7.025 0.0569571123247503
7.05 0.0568261767809487
7.075 0.056695241237147
7.1 0.0565643056933454
7.125 0.0564333701495437
7.15 0.0563024346057421
7.175 0.0561714990619405
7.2 0.0560405635181388
7.225 0.0559096279743372
7.25 0.0557786924305355
7.275 0.0556477568867339
7.3 0.0555168213429323
7.325 0.0553858857991306
7.35 0.055254950255329
7.375 0.0551240147115273
7.4 0.0549930791677257
7.425 0.0548621436239241
7.45 0.0547312080801224
7.475 0.0546002725363208
7.5 0.0544693369925191
7.525 0.0543384014487175
7.55 0.0542074659049158
7.575 0.0540765303611142
7.6 0.0539455948173126
7.625 0.0538146592735109
7.65 0.0536837237297093
7.675 0.0535527881859076
7.7 0.053421852642106
7.725 0.0532909170983044
7.75 0.0531599815545027
7.775 0.0530290460107011
7.8 0.0528981104668994
7.825 0.0527671749230978
7.85 0.0526362393792962
7.875 0.0525053038354945
7.9 0.0523743682916929
7.925 0.0522434327478912
7.95 0.0521124972040896
7.975 0.051981561660288
8 0.0518506261164863
8.025 0.0517196905726847
8.05 0.051588755028883
8.075 0.0514578194850814
8.1 0.0513268839412798
8.125 0.0511959483974781
8.15 0.0510650128536765
8.175 0.0509340773098748
8.2 0.0508031417660732
8.225 0.0506722062222716
8.25 0.0505412706784699
8.275 0.0504103351346683
8.3 0.0502793995908666
8.325 0.050148464047065
8.35 0.0500175285032634
8.375 0.0498865929594617
8.4 0.0497556574156601
8.425 0.0496247218718584
8.45 0.0494937863280568
8.475 0.0493628507842552
8.5 0.0492319152404535
8.525 0.0491009796966519
8.55 0.0489700441528502
8.575 0.0488391086090486
8.6 0.048708173065247
8.625 0.0485772375214453
8.65 0.0484463019776437
8.675 0.048315366433842
8.7 0.0481844308900404
8.725 0.0480534953462388
8.75 0.0479225598024371
8.775 0.0477916242586355
8.8 0.0476606887148338
8.825 0.0475297531710322
8.85 0.0473988176272305
8.875 0.0472678820834289
8.9 0.0471369465396273
8.925 0.0470060109958256
8.95 0.046875075452024
8.975 0.0467441399082223
9 0.0466132043644207
9.025 0.0464822688206191
9.05 0.0463513332768174
9.075 0.0462203977330158
9.1 0.0460894621892141
9.125 0.0459585266454125
9.15 0.0458275911016109
9.175 0.0456966555578092
9.2 0.0455657200140076
9.225 0.0454347844702059
9.25 0.0453038489264043
9.275 0.0451729133826027
9.3 0.045041977838801
9.325 0.0449110422949994
9.35 0.0447801067511977
9.375 0.0446491712073961
9.4 0.0445182356635945
9.425 0.0443873001197928
9.45 0.0442563645759912
9.475 0.0441254290321895
9.5 0.0439944934883879
9.525 0.0438635579445863
9.55 0.0437326224007846
9.575 0.043601686856983
9.6 0.0434707513131813
9.625 0.0433398157693797
9.65 0.043208880225578
9.675 0.0430779446817764
9.7 0.0429470091379748
9.725 0.0428160735941731
9.75 0.0426851380503715
9.775 0.0425542025065698
9.8 0.0424232669627682
9.825 0.0422923314189666
9.85 0.0421613958751649
9.875 0.0420304603313633
9.9 0.0418995247875616
9.925 0.04176858924376
9.95 0.0416376536999584
9.975 0.0415067181561567
10 0.0413757826123551
10.025 0.0412448470685534
10.05 0.0411139115247518
10.075 0.0409829759809502
10.1 0.0408520404371485
10.125 0.0407211048933469
10.15 0.0405901693495452
10.175 0.0404592338057436
10.2 0.040328298261942
10.225 0.0401973627181403
10.25 0.0400664271743387
10.275 0.039935491630537
10.3 0.0398045560867354
10.325 0.0396736205429338
10.35 0.0395426849991321
10.375 0.0394117494553305
10.4 0.0392808139115288
10.425 0.0391498783677272
10.45 0.0390189428239256
10.475 0.0388880072801239
10.5 0.0387570717363223
10.525 0.0386261361925206
10.55 0.038495200648719
10.575 0.0383642651049173
10.6 0.0382333295611157
10.625 0.0381023940173141
10.65 0.0379714584735124
10.675 0.0378405229297108
10.7 0.0377095873859091
10.725 0.0375786518421075
10.75 0.0374477162983059
10.775 0.0373167807545042
10.8 0.0371858452107026
10.825 0.0370549096669009
10.85 0.0369239741230993
10.875 0.0367930385792977
10.9 0.036662103035496
10.925 0.0365311674916944
10.95 0.0364002319478927
10.975 0.0362692964040911
11 0.0361383608602895
11.025 0.0360074253164878
11.05 0.0358764897726862
11.075 0.0357455542288845
11.1 0.0356146186850829
11.125 0.0354836831412813
11.15 0.0353527475974796
11.175 0.035221812053678
11.2 0.0350908765098763
11.225 0.0349599409660747
11.25 0.0348290054222731
11.275 0.0346980698784714
11.3 0.0345671343346698
11.325 0.0344361987908681
11.35 0.0343052632470665
11.375 0.0341743277032649
11.4 0.0340433921594632
11.425 0.0339124566156616
11.45 0.0337815210718599
11.475 0.0336505855280583
11.5 0.0335196499842567
11.525 0.033388714440455
11.55 0.0332577788966534
11.575 0.0331268433528517
11.6 0.0329959078090501
11.625 0.0328649722652485
11.65 0.0327340367214468
11.675 0.0326031011776452
11.7 0.0324721656338435
11.725 0.0323412300900419
11.75 0.0322102945462402
11.775 0.0320793590024386
11.8 0.031948423458637
11.825 0.0318174879148353
11.85 0.0316865523710337
11.875 0.031555616827232
11.9 0.0314246812834304
11.925 0.0312937457396288
11.95 0.0311628101958271
11.975 0.0310318746520255
12 0.0309009391082238
12.025 0.0307700035644222
12.05 0.0306390680206206
12.075 0.0305081324768189
12.1 0.0303771969330173
12.125 0.0302462613892156
12.15 0.030115325845414
12.175 0.0299843903016124
12.2 0.0298534547578107
12.225 0.0297225192140091
12.25 0.0295915836702074
12.275 0.0294606481264058
12.3 0.0293297125826042
12.325 0.0291987770388025
12.35 0.0290678414950009
12.375 0.0289369059511992
12.4 0.0288059704073976
12.425 0.0286750348635959
12.45 0.0285440993197943
12.475 0.0284131637759927
12.5 0.028282228232191
12.525 0.0281512926883894
12.55 0.0280203571445878
12.575 0.0278894216007861
12.6 0.0277584860569845
12.625 0.0276275505131828
12.65 0.0274966149693812
12.675 0.0273656794255795
12.7 0.0272347438817779
12.725 0.0271038083379763
12.75 0.0269728727941746
12.775 0.026841937250373
12.8 0.0267110017065713
12.825 0.0265800661627697
12.85 0.0264491306189681
12.875 0.0263181950751664
12.9 0.0261872595313648
12.925 0.0260563239875631
12.95 0.0259253884437615
12.975 0.0257944528999599
13 0.0256635173561582
13.025 0.0255325818123566
13.05 0.0254016462685549
13.075 0.0252707107247533
13.1 0.0251397751809517
13.125 0.02500883963715
13.15 0.0248779040933484
13.175 0.0247469685495467
13.2 0.0246160330057451
13.225 0.0244850974619435
13.25 0.0243541619181418
13.275 0.0242232263743402
13.3 0.0240922908305385
13.325 0.0239613552867369
13.35 0.0238304197429352
13.375 0.0236994841991336
13.4 0.023568548655332
13.425 0.0234376131115303
13.45 0.0233066775677287
13.475 0.0231757420239271
13.5 0.0230448064801254
13.525 0.0229138709363238
13.55 0.0227829353925221
13.575 0.0226519998487205
13.6 0.0225210643049188
13.625 0.0223901287611172
13.65 0.0222591932173156
13.675 0.0221282576735139
13.7 0.0219973221297123
13.725 0.0218663865859106
13.75 0.021735451042109
13.775 0.0216045154983074
13.8 0.0214735799545057
13.825 0.0213426444107041
13.85 0.0212117088669024
13.875 0.0210807733231008
13.9 0.0209498377792992
13.925 0.0208189022354975
13.95 0.0206879666916959
13.975 0.0205570311478942
14 0.0204260956040926
14.025 0.020295160060291
14.05 0.0201642245164893
14.075 0.0200332889726877
14.1 0.019902353428886
14.125 0.0197714178850844
14.15 0.0196404823412828
14.175 0.0195095467974811
14.2 0.0193786112536795
14.225 0.0192476757098778
14.25 0.0191167401660762
14.275 0.0189858046222746
14.3 0.0188548690784729
14.325 0.0187239335346713
14.35 0.0185929979908696
14.375 0.018462062447068
14.4 0.0183311269032664
14.425 0.0182001913594647
14.45 0.0180692558156631
14.475 0.0179383202718614
14.5 0.0178073847280598
14.525 0.0176764491842581
14.55 0.0175455136404565
14.575 0.0174145780966549
14.6 0.0172836425528532
14.625 0.0171527070090516
14.65 0.01702177146525
14.675 0.0168908359214483
14.7 0.0167599003776467
14.725 0.016628964833845
14.75 0.0164980292900434
14.775 0.0163670937462417
14.8 0.0162361582024401
14.825 0.0161052226586385
14.85 0.0159742871148368
14.875 0.0158433515710352
14.9 0.0157124160272335
14.925 0.0155814804834319
14.95 0.0154505449396303
14.975 0.0153196093958286
15 0.015188673852027
15.025 0.0150577383082253
15.05 0.0149268027644237
15.075 0.0147958672206221
15.1 0.0146649316768204
15.125 0.0145339961330188
15.15 0.0144030605892171
15.175 0.0142721250454155
15.2 0.0141411895016139
15.225 0.0140102539578122
15.25 0.0138793184140106
15.275 0.0137483828702089
15.3 0.0136174473264073
15.325 0.0134865117826057
15.35 0.013355576238804
15.375 0.0132246406950024
15.4 0.0130937051512007
15.425 0.0129627696073991
15.45 0.0128318340635974
15.475 0.0127008985197958
15.5 0.0125699629759942
15.525 0.0124390274321925
15.55 0.0123080918883909
15.575 0.0121771563445893
15.6 0.0120462208007876
15.625 0.011915285256986
15.65 0.0117843497131843
15.675 0.0116534141693827
15.7 0.011522478625581
15.725 0.0113915430817794
15.75 0.0112606075379778
15.775 0.0111296719941761
15.8 0.0109987364503745
15.825 0.0108678009065728
15.85 0.0107368653627712
15.875 0.0106059298189696
15.9 0.0104749942751679
15.925 0.0103440587313663
15.95 0.0102131231875646
15.975 0.010082187643763
16 0.00995125209996137
16.025 0.00982031655615971
16.05 0.00968938101235808
16.075 0.00955844546855644
16.1 0.0094275099247548
16.125 0.00929657438095316
16.15 0.00916563883715152
16.175 0.00903470329334988
16.2 0.00890376774954825
16.225 0.00877283220574659
16.25 0.00864189666194497
16.275 0.00851096111814331
16.3 0.00838002557434167
16.325 0.00824909003054004
16.35 0.00811815448673839
16.375 0.00798721894293676
16.4 0.00785628339913511
16.425 0.00772534785533348
16.45 0.00759441231153184
16.475 0.0074634767677302
16.5 0.00733254122392855
16.525 0.0072016056801269
16.55 0.00707067013632527
16.575 0.00693973459252364
16.6 0.00680879904872199
16.625 0.00667786350492035
16.65 0.00654692796111871
16.675 0.00641599241731707
16.7 0.00628505687351544
16.725 0.00615412132971378
16.75 0.00602318578591216
16.775 0.0058922502421105
16.8 0.00576131469830887
16.825 0.00563037915450723
16.85 0.00549944361070558
16.875 0.00536850806690395
16.9 0.00523757252310229
16.925 0.00510663697930067
16.95 0.00497570143549902
16.975 0.00484476589169738
17 0.00700347978626206
17.025 0.00578320420254621
17.05 0.00512201173751309
17.075 0.00501416250648042
17.1 0.00545391662476623
17.125 0.00643553420768833
17.15 0.00795327537056491
17.175 0.0100014002287135
17.2 0.0125741688974521
17.225 0.0156658414920993
17.25 0.019270678127972
17.275 0.0233829389203892
17.3 0.0279968839846676
17.325 0.0331067734361258
17.35 0.0387068673900825
17.375 0.0447914259618539
17.4 0.0513547092667599
17.425 0.0583909774201164
17.45 0.0658944905372423
17.475 0.0738595087334566
17.5 0.0822802921240751
17.525 0.091151100824418
17.55 0.100466194949801
17.575 0.110219834615542
17.6 0.120406279936963
17.625 0.131019791029376
17.65 0.142054628008104
17.675 0.153505050988461
17.7 0.165365320085767
17.725 0.177629695415341
17.75 0.190292437092498
17.775 0.20334780523256
17.8 0.216790059950839
17.825 0.230613461362658
17.85 0.244812269583334
17.875 0.259380744728182
17.9 0.274313146912525
17.925 0.289603736251676
17.95 0.305246772860954
17.975 0.321236516855681
18 0.337567228351169
18.025 0.354233167462741
18.05 0.371228594305711
18.075 0.388547768995398
18.1 0.406184951647123
18.125 0.424134402376199
18.15 0.44239038129795
18.175 0.460947148527686
18.2 0.47979896418073
18.225 0.498940088372402
18.25 0.518364781218014
18.275 0.53806730283289
18.3 0.558041913332342
18.325 0.578282872831691
18.35 0.598784441446258
18.375 0.619540879291355
18.4 0.640546446482305
18.425 0.661795403134421
18.45 0.683282009363023
18.475 0.705000525283432
18.5 0.726945211010961
18.525 0.749110326660933
18.55 0.771490132348661
18.575 0.794078888189464
18.6 0.816870854298666
18.625 0.839860290791575
18.65 0.863041457783518
18.675 0.886408615389805
18.7 0.909956023725758
18.725 0.933677942906698
18.75 0.957568633047937
18.775 0.981622354264798
18.8 1.00583336667259
18.825 1.03019593038664
18.85 1.05470430552227
18.875 1.07935275219479
18.9 1.10413553051952
18.925 1.12904690061177
18.95 1.15408112258686
18.975 1.17923245656013
19 1.20449516264687
19.025 1.22986350096241
19.05 1.25533173162207
19.075 1.28089411474116
19.1 1.30654491043501
19.125 1.33227837881892
19.15 1.35808878000823
19.175 1.38397037411824
19.2 1.40991742126427
19.225 1.43592418156165
19.25 1.46198491512568
19.275 1.4880938820717
19.3 1.51424534251501
19.325 1.54043355657094
19.35 1.56665278435479
19.375 1.5928972859819
19.4 1.61916132156758
19.425 1.64543915122713
19.45 1.6717250350759
19.475 1.69801323322918
19.5 1.72429800580231
19.525 1.75057361291059
19.55 1.77683431466935
19.575 1.8030743711939
19.6 1.82928804259956
19.625 1.85546958900165
19.65 1.88161327051549
19.675 1.90771334725639
19.7 1.93376407933968
19.725 1.95975972688067
19.75 1.98569454999467
19.775 2.01156280879701
19.8 2.03735876340301
19.825 2.06307667392798
19.85 2.08871080048724
19.875 2.1142554031961
19.9 2.1397047421699
19.925 2.16505307752394
19.95 2.19029466937354
19.975 2.21542377783402
20 2.24043473011387
20.025 2.26532339656456
20.05 2.29008719068056
20.075 2.31472359304947
20.1 2.33923008425894
20.125 2.36360414489659
20.15 2.38784325555004
20.175 2.41194489680692
20.2 2.43590654925487
20.225 2.4597256934815
20.25 2.48339981007444
20.275 2.50692637962133
20.3 2.53030288270978
20.325 2.55352679992742
20.35 2.57659561186189
20.375 2.5995067991008
20.4 2.62225784223179
20.425 2.64484622184248
20.45 2.66726941852051
20.475 2.68952491285348
20.5 2.71161018542904
20.525 2.73352271683481
20.55 2.75525998765842
20.575 2.77681947848749
20.6 2.79819866990966
20.625 2.81939504251253
20.65 2.84040607688376
20.675 2.86122925361096
20.7 2.88186205328176
20.725 2.90230195648378
20.75 2.92254644380465
20.775 2.94259299583201
20.8 2.96243909315347
20.825 2.98208221635667
20.85 3.00151984602923
20.875 3.02074946275877
20.9 3.03976854713293
20.925 3.05857457973933
20.95 3.0771650411656
20.975 3.09553741199937
21 3.11368917282826
21.025 3.1316178042399
21.05 3.14932078682191
21.075 3.16679560116193
21.1 3.18403972784758
21.125 3.20105064746649
21.15 3.21782584060628
21.175 3.23436278785458
21.2 3.25065896979903
21.225 3.26671186702724
21.25 3.28251896012684
21.275 3.29807772968546
21.3 3.31338565629072
21.325 3.32844022053026
21.35 3.3432389029917
21.375 3.35777918426267
21.4 3.3720585449308
21.425 3.3860744655837
21.45 3.39982442680901
21.475 3.41330590919436
21.5 3.42651639332737
21.525 3.43945335979567
21.55 3.45211428918689
21.575 3.46449666208865
21.6 3.47659795908858
21.625 3.48841566077431
21.65 3.49994724773346
21.675 3.51119020055366
21.7 3.52214199982255
21.725 3.53280012612773
21.75 3.54316206005685
21.775 3.55322528219753
21.8 3.56298727313739
21.825 3.57244551346407
21.85 3.58159748376519
21.875 3.59044066462837
21.9 3.59897253664125
21.925 3.60719058039145
21.95 3.6150922764666
21.975 3.62267510545433
22 3.62993654794225
22.025 3.63687408451801
22.05 3.64348519576922
22.075 3.64976736228351
22.1 3.65571806464852
22.125 3.66133478345186
22.15 3.66661499928116
22.175 3.67155619272406
22.2 3.67615584436817
22.225 3.68041143480113
22.25 3.68432044461056
22.275 3.68788035438409
22.3 3.69108864470935
22.325 3.69394279617395
22.35 3.69644028936554
22.375 3.69857860487174
22.4 3.70035522328017
22.425 3.70176762517845
22.45 3.70281329115423
22.475 3.70348970179512
22.5 3.70379433768876
22.525 3.70372467942276
22.55 3.70327820758476
22.575 3.70245240276238
22.6 3.70124474554325
22.625 3.699652716515
22.65 3.69767379626525
22.675 3.69530546538163
22.7 3.69254520445177
22.725 3.68939049406329
22.75 3.68583881480383
22.775 3.681887647261
22.8 3.67753447202244
22.825 3.67277676967577
22.85 3.66761202080862
22.875 3.66203770600862
22.9 3.65605130586339
22.925 3.64965030096056
22.95 3.64283217188776
22.975 3.63559439923261
23 3.63312600310578
23.025 3.63242593469094
23.05 3.6317258662761
23.075 3.63102579786126
23.1 3.63032572944642
23.125 3.62962566103158
23.15 3.62892559261674
23.175 3.62822552420191
23.2 3.62752545578707
23.225 3.62682538737223
23.25 3.62612531895739
23.275 3.62542525054255
23.3 3.62472518212771
23.325 3.62402511371287
23.35 3.62332504529803
23.375 3.62262497688319
23.4 3.62192490846836
23.425 3.62122484005352
23.45 3.62052477163868
23.475 3.61982470322384
23.5 3.619124634809
23.525 3.61842456639416
23.55 3.61772449797932
23.575 3.61702442956448
23.6 3.61632436114964
23.625 3.6156242927348
23.65 3.61492422431997
23.675 3.61422415590513
23.7 3.61352408749029
23.725 3.61282401907545
23.75 3.61212395066061
23.775 3.61142388224577
23.8 3.61072381383093
23.825 3.61002374541609
23.85 3.60932367700125
23.875 3.60862360858642
23.9 3.60792354017158
23.925 3.60722347175674
23.95 3.6065234033419
23.975 3.60582333492706
24 3.60512326651222
24.025 3.60442319809738
24.05 3.60372312968254
24.075 3.6030230612677
24.1 3.60232299285286
24.125 3.60162292443803
24.15 3.60092285602319
24.175 3.60022278760835
24.2 3.59952271919351
24.225 3.59882265077867
24.25 3.59812258236383
24.275 3.59742251394899
24.3 3.59672244553415
24.325 3.59602237711931
24.35 3.59532230870447
24.375 3.59462224028964
24.4 3.5939221718748
24.425 3.59322210345996
24.45 3.59252203504512
24.475 3.59182196663028
24.5 3.59112189821544
24.525 3.5904218298006
24.55 3.58972176138576
24.575 3.58902169297092
24.6 3.58832162455609
24.625 3.58762155614125
24.65 3.58692148772641
24.675 3.58622141931157
24.7 3.58552135089673
24.725 3.58482128248189
24.75 3.58412121406705
24.775 3.58342114565221
24.8 3.58272107723737
24.825 3.58202100882253
24.85 3.5813209404077
24.875 3.58062087199286
24.9 3.57992080357802
24.925 3.57922073516318
24.95 3.57852066674834
24.975 3.5778205983335
25 3.57712052991866
25.025 3.57642046150382
25.05 3.57572039308898
25.075 3.57502032467414
25.1 3.57432025625931
25.125 3.57362018784447
25.15 3.57292011942963
25.175 3.57222005101479
25.2 3.57151998259995
25.225 3.57081991418511
25.25 3.57011984577027
25.275 3.56941977735543
25.3 3.56871970894059
25.325 3.56801964052576
25.35 3.56731957211092
25.375 3.56661950369608
25.4 3.56591943528124
25.425 3.5652193668664
25.45 3.56451929845156
25.475 3.56381923003672
25.5 3.56311916162188
25.525 3.56241909320704
25.55 3.5617190247922
25.575 3.56101895637737
25.6 3.56031888796253
25.625 3.55961881954769
25.65 3.55891875113285
25.675 3.55821868271801
25.7 3.55751861430317
25.725 3.55681854588833
25.75 3.55611847747349
25.775 3.55541840905865
25.8 3.55471834064382
25.825 3.55401827222898
25.85 3.55331820381414
25.875 3.5526181353993
25.9 3.55191806698446
25.925 3.55121799856962
25.95 3.55051793015478
25.975 3.54981786173994
26 3.5491177933251
26.025 3.54841772491026
26.05 3.54771765649543
26.075 3.54701758808059
26.1 3.54631751966575
26.125 3.54561745125091
26.15 3.54491738283607
26.175 3.54421731442123
26.2 3.54351724600639
26.225 3.54281717759155
26.25 3.54211710917671
26.275 3.54141704076187
26.3 3.54071697234704
26.325 3.5400169039322
26.35 3.53931683551736
26.375 3.53861676710252
26.4 3.53791669868768
26.425 3.53721663027284
26.45 3.536516561858
26.475 3.53581649344316
26.5 3.53511642502832
26.525 3.53441635661349
26.55 3.53371628819865
26.575 3.53301621978381
26.6 3.53231615136897
26.625 3.53161608295413
26.65 3.53091601453929
26.675 3.53021594612445
26.7 3.52951587770961
26.725 3.52881580929477
26.75 3.52811574087993
26.775 3.5274156724651
26.8 3.52671560405026
26.825 3.52601553563542
26.85 3.52531546722058
26.875 3.52461539880574
26.9 3.5239153303909
26.925 3.52321526197606
26.95 3.52251519356122
26.975 3.52181512514638
27 3.52111505673155
27.025 3.52041498831671
27.05 3.51971491990187
27.075 3.51901485148703
27.1 3.51831478307219
27.125 3.51761471465735
27.15 3.51691464624251
27.175 3.51621457782767
27.2 3.51551450941283
27.225 3.51481444099799
27.25 3.51411437258316
27.275 3.51341430416832
27.3 3.51271423575348
27.325 3.51201416733864
27.35 3.5113140989238
27.375 3.51061403050896
27.4 3.50991396209412
27.425 3.50921389367928
27.45 3.50851382526444
27.475 3.5078137568496
27.5 3.50711368843477
27.525 3.50641362001993
27.55 3.50571355160509
27.575 3.50501348319025
27.6 3.50431341477541
27.625 3.50361334636057
27.65 3.50291327794573
27.675 3.50221320953089
27.7 3.50151314111605
27.725 3.50081307270122
27.75 3.50011300428638
27.775 3.49941293587154
27.8 3.4987128674567
27.825 3.49801279904186
27.85 3.49731273062702
27.875 3.49661266221218
27.9 3.49591259379734
27.925 3.4952125253825
27.95 3.49451245696766
27.975 3.49381238855283
28 3.49311232013799
28.025 3.49241225172315
28.05 3.49171218330831
28.075 3.49101211489347
28.1 3.49031204647863
28.125 3.48961197806379
28.15 3.48891190964895
28.175 3.48821184123411
28.2 3.48751177281928
28.225 3.48681170440444
28.25 3.4861116359896
28.275 3.48541156757476
28.3 3.48471149915992
28.325 3.48401143074508
28.35 3.48331136233024
28.375 3.4826112939154
28.4 3.48191122550056
28.425 3.48121115708572
28.45 3.48051108867089
28.475 3.47981102025605
28.5 3.47911095184121
28.525 3.47841088342637
28.55 3.47771081501153
28.575 3.47701074659669
28.6 3.47631067818185
28.625 3.47561060976701
28.65 3.47491054135217
28.675 3.47421047293733
28.7 3.4735104045225
28.725 3.47281033610766
28.75 3.47211026769282
28.775 3.47141019927798
28.8 3.47071013086314
28.825 3.4700100624483
28.85 3.46930999403346
28.875 3.46860992561862
28.9 3.46790985720378
28.925 3.46720978878895
28.95 3.46650972037411
28.975 3.46580965195927
29 3.46510958354443
29.025 3.46440951512959
29.05 3.46370944671475
29.075 3.46300937829991
29.1 3.46230930988507
29.125 3.46160924147023
29.15 3.46090917305539
29.175 3.46020910464056
29.2 3.45950903622572
29.225 3.45880896781088
29.25 3.45810889939604
29.275 3.4574088309812
29.3 3.45670876256636
29.325 3.45600869415152
29.35 3.45530862573668
29.375 3.45460855732184
29.4 3.453908488907
29.425 3.45320842049217
29.45 3.45250835207733
29.475 3.45180828366249
29.5 3.45110821524765
29.525 3.45040814683281
29.55 3.44970807841797
29.575 3.44900801000313
29.6 3.44830794158829
29.625 3.44760787317345
29.65 3.44690780475862
29.675 3.44620773634378
29.7 3.44550766792894
29.725 3.4448075995141
29.75 3.44410753109926
29.775 3.44340746268442
29.8 3.44270739426958
29.825 3.44200732585474
29.85 3.4413072574399
29.875 3.44060718902506
29.9 3.43990712061023
29.925 3.43920705219539
29.95 3.43850698378055
29.975 3.43780691536571
30 3.43710684695087
30.025 3.43640677853603
30.05 3.43570671012119
30.075 3.43500664170635
30.1 3.43430657329151
30.125 3.43360650487668
30.15 3.43290643646184
30.175 3.432206368047
30.2 3.43150629963216
30.225 3.43080623121732
30.25 3.43010616280248
30.275 3.42940609438764
30.3 3.4287060259728
30.325 3.42800595755796
30.35 3.42730588914312
30.375 3.42660582072829
30.4 3.42590575231345
30.425 3.42520568389861
30.45 3.42450561548377
30.475 3.42380554706893
30.5 3.42310547865409
30.525 3.42240541023925
30.55 3.42170534182441
30.575 3.42100527340957
30.6 3.42030520499473
30.625 3.4196051365799
30.65 3.41890506816506
30.675 3.41820499975022
30.7 3.41750493133538
30.725 3.41680486292054
30.75 3.4161047945057
30.775 3.41540472609086
30.8 3.41470465767602
30.825 3.41400458926118
30.85 3.41330452084635
30.875 3.41260445243151
30.9 3.41190438401667
30.925 3.41120431560183
30.95 3.41050424718699
30.975 3.40980417877215
31 3.40910411035731
31.025 3.40840404194247
31.05 3.40770397352763
31.075 3.40700390511279
31.1 3.40630383669796
31.125 3.40560376828312
31.15 3.40490369986828
31.175 3.40420363145344
31.2 3.4035035630386
31.225 3.40280349462376
31.25 3.39846889378193
31.275 3.39513073016018
31.3 3.39130807823856
31.325 3.38700645866187
31.35 3.38223139207489
31.375 3.37698839912243
31.4 3.37128300044927
31.425 3.36512071670021
31.45 3.35850706852006
31.475 3.35144757655359
31.5 3.34394776144561
31.525 3.33601314384091
31.55 3.32764924438429
31.575 3.31886158372054
31.6 3.30965568249445
31.625 3.30003706135082
31.65 3.29001124093445
31.675 3.27958374189013
31.7 3.26876008486265
31.725 3.25754579049681
31.75 3.2459463794374
31.775 3.23396737232922
31.8 3.22161428981706
31.825 3.20889265254572
31.85 3.19580798116
31.875 3.18236579630468
31.9 3.16857161862456
31.925 3.15443096876444
31.95 3.13994936736911
31.975 3.12513233508336
32 3.109985392552
32.025 3.09451406041981
32.05 3.07872385933159
32.075 3.06262030993214
32.1 3.04620893286624
32.125 3.0294952487787
32.15 3.01248477831431
32.175 2.99518304211786
32.2 2.97759556083415
32.225 2.95972785510797
32.25 2.94158544558413
32.275 2.9231738529074
32.3 2.90449859772258
32.325 2.88556520067449
32.35 2.8663791824079
32.375 2.84694606356761
32.4 2.82727136479841
32.425 2.8073606067451
32.45 2.78721931005249
32.475 2.76685299536535
32.5 2.74626718332849
32.525 2.72546739458669
32.55 2.70445914978476
32.575 2.68324796956749
32.6 2.66183937457967
32.625 2.6402388854661
32.65 2.61845202287157
32.675 2.59648430744088
32.7 2.57434125981882
32.725 2.5520284006502
32.75 2.52955125057979
32.775 2.5069153302524
32.8 2.48412616031281
32.825 2.46118926140584
32.85 2.43811015417627
32.875 2.41489435926889
32.9 2.39154739732851
32.925 2.3680747889999
32.95 2.34448205492788
32.975 2.32077471575723
33 2.29695829213275
33.025 2.27303830469923
33.05 2.24902027410147
33.075 2.22490972098427
33.1 2.20071216599241
33.125 2.1764331297707
33.15 2.15207813296392
33.175 2.12765269621687
33.2 2.10316234017435
33.225 2.07861258548115
33.25 2.05400895278207
33.275 2.0293569627219
33.3 2.00466213594543
33.325 1.97992999309747
33.35 1.9551660548228
33.375 1.93037584176622
33.4 1.90556487457252
33.425 1.8807386738865
33.45 1.85590276035296
33.475 1.83106265461668
33.5 1.80622387732247
33.525 1.78139194911512
33.55 1.75657239063941
33.575 1.73177072254016
33.6 1.70699246546215
33.625 1.68224314005017
33.65 1.65752826694903
33.675 1.63285336680351
33.7 1.60822396025841
33.725 1.58364556795853
33.75 1.5591236754774
33.775 1.53466296174964
33.8 1.51026729907093
33.825 1.48594052466572
33.85 1.46168647575843
33.875 1.43750898957349
33.9 1.41341190333535
33.925 1.38939905426842
33.95 1.36547427959715
33.975 1.34164141654598
34 1.31790430233932
34.025 1.29426677420162
34.05 1.2707326693573
34.075 1.24730582503081
34.1 1.22399007844657
34.125 1.20078926682902
34.15 1.17770722740259
34.175 1.15474779739171
34.2 1.13191481402081
34.225 1.10921211451434
34.25 1.08664353609673
34.275 1.06421291599239
34.3 1.04192409142577
34.325 1.01978089962131
34.35 0.997787177803433
34.375 0.975946763196573
34.4 0.954263493025165
34.425 0.932741204513635
34.45 0.911383734886428
34.475 0.890194921367971
34.5 0.869178601182698
34.525 0.848338611555041
34.55 0.827678789709427
34.575 0.807202972870303
34.6 0.786914998262093
34.625 0.766818703109233
34.65 0.746917924636155
34.675 0.727216500067285
34.7 0.70771826662707
34.725 0.688427061539935
34.75 0.669346722030314
34.775 0.650481085322641
34.8 0.631833988641342
34.825 0.613409269210862
34.85 0.595210764255629
34.875 0.577242311000075
34.9 0.559507746668633
34.925 0.542010908485732
34.95 0.524755633675816
34.975 0.507745759463311
35 0.49098512307265
35.025 0.474477561728267
35.05 0.458226912654592
35.075 0.442237013076064
35.1 0.426511700217115
35.125 0.411054811302174
35.15 0.395870183555678
35.175 0.380961654202054
35.2 0.366333060465744
35.225 0.351988239571176
35.25 0.337931028742783
35.275 0.324165265204999
35.3 0.310694786182253
35.325 0.297523428898986
35.35 0.284655030579626
35.375 0.272093428448609
35.4 0.259842459730364
35.425 0.247905961649325
35.45 0.236287771429928
35.475 0.224991726296605
35.5 0.214021663473789
35.525 0.203381420185911
35.55 0.193074833657403
35.575 0.183105741112704
35.6 0.173477979776244
35.625 0.164195386872455
35.65 0.155261799625771
35.675 0.146681055260623
35.7 0.138456991001448
35.725 0.130593444072677
35.75 0.123094251698743
35.775 0.11596325110408
35.8 0.109204279513118
35.825 0.102821174150295
35.85 0.0968177722400401
35.875 0.0911979110067889
35.9 0.0859654276749726
35.925 0.0811241594690237
35.95 0.0766779436133787
35.975 0.0726306173324662
36 0.0689860178507242
36.025 0.0657479823925828
36.05 0.0629203481824741
36.075 0.0605069524448343
36.1 0.0585116324040937
36.125 0.0569382252846871
36.15 0.0557905683110462
36.175 0.0550724987076046
36.2 0.0547878536987958
36.225 0.0549404705090528
36.25 0.0544659123367125
36.275 0.0543200035081735
36.3 0.0541740946796345
36.325 0.0540281858510955
36.35 0.0538822770225564
36.375 0.0537363681940174
36.4 0.0535904593654784
36.425 0.0534445505369394
36.45 0.0532986417084004
36.475 0.0531527328798614
36.5 0.0530068240513224
36.525 0.0528609152227834
36.55 0.0527150063942443
36.575 0.0525690975657053
36.6 0.0524231887371663
36.625 0.0522772799086273
36.65 0.0521313710800883
36.675 0.0519854622515492
36.7 0.0518395534230102
36.725 0.0516936445944712
36.75 0.0515477357659322
36.775 0.0514018269373932
36.8 0.0512559181088542
36.825 0.0511100092803152
36.85 0.0509641004517762
36.875 0.0508181916232371
36.9 0.0506722827946981
36.925 0.0505263739661591
36.95 0.0503804651376201
36.975 0.0502345563090811
37 0.0500886474805421
37.025 0.0499427386520031
37.05 0.049796829823464
37.075 0.049650920994925
37.1 0.049505012166386
37.125 0.049359103337847
37.15 0.049213194509308
37.175 0.049067285680769
37.2 0.0489213768522299
37.225 0.0487754680236909
37.25 0.0486295591951519
37.275 0.0484836503666129
37.3 0.0483377415380739
37.325 0.0481918327095349
37.35 0.0480459238809959
37.375 0.0479000150524569
37.4 0.0477541062239179
37.425 0.0476081973953788
37.45 0.0474622885668398
37.475 0.0473163797383008
37.5 0.0471704709097618
37.525 0.0470245620812228
37.55 0.0468786532526837
37.575 0.0467327444241447
37.6 0.0465868355956057
37.625 0.0464409267670667
37.65 0.0462950179385277
37.675 0.0461491091099887
37.7 0.0460032002814497
37.725 0.0458572914529106
37.75 0.0457113826243716
37.775 0.0455654737958326
37.8 0.0454195649672936
37.825 0.0452736561387546
37.85 0.0451277473102156
37.875 0.0449818384816766
37.9 0.0448359296531376
37.925 0.0446900208245985
37.95 0.0445441119960595
37.975 0.0443982031675205
38 0.0442522943389815
38.025 0.0441063855104425
38.05 0.0439604766819034
38.075 0.0438145678533644
38.1 0.0436686590248254
38.125 0.0435227501962864
38.15 0.0433768413677474
38.175 0.0432309325392084
38.2 0.0430850237106694
38.225 0.0429391148821304
38.25 0.0427932060535914
38.275 0.0426472972250523
38.3 0.0425013883965133
38.325 0.0423554795679743
38.35 0.0422095707394353
38.375 0.0420636619108963
38.4 0.0419177530823572
38.425 0.0417718442538182
38.45 0.0416259354252792
38.475 0.0414800265967402
38.5 0.0413341177682012
38.525 0.0411882089396622
38.55 0.0410423001111232
38.575 0.0408963912825841
38.6 0.0407504824540451
38.625 0.0406045736255061
38.65 0.0404586647969671
38.675 0.0403127559684281
38.7 0.0401668471398891
38.725 0.0400209383113501
38.75 0.0398750294828111
38.775 0.039729120654272
38.8 0.039583211825733
38.825 0.039437302997194
38.85 0.039291394168655
38.875 0.039145485340116
38.9 0.0389995765115769
38.925 0.0388536676830379
38.95 0.0387077588544989
38.975 0.0385618500259599
39 0.0384159411974209
39.025 0.0382700323688819
39.05 0.0381241235403429
39.075 0.0379782147118039
39.1 0.0378323058832649
39.125 0.0376863970547258
39.15 0.0375404882261868
39.175 0.0373945793976478
39.2 0.0372486705691088
39.225 0.0371027617405698
39.25 0.0369568529120308
39.275 0.0368109440834917
39.3 0.0366650352549527
39.325 0.0365191264264137
39.35 0.0363732175978747
39.375 0.0362273087693357
39.4 0.0360813999407967
39.425 0.0359354911122576
39.45 0.0357895822837186
39.475 0.0356436734551796
39.5 0.0354977646266406
39.525 0.0353518557981016
39.55 0.0352059469695626
39.575 0.0350600381410236
39.6 0.0349141293124846
39.625 0.0347682204839456
39.65 0.0346223116554065
39.675 0.0344764028268675
39.7 0.0343304939983285
39.725 0.0341845851697895
39.75 0.0340386763412505
39.775 0.0338927675127114
39.8 0.0337468586841724
39.825 0.0336009498556334
39.85 0.0334550410270944
39.875 0.0333091321985554
39.9 0.0331632233700164
39.925 0.0330173145414774
39.95 0.0328714057129384
39.975 0.0327254968843994
40 0.0325795880558603
40.025 0.0324336792273213
40.05 0.0322877703987823
40.075 0.0321418615702433
40.1 0.0319959527417043
40.125 0.0318500439131653
40.15 0.0317041350846262
40.175 0.0315582262560872
40.2 0.0314123174275482
40.225 0.0312664085990092
40.25 0.0311204997704702
40.275 0.0309745909419312
40.3 0.0308286821133921
40.325 0.0306827732848531
40.35 0.0305368644563141
40.375 0.0303909556277751
40.4 0.0302450467992361
40.425 0.0300991379706971
40.45 0.0299532291421581
40.475 0.0298073203136191
40.5 0.0296614114850801
40.525 0.029515502656541
40.55 0.029369593828002
40.575 0.029223684999463
40.6 0.029077776170924
40.625 0.028931867342385
40.65 0.0287859585138459
40.675 0.0286400496853069
40.7 0.0284941408567679
40.725 0.0283482320282289
40.75 0.0282023231996899
40.775 0.0280564143711509
40.8 0.0279105055426119
40.825 0.0277645967140728
40.85 0.0276186878855338
40.875 0.0274727790569948
40.9 0.0273268702284558
40.925 0.0271809613999168
40.95 0.0270350525713778
40.975 0.0268891437428388
41 0.0267432349142998
41.025 0.0265973260857607
41.05 0.0264514172572217
41.075 0.0263055084286827
41.1 0.0261595996001437
41.125 0.0260136907716047
41.15 0.0258677819430656
41.175 0.0257218731145266
41.2 0.0255759642859876
41.225 0.0254300554574486
41.25 0.0252841466289096
41.275 0.0251382378003706
41.3 0.0249923289718316
41.325 0.0248464201432926
41.35 0.0247005113147536
41.375 0.0245546024862145
41.4 0.0244086936576755
41.425 0.0242627848291365
41.45 0.0241168760005975
41.475 0.0239709671720585
41.5 0.0238250583435195
41.525 0.0236791495149804
41.55 0.0235332406864414
41.575 0.0233873318579024
41.6 0.0232414230293634
41.625 0.0230955142008244
41.65 0.0229496053722854
41.675 0.0228036965437464
41.7 0.0226577877152073
41.725 0.0225118788866683
41.75 0.0223659700581293
41.775 0.0222200612295903
41.8 0.0220741524010513
41.825 0.0219282435725123
41.85 0.0217823347439733
41.875 0.0216364259154343
41.9 0.0214905170868952
41.925 0.0213446082583562
41.95 0.0211986994298172
41.975 0.0210527906012782
42 0.0209068817727392
42.025 0.0207609729442001
42.05 0.0206150641156611
42.075 0.0204691552871221
42.1 0.0203232464585831
42.125 0.0201773376300441
42.15 0.0200314288015051
42.175 0.0198855199729661
42.2 0.0197396111444271
42.225 0.019593702315888
42.25 0.019447793487349
42.275 0.01930188465881
42.3 0.019155975830271
42.325 0.019010067001732
};
\addplot [ultra thick, black, forget plot]
table {%
17 -0.2
17 3.8
};
\addplot [ultra thick, black, forget plot]
table {%
23 -0.2
23 3.8
};
\addplot [ultra thick, black, forget plot]
table {%
31.25 -0.2
31.25 3.8
};
\addplot [ultra thick, black, forget plot]
table {%
36.25 -0.2
36.25 3.8
};
\node at (axis cs:8.5,0.3)[
  anchor=base,
  text=black,
  rotate=0.0
]{ F};
\node at (axis cs:20,0.3)[
  anchor=base,
  text=black,
  rotate=0.0
]{ G};
\node at (axis cs:27.125,0.3)[
  anchor=base,
  text=black,
  rotate=0.0
]{ H};
\node at (axis cs:33.75,0.3)[
  anchor=base,
  text=black,
  rotate=0.0
]{ I};
\node at (axis cs:38,0.3)[
  anchor=base,
  text=black,
  rotate=0.0
]{ J};
\end{axis}

\end{tikzpicture}}
	\caption{States of the ego vehicle. The blue line shows the measured data and the orange dashed line shows the output of the model fitted to the data.}
\end{figure}

\color{black}
