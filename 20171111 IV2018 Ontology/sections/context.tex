\subsection{Context of scenario}
\label{sec:context}

% Scenario definitions are very diverse --> context is important
Because the notion of scenario is used in many different contexts, there is a high diversity of the definition of a scenario (for an overview of the diversity in the scenarios, see \cite{vannotten2003updated, bishop2007scentechniques}). Therefore, it is assumed that there is no `correct' scenario definition \cite{vannotten2003updated}. As a result, to define the notion of scenario, it is important to consider the context in which it will be used.

% Explain context
% - Scenarios can be defined by human expert (knowledge driven) or by using data (Stellet et al.)
% - Could be used for assessment (Adaptive paper, pegasus, stellet et al., )
In this paper, the context of a scenario is the assessment of automated vehicles. It is assumed that all assessment methodologies use so-called test scenarios for which some resulting metrics are compared with a reference \cite{stellet2015taxonomy}. Whether these scenarios are obtained with a knowledge-based approach \cite{gietelink2004systemvalidation, stellet2015taxonomy} or with a data-driven approach \cite{deGelder2017assessment, stellet2015taxonomy}, a clear and unambiguous definition of such a test scenario is required. In the case of a data-driven approach, the test scenarios are generated through analysis of observed scenarios in (real-life driving) data. Because the observed scenarios can roughly be described in a similar manner as the test scenarios, we will refer to these observed and test scenarios as scenarios.
% - Scenarios that an (automated) vehicle can encounter
The ultimate goal is to build a database with all relevant scenarios that an automated vehicle has to cope with \cite{putz2017pegasus}. Hence, a scenario should be a description of a potential use-case of an automated vehicle. 
