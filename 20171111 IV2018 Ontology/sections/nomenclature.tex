\section{Nomenclature}
\label{sec:nomenclature}

For the definition of \emph{scenario} and \emph{event}, several notions are adopted from literature. In this section, the concepts of \emph{ego vehicle}, \emph{actor}, \emph{state}, \emph{model}, \emph{mode}, \emph{activity}, \emph{static environment}, and \emph{dynamic environment} are detailed. 

\subsubsection{Ego vehicle}
\label{sec:ego vehicle}
The ego vehicle refers to the perspective from which the world is seen. Usually, the ego vehicle refers to the vehicle that is perceiving the world through its sensors (see, e.g.,~\cite{Bonnin2014}) or the vehicle that has to perform a specific task (see, e.g.,~\cite{althoff2017CommonRoad}). In the latter case, the ego vehicle is often referred to as the system under test \cite{stellet2015taxonomy} or the vehicle under test \cite{gietelink2006development}.
%The ontology presented by Geyer~et~al.\ ``is described from the ego-vehicle's point of view'' \cite{geyer2014}. 
%Note that in case a sensor-equipped vehicle is used to extract scenarios from real-world driving, the ego vehicle in an extracted scenario does not necessarily have to correspond to the sensor-equipped vehicle that is used to acquire the real-world data.

\subsubsection{Actor}
\label{sec:actor}
An actor is an element of a scenario acting on its own behalf \cite{ulbrich2015}. In practice, this can be a driver of a car, a bicyclist, a pedestrian, an automation system, or a combination of a driver and an automation system \cite{geyer2014}.
% Traffic light?

\subsubsection{State}
\label{sec:state}
According to the IEEE Standard Glossary of Software Engineering Terminology \cite{ieee1990glossary}, states are ``[...] variables that define the characteristics of a system, component, or simulation''. For example, a state could be the acceleration of an actor.

\subsubsection{Model}
\label{sec:model}
Typically, a system is modeled using a differential equation of the form $\dot{x}=f_{\theta}(x(t), u(t), t)$ \cite{norman2011control}, where $x(t)$ represents the state vector at time $t$, $u(t)$ represents an external input vector, and the function $f(\cdot)$ is parametrized by $\theta$.
% The input $u$ is a function of time, that needs to be quantified. For this purpose, a parametrized function can be used, i.e. $u=g_{\theta}(t)$ with parameter vector $\theta$, such that the differential equation can be rewritten to $\dot{x}=h_{\theta}(x,t)$. In the context of this paper, model refers to a parametrized function, such as $h_{\theta}(x,t)$. It might be more practical to directly model the state (i.e., the result of the differential equation) using a function $x=k_{\theta}(t)$, such that no explicit information is required about the system dynamics. For example, see \cite{deGelder2017assessment}.

\subsubsection{Mode}
\label{sec:mode}
In some systems, the behavior or evaluation of the system may all of a sudden change abruptly, e.g., due to a sudden change in an input, a model parameter, or the model function. Such a transient is called a mode switch.
In each mode, the behavior of the system is described by a particular model with a fixed function $f$, parameter $\theta$ and smooth input $u(t)$ \cite{deschutter2000optimal}.

\subsubsection{Activity}
\label{sec:activity}
An activity refers to the behavior of a particular mode. For example, an activity could be described by the label `braking' or `changing lane'.
%A scenario contains the quantitative description of the ongoing activity of the ego vehicle and its dynamic environment. Here, the description refers to the changing states that are relevant for the scenario, e.g., acceleration and velocity. The activity is described using the models that describe the way the state evolves over time.

\subsubsection{Static environment}
\label{sec:static environment}
The static environment refers to the part of a scenario that does not change during a scenario. This includes geo-spatially stationary elements \cite{ulbrich2015}. Although one might argue whether light and weather conditions are dynamic or not \cite{geyer2014,bach2016modelbased}, in most cases it is reasonable to assume that these conditions are not subject to significant changes during the time frame of a scenario. Hence, light and weather conditions are considered to be part of the static environment.

\subsubsection{Dynamic environment}
\label{sec:dynamic environment}
As opposed to the static environment, the dynamic environment refers to the part of a scenario that changes during the time frame of a scenario. The dynamic environment is described using the activities that describe the way the states evolve over time. In practice, the dynamic environment mainly consists of the moving actors (other than the ego vehicle) that are relevant to the ego vehicle. 
\cbstart
Furthermore, road side units that communicate with vehicles within the communication range \cite{alsultan2014comprehensive}, are also part of the dynamic environment.
\cbend

Note that it might not always be obvious whether a part of a scenario belongs to the static or dynamic environment. For example, the post of a traffic light can be considered as part of the static environment, while the signal of the traffic light can be considered as part of the dynamic environment. Most important, however, is that all parts of the environment that are relevant to the assessment are described in either the static or the dynamic environment.
