\subsection{Nomenclature}
\label{sec:nomenclature}

In the previous subsections, the definitions of a scenario and and event are presented. For this purpose, several notions are adopted. In this subsection, the notions of \emph{ego vehicle}, \emph{activity}, \emph{static environment}, \emph{dynamic environment}, \emph{model} and \emph{actor} are explained. 

\subsubsection{Ego vehicle}
Literally, `ego' means `I'. Thus, the ego vehicle refers to the perspective from which the world is seen. Usually, the ego vehicle refers to the vehicle which is perceiving the world through its sensors (see e.g.\ \cite{Bonnin2014}) or the vehicle which has to perform a specific task (see e.g.\ \cite{althoff2017CommonRoad}). The ontology presented by Geyer et al.\ ``is described from the ego-vehicle’s point of view'' \cite{geyer2014}. In this paper, the ego vehicle determines which events are `relevant'. Thus, the ego vehicle is responsible for determining the start and end time of the scenario. Furthermore, the tags assigned to the scenario may depend on the ego vehicle. For example, the `target in front' refers to the vehicle in front of the ego vehicle. Note that in case a sensor-equipped vehicle is used to extract scenarios from real-life data, the ego vehicle in a resulting scenario does not necessarily correspond to the sensor-equipped vehicle which is used to acquire the real-life data.


\subsubsection{Static environment}

\subsubsection{Dynamic environment}

\subsubsection{Activity}

\subsubsection{Model}

\subsubsection{Actor}
