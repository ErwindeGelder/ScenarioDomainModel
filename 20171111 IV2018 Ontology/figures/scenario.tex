\documentclass{standalone}
\usepackage{tikz}
\usepackage{pgfplots}
\usepackage{xcolor}
\pgfplotsset{compat=1.14}

\usetikzlibrary{backgrounds}
\pgfdeclarelayer{foreground}
\pgfsetlayers{background,main,foreground}
\tikzset{%
	on foreground layer/.style={%
		execute at begin scope={%
			\pgfonlayer{foreground}%
			\let\tikz@options=\pgfutil@empty%
			\tikzset{every on foreground layer/.try,#1}%
			\tikz@options%
		},
		execute at end scope={\endpgfonlayer}
	},
	% addasu o ateb Loop Space: https://tex.stackexchange.com/a/20426/
	%\url{https://tex.stackexchange.com/q/46957/86}
	on layer/.code={%
		\pgfonlayer{#1}\begingroup
		\aftergroup\endpgfonlayer
		\aftergroup\endgroup
	},
	node on layer/.code={%
		\gdef\node@@on@layer{%
			\setbox\tikz@tempbox=\hbox\bgroup\pgfonlayer{#1}\unhbox\tikz@tempbox\endpgfonlayer\egroup}
		\aftergroup\node@on@layer
	},
}
\def\node@on@layer{\aftergroup\node@@on@layer}


\newlength\figurewidth
\newlength\figureheight
\begin{document}
%	\setlength\figureheight{150pt}
%	\setlength\figurewidth{248pt}
%	% This file was created by matplotlib2tikz v0.6.14.
\begin{tikzpicture}

\definecolor{color3}{rgb}{0.549019607843137,0.337254901960784,0.294117647058824}
\definecolor{color0}{rgb}{0.172549019607843,0.627450980392157,0.172549019607843}
\definecolor{color2}{rgb}{0.580392156862745,0.403921568627451,0.741176470588235}
\definecolor{color1}{rgb}{0.83921568627451,0.152941176470588,0.156862745098039}

\begin{axis}[
xlabel={$t$ [s]},
ylabel={Relative lateral distance [m]},
xmin=0, xmax=34.1600000858307,
ymin=-3.6944648464502, ymax=3.79876173773052,
width=\figurewidth,
height=\figureheight,
tick align=outside,
tick pos=left,
xmajorgrids,
x grid style={lightgray!92.026143790849673!black},
ymajorgrids,
y grid style={lightgray!92.026143790849673!black},
clip marker paths
]
\addplot [semithick, color0, mark=*, mark size=1, mark options={solid}, only marks, forget plot]
table {%
0 -0.0417787396782297
0.0400002002716064 -0.0434833324633892
0.0800001621246338 -0.00793896066897679
0.120000123977661 -0.0290788596134219
0.160000085830688 -0.00511650671826534
0.200000047683716 0.026785123248344
0.240000009536743 0.0205590435513228
0.28000020980835 -0.0226561519444399
0.320000171661377 -0.00188009407999191
0.360000133514404 -0.00377719282237902
0.400000095367432 -0.0155240147251251
0.440000057220459 -0.0306548872097677
0.480000019073486 -0.0606875543387061
0.519999980926514 -0.0349109763945282
0.56000018119812 -0.0274273183054372
0.600000143051147 -0.0255048803860961
0.640000104904175 -0.0142116114614845
0.680000066757202 -0.0144869500468184
0.720000028610229 -0.00634319719655396
0.759999990463257 0.00264264047317925
0.800000190734863 0.0105393193251953
0.840000152587891 0.0138097292343392
0.880000114440918 0.0291535571020424
0.920000076293945 0.0366716656258401
0.960000038146973 0.0297297342137315
1 0.0217351281834865
1.04000020027161 0.0103010571788247
1.08000016212463 -0.012837636547751
1.12000012397766 -0.000666284051467028
1.16000008583069 -0.0196001067388685
1.20000004768372 -0.00508295675099915
1.24000000953674 -0.0302094671738396
1.28000020980835 -0.0652997387717573
1.32000017166138 -0.0518677885327045
1.3600001335144 -0.062191293287375
1.40000009536743 -0.0583230235330692
1.44000005722046 -0.0637485272072151
1.48000001907349 -0.0848910782785489
1.51999998092651 -0.0887539614874596
1.56000018119812 -0.0915527800488931
1.60000014305115 -0.094033839429671
1.64000010490417 -0.0986166093511816
1.6800000667572 -0.101533324659034
1.72000002861023 -0.097418266876103
1.75999999046326 -0.0951203112193729
1.80000019073486 -0.0768142142537029
1.84000015258789 -0.0758488769996015
1.88000011444092 -0.0657215850608496
1.92000007629395 -0.0596173475464898
1.96000003814697 -0.0774694105485675
2 -0.0819576342397531
2.04000020027161 -0.0737450374304574
2.08000016212463 -0.0632697980881775
2.12000012397766 -0.065146546111649
2.16000008583069 -0.0379296996072994
2.20000004768372 -0.0298124907636027
2.24000000953674 -0.0104287031121793
2.28000020980835 -0.000783934746713261
2.32000017166138 -0.0146727622936489
2.3600001335144 -0.0116230187023605
2.40000009536743 -0.00800980925905722
2.44000005722046 -0.0569500180955829
2.48000001907349 -0.0194943770969402
2.51999998092651 0.0151885988311344
2.56000018119812 0.0656892286827005
2.60000014305115 0.072081029845608
2.64000010490417 0.0845924477338064
2.6800000667572 0.105742618639308
2.72000002861023 0.119406130685862
2.75999999046326 0.12824628236362
2.80000019073486 0.139938514157885
2.84000015258789 0.148019094859861
2.88000011444092 0.25583690341537
2.92000007629395 0.306554447711603
2.96000003814697 0.278564315163056
3 0.244764365743769
3.04000020027161 0.234652405155853
3.08000016212463 0.237902782307596
3.12000012397766 0.235483204363214
3.16000008583069 0.239052413177167
3.20000004768372 0.232617816658026
3.24000000953674 0.260339993436241
3.28000020980835 0.258706864517433
3.32000017166138 0.3574681748221
3.3600001335144 0.901479815125867
3.40000009536743 0.918641901317467
3.44000005722046 0.952841176995706
3.48000001907349 0.986183120187692
3.51999998092651 0.988627684322505
3.56000018119812 1.00186064011514
3.60000014305115 1.01180667626192
3.64000010490417 1.01591165870319
3.6800000667572 1.04432666445691
3.72000002861023 1.06150147302882
3.75999999046326 1.09942249772943
3.80000019073486 1.11890528969187
3.84000015258789 0.636547295423912
3.88000011444092 0.405911449278643
3.92000007629395 0.389694720289353
3.96000003814697 0.41780990901604
4 0.414559403397376
4.04000020027161 0.420269918803582
4.08000016212463 0.449963792167126
4.12000012397766 0.418271949134033
4.16000008583069 0.429857636237069
4.20000004768372 0.407949681378507
4.24000000953674 0.415058137103386
4.28000020980835 0.431460320618351
4.32000017166138 0.413362124001608
4.3600001335144 0.415399411612331
4.40000009536743 0.395302178564078
4.44000005722046 0.415159251091263
4.48000001907349 0.40905699661753
4.51999998092651 0.392595507550862
4.56000018119812 0.253553725662647
4.60000014305115 0.250807804710817
4.64000010490417 0.263448388963936
4.6800000667572 0.283164182167046
4.72000002861023 0.260610309470746
4.75999999046326 0.264162076704724
4.80000019073486 0.249604925137301
4.84000015258789 0.225354232271675
4.88000011444092 0.22634417982975
4.92000007629395 0.198150775817807
4.96000003814697 0.197521672397681
5 0.182313265544689
5.04000020027161 0.194446681398852
5.08000016212463 0.221427069799978
5.12000012397766 0.189127232670235
5.16000008583069 0.209346408108924
5.20000004768372 0.163559505577604
5.24000000953674 0.139405583163256
5.28000020980835 0.16605103007943
5.32000017166138 0.119682918201199
5.3600001335144 0.131886410955644
5.40000009536743 0.112349399868415
5.44000005722046 0.11047751927303
5.48000001907349 0.144443713438279
5.51999998092651 0.117395918805469
5.56000018119812 0.1614418734515
5.60000014305115 0.12118195280291
5.64000010490417 0.168949677401619
5.6800000667572 0.213932440600319
5.72000002861023 0.18063949003915
5.75999999046326 0.220967184431904
5.80000019073486 0.191980929452678
5.84000015258789 0.205817819563089
5.88000011444092 0.234895748189027
5.92000007629395 0.1917900253487
5.96000003814697 0.201873463051521
6 0.191549293366018
6.04000020027161 0.165585265412575
6.08000016212463 0.214685869194678
6.12000012397766 0.329367520623184
6.16000008583069 0.337115141512145
6.20000004768372 0.325935692683958
6.24000000953674 0.372339743198021
6.28000020980835 0.388521744711257
6.32000017166138 0.393964485258088
6.3600001335144 0.456375310335519
6.40000009536743 0.469509591666124
6.44000005722046 0.45917285117698
6.48000001907349 0.527237683952156
6.51999998092651 0.502872396074831
6.56000018119812 0.524079771777374
6.60000014305115 0.523329068632831
6.64000010490417 0.527918935909131
6.6800000667572 0.545962045217118
6.72000002861023 0.502727470836427
6.75999999046326 0.482184477662969
6.80000019073486 0.477095554564596
6.84000015258789 0.444204777600845
6.88000011444092 0.44850776699175
6.92000007629395 0.416270276152863
6.96000003814697 0.420725740179746
7 0.414874046023105
7.04000020027161 0.38603477883713
7.08000016212463 0.399460643621401
7.12000012397766 0.392432457259056
7.16000008583069 0.383393069805033
7.20000004768372 0.37759507821646
7.24000000953674 0.360152991313975
7.28000020980835 0.369336407770393
7.32000017166138 0.318961385583633
7.3600001335144 0.338294565931529
7.40000009536743 0.351662758020797
7.44000005722046 0.334317194704018
7.48000001907349 0.351272703445039
7.51999998092651 0.356151038601924
7.56000018119812 0.370647339093179
7.60000014305115 0.361366313709705
7.64000010490417 0.358361932543915
7.6800000667572 0.377377270780715
7.72000002861023 0.367017494780768
7.75999999046326 0.385762241170437
7.80000019073486 0.377993316734781
7.84000015258789 0.385367301479262
7.88000011444092 0.42649928924518
7.92000007629395 0.376952841485367
7.96000003814697 0.32511495293492
8 0.313859984558447
8.04000020027161 0.303630483685739
8.08000016212463 0.354531774151929
8.12000012397766 0.298820160892349
8.16000008583069 0.284601621841473
8.20000004768372 0.27365181123016
8.24000000953674 0.255467651706338
8.28000020980835 0.298369025216015
8.32000017166138 0.225502205215161
8.3600001335144 0.23583690821237
8.40000009536743 0.224470184450044
8.44000005722046 0.220576961146684
8.48000001907349 0.284973198074928
8.51999998092651 0.210160351948411
8.56000018119812 0.232056477010676
8.60000014305115 0.22463754242515
8.64000010490417 0.246830359009954
8.6800000667572 0.320988924900878
8.72000002861023 0.235447281501163
8.75999999046326 0.242505953399173
8.80000019073486 0.308828695113704
8.84000015258789 0.226868064053963
8.88000011444092 0.298516583395571
8.92000007629395 0.112239341909362
8.96000003814697 0.233319506270085
9 0.222930825871416
9.04000020027161 0.173799730495064
9.08000016212463 0.133034145241127
9.12000012397766 0.271600448852642
9.16000008583069 0.240262057773767
9.20000004768372 0.272490445382458
9.24000000953674 0.290509172202269
9.28000020980835 0.203546656494682
9.32000017166138 0.240886270162535
9.3600001335144 0.292646349281046
9.40000009536743 0.24357226985176
9.44000005722046 0.0457886916948231
9.48000001907349 0.113493234988786
9.51999998092651 -0.000772053611664832
9.56000018119812 0.112654417957984
9.60000014305115 0.19868405246894
9.64000010490417 0.289832259579926
9.6800000667572 0.336780099632129
9.72000002861023 0.546307860143487
9.75999999046326 0.605127321445179
9.80000019073486 0.709152583080343
9.84000015258789 0.86787252442825
9.88000011444092 0.855295164046075
9.92000007629395 1.01077547057974
9.96000003814697 1.03499509359713
10 1.09273377718478
10.0400002002716 1.1245075426224
10.0800001621246 1.04513123852704
10.1200001239777 0.590629446494773
10.1600000858307 0.490265626919553
10.2000000476837 0.525338806170876
10.2400000095367 0.582509165031396
10.2800002098083 0.550542282891613
10.3200001716614 0.569727304410648
10.3600001335144 0.58622729699181
10.4000000953674 0.600210523025847
10.4400000572205 0.745514249594854
10.4800000190735 0.744749887615746
10.5199999809265 0.891990517277949
10.5600001811981 0.899691473428397
10.6000001430511 1.20455610658641
10.6400001049042 1.70099379394201
10.6800000667572 1.76319378604241
10.7200000286102 1.8848609444214
10.7599999904633 1.942461616096
10.8000001907349 2.00791692021982
10.8400001525879 2.08900244115099
10.8800001144409 2.15631064645281
10.9200000762939 2.24277276393423
10.960000038147 2.26034880712028
11 2.2847860362985
11.0400002002716 2.38502518267754
11.0800001621246 2.41149425484073
11.1200001239777 2.46287144341076
11.1600000858307 2.45554859483347
11.2000000476837 2.50325144504693
11.2400000095367 2.4748176006217
11.2800002098083 2.53893538470829
11.3200001716614 2.6149336120236
11.3600001335144 2.58903833083674
11.4000000953674 2.6709464333538
11.4400000572205 2.71302412942567
11.4800000190735 2.71737212973987
11.5199999809265 2.73644748425465
11.5600001811981 2.78274488624952
11.6000001430511 2.81908712925223
11.6400001049042 2.84631088708817
11.6800000667572 2.87858715226997
11.7200000286102 2.8526665423972
11.7599999904633 2.82163607458537
11.8000001907349 2.81946235112765
11.8400001525879 2.77045666703232
11.8800001144409 2.81113021465049
11.9200000762939 2.83063097523508
11.960000038147 2.8547743756347
12 2.85652138845991
12.0400002002716 2.83735527463517
12.0800001621246 2.84826293840178
12.1200001239777 2.94650899192099
12.1600000858307 2.96834716525122
12.2000000476837 2.9459678874127
12.2400000095367 2.95552809754005
12.2800002098083 2.98052639741588
12.3200001716614 3.03308143254575
12.3600001335144 3.03516685628485
12.4000000953674 3.03830755409455
12.4400000572205 3.05198260959917
12.4800000190735 3.0678785048125
12.5199999809265 3.07061290230589
12.5600001811981 3.08839716534139
12.6000001430511 3.0832046156668
12.6400001049042 3.09135883909449
12.6800000667572 3.10858451990355
12.7200000286102 3.03941106173655
12.7599999904633 3.0286950647477
12.8000001907349 3.0340884180457
12.8400001525879 2.99966137050916
12.8800001144409 3.00007675451334
12.9200000762939 2.92460981991743
12.960000038147 2.91652296508919
13 2.84387097629045
13.0400002002716 2.82922891089638
13.0800001621246 2.784352756705
13.1200001239777 2.7617945740745
13.1600000858307 2.75657701446277
13.2000000476837 2.63835897994763
13.2400000095367 2.61485943274389
13.2800002098083 2.59233925987106
13.3200001716614 2.54632233644069
13.3600001335144 2.53502060367273
13.4000000953674 2.53048062053379
13.4400000572205 2.4933923496161
13.4800000190735 2.49556359346183
13.5199999809265 2.49158551094374
13.5600001811981 2.52205181480324
13.6000001430511 2.55991862959851
13.6400001049042 2.55413321029144
13.6800000667572 2.54124768509059
13.7200000286102 2.60696714951397
13.7599999904633 2.60402189107454
13.8000001907349 2.66285799233406
13.8400001525879 2.71306910117497
13.8800001144409 2.71683005828622
13.9200000762939 2.72031786075228
13.960000038147 2.73824721379009
14 2.74174426664553
14.0400002002716 2.78187433127749
14.0800001621246 2.86799681689796
14.1200001239777 2.87925312794838
14.1600000858307 2.89610280642867
14.2000000476837 2.91435251281274
14.2400000095367 2.92266929287403
14.2800002098083 2.93546697106999
14.3200001716614 3.00334255019609
14.3600001335144 3.01768822302289
14.4000000953674 3.03556724485131
14.4400000572205 3.01297996432891
14.4800000190735 3.01002518731099
14.5199999809265 3.0251960717066
14.5600001811981 3.00817321327189
14.6000001430511 3.01049629972815
14.6400001049042 2.99215120632159
14.6800000667572 2.99225186649551
14.7200000286102 2.95555593367695
14.7599999904633 2.9528534374746
14.8000001907349 2.95507923768087
14.8400001525879 2.99711772569004
14.8800001144409 2.99898927695971
14.9200000762939 3.00566460827698
14.960000038147 3.01015581961001
15 3.01644627112053
15.2000000476837 3.3620217705862
15.2400000095367 3.36236510006473
15.2800002098083 3.35264191549625
15.3200001716614 3.45816052935867
15.3600001335144 3.28908788760309
15.4000000953674 3.38583134435336
15.4400000572205 3.25156133938931
15.4800000190735 3.23405095981175
15.5199999809265 3.32269457553571
15.5600001811981 3.31887655673157
15.6000001430511 3.40343196799003
15.6400001049042 3.40716664924312
15.6800000667572 3.40956343709971
15.7200000286102 3.31672020465732
15.7599999904633 3.24866368285574
15.8000001907349 3.32397850709351
15.8400001525879 3.32650928617088
15.8800001144409 3.20824210039003
15.9200000762939 3.18228634701947
15.960000038147 3.15800436307625
16 3.2586327824088
16.0400002002716 3.25299428130253
16.0800001621246 3.22557641523254
16.1200001239777 3.18870393876074
16.1600000858307 3.21108888850197
16.2000000476837 3.16667358794859
16.2400000095367 3.16149233014769
16.2800002098083 3.12760682312541
16.3200001716614 3.02746670675659
16.3600001335144 2.99932713581388
16.4000000953674 3.0794154043405
16.4400000572205 3.07811237480725
16.4800000190735 3.17230242186235
16.5199999809265 3.13040485931683
16.5600001811981 3.16553648913305
16.6000001430511 3.32523983633196
16.6400001049042 3.33552316814808
16.6800000667572 3.20596796864897
16.7200000286102 3.15341218549578
16.7599999904633 3.14067369762768
16.8000001907349 3.12112796439612
16.8400001525879 3.11060822932094
16.8800001144409 3.15898096982907
16.9200000762939 3.09511129442034
16.960000038147 3.23231214923136
17 3.19656782795671
17.0400002002716 3.23766812528487
17.0800001621246 3.20983969689052
17.1200001239777 3.25479426494558
17.1600000858307 3.18412109014656
17.2000000476837 3.18202625105208
17.2400000095367 3.16884716106373
17.2800002098083 3.20181181156275
17.3200001716614 3.19369118608081
17.3600001335144 3.20997527150776
17.4000000953674 3.24944315752324
17.4400000572205 3.12758829404063
17.4800000190735 3.10249432758436
17.5199999809265 3.13525716009609
17.5600001811981 3.02941396537664
17.6000001430511 3.04261737822625
17.6400001049042 3.05885295114235
17.6800000667572 3.08026320160039
17.7200000286102 2.98320854652992
};
\addplot [semithick, color2, mark=*, mark size=1, mark options={solid}, only marks, forget plot]
table {%
0.800000190734863 0.566310207694819
0.840000152587891 0.610843449352141
0.880000114440918 0.646629711214613
0.920000076293945 0.675662457956674
0.960000038146973 0.677383853894884
1 0.652233842837334
1.04000020027161 0.617705696642899
1.08000016212463 0.59707644187842
1.12000012397766 0.582773435789414
1.16000008583069 0.57319667237221
1.20000004768372 0.544899207724968
1.24000000953674 0.520278560671475
1.28000020980835 0.486819697340015
1.32000017166138 0.471235098019191
1.3600001335144 0.482251080844038
1.40000009536743 0.498766569171103
1.44000005722046 0.525970282627713
1.48000001907349 0.546432838878246
1.51999998092651 0.570399635333319
1.56000018119812 0.603940263122975
1.60000014305115 0.637042374110034
1.64000010490417 0.671570818616942
1.6800000667572 0.768573049015447
1.72000002861023 0.854839414125309
1.75999999046326 0.921061127208674
1.80000019073486 0.972855269388712
1.84000015258789 1.02464311800119
1.88000011444092 1.06445564214345
1.92000007629395 0.997485319848947
1.96000003814697 0.905785809928621
2 0.826862849189619
2.04000020027161 0.833231959205346
2.08000016212463 0.651094119210421
2.12000012397766 0.258213124322324
2.16000008583069 0.236399503264688
2.20000004768372 0.0448681660358898
2.24000000953674 0.0231399156933656
2.28000020980835 -0.0463956041838169
2.32000017166138 0.0291771100656393
2.3600001335144 0.043865510420214
2.40000009536743 0.0832582982314672
2.44000005722046 -0.0523143689140556
2.48000001907349 0.0410986709152589
2.51999998092651 0.0119072781358935
2.56000018119812 -0.0342830780656945
2.60000014305115 -0.0409483506911916
2.64000010490417 0.00288127999770402
2.6800000667572 0.0280583743270773
2.72000002861023 -0.00967951797540012
2.75999999046326 -0.0235366700101965
2.80000019073486 0.0118969778108895
2.84000015258789 0.0404327908569302
2.88000011444092 -0.0160557699592126
2.92000007629395 0.00551272847477687
2.96000003814697 0.0458368679025721
3 0.00325816038120176
3.04000020027161 0.0808821824303544
3.08000016212463 0.104864896536889
3.12000012397766 0.0942851146853216
3.16000008583069 0.12244417347231
3.20000004768372 0.142543943031819
3.24000000953674 0.158265555786577
3.28000020980835 0.178455439635281
3.32000017166138 0.0464456730218337
3.3600001335144 1.71684116383779
3.40000009536743 1.80984270687744
3.44000005722046 1.8959152879875
3.48000001907349 1.96452916292364
3.51999998092651 2.01682161590802
3.56000018119812 2.07905052524954
3.60000014305115 2.14780912078468
3.64000010490417 2.2216339958221
3.6800000667572 2.27966431102876
3.72000002861023 2.35061349917311
3.75999999046326 2.42485896977757
3.80000019073486 2.48595951961801
3.84000015258789 1.35608615843019
3.88000011444092 0.800633217619586
3.92000007629395 0.806348912787322
3.96000003814697 0.793682823714538
4 0.781876202946178
4.04000020027161 0.779983413198837
4.08000016212463 0.743868245406269
4.12000012397766 0.714927957546375
4.16000008583069 0.673641919155063
4.20000004768372 0.640113584474457
4.24000000953674 0.609827272549596
4.28000020980835 0.619925098724548
4.32000017166138 0.593513720024116
4.3600001335144 0.585118610765496
4.40000009536743 0.542491445026684
4.44000005722046 0.514798994022574
4.48000001907349 0.495125997341028
4.51999998092651 0.455079345551271
4.56000018119812 0.411902997749079
4.60000014305115 0.414135237819253
4.64000010490417 0.392015050219518
4.6800000667572 0.381675412418476
4.72000002861023 0.373460900595211
4.75999999046326 0.356430183844504
4.80000019073486 0.353605795312068
4.84000015258789 0.338985483920712
4.88000011444092 0.297862774116112
4.92000007629395 0.270229610660648
4.96000003814697 0.234806377942342
5 0.197459938713907
5.04000020027161 0.160225007734228
5.08000016212463 0.129040816772171
5.12000012397766 0.10975342962631
5.16000008583069 0.102104780014038
5.20000004768372 0.070453461075896
5.24000000953674 0.0478691878701982
5.28000020980835 0.0317514027692696
5.32000017166138 0.0180137507933136
5.3600001335144 0.0146181817805523
5.40000009536743 -0.0146888413641624
5.44000005722046 -0.0213805046728804
5.48000001907349 -0.00648853465014628
5.51999998092651 0.00530702145781812
5.56000018119812 0.00225931900450016
5.60000014305115 -0.00310089537004378
5.64000010490417 -0.00748847930732433
5.6800000667572 0.030737689679129
5.72000002861023 0.0600183831049996
5.75999999046326 0.0508558302315239
5.80000019073486 0.0936188338125774
5.84000015258789 0.102305691474107
5.88000011444092 0.105565447949982
5.92000007629395 0.0765239492002127
5.96000003814697 0.112394011334227
6 0.120328411551254
6.04000020027161 0.110537871326804
6.08000016212463 0.138945929325808
6.12000012397766 0.166163563309374
6.16000008583069 0.183933603125409
6.20000004768372 0.1953600528994
6.24000000953674 0.197118693448992
6.28000020980835 0.195238790931309
6.32000017166138 0.199424579114837
6.3600001335144 0.197564247717759
6.40000009536743 0.190046991038282
6.44000005722046 0.183229822318331
6.48000001907349 0.152979201536525
6.51999998092651 0.125625637089855
6.56000018119812 0.0841914873662651
6.60000014305115 0.0510819772389325
6.64000010490417 0.0504747655169298
6.6800000667572 0.00932592477742102
6.72000002861023 -0.0129447702698416
6.75999999046326 -0.0164690498439564
6.80000019073486 -0.0262103437160532
6.84000015258789 -0.0329670587294698
6.88000011444092 -0.0384263744283208
6.92000007629395 -0.0262605696872765
6.96000003814697 -0.0351780229515891
7 -0.0432309925837881
7.04000020027161 -0.0151981113250474
7.08000016212463 0.0338100476016889
7.12000012397766 0.0357679562669317
7.16000008583069 0.0592956410198714
7.20000004768372 0.0931031224978316
7.24000000953674 0.109539192761189
7.28000020980835 0.146419504304959
7.32000017166138 0.177827589876486
7.3600001335144 0.208639028888138
7.40000009536743 0.241338746802459
7.44000005722046 0.242672236542932
7.48000001907349 0.289665332091754
7.51999998092651 0.321895144934148
7.56000018119812 0.328060234949125
7.60000014305115 0.34673592934652
7.64000010490417 0.389457517074822
7.6800000667572 0.427037136242893
7.72000002861023 0.457571806512278
7.75999999046326 0.469860417641607
7.80000019073486 0.477150907636018
7.84000015258789 0.50638483287477
7.88000011444092 0.517117167656501
7.92000007629395 0.481972901194229
7.96000003814697 0.520853194232856
8 0.555521718227795
8.04000020027161 0.524745312629409
8.08000016212463 0.534539829863825
8.12000012397766 0.570375525037023
8.16000008583069 0.516393241180434
8.20000004768372 0.537096404126594
8.24000000953674 0.482310099671144
8.28000020980835 0.501713631853749
8.32000017166138 0.523627180729363
8.3600001335144 0.487648713697725
8.40000009536743 0.510649260771908
8.44000005722046 0.470145698223579
8.48000001907349 0.481780408790086
8.51999998092651 0.499816324526014
8.56000018119812 0.489557610179938
8.60000014305115 0.513996777222232
8.64000010490417 0.513186447871466
8.6800000667572 0.535450528900637
8.72000002861023 0.53446231575872
8.75999999046326 0.546404819940854
8.80000019073486 0.564148460926211
8.84000015258789 0.551930000078432
8.88000011444092 0.58428951203691
8.92000007629395 0.764741188347422
8.96000003814697 1.55375805788906
9 1.49972280383137
9.04000020027161 1.42908013881005
9.08000016212463 1.34565933230997
9.12000012397766 1.28410794596156
9.16000008583069 1.20190537542625
9.20000004768372 1.05106507492809
9.24000000953674 0.97219118376518
9.28000020980835 0.85186529657332
9.32000017166138 0.793467436620649
9.3600001335144 0.711929168654996
9.40000009536743 0.636589809799067
9.44000005722046 0.0111924741370204
9.48000001907349 0.184432560387042
9.51999998092651 0.0928070148279042
9.56000018119812 -0.0035604359525262
9.60000014305115 0.00764646137104876
9.64000010490417 0.109509650398619
9.6800000667572 0.183795393463004
9.72000002861023 0.28361730148163
9.75999999046326 0.369584729844607
9.80000019073486 0.436554177496068
9.84000015258789 0.53993090817934
9.88000011444092 0.60936556681791
9.92000007629395 0.701576671484731
9.96000003814697 0.766588155874757
10 0.795320818150348
10.0400002002716 0.822163739214917
10.0800001621246 0.819465444767499
10.1200001239777 0.160630832338604
10.1600000858307 0.368721974828637
10.2000000476837 0.429991782721702
10.2400000095367 0.491752522703567
10.2800002098083 0.55212876541671
10.3200001716614 0.57109841778446
10.3600001335144 0.623836037549689
10.4000000953674 0.67635406648016
10.4400000572205 0.702848258560234
10.4800000190735 0.738574248868264
10.5199999809265 0.787722261072312
10.5600001811981 0.829154987026429
10.6000001430511 0.390461506995752
10.6400001049042 0.088210622555838
10.6800000667572 0.118442428606877
10.7200000286102 0.0862487188791029
10.7599999904633 0.109056080858271
10.8000001907349 0.0999966986860275
10.8400001525879 0.112223766509256
10.8800001144409 0.125940889207166
10.9200000762939 0.0939719427990252
10.960000038147 0.0954343909285349
11 0.0495611679202714
11.0400002002716 0.0499953368987809
11.0800001621246 0.0522129403243237
11.1200001239777 0.000892564743278264
11.1600000858307 0.00416056760696292
11.2000000476837 -0.0114628756722205
11.2400000095367 -0.0287236270662935
11.2800002098083 -0.0303494237736359
11.3200001716614 0.00908528571427872
11.3600001335144 0.00101057931345877
11.4000000953674 0.031681532710757
11.4400000572205 0.0330757205978438
11.4800000190735 0.0441293978755079
11.5199999809265 0.0546038547184303
11.5600001811981 0.0586091412894697
11.6000001430511 0.0628140335914317
11.6400001049042 0.0870766563838677
11.6800000667572 0.0898512604110598
11.7200000286102 0.0952771450046767
11.7599999904633 0.0934841837491885
11.8000001907349 0.0864083913513956
11.8400001525879 0.0849029112771754
11.8800001144409 0.0901387050616518
11.9200000762939 0.109423038511248
11.960000038147 0.117090565579414
12 0.128960814528231
12.0400002002716 0.146061296707827
12.0800001621246 0.140733140537947
12.1200001239777 0.154080838830084
12.1600000858307 0.133019293641508
12.2000000476837 0.126440674501128
12.2400000095367 0.11672144334535
12.2800002098083 0.0787771936971545
12.3200001716614 0.0256632256961259
12.3600001335144 0.0261733034384334
12.4000000953674 0.0303289529064699
12.4400000572205 -0.000944586122840076
12.4800000190735 -0.00968484800864877
12.5199999809265 -0.000273297049830694
12.5600001811981 -0.0052414452892844
12.6000001430511 0.0147797617976586
12.6400001049042 0.0146422796429012
12.6800000667572 0.0333753775917582
12.7200000286102 0.0594680698933437
12.7599999904633 0.0433029877477595
12.8000001907349 0.0758864451535606
12.8400001525879 0.132975106045697
12.8800001144409 0.152004155587209
12.9200000762939 0.193625398094346
12.960000038147 0.193292114022355
13 0.234876265889778
13.0400002002716 0.244397281672815
13.0800001621246 0.292911429532265
13.1200001239777 0.345180930434846
13.1600000858307 0.341011107825739
13.2000000476837 0.35596996360535
13.2400000095367 0.353265448015626
13.2800002098083 0.384979407064744
13.3200001716614 0.435744258130932
13.3600001335144 0.411237814555455
13.4000000953674 0.436464773778761
13.4400000572205 0.443470334011553
13.4800000190735 0.454673119389233
13.5199999809265 0.463302464134554
13.5600001811981 0.425825911697083
13.6000001430511 0.44239121962162
13.6400001049042 0.397786357296248
13.6800000667572 0.386133315776477
13.7200000286102 0.346914837932545
13.7599999904633 0.33891940100642
13.8000001907349 0.341350930766195
13.8400001525879 0.309790035946524
13.8800001144409 0.29750845610216
13.9200000762939 0.291639044370678
13.960000038147 0.288427938829084
14 0.274808572329144
14.0400002002716 0.25634174722573
14.0800001621246 0.257866649496806
14.1200001239777 0.251856026769563
14.1600000858307 0.244557503850695
14.2000000476837 0.23708568710016
14.2400000095367 0.233878568140892
14.2800002098083 0.23902251210239
14.3200001716614 0.197539551329336
14.3600001335144 0.208320735808366
14.4000000953674 0.196750895956934
14.4400000572205 0.204102732860377
14.4800000190735 0.212694612034852
14.5199999809265 0.218067670237961
14.5600001811981 0.211442809596015
14.6000001430511 0.175364470214746
14.6400001049042 0.173946624046427
14.6800000667572 0.168364948373697
14.7200000286102 0.168050266272658
14.7599999904633 0.17601600013847
14.8000001907349 0.181933060884221
14.8400001525879 0.177373849603686
14.8800001144409 0.197899952538733
14.9200000762939 0.213707804393029
14.960000038147 0.271238492358582
15 0.251444165503166
15.0400002002716 0.301896236658166
15.0800001621246 0.371651485747634
15.1200001239777 0.346971511720756
15.1600000858307 0.39226866970485
15.2000000476837 0.380482063803515
15.2400000095367 0.399623485315372
15.2800002098083 0.437694281824329
15.3200001716614 0.429529490736138
15.3600001335144 0.444422087994565
15.4000000953674 0.424034117137449
15.4400000572205 0.466133375252341
15.4800000190735 0.518911727566767
15.5199999809265 0.481066425741159
15.5600001811981 0.509581586610643
15.6000001430511 0.466325235525048
15.6400001049042 0.488396030737598
15.6800000667572 0.494905862370299
15.7200000286102 0.473714082063661
15.7599999904633 0.486474816098836
15.8000001907349 0.533609134022427
15.8400001525879 0.541887694552721
15.8800001144409 0.570135719502637
15.9200000762939 0.515307093370158
15.960000038147 0.516284628827609
16 0.330152038730749
16.0400002002716 0.367435154613507
16.0800001621246 0.327668306125615
16.1200001239777 0.34750467826179
16.1600000858307 0.352935054294176
16.2000000476837 0.391503327390787
16.2400000095367 0.432523304960908
16.2800002098083 0.406801371463973
16.3200001716614 0.45341647350826
16.3600001335144 0.505841648368469
16.4000000953674 0.467727895771039
16.4400000572205 0.506102889100949
16.4800000190735 0.473141046614069
16.5199999809265 0.479638922135009
16.5600001811981 0.525803814491428
16.6000001430511 0.515817534573067
16.6400001049042 0.514998743522826
16.6800000667572 0.480735315134913
16.7200000286102 0.478211176518587
16.7599999904633 0.477970710889447
16.8000001907349 0.274700255046721
16.8400001525879 0.260458180035826
16.8800001144409 0.0784179343925006
16.9200000762939 0.0813456511950165
16.960000038147 0.0119598054522513
17 0.00689240438001953
17.0400002002716 0.0203330215755822
17.0800001621246 -0.0136044825773348
17.1200001239777 0.0546710789614252
17.1600000858307 0.0640013333894688
17.2000000476837 0.193876301265352
17.2400000095367 0.12437566054101
17.2800002098083 0.0964831749665264
17.3200001716614 0.0776572298173204
17.3600001335144 0.137382696135798
17.4000000953674 0.0842097948356671
17.4400000572205 0.0489896024131397
17.4800000190735 0.116478444927321
17.5199999809265 0.0354244869242144
17.5600001811981 0.0598586379706848
17.6000001430511 0.0075663268172757
17.6400001049042 -0.0277578415043747
17.6800000667572 -0.11066865723301
17.7200000286102 -0.0933330142122363
17.7599999904633 -0.132659020156006
17.8000001907349 -0.1190174236814
17.8400001525879 -0.135165294037518
17.8800001144409 -0.176077792953201
17.9200000762939 -0.233190318349368
17.960000038147 -0.224460237723129
18 -0.276026780436551
18.0400002002716 -0.27100310888099
18.0800001621246 -0.281742996626321
18.1200001239777 -0.261303698134667
18.1600000858307 -0.305963607402613
18.2000000476837 -0.366999641402713
18.2400000095367 -0.415860436203306
18.2800002098083 -0.381925922617564
18.3200001716614 -0.449515713244959
18.3600001335144 -0.479008649191778
18.4000000953674 -0.526065153924264
18.4400000572205 -0.555062775086706
18.4800000190735 -0.563840935419121
18.5199999809265 -0.650808802236333
18.5600001811981 -0.645287875088315
18.6000001430511 -0.674056840978301
18.6400001049042 -0.696454080488008
18.6800000667572 -0.705973488418374
18.7200000286102 -0.687588581678176
18.7599999904633 -0.651796222040873
18.8000001907349 -0.669318613447536
18.8400001525879 -0.668670633539845
18.8800001144409 -0.699659836064121
18.9200000762939 -0.684550010908187
18.960000038147 -0.774139808445955
19 -0.847483846314773
19.0400002002716 -0.789853684681743
19.0800001621246 -0.882705806092499
19.1200001239777 -0.887991217056227
19.1600000858307 -1.00670539696924
19.2000000476837 -1.02696917033419
19.2400000095367 -1.10683648923957
19.2800002098083 -1.17721374071009
19.3200001716614 -1.14842622928867
19.3600001335144 -1.26307464292016
19.4000000953674 -1.33711854363836
19.4400000572205 -1.37062088606972
19.4800000190735 -1.45597983487223
19.5199999809265 -1.51829022184806
19.5600001811981 -1.59490033597692
19.6000001430511 -1.71039401653401
19.6400001049042 -1.79332094032618
19.6800000667572 -1.90688166443455
19.7200000286102 -1.93277090809491
19.7599999904633 -1.98157763298714
19.8000001907349 -2.01415135951048
19.8400001525879 -2.03603167152783
19.8800001144409 -2.08145048839823
19.9200000762939 -2.13427038389052
19.960000038147 -2.17986351736825
20 -2.2215365356194
20.0400002002716 -2.25179126968644
20.0800001621246 -2.27575958698248
20.1200001239777 -2.29509360091733
20.1600000858307 -2.35356574807719
20.2000000476837 -2.37988602596902
20.2400000095367 -2.39586358905932
20.2800002098083 -2.39791396949491
20.3200001716614 -2.41858756009156
20.3600001335144 -2.43708588184386
20.4000000953674 -2.44596525147163
20.4400000572205 -2.47621739818731
20.4800000190735 -2.49832537918992
20.5199999809265 -2.5289021242192
20.5600001811981 -2.54101956334867
20.6000001430511 -2.51667965594035
20.6400001049042 -2.52324005282374
20.6800000667572 -2.53152789311519
20.7200000286102 -2.56837064707082
20.7599999904633 -2.58717134270261
20.8000001907349 -2.60497636494775
20.8400001525879 -2.65142507249725
20.8800001144409 -2.66978526305454
20.9200000762939 -2.72750643483382
20.960000038147 -2.75167688992973
21 -2.77789235085584
21.0400002002716 -2.81564277616127
21.0800001621246 -2.82378367018273
21.1200001239777 -2.88394419058188
21.1600000858307 -2.92664984135118
21.2000000476837 -2.96763101280468
21.2400000095367 -3.00754755581002
21.2800002098083 -3.01767951747502
21.3200001716614 -3.04959447355103
21.3600001335144 -3.06483539860865
21.4000000953674 -3.08108535471511
21.4400000572205 -3.09988038177406
21.4800000190735 -3.07938192962744
21.5199999809265 -3.07450179160106
21.5600001811981 -3.11126867081998
21.6000001430511 -3.0772164112275
21.6400001049042 -3.09095531866039
21.6800000667572 -3.10269302185509
21.7200000286102 -3.10620439626484
21.7599999904633 -3.14846644360885
21.8000001907349 -3.15143292877913
21.8400001525879 -3.17539500684569
21.8800001144409 -3.16336295867154
21.9200000762939 -3.16684817754278
21.960000038147 -3.19276571448159
22 -3.16221074430849
22.0400002002716 -3.14962689474149
22.0800001621246 -3.19963011182661
22.1200001239777 -3.15126987289506
22.1600000858307 -3.17046951271812
22.2000000476837 -3.17009791109698
22.2400000095367 -3.18408447135549
22.2800002098083 -3.20265986033878
22.3200001716614 -3.18135070467814
22.3600001335144 -3.23107140953676
22.4000000953674 -3.20049992058705
22.4400000572205 -3.205351533558
22.4800000190735 -3.24689020341992
22.5199999809265 -3.24618794564699
22.5600001811981 -3.31296933309177
22.6000001430511 -3.32238731323191
22.6400001049042 -3.32085445451615
22.6800000667572 -3.35386363807835
22.7200000286102 -3.33440467245594
22.7599999904633 -3.27004764585176
22.8000001907349 -3.31705378775831
22.8400001525879 -3.32534795773888
22.8800001144409 -3.35238068703109
22.9200000762939 -3.30895890740336
22.960000038147 -3.30847977971329
23 -3.31484166481725
23.0400002002716 -3.31039253919606
23.0800001621246 -3.30921887967748
23.1200001239777 -3.31496718737862
23.1600000858307 -3.30344205879509
23.2000000476837 -3.3262146420195
23.2400000095367 -3.27484787419676
23.2800002098083 -3.26676867533237
23.3200001716614 -3.27966046112407
23.3600001335144 -3.25184571148178
23.4000000953674 -3.25812513888116
23.4400000572205 -3.23622625108198
23.4800000190735 -3.20261363734703
23.5199999809265 -3.2068762297282
23.5600001811981 -3.17399343737053
23.6000001430511 -3.17085537683501
23.6400001049042 -3.09102151167569
23.6800000667572 -3.07990074763016
23.7200000286102 -3.09147149598831
23.7599999904633 -3.04601365509262
23.8000001907349 -3.00431183095612
23.8400001525879 -2.96059977383607
23.8800001144409 -2.92070586405343
23.9200000762939 -2.88615294547282
23.960000038147 -2.84515825858977
24 -2.81476280561682
24.0400002002716 -2.76147381586448
24.0800001621246 -2.74818468608631
24.1200001239777 -2.71746692050952
24.1600000858307 -2.6698666124587
24.2000000476837 -2.63814799391607
24.2400000095367 -2.5860266896065
24.2800002098083 -2.57211902024349
24.3200001716614 -2.51950719128724
24.3600001335144 -2.58940700535638
24.4000000953674 -2.59120146933747
24.4400000572205 -2.56571001634112
24.4800000190735 -2.55655367996439
24.5199999809265 -2.53298952777565
24.5600001811981 -2.50554421944333
24.6000001430511 -2.4825804451454
24.6400001049042 -2.43864632084909
24.6800000667572 -2.43142804747703
24.7200000286102 -2.41183840668535
24.7599999904633 -2.39964292712566
24.8000001907349 -2.39825341311332
24.8400001525879 -2.46718946023636
24.8800001144409 -2.46491657907907
24.9200000762939 -2.42284496173631
24.960000038147 -2.40576869199577
25 -2.39544185452559
25.0400002002716 -2.35186107948322
25.0800001621246 -2.34811620717847
25.1200001239777 -2.32657782026529
25.1600000858307 -2.31207744852484
25.2000000476837 -2.31167722606531
25.2400000095367 -2.29977998642911
25.2800002098083 -2.29588080764236
25.3200001716614 -2.27847447474836
25.3600001335144 -2.2741632278136
25.4000000953674 -2.27419167934333
25.4400000572205 -2.26377251310878
25.4800000190735 -2.2628438239207
25.5199999809265 -2.2697114643276
25.5600001811981 -2.24988932495678
25.6000001430511 -2.23938165513604
25.6400001049042 -2.23462903161494
25.6800000667572 -2.22760710015989
25.7200000286102 -2.22802254104448
25.7599999904633 -2.2445938752599
25.8000001907349 -2.24200788333704
25.8400001525879 -2.23557840532604
25.8800001144409 -2.22249489206524
25.9200000762939 -2.21677273716648
25.960000038147 -2.22533793896382
26 -2.23404312104643
26.0400002002716 -2.2593390499648
26.0800001621246 -2.27615922686597
26.1200001239777 -2.28648302737267
26.1600000858307 -2.30341907296632
26.2000000476837 -2.34453838697522
26.2400000095367 -2.4281880745939
26.2800002098083 -2.43865109641466
26.3200001716614 -2.48650210017652
26.3600001335144 -2.44970634277806
26.4000000953674 -2.45475809371296
26.4400000572205 -2.46755026655577
26.4800000190735 -2.47681788892842
26.5199999809265 -2.48396338592078
26.5600001811981 -2.49725018920557
26.6000001430511 -2.50490773581862
26.6400001049042 -2.49390751057243
26.6800000667572 -2.50288433766719
26.7200000286102 -2.4513887598724
26.7599999904633 -2.45362101184234
26.8000001907349 -2.47598063118207
26.8400001525879 -2.50348544767923
26.8800001144409 -2.52222594099045
26.9200000762939 -2.53797841460131
26.960000038147 -2.54510518718761
27 -2.57157083690572
27.0400002002716 -2.62029681031889
27.0800001621246 -2.62464812958259
27.1200001239777 -2.64189720240698
27.1600000858307 -2.6526236577082
27.2000000476837 -2.67282090842203
27.2400000095367 -2.65491264456947
27.2800002098083 -2.6574144228119
27.3200001716614 -2.70728465858581
27.3600001335144 -2.71422010483035
27.4000000953674 -2.73367314242464
27.4400000572205 -2.75768024577716
27.4800000190735 -2.77047956984109
27.5199999809265 -2.74682595589388
27.5600001811981 -2.75181138442918
27.6000001430511 -2.76049805157132
27.6400001049042 -2.73587135978063
27.6800000667572 -2.71002586255824
27.7200000286102 -2.73210615906794
27.7599999904633 -2.73611180861448
27.8000001907349 -2.71526808339848
27.8400001525879 -2.71963398659206
27.8800001144409 -2.74113967170981
27.9200000762939 -2.80385364035568
27.960000038147 -2.81192461418436
28 -2.82151117847386
28.0400002002716 -2.8662003204457
28.0800001621246 -2.86789730066448
28.1200001239777 -2.89002520328185
28.1600000858307 -2.8973569880417
28.2000000476837 -2.88877312670165
28.2400000095367 -2.92543677157336
28.2800002098083 -2.87852247512787
28.3200001716614 -2.87398743011126
28.3600001335144 -2.88075960259602
28.4000000953674 -2.83739239807695
28.4400000572205 -2.81173675389736
28.4800000190735 -2.81765386803174
28.5199999809265 -2.78147043421476
28.5600001811981 -2.78150852027185
28.6000001430511 -2.75318300056105
28.6400001049042 -2.76884247961643
28.6800000667572 -2.75838117614253
28.7200000286102 -2.76060105664844
28.7599999904633 -2.75837731917145
28.8000001907349 -2.76579487820908
28.8400001525879 -2.87378495935205
28.8800001144409 -2.8760473519881
28.9200000762939 -2.90582607072331
28.960000038147 -2.91425939096074
29 -2.93657455293848
29.0400002002716 -2.94575688980919
29.0800001621246 -2.97916976518797
29.1200001239777 -3.01724646396958
29.1600000858307 -3.02416281957772
29.2000000476837 -2.99568116818603
29.2400000095367 -3.02261579426044
29.2800002098083 -3.00669704736426
29.3200001716614 -3.05927785604151
29.3600001335144 -3.06683659375732
29.4000000953674 -3.0684727393434
29.4400000572205 -3.0380287586231
29.4800000190735 -3.0375686355351
29.5199999809265 -3.04340467587728
29.5600001811981 -3.04687883838274
29.6000001430511 -2.98890304163213
29.6400001049042 -2.9990769212831
29.6800000667572 -2.95547896174324
29.7200000286102 -2.98736340999783
29.7599999904633 -2.98522332973768
29.8000001907349 -2.99763381204247
29.8400001525879 -3.01471429135446
29.8800001144409 -3.0108347675208
29.9200000762939 -3.03038114623164
29.960000038147 -3.02784021447056
30 -3.00634249675779
30.0400002002716 -3.04254666053674
30.0800001621246 -3.03064264155103
30.1200001239777 -3.07279285756794
30.1600000858307 -3.07773873836244
30.2000000476837 -3.06462669168446
30.2400000095367 -3.12608503955543
30.2800002098083 -3.12523114745277
30.3200001716614 -3.09766465266305
30.3600001335144 -3.14790265819908
30.4000000953674 -3.26171812992882
30.4400000572205 -3.23586516421113
30.4800000190735 -3.22879795246832
30.5199999809265 -3.24008360942349
30.5600001811981 -3.23119952952039
30.6000001430511 -3.22169658525557
30.6400001049042 -3.23104687855408
30.6800000667572 -3.21673438359682
30.7200000286102 -3.20422933458226
30.7599999904633 -3.19849504325427
30.8000001907349 -3.15316621136524
30.8400001525879 -3.11502467137735
30.8800001144409 -3.09366118605176
30.9200000762939 -3.12221302091524
30.960000038147 -3.10746115952173
31 -3.11950373313499
31.0400002002716 -3.11420236875809
31.0800001621246 -3.1462762997198
31.1200001239777 -3.12190182512929
31.1600000858307 -3.1413995377537
31.2000000476837 -3.07391465867788
31.2400000095367 -3.07354459741406
31.2800002098083 -3.01698151800646
31.3200001716614 -2.99326740638999
31.3600001335144 -2.97976676209942
31.4000000953674 -2.9505046102769
31.4400000572205 -2.92490161649248
31.4800000190735 -2.90618219454163
31.5199999809265 -2.86002318940931
31.5600001811981 -2.90033390738575
31.6000001430511 -2.96704108478009
31.6400001049042 -3.07472301259415
31.6800000667572 -3.06883054768239
31.7200000286102 -3.00946945013467
31.7599999904633 -2.97002807907616
31.8000001907349 -2.95274972630053
31.8400001525879 -2.85992727628906
31.8800001144409 -2.78052903604534
31.9200000762939 -2.70911716382429
31.960000038147 -2.64946456537842
32 -2.51962526279854
32.0400002002716 -2.46365620557724
32.0800001621246 -2.47920235855583
32.1200001239777 -2.26170462784135
32.1600000858307 -2.20704535742009
32.2000000476837 -2.24299002479363
32.2400000095367 -2.20396445839646
32.2800002098083 -2.09618641617809
32.3200001716614 -1.9770392294933
32.3600001335144 -1.97561668216405
32.4000000953674 -1.9571304304276
32.4400000572205 -1.92008592337765
32.4800000190735 -1.84891699494613
32.5199999809265 -1.82512187252416
32.5600001811981 -1.71903130640654
32.6000001430511 -1.62936754554125
32.6400001049042 -1.59266188297517
32.6800000667572 -1.62128089127122
32.7200000286102 -1.60972519106358
32.7599999904633 -1.58332909982517
32.8000001907349 -1.64517557763296
32.8400001525879 -1.55431943700714
32.8800001144409 -1.09901995845896
32.9200000762939 -1.01675410008137
32.960000038147 -0.933309716740302
33 -0.898074741824657
33.0400002002716 -0.829436605363087
33.0800001621246 -0.783736095627514
33.1200001239777 -0.742654390037021
33.1600000858307 -0.70801887724927
33.2000000476837 -0.601050299094892
33.2400000095367 -0.522436921926209
33.2800002098083 -0.381034988096312
33.3200001716614 -0.350820764074171
33.3600001335144 -0.683945456779577
33.4000000953674 -0.879792398929308
33.4400000572205 -0.792223194796082
33.4800000190735 -0.760605237564636
33.5199999809265 -0.691684602925099
33.5600001811981 -0.668958795737696
33.6000001430511 -0.575949471987297
33.6400001049042 -0.661034602890749
33.6800000667572 -0.70473204934717
33.7200000286102 -0.685390736011053
33.7599999904633 -0.675678718792236
33.8000001907349 -0.615669240678282
33.8400001525879 -0.602849580566696
33.8800001144409 -0.565048967036272
33.9200000762939 -0.577432030661821
33.960000038147 -0.555647910091302
34 -0.634970574233768
34.0400002002716 -0.680970577002872
34.0800001621246 -0.736431421807733
34.1200001239777 -0.775654138014481
34.1600000858307 -0.800359213922397
};
\path [draw=black, fill=black] (axis cs:15,0.12)
--(axis cs:15.5,-0.1)
--(axis cs:15.125,-0.1)
--(axis cs:15.125,-2)
--(axis cs:14.875,-2)
--(axis cs:14.875,-0.1)
--(axis cs:14.5,-0.1)
--cycle;

\path [draw=black, fill=black] (axis cs:17,2.97)
--(axis cs:17.5,2.75)
--(axis cs:17.125,2.75)
--(axis cs:17.125,2.2)
--(axis cs:16.875,2.2)
--(axis cs:16.875,2.75)
--(axis cs:16.5,2.75)
--cycle;

\addplot [line width=2.4000000000000004pt, color1, dashed, forget plot]
table {%
0 -0.0417787396782297
0.0400002002716064 -0.0234894145660781
0.0800001621246338 -0.00316769069747583
0.120000123977661 0.0171410027993888
0.160000085830688 0.0353008948896494
0.200000047683716 0.0519132876052167
0.240000009536743 0.047620933155939
0.28000020980835 0.0354367334666362
0.320000171661377 0.0243375702749589
0.360000133514404 0.0147874162005871
0.400000095367432 0.00626683768990029
0.440000057220459 -0.00117534729387254
0.480000019073486 -0.00759592921282741
0.519999980926514 -0.0134284232414302
0.56000018119812 -0.0182270835009956
0.600000143051147 -0.022438744343384
0.640000104904175 -0.026070568590947
0.680000066757202 -0.0291511821203337
0.720000028610229 -0.0319400999575896
0.759999990463257 -0.0338169274746991
0.800000190734863 -0.0352436446804821
0.840000152587891 -0.0363389452063536
0.880000114440918 -0.0371832000288373
0.920000076293945 -0.0378755526499222
0.960000038146973 -0.0379489570409571
1 -0.0377584947677796
1.04000020027161 -0.0376390233325493
1.08000016212463 -0.0374952551139339
1.12000012397766 -0.0372917578207389
1.16000008583069 -0.0369938784342557
1.20000004768372 -0.0366344301219262
1.24000000953674 -0.0360929457393068
1.28000020980835 -0.0354371027776929
1.32000017166138 -0.0345109548601747
1.3600001335144 -0.0334581229483179
1.40000009536743 -0.0324192791695158
1.44000005722046 -0.030980956112695
1.48000001907349 -0.0293790907626932
1.51999998092651 -0.0272088269687246
1.56000018119812 -0.024882385056818
1.60000014305115 -0.0226886086533483
1.64000010490417 -0.0190998139744281
1.6800000667572 -0.0151483238709784
1.72000002861023 -0.0110182941941803
1.75999999046326 -0.00714158725049169
1.80000019073486 -0.00356854737946884
1.84000015258789 0.00048544445960752
1.88000011444092 0.00451394310810775
1.92000007629395 0.00841177825825727
1.96000003814697 0.0119801034553173
2 0.01527436559098
2.04000020027161 0.0187924828262535
2.08000016212463 0.0222216665186688
2.12000012397766 0.0254353989836296
2.16000008583069 0.0283560527972835
2.20000004768372 0.0310911488127918
2.24000000953674 0.0341602267221054
2.28000020980835 0.037264868811495
2.32000017166138 0.0403389177592532
2.3600001335144 0.0432238920390264
2.40000009536743 0.0463886045314703
2.44000005722046 0.0496490639713337
2.48000001907349 0.0526877993519101
2.51999998092651 0.0564129153244892
2.56000018119812 0.0602846008974836
2.60000014305115 0.0638569429020855
2.64000010490417 0.0677044802291149
2.6800000667572 0.0715338351926627
2.72000002861023 0.0756254784338014
2.75999999046326 0.079603058963355
2.80000019073486 0.0832873003763861
2.84000015258789 0.0874524012518538
2.88000011444092 0.0916925884473081
2.92000007629395 0.0960618042740521
2.96000003814697 0.100200412245092
3 0.104271661953581
3.04000020027161 0.108167817374537
3.08000016212463 0.111783227709938
3.12000012397766 0.115684704289017
3.16000008583069 0.119522929461469
3.20000004768372 0.12315851979343
3.24000000953674 0.125141227311931
3.28000020980835 0.124029761602394
3.32000017166138 0.122598006925633
3.3600001335144 0.121279407271869
3.40000009536743 0.119340532654102
3.44000005722046 0.117189462631772
3.48000001907349 0.127221066263764
3.51999998092651 0.113301125155132
3.56000018119812 0.11144265952714
3.60000014305115 0.109868516394461
3.64000010490417 0.107719769339863
3.6800000667572 0.106500928672304
3.72000002861023 0.105042468454708
3.75999999046326 0.103877521716361
3.80000019073486 0.102937771611001
3.84000015258789 0.102171915711101
3.88000011444092 0.101579580501347
3.92000007629395 0.101226654302596
3.96000003814697 0.101113989199034
4 0.101616293426719
4.04000020027161 0.102418837022119
4.08000016212463 0.103231067324208
4.12000012397766 0.104120570963666
4.16000008583069 0.105118444100313
4.20000004768372 0.1061130722062
4.24000000953674 0.107099502125028
4.28000020980835 0.10808039216406
4.32000017166138 0.10905367990172
4.3600001335144 0.110000619620419
4.40000009536743 0.110710271456434
4.44000005722046 0.121753809426045
4.48000001907349 0.111943376392363
4.51999998092651 0.112505432724912
4.56000018119812 0.113037328530031
4.60000014305115 0.113606122874255
4.64000010490417 0.114212016721308
4.6800000667572 0.114937984492269
4.72000002861023 0.115942378212514
4.75999999046326 0.117000394895368
4.80000019073486 0.118307752468225
4.84000015258789 0.119605674340697
4.88000011444092 0.120902193469763
4.92000007629395 0.122538104963462
4.96000003814697 0.124289500251855
5 0.126108929838909
5.04000020027161 0.127843713065531
5.08000016212463 0.129489474225519
5.12000012397766 0.131066693075334
5.16000008583069 0.132579986193214
5.20000004768372 0.13395889045711
5.24000000953674 0.135024695265243
5.28000020980835 0.136097275109658
5.32000017166138 0.137301891448054
5.3600001335144 0.138522282896608
5.40000009536743 0.140322133246502
5.44000005722046 0.142360937875605
5.48000001907349 0.144572170213079
5.51999998092651 0.146722021325224
5.56000018119812 0.148727839038772
5.60000014305115 0.150742032117834
5.64000010490417 0.152816424101772
5.6800000667572 0.154949766866354
5.72000002861023 0.157208720762795
5.75999999046326 0.159367421435761
5.80000019073486 0.161433522444049
5.84000015258789 0.163369160557194
5.88000011444092 0.164821837764834
5.92000007629395 0.166124342488959
5.96000003814697 0.167369968847014
6 0.168116567065514
6.04000020027161 0.168625163666823
6.08000016212463 0.168751701543232
6.12000012397766 0.168478845217233
6.16000008583069 0.168307856437836
6.20000004768372 0.168231728600646
6.24000000953674 0.168188884458664
6.28000020980835 0.167959200938365
6.32000017166138 0.167825517347502
6.3600001335144 0.167778793190581
6.40000009536743 0.167686063493615
6.44000005722046 0.167577660315265
6.48000001907349 0.167414043164538
6.51999998092651 0.167352124229287
6.56000018119812 0.16730211208751
6.60000014305115 0.16726784797416
6.64000010490417 0.166864112795472
6.6800000667572 0.165765939475376
6.72000002861023 0.164868805072815
6.75999999046326 0.164104203344044
6.80000019073486 0.163175634034676
6.84000015258789 0.162404237764292
6.88000011444092 0.161376542242743
6.92000007629395 0.160366243110981
6.96000003814697 0.159332653747468
7 0.158396458797799
7.04000020027161 0.157743728819583
7.08000016212463 0.157670882961741
7.12000012397766 0.157998181690912
7.16000008583069 0.158570906013077
7.20000004768372 0.159516228425704
7.24000000953674 0.160643503265796
7.28000020980835 0.162220036107218
7.32000017166138 0.163808533736528
7.3600001335144 0.165476506757328
7.40000009536743 0.167105373620904
7.44000005722046 0.168665782559703
7.48000001907349 0.169805000798356
7.51999998092651 0.170751800959562
7.56000018119812 0.171827140200371
7.60000014305115 0.172935017713235
7.64000010490417 0.174222904089218
7.6800000667572 0.175631408882665
7.72000002861023 0.176995424582933
7.75999999046326 0.178355224513547
7.80000019073486 0.179779153121134
7.84000015258789 0.181321531430517
7.88000011444092 0.183416695002654
7.92000007629395 0.185557119325337
7.96000003814697 0.188003976170535
8 0.190525844739645
8.04000020027161 0.193066122560267
8.08000016212463 0.195395145382109
8.12000012397766 0.197570964148006
8.16000008583069 0.199726411527884
8.20000004768372 0.201854983180803
8.24000000953674 0.203986200810348
8.28000020980835 0.206255365030284
8.32000017166138 0.208366027970262
8.3600001335144 0.210387484551534
8.40000009536743 0.212325539830309
8.44000005722046 0.214381526209379
8.48000001907349 0.216517628578634
8.51999998092651 0.218616770703851
8.56000018119812 0.221095809107547
8.60000014305115 0.22355554916839
8.64000010490417 0.226048185659467
8.6800000667572 0.228557681507201
8.72000002861023 0.230702647590885
8.75999999046326 0.23243481748732
8.80000019073486 0.234081004254391
8.84000015258789 0.23558011349421
8.88000011444092 0.236788258690557
8.92000007629395 0.237924488952279
8.96000003814697 0.238731923484593
9 0.239499077388328
9.04000020027161 0.239734043719992
9.08000016212463 0.239799866402045
9.12000012397766 0.239988545359118
9.16000008583069 0.240262057773767
9.20000004768372 0.27249044538246
9.24000000953674 0.289991062240143
9.28000020980835 0.306487324447838
9.32000017166138 0.322004096110039
9.3600001335144 0.336267088026604
9.40000009536743 0.350096351134862
9.44000005722046 0.363051527791928
9.48000001907349 0.375882951802808
9.51999998092651 0.388226280791758
9.56000018119812 0.400754764070436
9.60000014305115 0.414008879288613
9.64000010490417 0.427690903121654
9.6800000667572 0.441974399828963
9.72000002861023 0.457064922618433
9.75999999046326 0.473005982408575
9.80000019073486 0.490280442656957
9.84000015258789 0.509064261184364
9.88000011444092 0.529485993180569
9.92000007629395 0.552115731425192
9.96000003814697 0.576684401166912
10 0.60325270940883
10.0400002002716 0.649831130856617
10.0800001621246 0.663265913770234
10.1200001239777 0.697282600223065
10.1600000858307 0.734081533815605
10.2000000476837 0.773597182126251
10.2400000095367 0.815988606103267
10.2800002098083 0.861317943809147
10.3200001716614 0.909598423165704
10.3600001335144 0.961256891333492
10.4000000953674 1.01645285377695
10.4400000572205 1.07552925798337
10.4800000190735 1.13846920296723
10.5199999809265 1.20554806746357
10.5600001811981 1.27517714672044
10.6000001430511 1.35045689811077
10.6400001049042 1.42734502594035
10.6800000667572 1.5065466494216
10.7200000286102 1.58753967702282
10.7599999904633 1.66973949662613
10.8000001907349 1.75282800954276
10.8400001525879 1.8364926698402
10.8800001144409 1.92004994454084
10.9200000762939 2.00314700538797
10.960000038147 2.08522734424601
11 2.16606889393365
11.0400002002716 2.2451298367563
11.0800001621246 2.32178394087486
11.1200001239777 2.39562664107728
11.1600000858307 2.46621406025363
11.2000000476837 2.55660417158619
11.2400000095367 2.59551971083539
11.2800002098083 2.65369595965046
11.3200001716614 2.70647836614852
11.3600001335144 2.75369088275477
11.4000000953674 2.79472028564417
11.4400000572205 2.82932207898036
11.4800000190735 2.85686386213842
11.5199999809265 2.87708908193203
11.5600001811981 2.88963753824534
11.6000001430511 2.89418151611971
11.6400001049042 2.89020334658292
11.6800000667572 2.87707720121421
11.7200000286102 2.85438721110032
11.7599999904633 2.82163607458536
11.8000001907349 2.81946235112765
11.8400001525879 2.82116591494254
11.8800001144409 2.82292899929882
11.9200000762939 2.82487931454895
11.960000038147 2.82687799052953
12 2.82901084490146
12.0400002002716 2.83118772363196
12.0800001621246 2.83336431404865
12.1200001239777 2.83554041684541
12.1600000858307 2.83770421958513
12.2000000476837 2.83941622049006
12.2400000095367 2.84091721275971
12.2800002098083 2.84239674481811
12.3200001716614 2.84384764503737
12.3600001335144 2.84521260282666
12.4000000953674 2.84626790164339
12.4400000572205 2.84747487840261
12.4800000190735 2.84850920790293
12.5199999809265 2.8494915962145
12.5600001811981 2.85025217788525
12.6000001430511 2.85099510747219
12.6400001049042 2.85197948850158
12.6800000667572 2.85318728562048
12.7200000286102 2.8545628921675
12.7599999904633 2.85601886621953
12.8000001907349 2.8575184955966
12.8400001525879 2.85890121794457
12.8800001144409 2.85989187934172
12.9200000762939 2.86090450652759
12.960000038147 2.86173581345884
13 2.86263661943759
13.0400002002716 2.86365100062622
13.0800001621246 2.86510819226026
13.1200001239777 2.86678465209858
13.1600000858307 2.86855735242891
13.2000000476837 2.87037406523006
13.2400000095367 2.87229955612768
13.2800002098083 2.87434111080212
13.3200001716614 2.87639503129837
13.3600001335144 2.8784233679331
13.4000000953674 2.88032740506776
13.4400000572205 2.88234655692363
13.4800000190735 2.88488259176671
13.5199999809265 2.88750131518647
13.5600001811981 2.89008614787922
13.6000001430511 2.89265031655973
13.6400001049042 2.89572143181652
13.6800000667572 2.89886507095544
13.7200000286102 2.90189527304969
13.7599999904633 2.90485947248228
13.8000001907349 2.90775999261128
13.8400001525879 2.91064154555121
13.8800001144409 2.9135789669666
13.9200000762939 2.91669274001231
13.960000038147 2.92104549711275
14 2.92549185329583
14.0400002002716 2.9310390710285
14.0800001621246 2.93652863347197
14.1200001239777 2.94194056627251
14.1600000858307 2.9481758322133
14.2000000476837 2.95432151438774
14.2400000095367 2.96028160901867
14.2800002098083 2.96602889185049
14.3200001716614 2.97117268010714
14.3600001335144 2.97577392363073
14.4000000953674 2.98027180157868
14.4400000572205 2.98479270070921
14.4800000190735 2.9891602958126
14.5199999809265 2.99374931884239
14.5600001811981 2.99818193068313
14.6000001430511 3.00241745373083
14.6400001049042 3.0061846823938
14.6800000667572 3.00978859099482
14.7200000286102 3.01305770668934
14.7599999904633 3.01603644197872
14.8000001907349 3.01867789969351
14.8400001525879 3.02114224926212
14.8800001144409 3.02332188552257
14.9200000762939 3.02522324269404
14.960000038147 3.02714463113932
15 3.02902239115438
15.2000000476837 3.03581285876219
15.2400000095367 3.03682919669528
15.2800002098083 3.03647918557522
15.3200001716614 3.03588694612308
15.3600001335144 3.03560266712244
15.4000000953674 3.03519843673356
15.4400000572205 3.03466744154129
15.4800000190735 3.03419394789522
15.5199999809265 3.03361529292566
15.5600001811981 3.03314342764902
15.6000001430511 3.03218547747298
15.6400001049042 3.03154820456549
15.6800000667572 3.03083075783232
15.7200000286102 3.03019726795694
15.7599999904633 3.02973953767429
15.8000001907349 3.02993220761341
15.8400001525879 3.03059178639963
15.8800001144409 3.032015592186
15.9200000762939 3.03396743956902
15.960000038147 3.0362630955797
16 3.03912061617393
16.0400002002716 3.0419153660061
16.0800001621246 3.04472339185263
16.1200001239777 3.04807617225551
16.1600000858307 3.05143227117412
16.2000000476837 3.05530167094487
16.2400000095367 3.05923011438077
16.2800002098083 3.06321889489764
16.3200001716614 3.06716322145259
16.3600001335144 3.07085821029738
16.4000000953674 3.07493434208533
16.4400000572205 3.07912470784107
16.4800000190735 3.08352294577925
16.5199999809265 3.08816542401901
16.5600001811981 3.0928191337742
16.6000001430511 3.09779896709111
16.6400001049042 3.10257366390395
16.6800000667572 3.10763294493057
16.7200000286102 3.11255837134667
16.7599999904633 3.11724023618146
16.8000001907349 3.12234074492009
16.8400001525879 3.12758011258413
16.8800001144409 3.13321495302348
16.9200000762939 3.13822630652863
16.960000038147 3.13753400168899
17 3.13608046020745
17.0400002002716 3.13498617028145
17.0800001621246 3.13327969580245
17.1200001239777 3.12992708185703
17.1600000858307 3.12667921847227
17.2000000476837 3.12170348773769
17.2400000095367 3.11752597903024
17.2800002098083 3.11183076637403
17.3200001716614 3.10599601329874
17.3600001335144 3.09492581492654
17.4000000953674 3.08307202517713
17.4400000572205 3.07334654246205
17.4800000190735 3.0628729101452
17.5199999809265 3.05207030549858
17.5600001811981 3.04111577457144
17.6000001430511 3.02891162055312
17.6400001049042 3.01603343285441
17.6800000667572 2.99881223059587
17.7200000286102 2.98320854652992
};
\addplot [line width=2.4000000000000004pt, color3, dashed, forget plot]
table {%
0.800000190734863 0.566310207694819
0.840000152587891 0.563927639889975
0.880000114440918 0.561796117788519
0.920000076293945 0.559816497888462
0.960000038146973 0.558455826218455
1 0.55735902121266
1.04000020027161 0.5561912176962
1.08000016212463 0.555047718635843
1.12000012397766 0.553963948650066
1.16000008583069 0.552974560757577
1.20000004768372 0.552046741790935
1.24000000953674 0.551300958894582
1.28000020980835 0.550669526904506
1.32000017166138 0.550308407543052
1.3600001335144 0.550073972175936
1.40000009536743 0.549825548675766
1.44000005722046 0.549976604453615
1.48000001907349 0.550291202524645
1.51999998092651 0.551174199039641
1.56000018119812 0.552213365999857
1.60000014305115 0.553119875124355
1.64000010490417 0.555421402524303
1.6800000667572 0.558085625348781
1.72000002861023 0.560928387746607
1.75999999046326 0.563517827411323
1.80000019073486 0.565803592330656
1.84000015258789 0.56857031689076
1.88000011444092 0.571311548260288
1.92000007629395 0.573922116131465
1.96000003814697 0.576203174049553
2 0.578210168906244
2.04000020027161 0.580441011189827
2.08000016212463 0.58258292760327
2.12000012397766 0.584509392789259
2.16000008583069 0.58614277932394
2.20000004768372 0.587590608060477
2.24000000953674 0.589372418690818
2.28000020980835 0.591189785828517
2.32000017166138 0.592976567497303
2.3600001335144 0.594574274498105
2.40000009536743 0.596451719711576
2.44000005722046 0.598424911872467
2.48000001907349 0.600176379974072
2.51999998092651 0.602614228667679
2.56000018119812 0.605198639288983
2.60000014305115 0.607483714014613
2.64000010490417 0.61004398406267
2.6800000667572 0.612586071747246
2.72000002861023 0.615390447709412
2.75999999046326 0.618080760959994
2.80000019073486 0.620477727421334
2.84000015258789 0.62335556101783
2.88000011444092 0.626308480934312
2.92000007629395 0.629390429482084
2.96000003814697 0.632241770174151
3 0.635025752603668
3.04000020027161 0.637634633072934
3.08000016212463 0.639962776129363
3.12000012397766 0.64257698542947
3.16000008583069 0.64512794332295
3.20000004768372 0.647476266375938
3.24000000953674 0.648171706615468
3.28000020980835 0.64577296595424
3.32000017166138 0.643053943998507
3.3600001335144 0.640448077065771
3.40000009536743 0.637221935169032
3.44000005722046 0.633783597867729
3.48000001907349 0.642527934220749
3.51999998092651 0.627320725833145
3.56000018119812 0.624174985253463
3.60000014305115 0.621313574841812
3.64000010490417 0.617877560508242
3.6800000667572 0.615371452561711
3.72000002861023 0.612625725065143
3.75999999046326 0.610173511047823
3.80000019073486 0.607946485990773
3.84000015258789 0.605893362811901
3.88000011444092 0.604013760323175
3.92000007629395 0.602373566845452
3.96000003814697 0.600973634462917
4 0.600188671411631
4.04000020027161 0.59970394005534
4.08000016212463 0.599228903078457
4.12000012397766 0.598831139438943
4.16000008583069 0.598541745296617
4.20000004768372 0.598249106123532
4.24000000953674 0.597948268763388
4.28000020980835 0.59764188385073
4.32000017166138 0.597327904309418
4.3600001335144 0.596987576749144
4.40000009536743 0.596409961306188
4.44000005722046 0.606166231996826
4.48000001907349 0.595068531684172
4.51999998092651 0.594343320737749
4.56000018119812 0.593587941591178
4.60000014305115 0.592869468656429
4.64000010490417 0.59218809522451
4.6800000667572 0.591626795716499
4.72000002861023 0.591343922157773
4.75999999046326 0.591114671561654
4.80000019073486 0.591134754182821
4.84000015258789 0.591145408776321
4.88000011444092 0.591154660626415
4.92000007629395 0.591503304841141
4.96000003814697 0.591967432850562
5 0.592499595158644
5.04000020027161 0.592947103433576
5.08000016212463 0.593305597314591
5.12000012397766 0.593595548885434
5.16000008583069 0.593821574724342
5.20000004768372 0.593913211709266
5.24000000953674 0.593691749238427
5.28000020980835 0.593477054131152
5.32000017166138 0.593394403190575
5.3600001335144 0.593327527360158
5.40000009536743 0.593840110431079
5.44000005722046 0.59459164778121
5.48000001907349 0.595515612839712
5.51999998092651 0.596378196672885
5.56000018119812 0.597096739434742
5.60000014305115 0.597823665234832
5.64000010490417 0.598610789939798
5.6800000667572 0.599456865425409
5.72000002861023 0.600428552042877
5.75999999046326 0.601299985436871
5.80000019073486 0.602078811493469
5.84000015258789 0.602727182327641
5.88000011444092 0.602892592256309
5.92000007629395 0.602907829701462
5.96000003814697 0.602866188780546
6 0.602325519720073
6.04000020027161 0.601546841369692
6.08000016212463 0.600386111967129
6.12000012397766 0.598825988362157
6.16000008583069 0.597367732303788
6.20000004768372 0.596004337187626
6.24000000953674 0.594674225766672
6.28000020980835 0.593157267294683
6.32000017166138 0.591736316424847
6.3600001335144 0.590402324988954
6.40000009536743 0.589022328013016
6.44000005722046 0.587626657555694
6.48000001907349 0.586175773125995
6.51999998092651 0.584826586911772
6.56000018119812 0.583489299818304
6.60000014305115 0.582167768425982
6.64000010490417 0.580476765968322
6.6800000667572 0.578091325369254
6.72000002861023 0.575906923687721
6.75999999046326 0.573855054679978
6.80000019073486 0.571639210418919
6.84000015258789 0.569580546869564
6.88000011444092 0.567265584069042
6.92000007629395 0.564968017658308
6.96000003814697 0.562647161015823
7 0.560423698787181
7.04000020027161 0.558483693857275
7.08000016212463 0.557123580720461
7.12000012397766 0.55616361217066
7.16000008583069 0.555449069213853
7.20000004768372 0.555107124347508
7.24000000953674 0.554947131908628
7.28000020980835 0.555236389798359
7.32000017166138 0.555537620148697
7.3600001335144 0.555918325890525
7.40000009536743 0.556259925475129
7.44000005722046 0.556533067134955
7.48000001907349 0.556385018094637
7.51999998092651 0.556044550976871
7.56000018119812 0.555832615265989
7.60000014305115 0.555653225499881
7.64000010490417 0.555653844596892
7.6800000667572 0.555775082111367
7.72000002861023 0.555851830532663
7.75999999046326 0.555924363184304
7.80000019073486 0.556061016840202
7.84000015258789 0.556316127870613
7.88000011444092 0.557124024163776
7.92000007629395 0.557977181207488
7.96000003814697 0.559136770773713
8 0.560371372063851
8.04000020027161 0.561624374932784
8.08000016212463 0.562666130475653
8.12000012397766 0.563554681962578
8.16000008583069 0.564422862063484
8.20000004768372 0.565264166437431
8.24000000953674 0.566108116788004
8.28000020980835 0.56709000605625
8.32000017166138 0.567913401717255
8.3600001335144 0.568647591019555
8.40000009536743 0.569298379019357
8.44000005722046 0.570067098119456
8.48000001907349 0.570915933209738
8.51999998092651 0.571727808055983
8.56000018119812 0.572919571507989
8.60000014305115 0.57409204428986
8.64000010490417 0.575297413501965
8.6800000667572 0.576519642070727
8.72000002861023 0.577377340875438
8.75999999046326 0.577822243492902
8.80000019073486 0.578181155308282
8.84000015258789 0.578392997269129
8.88000011444092 0.578313875186503
8.92000007629395 0.578162838169254
8.96000003814697 0.577683005422596
9 0.577162892047358
9.04000020027161 0.576110583427332
9.08000016212463 0.574889138830413
9.12000012397766 0.573790550508514
9.16000008583069 0.572776795644191
9.20000004768372 0.57182208536616
9.24000000953674 0.570855994111366
9.28000020980835 0.569824442798667
9.32000017166138 0.568593387227091
9.3600001335144 0.566729314620851
9.40000009536743 0.564893159455378
9.44000005722046 0.56248544762589
9.48000001907349 0.560097396475496
9.51999998092651 0.557205547166552
9.56000018119812 0.554323939860843
9.60000014305115 0.551834118657573
9.64000010490417 0.549279152598448
9.6800000667572 0.546673489480972
9.72000002861023 0.544063566051141
9.75999999046326 0.54133377676557
9.80000019073486 0.53880774691803
9.84000015258789 0.536502553116121
9.88000011444092 0.534387519592495
9.92000007629395 0.532873622664875
9.96000003814697 0.53153267112004
10 0.530266255499196
10.0400002002716 0.546925549187508
10.0800001621246 0.528198039854712
10.1200001239777 0.527649981150819
10.1600000858307 0.527322600214427
10.2000000476837 0.526991248162036
10.2400000095367 0.526655869480012
10.2800002098083 0.526219203610849
10.3200001716614 0.525535906794667
10.3600001335144 0.524873446159609
10.4000000953674 0.524232210708219
10.4400000572205 0.52379603146589
10.4800000190735 0.523388890985208
10.5199999809265 0.523228539236435
10.5600001811981 0.522074161923519
10.6000001430511 0.523476872174834
10.6400001049042 0.523843772231425
10.6800000667572 0.524329791026581
10.7200000286102 0.524862646749479
10.7599999904633 0.525307537003112
10.8000001907349 0.525795678861039
10.8400001525879 0.526465323014089
10.8800001144409 0.5270822559427
10.9200000762939 0.527743459111048
10.960000038147 0.528342234104429
11 0.529106323462411
11.0400002002716 0.529943259380987
11.0800001621246 0.530677554474344
11.1200001239777 0.531353981030273
11.1600000858307 0.531978471659719
11.2000000476837 0.556058809265848
11.2400000095367 0.532767539329965
11.2800002098083 0.533289424039148
11.3200001716614 0.533420410755758
11.3600001335144 0.53343390396562
11.4000000953674 0.533166489564569
11.4400000572205 0.53282348143712
11.4800000190735 0.532222288679252
11.5199999809265 0.531556167825498
11.5600001811981 0.530914673708877
11.6000001430511 0.530420060561538
11.6400001049042 0.53000436711532
11.6800000667572 0.529491574670342
11.7200000286102 0.528915624034234
11.7599999904633 0.528229023271985
11.8000001907349 0.527339178792165
11.8400001525879 0.526501261177906
11.8800001144409 0.525722864105031
11.9200000762939 0.525131697926006
11.960000038147 0.524588892477429
12 0.524180265420195
12.0400002002716 0.523815647573118
12.0800001621246 0.523450756560656
12.1200001239777 0.52308537792826
12.1600000858307 0.522707699238823
12.2000000476837 0.521878218714596
12.2400000095367 0.52083772955509
12.2800002098083 0.519775765035912
12.3200001716614 0.518685183826015
12.3600001335144 0.517508660186145
12.4000000953674 0.516022477573725
12.4400000572205 0.514687972903787
12.4800000190735 0.513180820974952
12.5199999809265 0.511621727857366
12.5600001811981 0.50984081295053
12.6000001430511 0.508042261108313
12.6400001049042 0.506485160708548
12.6800000667572 0.505151476398292
12.7200000286102 0.50398560151616
12.7599999904633 0.502900094139036
12.8000001907349 0.501858226938522
12.8400001525879 0.500699467857341
12.8800001144409 0.499148647825328
12.9200000762939 0.497619793582047
12.960000038147 0.495909619084141
13 0.494268943633728
13.0400002002716 0.492741828244786
13.0800001621246 0.49165753844967
13.1200001239777 0.490792516858827
13.1600000858307 0.490023735760005
13.2000000476837 0.489298967132004
13.2400000095367 0.48868297660046
13.2800002098083 0.488183034697325
13.3200001716614 0.487695473764414
13.3600001335144 0.487182328969994
13.4000000953674 0.48654488467549
13.4400000572205 0.486022555102209
13.4800000190735 0.486017108516137
13.5199999809265 0.486094350506732
13.5600001811981 0.486137686621905
13.6000001430511 0.486160373873264
13.6400001049042 0.486690007700896
13.6800000667572 0.487292165410659
13.7200000286102 0.487780886075753
13.7599999904633 0.488203604079186
13.8000001907349 0.488562627630611
13.8400001525879 0.488902699141377
13.8800001144409 0.489298639127612
13.9200000762939 0.489870930744165
13.960000038147 0.491682206415451
14 0.493587081169375
14.0400002002716 0.496592802324463
14.0800001621246 0.49954088333878
14.1200001239777 0.502411334710166
14.1600000858307 0.506105119221799
14.2000000476837 0.509709319967081
14.2400000095367 0.513127933168859
14.2800002098083 0.516333719423099
14.3200001716614 0.518936026250591
14.3600001335144 0.520995788345029
14.4000000953674 0.522952184863815
14.4400000572205 0.524931602565191
14.4800000190735 0.526757716239426
14.5199999809265 0.528805257840063
14.5600001811981 0.530696373103217
14.6000001430511 0.532390414721763
14.6400001049042 0.533616161955578
14.6800000667572 0.534678589127438
14.7200000286102 0.535406223392801
14.7599999904633 0.53584347725303
14.8000001907349 0.535943438390238
14.8400001525879 0.535866306529692
14.8800001144409 0.535504461360984
14.9200000762939 0.534864337103298
14.960000038147 0.534244244119427
15 0.53358052270533
15.0400002002716 0.532662227247694
15.0800001621246 0.531738019324988
15.1200001239777 0.530434322229276
15.1600000858307 0.529142164816211
15.2000000476837 0.527663568018939
15.2400000095367 0.526138424522874
15.2800002098083 0.523246916825231
15.3200001716614 0.520113195943936
15.3600001335144 0.517287435514134
15.4000000953674 0.514341723696098
15.4400000572205 0.51126924707468
15.4800000190735 0.508254271999453
15.5199999809265 0.505134135600738
15.5600001811981 0.502120773746518
15.6000001430511 0.49862134214132
15.6400001049042 0.495442587804676
15.6800000667572 0.492183659642341
15.7200000286102 0.489008688337813
15.7599999904633 0.486009476626005
15.8000001907349 0.483660649987539
15.8400001525879 0.481778747344605
15.8800001144409 0.48066107170182
15.9200000762939 0.480071437655686
15.960000038147 0.479825612237212
16 0.480141651402282
16.0400002002716 0.480394904656874
16.0800001621246 0.480661449074243
16.1200001239777 0.481472748047967
16.1600000858307 0.482287365537429
16.2000000476837 0.483615283879024
16.2400000095367 0.485002245885759
16.2800002098083 0.486449529825051
16.3200001716614 0.487852374950846
16.3600001335144 0.489005882366482
16.4000000953674 0.490540532725273
16.4400000572205 0.492189417051856
16.4800000190735 0.494046173560887
16.5199999809265 0.496147170371491
16.5600001811981 0.498259383549096
16.6000001430511 0.500697735436852
16.6400001049042 0.502930950820535
16.6800000667572 0.505448750417994
16.7200000286102 0.507832695404942
16.7599999904633 0.509973078810582
16.8000001907349 0.512532090971624
16.8400001525879 0.515229977206515
16.8800001144409 0.518323336216702
16.9200000762939 0.520793208292698
16.960000038147 0.517559422023902
17 0.513564399113204
17.0400002002716 0.509928612609625
17.0800001621246 0.505680656701469
17.1200001239777 0.499786561326895
17.1600000858307 0.493997216512978
17.2000000476837 0.486480004349246
17.2400000095367 0.479761014212639
17.2800002098083 0.471524304978844
17.3200001716614 0.463148070474397
17.3600001335144 0.449536390673046
17.4000000953674 0.435141119494483
17.4400000572205 0.422874155350239
17.4800000190735 0.409859041604239
17.5199999809265 0.396514955528463
17.5600001811981 0.383018928023737
17.6000001430511 0.368273292576262
17.6400001049042 0.352853623448394
17.6800000667572 0.333090939760707
17.7200000286102 0.314945774265597
17.7599999904633 0.292931715366997
17.8000001907349 0.272382267142471
17.8400001525879 0.249909796719576
17.8800001144409 0.227007533258429
17.9200000762939 0.205556134603106
17.960000038147 0.183123015003599
18 0.161866119933623
18.0400002002716 0.136477825554885
18.0800001621246 0.10859364002154
18.1200001239777 0.0820231452406671
18.1600000858307 0.0551840037930995
18.2000000476837 0.0281780294691293
18.2400000095367 -0.00166156217087476
18.2800002098083 -0.0294411269390502
18.3200001716614 -0.0647262780977721
18.3600001335144 -0.100047232038578
18.4000000953674 -0.133356194565386
18.4400000572205 -0.168643587611325
18.4800000190735 -0.20022827116965
18.5199999809265 -0.231791290968456
18.5600001811981 -0.263616250127143
18.6000001430511 -0.295076251382716
18.6400001049042 -0.32779320112527
18.6800000667572 -0.360207325313561
18.7200000286102 -0.40085453669958
18.7599999904633 -0.44070215252467
18.8000001907349 -0.481750767354482
18.8400001525879 -0.520361587622894
18.8800001144409 -0.555477954414314
18.9200000762939 -0.594487755733066
18.960000038147 -0.63305777529849
19 -0.676109740801417
19.0400002002716 -0.717576484920324
19.0800001621246 -0.757689262558619
19.1200001239777 -0.796505524920395
19.1600000858307 -0.834101763723671
19.2000000476837 -0.872172195942563
19.2400000095367 -0.913016155435836
19.2800002098083 -0.968049491227328
19.3200001716614 -1.01991097991935
19.3600001335144 -1.06765599933218
19.4000000953674 -1.1180306563882
19.4400000572205 -1.16629438798686
19.4800000190735 -1.21530488378435
19.5199999809265 -1.26282637580694
19.5600001811981 -1.30827650525853
19.6000001430511 -1.3483711088362
19.6400001049042 -1.3877826525578
19.6800000667572 -1.42670953515843
19.7200000286102 -1.46196574961523
19.7599999904633 -1.49793501433718
19.8000001907349 -1.54051604284473
19.8400001525879 -1.5799879611256
19.8800001144409 -1.62026129007031
19.9200000762939 -1.65852213254711
19.960000038147 -1.69807372700624
20 -1.73579085751822
20.0400002002716 -1.78144333065173
20.0800001621246 -1.82696101900612
20.1200001239777 -1.86554507952709
20.1600000858307 -1.90319968209045
20.2000000476837 -1.93972804944744
20.2400000095367 -1.97513656869397
20.2800002098083 -2.01177486442198
20.3200001716614 -2.04732632861467
20.3600001335144 -2.08936807891758
20.4000000953674 -2.12920894768779
20.4400000572205 -2.17014513101829
20.4800000190735 -2.20857703921015
20.5199999809265 -2.24874836858047
20.5600001811981 -2.28833880398407
20.6000001430511 -2.32398723712702
20.6400001049042 -2.35779262868662
20.6800000667572 -2.39064231061415
20.7200000286102 -2.42490908481529
20.7599999904633 -2.46052811872392
20.8000001907349 -2.49493405280961
20.8400001525879 -2.52879016170607
20.8800001144409 -2.56051054590224
20.9200000762939 -2.59280653394196
20.960000038147 -2.6222437538717
21 -2.6493792230323
21.0400002002716 -2.68040269635519
21.0800001621246 -2.71216477842703
21.1200001239777 -2.7442752944313
21.1600000858307 -2.77340037722636
21.2000000476837 -2.80056889123675
21.2400000095367 -2.82618478350346
21.2800002098083 -2.84921539358482
21.3200001716614 -2.87201483468714
21.3600001335144 -2.89643900154388
21.4000000953674 -2.91935013798375
21.4400000572205 -2.94074032305327
21.4800000190735 -2.96085979618567
21.5199999809265 -2.98073681292527
21.5600001811981 -2.99899311492534
21.6000001430511 -3.01748447248947
21.6400001049042 -3.03606762567086
21.6800000667572 -3.05296805890715
21.7200000286102 -3.06780476079598
21.7599999904633 -3.0814757779189
21.8000001907349 -3.09411753194364
21.8400001525879 -3.10574577956912
21.8800001144409 -3.11647873874857
21.9200000762939 -3.12586667343871
21.960000038147 -3.1343722840652
22 -3.14219651637647
22.0400002002716 -3.14867125157525
22.0800001621246 -3.15464672327943
22.1200001239777 -3.16005534907116
22.1600000858307 -3.16506436122899
22.2000000476837 -3.16960233300277
22.2400000095367 -3.17346310079766
22.2800002098083 -3.17698262871556
22.3200001716614 -3.18022028826922
22.3600001335144 -3.18318574715157
22.4000000953674 -3.18569926685406
22.4400000572205 -3.18734445343163
22.4800000190735 -3.18889711749647
22.5199999809265 -3.19029333838841
22.5600001811981 -3.19115479067629
22.6000001430511 -3.19191818456762
22.6400001049042 -3.19096453474222
22.6800000667572 -3.18954611464741
22.7200000286102 -3.18824063325985
22.7599999904633 -3.18625309557559
22.8000001907349 -3.18421658952843
22.8400001525879 -3.18178541444866
22.8800001144409 -3.17917107832383
22.9200000762939 -3.17612119904749
22.960000038147 -3.17219201616058
23 -3.16857081015958
23.0400002002716 -3.16515725619533
23.0800001621246 -3.16115810188355
23.1200001239777 -3.15739879649366
23.1600000858307 -3.1527030006046
23.2000000476837 -3.14811909138904
23.2400000095367 -3.14373820703735
23.2800002098083 -3.13677307923692
23.3200001716614 -3.12980436707365
23.3600001335144 -3.1230111220286
23.4000000953674 -3.11638853139072
23.4400000572205 -3.11044626452857
23.4800000190735 -3.10465819940002
23.5199999809265 -3.09872369727516
23.5600001811981 -3.09313368842693
23.6000001430511 -3.08761979197185
23.6400001049042 -3.08266926256685
23.6800000667572 -3.07768587405984
23.7200000286102 -3.07279755586332
23.7599999904633 -3.06819058357103
23.8000001907349 -3.06385226643817
23.8400001525879 -3.06017879579008
23.8800001144409 -3.05650434515481
23.9200000762939 -3.05283166234428
23.960000038147 -3.04953571408226
24 -3.04625645159661
24.0400002002716 -3.04241159189465
24.0800001621246 -3.03884764818621
24.1200001239777 -3.03541160452394
24.1600000858307 -3.03227214109481
24.2000000476837 -3.02930256644047
24.2400000095367 -3.02656384632114
24.2800002098083 -3.02398035216542
24.3200001716614 -3.02168990780746
24.3600001335144 -3.01999081570264
24.4000000953674 -3.01861795719642
24.4400000572205 -3.01744409467171
24.4800000190735 -3.01725621068551
24.5199999809265 -3.01859008113163
24.5600001811981 -3.01993128343023
24.6000001430511 -3.02102215142922
24.6400001049042 -3.02229521646069
24.6800000667572 -3.0235127758207
24.7200000286102 -3.0250542217152
24.7599999904633 -3.02663353472275
24.8000001907349 -3.02813112776014
24.8400001525879 -3.03131457513377
24.8800001144409 -3.03476407440966
24.9200000762939 -3.03863570358227
24.960000038147 -3.04284474806492
25 -3.04722629528103
25.0400002002716 -3.05212834414081
25.0800001621246 -3.05671280009892
25.1200001239777 -3.06101180433907
25.1600000858307 -3.06494335731499
25.2000000476837 -3.06861409720777
25.2400000095367 -3.07206865959385
25.2800002098083 -3.07538054795567
25.3200001716614 -3.07922833770367
25.3600001335144 -3.0830691049182
25.4000000953674 -3.08645949521743
25.4400000572205 -3.08964065330216
25.4800000190735 -3.09268463359324
25.5199999809265 -3.09573090250841
25.5600001811981 -3.09879765826611
25.6000001430511 -3.10155757056799
25.6400001049042 -3.10380737601632
25.6800000667572 -3.10588661248871
25.7200000286102 -3.1076445877
25.7599999904633 -3.10926472539818
25.8000001907349 -3.1107860064564
25.8400001525879 -3.11151635221304
25.8800001144409 -3.11193606990889
25.9200000762939 -3.11123573900057
25.960000038147 -3.11012607009645
26 -3.10921695826847
26.0400002002716 -3.10813323274568
26.0800001621246 -3.10695589521532
26.1200001239777 -3.10585963500949
26.1600000858307 -3.1047741822136
26.2000000476837 -3.10367425916209
26.2400000095367 -3.10199112543039
26.2800002098083 -3.10040470990576
26.3200001716614 -3.09842523185473
26.3600001335144 -3.09660275143389
26.4000000953674 -3.09491983696476
26.4400000572205 -3.09210404374657
26.4800000190735 -3.08904975209152
26.5199999809265 -3.08598222497896
26.5600001811981 -3.08295351500312
26.6000001430511 -3.07989391487829
26.6400001049042 -3.0768073492051
26.6800000667572 -3.07398176848043
26.7200000286102 -3.07142041446374
26.7599999904633 -3.06884085519035
26.8000001907349 -3.0664312114736
26.8400001525879 -3.06443046921649
26.8800001144409 -3.06259046814647
26.9200000762939 -3.06088428597908
26.960000038147 -3.05927089701911
27 -3.057863201863
27.0400002002716 -3.05614577467182
27.0800001621246 -3.05434368041142
27.1200001239777 -3.05255039938368
27.1600000858307 -3.05076923499717
27.2000000476837 -3.04912681307888
27.2400000095367 -3.04752941532641
27.2800002098083 -3.0461495349667
27.3200001716614 -3.04502254463527
27.3600001335144 -3.04519891389882
27.4000000953674 -3.04554464894153
27.4400000572205 -3.0467402093181
27.4800000190735 -3.04819348534165
27.5199999809265 -3.05020248015308
27.5600001811981 -3.05228497304395
27.6000001430511 -3.05413729221519
27.6400001049042 -3.05603369881925
27.6800000667572 -3.05795171264271
27.7200000286102 -3.05980518403021
27.7599999904633 -3.0616272723732
27.8000001907349 -3.06332656566938
27.8400001525879 -3.06506498571582
27.8800001144409 -3.06670845008976
27.9200000762939 -3.06819324484233
27.960000038147 -3.07142120880214
28 -3.0747425711339
28.0400002002716 -3.07892630048381
28.0800001621246 -3.08315042161306
28.1200001239777 -3.08676719839462
28.1600000858307 -3.09083348408661
28.2000000476837 -3.09518268124913
28.2400000095367 -3.09929184238516
28.2800002098083 -3.10331192038176
28.3200001716614 -3.10690328534861
28.3600001335144 -3.11011354009615
28.4000000953674 -3.1131590884315
28.4400000572205 -3.11524478376099
28.4800000190735 -3.11721636773006
28.5199999809265 -3.11916297325803
28.5600001811981 -3.12098541192642
28.6000001430511 -3.12266619658216
28.6400001049042 -3.12413676233627
28.6800000667572 -3.1254747602067
28.7200000286102 -3.12691185707381
28.7599999904633 -3.12846578872467
28.8000001907349 -3.12989501329388
28.8400001525879 -3.13126186887142
28.8800001144409 -3.13267266867584
28.9200000762939 -3.13395580594061
28.960000038147 -3.13471372949924
29 -3.13538757328855
29.0400002002716 -3.13490932103054
29.0800001621246 -3.1341252739791
29.1200001239777 -3.13341849006276
29.1600000858307 -3.13232679834441
29.2000000476837 -3.13122018636376
29.2400000095367 -3.12997079926449
29.2800002098083 -3.12863161921327
29.3200001716614 -3.12766667516988
29.3600001335144 -3.12713027711098
29.4000000953674 -3.1266910232181
29.4400000572205 -3.12629975284805
29.4800000190735 -3.12583140628429
29.5199999809265 -3.12544686715797
29.5600001811981 -3.12390693799057
29.6000001430511 -3.12227947671975
29.6400001049042 -3.11978546323652
29.6800000667572 -3.11714347716993
29.7200000286102 -3.11485986613387
29.7599999904633 -3.11271153060996
29.8000001907349 -3.11064119611469
29.8400001525879 -3.10852454494824
29.8800001144409 -3.10626817302645
29.9200000762939 -3.10426862792158
29.960000038147 -3.102470316615
30 -3.10080797207185
30.0400002002716 -3.09947724489471
30.0800001621246 -3.09852574292678
30.1200001239777 -3.09769334645661
30.1600000858307 -3.09665787860721
30.2000000476837 -3.09564611681417
30.2400000095367 -3.0944041163228
30.2800002098083 -3.09322931651988
30.3200001716614 -3.09221760720683
30.3600001335144 -3.09091134842154
30.4000000953674 -3.08927227027325
30.4400000572205 -3.08761952013195
30.4800000190735 -3.08567962711249
30.5199999809265 -3.08391170246022
30.5600001811981 -3.08215708845827
30.6000001430511 -3.08046221756638
30.6400001049042 -3.07844508410792
30.6800000667572 -3.07642796688965
30.7200000286102 -3.07455139759923
30.7599999904633 -3.07154034670631
30.8000001907349 -3.06842096207526
30.8400001525879 -3.06495137941348
30.8800001144409 -3.06127082902975
30.9200000762939 -3.05795891665044
30.960000038147 -3.05451936904564
31 -3.05032685485218
31.0400002002716 -3.0458846782397
31.0800001621246 -3.03942147273481
31.1200001239777 -3.03337294953614
31.1600000858307 -3.02621415403253
31.2000000476837 -3.01921699007315
31.2400000095367 -3.01257162831714
31.2800002098083 -3.00489113274947
31.3200001716614 -2.99745877256903
31.3600001335144 -2.98760299835945
31.4000000953674 -2.97698141416926
31.4400000572205 -2.96584412151992
31.4800000190735 -2.95404659434919
31.5199999809265 -2.94226596984242
31.5600001811981 -2.92936315032436
31.6000001430511 -2.91695392775296
31.6400001049042 -2.89905599374622
31.6800000667572 -2.88168217302807
31.7200000286102 -2.86509627952803
31.7599999904633 -2.84596053977222
31.8000001907349 -2.82661375479393
31.8400001525879 -2.80698480522511
31.8800001144409 -2.78839463494095
31.9200000762939 -2.76614228427842
31.960000038147 -2.74142163191376
32 -2.71894206978041
32.0400002002716 -2.69695768531981
32.0800001621246 -2.67326320789268
32.1200001239777 -2.65080879597698
32.1600000858307 -2.62572252576284
32.2000000476837 -2.6020264114522
32.2400000095367 -2.56920242720012
32.2800002098083 -2.53735767543013
32.3200001716614 -2.50207906979168
32.3600001335144 -2.46802778932719
32.4000000953674 -2.43880097392883
32.4400000572205 -2.40988958326224
32.4800000190735 -2.38184194627937
32.5199999809265 -2.35352341977618
32.5600001811981 -2.32471803490088
32.6000001430511 -2.29536513174775
32.6400001049042 -2.260115290012
32.6800000667572 -2.23197929838065
32.7200000286102 -2.19604967731882
32.7599999904633 -2.15758113755654
32.8000001907349 -2.12269741627009
32.8400001525879 -2.08866960478418
32.8800001144409 -2.05553259784881
32.9200000762939 -2.01525945631235
32.960000038147 -1.97671398049917
33 -1.93924104769494
33.0400002002716 -1.90277044243617
33.0800001621246 -1.86712145972105
33.1200001239777 -1.83278821936665
33.1600000858307 -1.79745270023536
33.2000000476837 -1.76215683394709
33.2400000095367 -1.7274161989365
33.2800002098083 -1.68559335896546
33.3200001716614 -1.64719181793382
33.3600001335144 -1.59536739468671
33.4000000953674 -1.54347700936637
33.4400000572205 -1.49439716594584
33.4800000190735 -1.44719190199054
33.5199999809265 -1.40686581037722
33.5600001811981 -1.3645624196192
33.6000001430511 -1.32289170731313
33.6400001049042 -1.28580426174661
33.6800000667572 -1.24778905473918
33.7200000286102 -1.20982473069319
33.7599999904633 -1.17062406807163
33.8000001907349 -1.1339732081369
33.8400001525879 -1.09528476817552
33.8800001144409 -1.05709751020293
33.9200000762939 -1.02131671136402
33.960000038147 -0.951625583804719
34 -0.943213589384272
34.0400002002716 -0.905293540085858
34.0800001621246 -0.869052156940205
34.1200001239777 -0.834685359899977
34.1600000858307 -0.800359213922397
};
\node at (axis cs:13.5,-2)[
  anchor=north,
  text=black,
  rotate=0.0
]{ Pickup truck};
\node at (axis cs:17,2.2)[
  anchor=north,
  text=black,
  rotate=0.0
]{ Station wagon};
\end{axis}

\end{tikzpicture}

% Define main parts
\tikzstyle{block distance}=[node distance=5.5em]
\tikzstyle{block}=[draw, text width=4.5em, align=center, block distance, minimum height=2.8em]
\tikzstyle{dashed arrow}=[->, dashed, ultra thick]

% Define all kinds of distances
\tikzstyle{in between}=[coordinate, node distance=2.75em]
\tikzstyle{larger distance}=[node distance=6.5em]
\tikzstyle{environment text}=[coordinate, node distance=3.5em]
\tikzstyle{environment text b}=[node distance=0.2em, minimum width=6.5em, align=right]
\tikzstyle{entrance environment}=[coordinate, node distance=4.8em]
\tikzstyle{helper arrow}=[coordinate, node distance=2em]
\tikzstyle{top env box hwidth}=[coordinate, node distance=5.75em]
\tikzstyle{length dashed arrow}=[coordinate, node distance=2.5em]
\tikzstyle{distance heading}=[node distance=3em]
\tikzstyle{entrance states}=[coordinate, node distance=3.5em]
\tikzstyle{state below}=[coordinate, node distance=2em]
\tikzstyle{distance activities}=[node distance=6em]
\tikzstyle{entrance activities}=[coordinate, node distance=3.5em]
\tikzstyle{distance activities 2}=[node distance=5em]
\tikzstyle{margin dyn box}=[coordinate, node distance=0.3em]

% Define all the colors
\tikzstyle{color scen}=[fill=yellow!20]
\tikzstyle{color ego}=[fill=red!20]
\tikzstyle{color dyn}=[fill={rgb:red,5;green,10;yellow,1;black,5}]
\tikzstyle{color heading}=[fill={rgb:red,5;green,10;yellow,1}]
\tikzstyle{color heading activities}=[fill={rgb:red,5;green,10;yellow,10}]
\tikzstyle{color heading activities 2}=[fill={rgb:red,5;green,10;yellow,20}]
\tikzstyle{color speed}=[fill={rgb:red,5;green,10;white,1}]
\tikzstyle{color speed activities}=[fill={rgb:red,5;green,10;white,5}]
\tikzstyle{color speed activities 2}=[fill={rgb:red,5;green,10;white,10}]
\tikzstyle{color stat}=[fill=blue!50]
\tikzstyle{line color dyn}=[black!40!green!80]
\tikzstyle{line color stat}=[black!40!blue!80]

\hspace{-10.3em} % For some strange reason, there is some white space on left side
\begin{tikzpicture}
	\node[color scen, block](scenario){Scenario};
	\node[coordinate, below of=scenario, node distance=2em](helper scenario){};
	
	% Create dynamic environment
	\node[coordinate, below of=scenario, node distance=7.5em](below scenario){};
	\node[color dyn, block, left of=below scenario, node distance=0em](otherdyn){Other road user};
	\node[color dyn, block, left of=otherdyn](target){Target vehicle};
	\node[above of=otherdyn, environment text](helper1){};
	\node[right of=helper1, environment text b]{Dynamic\\environment};
	\node[left of=otherdyn, in between](helper2){};
	\node[entrance environment, above of=helper2](entrance dyn){};
	\node[helper arrow, above of=helper2](helper dyn){};
	\draw[ultra thick] (scenario) -- (helper scenario) -| (entrance dyn);
	\draw[->] (entrance dyn) -- (helper dyn) -| (target);
	\draw[->] (entrance dyn) -- (helper dyn) -| (otherdyn);
	
	% Ego vehicle
	\node[color ego, block, larger distance, left of=target](ego){Ego vehicle};
	\draw[ultra thick, ->] (scenario) --(helper scenario) -| (ego);
	
	% Static environment
	\node[color stat, block, right of=otherdyn, larger distance](road){Road};
	\node[color stat, block, right of=road](weather){Weather};
	\node[above of=weather, environment text](helper3){};
	\node[right of=helper3, environment text b]{Static\\environment};
	\node[left of=weather, in between](helper4){};
	\node[entrance environment, above of=helper4](entrance stat){};
	\node[helper arrow, above of=helper4](helper stat){};
	\draw[ultra thick] (scenario) -- (helper scenario) -| (entrance stat);
	\draw[->] (entrance stat) -- (helper stat) - | (road);
	\draw[->] (entrance stat) -- (helper stat) - | (weather);
	\node[length dashed arrow, below of=road](road arrow){};
	\draw[dashed arrow] (road) -- (road arrow);
	\node[length dashed arrow, below of=weather](weather arrow){};
	\draw[dashed arrow] (weather) -- (weather arrow);
	
	% States
	\node[coordinate, below of=target, node distance=7.4em](below target){};
	\node[color heading, block, left of=below target, distance heading](heading){Heading};
	\node[color speed, block, right of=below target, node distance=14em](speed){Speed};
	\node[coordinate, above of=speed, node distance=2.8em](helper10){};
	\node[right of=helper10, node distance=1.4em, minimum width=4em, align=right]{States};
	\node[above of=heading, entrance states](helper17){};
	\node[coordinate, right of=helper17, distance heading](entrance states){};
	\node[coordinate, below of=target, node distance=5.4em](helper states){};
	\draw[ultra thick] (target) -- (entrance states);
	\draw[->] (entrance states) -- (helper states) -| (heading);
	\draw[->] (entrance states) -- (helper states) -| (speed);
	
	% Activities for heading
	\node[color heading activities, block, below of=heading, distance activities](lane following){Lane following};
	\node[color heading activities, block, left of=lane following](left){Turn left};
	\node[color heading activities, block, right of=lane following](right){Turn right};
	\node[entrance activities, above of=lane following](entrance heading){};
	\node[helper arrow, above of=lane following](helper heading){};
	\draw[ultra thick] (heading) -- (entrance heading);
	\draw[->] (entrance heading) -- (helper heading) -| (left);
	\draw[->] (entrance heading) -- (lane following);
	\draw[->] (entrance heading) -- (helper heading) -| (right);
	\node[length dashed arrow, below of=lane following](lane following arrow){};
	\node[length dashed arrow, below of=right](right arrow){};
	\draw[dashed arrow] (lane following) -- (lane following arrow);
	\draw[dashed arrow] (right) -- (right arrow);
	
	% Activities for speed
	\node[color speed activities, block, below of=speed, distance activities](cruising){Cruising/ Full stop};
	\node[color speed activities, block, left of=cruising](decelerating){Deceler-ating};
	\node[color speed activities, block, right of=cruising](accelerating){Acceler-ating};
	\node[entrance activities, above of=cruising](entrance speed){};
	\node[helper arrow, above of=cruising](helper speed){};
	\draw[ultra thick] (speed) -- (entrance speed);
	\draw[->] (entrance speed) -- (helper speed) -| (decelerating);
	\draw[->] (entrance speed) -- (cruising);
	\draw[->] (entrance speed) -- (helper speed) -| (accelerating);
	\node[length dashed arrow, below of=cruising](cruising arrow){};
	\node[length dashed arrow, below of=accelerating](accelerating arrow){};
	\draw[dashed arrow] (cruising) -- (cruising arrow);
	\draw[dashed arrow] (accelerating) -- (accelerating arrow);
	
	% Possibilities for turning left
	\node[color heading activities 2, block, below of=left, distance activities 2](crossing){Turn left crossing};
	\node[color heading activities 2, block, right of=crossing](lane change){Lane change};
	\node[color heading activities 2, block, right of=lane change](other left){$\ldots$};
	\node[helper arrow, above of=crossing](helper left){};
	\draw[->] (left) -- (crossing);
	\draw[->] (left) -- (helper left) -| (lane change);
	\draw[->] (left) -- (helper left) -| (other left);
	
	% Possibilities for decelerating
	\node[color speed activities 2, block, below of=decelerating, distance activities 2](release){Release throttle};
	\node[color speed activities 2, block, right of=release](emergency){Emergency braking};
	\node[color speed activities 2, block, right of=emergency](other decelerating){$\ldots$};
	\node[helper arrow, above of=release](helper decelerating){};
	\draw[->] (decelerating) -- (release);
	\draw[->] (decelerating) -- (helper decelerating) -| (emergency);
	\draw[->] (decelerating) -- (helper decelerating) -| (other decelerating);
	
	% Define all the nodes for the dashed boxes %%%%%%%%%%%%%%%%%%%%%%%%%%%%%%%%%%%%%%%%%%%%%
	% Dashed box around static environment
	\node[top env box hwidth, left of=entrance stat](helper5){};
	\node[top env box hwidth, right of=entrance stat](helper6){};
	\node[coordinate, below of=helper4, node distance=3em](helper7){};
	\node[top env box hwidth, left of=helper7](helper8){};
	\node[top env box hwidth, right of=helper7](helper9){};
	
	% Dashed box around states (this is done in a strange manner, but this way
	% we make sure the margins are similar as for the environment blocks)
	\node[below of=speed, state below](helper13){};
	\node[left of=helper13, in between](helper11){};
	\node[right of=helper11, top env box hwidth](helper12){};
	\node[below of=heading, state below](helper16){};
	\node[right of=helper16, in between](helper14){};
	\node[left of=helper14, top env box hwidth, red](helper15){};
	\node[right of=helper17, in between](helper18){};
	\node[left of=helper18, top env box hwidth](helper19){};
	\node[above of=speed, entrance states](helper20){};
	\node[left of=helper20, in between](helper21){};
	\node[right of=helper21, top env box hwidth](helper22){};
	
	% Box around activities and label for activities
	\node[entrance activities, above of=left](helper23){};
	\node[right of=helper23, in between](helper24){};
	\node[left of=helper24, top env box hwidth](helper25){};
	\node[entrance activities, above of=accelerating](helper26){};
	\node[left of=helper26, in between](helper27){};
	\node[right of=helper27, top env box hwidth](helper28){};
	\node[below of=other decelerating, state below](helper29){};
	\node[left of=helper29, in between](helper30){};
	\node[right of=helper30, top env box hwidth](helper31){};
	\node[below of=crossing, state below](helper32){};
	\node[right of=helper32, in between](helper33){};
	\node[left of=helper33, top env box hwidth](helper34){};
	\node[coordinate, above of=accelerating, node distance=2.8em](helper35){};
	\node[right of=helper35, node distance=0.8em, minimum width=4em, align=right]{Activities};
	
	% Box for dynamic environment
	\node[top env box hwidth, left of=entrance dyn](helper36){};
	\node[top env box hwidth, right of=entrance dyn](helper37){};
	\node[margin dyn box, above of=entrance states](helper38){};
	\node[right of=helper38, in between](helper39){};
	\node[left of=helper39, top env box hwidth](helper40){};
	\node[right of=helper39, top env box hwidth](helper41){};
	\node[right of=helper31, margin dyn box](helper42){};
	\node[below of=helper42, margin dyn box](helper43){};
	\node[left of=helper34, margin dyn box](helper44){};
	\node[below of=helper44, margin dyn box](helper45){};
	
	% Draw the boxes
	% Static environment
	\draw[dashed, line color stat] (helper5) -- (helper6) -- (helper9) -- (helper8) -- (helper5);
	\fill[fill=blue!10, on layer=background] (helper5) -- (helper6) -- (helper9) -- (helper8) -- (helper5);
	
	% Dynamic environment
	\draw[dashed, line color dyn] (helper36) -- (helper37) -- (helper41) -| (helper43) -- (helper45) |- (helper40) -- (helper36);
	\fill[fill=green!10, on layer=background] (helper36) -- (helper37) -- (helper41) -| (helper43) -- (helper45) |- (helper40) -- (helper36);
	
	% States
	\draw[dashed, line color dyn] (helper22) -- (helper12) -- (helper15) -- (helper19) -- (helper22);
	\fill[fill=green!20, on layer=background] (helper22) |- (helper12) -- (helper15) -- (helper19) -- (helper22);
	
	% Activities
	\draw[dashed, line color dyn] (helper25) -- (helper28) -- (helper31) -- (helper34) -- (helper25);
	\fill[fill=green!40, on layer=background] (helper25) |- (helper28) -- (helper31) -- (helper34) -- (helper25);
\end{tikzpicture}
	
\end{document}
