\documentclass[10pt,final,a4paper,oneside,onecolumn]{article}

%%==========================================================================
%% Packages
%%==========================================================================
\usepackage[a4paper,left=3.5cm,right=3.5cm,top=3cm,bottom=3cm]{geometry} %% change page layout; remove for IEEE paper format
\usepackage[T1]{fontenc}                        %% output font encoding for international characters (e.g., accented)
\usepackage[cmex10]{amsmath}                    %% math typesetting; consider using the [cmex10] option
\usepackage{amssymb}                            %% special (symbol) fonts for math typesetting
\usepackage{amsthm}                             %% theorem styles
\usepackage{dsfont}                             %% double stroke roman fonts: the real numbers R: $\mathds{R}$
\usepackage{mathrsfs}                           %% formal script fonts: the Laplace transform L: $\mathscr{L}$
\usepackage[pdftex]{graphicx}                   %% graphics control; use dvips for TeXify; use pdftex for PDFTeXify
\usepackage{array}                              %% array functionality (array, tabular)
\usepackage{upgreek}                            %% upright Greek letters; add the prefix 'up', e.g. \upphi
\usepackage{stfloats}                           %% improved handling of floats
\usepackage{multirow}                           %% cells spanning multiple rows in tables
%\usepackage{subfigure}                         %% subfigures and corresponding captions (for use with IEEEconf.cls)
\usepackage{subfig}                             %% subfigures (IEEEtran.cls: set caption=false)
\usepackage{fancyhdr}                           %% page headers and footers
\usepackage[official,left]{eurosym}             %% the euro symbol; command: \euro
\usepackage{appendix}                           %% appendix layout
\usepackage{xspace}                             %% add space after macro depending on context
\usepackage{verbatim}                           %% provides the comment environment
\usepackage[dutch,USenglish]{babel}             %% language support
\usepackage{wrapfig}                            %% wrapping text around figures
\usepackage{longtable}                          %% tables spanning multiple pages
\usepackage{pgfplots}                           %% support for TikZ figures (Matlab/Python)
\pgfplotsset{compat=1.14}						%% Run in backwards compatibility mode
\usepackage[breaklinks=true,hidelinks,          %% implement hyperlinks (dvips yields minor problems with breaklinks;
bookmarksnumbered=true]{hyperref}   %% IEEEtran: set bookmarks=false)
%\usepackage[hyphenbreaks]{breakurl}            %% allow line breaks in URLs (don't use with PDFTeX)
\usepackage[final]{pdfpages}                    %% Include other pdfs
\usepackage[capitalize]{cleveref}				%% Referensing to figures, equations, etc.
\usepackage{units}								%% Appropriate behavior of units


\usepackage{silence}  							%% For filtering warnings
\usepackage{csquotes}
% Add doi=false if no DOI
\usepackage[style=authoryear-comp,isbn=false,url=false,date=year,backend=biber,maxbibnames=15,maxcitenames=2,uniquelist=false,uniquename=false,giveninits=true]{biblatex}
% Filter warnings issued by package biblatex starting with "Patching footnotes failed"
\WarningFilter{biblatex}{Patching footnotes failed}
%\renewcommand*{\bibfont}{\footnotesize}		%% Use this for papers
\setlength{\biblabelsep}{\labelsep}
\bibliography{../bib}


\pagestyle{fancy}                                       %% set page style
\fancyhf{}                                              %% clear all header & footer fields
\renewcommand*{\headrulewidth}{0pt}                  	%% No line in this case

\newcommand{\expectation}[1]{\textup{E} \left[ #1 \right]}
\newcommand{\mise}[2]{\textup{MISE}_{#1}\left( #2 \right)}
\newcommand{\amise}[2]{\textup{AMISE}_{#1} \left( #2 \right)}
\newcommand{\measure}[2]{J_{#1} \left( #2 \right)}
\newcommand*{\ud}{\mathrm{\,d}}

\usepackage{titlesec}
\titlespacing*{\paragraph}{0ex}{1ex}{1ex}
\newcommand{\toauthor}{\paragraph*{Comment to authors:} \itshape}
\newcommand{\fromauthor}{\paragraph*{Reply:} \normalfont}
\newcommand{\additionend}[1]{\color{black}[#1]\color{red}}
\newcommand{\addition}[1]{\additionend{#1}\ }

\usepackage{changebar}
\newcommand{\cstart}{\cbstart\color{red}}
\newcommand{\cend}{\cbend\color{black}}


% Notations
\newcommand{\amplitude}{\Delta v}
\newcommand{\attrtstart}{start time}
\newcommand{\attrtend}{end time}
\newcommand{\distancecondition}{d_{\mathrm{v,p}}}
\newcommand{\duration}{T}
\newcommand{\east}{x}
\newcommand{\north}{y}
\newcommand{\head}{\phi}
\newcommand{\egosub}{ego}
\newcommand{\egoeast}{\east_{\mathrm{\egosub}}}
\newcommand{\egonorth}{\north_{\mathrm{\egosub}}}
\newcommand{\egospeed}{\dot{\east}_{\mathrm{\egosub}}}
\newcommand{\egospeedinitsymbol}{v}
\newcommand{\egospeedinit}{\egospeedinitsymbol_{0}}
\newcommand{\egospeedinitb}{\egospeedinitsymbol_{0}}
\newcommand{\egospeedinitc}{\egospeedinitsymbol_{0}}
\newcommand{\egoacceleration}{\ddot{\east}_{\mathrm{\egosub}}}
\newcommand{\egoheading}{\head_{\mathrm{\egosub}}}
\newcommand{\function}{f}
\newcommand{\inputsystem}{u}
\newcommand{\dimensionstate}{n}
\newcommand{\parameter}{\theta}
\newcommand{\parametera}{a}
\newcommand{\parameterb}{b}
\newcommand{\pedsub}{ped}
\newcommand{\pedeast}{\east_{\mathrm{\pedsub}}}
\newcommand{\pednorth}{\north_{\mathrm{\pedsub}}}
\newcommand{\pednorthinit}{\north_{0}}
\newcommand{\pedheading}{\head_{\mathrm{\pedsub}}}
\newcommand{\pedspeed}{\dot{\north}_{\mathrm{\pedsub}}}
\newcommand{\origin}{O}
\newcommand{\scenario}{S}
\newcommand{\scenarioa}{\scenario_{1}}
\newcommand{\scenariob}{\scenario_{2}}
\newcommand{\scenarioc}{\scenario_{3}}
\newcommand{\scenariofallsinto}{\in}
\newcommand{\scenariocategory}{\mathcal{C}}
\newcommand{\scenariocategorya}{\scenariocategory_{1}}
\newcommand{\scenariocategoryb}{\scenariocategory_{2}}
\newcommand{\scenariocategoryc}{\scenariocategory_{3}}
\newcommand{\scenariocategoryfallsinto}{\subseteq}
\newcommand{\slopeego}{a_{\mathrm{\egosub}}}
\newcommand{\slopepedestrian}{v_{\mathrm{\pedsub}}}
\newcommand{\state}{x}
\newcommand{\statedot}{\dot{\state}}
\renewcommand{\time}{t}
\newcommand{\inittime}{\time_{0}}
\newcommand{\inittimeb}{\time_{1}}
\newcommand{\inittimec}{\time_{2}}
\newcommand{\hasone}{$1$}
\newcommand{\hasn}{$N$}

\begin{document}
	
\section*{Cover letter}

Dear Editor and Reviewers,

We thank you for your time and efforts to review the paper and for your constructive comments that were helpful in further improving the overall quality of the paper. We have carefully examined the suggestions raised by the reviewers. In response to these comments, we have prepared a revised version of the paper. A detailed reply to the review comments can be found below.  Additions to the paper are indicated in \cstart red \cend font. Furthermore, a gray bar is shown next to the updated parts of the paper.

Yours sincerely,
Erwin de Gelder, Jan-Pieter Paardekooper, Arash Khabbaz Saberi, Hala Elrofai, Olaf Op den Camp, Jeroen Ploeg, Ludwig Friendmann, Bart De Schutter


	
\section*{Reviewer 1}

\toauthor Novelty: Similar work has been published before, see e.g.\ \autocite{provine2004ontology, morignot2013ontology, schlenoff2003using, zhao2015core}. Thus, it is unclear what the novelty of the presented ontology is.

\fromauthor Thank you for the mentioned references. In the introduction, we added the following sentences to explain the new addition of our work in the automotive domain and how our work differs from previous published works: 
\cstart Ontologies have been widely used in the field of automated driving \autocite{provine2004ontology, morignot2013ontology, schlenoff2003using, zhao2015core, maiti2017conceptualization, benvenuti2017ontologybased, bagschik2017ontology}. However, to the best of our knowledge, we are the first to propose an ontology for scenarios for the assessment of AVs. 
From the implementation side, there are several file formats and methods of scenario ontologies, e.g., OpenSCENARIO \autocite{openscenario} and CommonRoad \autocite{althoff2017CommonRoad}. Our proposed ontology differs from these implementations as our ontology also allows for qualitative descriptions of scenarios, which is useful because it enables to group scenarios and to interpret the scenarios more easily.
Furthermore, our ontology is supported with the definitions and justifications of each of the terms.\cend

Furthermore, the manuscript's title (``Ontology for Scenarios for the Assessment of Automated Vehicles'') indicates the purpose of the ontology and, therefore, indicates the different purpose compared to other ontologies (among which the provided examples).

Although this manuscript is not the first article to provide a definition of scenario in the context of automated driving, we indicate that, as opposed to definitions from other literature, our definition ``is concrete enough to be used in quantitative analysis required for the assessment of AVs''.

We have also added \textcite{provine2004ontology, schlenoff2003using} to the references of ontologies mentioned in Section~II-A: ``Ontologies are widely applied in all kinds of research areas, e.g., [...], \cstart and vehicle path planning \autocite{provine2004ontology, schlenoff2003using}.\cend''



\toauthor Implementations of ontologies exist. The paper cites OpenScenario and CommonRoad, which are both implementations of ontologies. However, the authors do not mention in the introduction that these are concrete implementations with a concrete XML structure. What aspects are described in this paper that are not already implemented in OpenScenario or CommonRoad? This should be discussed in the paper. 

\fromauthor To address this comment, we have added the following sentence to the introduction: \cstart From the implementation side, there are several file formats and methods of scenario ontologies, e.g., OpenSCENARIO \autocite{openscenario} and CommonRoad \autocite{althoff2017CommonRoad}. Our proposed ontology differs from these implementations as our ontology also allows for qualitative descriptions of scenarios, which is useful because it enables to group scenarios and to interpret the scenarios more easily.
Furthermore, our ontology is supported with the definitions and justifications of each of the terms.\cend



\toauthor It is unclear whether an executable implementation of the proposed ontology exists. This should be clarified in the paper.

\fromauthor To clarify that we implemented the proposed ontology in a coding language, we have added to following line to the introduction (including the footnote): \cstart The implementation code of our ontology is publicly available\footnote{\cstart As a coding language, Python is used. The code is publicly available at \url{https://github.com/ErwindeGelder/ScenarioDomainModel}.\cend}\cend.



\toauthor The paper has several mistakes: 1. event: It is written that an event requires a time step. This, however, is not required, see e.g.\ finite state machines. Only in timed automata and in hybrid systems timing information is required.

\fromauthor To address this comment, we have clarified that we use a hybrid-system setting to define events and that, therefore, an event happens at a time instant rather than a time span. However, the timing itself may be undefined: ``\cstart We considered the definitions of event in hybrid systems and control and event-based control. Therefore, an event happens at some time instant rather than a time span. Note, however, that the timing itself may be undefined.\cend''

%\fromauthor This comment is not clear to us. In our manuscript, we did not write that an event requires a time step; we write that ``an event corresponds to a time instant [rather than a time span/step]''. Since hybrid systems and finite state machines are closely related, we assume mentioning hybrid systems suffices. Could the reviewer clarify what the intention is of this comment?



\toauthor 2. braking trajectory: Since a sine function is multiplied by a constant, it is hard to see why the acceleration should always be negative when $t_0$ can be freely chosen

\fromauthor There was indeed a mistake in the corresponding equation. We have corrected the corresponding equation as follows:
\begin{equation}
\egoacceleration(\time) = \frac{\pi \amplitude}{2\duration} \sin \left( \frac{\pi \color{red}\left(\color{black} \time \color{red} - \inittime\right)\color{black}}{\duration} \right),\ \egospeed(\inittime) = \egospeedinit,\ \time \in [\inittime, \inittime+\duration].
\end{equation}
Now the part inside the $\sin(\cdot)$ is in the range of $[0, \pi]$. Hence, $\egoacceleration(\time)$ always has the same sign as $\amplitude/\duration$.



\toauthor 3. state/state variable: A state consists of many state variables and each system has only one state. This is often confused in this paper

\fromauthor Thank you for pointing this out. We have now added a definition of the notion of a \emph{state variable} to the nomenclature (Section II-C):

\cstart State variable: \textcite[p.~163]{dorf2011modern} write that ``the state variables describe the present configuration of a system and can be used to determine the future response, given the excitation inputs and the equations describing the dynamics.'' In our case, ``the system'' could refer to an actor, a component, or a simulation. E.g., a state variable could be the acceleration of an actor.\cend

To avoid any confusion between a \emph{state variable} and a \emph{state}, we have changed \emph{state} into \emph{state \cstart vector\cend}. We have now defined \emph{state vector} as follows (replacing the definition of \emph{state}):

\cstart State vector: a state vector refers to ``the vector containing all $n$ state variables'' \autocite[p.~233]{dorf2011modern}.\cend

We have added the word \cstart \emph{variable} \cend and \cstart \emph{vector} \cend to \emph{state} at many occasions in the paper in order to avoid confusion.



\toauthor 4. ``an event marks a mode transition or the moment a system reaches a threshold'': This is not correct, since an event can be an external event that is broadcast or a user input.

% \fromauthor This comment is not clear to us. We do not claim that our definition of event is ``globally'' correct, but as far as we know, it fits the purpose of our ontology. Furthermore, the provided examples are possible to program using our proposed ontology. For example, if a state variable is defined to take the value of a certain user input, a threshold on that state variable can be used to trigger an event at a certain user input. Alternatively, a user input might trigger a mode transition, and, therefore, an event.
%Could the reviewer indicate whether this explanation suffices? If not, should the definition of \emph{event} be reviewed or would a more elaborate explanation suffice?

\fromauthor To address this comment, we have clarified that there are two options for defining an event and we provide an example of an event based on a user input:

\cstart
Definition~2 indicates that the moment of an event can be defined in two different ways: (1) by the system reaching a threshold or (2) by a mode transition. \cend [...] \cstart The second type of event, i.e., a mode transition, can, e.g., occur at the moment of a driver input.\cend



\toauthor Scope. The paper often discusses issues that are not related to autonomous driving. E.g.\ ontologies on computing, security, etc.\ are mentioned. It is unclear why these should be relevant for this paper.

\fromauthor In order to clearly justify some of the claims we make, we mention other fields of research for which ontologies have proven to be useful. This is used as an implicit argument that an ontology might also be useful for our purpose. However, to make the list of examples shorter and more compact, we have removed ``biology \autocite{gkoutos2004mouse}, security-related research \autocite{kim2005security}, socio-technical systems \autocite{vanDamPhDThesis2009}''.



\toauthor Focus. The paper is not very focused and overly lengthy. For instance, the term ``event'' is described using almost an entire page. I would have spent around 5 sentences to explain this.  

\fromauthor  In our discussions with the community (conferences, (ISO) working groups), we have noticed that the definitions of these terms is often opinion based. Hence, we have opted to use the following approach to define the terms \emph{scenario}, \emph{event}, \emph{activity}, and \emph{scenario category}: (1) list existing definitions, (2) draw conclusions from the existing literature, (3) define the term, and (4) finish with some remarks/implications. 

The arguments (1-2) and remarks (4) help to justify our proposed definitions. This is confirmed by the second reviewer (``The background research is extensive, and the definition development is justified.''). Therefore, we have opted to keep the current approach to define the terms \emph{scenario}, \emph{event}, \emph{activity}, and \emph{scenario category}. Please note that, considering this comment, we shortened the manuscript by almost 2 pages.



\toauthor Also, Fig.~2 is not required to understand the difference between categories and instances. 

\fromauthor The figure is a visualization of the text to improve the understanding of the reader. Therefore, we have opted to keep the figure. However, to shorten the paper, the figure is made smaller.



\toauthor The explanation after Def.~4 is also overly lengthy. 

\fromauthor As suggested by the reviewer, we have shortened this part of the manuscript:
\begin{itemize}
	\item The explanation of the ``fall into'' method (changed to \cstart ``comprise'' \cend and \cstart ``encompass''\cend, see comments of reviewer 2) has been shortened.
	\item Examples of the use of tags have been removed.
	\item The explanation of all tags in Fig.~3 has been omitted because we assume this is self-explanatory.
\end{itemize}



\toauthor Another example is the explanation of ``category'' using music styles. Most people know what a category is.  

\fromauthor We assume that this comment refers to the examples of the use of tags: ``Applications are very broad, e.g., from classification of audio data \autocite{kong2017joint} capturing musical characteristics from songs \autocite{ellis2011semantic} to tagging of Wikipedia pages \autocite{voss2006collaborative}.'' We have removed this sentence.



\toauthor One could probably write the whole paper with half the pages without loosing any information.

\fromauthor We considered this comment and shortened the manuscript by almost 2 pages.



\toauthor Renaming of existing concepts. To me, an activity is a state trajectory. Why do we need a new name?

\fromauthor We have chosen the name ``activity'' because --- as explained in Section III-C --- ``activity'' \autocite{elrofai2018scenario,catapult2018musicc} or ``action'' \autocite{ulbrich2015} are part of the existing literature on the subject. In the manuscript, we provide the reason why we opt for ``activity'' over ``action''. To the best of our knowledge, ``state (variable) trajectory'' is less common in the field of safety assessment of (automated) vehicles.



\toauthor Schematic overview in Fig.~4. To me this is quite trivial and does not require the lengthy explanation in the paper. The figure could also be made smaller.

\fromauthor Fig.~4 is now made smaller. For someone who is not familiar with the UML notation (which is conceivable because the manuscript is not submitted to a dedicated computer science journal), we think this explanation is useful. However, we considered to shorten the explanation, so we have removed the short examples.



\toauthor Example. The example is not overly exciting. It would be good if the authors could share the corresponding code for other researchers.

\fromauthor As suggested by the reviewer, we have made the code available on a github repository. This is now mentioned in the manuscript (including the footnote): ``\cstart The code of this example is publicly available\footnote{\cstart See \url{https://github.com/ErwindeGelder/ScenarioDomainModel}. The repository also contains more examples that the one presented here.\cend}\cend.'' To not further lengthen the paper, we use this rather straightforward example. However, as now mentioned in the manuscript, more examples will be shared on the github repository.



\section*{Reviewer 2}

\toauthor This is a very useful contribution to the AV industry. The background research is extensive, and the definition development is justified. Some thoughts:

Page 1, opening paragraph: The authors mix ADAS with automation (citation of Bengler et al.\ (2014)). Care should be taken to distinguish between the two.

\fromauthor Thank you for pointing this out. After citing \textcite{bengler2014threedecades}, we have added the following sentence to clarify this: \cstart Whereas \textcite{bengler2014threedecades} address driver assistance systems, it is not hard to imagine that the same holds for AVs.\cend



\toauthor Page 1, opening paragraph: ``too much resources''.

\fromauthor This is rephrased to ``\cstart too many resources\cend''.



\toauthor Page 5, 6: It might be useful to mention that if the activity of the ego vehicle is described rather than its goal, that an observer might not be able to determine if the ego vehicle successfully navigated the scenario.

\fromauthor Thank you for this suggestion. We have now added the following sentence at point 6 of Section~III-A: \cstart Note that if the activity of the ego vehicle is described rather than its goal, an observer might not be able to determine whether the ego vehicle has successfully responded to the scenario.\cend



\toauthor Page 6, under Definition 2: The example of the event involving the vehicle and pedestrian -- the point could be made that the threshold value is set by whomever is setting up the scenario, i.e., that it is flexible.

\fromauthor This is indeed true. We have added the following sentence after Definition 2 in Section~III-B: \cstart By using a variable threshold $\distancecondition$, the value is flexible and can be set differently to define multiple scenarios.\cend



\toauthor Page 7, Definition 4: I am not sure that ``scenario category'' is the best choice of nomenclature. Perhaps instead of ``category'', ``account'' or ``depiction'' could be used. The term ``category'' is a grouping term, but each scenario will be unique, despite having common features to other scenarios. It's not clear to me that multiple scenarios could fall under the scenario category -- an example should be provided. If the qualitative description is sufficiently detailed, multiple scenarios should not be able to ``fall into'' a scenario category.

\fromauthor We appreciate the different options for ``category''. Although ``depiction'' might be closer to the actual definition of the term, we opt for using ``category'', because this also addresses the fact that scenarios can be grouped into scenario categories. To clarify that multiple scenarios could fall under a scenario category, we have added the following to the example (Section~V-B): \cstart A different scenario can be defined by, e.g., changing the parameter values. This illustrates that the scenario category in Fig.~6 comprises multiple scenarios, namely all scenarios that only differ from the scenario in Fig.~7 because of different parameter values.\cend

\toauthor Further, ``fall into'' is colloquial, and I would use ``comprises'' instead (although the terms would have to be reversed).

\fromauthor We have changed the ``fall into'' method to the ``\cstart comprise\cend'' method. Furthermore, because it might be confusing that we use different symbols ($\scenariofallsinto$ and $\scenariocategoryfallsinto$), we choose to use different names:
\begin{itemize}
	\item A scenario category can \cstart comprise \cend one or multiple scenarios, e.g., $\scenarioa \scenariofallsinto \scenariocategoryb$ means that $\scenariocategoryb$ \cstart comprises \cend $\scenarioa$.
	\item A scenario category can \cstart encompass \cend one or multiple scenario categories, e.g., $\scenariocategorya \scenariocategoryfallsinto \scenariocategoryb$ means that $\scenariocategoryb$ \cstart encompasses \cend $\scenariocategorya$.
\end{itemize}

\toauthor Page 13, Static Environment block: You should mention that there are many facets to this quantitative description -- more than just number of lanes, lane width, and coordinates -- e.g., road friction.

\fromauthor To clarify that there are many facets to the quantitative description of the static environment, we have added the following sentence: \cstart Note that, in practice, the quantitative description of the static environment contains many more facets than the ones mentioned in Fig.~7.\cend



\section*{Reviewer 3}

\toauthor This paper describes a proposed ontology for defining scenarios in the context of assessing automated vehicle (AV) performance. This is a timely subject as the testing and use of AVs continues to increase. The paper does a nice job of describing and summarizing existing work in this subject area and providing definitions for components of the proposed ontology. The paper is extremely well-written and provides important contributions to the research literature in the area of automated vehicle assessment.

\fromauthor We thank the reviewer for this review.



\printbibliography

\end{document}