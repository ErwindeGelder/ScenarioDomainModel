\section{Case study}
\label{sec:case study}

To illustrate the proposed methods for generating scenarios (\cref{sec:generation}) and comparing a set of generated scenarios with a set of observed real-world scenarios (\cref{sec:comparison}), we apply the proposed methods in a case study.
We will first explain the scenario categories that we consider and how we parameterize the scenarios.
Next, we will demonstrate the scenario generation method in \cref{sec:case study parameter reduction}. 
Here, we will also show how our metric of \cref{eq:proposed metric} can be used to choose $\dimension$.
In \cref{sec:case study comparison metric}, we will demonstrate that our proposed metric of \cref{eq:proposed metric} better correlates with the Wasserstein metric of \cref{eq:wasserstein} than the empirical Wasserstein metric of \cref{eq:empirical wasserstein}.



\subsection{Scenario categories and parameterization}
\label{sec:case study intro}

In this case study, we looked at two scenario categories.
The first scenario category considers an ego vehicle that is following another vehicle that decelerates, see \cref{fig:lead vehicle decelerating}.
As a result, the ego vehicle might need to brake or change direction to avoid contact with the vehicle that decelerates.
The second scenario category considers a vehicle that performs a cut in, such that is becomes the leader vehicle of the ego vehicle, see \cref{fig:cut in}.
Depending on the speed and timing of the vehicle that performs a cut in, the ego vehicle might need to brake or change direction to avoid a collision.

\setlength{\figurewidth}{.99\linewidth}
\begin{figure}
	\centering
	% This file was created by matplotlib2tikz v0.6.17.
\begin{tikzpicture}

\definecolor{color0}{rgb}{0.8,1,0.8}
\definecolor{color1}{rgb}{0,0.4375,0.75}

\begin{axis}[
xmin=-15, xmax=15,
ymin=-5, ymax=5,
width=\figurewidth,
height=0.33\figurewidth,
tick align=outside,
xticklabel style = {align=center,text width=1},
yticklabel style = {align=right,text width=1},
tick pos=left,
x grid style={white!69.01960784313725!black},
y grid style={white!69.01960784313725!black},
axis background/.style={fill=color0},
ticks=none,
scale only axis
]
\path [draw=white!80.0!black, fill=white!80.0!black] (axis cs:-15,2.8)
--(axis cs:15,2.8)
--(axis cs:15,-2.8)
--(axis cs:-15,-2.8)
--cycle;

\addplot [semithick, black, forget plot]
table {%
-15 2.8
15 2.8
};
\addplot [semithick, black, forget plot]
table {%
15 -2.8
-15 -2.8
};
\addplot [semithick, black, forget plot]
table {%
-15 0
-14.5 0
};
\addplot [semithick, black, forget plot]
table {%
-12.5 0
-11.5 0
};
\addplot [semithick, black, forget plot]
table {%
-9.5 0
-8.5 0
};
\addplot [semithick, black, forget plot]
table {%
-6.5 0
-5.5 0
};
\addplot [semithick, black, forget plot]
table {%
-3.5 0
-2.5 0
};
\addplot [semithick, black, forget plot]
table {%
-0.499999999999999 0
0.500000000000002 0
};
\addplot [semithick, black, forget plot]
table {%
2.5 0
3.5 0
};
\addplot [semithick, black, forget plot]
table {%
5.5 0
6.5 0
};
\addplot [semithick, black, forget plot]
table {%
8.5 0
9.5 0
};
\addplot [semithick, black, forget plot]
table {%
11.5 0
12.5 0
};
\addplot [semithick, black, forget plot]
table {%
14.5 0
15 0
};
\path [fill=black, draw opacity=0] (axis cs:5,0.86)
--(axis cs:0.75,0.86)
--(axis cs:0.75,0.59)
--(axis cs:5,0.59)
--cycle;

\path [fill=black, draw opacity=0] (axis cs:5,1.94)
--(axis cs:0.75,1.94)
--(axis cs:0.75,2.21)
--(axis cs:5,2.21)
--cycle;

\path [draw=black, fill=color1] (axis cs:-12.25,1.40401785714286)
--(axis cs:-12.2418918918919,1.92633928571429)
--(axis cs:-12.1851351351351,2.12723214285714)
--(axis cs:-12.1040540540541,2.19955357142857)
--(axis cs:-11.9743243243243,2.24776785714286)
--(axis cs:-11.5121621621622,2.30401785714286)
--(axis cs:-8.09864864864865,2.26383928571429)
--(axis cs:-8.00945945945946,2.20758928571429)
--(axis cs:-7.87162162162162,2.03080357142857)
--(axis cs:-7.81486486486487,1.83794642857143)
--(axis cs:-7.75,1.40401785714286)
--(axis cs:-7.75,1.39598214285714)
--(axis cs:-7.81486486486487,0.962053571428571)
--(axis cs:-7.87162162162162,0.769196428571429)
--(axis cs:-8.00945945945946,0.592410714285714)
--(axis cs:-8.09864864864865,0.536160714285714)
--(axis cs:-11.5121621621622,0.495982142857143)
--(axis cs:-11.9743243243243,0.552232142857143)
--(axis cs:-12.1040540540541,0.600446428571428)
--(axis cs:-12.1851351351351,0.672767857142857)
--(axis cs:-12.2418918918919,0.873660714285714)
--(axis cs:-12.25,1.39598214285714)
--cycle;

\path [draw=black, fill=color1] (axis cs:-9.33108108108108,2.23169642857143)
--(axis cs:-9.33918918918919,2.43258928571429)
--(axis cs:-9.33108108108108,2.48080357142857)
--(axis cs:-9.29864864864865,2.45669642857143)
--(axis cs:-9.25,2.24776785714286)
--cycle;

\path [draw=black, fill=color1] (axis cs:-9.33108108108108,0.568303571428571)
--(axis cs:-9.33918918918919,0.367410714285714)
--(axis cs:-9.33108108108108,0.319196428571428)
--(axis cs:-9.29864864864865,0.343303571428571)
--(axis cs:-9.25,0.552232142857143)
--cycle;

\path [draw=black, fill=white] (axis cs:-11.7148648648649,1.40401785714286)
--(axis cs:-11.6986486486486,1.77366071428571)
--(axis cs:-11.65,1.99866071428571)
--(axis cs:-11.5851351351351,2.11919642857143)
--(axis cs:-11.5283783783784,2.12723214285714)
--(axis cs:-11.0175675675676,1.990625)
--(axis cs:-11.0662162162162,1.85401785714286)
--(axis cs:-11.0743243243243,1.64508928571429)
--(axis cs:-11.0743243243243,1.15491071428571)
--(axis cs:-11.0662162162162,0.945982142857143)
--(axis cs:-11.0175675675676,0.809375)
--(axis cs:-11.5283783783784,0.672767857142857)
--(axis cs:-11.5851351351351,0.680803571428571)
--(axis cs:-11.65,0.801339285714285)
--(axis cs:-11.6986486486486,1.02633928571429)
--(axis cs:-11.7148648648649,1.39598214285714)
--cycle;

\path [draw=black, fill=white] (axis cs:-8.91756756756757,1.40401785714286)
--(axis cs:-8.94189189189189,1.83794642857143)
--(axis cs:-9.03108108108108,2.11116071428571)
--(axis cs:-9.08783783783784,2.16741071428571)
--(axis cs:-9.61486486486486,1.95848214285714)
--(axis cs:-9.56621621621622,1.765625)
--(axis cs:-9.55,1.596875)
--(axis cs:-9.55,1.203125)
--(axis cs:-9.56621621621622,1.034375)
--(axis cs:-9.61486486486486,0.841517857142857)
--(axis cs:-9.08783783783784,0.632589285714286)
--(axis cs:-9.03108108108108,0.688839285714286)
--(axis cs:-8.94189189189189,0.962053571428571)
--(axis cs:-8.91756756756757,1.39598214285714)
--cycle;

\path [draw=black, fill=white] (axis cs:-11.0256756756757,2.23169642857143)
--(axis cs:-10.8148648648649,2.23169642857143)
--(axis cs:-10.8148648648649,2.07901785714286)
--(axis cs:-10.9202702702703,2.12723214285714)
--cycle;

\path [draw=black, fill=white] (axis cs:-11.0256756756757,0.568303571428571)
--(axis cs:-10.8148648648649,0.568303571428571)
--(axis cs:-10.8148648648649,0.720982142857143)
--(axis cs:-10.9202702702703,0.672767857142857)
--cycle;

\path [draw=black, fill=white] (axis cs:-10.7662162162162,2.06294642857143)
--(axis cs:-10.7662162162162,2.19151785714286)
--(axis cs:-10.7337837837838,2.22366071428571)
--(axis cs:-10.2067567567568,2.22366071428571)
--(axis cs:-10.1824324324324,2.18348214285714)
--(axis cs:-10.2310810810811,2.03883928571429)
--(axis cs:-10.3040540540541,2.00669642857143)
--(axis cs:-10.5716216216216,2.02276785714286)
--cycle;

\path [draw=black, fill=white] (axis cs:-10.7662162162162,0.737053571428571)
--(axis cs:-10.7662162162162,0.608482142857143)
--(axis cs:-10.7337837837838,0.576339285714286)
--(axis cs:-10.2067567567568,0.576339285714286)
--(axis cs:-10.1824324324324,0.616517857142857)
--(axis cs:-10.2310810810811,0.761160714285714)
--(axis cs:-10.3040540540541,0.793303571428571)
--(axis cs:-10.5716216216216,0.777232142857143)
--cycle;

\path [draw=black, fill=white] (axis cs:-10.15,1.98258928571429)
--(axis cs:-10.0202702702703,2.23169642857143)
--(axis cs:-9.28243243243243,2.23169642857143)
--(axis cs:-9.29054054054054,2.18348214285714)
--(axis cs:-9.70405405405405,2.00669642857143)
--cycle;

\path [draw=black, fill=white] (axis cs:-10.15,0.817410714285714)
--(axis cs:-10.0202702702703,0.568303571428571)
--(axis cs:-9.28243243243243,0.568303571428571)
--(axis cs:-9.29054054054054,0.616517857142857)
--(axis cs:-9.70405405405405,0.793303571428571)
--cycle;

\path [draw=black, fill=white] (axis cs:-8.2527027027027,2.24776785714286)
--(axis cs:-8.09054054054054,2.23973214285714)
--(axis cs:-7.96891891891892,2.11919642857143)
--(axis cs:-7.91216216216216,2.02276785714286)
--(axis cs:-7.89594594594595,1.92633928571429)
--(axis cs:-7.89594594594595,1.75758928571429)
--(axis cs:-7.98513513513513,1.91026785714286)
--cycle;

\path [draw=black, fill=white] (axis cs:-8.2527027027027,0.552232142857143)
--(axis cs:-8.09054054054054,0.560267857142857)
--(axis cs:-7.96891891891892,0.680803571428571)
--(axis cs:-7.91216216216216,0.777232142857143)
--(axis cs:-7.89594594594595,0.873660714285714)
--(axis cs:-7.89594594594595,1.04241071428571)
--(axis cs:-7.98513513513513,0.889732142857143)
--cycle;

\path [draw=black, fill=red] (axis cs:2.75,1.40401785714286)
--(axis cs:2.75810810810811,1.92633928571429)
--(axis cs:2.81486486486487,2.12723214285714)
--(axis cs:2.89594594594595,2.19955357142857)
--(axis cs:3.02567567567568,2.24776785714286)
--(axis cs:3.48783783783784,2.30401785714286)
--(axis cs:6.90135135135135,2.26383928571429)
--(axis cs:6.99054054054054,2.20758928571429)
--(axis cs:7.12837837837838,2.03080357142857)
--(axis cs:7.18513513513513,1.83794642857143)
--(axis cs:7.25,1.40401785714286)
--(axis cs:7.25,1.39598214285714)
--(axis cs:7.18513513513513,0.962053571428572)
--(axis cs:7.12837837837838,0.769196428571429)
--(axis cs:6.99054054054054,0.592410714285714)
--(axis cs:6.90135135135135,0.536160714285714)
--(axis cs:3.48783783783784,0.495982142857143)
--(axis cs:3.02567567567568,0.552232142857143)
--(axis cs:2.89594594594595,0.600446428571429)
--(axis cs:2.81486486486487,0.672767857142857)
--(axis cs:2.75810810810811,0.873660714285714)
--(axis cs:2.75,1.39598214285714)
--cycle;

\path [draw=black, fill=red] (axis cs:5.66891891891892,2.23169642857143)
--(axis cs:5.66081081081081,2.43258928571429)
--(axis cs:5.66891891891892,2.48080357142857)
--(axis cs:5.70135135135135,2.45669642857143)
--(axis cs:5.75,2.24776785714286)
--cycle;

\path [draw=black, fill=red] (axis cs:5.66891891891892,0.568303571428571)
--(axis cs:5.66081081081081,0.367410714285714)
--(axis cs:5.66891891891892,0.319196428571429)
--(axis cs:5.70135135135135,0.343303571428571)
--(axis cs:5.75,0.552232142857143)
--cycle;

\path [draw=black, fill=white] (axis cs:3.28513513513514,1.40401785714286)
--(axis cs:3.30135135135135,1.77366071428571)
--(axis cs:3.35,1.99866071428571)
--(axis cs:3.41486486486486,2.11919642857143)
--(axis cs:3.47162162162162,2.12723214285714)
--(axis cs:3.98243243243243,1.990625)
--(axis cs:3.93378378378378,1.85401785714286)
--(axis cs:3.92567567567568,1.64508928571429)
--(axis cs:3.92567567567568,1.15491071428571)
--(axis cs:3.93378378378378,0.945982142857143)
--(axis cs:3.98243243243243,0.809375)
--(axis cs:3.47162162162162,0.672767857142857)
--(axis cs:3.41486486486486,0.680803571428571)
--(axis cs:3.35,0.801339285714286)
--(axis cs:3.30135135135135,1.02633928571429)
--(axis cs:3.28513513513514,1.39598214285714)
--cycle;

\path [draw=black, fill=white] (axis cs:6.08243243243243,1.40401785714286)
--(axis cs:6.05810810810811,1.83794642857143)
--(axis cs:5.96891891891892,2.11116071428571)
--(axis cs:5.91216216216216,2.16741071428571)
--(axis cs:5.38513513513514,1.95848214285714)
--(axis cs:5.43378378378378,1.765625)
--(axis cs:5.45,1.596875)
--(axis cs:5.45,1.203125)
--(axis cs:5.43378378378378,1.034375)
--(axis cs:5.38513513513514,0.841517857142857)
--(axis cs:5.91216216216216,0.632589285714286)
--(axis cs:5.96891891891892,0.688839285714286)
--(axis cs:6.05810810810811,0.962053571428572)
--(axis cs:6.08243243243243,1.39598214285714)
--cycle;

\path [draw=black, fill=white] (axis cs:3.97432432432432,2.23169642857143)
--(axis cs:4.18513513513513,2.23169642857143)
--(axis cs:4.18513513513513,2.07901785714286)
--(axis cs:4.07972972972973,2.12723214285714)
--cycle;

\path [draw=black, fill=white] (axis cs:3.97432432432432,0.568303571428571)
--(axis cs:4.18513513513513,0.568303571428571)
--(axis cs:4.18513513513513,0.720982142857143)
--(axis cs:4.07972972972973,0.672767857142857)
--cycle;

\path [draw=black, fill=white] (axis cs:4.23378378378378,2.06294642857143)
--(axis cs:4.23378378378378,2.19151785714286)
--(axis cs:4.26621621621622,2.22366071428571)
--(axis cs:4.79324324324324,2.22366071428571)
--(axis cs:4.81756756756757,2.18348214285714)
--(axis cs:4.76891891891892,2.03883928571429)
--(axis cs:4.69594594594595,2.00669642857143)
--(axis cs:4.42837837837838,2.02276785714286)
--cycle;

\path [draw=black, fill=white] (axis cs:4.23378378378378,0.737053571428572)
--(axis cs:4.23378378378378,0.608482142857143)
--(axis cs:4.26621621621622,0.576339285714286)
--(axis cs:4.79324324324324,0.576339285714286)
--(axis cs:4.81756756756757,0.616517857142857)
--(axis cs:4.76891891891892,0.761160714285714)
--(axis cs:4.69594594594595,0.793303571428572)
--(axis cs:4.42837837837838,0.777232142857143)
--cycle;

\path [draw=black, fill=white] (axis cs:4.85,1.98258928571429)
--(axis cs:4.97972972972973,2.23169642857143)
--(axis cs:5.71756756756757,2.23169642857143)
--(axis cs:5.70945945945946,2.18348214285714)
--(axis cs:5.29594594594595,2.00669642857143)
--cycle;

\path [draw=black, fill=white] (axis cs:4.85,0.817410714285714)
--(axis cs:4.97972972972973,0.568303571428571)
--(axis cs:5.71756756756757,0.568303571428571)
--(axis cs:5.70945945945946,0.616517857142857)
--(axis cs:5.29594594594595,0.793303571428572)
--cycle;

\path [draw=black, fill=white] (axis cs:6.7472972972973,2.24776785714286)
--(axis cs:6.90945945945946,2.23973214285714)
--(axis cs:7.03108108108108,2.11919642857143)
--(axis cs:7.08783783783784,2.02276785714286)
--(axis cs:7.10405405405405,1.92633928571429)
--(axis cs:7.10405405405405,1.75758928571429)
--(axis cs:7.01486486486487,1.91026785714286)
--cycle;

\path [draw=black, fill=white] (axis cs:6.7472972972973,0.552232142857143)
--(axis cs:6.90945945945946,0.560267857142857)
--(axis cs:7.03108108108108,0.680803571428572)
--(axis cs:7.08783783783784,0.777232142857143)
--(axis cs:7.10405405405405,0.873660714285714)
--(axis cs:7.10405405405405,1.04241071428571)
--(axis cs:7.01486486486487,0.889732142857143)
--cycle;

\addplot [black, forget plot]
table {%
-11.6337837837838 2.14330357142857
-11.8608108108108 2.13526785714286
-12.0716216216216 2.07901785714286
-12.1364864864865 1.99866071428571
-12.1364864864865 0.801339285714285
-12.0716216216216 0.720982142857143
-11.8608108108108 0.664732142857143
-11.6337837837838 0.656696428571428
};
\addplot [black, forget plot]
table {%
-8.00945945945946 1.82991071428571
-7.98513513513513 1.85401785714286
-7.89594594594595 1.71741071428571
-7.89594594594595 1.08258928571429
-7.98513513513513 0.945982142857143
-8.00945945945946 0.970089285714286
};
\addplot [black, forget plot]
table {%
-8.26081081081081 2.13526785714286
-9.0472972972973 2.16741071428571
-8.00945945945946 1.82991071428571
-8.00945945945946 0.970089285714286
-9.0472972972973 0.632589285714286
-8.26081081081081 0.664732142857143
};
\addplot [semithick, red, forget plot]
table {%
-7.75 1.4
-2.75 1.4
};
\addplot [semithick, red, forget plot]
table {%
-3.5 2.15
-2.75 1.4
-3.5 0.65
};
\addplot [black, forget plot]
table {%
3.36621621621622 2.14330357142857
3.13918918918919 2.13526785714286
2.92837837837838 2.07901785714286
2.86351351351351 1.99866071428571
2.86351351351351 0.801339285714286
2.92837837837838 0.720982142857143
3.13918918918919 0.664732142857143
3.36621621621622 0.656696428571429
};
\addplot [black, forget plot]
table {%
6.99054054054054 1.82991071428571
7.01486486486487 1.85401785714286
7.10405405405405 1.71741071428571
7.10405405405405 1.08258928571429
7.01486486486487 0.945982142857143
6.99054054054054 0.970089285714286
};
\addplot [black, forget plot]
table {%
6.73918918918919 2.13526785714286
5.9527027027027 2.16741071428571
6.99054054054054 1.82991071428571
6.99054054054054 0.970089285714286
5.9527027027027 0.632589285714286
6.73918918918919 0.664732142857143
};
\addplot [semithick, red, forget plot]
table {%
7.25 1.4
10.25 1.4
};
\addplot [semithick, red, forget plot]
table {%
9.5 2.15
10.25 1.4
9.5 0.650000000000001
};
\end{axis}

\end{tikzpicture}
	\caption{Schematic representation of the scenario category ``lead vehicle decelerating''. The blue vehicle denotes the ego vehicle.}
	\label{fig:lead vehicle decelerating}
\end{figure}

\begin{figure}
	\centering
	% This file was created by matplotlib2tikz v0.7.5.
\begin{tikzpicture}

\definecolor{color0}{rgb}{0.8,1,0.8}
\definecolor{color1}{rgb}{0,0.4375,0.75}

\begin{axis}[
axis background/.style={fill=color0},
height=0.33\figurewidth,
scale only axis,
tick align=outside,
tick pos=left,
ticks=none,
width=\figurewidth,
x grid style={white!69.01960784313725!black},
xmin=-15, xmax=15,
xtick style={color=black},
y grid style={white!69.01960784313725!black},
ymin=-5, ymax=5,
ytick style={color=black}
]
\path [draw=white!80.0!black, fill=white!80.0!black]
(axis cs:-15,2.8)
--(axis cs:15,2.8)
--(axis cs:15,-2.8)
--(axis cs:-15,-2.8)
--cycle;
\addplot [semithick, black]
table {%
-15 2.8
15 2.8
};
\addplot [semithick, black]
table {%
15 -2.8
-15 -2.8
};
\addplot [semithick, black]
table {%
-15 0
-14.5 0
};
\addplot [semithick, black]
table {%
-12.5 0
-11.5 0
};
\addplot [semithick, black]
table {%
-9.5 0
-8.5 0
};
\addplot [semithick, black]
table {%
-6.5 0
-5.5 0
};
\addplot [semithick, black]
table {%
-3.5 0
-2.5 0
};
\addplot [semithick, black]
table {%
-0.499999999999999 0
0.500000000000002 0
};
\addplot [semithick, black]
table {%
2.5 0
3.5 0
};
\addplot [semithick, black]
table {%
5.5 0
6.5 0
};
\addplot [semithick, black]
table {%
8.5 0
9.5 0
};
\addplot [semithick, black]
table {%
11.5 0
12.5 0
};
\addplot [semithick, black]
table {%
14.5 0
15 0
};
\path [draw=black, fill=color1]
(axis cs:-12.25,-1.39598214285714)
--(axis cs:-12.2418918918919,-0.873660714285714)
--(axis cs:-12.1851351351351,-0.672767857142857)
--(axis cs:-12.1040540540541,-0.600446428571429)
--(axis cs:-11.9743243243243,-0.552232142857143)
--(axis cs:-11.5121621621622,-0.495982142857143)
--(axis cs:-8.09864864864865,-0.536160714285714)
--(axis cs:-8.00945945945946,-0.592410714285714)
--(axis cs:-7.87162162162162,-0.769196428571428)
--(axis cs:-7.81486486486487,-0.962053571428571)
--(axis cs:-7.75,-1.39598214285714)
--(axis cs:-7.75,-1.40401785714286)
--(axis cs:-7.81486486486487,-1.83794642857143)
--(axis cs:-7.87162162162162,-2.03080357142857)
--(axis cs:-8.00945945945946,-2.20758928571429)
--(axis cs:-8.09864864864865,-2.26383928571429)
--(axis cs:-11.5121621621622,-2.30401785714286)
--(axis cs:-11.9743243243243,-2.24776785714286)
--(axis cs:-12.1040540540541,-2.19955357142857)
--(axis cs:-12.1851351351351,-2.12723214285714)
--(axis cs:-12.2418918918919,-1.92633928571429)
--(axis cs:-12.25,-1.40401785714286)
--cycle;
\path [draw=black, fill=color1]
(axis cs:-9.33108108108108,-0.568303571428571)
--(axis cs:-9.33918918918919,-0.367410714285714)
--(axis cs:-9.33108108108108,-0.319196428571428)
--(axis cs:-9.29864864864865,-0.343303571428571)
--(axis cs:-9.25,-0.552232142857143)
--cycle;
\path [draw=black, fill=color1]
(axis cs:-9.33108108108108,-2.23169642857143)
--(axis cs:-9.33918918918919,-2.43258928571429)
--(axis cs:-9.33108108108108,-2.48080357142857)
--(axis cs:-9.29864864864865,-2.45669642857143)
--(axis cs:-9.25,-2.24776785714286)
--cycle;
\path [draw=black, fill=white]
(axis cs:-11.7148648648649,-1.39598214285714)
--(axis cs:-11.6986486486486,-1.02633928571429)
--(axis cs:-11.65,-0.801339285714286)
--(axis cs:-11.5851351351351,-0.680803571428571)
--(axis cs:-11.5283783783784,-0.672767857142857)
--(axis cs:-11.0175675675676,-0.809375)
--(axis cs:-11.0662162162162,-0.945982142857143)
--(axis cs:-11.0743243243243,-1.15491071428571)
--(axis cs:-11.0743243243243,-1.64508928571429)
--(axis cs:-11.0662162162162,-1.85401785714286)
--(axis cs:-11.0175675675676,-1.990625)
--(axis cs:-11.5283783783784,-2.12723214285714)
--(axis cs:-11.5851351351351,-2.11919642857143)
--(axis cs:-11.65,-1.99866071428571)
--(axis cs:-11.6986486486486,-1.77366071428571)
--(axis cs:-11.7148648648649,-1.40401785714286)
--cycle;
\path [draw=black, fill=white]
(axis cs:-8.91756756756757,-1.39598214285714)
--(axis cs:-8.94189189189189,-0.962053571428571)
--(axis cs:-9.03108108108108,-0.688839285714286)
--(axis cs:-9.08783783783784,-0.632589285714286)
--(axis cs:-9.61486486486486,-0.841517857142857)
--(axis cs:-9.56621621621622,-1.034375)
--(axis cs:-9.55,-1.203125)
--(axis cs:-9.55,-1.596875)
--(axis cs:-9.56621621621622,-1.765625)
--(axis cs:-9.61486486486486,-1.95848214285714)
--(axis cs:-9.08783783783784,-2.16741071428571)
--(axis cs:-9.03108108108108,-2.11116071428571)
--(axis cs:-8.94189189189189,-1.83794642857143)
--(axis cs:-8.91756756756757,-1.40401785714286)
--cycle;
\path [draw=black, fill=white]
(axis cs:-11.0256756756757,-0.568303571428571)
--(axis cs:-10.8148648648649,-0.568303571428571)
--(axis cs:-10.8148648648649,-0.720982142857143)
--(axis cs:-10.9202702702703,-0.672767857142857)
--cycle;
\path [draw=black, fill=white]
(axis cs:-11.0256756756757,-2.23169642857143)
--(axis cs:-10.8148648648649,-2.23169642857143)
--(axis cs:-10.8148648648649,-2.07901785714286)
--(axis cs:-10.9202702702703,-2.12723214285714)
--cycle;
\path [draw=black, fill=white]
(axis cs:-10.7662162162162,-0.737053571428571)
--(axis cs:-10.7662162162162,-0.608482142857143)
--(axis cs:-10.7337837837838,-0.576339285714286)
--(axis cs:-10.2067567567568,-0.576339285714286)
--(axis cs:-10.1824324324324,-0.616517857142857)
--(axis cs:-10.2310810810811,-0.761160714285714)
--(axis cs:-10.3040540540541,-0.793303571428571)
--(axis cs:-10.5716216216216,-0.777232142857143)
--cycle;
\path [draw=black, fill=white]
(axis cs:-10.7662162162162,-2.06294642857143)
--(axis cs:-10.7662162162162,-2.19151785714286)
--(axis cs:-10.7337837837838,-2.22366071428571)
--(axis cs:-10.2067567567568,-2.22366071428571)
--(axis cs:-10.1824324324324,-2.18348214285714)
--(axis cs:-10.2310810810811,-2.03883928571429)
--(axis cs:-10.3040540540541,-2.00669642857143)
--(axis cs:-10.5716216216216,-2.02276785714286)
--cycle;
\path [draw=black, fill=white]
(axis cs:-10.15,-0.817410714285714)
--(axis cs:-10.0202702702703,-0.568303571428571)
--(axis cs:-9.28243243243243,-0.568303571428571)
--(axis cs:-9.29054054054054,-0.616517857142857)
--(axis cs:-9.70405405405405,-0.793303571428571)
--cycle;
\path [draw=black, fill=white]
(axis cs:-10.15,-1.98258928571429)
--(axis cs:-10.0202702702703,-2.23169642857143)
--(axis cs:-9.28243243243243,-2.23169642857143)
--(axis cs:-9.29054054054054,-2.18348214285714)
--(axis cs:-9.70405405405405,-2.00669642857143)
--cycle;
\path [draw=black, fill=white]
(axis cs:-8.2527027027027,-0.552232142857143)
--(axis cs:-8.09054054054054,-0.560267857142857)
--(axis cs:-7.96891891891892,-0.680803571428571)
--(axis cs:-7.91216216216216,-0.777232142857143)
--(axis cs:-7.89594594594595,-0.873660714285714)
--(axis cs:-7.89594594594595,-1.04241071428571)
--(axis cs:-7.98513513513513,-0.889732142857143)
--cycle;
\path [draw=black, fill=white]
(axis cs:-8.2527027027027,-2.24776785714286)
--(axis cs:-8.09054054054054,-2.23973214285714)
--(axis cs:-7.96891891891892,-2.11919642857143)
--(axis cs:-7.91216216216216,-2.02276785714286)
--(axis cs:-7.89594594594595,-1.92633928571429)
--(axis cs:-7.89594594594595,-1.75758928571429)
--(axis cs:-7.98513513513513,-1.91026785714286)
--cycle;
\path [draw=black, fill=red]
(axis cs:-7.25,1.40401785714286)
--(axis cs:-7.24189189189189,1.92633928571429)
--(axis cs:-7.18513513513513,2.12723214285714)
--(axis cs:-7.10405405405405,2.19955357142857)
--(axis cs:-6.97432432432432,2.24776785714286)
--(axis cs:-6.51216216216216,2.30401785714286)
--(axis cs:-3.09864864864865,2.26383928571429)
--(axis cs:-3.00945945945946,2.20758928571429)
--(axis cs:-2.87162162162162,2.03080357142857)
--(axis cs:-2.81486486486487,1.83794642857143)
--(axis cs:-2.75,1.40401785714286)
--(axis cs:-2.75,1.39598214285714)
--(axis cs:-2.81486486486487,0.962053571428571)
--(axis cs:-2.87162162162162,0.769196428571429)
--(axis cs:-3.00945945945946,0.592410714285714)
--(axis cs:-3.09864864864865,0.536160714285714)
--(axis cs:-6.51216216216216,0.495982142857143)
--(axis cs:-6.97432432432432,0.552232142857143)
--(axis cs:-7.10405405405405,0.600446428571428)
--(axis cs:-7.18513513513513,0.672767857142857)
--(axis cs:-7.24189189189189,0.873660714285714)
--(axis cs:-7.25,1.39598214285714)
--cycle;
\path [draw=black, fill=red]
(axis cs:-4.33108108108108,2.23169642857143)
--(axis cs:-4.33918918918919,2.43258928571429)
--(axis cs:-4.33108108108108,2.48080357142857)
--(axis cs:-4.29864864864865,2.45669642857143)
--(axis cs:-4.25,2.24776785714286)
--cycle;
\path [draw=black, fill=red]
(axis cs:-4.33108108108108,0.568303571428571)
--(axis cs:-4.33918918918919,0.367410714285714)
--(axis cs:-4.33108108108108,0.319196428571428)
--(axis cs:-4.29864864864865,0.343303571428571)
--(axis cs:-4.25,0.552232142857143)
--cycle;
\path [draw=black, fill=white]
(axis cs:-6.71486486486486,1.40401785714286)
--(axis cs:-6.69864864864865,1.77366071428571)
--(axis cs:-6.65,1.99866071428571)
--(axis cs:-6.58513513513514,2.11919642857143)
--(axis cs:-6.52837837837838,2.12723214285714)
--(axis cs:-6.01756756756757,1.990625)
--(axis cs:-6.06621621621622,1.85401785714286)
--(axis cs:-6.07432432432432,1.64508928571429)
--(axis cs:-6.07432432432432,1.15491071428571)
--(axis cs:-6.06621621621622,0.945982142857143)
--(axis cs:-6.01756756756757,0.809375)
--(axis cs:-6.52837837837838,0.672767857142857)
--(axis cs:-6.58513513513514,0.680803571428571)
--(axis cs:-6.65,0.801339285714285)
--(axis cs:-6.69864864864865,1.02633928571429)
--(axis cs:-6.71486486486486,1.39598214285714)
--cycle;
\path [draw=black, fill=white]
(axis cs:-3.91756756756757,1.40401785714286)
--(axis cs:-3.94189189189189,1.83794642857143)
--(axis cs:-4.03108108108108,2.11116071428571)
--(axis cs:-4.08783783783784,2.16741071428571)
--(axis cs:-4.61486486486486,1.95848214285714)
--(axis cs:-4.56621621621622,1.765625)
--(axis cs:-4.55,1.596875)
--(axis cs:-4.55,1.203125)
--(axis cs:-4.56621621621622,1.034375)
--(axis cs:-4.61486486486486,0.841517857142857)
--(axis cs:-4.08783783783784,0.632589285714286)
--(axis cs:-4.03108108108108,0.688839285714286)
--(axis cs:-3.94189189189189,0.962053571428571)
--(axis cs:-3.91756756756757,1.39598214285714)
--cycle;
\path [draw=black, fill=white]
(axis cs:-6.02567567567568,2.23169642857143)
--(axis cs:-5.81486486486487,2.23169642857143)
--(axis cs:-5.81486486486487,2.07901785714286)
--(axis cs:-5.92027027027027,2.12723214285714)
--cycle;
\path [draw=black, fill=white]
(axis cs:-6.02567567567568,0.568303571428571)
--(axis cs:-5.81486486486487,0.568303571428571)
--(axis cs:-5.81486486486487,0.720982142857143)
--(axis cs:-5.92027027027027,0.672767857142857)
--cycle;
\path [draw=black, fill=white]
(axis cs:-5.76621621621622,2.06294642857143)
--(axis cs:-5.76621621621622,2.19151785714286)
--(axis cs:-5.73378378378378,2.22366071428571)
--(axis cs:-5.20675675675676,2.22366071428571)
--(axis cs:-5.18243243243243,2.18348214285714)
--(axis cs:-5.23108108108108,2.03883928571429)
--(axis cs:-5.30405405405405,2.00669642857143)
--(axis cs:-5.57162162162162,2.02276785714286)
--cycle;
\path [draw=black, fill=white]
(axis cs:-5.76621621621622,0.737053571428571)
--(axis cs:-5.76621621621622,0.608482142857143)
--(axis cs:-5.73378378378378,0.576339285714286)
--(axis cs:-5.20675675675676,0.576339285714286)
--(axis cs:-5.18243243243243,0.616517857142857)
--(axis cs:-5.23108108108108,0.761160714285714)
--(axis cs:-5.30405405405405,0.793303571428571)
--(axis cs:-5.57162162162162,0.777232142857143)
--cycle;
\path [draw=black, fill=white]
(axis cs:-5.15,1.98258928571429)
--(axis cs:-5.02027027027027,2.23169642857143)
--(axis cs:-4.28243243243243,2.23169642857143)
--(axis cs:-4.29054054054054,2.18348214285714)
--(axis cs:-4.70405405405405,2.00669642857143)
--cycle;
\path [draw=black, fill=white]
(axis cs:-5.15,0.817410714285714)
--(axis cs:-5.02027027027027,0.568303571428571)
--(axis cs:-4.28243243243243,0.568303571428571)
--(axis cs:-4.29054054054054,0.616517857142857)
--(axis cs:-4.70405405405405,0.793303571428571)
--cycle;
\path [draw=black, fill=white]
(axis cs:-3.2527027027027,2.24776785714286)
--(axis cs:-3.09054054054054,2.23973214285714)
--(axis cs:-2.96891891891892,2.11919642857143)
--(axis cs:-2.91216216216216,2.02276785714286)
--(axis cs:-2.89594594594595,1.92633928571429)
--(axis cs:-2.89594594594595,1.75758928571429)
--(axis cs:-2.98513513513514,1.91026785714286)
--cycle;
\path [draw=black, fill=white]
(axis cs:-3.2527027027027,0.552232142857143)
--(axis cs:-3.09054054054054,0.560267857142857)
--(axis cs:-2.96891891891892,0.680803571428571)
--(axis cs:-2.91216216216216,0.777232142857143)
--(axis cs:-2.89594594594595,0.873660714285714)
--(axis cs:-2.89594594594595,1.04241071428571)
--(axis cs:-2.98513513513514,0.889732142857143)
--cycle;
\addplot [black]
table {%
-11.6337837837838 -0.656696428571429
-11.8608108108108 -0.664732142857143
-12.0716216216216 -0.720982142857143
-12.1364864864865 -0.801339285714286
-12.1364864864865 -1.99866071428571
-12.0716216216216 -2.07901785714286
-11.8608108108108 -2.13526785714286
-11.6337837837838 -2.14330357142857
};
\addplot [black]
table {%
-8.00945945945946 -0.970089285714286
-7.98513513513513 -0.945982142857143
-7.89594594594595 -1.08258928571429
-7.89594594594595 -1.71741071428571
-7.98513513513513 -1.85401785714286
-8.00945945945946 -1.82991071428571
};
\addplot [black]
table {%
-8.26081081081081 -0.664732142857143
-9.0472972972973 -0.632589285714286
-8.00945945945946 -0.970089285714286
-8.00945945945946 -1.82991071428571
-9.0472972972973 -2.16741071428571
-8.26081081081081 -2.13526785714286
};
\addplot [semithick, red]
table {%
-7.75 -1.4
-2.75 -1.4
};
\addplot [semithick, red]
table {%
-3.5 -0.649999999999999
-2.75 -1.4
-3.5 -2.15
};
\addplot [black]
table {%
-6.63378378378378 2.14330357142857
-6.86081081081081 2.13526785714286
-7.07162162162162 2.07901785714286
-7.13648648648649 1.99866071428571
-7.13648648648649 0.801339285714285
-7.07162162162162 0.720982142857143
-6.86081081081081 0.664732142857143
-6.63378378378378 0.656696428571428
};
\addplot [black]
table {%
-3.00945945945946 1.82991071428571
-2.98513513513514 1.85401785714286
-2.89594594594595 1.71741071428571
-2.89594594594595 1.08258928571429
-2.98513513513514 0.945982142857143
-3.00945945945946 0.970089285714286
};
\addplot [black]
table {%
-3.26081081081081 2.13526785714286
-4.0472972972973 2.16741071428571
-3.00945945945946 1.82991071428571
-3.00945945945946 0.970089285714286
-4.0472972972973 0.632589285714286
-3.26081081081081 0.664732142857143
};
\addplot [semithick, red]
table {%
-2.75 1.4
-0.75 1.4
-0.434210526315789 1.38090582476381
-0.118421052631579 1.32414413838089
0.197368421052632 1.23126325168908
0.513157894736842 1.10479671315495
0.828947368421053 0.948194200276037
1.14473684210526 0.765727421371397
1.46052631578947 0.562373594514157
1.77631578947368 0.343679681997119
2.09210526315789 0.115611083661265
2.40789473684211 -0.115611083661265
2.72368421052632 -0.343679681997119
3.03947368421053 -0.562373594514157
3.35526315789474 -0.765727421371398
3.67105263157895 -0.948194200276037
3.98684210526316 -1.10479671315495
4.30263157894737 -1.23126325168908
4.61842105263158 -1.32414413838089
4.93421052631579 -1.38090582476381
5.25 -1.4
7.25 -1.4
};
\addplot [semithick, red]
table {%
6.5 -0.65
7.25 -1.4
6.5 -2.15
};
\end{axis}

\end{tikzpicture}
	\caption{Schematic representation of the scenario category ``cut in''. The blue vehicle denotes the ego vehicle.}
	\label{fig:cut in}
\end{figure}

To obtain the scenarios, the data set described in \autocite{paardekooper2019dataset6000km} was used. 
The data were recorded from a single vehicle in which different drivers were asked to drive a prescribed route. 
The majority of the route was on the highway.
To measure the surrounding traffic, the vehicle was equipped with three radars and one camera. 
The surrounding traffic was measured by fusing the data of the radars and the camera \autocite{elfring2016effective}.
To extract the lead-vehicle-braking and cut-in scenarios from the data set, the data was automatically annotated with tags that described the behavior and status of the ego vehicle and its surrounding traffic \autocite{degelder2020scenariomining}.
To find the scenarios, we searches for a particular combination of tags \autocite{degelder2020scenariomining}.

In total, 1150 lead-vehicle-decelerating scenarios were found. 
One quarter of these scenarios were used for testing ($\scenariostestnumberof=287$) and the $\scenariosnumberof=863$ remaining scenarios were used for generating $\scenariosgeneratednumberof=10000$ new scenarios.
To describe a lead-vehicle-decelerating scenario, we considered the the speed of the lead vehicle ($\dimensiontimeseries=1$) at $\numberoftimesegments=50$ time instants. 
In \cref{fig:speed lvd observed}, the speed of the lead vehicle of 100 randomly-selected observed lead-vehicle-decelerating scenarios are shown.
As the only additional parameter ($\dimensionextraparameters=1$), we used the duration of the scenario, i.e., $\timeend-\timestart$.
Thus, $\dimensionscenariopars=51$.
For the weighting of the scenario parameters, we used:
\begin{equation*}
	\weight{1}=\weight{2}=\ldots=\weight{50}\approx 0.00543, \weight{51}\approx 0.135.
\end{equation*}

In total, 289 cut-in scenarios were found. 
One quarter of these scenarios were used for testing ($\scenariostestnumberof=72$) and the $\scenariosnumberof=217$ remaining scenarios were used for generating $\scenariosgeneratednumberof=5000$ new scenarios.
To describe a cut-in scenario, we considered the speed of the lead vehicle and the lateral position of the lead vehicle with respect to the center of the ego vehicle's lane ($\dimensiontimeseries=2$) at $\numberoftimesegments=50$ time instants.
We used $\dimensionextraparameters=3$ extra parameters: the duration of the scenario, the initial speed of the ego vehicle, and the initial longitudinal position of the cutting-in vehicle with respect to the ego vehicle.
Thus, $\dimensionscenariopars=103$.
For the weighting of the scenario parameters, we used:
\begin{equation*}
	\weight{1}=\weight{3}=\ldots=\weight{99}\approx 0.00349,
\end{equation*}
\begin{equation*}
	\weight{2}=\weight{4}=\ldots=\weight{100}\approx 0.0116,
\end{equation*}
\begin{equation*}
	\weight{101}\approx 0.571, \weight{102}\approx 0.231, \weight{103}\approx 0.0672.
\end{equation*}



\subsection{Generating scenarios}
\label{sec:case study parameter reduction}

In our case study, we used a bandwidth matrix of the form $\bandwidthmatrix=\bandwidth^2 \identitymatrix{\dimension}$, where $\identitymatrix{\dimension}$ denotes the $\dimension$-by-$\dimension$ identity matrix.
The bandwidth $\bandwidth$ was determined with leave-one-out cross-validation \autocite{duin1976parzen}, because this minimizes the Kullback-Leibler divergence between the real \ac{pdf} and the estimated \ac{pdf} \autocite{turlach1993bandwidthselection, zambom2013review}.

An important parameter for the generation of new scenarios is the number of reduced parameters ($\dimension$).
One approach is to look at the ``explained variance'' of \cref{eq:explained variance} of the first $\dimension$ singular values, see \cref{tab:explained variance}.
Because the first 2 singular values already explain 99.8 \% of the variance for the lead-vehicle-decelerating scenarios, we selected $\dimension=2$.
In \cref{fig:speed lvd generated}, the speed of the lead vehicle of 100 generated lead-vehicle-decelerating scenarios are shown.

\setlength{\figurewidth}{.97\linewidth}
\setlength{\figureheight}{.7\figurewidth}
\begin{figure}
	\centering
	% This file was created by tikzplotlib v0.9.8.
\begin{tikzpicture}

\begin{axis}[
height=\figureheight,
scaled y ticks=false,
tick align=outside,
tick pos=left,
width=\figurewidth,
x grid style={white!69.0196078431373!black},
xlabel={Time [s]},
xmin=-1.1368993993994, xmax=23.8748873873873,
xtick style={color=black},
xticklabel style={align=center},
y grid style={white!69.0196078431373!black},
ylabel={Speed [km/h]},
ymin=3.55360828236207, ymax=135.7515143592,
ytick style={color=black},
yticklabel style={/pgf/number format/fixed,/pgf/number format/precision=3}
]
\addplot [semithick, white!20!black]
table {%
0 73.5720451875602
0.115510204081639 73.5069699786363
0.231020408163278 73.4298403294583
0.346530612244917 73.3405446588219
0.462040816326556 73.2390615712309
0.577551020408195 73.1253696711893
0.693061224489834 72.9994475632007
0.808571428571473 72.8612738517685
0.924081632653112 72.7108271413976
1.03959183673475 72.5480860365914
1.15510204081639 72.3730291418536
1.27061224489803 72.1856350616882
1.38612244897967 71.9858824005989
1.50163265306131 71.7737497630895
1.61714285714295 71.549215753664
1.73265306122458 71.3122589768262
1.84816326530622 71.0628580370799
1.96367346938786 70.8009915389289
2.0791836734695 70.5266380868772
2.19469387755114 70.2397762854273
2.31020408163278 69.9403847390853
2.42571428571442 69.6284420523541
2.54122448979606 69.3039268297374
2.6567346938777 68.9668176757392
2.77224489795934 68.6170931948631
2.88775510204097 68.2547319916131
3.00326530612261 67.879712670493
3.11877551020425 67.4920138360067
3.23428571428589 67.0916140926579
3.34979591836753 66.6784920449506
3.46530612244917 66.2526262973886
3.58081632653081 65.8139954544739
3.69632653061245 65.3625781207138
3.81183673469409 64.8983529006105
3.92734693877572 64.4212983986679
4.04285714285736 63.9313932193897
4.158367346939 63.4286198338727
4.27387755102064 62.9131453232066
4.38938775510228 62.3853985579746
4.50489795918392 61.845828304056
4.62040816326556 61.2948833273301
4.7359183673472 60.7330123936763
4.85142857142884 60.1606642689739
4.96693877551048 59.5782877190999
5.08244897959211 58.9863315099382
5.19795918367375 58.385244407366
5.31346938775539 57.7754751772624
5.42897959183703 57.1574725855069
5.54448979591867 56.5316853979788
5.66000000000031 55.8985623805574
};
\addplot [semithick, white!20!black]
table {%
0 48.7902510131782
0.109591836734694 48.6303546449996
0.219183673469388 48.4714575631724
0.328775510204082 48.3130136689217
0.438367346938776 48.1544768634724
0.54795918367347 47.9953010480497
0.657551020408164 47.8349401238785
0.767142857142858 47.6728479921838
0.876734693877552 47.5084785541908
0.986326530612246 47.3412857111243
1.09591836734694 47.1707233642096
1.20551020408163 46.9962454146715
1.31510204081633 46.8173057637351
1.42469387755102 46.6333583126254
1.53428571428572 46.4438569625675
1.64387755102041 46.2482556147864
1.7534693877551 46.0460081705071
1.8630612244898 45.8365685309547
1.97265306122449 45.6195828438909
2.08224489795919 45.3956275104023
2.19183673469388 45.1655599248898
2.30142857142857 44.9302375411279
2.41102040816327 44.690517812891
2.52061224489796 44.4472581939536
2.63020408163266 44.20131613809
2.73979591836735 43.9535490990746
2.84938775510204 43.704814530682
2.95897959183674 43.4559698866865
3.06857142857143 43.2078726208626
3.17816326530613 42.9613801869846
3.28775510204082 42.7173500388269
3.39734693877551 42.4766396301641
3.50693877551021 42.2401064147705
3.6165306122449 42.0086078464205
3.72612244897959 41.7830013788885
3.83571428571429 41.5641444659491
3.94530612244898 41.3528945135279
4.05489795918368 41.1496232999591
4.16448979591837 40.9530221000339
4.27408163265306 40.7614189332101
4.38367346938776 40.5731418189458
4.49326530612245 40.3865187766989
4.60285714285715 40.1998778259274
4.71244897959184 40.0115469860892
4.82204081632653 39.8198542766423
4.93163265306123 39.6231277170447
5.04122448979592 39.4196953267543
5.15081632653062 39.207885125229
5.26040816326531 38.9860251319269
5.37 38.7524433663058
};
\addplot [semithick, white!20!black]
table {%
0 36.679786964858
0.163469387755102 36.5951013909426
0.326938775510204 36.5349623497997
0.490408163265306 36.4911986192792
0.653877551020407 36.455638977231
0.817346938775509 36.4201122015052
0.980816326530611 36.3764470699517
1.14428571428571 36.3164723604203
1.30775510204081 36.2320168507612
1.47122448979592 36.1149093188243
1.63469387755102 35.9569985428384
1.79816326530612 35.7543733813568
1.96163265306122 35.5125828721479
2.12510204081632 35.2384560772284
2.28857142857143 34.9388220586157
2.45204081632653 34.6205098783266
2.61551020408163 34.2903485983783
2.77897959183673 33.9551672807876
2.94244897959183 33.6217949875718
3.10591836734694 33.2970607807479
3.26938775510204 32.9876841324202
3.43285714285714 32.6961242068378
3.59632653061224 32.4193058776593
3.75979591836734 32.1537841525885
3.92326530612244 31.8961140393287
4.08673469387755 31.6428505455836
4.25020408163265 31.3905486790565
4.41367346938775 31.135763447451
4.57714285714285 30.8750498584706
4.74061224489795 30.6049629198189
4.90408163265306 30.3224058515463
5.06755102040816 30.0294921621558
5.23102040816326 29.732346250518
5.39448979591836 29.4371956895326
5.55795918367346 29.1502680520992
5.72142857142857 28.8777909111171
5.88489795918367 28.6259918394861
6.04836734693877 28.4010984101055
6.21183673469387 28.2093381958751
6.37530612244897 28.0569387696942
6.53877551020407 27.9481955748432
6.70224489795918 27.8731244091339
6.86571428571428 27.8153408910142
7.02918367346938 27.7584304494064
7.19265306122448 27.6859785132332
7.35612244897958 27.5815705114169
7.51959183673469 27.4287918728802
7.68306122448979 27.2112280265452
7.84653061224489 26.9124644013349
8.00999999999999 26.5160864261712
};
\addplot [semithick, white!20!black]
table {%
0 99.5374444092286
0.129813486956339 99.5311008183187
0.259626973912678 99.5087663963913
0.389440460869017 99.4650535154268
0.519253947825356 99.3945745474057
0.649067434781695 99.2919418643085
0.778880921738034 99.1524512411667
0.908694408694373 98.9772467720188
1.03850789565071 98.7704420194526
1.16832138260705 98.5361738668855
1.29813486956339 98.2785791977372
1.42794835651973 98.001794895425
1.55776184347607 97.709957843367
1.68757533043241 97.4072049249803
1.81738881738875 97.0976730236851
1.94720230434509 96.7854990228984
2.07701579130142 96.4748198060374
2.20682927825776 96.1695134979465
2.3366427652141 95.8716429167676
2.46645625217044 95.5824937486758
2.59626973912678 95.3033487316164
2.72608322608312 95.0354906035376
2.85589671303946 94.7802021023846
2.9857101999958 94.5387659661029
3.11552368695214 94.3124649326404
3.24533717390848 94.1025817399425
3.37515066086481 93.9103991259554
3.50496414782115 93.7371998286249
3.63477763477749 93.5838017664109
3.76459112173383 93.4483296388583
3.89440460869017 93.3279404705572
4.02421809564651 93.219790022983
4.15403158260285 93.1210340576096
4.28384506955919 93.0288283359115
4.41365855651553 92.9403286193628
4.54347204347187 92.8526906694384
4.6732855304282 92.7630702476126
4.80309901738454 92.6686231153595
4.93291250434088 92.566505034154
5.06272599129722 92.454718217194
5.19253947825356 92.335335017215
5.3223529652099 92.2116496341041
5.45216645216624 92.0869565026023
5.58197993912258 91.9645500574498
5.71179342607892 91.8477247333864
5.84160691303526 91.7397749651533
5.97142039999159 91.6439951874904
6.10123388694793 91.5636798351382
6.23104737390427 91.5021233428367
6.36086086086061 91.4626201453267
};
\addplot [semithick, white!20!black]
table {%
0 105.298772401465
0.149387755102035 105.046297920715
0.29877551020407 104.856180578652
0.448163265306105 104.715498997485
0.597551020408139 104.611331799425
0.746938775510174 104.530757606681
0.896326530612209 104.460855041462
1.04571428571424 104.388702725979
1.19510204081628 104.30137928244
1.34448979591831 104.185963333056
1.49387755102035 104.029559283684
1.64326530612238 103.824737673606
1.79265306122442 103.576264707715
1.94204081632645 103.29055674439
2.09142857142849 102.974030142012
2.24081632653052 102.63310125896
2.39020408163256 102.274186453613
2.53959183673459 101.90370208435
2.68897959183663 101.528064509552
2.83836734693866 101.153690087596
2.9877551020407 100.786919338485
3.13714285714273 100.431144565169
3.28653061224477 100.085928232383
3.4359183673468 99.7505768503253
3.58530612244884 99.424396929195
3.73469387755087 99.1066949791911
3.88408163265291 98.7967775105123
4.03346938775494 98.4939510333577
4.18285714285698 98.1975220579262
4.33224489795901 97.9067970944166
4.48163265306105 97.621190419245
4.63102040816308 97.3417288107397
4.78040816326512 97.0706803543978
4.92979591836715 96.8103450664449
5.07918367346919 96.5630229631072
5.22857142857122 96.331014060611
5.37795918367325 96.1166183751827
5.52734693877529 95.9221359230485
5.67673469387733 95.7498667204349
5.82612244897936 95.602110783568
5.97551020408139 95.4802647993052
6.12489795918343 95.3790492858848
6.27428571428546 95.2901924830103
6.4236734693875 95.2054085158631
6.57306122448953 95.1164115096258
6.72244897959157 95.0149155894799
6.8718367346936 94.8926348806073
7.02122448979564 94.7412835081899
7.17061224489767 94.5525755974096
7.31999999999971 94.3182252734482
};
\addplot [semithick, white!20!black]
table {%
0 76.6843818831066
0.0257142857142902 76.656370782497
0.0514285714285803 76.6280862038097
0.0771428571428705 76.5995313358053
0.102857142857161 76.5707093672458
0.128571428571451 76.5416234868931
0.154285714285741 76.5122768835078
0.180000000000031 76.4826727458522
0.205714285714321 76.452814262687
0.231428571428612 76.4227046227739
0.257142857142902 76.3923470148753
0.282857142857192 76.3617446277512
0.308571428571482 76.3309006501638
0.334285714285772 76.2998182708752
0.360000000000062 76.2685006786458
0.385714285714353 76.2369510622379
0.411428571428643 76.2051726104121
0.437142857142933 76.1731685119305
0.462857142857223 76.1409419555549
0.488571428571513 76.1084961300459
0.514285714285803 76.0758342241653
0.540000000000094 76.0429594266754
0.565714285714384 76.0098749263364
0.591428571428674 75.9765839119102
0.617142857142964 75.9430895721592
0.642857142857254 75.9093950958435
0.668571428571544 75.8755036717257
0.694285714285835 75.8414184885663
0.720000000000125 75.8071427351272
0.745714285714415 75.7726796001706
0.771428571428705 75.7380322724566
0.797142857142995 75.7032039407475
0.822857142857285 75.6681977938052
0.848571428571576 75.6330170203901
0.874285714285866 75.5976648092641
0.900000000000156 75.5621443491896
0.925714285714446 75.5264588289266
0.951428571428736 75.4906114372379
0.977142857143026 75.4546053628837
1.00285714285732 75.4184437946261
1.02857142857161 75.3821299212272
1.0542857142859 75.3456669314473
1.08000000000019 75.3090580140484
1.10571428571448 75.2723063577928
1.13142857142877 75.2354151514405
1.15714285714306 75.1983875837543
1.18285714285735 75.1612268434945
1.20857142857164 75.1239361194231
1.23428571428593 75.0865186003023
1.26000000000022 75.0489774748923
};
\addplot [semithick, white!20!black]
table {%
0 64.8023849565924
0.0722994422994445 64.7883443192993
0.144598884598889 64.7487220147258
0.216898326898333 64.6856288745591
0.289197769197778 64.6011757304864
0.361497211497222 64.4974734141943
0.433796653796667 64.37663275737
0.506096096096111 64.2407645917006
0.578395538395556 64.091979748873
0.650694980695 63.9323890605743
0.722994422994445 63.7641033584921
0.795293865293889 63.5892334743122
0.867593307593334 63.4098902397225
0.939892749892778 63.2281844864097
1.01219219219222 63.0462270460609
1.08449163449167 62.8661287503633
1.15679107679111 62.6900004310043
1.22909051909056 62.5199529196698
1.30138996139 62.3580970480475
1.37368940368944 62.206080007549
1.44598884598889 62.0635930264027
1.51828828828833 61.929811620639
1.59058773058778 61.8039113035566
1.66288717288722 61.685067588453
1.73518661518667 61.5724559886269
1.80748605748611 61.4652520173768
1.87978549978556 61.3626311880008
1.952084942085 61.2637690137974
2.02438438438445 61.1678410080652
2.09668382668389 61.074022684102
2.16898326898333 60.9814895552063
2.24128271128278 60.8894171346764
2.31358215358222 60.7969809358108
2.38588159588167 60.7033564719078
2.45818103818111 60.6077192562659
2.53048048048056 60.5092448021829
2.60277992278 60.4071086229575
2.67507936507945 60.300486231888
2.74737880737889 60.1885531422727
2.81967824967833 60.0704848674099
2.89197769197778 59.9454569205984
2.96427713427722 59.8126448151357
3.03657657657667 59.6712240643205
3.10887601887611 59.5204120179942
3.18117546117556 59.3603849234592
3.253474903475 59.1922760719123
3.32577434577445 59.0172602203798
3.39807378807389 58.8365121258863
3.47037323037334 58.6512065454583
3.54267267267278 58.4625182361212
};
\addplot [semithick, white!20!black]
table {%
0 98.8598975707998
0.0950799779371126 98.8040575993935
0.190159955874225 98.6581201846343
0.285239933811338 98.4466227716915
0.38031991174845 98.1941028057342
0.475399889685563 97.9250977319318
0.570479867622675 97.6641449954533
0.665559845559788 97.43419314633
0.760639823496901 97.2391840234698
0.855719801434013 97.0705377604498
0.950799779371126 96.9194538300685
1.04587975730824 96.7771317051243
1.14095973524535 96.6347708584158
1.23603971318246 96.4835707627415
1.33111969111958 96.3147308908999
1.42619966905669 96.1194507156895
1.5212796469938 95.8889297099089
1.61635962493091 95.6143673463564
1.71143960286803 95.2902927544929
1.80651958080514 94.9247908159246
1.90159955874225 94.5293956833132
1.99667953667936 94.1156415100057
2.09175951461648 93.6950624493502
2.18683949255359 93.279192654695
2.2819194704907 92.8795662793885
2.37699944842781 92.5077174767789
2.47207942636493 92.1751804002145
2.56715940430204 91.8934892030438
2.66223938223915 91.6740363708842
2.75731936017626 91.5194692917039
2.85239933811338 91.4187376482412
2.94747931605049 91.3595959760012
3.0425592939876 91.3297988104895
3.13763927192472 91.3171006872109
3.23271924986183 91.3092561416706
3.32779922779894 91.2940197093739
3.42287920573605 91.2591459258259
3.51795918367317 91.1923893265317
3.61303916161028 91.0815044469966
3.70811913954739 90.9152714374495
3.8031991174845 90.6944784596886
3.89827909542162 90.4277014482213
3.99335907335873 90.1236479731931
4.08843905129584 89.791025604748
4.18351902923295 89.4385419130346
4.27859900717007 89.0749044681971
4.37367898510718 88.708820840381
4.46875896304429 88.3489985997317
4.5638389409814 88.0041453163949
4.65891891891852 87.6829685605162
};
\addplot [semithick, white!20!black]
table {%
0 91.3142464263241
0.140310106024391 91.04581425733
0.280620212048782 90.7886422837588
0.420930318073172 90.5441406006076
0.561240424097563 90.3137193028744
0.701550530121954 90.0987884855562
0.841860636146345 89.9007582436503
0.982170742170735 89.7209534623201
1.12248084819513 89.5589794382682
1.26279095421952 89.412845618686
1.40310106024391 89.2805010319932
1.5434111662683 89.1598947066098
1.68372127229269 89.048975670956
1.82403137831708 88.9456929534514
1.96434148434147 88.8479955825165
2.10465159036586 88.7538325865712
2.24496169639025 88.6611529940354
2.38527180241464 88.5679058333293
2.52558190843903 88.4720401328726
2.66589201446342 88.3715049210857
2.80620212048782 88.2642492263885
2.94651222651221 88.148222077201
3.0868223325366 88.0213725019432
3.22713243856099 87.8816495290349
3.36744254458538 87.7270021868966
3.50775265060977 87.5553795039482
3.64806275663416 87.3647305086096
3.78837286265855 87.1530042293004
3.92868296868294 86.9181496944416
4.06899307470733 86.6587797013147
4.20930318073172 86.3771992332606
4.34961328675611 86.0769895820705
4.4899233927805 85.7617332610046
4.63023349880489 85.4350127833252
4.77054360482929 85.1004106622928
4.91085371085368 84.7615094111685
5.05116381687807 84.4218915432133
5.19147392290246 84.0851395716877
5.33178402892685 83.7548360098539
5.47209413495124 83.4345633709724
5.61240424097563 83.1279041683042
5.75271434700002 82.8384409151104
5.89302445302441 82.5697561246517
6.0333345590488 82.3254323101899
6.17364466507319 82.1090519849856
6.31395477109758 81.9241976623
6.45426487712198 81.7744518553941
6.59457498314637 81.6633970775287
6.73488508917076 81.5946158419653
6.87519519519515 81.5716906619648
};
\addplot [semithick, white!20!black]
table {%
0 37.7067246562798
0.0120408163265336 37.6218650562152
0.0240816326530672 37.5334702887071
0.0361224489796007 37.4416989423712
0.0481632653061343 37.3467096058231
0.0602040816326679 37.2486608676822
0.0722448979592015 37.1477113165569
0.0842857142857351 37.0440195410664
0.0963265306122687 36.9377441298263
0.108367346938802 36.8290436714524
0.120408163265336 36.7180767545602
0.132448979591869 36.6050019677655
0.144489795918403 36.4899778996838
0.156530612244937 36.3731631389354
0.16857142857147 36.254716274127
0.180612244898004 36.1347958938786
0.192653061224537 36.013560586806
0.204693877551071 35.8911689415247
0.216734693877604 35.7677795466505
0.228775510204138 35.643550990799
0.240816326530672 35.5186418625858
0.252857142857205 35.3932107506314
0.264897959183739 35.2674162435419
0.276938775510272 35.1414169299377
0.288979591836806 35.0153713984345
0.30102040816334 34.8894382376479
0.313061224489873 34.7637760361937
0.325102040816407 34.6385433826873
0.33714285714294 34.5138988657446
0.349183673469474 34.3900010739858
0.361224489796007 34.2670085960173
0.373265306122541 34.1450800204593
0.385306122449075 34.0243739359275
0.397346938775608 33.9050489310376
0.409387755102142 33.7872635944053
0.421428571428675 33.6711765146462
0.433469387755209 33.5569462803759
0.445510204081743 33.4447314802144
0.457551020408276 33.3346907027687
0.46959183673481 33.2269825366589
0.481632653061343 33.1217655705006
0.493673469387877 33.0191983929094
0.50571428571441 32.919439592501
0.517755102040944 32.8226477578911
0.529795918367478 32.7289814776953
0.541836734694011 32.6385993405327
0.553877551020545 32.5516599350119
0.565918367347078 32.4683218497523
0.577959183673612 32.3887436733694
0.590000000000146 32.313083994479
};
\addplot [semithick, white!20!black]
table {%
0 92.5894501092809
0.0143032828747186 92.5849038993512
0.0286065657494372 92.5707231269776
0.0429098486241558 92.5467181657974
0.0572131314988745 92.5126993894504
0.0715164143735931 92.4684771715742
0.0858196972483117 92.413861885807
0.10012298012303 92.3486639057848
0.114426262997749 92.2726936051501
0.128729545872468 92.185761357539
0.143032828747186 92.0876775365899
0.157336111621905 91.9782525159373
0.171639394496623 91.8572966692265
0.185942677371342 91.7246203700923
0.200245960246061 91.5800634118801
0.214549243120779 91.4237626056137
0.228852525995498 91.2560269848151
0.243155808870216 91.0771676953306
0.257459091744935 90.8874958830113
0.271762374619654 90.6873226937018
0.286065657494372 90.476959273266
0.300368940369091 90.2567167675486
0.31467222324381 90.0269063224009
0.328975506118528 89.7878390836663
0.343278788993247 89.5398261972109
0.357582071867965 89.2831788088786
0.371885354742684 89.0182080645203
0.386188637617403 88.7452251099783
0.400491920492121 88.4645410911212
0.41479520336684 88.1764671537916
0.429098486241558 87.8813144438404
0.443401769116277 87.5793941071091
0.457705051990996 87.2712231617201
0.472008334865714 86.9587997280883
0.486311617740433 86.6447702671666
0.500614900615152 86.3317839803705
0.51491818348987 86.0224900691455
0.529221466364589 85.7195377349071
0.543524749239307 85.4255761790807
0.557828032114026 85.1432546030831
0.572131314988745 84.8752222083576
0.586434597863463 84.6241281963205
0.600737880738182 84.3926217683973
0.615041163612901 84.1833521260073
0.629344446487619 83.9989684705891
0.643647729362338 83.8421200035613
0.657951012237056 83.7154559263493
0.672254295111775 83.621625440376
0.686557577986494 83.5632777470732
0.700860860861212 83.5430620478625
};
\addplot [semithick, white!20!black]
table {%
0 125.94687062163
0.0262346019488883 125.945721304931
0.0524692038977765 125.942278653172
0.0787038058466648 125.936760007106
0.104938407795553 125.929382707484
0.131173009744441 125.920364095058
0.15740761169333 125.909921510579
0.183642213642218 125.898272294799
0.209876815591106 125.88563378847
0.236111417539994 125.872223332343
0.262346019488883 125.85825826717
0.288580621437771 125.843955933703
0.314815223386659 125.829533672693
0.341049825335547 125.815208824892
0.367284427284436 125.801198731052
0.393519029233324 125.787720731925
0.419753631182212 125.774990592809
0.4459882331311 125.763115580708
0.472222835079989 125.752026223343
0.498457437028877 125.741636797727
0.524692038977765 125.73186158087
0.550926640926653 125.722614849785
0.577161242875542 125.713810881486
0.60339584482443 125.705363952984
0.629630446773318 125.697188341292
0.655865048722207 125.689198323422
0.682099650671095 125.681308176386
0.708334252619983 125.673432177197
0.734568854568871 125.665484602867
0.760803456517759 125.657379730408
0.787038058466648 125.649031836834
0.813272660415536 125.640355199156
0.839507262364424 125.631264094386
0.865741864313313 125.621672799538
0.891976466262201 125.611495591623
0.918211068211089 125.600646747654
0.944445670159977 125.589040544642
0.970680272108866 125.576591259602
0.996914874057754 125.563213169544
1.02314947600664 125.548820551482
1.04938407795553 125.533327682427
1.07561867990442 125.516648839392
1.10185328185331 125.49869829939
1.1280878838022 125.479390339433
1.15432248575108 125.458639236532
1.18055708769997 125.436359267701
1.20679168964886 125.412464709952
1.23302629159775 125.386869840298
1.25926089354664 125.359491466397
1.28549549549552 125.330321536692
};
\addplot [semithick, white!20!black]
table {%
0 47.8436150642019
0.0128553042838775 47.8431585384769
0.025710608567755 47.8417758080661
0.0385659128516325 47.8395063584208
0.05142121713551 47.8363896749919
0.0642765214193875 47.8324652432306
0.077131825703265 47.8277725485882
0.0899871299871425 47.8223510765156
0.10284243427102 47.816240312464
0.115697738554898 47.8094797418846
0.128553042838775 47.8021088502285
0.141408347122653 47.7941671229468
0.15426365140653 47.7856940454907
0.167118955690408 47.7767291033111
0.179974259974285 47.7673117818594
0.192829564258163 47.7574815665866
0.20568486854204 47.747277942944
0.218540172825918 47.7367403963824
0.231395477109795 47.7259084123531
0.244250781393673 47.7148214763073
0.25710608567755 47.7035190736961
0.269961389961428 47.6920406899706
0.282816694245305 47.6804258105818
0.295671998529183 47.668713920981
0.30852730281306 47.6569445066193
0.321382607096938 47.6451570529479
0.334237911380815 47.6333910454178
0.347093215664693 47.6216859694801
0.35994851994857 47.6100813105859
0.372803824232448 47.5986165541867
0.385659128516325 47.5873311857331
0.398514432800203 47.5762646906766
0.41136973708408 47.5654565544681
0.424225041367958 47.5549462625589
0.437080345651835 47.5447733004002
0.449935649935713 47.5349771534428
0.46279095421959 47.5255973071381
0.475646258503468 47.5166732469371
0.488501562787345 47.5082444582911
0.501356867071223 47.500350426651
0.5142121713551 47.493030637468
0.527067475638978 47.4863245761933
0.539922779922855 47.480271728278
0.552778084206733 47.4749115791733
0.56563338849061 47.4702836143301
0.578488692774488 47.4664273191998
0.591343997058365 47.4633821591263
0.604199301342243 47.4611677885621
0.61705460562612 47.4597469496293
0.629909909909998 47.4590722275873
};
\addplot [semithick, white!20!black]
table {%
0 65.6331966427301
0.0591836734693873 65.5927452669665
0.118367346938775 65.551626912996
0.177551020408162 65.5098627817609
0.236734693877549 65.4674740742041
0.295918367346936 65.4244819912681
0.355102040816324 65.3809077338958
0.414285714285711 65.3367725030296
0.473469387755098 65.2920974996123
0.532653061224486 65.2469039245866
0.591836734693873 65.201212978895
0.65102040816326 65.1550458634803
0.710204081632647 65.108423779285
0.769387755102035 65.061367927252
0.828571428571422 65.0138995083238
0.887755102040809 64.966039723443
0.946938775510197 64.9178097735524
1.00612244897958 64.8692308595946
1.06530612244897 64.8203241825122
1.12448979591836 64.7711109432479
1.18367346938775 64.7216123427445
1.24285714285713 64.6718495819444
1.30204081632652 64.6218438617904
1.36122448979591 64.5716163832252
1.42040816326529 64.5211883471913
1.47959183673468 64.4705809546315
1.53877551020407 64.4198154064885
1.59795918367346 64.3689129037049
1.65714285714284 64.3178946472232
1.71632653061223 64.2667818379863
1.77551020408162 64.2155956769368
1.83469387755101 64.1643573650172
1.89387755102039 64.1130881031703
1.95306122448978 64.0618090923387
2.01224489795917 64.0105415334652
2.07142857142856 63.9593066274922
2.13061224489794 63.9081255753626
2.18979591836733 63.857019578019
2.24897959183672 63.8060098364039
2.3081632653061 63.7551175514602
2.36734693877549 63.7043639241303
2.42653061224488 63.6537701553572
2.48571428571427 63.6033574460832
2.54489795918365 63.5531469972511
2.60408163265304 63.5031600098037
2.66326530612243 63.4534176846834
2.72244897959182 63.403941222833
2.7816326530612 63.3547518251952
2.84081632653059 63.3058706927127
2.89999999999998 63.2573190263279
};
\addplot [semithick, white!20!black]
table {%
0 71.8096666962024
0.0355308369594101 71.7965465247852
0.0710616739188203 71.7576460430977
0.10659251087823 71.6944008227741
0.142123347837641 71.6082464354491
0.177654184797051 71.5006184527574
0.213185021756461 71.3729524463328
0.248715858715871 71.2266839878102
0.284246695675281 71.0632486488239
0.319777532634691 70.8840820010085
0.355308369594101 70.6906196159983
0.390839206553511 70.4842970654279
0.426370043512922 70.2665590823882
0.461900880472332 70.0394581868344
0.497431717431742 69.8060239871629
0.532962554391152 69.5693743728663
0.568493391350562 69.3326272334367
0.604024228309972 69.0989004583658
0.639555065269382 68.8713119371458
0.675085902228793 68.6529795592686
0.710616739188203 68.4470212142261
0.746147576147613 68.256554791511
0.781678413107023 68.084698180614
0.817209250066433 67.9345692710276
0.852740087025843 67.8092859522439
0.888270923985253 67.7112061125669
0.923801760944663 67.6383145437213
0.959332597904074 67.5870350257448
0.994863434863484 67.5537894003946
1.03039427182289 67.5349995094284
1.0659251087823 67.5270871946036
1.10145594574171 67.5264742976778
1.13698678270112 67.5295826604084
1.17251761966053 67.532834124553
1.20804845661994 67.532650531869
1.24357929357935 67.525453724114
1.27911013053876 67.5076655430454
1.31464096749817 67.4757217484703
1.35017180445759 67.4264905370675
1.385702641417 67.3573460296773
1.42123347837641 67.2656910762341
1.45676431533582 67.1489285266724
1.49229515229523 67.0044612309267
1.52782598925464 66.8296920389314
1.56335682621405 66.6220238006212
1.59888766317346 66.3788593659311
1.63441850013287 66.0976015847944
1.66994933709228 65.775653307146
1.70548017405169 65.4104173829206
1.7410110110111 64.9992966620525
};
\addplot [semithick, white!20!black]
table {%
0 109.832368863295
0.121298441298445 109.807189540695
0.24259688259689 109.737243549087
0.363895323895335 109.628415901701
0.48519376519378 109.486591611765
0.606492206492225 109.31765569251
0.72779064779067 109.127493157165
0.849089089089115 108.921944220975
0.97038753038756 108.705679709589
1.09168597168601 108.482118167633
1.21298441298445 108.25461673908
1.3342828542829 108.026532567905
1.45558129558134 107.80122279808
1.57687973687979 107.582044573583
1.69817817817823 107.372355038384
1.81947661947668 107.175511336459
1.94077506077512 106.994870611782
2.06207350207357 106.833790008327
2.18337194337201 106.694780703077
2.30467038467046 106.575103413782
2.4259688259689 106.470007394874
2.54726726726735 106.374737805819
2.66856570856579 106.28453980608
2.78986414986424 106.194658555123
2.91116259116268 106.100339212413
3.03246103246113 105.996826937414
3.15375947375957 105.87936688959
3.27505791505802 105.743204228408
3.39635635635646 105.583607456264
3.51765479765491 105.399075000651
3.63895323895335 105.194609112695
3.7602516802518 104.975992547285
3.88155012155024 104.749008059306
4.00284856284869 104.519438403644
4.12414700414713 104.293066335185
4.24544544544558 104.075674608817
4.36674388674402 103.873045979423
4.48804232804247 103.690963201894
4.60934076934091 103.535209031112
4.73063921063936 103.41093140868
4.8519376519378 103.314440065205
4.97323609323625 103.235563787814
5.09453453453469 103.163982272483
5.21583297583314 103.089375215186
5.33713141713158 103.001422311897
5.45842985843003 102.889803258593
5.57972829972847 102.744197751246
5.70102674102692 102.554285485833
5.82232518232536 102.309746158327
5.94362362362381 102.000259464704
};
\addplot [semithick, white!20!black]
table {%
0 57.2877202719939
0.0655537169822848 57.2805460881481
0.13110743396457 57.2586153448431
0.196661150946854 57.2215850588022
0.262214867929139 57.1691250923336
0.327768584911424 57.101561995897
0.393322301893709 57.020179522178
0.458876018875994 56.9263370602266
0.524429735858278 56.821393999091
0.589983452840563 56.7067097278213
0.655537169822848 56.5836436354668
0.721090886805133 56.4535551110772
0.786644603787418 56.3178035437017
0.852198320769702 56.17774832239
0.917752037751987 56.0347488361913
0.983305754734272 55.8901644741552
1.04885947171656 55.7452329195609
1.11441318869884 55.6006191265571
1.17996690568113 55.4568197307124
1.24552062266341 55.3143313448543
1.3110743396457 55.1736505818069
1.37662805662798 55.0352740543978
1.44218177361027 54.8996983754531
1.50773549059255 54.7674201577991
1.57328920757484 54.6389360142622
1.63884292455712 54.5147425576686
1.7043966415394 54.3953364008447
1.76995035852169 54.2811784153567
1.83550407550397 54.1720295886388
1.90105779248626 54.0670128355852
1.96661150946854 53.9652277252898
2.03216522645083 53.8657738268471
2.09771894343311 53.7677507093493
2.1632726604154 53.6702579418911
2.22882637739768 53.5723950935662
2.29438009437997 53.4732617334683
2.35993381136225 53.3719574306913
2.42548752834454 53.2675817543288
2.49104124532682 53.1592354209432
2.55659496230911 53.0468521025322
2.62214867929139 52.9326671143719
2.68770239627368 52.8193099831232
2.75325611325596 52.7094102354474
2.81880983023825 52.6055973980062
2.88436354722053 52.5105009974591
2.94991726420282 52.426750560468
3.0154709811851 52.3569756136939
3.08102469816739 52.3038056837979
3.14657841514967 52.2698702974411
3.21213213213196 52.2577989812846
};
\addplot [semithick, white!20!black]
table {%
0 60.7101412104466
0.0257142857142855 60.6631266403152
0.0514285714285711 60.6183519510634
0.0771428571428566 60.5757579005331
0.102857142857142 60.5352852465671
0.128571428571428 60.4968747470072
0.154285714285713 60.4604671596957
0.179999999999999 60.4260032424742
0.205714285714284 60.3934237531857
0.23142857142857 60.3626694496719
0.257142857142855 60.3336810897749
0.282857142857141 60.3063994313369
0.308571428571426 60.2807652322002
0.334285714285712 60.2567192502069
0.359999999999997 60.2342022431991
0.385714285714283 60.2131549690188
0.411428571428568 60.1935181855086
0.437142857142854 60.1752326505103
0.46285714285714 60.1582391218661
0.488571428571425 60.1424783574181
0.514285714285711 60.1278911150087
0.539999999999996 60.1144181524799
0.565714285714282 60.1020002276739
0.591428571428567 60.0905780984327
0.617142857142853 60.0800925225986
0.642857142857138 60.0704842580137
0.668571428571424 60.0616940625203
0.694285714285709 60.0536626939603
0.719999999999995 60.0463309101761
0.74571428571428 60.0396394690097
0.771428571428566 60.0335291283033
0.797142857142851 60.027940645899
0.822857142857137 60.022814779639
0.848571428571423 60.0180922873655
0.874285714285708 60.0137139269207
0.899999999999994 60.0096204561466
0.925714285714279 60.0057526328854
0.951428571428565 60.0020512149794
0.97714285714285 59.9984569602705
1.00285714285714 59.9949106266011
1.02857142857142 59.9913529718132
1.05428571428571 59.987724753749
1.07999999999999 59.9839667302507
1.10571428571428 59.9800196591603
1.13142857142856 59.9758243687383
1.15714285714285 59.9713347010156
1.18285714285713 59.9665326450989
1.20857142857142 59.9614038645957
1.23428571428571 59.9559340231139
1.25999999999999 59.9501087842608
};
\addplot [semithick, white!20!black]
table {%
0 53.0007479705202
0.0416416416416389 52.9018804633317
0.0832832832832778 52.7926371228523
0.124924924924917 52.6732309731792
0.166566566566556 52.5438750384062
0.208208208208195 52.4047823426265
0.249849849849833 52.256165909938
0.291491491491472 52.098238764434
0.333133133133111 51.9312139302075
0.37477477477475 51.755304431357
0.416416416416389 51.5707232919755
0.458058058058028 51.3776835361558
0.499699699699667 51.1763981879968
0.541341341341306 50.9670802715913
0.582982982982945 50.7499428110318
0.624624624624584 50.5251988304177
0.666266266266223 50.2930613538417
0.707907907907862 50.0537434053959
0.7495495495495 49.8074580091803
0.791191191191139 49.5544181892873
0.832832832832778 49.2948369698087
0.874474474474417 49.0289273748451
0.916116116116056 48.7569024284885
0.957757757757695 48.4789751548306
0.999399399399334 48.1953585779724
1.04104104104097 47.9062694808629
1.08268268268261 47.6123179580468
1.12432432432425 47.3148469412517
1.16596596596589 47.0152793853569
1.20760760760753 46.7150382452316
1.24924924924917 46.4155464757452
1.29089089089081 46.118227031777
1.33253253253245 45.8245028681963
1.37417417417408 45.5357969398725
1.41581581581572 45.2535322016847
1.45745745745736 44.9791316085022
1.499099099099 44.7140181151949
1.54074074074064 44.4596146766412
1.58238238238228 44.2173442477108
1.62402402402392 43.9886297832738
1.66566566566556 43.774894238208
1.7073073073072 43.5775605673833
1.74894894894883 43.3980517256706
1.79059059059047 43.2377906679463
1.83223223223211 43.0982003490811
1.87387387387375 42.9807037239465
1.91551551551539 42.8867237474176
1.95715715715703 42.8176833743656
1.99879879879867 42.7750055596629
2.04044044044031 42.7601132581831
};
\addplot [semithick, white!20!black]
table {%
0 41.0279589339943
0.0375510204081662 40.7969701320388
0.0751020408163325 40.567655717845
0.112653061224499 40.3399883294824
0.150204081632665 40.1139406050257
0.187755102040831 39.8894851825444
0.225306122448997 39.6665947001132
0.262857142857164 39.4452417958016
0.30040816326533 39.2253991076844
0.337959183673496 39.0070392738309
0.375510204081662 38.7901349323161
0.413061224489829 38.5746587212092
0.450612244897995 38.3605832785852
0.488163265306161 38.1478812425134
0.525714285714327 37.9365252510686
0.563265306122493 37.7264879423203
0.60081632653066 37.5177419543432
0.638367346938826 37.3102599252068
0.675918367346992 37.1040144929845
0.713469387755158 36.8989782957497
0.751020408163325 36.695123971572
0.788571428571491 36.4924241585259
0.826122448979657 36.2908514946812
0.863673469387823 36.0903786181123
0.90122448979599 35.8909781668891
0.938775510204156 35.6926227790858
0.976326530612322 35.4952850927724
1.01387755102049 35.2989377460232
1.05142857142865 35.1035533769081
1.08897959183682 34.9091046235014
1.12653061224499 34.715564123873
1.16408163265315 34.5229045160972
1.20163265306132 34.331098438244
1.23918367346949 34.1401185283864
1.27673469387765 33.9499374245979
1.31428571428582 33.7605277649482
1.35183673469398 33.5718621875117
1.38938775510215 33.3839133303582
1.42693877551032 33.1966538315622
1.46448979591848 33.0100563291934
1.50204081632665 32.8240934613262
1.53959183673482 32.6387378660305
1.57714285714298 32.4539621813806
1.61469387755115 32.2697390454465
1.65224489795931 32.0860410963024
1.68979591836748 31.9028409720182
1.72734693877565 31.7201113106683
1.76489795918381 31.5378247503225
1.80244897959198 31.3559539290552
1.84000000000015 31.1744714849362
};
\addplot [semithick, white!20!black]
table {%
0 85.7482793719989
0.226563706563699 85.703841596462
0.453127413127397 85.580741234861
0.679691119691096 85.392651974346
0.906254826254795 85.1532475020677
1.13281853281849 84.8762015051777
1.35938223938219 84.5751876708253
1.58594594594589 84.2638796861624
1.81250965250959 83.9559512383386
2.03907335907329 83.6650757952101
2.26563706563699 83.4015132989742
2.49220077220069 83.1634729722355
2.71876447876438 82.946506099946
2.94532818532808 82.7461639670576
3.17189189189178 82.5579978585212
3.39845559845548 82.377559059289
3.62501930501918 82.2003988543119
3.85158301158288 82.0220685285417
4.07814671814658 81.8381193669304
4.30471042471027 81.6441026870482
4.53127413127397 81.4367377460401
4.75783783783767 81.2169974372214
4.98440154440137 80.9868227358916
5.21096525096507 80.7481546173517
5.43752895752877 80.5029340569025
5.66409266409247 80.2531020298436
5.89065637065617 80.0005995114764
6.11722007721986 79.7473674771003
6.34378378378356 79.4953469020163
6.57034749034726 79.2464787582743
6.79691119691096 79.0023229125258
7.02347490347466 78.7630072341737
7.25003861003836 78.5283233144987
7.47660231660206 78.2980627447804
7.70316602316575 78.0720171162996
7.92972972972945 77.8499780203373
8.15629343629315 77.6317370481728
8.38285714285685 77.4170857910875
8.60942084942055 77.205815840361
8.83598455598425 76.9977187875849
9.06254826254795 76.7928689778903
9.28911196911164 76.5924369353901
9.51567567567534 76.3978587867634
9.74223938223904 76.2105706586891
9.96880308880274 76.0320086778454
10.1953667953664 75.8636089709114
10.4219305019301 75.7068076645653
10.6484942084938 75.563040885486
10.8750579150575 75.4337447603524
11.1016216216212 75.3203554158427
};
\addplot [semithick, white!20!black]
table {%
0 90.0162987559067
0.288571428571426 89.7826618259958
0.577142857142852 89.5735100857119
0.865714285714278 89.3826041196059
1.1542857142857 89.2037045122271
1.44285714285713 89.0305718481265
1.73142857142856 88.8569667118536
2.01999999999998 88.6766496879592
2.30857142857141 88.4833813609931
2.59714285714283 88.2709223155051
2.88571428571426 88.0330521755634
3.17428571428569 87.7675869428799
3.46285714285711 87.4813483108023
3.75142857142854 87.1823765017766
4.03999999999996 86.8787117382509
4.32857142857139 86.5783942426719
4.61714285714282 86.2894642374863
4.90571428571424 86.0199619451419
5.19428571428567 85.7779275880852
5.48285714285709 85.5714013887639
5.77142857142852 85.40824851286
6.05999999999995 85.2895287943303
6.34857142857137 85.2074617005159
6.6371428571428 85.1536758821771
6.92571428571422 85.1197999900748
7.21428571428565 85.0974626749693
7.50285714285708 85.0782925876214
7.7914285714285 85.0539183787913
8.07999999999993 85.0159686992398
8.36857142857135 84.9560721997274
8.65714285714278 84.8660555424528
8.9457142857142 84.7407082274315
9.23428571428563 84.5771005530969
9.52285714285706 84.3723614879389
9.81142857142848 84.1236200004464
10.0999999999999 83.8280050591094
10.3885714285713 83.4826456324172
10.6771428571428 83.0846706888583
10.9657142857142 82.6312091969237
11.2542857142856 82.1193901251012
11.542857142857 81.5486355902151
11.8314285714285 80.9353155084862
12.1199999999999 80.3033958499926
12.4085714285713 79.6768784152525
12.6971428571427 79.0797650047837
12.9857142857142 78.5360574191072
13.2742857142856 78.0697574587402
13.562857142857 77.7048669242032
13.8514285714284 77.465387616014
14.1399999999999 77.3753213346924
};
\addplot [semithick, white!20!black]
table {%
0 26.2300509819158
0.029795918367343 26.1652483397838
0.0595918367346861 26.1004029328041
0.0893877551020291 26.0355139864706
0.119183673469372 25.9705807262784
0.148979591836715 25.905602377723
0.178775510204058 25.8405781662981
0.208571428571401 25.7755073174994
0.238367346938744 25.7103890568207
0.268163265306087 25.6452226097575
0.29795918367343 25.5800072018048
0.327755102040773 25.5147420584566
0.357551020408116 25.449426405208
0.38734693877546 25.3840594675544
0.417142857142803 25.3186404709896
0.446938775510146 25.2531686410094
0.476734693877489 25.1876432031075
0.506530612244832 25.1220633827792
0.536326530612175 25.0564284055199
0.566122448979518 24.9907374968232
0.595918367346861 24.9249898821846
0.625714285714204 24.8591847870992
0.655510204081547 24.7933214370608
0.68530612244889 24.7273990575648
0.715102040816233 24.6614168741065
0.744897959183576 24.5953741121795
0.774693877550919 24.5292699972798
0.804489795918262 24.463103754901
0.834285714285605 24.3968746105385
0.864081632652948 24.3305817896876
0.893877551020291 24.264224517842
0.923673469387634 24.1978020204971
0.953469387754977 24.1313135231481
0.98326530612232 24.0647582512887
1.01306122448966 23.9981354304144
1.04285714285701 23.9314442860203
1.07265306122435 23.8646840436004
1.10244897959169 23.7978539286502
1.13224489795904 23.7309531666637
1.16204081632638 23.6639809831361
1.19183673469372 23.5969366035627
1.22163265306106 23.5298192534373
1.25142857142841 23.4626281582552
1.28122448979575 23.3953625435116
1.31102040816309 23.3280216347004
1.34081632653044 23.2606046573174
1.37061224489778 23.1931108368562
1.40040816326512 23.1255393988122
1.43020408163247 23.0578895686808
1.45999999999981 22.9901605719556
};
\addplot [semithick, white!20!black]
table {%
0 77.1404673439301
0.156530612244899 75.9947858904295
0.313061224489799 75.0279948584087
0.469591836734698 74.2218988660977
0.626122448979598 73.558302531731
0.782653061224497 73.0190104735419
0.939183673469397 72.5858273097618
1.0957142857143 72.2405576586244
1.2522448979592 71.9650061383629
1.4087755102041 71.7409773672094
1.56530612244899 71.550275963397
1.72183673469389 71.374706545159
1.87836734693879 71.1960737307275
2.03489795918369 70.9961821383362
2.19142857142859 70.7568363862172
2.34795918367349 70.4613341986153
2.50448979591839 70.1082547386741
2.66102040816329 69.7051323243256
2.81755102040819 69.2596147105216
2.97408163265309 68.7793496522143
3.13061224489799 68.2719849043537
3.28714285714289 67.7451682218918
3.44367346938779 67.2065473597808
3.60020408163269 66.6637700729706
3.75673469387759 66.1244841164133
3.91326530612249 65.5963372450612
4.06979591836739 65.0869772138639
4.22632653061229 64.6040517777739
4.38285714285718 64.155208691743
4.53938775510208 63.7479667486752
4.69591836734698 63.3849750365957
4.85244897959188 63.0626386211848
5.00897959183678 62.7769534068974
5.16551020408168 62.5239152981904
5.32204081632658 62.29952019952
5.47857142857148 62.0997640153418
5.63510204081638 61.9206426501123
5.79163265306128 61.7581520082877
5.94816326530618 61.6082879943237
6.10469387755108 61.467046512677
6.26122448979598 61.3304234678032
6.41775510204088 61.1944147641586
6.57428571428578 61.0550163061997
6.73081632653068 60.9082243056948
6.88734693877558 60.7503703618863
7.04387755102047 60.5787612237582
7.20040816326537 60.3908798823711
7.35693877551027 60.1842093287866
7.51346938775517 59.9562325540662
7.67000000000007 59.7044325492705
};
\addplot [semithick, white!20!black]
table {%
0 90.9807215627592
0.217721394864254 90.9446221473781
0.435442789728508 90.8390851835365
0.653164184592762 90.6732706498619
0.870885579457016 90.4563385249825
1.08860697432127 90.1974487875263
1.30632836918552 89.9057614161217
1.52404976404978 89.5904363893956
1.74177115891403 89.2606336859766
1.95949255377829 88.9253269350429
2.17721394864254 88.5881516933395
2.39493534350679 88.2467459577014
2.61265673837105 87.8984289167047
2.8303781332353 87.5405197589259
3.04809952809956 87.1703376729393
3.26582092296381 86.7852018473216
3.48354231782806 86.3824314706494
3.70126371269232 85.9593457314968
3.91898510755657 85.5132638184406
4.13670650242083 85.0415784416816
4.35442789728508 84.5458440973196
4.57214929214933 84.0339265298404
4.78987068701359 83.5142184013604
5.00759208187784 82.9951123739956
5.2253134767421 82.485001109859
5.44303487160635 81.9922772710669
5.6607562664706 81.5253335197345
5.87847766133486 81.0925625179776
6.09619905619911 80.7023569279097
6.31392045106337 80.3630764797477
6.53164184592762 80.0780500139735
6.74936324079187 79.8402153000872
6.96708463565613 79.6412248508694
7.18480603052038 79.4727311790997
7.40252742538464 79.3263867975584
7.62024882024889 79.1938442190248
7.83797021511314 79.066755956279
8.0556916099774 78.9367745221011
8.27341300484165 78.7955524292704
8.49113439970591 78.6347419077557
8.70885579457016 78.4457314381259
8.92657718943441 78.2191568351881
9.14429858429867 77.945520527711
9.36201997916292 77.6153249444647
9.57974137402718 77.2190725142198
9.79746276889143 76.7472656657442
10.0151841637557 76.1904068278085
10.2329055586199 75.5389984291823
10.4506269534842 74.783542898637
10.6683483483484 73.9145426649391
};
\addplot [semithick, white!20!black]
table {%
0 90.0162987559067
0.288571428571426 89.7826618259958
0.577142857142852 89.5735100857119
0.865714285714278 89.3826041196059
1.1542857142857 89.2037045122271
1.44285714285713 89.0305718481265
1.73142857142856 88.8569667118536
2.01999999999998 88.6766496879592
2.30857142857141 88.4833813609931
2.59714285714283 88.2709223155051
2.88571428571426 88.0330521755634
3.17428571428569 87.7675869428799
3.46285714285711 87.4813483108023
3.75142857142854 87.1823765017766
4.03999999999996 86.8787117382509
4.32857142857139 86.5783942426719
4.61714285714282 86.2894642374863
4.90571428571424 86.0199619451419
5.19428571428567 85.7779275880852
5.48285714285709 85.5714013887639
5.77142857142852 85.40824851286
6.05999999999995 85.2895287943303
6.34857142857137 85.2074617005159
6.6371428571428 85.1536758821771
6.92571428571422 85.1197999900748
7.21428571428565 85.0974626749693
7.50285714285708 85.0782925876214
7.7914285714285 85.0539183787913
8.07999999999993 85.0159686992398
8.36857142857135 84.9560721997274
8.65714285714278 84.8660555424528
8.9457142857142 84.7407082274315
9.23428571428563 84.5771005530969
9.52285714285706 84.3723614879389
9.81142857142848 84.1236200004464
10.0999999999999 83.8280050591094
10.3885714285713 83.4826456324172
10.6771428571428 83.0846706888583
10.9657142857142 82.6312091969237
11.2542857142856 82.1193901251012
11.542857142857 81.5486355902151
11.8314285714285 80.9353155084862
12.1199999999999 80.3033958499926
12.4085714285713 79.6768784152525
12.6971428571427 79.0797650047837
12.9857142857142 78.5360574191072
13.2742857142856 78.0697574587402
13.562857142857 77.7048669242032
13.8514285714284 77.465387616014
14.1399999999999 77.3753213346924
};
\addplot [semithick, white!20!black]
table {%
0 44.8599713350951
0.0526530612244883 44.5743026809186
0.105306122448977 44.2874488728645
0.157959183673465 43.9995258962801
0.210612244897953 43.7106497365092
0.263265306122442 43.4209363788957
0.31591836734693 43.1305018087871
0.368571428571418 42.8394620115271
0.421224489795906 42.5479329724607
0.473877551020395 42.2560306769319
0.526530612244883 41.9638711102879
0.579183673469371 41.6715702578728
0.63183673469386 41.3792441050301
0.684489795918348 41.0870086371075
0.737142857142836 40.7949798394488
0.789795918367325 40.5032736973976
0.842448979591813 40.2120061963016
0.895102040816301 39.9212933215046
0.94775510204079 39.6312510583501
1.00040816326528 39.3419953921859
1.05306122448977 39.0536423083558
1.10571428571425 38.7663077922046
1.15836734693874 38.4801078290762
1.21102040816323 38.1951584043182
1.26367346938772 37.9115755032742
1.31632653061221 37.6294751112882
1.3689795918367 37.3489732137076
1.42163265306118 37.0701857958761
1.47428571428567 36.7932288431376
1.52693877551016 36.5182183408396
1.57959183673465 36.2452702743258
1.63224489795914 35.9745006289414
1.68489795918363 35.7060253900301
1.73755102040811 35.4399605429393
1.7902040816326 35.1764220730129
1.84285714285709 34.9155259655947
1.89551020408158 34.6573882060321
1.94816326530607 34.4021106249819
2.00081632653056 34.1496907011097
2.05346938775504 33.9000792437563
2.10612244897953 33.6532268442882
2.15877551020402 33.409084094072
2.21142857142851 33.1676015844773
2.264081632653 32.9287299068706
2.31673469387749 32.6924196526196
2.36938775510197 32.4586214130908
2.42204081632646 32.2272857796539
2.47469387755095 31.9983633436753
2.52734693877544 31.7718046965217
2.57999999999993 31.5475604295627
};
\addplot [semithick, white!20!black]
table {%
0 39.93977894749
0.0504081632653113 39.7033606019852
0.100816326530623 39.4605856669462
0.151224489795934 39.2113905681033
0.201632653061245 38.9557117311853
0.252040816326557 38.6934855819246
0.302448979591868 38.4246485460503
0.352857142857179 38.1491370492925
0.403265306122491 37.8668875173812
0.453673469387802 37.5778363760466
0.504081632653113 37.2819200510175
0.554489795918425 36.9790749680266
0.604897959183736 36.6692375528028
0.655306122449047 36.3523442310761
0.705714285714358 36.0283314285767
0.75612244897967 35.6971355710347
0.806530612244981 35.3586930841787
0.856938775510292 35.0129403937418
0.907346938775604 34.6598139254526
0.957755102040915 34.2992501050413
1.00816326530623 33.931185358238
1.05857142857154 33.5555561107728
1.10897959183685 33.172298788374
1.15938775510216 32.7813498167753
1.20979591836747 32.3826456217051
1.26020408163278 31.9761226288934
1.31061224489809 31.5617172640704
1.36102040816341 31.1393659529662
1.41142857142872 30.7090051213089
1.46183673469403 30.2705711948326
1.51224489795934 29.8240005992655
1.56265306122465 29.3692297603376
1.61306122448996 28.9061951037791
1.66346938775527 28.43483305532
1.71387755102058 27.9550800406885
1.7642857142859 27.4668724856188
1.81469387755121 26.9701468158389
1.86510204081652 26.4648394570791
1.91551020408183 25.9508868350693
1.96591836734714 25.4282875732984
2.01632653061245 24.8975776432549
2.06673469387776 24.3595680058747
2.11714285714308 23.8150718448428
2.16755102040839 23.2649023438468
2.2179591836737 22.7098726865742
2.26836734693901 22.1507960567125
2.31877551020432 21.5884856379466
2.36918367346963 21.023754613969
2.41959183673494 20.4574161684648
2.47000000000025 19.8902834851214
};
\addplot [semithick, white!20!black]
table {%
0 52.4169596217138
0.23326530612245 52.2613975238401
0.466530612244901 52.0303607882166
0.699795918367351 51.735398097441
0.933061224489801 51.3880581341094
1.16632653061225 50.9998895808196
1.3995918367347 50.5824411201676
1.63285714285715 50.1472614347513
1.8661224489796 49.7058992071667
2.09938775510205 49.2699031200116
2.3326530612245 48.8508218558825
2.56591836734695 48.460204097376
2.7991836734694 48.1095985270898
3.03244897959185 47.81055382762
3.2657142857143 47.5746186815643
3.49897959183675 47.4125526192858
3.7322448979592 47.3222479373645
3.96551020408165 47.2911204090322
4.19877551020411 47.3062811389514
4.43204081632656 47.3548412317849
4.66530612244901 47.4239117921953
4.89857142857146 47.5006039248451
5.13183673469391 47.5720287343972
5.36510204081636 47.6252973255141
5.59836734693881 47.6475208028584
5.83163265306126 47.6258102710929
6.06489795918371 47.54727683488
6.29816326530616 47.3990315988826
6.53142857142861 47.1681856677631
6.76469387755106 46.8418617995045
6.99795918367351 46.4117766216555
7.23122448979596 45.8812123404008
7.46448979591841 45.2552427164701
7.69775510204086 44.5389415105907
7.93102040816331 43.7373824834927
8.16428571428576 42.8556393959032
8.39755102040821 41.8987860085525
8.63081632653066 40.8718960821673
8.86408163265311 39.7800433774782
9.09734693877556 38.6283016552118
9.33061224489801 37.4217446760988
9.56387755102046 36.1654462008669
9.79714285714291 34.8644799902433
10.0304081632654 33.5239198049592
10.2636734693878 32.1491336259033
10.4969387755103 30.7481738304702
10.7302040816327 29.3305291729927
10.9634693877552 27.9057019950836
11.1967346938776 26.4831946383503
11.4300000000001 25.0725094444055
};
\addplot [semithick, white!20!black]
table {%
0 117.455236768141
0.012925782925786 117.450648606469
0.025851565851572 117.43674468898
0.038777348777358 117.4123491251
0.051703131703144 117.376286024255
0.0646289146289301 117.32737949587
0.0775546975547161 117.264453649372
0.0904804804805021 117.186332594187
0.103406263406288 117.091840439739
0.116332046332074 116.979858157923
0.12925782925786 116.850362303004
0.142183612183646 116.704318649311
0.155109395109432 116.542728491603
0.168035178035218 116.366593124642
0.180960960961004 116.176913843187
0.19388674388679 115.974691941997
0.206812526812576 115.760928715834
0.219738309738362 115.536625459458
0.232664092664148 115.302783467628
0.245589875589934 115.060404035104
0.25851565851572 114.810488456647
0.271441441441506 114.554038027017
0.284367224367292 114.292054040973
0.297293007293078 114.025537793276
0.310218790218864 113.755490578687
0.32314457314465 113.482913691954
0.336070356070436 113.208808427859
0.348996138996222 112.934176081151
0.361921921922008 112.66001794659
0.374847704847794 112.387333577792
0.38777348777358 112.116947603167
0.400699270699366 111.849362630094
0.413625053625152 111.585046625179
0.426550836550938 111.324467555026
0.439476619476724 111.06809338624
0.45240240240251 110.816392085426
0.465328185328296 110.56983161919
0.478253968254082 110.328879954136
0.491179751179868 110.094005056869
0.504105534105654 109.865674893994
0.51703131703144 109.644357432116
0.529957099957226 109.430520637839
0.542882882883012 109.22463247777
0.555808665808799 109.027160918513
0.568734448734584 108.838573926673
0.581660231660371 108.659339468854
0.594586014586156 108.489925511662
0.607511797511943 108.330800021702
0.620437580437728 108.182430965578
0.633363363363515 108.045286309896
};
\addplot [semithick, white!20!black]
table {%
0 52.7283399147053
0.114291025719597 52.5938901711735
0.228582051439194 52.4552727155047
0.342873077158791 52.312849164538
0.457164102878388 52.1669857804113
0.571455128597985 52.0180488252632
0.685746154317582 51.8664045612317
0.800037180037179 51.7124192504564
0.914328205756777 51.5564591550755
1.02861923147637 51.3988905372267
1.14291025719597 51.24007965905
1.25720128291557 51.0803927826832
1.37149230863516 50.920196170265
1.48578333435476 50.7598560839332
1.60007436007436 50.5997387858277
1.71436538579396 50.4402105380864
1.82865641151355 50.2816376028472
1.94294743723315 50.12438624225
2.05723846295275 49.9688198655687
2.17152948867234 49.8151424020172
2.28582051439194 49.6633170848968
2.40011154011154 49.5132871837812
2.51440256583114 49.3649959682425
2.62869359155073 49.2183867078528
2.74298461727033 49.0734026721858
2.85727564298993 48.9299871308135
2.97156666870952 48.7880833533086
3.08585769442912 48.6476346092432
3.20014872014872 48.5085841681909
3.31443974586831 48.370875299724
3.42873077158791 48.2344512734143
3.54302179730751 48.0992553588357
3.65731282302711 47.9652308255601
3.7716038487467 47.8323209431604
3.8858948744663 47.7004689812084
4.0001859001859 47.569618209278
4.11447692590549 47.4399011495429
4.22876795162509 47.3122871007636
4.34305897734469 47.1879765547869
4.45735000306429 47.0681700112456
4.57164102878388 46.9540679697733
4.68593205450348 46.8468709300028
4.80022308022308 46.7477793915685
4.91451410594267 46.6579938541032
5.02880513166227 46.5787148172401
5.14309615738187 46.5111427806133
5.25738718310147 46.4564782438558
5.37167820882106 46.4159217066009
5.48596923454066 46.390673668482
5.60026026026026 46.3819346291328
};
\addplot [semithick, white!20!black]
table {%
0 63.200041209007
0.0324103695532274 63.1749118823594
0.0648207391064549 63.1000947822115
0.0972311086596823 62.9797115167275
0.12964147821291 62.8178836940701
0.162051847766137 62.6187329224055
0.194462217319365 62.3863808098969
0.226872586872592 62.1249489647083
0.25928295642582 61.8385589950037
0.291693325979047 61.5313325089471
0.324103695532274 61.2073911147003
0.356514065085502 60.8708564204318
0.388924434638729 60.5258500343035
0.421334804191957 60.1764935644794
0.453745173745184 59.8269086191235
0.486155543298412 59.4812168063999
0.518565912851639 59.1435397344703
0.550976282404867 58.8177522523336
0.583386651958094 58.5052308779588
0.615797021511321 58.2059002981879
0.648207391064549 57.9196672721986
0.680617760617776 57.6464385591685
0.713028130171004 57.3861209182735
0.745438499724231 57.1386211086948
0.777848869277459 56.9038458896082
0.810259238830686 56.6817020201913
0.842669608383914 56.4720962596216
0.875079977937141 56.2749353670769
0.907490347490368 56.0901261017336
0.939900717043596 55.9175752227717
0.972311086596823 55.7571894893676
1.00472145615005 55.6088756606989
1.03713182570328 55.4725404959433
1.06954219525651 55.3480907542783
1.10195256480973 55.2354331948809
1.13436293436296 55.1344735466123
1.16677330391619 55.0448782789049
1.19918367346942 54.9657694295127
1.23159404302264 54.8961936977314
1.26400441257587 54.8351977828566
1.2964147821291 54.781828384184
1.32882515168233 54.7351322010099
1.36123552123555 54.6941559326297
1.39364589078878 54.6579462783392
1.42605626034201 54.6255499374341
1.45846662989523 54.5960136092102
1.49087699944846 54.5683839929631
1.52328736900169 54.5417077879889
1.55569773855492 54.5150316935832
1.58810810810814 54.4874024090417
};
\addplot [semithick, white!20!black]
table {%
0 58.0705203232099
0.0416326530612261 58.0100067072057
0.0832653061224521 57.9494560566907
0.124897959183678 57.8889721413454
0.166530612244904 57.8286587308504
0.20816326530613 57.7686195948858
0.249795918367356 57.7089585031324
0.291428571428582 57.6497792252705
0.333061224489809 57.5911855309806
0.374693877551035 57.5332811899432
0.416326530612261 57.4761699718388
0.457959183673487 57.419955646348
0.499591836734713 57.3647419831509
0.541224489795939 57.3106327519282
0.582857142857165 57.2577317223603
0.624489795918391 57.2061426641278
0.666122448979617 57.1559693469111
0.707755102040843 57.1073155403906
0.749387755102069 57.0602850142469
0.791020408163295 57.0149815381604
0.832653061224521 56.9715088818115
0.874285714285747 56.929970804879
0.915918367346973 56.8904001583502
0.9575510204082 56.8525891053307
0.999183673469425 56.8162787891302
1.04081632653065 56.7812103530584
1.08244897959188 56.7471249404246
1.1240816326531 56.7137636945386
1.16571428571433 56.6808677587098
1.20734693877556 56.6481782762479
1.24897959183678 56.6154363904625
1.29061224489801 56.5823832446632
1.33224489795923 56.5487599821595
1.37387755102046 56.514307746261
1.41551020408169 56.4787676802773
1.45714285714291 56.441880927518
1.49877551020414 56.4033886312926
1.54040816326536 56.3630319349108
1.58204081632659 56.320551981682
1.62367346938782 56.2756899149162
1.66530612244904 56.2281868779225
1.70693877551027 56.1777840140106
1.74857142857149 56.1242224664902
1.79020408163272 56.0672433786709
1.83183673469395 56.0065878938621
1.87346938775517 55.9419971553736
1.9151020408164 55.8732123065151
1.95673469387763 55.7999744905957
1.99836734693885 55.7220248509252
2.04000000000008 55.6391045308133
};
\addplot [semithick, white!20!black]
table {%
0 46.4075398812719
0.086734693877551 46.3682812169819
0.173469387755102 46.326762778459
0.260204081632653 46.2826905499352
0.346938775510204 46.2357705156427
0.433673469387755 46.1857086598136
0.520408163265306 46.1322109666799
0.607142857142857 46.0749834204739
0.693877551020408 46.0137320054277
0.780612244897959 45.9481627057733
0.86734693877551 45.8779815057428
0.954081632653061 45.8028943895686
1.04081632653061 45.7226073414825
1.12755102040816 45.6368272204098
1.21428571428571 45.5453683463139
1.30102040816327 45.448254623248
1.38775510204082 45.3455341582379
1.47448979591837 45.2372550583096
1.56122448979592 45.1234654304887
1.64795918367347 45.0042133818011
1.73469387755102 44.8795470192728
1.82142857142857 44.7495144499293
1.90816326530612 44.6141637807964
1.99489795918367 44.4735431189003
2.08163265306122 44.3277005712663
2.16836734693878 44.1766842449205
2.25510204081633 44.0205422468887
2.34183673469388 43.8593226841968
2.42857142857143 43.6930736638703
2.51530612244898 43.5218432929352
2.60204081632653 43.3456796784174
2.68877551020408 43.1646309273425
2.77551020408163 42.9787451467364
2.86224489795918 42.7880704436251
2.94897959183673 42.592654925034
3.03571428571429 42.3925466979892
3.12244897959184 42.1877938695166
3.20918367346939 41.9784445466416
3.29591836734694 41.7645468363903
3.38265306122449 41.5461488457888
3.46938775510204 41.3232986818622
3.55612244897959 41.0960444516367
3.64285714285714 40.8644342621384
3.72959183673469 40.6285162203925
3.81632653061224 40.3883384334251
3.9030612244898 40.1439490082623
3.98979591836735 39.8953960519293
4.0765306122449 39.6427276714523
4.16326530612245 39.3859919738573
4.25 39.1252370661696
};
\addplot [semithick, white!20!black]
table {%
0 62.8504839931617
0.0889795918367327 62.7428389359766
0.177959183673465 62.6330938820112
0.266938775510198 62.5212812324767
0.355918367346931 62.4074333885845
0.444897959183663 62.2915827515456
0.533877551020396 62.1737617225712
0.622857142857129 62.0540027028727
0.711836734693861 61.9323380936612
0.800816326530594 61.8088002961479
0.889795918367327 61.6834217115441
0.978775510204059 61.556234741061
1.06775510204079 61.4272717859097
1.15673469387752 61.2965652473016
1.24571428571426 61.1641475264478
1.33469387755099 61.0300510245596
1.42367346938772 60.8943081428481
1.51265306122446 60.7569512825246
1.60163265306119 60.6180128448003
1.69061224489792 60.4775252308864
1.77959183673465 60.3355208419942
1.86857142857139 60.1920320793349
1.95755102040812 60.0470913441196
2.04653061224485 59.9007310375596
2.13551020408158 59.7529835608661
2.22448979591832 59.6038813152503
2.31346938775505 59.4534567019235
2.40244897959178 59.3017473155587
2.49142857142851 59.1488301679437
2.58040816326525 58.994800350688
2.66938775510198 58.8397530499162
2.75836734693871 58.6837834517529
2.84734693877544 58.5269867423229
2.93632653061218 58.3694581077507
3.02530612244891 58.2112927341609
3.11428571428564 58.0525858076784
3.20326530612238 57.8934325144277
3.29224489795911 57.7339280405334
3.38122448979584 57.5741675721202
3.47020408163257 57.4142462953129
3.55918367346931 57.2542593962359
3.64816326530604 57.094302061014
3.73714285714277 56.9344694757719
3.8261224489795 56.7748568266341
3.91510204081624 56.6155592997253
4.00408163265297 56.4566720811703
4.0930612244897 56.2982903570936
4.18204081632643 56.1405093136199
4.27102040816317 55.9834241368738
4.3599999999999 55.82713001298
};
\addplot [semithick, white!20!black]
table {%
0 89.2897734602013
0.137050519907661 89.2686328263344
0.274101039815322 89.2077410551013
0.411151559722983 89.1128698247283
0.548202079630644 88.9897908134411
0.685252599538305 88.844275699466
0.822303119445966 88.6820961610286
0.959353639353627 88.509023876355
1.09640415926129 88.3308305236712
1.23345467916895 88.1532877812031
1.37050519907661 87.9821673271768
1.50755571898427 87.8232408398182
1.64460623889193 87.6811996969126
1.78165675879959 87.5556367813455
1.91870727870725 87.4446422956554
2.05575779861491 87.3463062288993
2.19280831852258 87.2587185701341
2.32985883843024 87.1799693084167
2.4669093583379 87.1081484328039
2.60395987824556 87.0413459323526
2.74101039815322 86.9776517961196
2.87806091806088 86.9151560131619
3.01511143796854 86.8519485725361
3.1521619578762 86.7861194632993
3.28921247778386 86.7157586745081
3.42626299769152 86.6390116190532
3.56331351759918 86.5548310474703
3.70036403750685 86.4627801535492
3.83741455741451 86.3624372087729
3.97446507732217 86.2533804846247
4.11151559722983 86.135188252587
4.24856611713749 86.0074387841432
4.38561663704515 85.8697103507762
4.52266715695281 85.721581223969
4.65971767686047 85.5626296752046
4.79676819676813 85.392433975966
4.93381871667579 85.2105723977362
5.07086923658346 85.0166232119986
5.20791975649112 84.8101840732319
5.34497027639878 84.5928160385287
5.48202079630644 84.3697034914063
5.6190713162141 84.1464217802269
5.75612183612176 83.9285462533525
5.89317235602942 83.7216522591452
6.03022287593708 83.5313151459671
6.16727339584474 83.3631102621805
6.3043239157524 83.222612956147
6.44137443566007 83.115398576229
6.57842495556773 83.0470424707884
6.71547547547539 83.0231199881875
};
\addplot [semithick, white!20!black]
table {%
0 103.459094389727
0.11551020408163 103.247331928346
0.231020408163259 103.075581599288
0.346530612244889 102.93835874716
0.462040816326519 102.830178716572
0.577551020408148 102.74555685213
0.693061224489778 102.679008498444
0.808571428571408 102.625049000121
0.924081632653037 102.578193701768
1.03959183673467 102.532957947995
1.1551020408163 102.483857083409
1.27061224489793 102.425418348854
1.38612244897956 102.354316625359
1.50163265306119 102.271846495589
1.61714285714282 102.179901806373
1.73265306122445 102.080376404539
1.84816326530607 101.975164136914
1.9636734693877 101.866158850328
2.07918367346933 101.755254391607
2.19469387755096 101.644344607581
2.31020408163259 101.535323345078
2.42571428571422 101.430084450925
2.54122448979585 101.330473878203
2.65673469387748 101.236734047868
2.77224489795911 101.147162562934
2.88775510204074 101.059940077833
3.00326530612237 100.973247246995
3.118775510204 100.885264724851
3.23428571428563 100.794173165831
3.34979591836726 100.698153224366
3.46530612244889 100.595385554887
3.58081632653052 100.484050811825
3.69632653061215 100.362329649609
3.81183673469378 100.228496173629
3.92734693877541 100.082026630861
4.04285714285704 99.9232316102088
4.15836734693867 99.7524385019649
4.2738775510203 99.5699746964239
4.38938775510193 99.37616758388
4.50489795918356 99.1713445546273
4.62040816326519 98.9558329989599
4.73591836734682 98.7299603071711
4.85142857142845 98.4940538695569
4.96693877551008 98.2484410764105
5.08244897959171 97.993857384858
5.19795918367334 97.7335947925191
5.31346938775497 97.4719332330259
5.42897959183659 97.2131547638818
5.54448979591822 96.9615414425906
5.65999999999985 96.7213753266559
};
\addplot [semithick, white!20!black]
table {%
0 71.5758038765986
0.11408163265306 71.5533564180488
0.228163265306119 71.5248159751362
0.342244897959179 71.4889457312487
0.456326530612238 71.4445088697743
0.570408163265298 71.3902685741006
0.684489795918357 71.3249880276154
0.798571428571417 71.2474304137066
0.912653061224476 71.156358915762
1.02673469387754 71.0505367171694
1.1408163265306 70.9287315788518
1.25489795918366 70.7906816991346
1.36897959183671 70.6382904503618
1.48306122448977 70.473754167112
1.59714285714283 70.2992691839636
1.71122448979589 70.1170318354954
1.82530612244895 69.9292384562862
1.93938775510201 69.7380853809139
2.05346938775507 69.5457689439575
2.16755102040813 69.3544854799955
2.28163265306119 69.1663299020453
2.39571428571425 68.9794543599327
2.50979591836731 68.7868892146434
2.62387755102037 68.5813225293929
2.73795918367343 68.3554423673977
2.85204081632649 68.1019367918742
2.96612244897955 67.8134938660384
3.08020408163261 67.4828016531068
3.19428571428567 67.1025482162962
3.30836734693873 66.6654216188216
3.42244897959179 66.1648466994683
3.53653061224485 65.6052726425671
3.65061224489791 64.9996351954799
3.76469387755097 64.3610884094409
3.87877551020402 63.7027863356856
3.99285714285708 63.0378830254456
4.10693877551014 62.3795325299565
4.2210204081632 61.7408889004523
4.33510204081626 61.1351061881674
4.44918367346932 60.5753384443359
4.56326530612238 60.0728150044584
4.67734693877544 59.6245403518061
4.7914285714285 59.2211433487824
4.90551020408156 58.8532227841063
5.01959183673462 58.5113774464967
5.13367346938768 58.1862061246726
5.24775510204074 57.8683076073536
5.3618367346938 57.5482806832575
5.47591836734686 57.2167241411037
5.58999999999992 56.8642367696114
};
\addplot [semithick, white!20!black]
table {%
0 76.7645927330305
0.159183673469387 76.2922654699348
0.318367346938774 75.9366837572254
0.47755102040816 75.6727917396234
0.636734693877547 75.4755335618508
0.795918367346934 75.3198533686289
0.955102040816321 75.1806953046798
1.11428571428571 75.0330035147249
1.27346938775509 74.851722143486
1.43265306122448 74.6117953356844
1.59183673469387 74.2881672360424
1.75102040816325 73.8557819892809
1.91020408163264 73.2898481915921
2.06938775510203 72.583702558452
2.22857142857142 71.7601638908812
2.3877551020408 70.8447559722432
2.54693877551019 69.863002585904
2.70612244897958 68.840427515227
2.86530612244896 67.8025545435784
3.02448979591835 66.7749074543215
3.18367346938774 65.7830100308224
3.34285714285712 64.8523860564444
3.50204081632651 64.0085593145538
3.6612244897959 63.2770535885139
3.82040816326528 62.6813696448527
3.97959183673467 62.2181407924513
4.13877551020406 61.8649306891694
4.29795918367344 61.5988972212607
4.45714285714283 61.3971982749798
4.61632653061222 61.2369917365806
4.7755102040816 61.0954354923175
4.93469387755099 60.9496874284445
5.09387755102038 60.7769054312162
5.25306122448976 60.5542473868864
5.41224489795915 60.2588711817097
5.57142857142854 59.8679347019399
5.73061224489792 59.3625007361781
5.88979591836731 58.7419950482374
6.0489795918367 58.0112363315922
6.20816326530609 57.1750439994772
6.36734693877547 56.238237465129
6.52653061224486 55.205636141782
6.68571428571425 54.0820594426731
6.84489795918363 52.8723267810365
7.00408163265302 51.5812575701093
7.16326530612241 50.2136712231254
7.32244897959179 48.7743871533223
7.48163265306118 47.2682975353014
7.64081632653057 45.708169240076
7.79999999999995 44.1215869401168
};
\addplot [semithick, white!20!black]
table {%
0 65.0569800350152
0.174897959183674 64.7631607655319
0.349795918367349 64.5355961714279
0.524693877551023 64.3538087199994
0.699591836734698 64.1973208785428
0.874489795918372 64.0456551143544
1.04938775510205 63.8783338947305
1.22428571428572 63.6748796869674
1.3991836734694 63.4148149583615
1.57408163265307 63.0776621762093
1.74897959183674 62.6430080520463
1.92387755102042 62.1040590762283
2.09877551020409 61.4844092644951
2.27367346938777 60.8117642639285
2.44857142857144 60.1138297216102
2.62346938775512 59.4183112846219
2.79836734693879 58.7529146000453
2.97326530612247 58.1453453149621
3.14816326530614 57.623309076454
3.32306122448982 57.2145115316028
3.49795918367349 56.946151170694
3.67285714285716 56.8257107635619
3.84775510204084 56.8350616618346
4.02265306122451 56.9543635629527
4.19755102040819 57.1637761643571
4.37244897959186 57.4434591634883
4.54734693877554 57.7735722577876
4.72224489795921 58.1342751446955
4.89714285714289 58.5057275216528
5.07204081632656 58.8680890861003
5.24693877551023 59.2018482237978
5.42183673469391 59.492411471646
5.59673469387758 59.7289713690335
5.77163265306126 59.9008178444801
5.94653061224493 59.997240826506
6.12142857142861 60.0075302436308
6.29632653061228 59.9209760243747
6.47122448979596 59.7268680972576
6.64612244897963 59.4144963907994
6.8210204081633 58.97315083352
6.99591836734698 58.3944894303276
7.17081632653065 57.6876717506878
7.34571428571433 56.8697016171006
7.520612244898 55.9576198532609
7.69551020408168 54.9684672828634
7.87040816326535 53.9192847296026
8.04530612244903 52.8271130171731
8.2202040816327 51.7089929692694
8.39510204081638 50.5819654095862
8.57000000000005 49.463071161818
};
\addplot [semithick, white!20!black]
table {%
0 95.1059769163613
0.028548548548547 95.103927338339
0.057097097097094 95.0980642972316
0.0856456456456411 95.0885269949566
0.114194194194188 95.0754546334314
0.142742742742735 95.0589864145731
0.171291291291282 95.0392615402995
0.199839839839829 95.0164192125275
0.228388388388376 94.9905986331751
0.256936936936923 94.9619390041589
0.28548548548547 94.9305795273971
0.314034034034017 94.8966594048062
0.342582582582564 94.8603178383045
0.371131131131111 94.8216940298085
0.399679679679658 94.7809268182487
0.428228228228205 94.7381389420893
0.456776776776752 94.6934309950662
0.485325325325299 94.6469019604376
0.513873873873846 94.5986508214646
0.542422422422393 94.5487765614052
0.570970970970941 94.4973781635206
0.599519519519487 94.4445546110687
0.628068068068035 94.3904048873108
0.656616616616582 94.3350279755049
0.685165165165129 94.2785228589122
0.713713713713676 94.2209885207905
0.742262262262223 94.1625239444013
0.77081081081077 94.1032281130023
0.799359359359317 94.0432000098549
0.827907907907864 93.9825386182169
0.856456456456411 93.9213429213498
0.885005005004958 93.8597119025113
0.913553553553505 93.7977445449628
0.942102102102052 93.7355398319619
0.970650650650599 93.6731967467704
0.999199199199146 93.6108142726458
1.02774774774769 93.5484913928495
1.05629629629624 93.4863270906394
1.08484484484479 93.4244203492769
1.11339339339333 93.3628701520197
1.14194194194188 93.3017754821292
1.17049049049043 93.2412353228633
1.19903903903897 93.1813486574833
1.22758758758752 93.1222144692469
1.25613613613607 93.0639317414157
1.28468468468462 93.0065994572474
1.31323323323316 92.9503166000034
1.34178178178171 92.8951821253864
1.37033033033026 92.8412600543377
1.3988788788788 92.7885111137525
};
\addplot [semithick, white!20!black]
table {%
0 85.0195634526627
0.0750137892995025 85.0076467894167
0.150027578599005 84.9741217769066
0.225041367898507 84.9206435848027
0.30005515719801 84.8488673827751
0.375068946497512 84.7604483404938
0.450082735797015 84.6570416276295
0.525096525096517 84.5403024138518
0.60011031439602 84.4118858688316
0.675124103695522 84.2734471622387
0.750137892995025 84.126641463743
0.825151682294527 83.9731239430155
0.900165471594029 83.8144570056994
0.975179260893532 83.6516351198172
1.05019305019303 83.4854379044821
1.12520683949254 83.3166445756724
1.20022062879204 83.1460343493683
1.27523441809154 82.9743864415484
1.35024820739104 82.8024800681916
1.42526199669055 82.6310944452781
1.50027578599005 82.4610087887866
1.57528957528955 82.2930023146959
1.65030336458905 82.1278542389861
1.72531715388856 81.966343777636
1.80033094318806 81.8092501466244
1.87534473248756 81.6573525619315
1.95035852178706 81.5114302395355
2.02537231108657 81.3722623954167
2.10038610038607 81.2406199934981
2.17539988968557 81.1167646013045
2.25041367898507 81.0001599037085
2.32542746828458 80.8901999690122
2.40044125758408 80.7862788655181
2.47545504688358 80.6877906615294
2.55046883618308 80.5941294253484
2.62548262548259 80.5046892252774
2.70049641478209 80.4188641296195
2.77551020408159 80.336048206677
2.85052399338109 80.2556355247529
2.9255377826806 80.1770201521496
3.0005515719801 80.0995961571694
3.0755653612796 80.0227576081155
3.1505791505791 79.9458985732902
3.22559293987861 79.8684131209959
3.30060672917811 79.7896953195358
3.37562051847761 79.709202530206
3.45063430777711 79.6268612166352
3.52564809707662 79.5428086161043
3.60066188637612 79.4571829713169
3.67567567567562 79.3701225249777
};
\addplot [semithick, white!20!black]
table {%
0 61.280679224693
0.212244897959184 61.13267308069
0.424489795918368 60.9656480649313
0.636734693877551 60.7817797794168
0.848979591836735 60.5832438261466
1.06122448979592 60.3722158071208
1.2734693877551 60.1508713243393
1.48571428571429 59.9213859798022
1.69795918367347 59.6859353755097
1.91020408163265 59.4466951134616
2.12244897959184 59.2058346931948
2.33469387755102 58.9642298920185
2.54693877551021 58.7198700220843
2.75918367346939 58.4703538378897
2.97142857142857 58.2132800939327
3.18367346938776 57.9462475447109
3.39591836734694 57.6668549447223
3.60816326530612 57.3727010484646
3.82040816326531 57.0613846104357
4.03265306122449 56.7305043851332
4.24489795918368 56.3777038607132
4.45714285714286 56.0023655462946
4.66938775510204 55.606131000736
4.88163265306123 55.1907927589921
5.09387755102041 54.758143356018
5.30612244897959 54.3099753267685
5.51836734693878 53.8480812061987
5.73061224489796 53.3742535292635
5.94285714285715 52.8902848309178
6.15510204081633 52.3979676461166
6.36734693877551 51.899149392402
6.5795918367347 51.3964986934377
6.79183673469388 50.8933163389854
7.00408163265306 50.3929193803145
7.21632653061225 49.898624868694
7.42857142857143 49.4137498553932
7.64081632653062 48.9416113916814
7.8530612244898 48.4855265288277
8.06530612244898 48.0488123181013
8.27755102040817 47.6347858107715
8.48979591836735 47.2456354394863
8.70204081632654 46.8752084398975
8.91428571428572 46.5136134984744
9.1265306122449 46.1509416670204
9.33877551020409 45.7772839973389
9.55102040816327 45.3827315412334
9.76326530612245 44.9573753505073
9.97551020408164 44.4913064769639
10.1877551020408 43.9746159724067
10.4 43.3973948886392
};
\addplot [semithick, white!20!black]
table {%
0 61.280679224693
0.212244897959184 61.13267308069
0.424489795918368 60.9656480649313
0.636734693877551 60.7817797794168
0.848979591836735 60.5832438261466
1.06122448979592 60.3722158071208
1.2734693877551 60.1508713243393
1.48571428571429 59.9213859798022
1.69795918367347 59.6859353755097
1.91020408163265 59.4466951134616
2.12244897959184 59.2058346931948
2.33469387755102 58.9642298920185
2.54693877551021 58.7198700220843
2.75918367346939 58.4703538378897
2.97142857142857 58.2132800939327
3.18367346938776 57.9462475447109
3.39591836734694 57.6668549447223
3.60816326530612 57.3727010484646
3.82040816326531 57.0613846104357
4.03265306122449 56.7305043851332
4.24489795918368 56.3777038607132
4.45714285714286 56.0023655462946
4.66938775510204 55.606131000736
4.88163265306123 55.1907927589921
5.09387755102041 54.758143356018
5.30612244897959 54.3099753267685
5.51836734693878 53.8480812061987
5.73061224489796 53.3742535292635
5.94285714285715 52.8902848309178
6.15510204081633 52.3979676461166
6.36734693877551 51.899149392402
6.5795918367347 51.3964986934377
6.79183673469388 50.8933163389854
7.00408163265306 50.3929193803145
7.21632653061225 49.898624868694
7.42857142857143 49.4137498553932
7.64081632653062 48.9416113916814
7.8530612244898 48.4855265288277
8.06530612244898 48.0488123181013
8.27755102040817 47.6347858107715
8.48979591836735 47.2456354394863
8.70204081632654 46.8752084398975
8.91428571428572 46.5136134984744
9.1265306122449 46.1509416670204
9.33877551020409 45.7772839973389
9.55102040816327 45.3827315412334
9.76326530612245 44.9573753505073
9.97551020408164 44.4913064769639
10.1877551020408 43.9746159724067
10.4 43.3973948886392
};
\addplot [semithick, white!20!black]
table {%
0 129.742518628435
0.0679591836734679 129.683493510122
0.135918367346936 129.624268345419
0.203877551020404 129.564891937017
0.271836734693872 129.505413087608
0.339795918367339 129.44588059988
0.407755102040807 129.386343276524
0.475714285714275 129.326849920232
0.543673469387743 129.267449333693
0.611632653061211 129.208190319599
0.679591836734679 129.149094179427
0.747551020408147 129.090007295013
0.815510204081615 129.030707507478
0.883469387755083 128.970972497517
0.951428571428551 128.910579945826
1.01938775510202 128.849307533101
1.08734693877549 128.786932940037
1.15530612244895 128.72323384733
1.22326530612242 128.657987935677
1.29122448979589 128.590972885772
1.35918367346936 128.521966378312
1.42714285714283 128.450746093992
1.49510204081629 128.377089713508
1.56306122448976 128.300774917556
1.63102040816323 128.221579386832
1.6989795918367 128.139280802031
1.76693877551017 128.053656843849
1.83489795918363 127.964485192981
1.9028571428571 127.871543530125
1.97081632653057 127.774609535975
2.03877551020404 127.673604929773
2.1067346938775 127.569176286294
2.17469387755097 127.462197692641
2.24265306122444 127.353543317018
2.31061224489791 127.244087327628
2.37857142857138 127.134703892678
2.44653061224484 127.02626718037
2.51448979591831 126.919651358908
2.58244897959178 126.815730596497
2.65040816326525 126.715379061341
2.71836734693872 126.619470921643
2.78632653061218 126.528880345607
2.85428571428565 126.444481501439
2.92224489795912 126.367148557341
2.99020408163259 126.297755681518
3.05816326530606 126.237177042175
3.12612244897952 126.186286807514
3.19408163265299 126.145959145741
3.26204081632646 126.117068225058
3.32999999999993 126.100488213671
};
\addplot [semithick, white!20!black]
table {%
0 85.4239203643762
0.0876611305182692 85.4019511220349
0.175322261036538 85.3416828705954
0.262983391554808 85.2501177158208
0.350644522073077 85.1342577634755
0.438305652591346 85.0011051193242
0.525966783109615 84.8576618891294
0.613627913627884 84.710238713023
0.701289044146153 84.5613188678926
0.788950174664423 84.4120687851671
0.876611305182692 84.2636536858129
0.964272435700961 84.1172387907966
1.05193356621923 83.9739893210826
1.1395946967375 83.8350704976382
1.22725582725577 83.7016475414277
1.31491695777404 83.5748856734177
1.40257808829231 83.4559501145747
1.49023921881058 83.3460036497044
1.57790034932885 83.2454165710341
1.66556147984711 83.1526311741187
1.75322261036538 83.0658021656364
1.84088374088365 82.9830842522668
1.92854487140192 82.9026321406895
2.01620600192019 82.8226005375828
2.10386713243846 82.7411441496264
2.19152826295673 82.6564176834996
2.279189393475 82.566575845881
2.36685052399327 82.46977334345
2.45451165451154 82.3642146218317
2.54217278502981 82.2495985337889
2.62983391554808 82.1273436136081
2.71749504606634 81.9989634680275
2.80515617658461 81.8659717037837
2.89281730710288 81.729881927615
2.98047843762115 81.5922077462598
3.06813956813942 81.4544627664546
3.15580069865769 81.3181605949379
3.24346182917596 81.1848148384479
3.33112295969423 81.0559391037212
3.4187840902125 80.9328507075669
3.50644522073077 80.8149961558521
3.59410635124904 80.7007852408735
3.68176748176731 80.5886167343926
3.76942861228557 80.4768894081696
3.85708974280384 80.3640020339663
3.94475087332211 80.2483533835444
4.03241200384038 80.1283422286639
4.12007313435865 80.0023673410866
4.20773426487692 79.8688274925745
4.29539539539519 79.7261214548872
};
\addplot [semithick, white!20!black]
table {%
0 58.1271949574006
0.0810204081632612 57.8982618841691
0.162040816326522 57.6536863309316
0.243061224489784 57.3963880764053
0.324081632653045 57.1292868993096
0.405102040816306 56.855302578364
0.486122448979567 56.5773548922856
0.567142857142829 56.2983636197939
0.64816326530609 56.0212485396082
0.729183673469351 55.7489294304458
0.810204081632612 55.484326071026
0.891224489795873 55.2299632959471
0.972244897959135 54.986088483454
1.0532653061224 54.7521343942542
1.13428571428566 54.5275327615496
1.21530612244892 54.3117153185438
1.29632653061218 54.1041137984406
1.37734693877544 53.9041599344419
1.4583673469387 53.7112854597516
1.53938775510196 53.5249221075731
1.62040816326522 53.3445016111086
1.70142857142849 53.1694557035618
1.78244897959175 52.9992161181362
1.86346938775501 52.833214588034
1.94448979591827 52.6709340388133
2.02551020408153 52.5121184179704
2.10653061224479 52.3565946536461
2.18755102040805 52.204189708479
2.26857142857131 52.0547305451077
2.34959183673458 51.9080441261696
2.43061224489784 51.7639574143032
2.5116326530611 51.6222973721473
2.59265306122436 51.4828909623392
2.67367346938762 51.3455651475174
2.75469387755088 51.2101468903207
2.83571428571414 51.0764631533864
2.9167346938774 50.9443408993531
2.99775510204067 50.8135666908066
3.07877551020393 50.6837447005595
3.15979591836719 50.5544276548895
3.24081632653045 50.4251682770806
3.32183673469371 50.2955192904182
3.40285714285697 50.1650334181875
3.48387755102023 50.0332633836728
3.56489795918349 49.8997619101593
3.64591836734676 49.7640817209322
3.72693877551002 49.6257755392757
3.80795918367328 49.4843960884751
3.88897959183654 49.3394960918158
3.9699999999998 49.1906282725817
};
\addplot [semithick, white!20!black]
table {%
0 94.3956294618508
0.0314285714285707 94.1822124810298
0.0628571428571414 93.9651778964838
0.0942857142857121 93.7450739746752
0.125714285714283 93.5224489820759
0.157142857142853 93.2978511851485
0.188571428571424 93.0718288503588
0.219999999999995 92.8449302441722
0.251428571428566 92.6177036330512
0.282857142857136 92.390697283468
0.314285714285707 92.164459461885
0.345714285714278 91.9395384347678
0.377142857142848 91.7164824685821
0.408571428571419 91.4958398297903
0.43999999999999 91.2781587848645
0.47142857142856 91.0639876002672
0.502857142857131 90.8538741818686
0.534285714285702 90.6482175344489
0.565714285714272 90.4470388193637
0.597142857142843 90.2503001612526
0.628571428571414 90.0579636847461
0.659999999999984 89.8699915144778
0.691428571428555 89.6863457750813
0.722857142857126 89.5069885911877
0.754285714285696 89.3318820874355
0.785714285714267 89.160988388456
0.817142857142838 88.9942696188825
0.848571428571409 88.8316879033464
0.879999999999979 88.6732053664859
0.91142857142855 88.5187841329324
0.942857142857121 88.3683863273192
0.974285714285691 88.22197407428
1.00571428571426 88.0795094984464
1.03714285714283 87.9409547244561
1.0685714285714 87.8062718769405
1.09999999999997 87.6754230805332
1.13142857142854 87.5483704598679
1.16285714285712 87.4250761395762
1.19428571428569 87.3055022442955
1.22571428571426 87.1896108986574
1.25714285714283 87.0773642272955
1.2885714285714 86.9687243548435
1.31999999999997 86.8636534059333
1.35142857142854 86.7621135052017
1.38285714285711 86.6640667772805
1.41428571428568 86.5694753468036
1.44571428571425 86.4783013384043
1.47714285714282 86.3905068767151
1.50857142857139 86.306054086372
1.53999999999996 86.2249050920074
};
\addplot [semithick, white!20!black]
table {%
0 62.5067430615558
0.150863516577801 62.4750389752515
0.301727033155602 62.3888222020313
0.452590549733403 62.2594423885896
0.603454066311203 62.0982491816209
0.754317582889004 61.9165922278197
0.905181099466805 61.7258211738799
1.05604461604461 61.5372856664961
1.20690813262241 61.3622660585996
1.35777164920021 61.2070465319753
1.50863516577801 61.0696421833248
1.65949868235581 60.9472916788341
1.81036219893361 60.8372336846891
1.96122571551141 60.736706867076
2.11208923208921 60.6429498921807
2.26295274866701 60.5532014261891
2.41381626524481 60.4647001352874
2.56467978182261 60.3746846856615
2.71554329840042 60.2803937434975
2.86640681497822 60.1795518661417
3.01727033155602 60.0719372180319
3.16813384813382 59.9578704159199
3.31899736471162 59.8376720797326
3.46986088128942 59.7116628293978
3.62072439786722 59.580163284843
3.77158791444502 59.4434940659957
3.92245143102282 59.3019757927833
4.07331494760062 59.1559290851334
4.22417846417842 59.0056745629734
4.37504198075623 58.8515456340073
4.52590549733403 58.6940311642747
4.67676901391183 58.5337236420049
4.82763253048963 58.371217438258
4.97849604706743 58.2071069240943
5.12935956364523 58.0419864705739
5.28022308022303 57.8764504487568
5.43108659680083 57.7110932297034
5.58195011337863 57.546509184474
5.73281362995643 57.3832926841283
5.88367714653423 57.2220358056755
6.03454066311203 57.0631867194064
6.18540417968984 56.9069668620486
6.33626769626764 56.7535777300499
6.48713121284544 56.6032208198583
6.63799472942324 56.4560976279217
6.78885824600104 56.3124096506881
6.93972176257884 56.172358384605
7.09058527915664 56.0361453261205
7.24144879573444 55.9039719716826
7.39231231231224 55.776039817739
};
\addplot [semithick, white!20!black]
table {%
0 41.3059277763755
0.0142857142857152 41.2183182411904
0.0285714285714304 41.1305610658829
0.0428571428571456 41.0426531125932
0.0571428571428609 40.9545912434623
0.0714285714285761 40.8663723206291
0.0857142857142913 40.7779932062344
0.100000000000007 40.6894507624186
0.114285714285722 40.600741851322
0.128571428571437 40.5118633350847
0.142857142857152 40.4228120758479
0.157142857142867 40.3335849357503
0.171428571428583 40.2441787769329
0.185714285714298 40.1545904615359
0.200000000000013 40.0648168516998
0.214285714285728 39.9748548095646
0.228571428571443 39.8847011972714
0.242857142857159 39.794352876959
0.257142857142874 39.7038067107685
0.271428571428589 39.61305956084
0.285714285714304 39.5221082893139
0.300000000000019 39.4309497583304
0.314285714285735 39.3395808300305
0.32857142857145 39.247998366553
0.342857142857165 39.156199230039
0.35714285714288 39.0641802826286
0.371428571428596 38.9719383864622
0.385714285714311 38.87947040368
0.400000000000026 38.786773196423
0.414285714285741 38.69384362683
0.428571428571456 38.600678557042
0.442857142857172 38.5072748491993
0.457142857142887 38.4136293654422
0.471428571428602 38.3197389679109
0.485714285714317 38.2256005187464
0.500000000000032 38.1312108800876
0.514285714285748 38.0365669140753
0.528571428571463 37.9416654828499
0.542857142857178 37.8465034485517
0.557142857142893 37.7510776733209
0.571428571428608 37.6553850192986
0.585714285714324 37.5594223486235
0.600000000000039 37.4631865234365
0.614285714285754 37.3666744058781
0.628571428571469 37.2698828580884
0.642857142857185 37.1728087422078
0.6571428571429 37.0754489203772
0.671428571428615 36.9778002547354
0.68571428571433 36.8798596074234
0.700000000000045 36.7816238405815
};
\addplot [semithick, white!20!black]
table {%
0 46.4075398812719
0.086734693877551 46.3682812169819
0.173469387755102 46.326762778459
0.260204081632653 46.2826905499352
0.346938775510204 46.2357705156427
0.433673469387755 46.1857086598136
0.520408163265306 46.1322109666799
0.607142857142857 46.0749834204739
0.693877551020408 46.0137320054277
0.780612244897959 45.9481627057733
0.86734693877551 45.8779815057428
0.954081632653061 45.8028943895686
1.04081632653061 45.7226073414825
1.12755102040816 45.6368272204098
1.21428571428571 45.5453683463139
1.30102040816327 45.448254623248
1.38775510204082 45.3455341582379
1.47448979591837 45.2372550583096
1.56122448979592 45.1234654304887
1.64795918367347 45.0042133818011
1.73469387755102 44.8795470192728
1.82142857142857 44.7495144499293
1.90816326530612 44.6141637807964
1.99489795918367 44.4735431189003
2.08163265306122 44.3277005712663
2.16836734693878 44.1766842449205
2.25510204081633 44.0205422468887
2.34183673469388 43.8593226841968
2.42857142857143 43.6930736638703
2.51530612244898 43.5218432929352
2.60204081632653 43.3456796784174
2.68877551020408 43.1646309273425
2.77551020408163 42.9787451467364
2.86224489795918 42.7880704436251
2.94897959183673 42.592654925034
3.03571428571429 42.3925466979892
3.12244897959184 42.1877938695166
3.20918367346939 41.9784445466416
3.29591836734694 41.7645468363903
3.38265306122449 41.5461488457888
3.46938775510204 41.3232986818622
3.55612244897959 41.0960444516367
3.64285714285714 40.8644342621384
3.72959183673469 40.6285162203925
3.81632653061224 40.3883384334251
3.9030612244898 40.1439490082623
3.98979591836735 39.8953960519293
4.0765306122449 39.6427276714523
4.16326530612245 39.3859919738573
4.25 39.1252370661696
};
\addplot [semithick, white!20!black]
table {%
0 81.4207530839561
0.126326530612246 81.396107983588
0.252653061224492 81.3672418695219
0.378979591836738 81.3337043790336
0.505306122448984 81.2950451493988
0.63163265306123 81.2508138178931
0.757959183673476 81.2005600217923
0.884285714285722 81.143833398372
1.01061224489797 81.0801835849078
1.13693877551021 81.0091602186753
1.26326530612246 80.9303142902344
1.38959183673471 80.8434836863341
1.51591836734695 80.7491463970143
1.6422448979592 80.6478670224856
1.76857142857144 80.5402101629584
1.89489795918369 80.4267404186432
2.02122448979594 80.3080223897504
2.14755102040818 80.1846206764905
2.27387755102043 80.0570998790739
2.40020408163267 79.9260245977112
2.52653061224492 79.7919623688394
2.65285714285717 79.6555948747054
2.77918367346941 79.5177520770007
2.90551020408166 79.3792738471813
3.0318367346939 79.2410000567033
3.15816326530615 79.1037705770228
3.2844897959184 78.9684252795965
3.41081632653064 78.8358040358801
3.53714285714289 78.7067467173299
3.66346938775513 78.5820931954021
3.78979591836738 78.4626263140287
3.91612244897963 78.348275616412
4.04244897959187 78.2383137731626
4.16877551020412 78.131996557847
4.29510204081636 78.0285797440316
4.42142857142861 77.9273191052829
4.54775510204086 77.8274704151673
4.6740816326531 77.7282894472512
4.80040816326535 77.6290319751012
4.9267346938776 77.5289537722837
5.05306122448984 77.4272079650952
5.17938775510209 77.3221890523529
5.30571428571433 77.2119515137928
5.43204081632658 77.0945482252869
5.55836734693882 76.9680320627073
5.68469387755107 76.830455901926
5.81102040816332 76.6798726188153
5.93734693877556 76.5143350892473
6.06367346938781 76.331896189094
6.19000000000005 76.1306087942276
};
\addplot [semithick, white!20!black]
table {%
0 46.4968421831844
0.112053482053481 46.4172642693891
0.224106964106962 46.2018413320367
0.336160446160443 45.8800531642522
0.448213928213925 45.4813795591591
0.560267410267406 45.0353003098804
0.672320892320887 44.5712952095418
0.784374374374368 44.1188440311956
0.896427856427849 43.7019419937814
1.00848133848133 43.3236520733738
1.12053482053481 42.982044248125
1.23258830258829 42.6751884961893
1.34464178464177 42.4011547957187
1.45669526669525 42.1580131248657
1.56874874874874 41.9438334617842
1.68080223080222 41.756685784626
1.7928557128557 41.5946400715448
1.90490919490918 41.455766300693
2.01696267696266 41.3381344502234
2.12901615901614 41.2398144982891
2.24106964106962 41.1588764230429
2.3531231231231 41.0929721756752
2.46517660517658 41.037914728404
2.57723008723007 40.9890115100215
2.68928356928355 40.9415699346666
2.80133705133703 40.8908974164787
2.91339053339051 40.8323013695967
3.02544401544399 40.7610892081598
3.13749749749747 40.6725683463071
3.24955097955095 40.5620461981774
3.36160446160443 40.4248301779104
3.47365794365792 40.2562276996448
3.5857114257114 40.0515461775193
3.69776490776488 39.8060930256739
3.80981838981836 39.515825209072
3.92187187187184 39.1799958847671
4.03392535392532 38.7989012405544
4.1459788359788 38.3728378746878
4.25803231803228 37.9021023854246
4.37008580008576 37.3869913710195
4.48213928213925 36.8278014297269
4.59419276419273 36.2248291598045
4.70624624624621 35.578371159507
4.81829972829969 34.8887240270883
4.93035321035317 34.1561843608069
5.04240669240665 33.3810487589152
5.15446017446013 32.5636138196719
5.26651365651361 31.706980429177
5.37856713856709 30.830694936757
5.49062062062058 29.9602799420489
};
\addplot [semithick, white!20!black]
table {%
0 35.0525245506944
0.128775510204081 34.5929966863877
0.257551020408161 34.1322138149291
0.386326530612242 33.679303915387
0.515102040816322 33.2433949668324
0.643877551020403 32.8336149483362
0.772653061224483 32.459091838967
0.901428571428564 32.1289536177959
1.03020408163264 31.8511299231657
1.15897959183672 31.6249506977456
1.28775510204081 31.4459897143136
1.41653061224489 31.3098050341602
1.54530612244897 31.2119547185756
1.67408163265305 31.1479968288493
1.80285714285713 31.1134894262716
1.93163265306121 31.1039905721323
2.06040816326529 31.1150583277214
2.18918367346937 31.1422507543289
2.31795918367345 31.1811259132447
2.44673469387753 31.2272418657589
2.57551020408161 31.2761566731614
2.70428571428569 31.3234283967422
2.83306122448977 31.3646150977911
2.96183673469385 31.3952748375982
3.09061224489793 31.4109656774536
3.21938775510201 31.4072456786471
3.34816326530609 31.3796729024687
3.47693877551017 31.3238054102083
3.60571428571425 31.2352012914043
3.73448979591833 31.1107005950831
3.86326530612242 30.9518465399861
3.9920408163265 30.7612606292368
4.12081632653058 30.541564365959
4.24959183673466 30.2953792532753
4.37836734693874 30.0253267943097
4.50714285714282 29.734028492186
4.6359183673469 29.4241058500267
4.76469387755098 29.0981803709557
4.89346938775506 28.7588735580971
5.02224489795914 28.408806914573
5.15102040816322 28.0506019435076
5.2797959183673 27.686880148025
5.40857142857138 27.3202630312472
5.53734693877546 26.9533720962984
5.66612244897954 26.5888288463019
5.79489795918362 26.229254784382
5.9236734693877 25.8772714136607
6.05244897959178 25.5355002372621
6.18122448979586 25.2065627583103
6.30999999999995 24.8930715003804
};
\addplot [semithick, white!20!black]
table {%
0 120.026896970323
0.145918367346941 119.777383123404
0.291836734693881 119.5954157664
0.437755102040822 119.466815839458
0.583673469387763 119.377404282727
0.729591836734703 119.313002036354
0.875510204081644 119.259430040487
1.02142857142858 119.202509235273
1.16734693877553 119.128060560861
1.31326530612247 119.021904957398
1.45918367346941 118.869893089009
1.60510204081635 118.664177102907
1.75102040816329 118.410968587353
1.89693877551023 118.118381465121
2.04285714285717 117.794529658989
2.18877551020411 117.447527091731
2.33469387755105 117.085487686126
2.48061224489799 116.716525364949
2.62653061224493 116.348754050977
2.77244897959187 115.990287666985
2.91836734693881 115.649133773648
3.06428571428575 115.329165104887
3.21020408163269 115.028883108425
3.35612244897963 114.746430259889
3.50204081632658 114.479949034903
3.64795918367352 114.227581909094
3.79387755102046 113.987471358089
3.9397959183674 113.757759857513
4.08571428571434 113.536589882992
4.23163265306128 113.322103910152
4.37755102040822 113.112534843415
4.52346938775516 112.907468669943
4.6693877551021 112.707532982669
4.81530612244904 112.513382168243
4.96122448979598 112.325670613314
5.10714285714292 112.145052704533
5.25306122448986 111.972182828552
5.3989795918368 111.807715372018
5.54489795918374 111.652304721584
5.69081632653069 111.506605263898
5.83673469387763 111.371139898916
5.98265306122457 111.245459757721
6.12857142857151 111.12868042172
6.27448979591845 111.019915417835
6.42040816326539 110.91827827299
6.56632653061233 110.822882514109
6.71224489795927 110.732841668117
6.85816326530621 110.647269261937
7.00408163265315 110.565278822493
7.15000000000009 110.48598387671
};
\addplot [semithick, white!20!black]
table {%
0 80.048792177872
0.104897959183671 79.8461064397409
0.209795918367342 79.7151191555056
0.314693877551013 79.6402079904854
0.419591836734683 79.6057506099998
0.524489795918354 79.596124679368
0.629387755102025 79.5957078639096
0.734285714285696 79.5888778289439
0.839183673469367 79.5600122397905
0.944081632653038 79.4934887617688
1.04897959183671 79.3737135339853
1.15387755102038 79.1911291382942
1.25877551020405 78.9496462575909
1.36367346938772 78.6549978971096
1.46857142857139 78.3129170620847
1.57346938775506 77.9291367577504
1.67836734693873 77.5093899893428
1.7832653061224 77.0594097620927
1.88816326530608 76.5849290812361
1.99306122448975 76.0916809520072
2.09795918367342 75.5853936127094
2.20285714285709 75.0716099872063
2.30775510204076 74.555632269352
2.41265306122443 74.0427465646018
2.5175510204081 73.5382389784178
2.62244897959177 73.0473956162601
2.72734693877544 72.5755025835884
2.83224489795911 72.1278459858628
2.93714285714278 71.7097119285447
3.04204081632646 71.3263865170906
3.14693877551013 70.9829029792293
3.2518367346938 70.6805107425363
3.35673469387747 70.4175464577382
3.46163265306114 70.1922718488255
3.56653061224481 70.0029486397898
3.67142857142848 69.84783855462
3.77632653061215 69.7252033173079
3.88122448979582 69.633304651844
3.98612244897949 69.5704042822191
4.09102040816316 69.5347639324239
4.19591836734683 69.5241636579423
4.30081632653051 69.532823682951
4.40571428571418 69.5533687046979
4.51061224489785 69.5784158943607
4.61551020408152 69.6005824231171
4.72040816326519 69.6124854621447
4.82530612244886 69.6067421826213
4.93020408163253 69.5759697557243
5.0351020408162 69.5127853526316
5.13999999999987 69.4098061445206
};
\addplot [semithick, white!20!black]
table {%
0 100.433493514115
0.0129378357949797 100.431506961157
0.0258756715899594 100.42535454495
0.0388135073849391 100.414766172782
0.0517513431799188 100.399471751939
0.0646891789748985 100.379201189711
0.0776270147698782 100.353745516578
0.0905648505648579 100.323224111265
0.103502686359838 100.287866133655
0.116440522154817 100.247900821376
0.129378357949797 100.20355741206
0.142316193744777 100.155065143337
0.155254029539756 100.102653252836
0.168191865334736 100.046550978187
0.181129701129716 99.9869875570238
0.194067536924696 99.9241922269715
0.207005372719675 99.8583942256625
0.219943208514655 99.7898227907296
0.232881044309635 99.7187071597981
0.245818880104614 99.6452765705005
0.258756715899594 99.5697602604697
0.271694551694574 99.4923874673307
0.284632387489553 99.4133874287161
0.297570223284533 99.3329893822592
0.310508059079513 99.2514225655843
0.323445894874493 99.1689162163245
0.336383730669472 99.08569957211
0.349321566464452 99.0020018705739
0.362259402259432 98.9179627144448
0.375197238054411 98.8333064573807
0.388135073849391 98.7476372902685
0.401072909644371 98.6605593918203
0.414010745439351 98.5716769407572
0.42694858123433 98.4805941158005
0.43988641702931 98.386915095662
0.45282425282429 98.2902440590629
0.465762088619269 98.190185184725
0.478699924414249 98.0863426513593
0.491637760209229 97.9783206376875
0.504575596004209 97.8657233224317
0.517513431799188 97.7481548843021
0.530451267594168 97.6252195020207
0.543389103389148 97.4965213543103
0.556326939184127 97.3616646198799
0.569264774979107 97.2202534774522
0.582202610774087 97.0718921057504
0.595140446569066 96.9161846834825
0.608078282364046 96.7527353893714
0.621016118159026 96.5811484021417
0.633953953954006 96.4010279004993
};
\addplot [semithick, white!20!black]
table {%
0 51.7661246092302
0.183877551020408 51.316027160427
0.367755102040816 50.9748909735591
0.551632653061224 50.7217852805209
0.735510204081632 50.5357793132073
0.91938775510204 50.3959423035129
1.10326530612245 50.2813434833325
1.28714285714286 50.1710520845606
1.47102040816326 50.044137339092
1.65489795918367 49.8796684788213
1.83877551020408 49.6567466149137
2.02265306122449 49.3612312638511
2.2065306122449 48.9940608369974
2.3904081632653 48.5582140190194
2.57428571428571 48.0566694945844
2.75816326530612 47.4924059483595
2.94204081632653 46.8684020650115
3.12591836734694 46.1876365292077
3.30979591836734 45.4530880256148
3.49367346938775 44.6677352389001
3.67755102040816 43.8345620566319
3.86142857142857 42.9567546291572
4.04530612244898 42.0377618533285
4.22918367346938 41.0810501857884
4.41306122448979 40.0900860831799
4.5969387755102 39.0683360021467
4.78081632653061 38.0192663993311
4.96469387755102 36.9463437313764
5.14857142857142 35.853034454926
5.33244897959183 34.7428050266224
5.51632653061224 33.6195092871289
5.70020408163265 32.4927974898461
5.88408163265306 31.3767819781797
6.06795918367346 30.2856898759857
6.25183673469387 29.2337483071191
6.43571428571428 28.2351843954364
6.61959183673469 27.3042252647925
6.8034693877551 26.4550980390434
6.9873469387755 25.702029842045
7.17122448979591 25.0592477976526
7.35510204081632 24.5384694445579
7.53897959183673 24.132864918599
7.72285714285714 23.8272913547576
7.90673469387754 23.6065666757474
8.09061224489795 23.455508804282
8.27448979591836 23.358935663075
8.45836734693877 23.3016651748401
8.64224489795918 23.2685152622908
8.82612244897958 23.244303848141
9.00999999999999 23.2138488551041
};
\addplot [semithick, white!20!black]
table {%
0 104.704500385144
0.12468019039448 104.671749658374
0.24936038078896 104.576928374617
0.374040571183439 104.427359292112
0.498720761577919 104.230384604188
0.623400951972399 103.99334650417
0.748081142366879 103.723587185387
0.872761332761359 103.428448841164
0.997441523155839 103.115273664828
1.12212171355032 102.791403849706
1.2468019039448 102.464181589126
1.37148209433928 102.140949076412
1.49616228473376 101.829048504893
1.62084247512824 101.535396607418
1.74552266552272 101.261060451399
1.8702028559172 101.002854369332
1.99488304631168 100.757496831131
2.11956323670616 100.521706306708
2.24424342710064 100.292201265978
2.36892361749512 100.065700178851
2.4936038078896 99.838921515242
2.61828399828408 99.6085837450628
2.74296418867856 99.3714053382274
2.86764437907304 99.1241047646483
2.99232456946752 98.8634004942384
3.117004759862 98.5861144401373
3.24168495025648 98.2927664070656
3.36636514065096 97.9885007155962
3.49104533104544 97.6787533640383
3.61572552143992 97.3689603507004
3.74040571183439 97.064557673893
3.86508590222887 96.7709813319247
3.98976609262335 96.4936673231046
4.11444628301783 96.238051645742
4.23912647341231 96.0095702981457
4.36380666380679 95.8136592786257
4.48848685420127 95.6557545854909
4.61316704459575 95.5412632359726
4.73784723499023 95.4699786718151
4.86252742538471 95.429420677741
4.98720761577919 95.4054865446786
5.11188780617367 95.3840735635565
5.23656799656815 95.3510790253031
5.36124818696263 95.2924002208467
5.48592837735711 95.1939344411161
5.61060856775159 95.0415789770397
5.73528875814607 94.8212311195458
5.85996894854055 94.5187881595624
5.98464913893503 94.1201473880191
6.10932932932951 93.6112060958438
};
\addplot [semithick, white!20!black]
table {%
0 106.682297939778
0.151020408163267 106.549802592146
0.302040816326534 106.410962071392
0.453061224489802 106.265871003786
0.604081632653069 106.114624015596
0.755102040816336 105.957315733094
0.906122448979603 105.794040782549
1.05714285714287 105.624893790231
1.20816326530614 105.44996938241
1.3591836734694 105.269362185356
1.51020408163267 105.083166825338
1.66122448979594 104.891477928627
1.81224489795921 104.694390121493
1.96326530612247 104.491998030205
2.11428571428574 104.284396281033
2.26530612244901 104.071679500248
2.41632653061227 103.853942314118
2.56734693877554 103.631279348915
2.71836734693881 103.403785230908
2.86938775510208 103.171554586366
3.02040816326534 102.93468204156
3.17142857142861 102.693262222789
3.32244897959188 102.447390466201
3.47346938775515 102.197164657469
3.62448979591841 101.942683254444
3.77551020408168 101.684044714975
3.92653061224495 101.421347496915
4.07755102040821 101.154690058112
4.22857142857148 100.884170856419
4.37959183673475 100.609888349685
4.53061224489802 100.331940995761
4.68163265306128 100.050427252498
4.83265306122455 99.7654455777464
4.98367346938782 99.4770944293566
5.13469387755108 99.1854722651793
5.28571428571435 98.8906775430652
5.43673469387762 98.5928087208648
5.58775510204089 98.2919642564292
5.73877551020415 97.9882426076081
5.88979591836742 97.6817422322526
6.04081632653069 97.3725615882133
6.19183673469395 97.0607991333407
6.34285714285722 96.7465531883291
6.49387755102049 96.4294680226355
6.64489795918376 96.1077224313344
6.79591836734702 95.7792001075746
6.94693877551029 95.4417847445051
7.09795918367356 95.0933600352748
7.24897959183682 94.7318096730327
7.40000000000009 94.3550173509275
};
\addplot [semithick, white!20!black]
table {%
0 81.5616733197074
0.0359183673469386 81.4931244973137
0.0718367346938772 81.4246449151976
0.107755102040816 81.3562480505138
0.143673469387754 81.2879473804174
0.179591836734693 81.2197563820632
0.215510204081632 81.1516885326063
0.25142857142857 81.0837573092014
0.287346938775509 81.0159761890036
0.323265306122447 80.9483586491677
0.359183673469386 80.8809181668488
0.395102040816324 80.8136682192016
0.431020408163263 80.7466222833812
0.466938775510202 80.6797938365424
0.50285714285714 80.6131963558402
0.538775510204079 80.5468433184296
0.574693877551017 80.4807482014653
0.610612244897956 80.4149244423309
0.646530612244895 80.3493801395674
0.682448979591833 80.2841127269132
0.718367346938772 80.2191183704459
0.75428571428571 80.1543932362429
0.790204081632649 80.0899334903815
0.826122448979588 80.0257352989392
0.862040816326526 79.9617948279933
0.897959183673465 79.8981082436212
0.933877551020403 79.8346717119005
0.969795918367342 79.7714813989086
1.00571428571428 79.7085334707227
1.04163265306122 79.6458240934204
1.07755102040816 79.583349433079
1.1134693877551 79.5211056557759
1.14938775510203 79.4590889275886
1.18530612244897 79.3972954145945
1.22122448979591 79.3357212828709
1.25714285714285 79.2743626984953
1.29306122448979 79.2132158275452
1.32897959183673 79.1522768360978
1.36489795918367 79.0915418902307
1.40081632653061 79.0310071560211
1.43673469387754 78.9706687995466
1.47265306122448 78.9105229868846
1.50857142857142 78.8505658841125
1.54448979591836 78.7907936573075
1.5804081632653 78.7312024725473
1.61632653061224 78.6717884959092
1.65224489795918 78.6125478934705
1.68816326530611 78.5534768313088
1.72408163265305 78.4945714755014
1.75999999999999 78.4358279921257
};
\addplot [semithick, white!20!black]
table {%
0 91.3502277059725
0.0154583154583156 91.332062157848
0.0309166309166311 91.2779553931535
0.0463749463749467 91.1891742238767
0.0618332618332623 91.0684265906942
0.0772915772915778 90.9184549456942
0.0927498927498934 90.7420017409649
0.108208208208209 90.5418094285949
0.123666523666525 90.3206204606714
0.13912483912484 90.0811772892831
0.154583154583156 89.8262223665182
0.170041470041471 89.5584981444648
0.185499785499787 89.2807470752111
0.200958100958102 88.9957116108452
0.216416416416418 88.7061342034554
0.231874731874733 88.4147381357703
0.247333047333049 88.1235640300423
0.262791362791365 87.8338002866341
0.27824967824968 87.546581700194
0.293707993707996 87.2630430653708
0.309166309166311 86.9843191768116
0.324624624624627 86.7115448291654
0.340082940082942 86.4458548170803
0.355541255541258 86.1883839352046
0.370999570999574 85.9402669781865
0.386457886457889 85.7026387406744
0.401916201916205 85.4766340173165
0.41737451737452 85.263387602761
0.432832832832836 85.0640312376842
0.448291148291151 84.8791783545407
0.463749463749467 84.7083437442229
0.479207779207783 84.5509021045445
0.494666094666098 84.4062281333186
0.510124410124414 84.273696528359
0.525582725582729 84.1526819874792
0.541041041041045 84.0425592084928
0.55649935649936 83.9427028892133
0.571957671957676 83.8524877274542
0.587415987415991 83.7712884210292
0.602874302874307 83.6984796677516
0.618332618332623 83.6334361654351
0.633790933790938 83.575532612202
0.649249249249254 83.5241764964841
0.664707564707569 83.4788982526377
0.680165880165885 83.4392571247985
0.6956241956242 83.4048123571027
0.711082511082516 83.375123193686
0.726540826540832 83.3497488786844
0.741999141999147 83.328248656234
0.757457457457463 83.3101817704706
};
\addplot [semithick, white!20!black]
table {%
0 68.2473157705309
0.146276889134033 68.2356915064409
0.292553778268067 68.2037000157142
0.4388306674021 68.1542067547507
0.585107556536133 68.09007717995
0.731384445670166 68.0141767477118
0.8776613348042 67.9293709144359
1.02393822393823 67.8385251365218
1.17021511307227 67.7445048703694
1.3164920022063 67.6501289833104
1.46276889134033 67.5560742797189
1.60904578047437 67.4600267658771
1.7553226696084 67.3594501617485
1.90159955874243 67.2518081872963
2.04787644787647 67.1345645624839
2.1941533370105 67.0051830072746
2.34043022614453 66.8611272416318
2.48670711527857 66.6998609855188
2.6329840044126 66.5188479588989
2.77926089354663 66.3155518817351
2.92553778268067 66.0883448099102
3.0718146718147 65.8403205923932
3.21809156094873 65.5761025532585
3.36436845008277 65.300314801511
3.5106453392168 65.0175814461555
3.65692222835083 64.7325265961968
3.80319911748487 64.4497743606396
3.9494760066189 64.1739488484889
4.09575289575293 63.9096741687493
4.24202978488696 63.6615744304257
4.388306674021 63.434084692553
4.53458356315503 63.2261696228013
4.68086045228906 63.0306162109046
4.8271373414231 62.8398802833777
4.97341423055713 62.6464176667357
5.11969111969116 62.4426841874938
5.2659680088252 62.2211356721673
5.41224489795923 61.9742279472712
5.55852178709326 61.6944168393206
5.7047986762273 61.3741581748309
5.85107556536133 61.0059077803172
5.99735245449536 60.5869791576598
6.1436293436294 60.1344811085087
6.28990623276343 59.670564112862
6.43618312189746 59.2173786518976
6.5824600110315 58.7970752067909
6.72873690016553 58.4318042587185
6.87501378929956 58.1437162888567
7.0212906784336 57.9549617783818
7.16756756756763 57.8876912084702
};
\addplot [semithick, white!20!black]
table {%
0 72.0756027679118
0.055918367346939 71.9992958814058
0.111836734693878 71.9510806452872
0.167755102040817 71.9237713389995
0.223673469387756 71.9101822419861
0.279591836734695 71.9031276336905
0.335510204081634 71.8954217935563
0.391428571428573 71.879879001027
0.447346938775512 71.8493135355459
0.503265306122451 71.7965396765567
0.55918367346939 71.7143897248783
0.615102040816329 71.5995165128559
0.671020408163268 71.4570969832717
0.726938775510206 71.2934614469154
0.782857142857145 71.1149402145772
0.838775510204084 70.927863597047
0.894693877551023 70.7385619051156
0.950612244897962 70.5533654495712
1.0065306122449 70.3786045412048
1.06244897959184 70.2206094908062
1.11836734693878 70.0856036102372
1.17428571428572 69.97565062802
1.23020408163266 69.8874108267961
1.2861224489796 69.8171833678228
1.34204081632654 69.7612674123586
1.39795918367347 69.7159621216613
1.45387755102041 69.6775666569889
1.50979591836735 69.6423801795995
1.56571428571429 69.6067018507512
1.62163265306123 69.5668308317017
1.67755102040817 69.5191959078817
1.73346938775511 69.4621654264146
1.78938775510205 69.3956008128559
1.84530612244899 69.3194018999232
1.90122448979592 69.2334685203345
1.95714285714286 69.1377005068069
2.0130612244898 69.0319976920582
2.06897959183674 68.9162599088061
2.12489795918368 68.7903869897683
2.18081632653062 68.6542787676625
2.23673469387756 68.5077489826467
2.2926530612245 68.3499750970499
2.34857142857144 68.1798493915972
2.40448979591838 67.9962628018171
2.46040816326531 67.7981062632379
2.51632653061225 67.5842707113879
2.57224489795919 67.3536470817966
2.62816326530613 67.1051263099904
2.68408163265307 66.8375993314984
2.74000000000001 66.5499570818493
};
\addplot [semithick, white!20!black]
table {%
0 69.3846179800352
0.0249212477783913 69.3827285328042
0.0498424955567826 69.3771405270257
0.074763743335174 69.3680243885157
0.0996849911135653 69.3555505430899
0.124606238891957 69.3398894165644
0.149527486670348 69.3212114347547
0.174448734448739 69.2996870234768
0.199369982227131 69.2754866085465
0.224291230005522 69.2487806157798
0.249212477783913 69.2197394709922
0.274133725562304 69.1885335999998
0.299054973340696 69.1553334286183
0.323976221119087 69.1203093826635
0.348897468897478 69.0836318879515
0.37381871667587 69.0454713702977
0.398739964454261 69.0059982555183
0.423661212232652 68.965382969429
0.448582460011044 68.9237959378457
0.473503707789435 68.881407586584
0.498424955567826 68.8383883414601
0.523346203346218 68.7949086282895
0.548267451124609 68.7511388728882
0.573188698903 68.7072493647589
0.598109946681392 68.6633609639179
0.623031194459783 68.6194718866077
0.647952442238174 68.5755616527363
0.672873690016566 68.5316097822122
0.697794937794957 68.4875957949433
0.722716185573348 68.4434992108382
0.747637433351739 68.3992995498048
0.772558681130131 68.3549763317515
0.797479928908522 68.3105090765865
0.822401176686913 68.2658773042179
0.847322424465305 68.2210605345543
0.872243672243696 68.1760382875034
0.897164920022087 68.1307900829739
0.922086167800479 68.0852954408738
0.94700741557887 68.0395338811114
0.971928663357261 67.9934849235949
0.996849911135653 67.9471280882326
1.02177115891404 67.9004428949326
1.04669240669244 67.8534088636031
1.07161365447083 67.8060055141526
1.09653490224922 67.758212366489
1.12145615002761 67.7100089405209
1.146377397806 67.6613747561561
1.17129864558439 67.6122893333032
1.19621989336278 67.5627321918702
1.22114114114117 67.5126828517654
};
\addplot [semithick, white!20!black]
table {%
0 99.4209531152114
0.111224489795924 99.1528745747465
0.222448979591848 98.9559241012902
0.333673469387772 98.8197713281425
0.444897959183696 98.7340858886032
0.55612244897962 98.6885374159733
0.667346938775544 98.6727955435523
0.778571428571468 98.6765299046402
0.889795918367391 98.6894101325374
1.00102040816332 98.7011058605437
1.11224489795924 98.7013038620497
1.22346938775516 98.6833246095925
1.33469387755109 98.6485958384283
1.44591836734701 98.5996422495933
1.55714285714294 98.5389885441239
1.66836734693886 98.4691594230564
1.77959183673478 98.392679587427
1.89081632653071 98.3120737382721
2.00204081632663 98.2298665766276
2.11326530612255 98.1485828035304
2.22448979591848 98.0706973792864
2.3357142857144 97.9967515933138
2.44693877551033 97.9247748281541
2.55816326530625 97.8526285913845
2.66938775510217 97.7781743905819
2.7806122448981 97.6992737333231
2.89183673469402 97.6137881271857
3.00306122448995 97.5195790797464
3.11428571428587 97.4145080985823
3.22551020408179 97.2964366912705
3.33673469387772 97.1634238145888
3.44795918367364 97.0164828503943
3.55918367346957 96.85890150282
3.67040816326549 96.6940259794677
3.78163265306141 96.5252024879371
3.89285714285734 96.3557772358287
4.00408163265326 96.1890964307431
4.11530612244919 96.0285062802807
4.22653061224511 95.8773529920419
4.33775510204103 95.7389827736267
4.44897959183696 95.6160561814825
4.56020408163288 95.5061663814936
4.67142857142881 95.4046353200963
4.78265306122473 95.3067742304281
4.89387755102065 95.207894345626
5.00510204081658 95.1033068988274
5.1163265306125 94.988323123169
5.22755102040843 94.858254251789
5.33877551020435 94.7084115178242
5.45000000000027 94.5341061544118
};
\addplot [semithick, white!20!black]
table {%
0 74.0868345688928
0.0246360646360699 74.0830275532693
0.0492721292721399 74.0721535787184
0.0739081939082098 74.0546001899189
0.0985442585442797 74.0307549315497
0.12318032318035 74.0010053482896
0.14781638781642 73.9657389848175
0.17245245245249 73.9253433858122
0.197088517088559 73.8802060959527
0.221724581724629 73.8307146599178
0.246360646360699 73.7772566223863
0.270996710996769 73.7202195280372
0.295632775632839 73.6599909215492
0.320268840268909 73.5969583476012
0.344904904904979 73.5315093508721
0.369540969541049 73.4640314760408
0.394177034177119 73.3949122677861
0.418813098813189 73.3245392707869
0.443449163449259 73.253300029722
0.468085228085329 73.1815820892703
0.492721292721399 73.1097729941107
0.517357357357469 73.038260288922
0.541993421993539 72.9674315183832
0.566629486629608 72.8976742271729
0.591265551265678 72.8293759599702
0.615901615901748 72.7629242614538
0.640537680537818 72.698703729735
0.665173745173888 72.6368694582974
0.689809809809958 72.5771866481827
0.714445874446028 72.5193826243935
0.739081939082098 72.463184711932
0.763718003718168 72.408320235801
0.788354068354238 72.3545165210028
0.812990132990308 72.30150089254
0.837626197626378 72.249000675415
0.862262262262448 72.1967431946303
0.886898326898518 72.1444557751885
0.911534391534587 72.091865742092
0.936170456170657 72.0387004203433
0.960806520806727 71.9846871349449
0.985442585442797 71.9295532108993
1.01007865007887 71.873025973209
1.03471471471494 71.8148327468765
1.05935077935101 71.7547008569043
1.08398684398708 71.6923576282948
1.10862290862315 71.6275303860506
1.13325897325922 71.5599464551742
1.15789503789529 71.489333160668
1.18253110253136 71.4154178275346
1.20716716716743 71.3379277807764
};
\addplot [semithick, white!20!black]
table {%
0 92.9015460749949
0.218907887479325 92.8912394902223
0.43781577495865 92.8565893211885
0.656723662437975 92.7911905978021
0.8756315499173 92.6886383499713
1.09453943739662 92.5425276076046
1.31344732487595 92.3464534006107
1.53235521235527 92.0940107588979
1.7512630998346 91.7787947123756
1.97017098731392 91.3944002909508
2.18907887479325 90.9364931639736
2.40798676227257 90.4161627382055
2.6268946497519 89.8514587878038
2.84580253723122 89.2604649475936
3.06471042471055 88.6612648523961
3.28361831218987 88.0719421370363
3.5025261996692 87.510580436338
3.72143408714852 86.9952633851249
3.94034197462785 86.5440746182216
4.15924986210717 86.17509777045
4.3781577495865 85.9033834958764
4.59706563706582 85.724734867628
4.81597352454515 85.6274285630382
5.03488141202447 85.5997238619619
5.2537892995038 85.6298800442539
5.47269718698312 85.7061563897694
5.69160507446245 85.8168121783631
5.91051296194177 85.9501066898899
6.1294208494211 86.0942992042052
6.34832873690042 86.2376490011636
6.56723662437975 86.3679120985377
6.78614451185907 86.4701123289657
7.0050523993384 86.5283507329436
7.22396028681772 86.5267276321379
7.44286817429705 86.449343348215
7.66177606177637 86.2802982028412
7.8806839492557 86.0036925176828
8.09959183673502 85.6036266144072
8.31849972421434 85.0642008146792
8.53740761169367 84.3695154401658
8.756315499173 83.5095812863493
8.97522338665232 82.5019335953828
9.19413127413165 81.3721136550469
9.41303916161097 80.1456636484607
9.63194704909029 78.8481257587504
9.85085493656962 77.5050421690398
10.0697628240489 76.1419550624529
10.2886707115283 74.7844066221162
10.5075785990076 73.4579390311481
10.7264864864869 72.1880944726753
};
\addplot [semithick, white!20!black]
table {%
0 105.568939488178
0.141633878776734 105.563527800476
0.283267757553468 105.547140234521
0.424901636330203 105.518919677571
0.566535515106937 105.47800901688
0.708169393883671 105.423551139705
0.849803272660405 105.354688933302
0.99143715143714 105.270565284926
1.13307103021387 105.170369733758
1.27470490899061 105.05407365865
1.41633878776734 104.922296256151
1.55797266654408 104.775676272885
1.69960654532081 104.614852455475
1.84124042409754 104.440463550544
1.98287430287428 104.253148304716
2.12450818165101 104.053545464614
2.26614206042775 103.842293776862
2.40777593920448 103.620031988081
2.54940981798122 103.387400589001
2.69104369675795 103.145531602703
2.83267757553468 102.896720597953
2.97431145431142 102.643431493201
3.11594533308815 102.388128206895
3.25757921186489 102.133274657484
3.39921309064162 101.881334763416
3.54084696941836 101.634772443141
3.68248084819509 101.396051615107
3.82411472697182 101.167636197763
3.96574860574856 100.951990109557
4.10738248452529 100.751199396959
4.24901636330203 100.564428231911
4.39065024207876 100.389478886346
4.5322841208555 100.224146004089
4.67391799963223 100.06622422896
4.81555187840896 99.9135082047846
4.9571857571857 99.7637925753837
5.09881963596243 99.6148719845805
5.24045351473917 99.4645410761977
5.3820873935159 99.310594494058
5.52372127229263 99.1508395289101
5.66535515106937 98.9835961009091
5.8069890298461 98.8078624520745
5.94862290862284 98.6226834268164
6.09025678739957 98.4271038695448
6.23189066617631 98.2201686246698
6.37352454495304 98.0009225366015
6.51515842372977 97.7684104497496
6.65679230250651 97.5216772085248
6.79842618128324 97.259767657337
6.94006006005998 96.981726640596
};
\addplot [semithick, white!20!black]
table {%
0 85.1970434498288
0.0171428571428601 85.1581333841955
0.0342857142857202 85.1153715181805
0.0514285714285803 85.0673375854235
0.0685714285714405 85.0126113195668
0.0857142857143006 84.9497724542497
0.102857142857161 84.8774007231149
0.120000000000021 84.7940758598014
0.137142857142881 84.6983775979522
0.154285714285741 84.5888856712056
0.171428571428601 84.4641798132056
0.188571428571461 84.3229109967092
0.205714285714321 84.165368694228
0.222857142857181 83.9934808780135
0.240000000000042 83.8092467594455
0.257142857142902 83.6146655498941
0.274285714285762 83.4117364607396
0.291428571428622 83.2024587033518
0.308571428571482 82.9888314891111
0.325714285714342 82.7728540293872
0.342857142857202 82.5565255355609
0.360000000000062 82.3418452190015
0.377142857142923 82.1308122910898
0.394285714285783 81.9254259631953
0.411428571428643 81.7276854466986
0.428571428571503 81.5395899529695
0.445714285714363 81.3631386933881
0.462857142857223 81.2003308793249
0.480000000000083 81.0531657221592
0.497142857142943 80.9236424332624
0.514285714285803 80.8137486985937
0.531428571428664 80.7240776283996
0.548571428571524 80.652513859451
0.565714285714384 80.5966308422
0.582857142857244 80.554002027102
0.600000000000104 80.52220086461
0.617142857142964 80.4988008051786
0.634285714285824 80.4813752992612
0.651428571428684 80.4674977973123
0.668571428571544 80.4547417497853
0.685714285714405 80.4406806071347
0.702857142857265 80.4228878198139
0.720000000000125 80.3989368382774
0.737142857142985 80.3664011129786
0.754285714285845 80.3228540943722
0.771428571428705 80.2658692329111
0.788571428571565 80.1930199790506
0.805714285714425 80.1018797832429
0.822857142857286 79.9900220959443
0.840000000000146 79.8550203676057
};
\addplot [semithick, white!20!black]
table {%
0 63.200041209007
0.0324103695532274 63.1749118823594
0.0648207391064549 63.1000947822115
0.0972311086596823 62.9797115167275
0.12964147821291 62.8178836940701
0.162051847766137 62.6187329224055
0.194462217319365 62.3863808098969
0.226872586872592 62.1249489647083
0.25928295642582 61.8385589950037
0.291693325979047 61.5313325089471
0.324103695532274 61.2073911147003
0.356514065085502 60.8708564204318
0.388924434638729 60.5258500343035
0.421334804191957 60.1764935644794
0.453745173745184 59.8269086191235
0.486155543298412 59.4812168063999
0.518565912851639 59.1435397344703
0.550976282404867 58.8177522523336
0.583386651958094 58.5052308779588
0.615797021511321 58.2059002981879
0.648207391064549 57.9196672721986
0.680617760617776 57.6464385591685
0.713028130171004 57.3861209182735
0.745438499724231 57.1386211086948
0.777848869277459 56.9038458896082
0.810259238830686 56.6817020201913
0.842669608383914 56.4720962596216
0.875079977937141 56.2749353670769
0.907490347490368 56.0901261017336
0.939900717043596 55.9175752227717
0.972311086596823 55.7571894893676
1.00472145615005 55.6088756606989
1.03713182570328 55.4725404959433
1.06954219525651 55.3480907542783
1.10195256480973 55.2354331948809
1.13436293436296 55.1344735466123
1.16677330391619 55.0448782789049
1.19918367346942 54.9657694295127
1.23159404302264 54.8961936977314
1.26400441257587 54.8351977828566
1.2964147821291 54.781828384184
1.32882515168233 54.7351322010099
1.36123552123555 54.6941559326297
1.39364589078878 54.6579462783392
1.42605626034201 54.6255499374341
1.45846662989523 54.5960136092102
1.49087699944846 54.5683839929631
1.52328736900169 54.5417077879889
1.55569773855492 54.5150316935832
1.58810810810814 54.4874024090417
};
\addplot [semithick, white!20!black]
table {%
0 109.95009410557
0.0734693877551002 109.505745406748
0.1469387755102 109.117808940618
0.220408163265301 108.782500690592
0.293877551020401 108.496036640077
0.367346938775501 108.254632772479
0.440816326530601 108.054505071208
0.514285714285701 107.891869519672
0.587755102040802 107.762942101279
0.661224489795902 107.663938799437
0.734693877551002 107.591075597554
0.808163265306102 107.540568479038
0.881632653061202 107.508633427297
0.955102040816302 107.49148642574
1.0285714285714 107.485343457774
1.1020408163265 107.486420506808
1.1755102040816 107.49093355625
1.2489795918367 107.495098589508
1.3224489795918 107.495131589989
1.3959183673469 107.487248541103
1.469387755102 107.467665426257
1.5428571428571 107.43259822886
1.6163265306122 107.378262932319
1.6897959183673 107.300875520042
1.7632653061224 107.196651975439
1.8367346938775 107.061808281916
1.9102040816326 106.892860418698
1.98367346938771 106.689609023423
2.05714285714281 106.453881386734
2.13061224489791 106.187534654782
2.20408163265301 105.892425973717
2.27755102040811 105.570412489688
2.35102040816321 105.223351348844
2.42448979591831 104.853099697336
2.49795918367341 104.461514681313
2.57142857142851 104.050453446926
2.64489795918361 103.621773140322
2.71836734693871 103.177330907653
2.79183673469381 102.718983895069
2.86530612244891 102.248589248719
2.93877551020401 101.768004114751
3.01224489795911 101.279085639318
3.08571428571421 100.783690968568
3.15918367346931 100.28367724865
3.23265306122441 99.7809016257142
3.30612244897951 99.2772212459117
3.37959183673461 98.7744932553912
3.45306122448971 98.2745748003024
3.52653061224481 97.7793230267934
3.59999999999991 97.2905950810173
};
\addplot [semithick, white!20!black]
table {%
0 98.3566494297471
0.114897959183676 98.199423270451
0.229795918367351 98.0445540299835
0.344693877551027 97.8934415283834
0.459591836734703 97.747485585688
0.574489795918378 97.6080860219342
0.689387755102054 97.476642657161
0.80428571428573 97.3545553114051
0.919183673469406 97.2432238047051
1.03408163265308 97.1440479570982
1.14897959183676 97.058421311458
1.26387755102043 96.9864066519405
1.37877551020411 96.9250976642384
1.49367346938778 96.8711862955635
1.60857142857146 96.8213644931284
1.72346938775514 96.7723242041446
1.83836734693881 96.7207573758245
1.95326530612249 96.66335595538
2.06816326530616 96.5968118900228
2.18306122448984 96.5178171269655
2.29795918367351 96.4231183067856
2.41285714285719 96.3115882746505
2.52775510204087 96.1848618906955
2.64265306122454 96.0447586051637
2.75755102040822 95.8930978682975
2.87244897959189 95.7316991303419
2.98734693877557 95.5623818415388
3.10224489795924 95.386965452133
3.21714285714292 95.2072694123674
3.3320408163266 95.0251131724844
3.44693877551027 94.8422179389967
3.56183673469395 94.6588349010857
3.67673469387762 94.4740836256824
3.7916326530613 94.2870545704658
3.90653061224497 94.0968381931128
4.02142857142865 93.9025249513003
4.13632653061233 93.7032053027073
4.251224489796 93.49796970501
4.36612244897968 93.2859086158875
4.48102040816335 93.0661124930166
4.59591836734703 92.8378060640445
4.7108163265307 92.6012063956217
4.82571428571438 92.356975323673
4.94061224489805 92.1057767820915
5.05551020408173 91.8482747047733
5.17040816326541 91.5851330256104
5.28530612244908 91.317015678499
5.40020408163276 91.0445865973321
5.51510204081643 90.7685097160029
5.63000000000011 90.4894489684074
};
\addplot [semithick, white!20!black]
table {%
0 45.5621740246808
0.267755102040816 45.1929035672979
0.535510204081633 44.904095757842
0.803265306122449 44.6824464625739
1.07102040816327 44.5146515477541
1.33877551020408 44.3874068796433
1.6065306122449 44.2874083245022
1.87428571428571 44.2013517485914
2.14204081632653 44.1159330181715
2.40979591836735 44.0178479995031
2.67755102040816 43.893792558847
2.94530612244898 43.7307558312271
3.2130612244898 43.5249266080137
3.48081632653061 43.2833072035133
3.74857142857143 43.0135186197706
4.01632653061225 42.7231818588305
4.28408163265306 42.4199179227375
4.55183673469388 42.1113478135365
4.8195918367347 41.8050925332724
5.08734693877551 41.5087730839897
5.35510204081633 41.2300104677333
5.62285714285714 40.9764256865479
5.89061224489796 40.7539722876679
6.15836734693878 40.5589029749861
6.42612244897959 40.3839715476991
6.69387755102041 40.2219271042022
6.96163265306123 40.0655187428907
7.22938775510204 39.90749556216
7.49714285714286 39.7406066604054
7.76489795918368 39.5576011360222
8.03265306122449 39.3512280874057
8.30040816326531 39.1142366129513
8.56816326530613 38.839379636944
8.83591836734694 38.5196744788506
9.10367346938776 38.148569669073
9.37142857142858 37.7195534502921
9.63918367346939 37.2261140651884
9.90693877551021 36.661739756443
10.174693877551 36.0199187667363
10.4424489795918 35.2941393387492
10.7102040816327 34.4778897151625
10.9779591836735 33.5646581386568
11.2457142857143 32.5479328519128
11.5134693877551 31.4245302911075
11.7812244897959 30.2199999919827
12.0489795918367 28.974587160438
12.3167346938776 27.7286555358942
12.5844897959184 26.5225688577729
12.8522448979592 25.3966908654956
13.12 24.3913852984832
};
\addplot [semithick, white!20!black]
table {%
0 107.439851393138
0.108843537414964 107.383667521073
0.217687074829929 107.324779397205
0.326530612244893 107.262895011015
0.435374149659858 107.197722351981
0.544217687074822 107.128969409582
0.653061224489787 107.056344173297
0.761904761904751 106.979554632605
0.870748299319716 106.898308776985
0.97959183673468 106.812314595916
1.08843537414964 106.721280078876
1.19727891156461 106.624913215345
1.30612244897957 106.522921994801
1.41496598639454 106.415014406724
1.5238095238095 106.300898440592
1.63265306122447 106.180282085885
1.74149659863943 106.052873332081
1.8503401360544 105.918380168659
1.95918367346936 105.77651090536
2.06802721088432 105.627226530997
2.17687074829929 105.471194011532
2.28571428571425 105.309202657507
2.39455782312922 105.142041779466
2.50340136054418 104.97050068795
2.61224489795915 104.795368693505
2.72108843537411 104.617435106671
2.82993197278908 104.437489237993
2.93877551020404 104.256320398014
3.047619047619 104.074717897277
3.15646258503397 103.893471046324
3.26530612244893 103.7133691557
3.3741496598639 103.535201535947
3.48299319727886 103.359757497608
3.59183673469383 103.187826351226
3.70068027210879 103.020197407344
3.80952380952376 102.857659976506
3.91836734693872 102.701003369254
4.02721088435368 102.551016896132
4.13605442176865 102.408489867683
4.24489795918361 102.27421159445
4.35374149659858 102.148971386976
4.46258503401354 102.033558555803
4.57142857142851 101.928762411476
4.68027210884347 101.835372264537
4.78911564625844 101.754177425529
4.8979591836734 101.685967204995
5.00680272108836 101.631530913479
5.11564625850333 101.591657861524
5.22448979591829 101.567137359672
5.33333333333326 101.558758718467
};
\addplot [semithick, white!20!black]
table {%
0 107.173249389072
0.0679591836734772 107.104471467914
0.135918367346954 107.031463677564
0.203877551020432 106.953681660352
0.271836734693909 106.870581058603
0.339795918367386 106.781617514643
0.407755102040863 106.6862466708
0.47571428571434 106.584081543146
0.543673469387817 106.475430264412
0.611632653061295 106.360792826777
0.679591836734772 106.240669228695
0.747551020408249 106.115559468619
0.815510204081726 105.985963545002
0.883469387755203 105.852381456297
0.951428571428681 105.715313200959
1.01938775510216 105.575258777438
1.08734693877563 105.432718184191
1.15530612244911 105.28819141967
1.22326530612259 105.142178482327
1.29122448979607 104.995179370617
1.35918367346954 104.847694082993
1.42714285714302 104.700222617907
1.4951020408165 104.553264973813
1.56306122448998 104.407321149165
1.63102040816345 104.262891142417
1.69897959183693 104.120474952019
1.76693877551041 103.980572576428
1.83489795918388 103.843684014095
1.90285714285736 103.710287491657
1.97081632653084 103.580319399421
2.03877551020432 103.453150433327
2.10673469387779 103.328124743113
2.17469387755127 103.204586478518
2.24265306122475 103.081879789283
2.31061224489822 102.959348825146
2.3785714285717 102.836337735845
2.44653061224518 102.712190671121
2.51448979591866 102.586251780713
2.58244897959213 102.457865214357
2.65040816326561 102.326375121796
2.71836734693909 102.191125652766
2.78632653061256 102.051460957008
2.85428571428604 101.90672518426
2.92224489795952 101.756262484262
2.990204081633 101.599417006753
3.05816326530647 101.43553290147
3.12612244897995 101.263954318156
3.19408163265343 101.084025406546
3.2620408163269 100.895090316381
3.33000000000038 100.6964931974
};
\addplot [semithick, white!20!black]
table {%
0 54.7187080519513
0.0733985005413531 54.715564051597
0.146797001082706 54.7063296172556
0.220195501624059 54.6913213955401
0.293594002165413 54.6708560330636
0.366992502706766 54.6452501764394
0.440391003248119 54.6148204722804
0.513789503789472 54.5798835671998
0.587188004330825 54.5407561078107
0.660586504872178 54.4977547407262
0.733985005413532 54.4511961125593
0.807383505954885 54.4013968699234
0.880782006496238 54.3486736594313
0.954180507037591 54.2933431276963
1.02757900757894 54.2357219213315
1.1009775081203 54.1761266869498
1.17437600866165 54.1148740711644
1.247774509203 54.0522807205886
1.32117300974436 53.9886632818354
1.39457151028571 53.9243384015178
1.46797001082706 53.8596227262491
1.54136851136842 53.7948329026422
1.61476701190977 53.7302855773101
1.68816551245112 53.6662973968664
1.76156401299248 53.6031850079239
1.83496251353383 53.5412650570957
1.90836101407518 53.4808541909949
1.98175951461653 53.4222690562346
2.05515801515789 53.3658262994279
2.12855651569924 53.3118031581346
2.20195501624059 53.2600427387476
2.27535351678195 53.2101190530219
2.3487520173233 53.1616020966735
2.42215051786465 53.1140618654186
2.49554901840601 53.067068354973
2.56894751894736 53.0201915610531
2.64234601948871 52.9730014793747
2.71574452003007 52.925068105654
2.78914302057142 52.875961435607
2.86254152111277 52.8252514649497
2.93594002165413 52.7725081893981
3.00933852219548 52.7173016046684
3.08273702273683 52.6592017064766
3.15613552327819 52.5977784905389
3.22953402381954 52.5326019525711
3.30293252436089 52.4632420882893
3.37633102490224 52.3892688934094
3.4497295254436 52.310252363648
3.52312802598495 52.2257624947207
3.5965265265263 52.1353692823437
};
\addplot [semithick, white!20!black]
table {%
0 56.182905104932
0.214285714285714 55.927194999314
0.428571428571429 55.7282919170863
0.642857142857143 55.5798805476232
0.857142857142857 55.4756455802991
1.07142857142857 55.4092717044878
1.28571428571429 55.3744436095637
1.5 55.3648459849011
1.71428571428571 55.3741635198742
1.92857142857143 55.3960809038572
2.14285714285714 55.4242838810419
2.35714285714286 55.4526818169304
2.57142857142857 55.4756830057041
2.78571428571429 55.4877632498642
3 55.4833983519122
3.21428571428571 55.4570641143493
3.42857142857143 55.4032363396768
3.64285714285714 55.3163908303961
3.85714285714286 55.1910033890089
4.07142857142857 55.021549818016
4.28571428571429 54.8026618144149
4.5 54.5350314737327
4.71428571428571 54.227223563542
4.92857142857143 53.8883289953393
5.14285714285714 53.5274386806207
5.35714285714286 53.1536435308834
5.57142857142857 52.7760344576223
5.78571428571429 52.4037023723344
6 52.0457381865161
6.21428571428571 51.7112328116637
6.42857142857143 51.4087162155089
6.64285714285714 51.1383249850083
6.85714285714286 50.8937344659777
7.07142857142857 50.6684537986715
7.28571428571429 50.4559921233456
7.5 50.2498585802556
7.71428571428571 50.0435623096568
7.92857142857143 49.8306124518048
8.14285714285714 49.6045181469548
8.35714285714286 49.3587885353628
8.57142857142857 49.0879412847708
8.78571428571428 48.7939477113832
9 48.4821198787063
9.21428571428571 48.1577856084879
9.42857142857143 47.8262727224763
9.64285714285714 47.4929090424194
9.85714285714286 47.1630223900661
10.0714285714286 46.8419405871629
10.2857142857143 46.5349914554587
10.5 46.2475028167014
};
\addplot [semithick, white!20!black]
table {%
0 72.3526215949654
0.0440816326530629 72.2493517400345
0.0881632653061258 72.0942317280801
0.132244897959189 71.897214669939
0.176326530612252 71.6682536764473
0.220408163265314 71.4173018584429
0.264489795918377 71.1543123267621
0.30857142857144 70.8892381922414
0.352653061224503 70.6320325657176
0.396734693877566 70.3926485580274
0.440816326530629 70.1810189732117
0.484897959183692 70.0027715745575
0.528979591836755 69.8539290108433
0.573061224489818 69.7292142959014
0.617142857142881 69.6233504435649
0.661224489795944 69.5310604676668
0.705306122449006 69.4470673820402
0.749387755102069 69.3660942005181
0.793469387755132 69.2828639369332
0.837551020408195 69.192099605119
0.881632653061258 69.0886136279656
0.925714285714321 68.9706942054662
0.969795918367384 68.8411446950079
1.01387755102045 68.7030702095463
1.05795918367351 68.5595758620368
1.10204081632657 68.4137667654348
1.14612244897964 68.2687480326965
1.1902040816327 68.1276247767771
1.23428571428576 67.9935021106321
1.27836734693882 67.8694851472172
1.32244897959189 67.7585009035736
1.36653061224495 67.6608115541756
1.41061224489801 67.5746278724116
1.45469387755108 67.4981078625112
1.49877551020414 67.4294095287033
1.5428571428572 67.3666908752169
1.58693877551026 67.3081099062811
1.63102040816333 67.2518246261248
1.67510204081639 67.1959930389774
1.71918367346945 67.1387731490675
1.76326530612252 67.0785785743487
1.80734693877558 67.0157120779525
1.85142857142864 66.9513231434717
1.8955102040817 66.8865652484633
1.93959183673477 66.8225918704843
1.98367346938783 66.7605564870917
2.02775510204089 66.7016125758425
2.07183673469396 66.6469136142939
2.11591836734702 66.5976130800029
2.16000000000008 66.5548644505265
};
\addplot [semithick, white!20!black]
table {%
0 93.6721294945298
0.0566580866580826 93.6670748364212
0.113316173316165 93.6515188901763
0.169974259974248 93.624135098888
0.22663234663233 93.5850633853686
0.283290433290413 93.5360968451228
0.339948519948495 93.479116937231
0.396606606606578 93.4160051207736
0.453264693264661 93.3486428548302
0.509922779922743 93.2789115984823
0.566580866580826 93.2086928108096
0.623238953238908 93.1398679508924
0.679897039896991 93.074318477811
0.736555126555073 93.0139258506453
0.793213213213156 92.9605715284766
0.849871299871239 92.9157989817026
0.906529386529321 92.8789551908722
0.963187473187404 92.8485093774639
1.01984555984549 92.8229284567584
1.07650364650357 92.8006793440369
1.13316173316165 92.78022895458
1.18981981981973 92.7600442036686
1.24647790647782 92.7385920065835
1.3031359931359 92.7143392786053
1.35979407979398 92.6857529350153
1.41645216645206 92.6512998910941
1.47311025311015 92.609452470883
1.52976833976823 92.55951512669
1.58642642642631 92.5025140830218
1.64308451308439 92.4396889745866
1.69974259974248 92.372279436093
1.75640068640056 92.3015251022497
1.81305877305864 92.2286656077645
1.86971685971673 92.1549405873473
1.92637494637481 92.081589675706
1.98303303303289 92.0098525075492
2.03969111969097 91.9409687175856
2.09634920634906 91.8761779405232
2.15300729300714 91.8162635529504
2.20966537966522 91.7555346274949
2.2663234663233 91.6834907348342
2.32298155298139 91.5895173974813
2.37963963963947 91.4630001379485
2.43629772629755 91.2933244787509
2.49295581295563 91.0698759424007
2.54961389961372 90.7820400514112
2.6062719862718 90.4192023282957
2.66293007292988 89.9707482955633
2.71958815958796 89.4260634757344
2.77624624624605 88.774533391319
};
\addplot [semithick, white!20!black]
table {%
0 53.2017501633263
0.0893877551020407 53.0001195622982
0.178775510204081 52.8136775708732
0.268163265306122 52.6412436602184
0.357551020408163 52.4816373015008
0.446938775510204 52.3336779658877
0.536326530612244 52.196185124546
0.625714285714285 52.067978248643
0.715102040816326 51.9478768093458
0.804489795918367 51.8347002778214
0.893877551020407 51.7272681252371
0.983265306122448 51.6243998227598
1.07265306122449 51.5249148415568
1.16204081632653 51.4276326527952
1.25142857142857 51.331372727642
1.34081632653061 51.2349545372644
1.43020408163265 51.1371975528295
1.51959183673469 51.0369212455044
1.60897959183673 50.9329450864563
1.69836734693877 50.8240887854268
1.78775510204081 50.7095515315358
1.87714285714286 50.5896814063719
1.9665306122449 50.4650430842829
2.05591836734694 50.3362012396166
2.14530612244898 50.2037205467208
2.23469387755102 50.0681656799434
2.32408163265306 49.9301013136321
2.4134693877551 49.7900921221349
2.50285714285714 49.6487027797994
2.59224489795918 49.5064979609735
2.68163265306122 49.3640423400051
2.77102040816326 49.2219005912419
2.8604081632653 49.080637389032
2.94979591836734 48.9408174077228
3.03918367346938 48.8030053216624
3.12857142857143 48.6677658051985
3.21795918367347 48.5356635326791
3.30734693877551 48.4072631784519
3.39673469387755 48.2831274298738
3.48612244897959 48.1633202066128
3.57551020408163 48.0467518901247
3.66489795918367 47.9321702679529
3.75428571428571 47.8183231276405
3.84367346938775 47.7039582567306
3.93306122448979 47.5878234427664
4.02244897959183 47.4686664732911
4.11183673469387 47.3452351358477
4.20122448979591 47.2162772179797
4.29061224489795 47.0805405072298
4.38 46.9367727911415
};
\addplot [semithick, white!20!black]
table {%
0 50.8861013688464
0.24795918367347 50.7734958779621
0.49591836734694 50.4684629461345
0.74387755102041 49.9956825831836
0.99183673469388 49.3798347989295
1.23979591836735 48.6455996031922
1.48775510204082 47.8176570057916
1.73571428571429 46.9206870165477
1.98367346938776 45.9793696452804
2.23163265306123 45.0183849018099
2.4795918367347 44.0624127959561
2.72755102040817 43.1361333375389
2.97551020408164 42.2642265363784
3.22346938775511 41.4683845061328
3.47142857142858 40.7501401716189
3.71938775510205 40.1026923189337
3.96734693877552 39.5192133114217
4.21530612244899 38.9928755124277
4.46326530612246 38.5168512852963
4.71122448979593 38.0843129933722
4.9591836734694 37.6884330000001
5.20714285714287 37.3223836685246
5.45510204081634 36.9793373622903
5.70306122448981 36.652466444642
5.95102040816328 36.3349432789243
6.19897959183675 36.0199374171337
6.44693877551022 35.7005845909386
6.69489795918369 35.369998159428
6.94285714285716 35.0212910831628
7.19081632653063 34.647576322704
7.4387755102041 34.2419668386126
7.68673469387757 33.7975755914494
7.93469387755104 33.3075155417753
8.18265306122451 32.7648996501513
8.43061224489798 32.1628408771385
8.67857142857145 31.4944521832977
8.92653061224492 30.7528465291898
9.17448979591839 29.9314857329552
9.42244897959186 29.0324115620181
9.67040816326533 28.0665666186801
9.9183673469388 27.0453065670652
10.1663265306123 25.9799870712975
10.4142857142857 24.8819637955008
10.6622448979592 23.7625924037995
10.9102040816327 22.6332285603174
11.1581632653062 21.5052279291786
11.4061224489796 20.3899461745071
11.6540816326531 19.2987389604266
11.9020408163266 18.2429619510617
12.15 17.2339561678047
};
\addplot [semithick, white!20!black]
table {%
0 43.6356920934673
0.0706122448979599 43.1340686249208
0.14122448979592 42.5946685419145
0.21183673469388 42.0216964921124
0.28244897959184 41.4193571231839
0.3530612244898 40.7918550827987
0.42367346938776 40.1433950186202
0.494285714285719 39.4781815783182
0.564897959183679 38.8004194095602
0.635510204081639 38.114313160016
0.706122448979599 37.4240619431894
0.776734693877559 36.732691630787
0.847346938775519 36.0406104370947
0.917959183673479 35.3478723901869
0.988571428571439 34.6545315181446
1.0591836734694 33.9606418490488
1.12979591836736 33.2662574109738
1.20040816326532 32.5714322320008
1.27102040816328 31.8762203402107
1.34163265306124 31.1806757636779
1.4122448979592 30.4847917802792
1.48285714285716 29.7862000037003
1.55346938775512 29.0794641623167
1.62408163265308 28.3589429525576
1.69469387755104 27.6189950708591
1.765306122449 26.8539792136576
1.83591836734696 26.0582540773818
1.90653061224492 25.2261783584682
1.97714285714288 24.3521107533533
2.04775510204084 23.4304099584654
2.1183673469388 22.4562010228858
2.18897959183676 21.4360759019326
2.25959183673472 20.3854537980574
2.33020408163268 19.3199809811519
2.40081632653064 18.2553037211179
2.4714285714286 17.2070682878571
2.54204081632656 16.1909209512613
2.61265306122452 15.2225079812323
2.68326530612248 14.3174756476713
2.75387755102044 13.4914702204709
2.8244897959184 12.7582571362537
2.89510204081636 12.1177012981891
2.96571428571432 11.563437349204
3.03632653061228 11.0890705442117
3.10693877551024 10.6882061381233
3.1775510204082 10.3544493858508
3.24816326530616 10.0814055423032
3.31877551020412 9.86267986239264
3.38938775510208 9.69187760103073
3.46000000000004 9.56260401312743
};
\addplot [semithick, white!20!black]
table {%
0 49.7781419561068
0.0855788441502747 49.7683382455017
0.171157688300549 49.738622467772
0.256736532450824 49.6888534110108
0.342315376601099 49.618889863311
0.427894220751374 49.5285906127661
0.513473064901648 49.4178144474689
0.599051909051923 49.2864201555131
0.684630753202198 49.1348493528843
0.770209597352472 48.966071821878
0.855788441502747 48.7837426857562
0.941367285653022 48.5915170802463
1.0269461298033 48.3930501410776
1.11252497395357 48.1919970039775
1.19810381810385 47.9920128046754
1.28368266225412 47.796752678899
1.3692615064044 47.6098717623763
1.45484035055467 47.4350251908365
1.54041919470494 47.2757570028651
1.62599803885522 47.1327954855435
1.71157688300549 47.0039003557762
1.79715572715577 46.8866896428288
1.88273457130604 46.7787813759657
1.96831341545632 46.6777935844511
2.05389225960659 46.5813442975505
2.13947110375687 46.487051544528
2.22504994790714 46.3925333546488
2.31062879205742 46.2954077571771
2.39620763620769 46.1932927813783
2.48178648035797 46.0846363170352
2.56736532450824 45.9713429577686
2.65294416865852 45.8562171957536
2.73852301280879 45.7420635252932
2.82410185695906 45.6316864406917
2.90968070110934 45.527890436252
2.99525954525961 45.4334800062784
3.08083838940989 45.351259645074
3.16641723356016 45.2840338469425
3.25199607771044 45.2346071061877
3.33757492186071 45.2055483795152
3.42315376601099 45.1938335452379
3.50873261016126 45.1907460071042
3.59431145431154 45.1873137500965
3.67989029846181 45.1745647591973
3.76546914261209 45.1435270193889
3.85104798676236 45.0852285156541
3.93662683091264 44.9906972329748
4.02220567506291 44.8509611563342
4.10778451921318 44.6570482707138
4.19336336336346 44.3999865610971
};
\addplot [semithick, white!20!black]
table {%
0 88.9119142711392
0.0253061224489798 88.8392299946697
0.0506122448979596 88.7662433944368
0.0759183673469393 88.6929645421113
0.101224489795919 88.619403509362
0.126530612244899 88.5455703678578
0.151836734693879 88.4714751892694
0.177142857142858 88.3971280452657
0.202448979591838 88.3225390075162
0.227755102040818 88.2477181476896
0.253061224489798 88.1726755374568
0.278367346938778 88.0974212484866
0.303673469387757 88.0219653524479
0.328979591836737 87.9463179210113
0.354285714285717 87.8704890258459
0.379591836734697 87.7944887386203
0.404897959183676 87.7183271310053
0.430204081632656 87.6420142746699
0.455510204081636 87.5655602412828
0.480816326530616 87.4889751025149
0.506122448979596 87.4122689300348
0.531428571428575 87.3354517955123
0.556734693877555 87.2585337706159
0.582040816326535 87.1815249270167
0.607346938775515 87.1044353363833
0.632653061224494 87.0272750703846
0.657959183673474 86.9500542006914
0.683265306122454 86.8727827989726
0.708571428571434 86.7954709368969
0.733877551020414 86.7181286861352
0.759183673469393 86.6407661183562
0.784489795918373 86.5633933052296
0.809795918367353 86.486020318424
0.835102040816333 86.4086572296102
0.860408163265312 86.3313141104572
0.885714285714292 86.2540010326337
0.911020408163272 86.1767280678106
0.936326530612252 86.0995052876566
0.961632653061232 86.0223427638406
0.986938775510211 85.9452505680334
1.01224489795919 85.8682387719039
1.03755102040817 85.7913174471206
1.06285714285715 85.7144966653547
1.08816326530613 85.6377864982749
1.11346938775511 85.5611970175505
1.13877551020409 85.4847382948505
1.16408163265307 85.4084204018457
1.18938775510205 85.3322534102049
1.21469387755103 85.2562473915969
1.24000000000001 85.1804124176925
};
\addplot [semithick, white!20!black]
table {%
0 79.3580205393976
0.129016363302076 79.0292813392406
0.258032726604151 78.6805308847068
0.387049089906227 78.3124270388433
0.516065453208303 77.9256276646969
0.645081816510379 77.5207906253145
0.774098179812454 77.098573783743
0.90311454311453 76.6596350030295
1.03213090641661 76.2046321462207
1.16114726971868 75.7342230763637
1.29016363302076 75.2490656565052
1.41917999632283 74.7498177496922
1.54819635962491 74.2371372189718
1.67721272292698 73.7116819273906
1.80622908622906 73.1741097379958
1.93524544953114 72.6250785138341
2.06426181283321 72.0652461179525
2.19327817613529 71.495270413398
2.32229453943736 70.9158092632174
2.45131090273944 70.3275205304576
2.58032726604152 69.7310620781656
2.70934362934359 69.1270917693882
2.83835999264567 68.5162674671724
2.96737635594774 67.8992470345652
3.09639271924982 67.2766883346133
3.22540908255189 66.6492492303627
3.35442544585397 66.0177561947698
3.48344180915605 65.3842449102302
3.61245817245812 64.7512789120414
3.7414745357602 64.1214239373534
3.87049089906227 63.4972457233161
3.99950726236435 62.8813100070795
4.12852362566642 62.2761825257936
4.2575399889685 61.6844290166084
4.38655635227058 61.108615216674
4.51557271557265 60.5513068631402
4.64458907887473 60.0150696931572
4.7736054421768 59.5024694438749
4.90262180547888 59.0160718524433
5.03163816878095 58.5584426560124
5.16065453208303 58.1321475917323
5.28967089538511 57.7397523967528
5.41868725868718 57.3838228082241
5.54770362198926 57.0669245632961
5.67671998529133 56.7916233991188
5.80573634859341 56.5604850528422
5.93475271189548 56.3760752616163
6.06376907519756 56.2409597625911
6.19278543849964 56.1577042929166
6.32180180180171 56.1288745897428
};
\addplot [semithick, white!20!black]
table {%
0 89.0823224867624
0.111080468223324 89.0751541349374
0.222160936446648 89.0531560957096
0.333241404669971 89.0167668367809
0.444321872893295 88.9664248258527
0.555402341116619 88.9025685306271
0.666482809339943 88.8256364188056
0.777563277563266 88.7359940938524
0.88864374578659 88.6330963271757
0.999724214009914 88.5157784786548
1.11080468223324 88.3828640582389
1.22188515045656 88.233176575878
1.33296561867989 88.0655395415209
1.44404608690321 87.8787764651176
1.55512655512653 87.6717108566174
1.66620702334986 87.4431662259691
1.77728749157318 87.1919660831231
1.8883679597965 86.9169339380278
1.99944842801983 86.6169689664818
2.11052889624315 86.2940226654017
2.22160936446648 85.9540769291286
2.3326898326898 85.6033896832253
2.44377030091312 85.2482188532519
2.55485076913645 84.8948223647712
2.66593123735977 84.5494581433439
2.77701170558309 84.2183841145307
2.88809217380642 83.9078582038944
2.99917264202974 83.6241383369958
3.11025311025307 83.3734824393957
3.22133357847639 83.1621428150297
3.33241404669971 82.9933980043969
3.44349451492304 82.8626745593111
3.55457498314636 82.7641181008392
3.66565545136968 82.6918742500485
3.77673591959301 82.6400886280066
3.88781638781633 82.6029068557806
3.99889685603965 82.574474554438
4.10997732426298 82.5489373450461
4.2210577924863 82.5204408486721
4.33213826070963 82.4831306863835
4.44321872893295 82.4311524792474
4.55429919715627 82.3600798213593
4.6653796653796 82.2735087851119
4.77646013360292 82.177835011219
4.88754060182625 82.0794570413913
4.99862107004957 81.9847734173392
5.10970153827289 81.9001826807727
5.22078200649622 81.8320833734027
5.33186247471954 81.7868740369393
5.44294294294286 81.770953213093
};
\addplot [semithick, white!20!black]
table {%
0 105.7688416238
0.0128571428571451 105.765172360696
0.0257142857142902 105.76071524894
0.0385714285714353 105.755480215654
0.0514285714285803 105.749477187963
0.0642857142857254 105.74271609299
0.0771428571428705 105.735206857858
0.0900000000000156 105.726959409692
0.102857142857161 105.717983675614
0.115714285714306 105.708289582749
0.128571428571451 105.69788705822
0.141428571428596 105.686786029151
0.154285714285741 105.674996422666
0.167142857142886 105.662528165887
0.180000000000031 105.64939118594
0.192857142857176 105.635595409947
0.205714285714321 105.621150765031
0.218571428571466 105.606067178318
0.231428571428612 105.59035457693
0.244285714285757 105.57402288799
0.257142857142902 105.557082038624
0.270000000000047 105.539541955953
0.282857142857192 105.521412567103
0.295714285714337 105.502703799196
0.308571428571482 105.483425579356
0.321428571428627 105.463587834707
0.334285714285772 105.443200492373
0.347142857142917 105.422273479476
0.360000000000062 105.400816723141
0.372857142857207 105.378840150492
0.385714285714353 105.356353688652
0.398571428571498 105.333367264744
0.411428571428643 105.309890805894
0.424285714285788 105.285934239222
0.437142857142933 105.261507491854
0.450000000000078 105.236620490914
0.462857142857223 105.211283163524
0.475714285714368 105.18550543681
0.488571428571513 105.159297237893
0.501428571428658 105.132668493898
0.514285714285803 105.105629131948
0.527142857142949 105.078189079168
0.540000000000094 105.05035826268
0.552857142857239 105.022146609609
0.565714285714384 104.993564047078
0.578571428571529 104.96462050221
0.591428571428674 104.93532590213
0.604285714285819 104.90569017396
0.617142857142964 104.875723244826
0.630000000000109 104.845435041849
};
\addplot [semithick, white!20!black]
table {%
0 64.0421267619347
0.0825874854446265 64.0258410425686
0.165174970889253 63.9792969473301
0.247762456333879 63.9068909850654
0.330349941778506 63.8130196646211
0.412937427223132 63.7020794948431
0.495524912667759 63.5784669845778
0.578112398112385 63.4465786426721
0.660699883557012 63.3107950754857
0.743287369001638 63.1738358139677
0.825874854446265 63.0353269789287
0.908462339890891 62.8945571164163
0.991049825335518 62.7508147724781
1.07363731078014 62.6033884931603
1.15622479622477 62.4515668245109
1.2388122816694 62.2946383125762
1.32139976711402 62.1318915034042
1.40398725255865 61.962614943041
1.48657473800328 61.7860971775342
1.5691622234479 61.6019166152497
1.65174970889253 61.411003115087
1.73433719433716 61.2146800909838
1.81692467978178 61.0142710018674
1.89951216522641 60.8110993066666
1.98209965067104 60.6064884643102
2.06468713611566 60.4017619337256
2.14727462156029 60.198243173842
2.22986210700491 59.9972556435868
2.31244959244954 59.8001228018887
2.39503707789417 59.6081455651016
2.47762456333879 59.4223276087169
2.56021204878342 59.2434626931482
2.64279953422805 59.0723401665211
2.72538701967267 58.9097493769631
2.8079745051173 58.7564796726013
2.89056199056193 58.6133204015618
2.97314947600655 58.4810609119721
3.05573696145118 58.3604905519582
3.13832444689581 58.2523986696478
3.22091193234043 58.1575455571102
3.30349941778506 58.0747140977255
3.38608690322969 57.9994794566412
3.46867438867431 57.9271234820713
3.55126187411894 57.8529280222308
3.63384935956356 57.7721749253339
3.71643684500819 57.6801460395953
3.79902433045282 57.5721232132298
3.88161181589744 57.4433882944513
3.96419930134207 57.2892231314752
4.0467867867867 57.1049095725151
};
\addplot [semithick, white!20!black]
table {%
0 111.739538575798
0.0398118526689925 111.733531932426
0.079623705337985 111.713942001842
0.119435558006978 111.6811313736
0.15924741067597 111.636349909916
0.199059263344963 111.580850445585
0.238871116013955 111.515885815404
0.278682968682948 111.442708854171
0.31849482135194 111.362572396683
0.358306674020933 111.276729277736
0.398118526689925 111.186432332127
0.437930379358918 111.092934394653
0.47774223202791 110.997488300112
0.517554084696903 110.901346883299
0.557365937365895 110.805762979013
0.597177790034888 110.711989422049
0.63698964270388 110.621279047206
0.676801495372873 110.534884689279
0.716613348041865 110.454059183066
0.756425200710858 110.380055363363
0.79623705337985 110.314060305131
0.836048906048843 110.256233680271
0.875860758717835 110.205921439943
0.915672611386828 110.162447241639
0.95548446405582 110.125134742851
0.995296316724813 110.09330760107
1.03510816939381 110.066289473787
1.0749200220628 110.043404018493
1.11473187473179 110.023974892679
1.15454372740078 110.007325753837
1.19435558006978 109.992780259458
1.23416743273877 109.979662067032
1.27397928540776 109.967294834051
1.31379113807675 109.955002218006
1.35360299074575 109.942107876389
1.39341484341474 109.92793546669
1.43322669608373 109.911808646401
1.47303854875272 109.893051073013
1.51285040142172 109.870992107043
1.55266225409071 109.84525861139
1.5924741067597 109.815914572317
1.63228595942869 109.783058905943
1.67209781209769 109.746790528384
1.71190966476668 109.70720835576
1.75172151743567 109.664411304189
1.79153337010466 109.61849828979
1.83134522277366 109.56956822868
1.87115707544265 109.517720036979
1.91096892811164 109.463052630804
1.95078078078063 109.405664926275
};
\addplot [semithick, white!20!black]
table {%
0 47.0651114845144
0.0820408163265302 46.8763971843141
0.16408163265306 46.6869239851464
0.246122448979591 46.4966979525932
0.328163265306121 46.3057251522366
0.410204081632651 46.1140116496587
0.492244897959181 45.9215635104416
0.574285714285712 45.7283868001672
0.656326530612242 45.5344875844178
0.738367346938772 45.3398719287753
0.820408163265302 45.1445458988219
0.902448979591833 44.9485155601396
0.984489795918363 44.7517869783105
1.06653061224489 44.5543662189167
1.14857142857142 44.3562593475402
1.23061224489795 44.157472429763
1.31265306122448 43.9580115311673
1.39469387755101 43.7578827173352
1.47673469387754 43.5570920538486
1.55877551020407 43.3556456062896
1.6408163265306 43.1535494402404
1.72285714285713 42.9508096212831
1.80489795918367 42.7474322149997
1.8869387755102 42.5434232869722
1.96897959183673 42.3387889027827
2.05102040816326 42.1335351280133
2.13306122448979 41.9276680282461
2.21510204081632 41.7211936690631
2.29714285714285 41.5141181160464
2.37918367346938 41.3064474347781
2.46122448979591 41.0981876908402
2.54326530612244 40.8893449498147
2.62530612244897 40.6799252772839
2.7073469387755 40.4699347388298
2.78938775510203 40.2593794000343
2.87142857142856 40.0482653264797
2.95346938775509 39.836598583748
3.03551020408162 39.6243852374212
3.11755102040815 39.4116313530813
3.19959183673468 39.1983429963106
3.28163265306121 38.984526232691
3.36367346938774 38.7701871278046
3.44571428571427 38.5553317472335
3.5277551020408 38.3399661565596
3.60979591836733 38.1240964213653
3.69183673469386 37.9077286072324
3.77387755102039 37.6908687797431
3.85591836734692 37.4735230044795
3.93795918367345 37.2556973470236
4.01999999999998 37.0373978729574
};
\addplot [semithick, white!20!black]
table {%
0 45.5621740246808
0.267755102040816 45.1929035672979
0.535510204081633 44.904095757842
0.803265306122449 44.6824464625739
1.07102040816327 44.5146515477541
1.33877551020408 44.3874068796433
1.6065306122449 44.2874083245022
1.87428571428571 44.2013517485914
2.14204081632653 44.1159330181715
2.40979591836735 44.0178479995031
2.67755102040816 43.893792558847
2.94530612244898 43.7307558312271
3.2130612244898 43.5249266080137
3.48081632653061 43.2833072035133
3.74857142857143 43.0135186197706
4.01632653061225 42.7231818588305
4.28408163265306 42.4199179227375
4.55183673469388 42.1113478135365
4.8195918367347 41.8050925332724
5.08734693877551 41.5087730839897
5.35510204081633 41.2300104677333
5.62285714285714 40.9764256865479
5.89061224489796 40.7539722876679
6.15836734693878 40.5589029749861
6.42612244897959 40.3839715476991
6.69387755102041 40.2219271042022
6.96163265306123 40.0655187428907
7.22938775510204 39.90749556216
7.49714285714286 39.7406066604054
7.76489795918368 39.5576011360222
8.03265306122449 39.3512280874057
8.30040816326531 39.1142366129513
8.56816326530613 38.839379636944
8.83591836734694 38.5196744788506
9.10367346938776 38.148569669073
9.37142857142858 37.7195534502921
9.63918367346939 37.2261140651884
9.90693877551021 36.661739756443
10.174693877551 36.0199187667363
10.4424489795918 35.2941393387492
10.7102040816327 34.4778897151625
10.9779591836735 33.5646581386568
11.2457142857143 32.5479328519128
11.5134693877551 31.4245302911075
11.7812244897959 30.2199999919827
12.0489795918367 28.974587160438
12.3167346938776 27.7286555358942
12.5844897959184 26.5225688577729
12.8522448979592 25.3966908654956
13.12 24.3913852984832
};
\addplot [semithick, white!20!black]
table {%
0 49.5729313166941
0.185521235521236 49.1265601847081
0.371042471042473 48.6648464534263
0.556563706563709 48.2059326175161
0.742084942084946 47.7679611716452
0.927606177606182 47.3673304559987
1.11312741312742 47.0069023448966
1.29864864864866 46.6832092616321
1.48416988416989 46.3927476996119
1.66969111969113 46.132014152243
1.85521235521236 45.8975051129322
2.0407335907336 45.6857170750863
2.22625482625484 45.4931465321122
2.41177606177607 45.316289977417
2.59729729729731 45.151643904407
2.78281853281855 44.9957048064893
2.96833976833978 44.8449692796846
3.15386100386102 44.6962113294984
3.33938223938226 44.5470861367087
3.52490347490349 44.3954234439906
3.71042471042473 44.2390529940192
3.89594594594597 44.0758045294696
4.0814671814672 43.9035077930166
4.26698841698844 43.7199925273354
4.45250965250968 43.5230884751011
4.63803088803091 43.3106253789888
4.82355212355215 43.0804329816733
5.00907335907339 42.83034102583
5.19459459459462 42.558216891407
5.38011583011586 42.2636909760851
5.56563706563709 41.9488742691482
5.75115830115833 41.6160641783316
5.93667953667957 41.2675581113702
6.1222007722008 40.9056534759989
6.30772200772204 40.5326476799526
6.49324324324328 40.1508381309663
6.67876447876451 39.7625222367747
6.86428571428575 39.3699974051133
7.04980694980699 38.9755610437161
7.23532818532822 38.5815105603183
7.42084942084946 38.1904023616964
7.6063706563707 37.807853216908
7.79189189189193 37.4414780830662
7.97741312741317 37.0989263043904
8.1629343629344 36.7878472250996
8.34845559845564 36.5158901894134
8.53397683397688 36.2907045415503
8.71949806949811 36.1199396257299
8.90501930501935 36.0112447861713
9.09054054054059 35.9722693670938
};
\addplot [semithick, white!20!black]
table {%
0 64.1666194060546
0.428979591836734 64.0447488801708
0.857959183673469 63.7448390776417
1.2869387755102 63.3040671113254
1.71591836734694 62.7596100940798
2.14489795918367 62.1486451387631
2.57387755102041 61.5083493582331
3.00285714285714 60.875899865348
3.43183673469387 60.2884737729657
3.86081632653061 59.7832481939444
4.28979591836734 59.397316593202
4.71877551020408 59.1500390724141
5.14775510204081 59.0212102577182
5.57673469387755 58.9852713071034
6.00571428571428 59.0166633785588
6.43469387755102 59.0898276300733
6.86367346938775 59.179205219636
7.29265306122448 59.2592373052359
7.72163265306122 59.3043650448619
8.15061224489795 59.2890295965031
8.57959183673469 59.1880255561945
9.00857142857142 58.9898874240078
9.43755102040816 58.700998321336
9.86653061224489 58.328934222977
10.2955102040816 57.881271103729
10.7244897959184 57.3655849383899
11.1534693877551 56.7894517017578
11.5824489795918 56.1604473686308
12.0114285714286 55.4861479138069
12.4404081632653 54.774129312084
12.869387755102 54.0306097407098
13.2983673469388 53.2414907024717
13.7273469387755 52.3770338839272
14.1563265306122 51.4070986612478
14.585306122449 50.3015444106059
15.0142857142857 49.0302305081731
15.4432653061224 47.5630163301212
15.8722448979592 45.8697612526224
16.3012244897959 43.9203246518483
16.7302040816326 41.6845659039713
17.1591836734694 39.1420266411201
17.5881632653061 36.34380641836
18.0171428571428 33.3730772636141
18.4461224489796 30.3131624900561
18.8751020408163 27.2473854108583
19.304081632653 24.2590693391945
19.7330612244898 21.4315375882369
20.1620408163265 18.8481134711594
20.5910204081632 16.5921203011342
21.02 14.746881391335
};
\addplot [semithick, white!20!black]
table {%
0 98.2204789023527
0.0223458152029577 98.2135181078808
0.0446916304059154 98.1920500457309
0.0670374456088731 98.1558343558441
0.0893832608118308 98.1046306781635
0.111729076014789 98.0382558387765
0.134074891217746 97.9568982517991
0.156420706420704 97.8608948074802
0.178766521623662 97.750582785911
0.201112336826619 97.626299467184
0.223458152029577 97.4883821313838
0.245803967232535 97.3371680586057
0.268149782435492 97.1729945289335
0.29049559763845 96.9961988224634
0.312841412841408 96.8071182192782
0.335187228044366 96.6060899994708
0.357533043247323 96.3934514431357
0.379878858450281 96.1695398303538
0.402224673653239 95.934692441224
0.424570488856196 95.6892465558264
0.446916304059154 95.4334041425576
0.469262119262112 95.1666211678542
0.49160793446507 94.8880978986149
0.513953749668027 94.5970343743313
0.536299564870985 94.2926306344775
0.558645380073943 93.9740867185515
0.5809911952769 93.6406026660264
0.603337010479858 93.2913785164013
0.625682825682816 92.925614309148
0.648028640885773 92.5425100837591
0.670374456088731 92.1412658797287
0.692720271291689 91.7210817365263
0.715066086494647 91.2811576936549
0.737411901697604 90.8206937905822
0.759757716900562 90.3388900668128
0.78210353210352 89.8360366707883
0.804449347306477 89.3175389766624
0.826795162509435 88.7903014257399
0.849140977712393 88.2612286518738
0.87148679291535 87.7372252889597
0.893832608118308 87.2251959708508
0.916178423321266 86.7320453314423
0.938524238524224 86.2646780045883
0.960870053727181 85.8299986241737
0.983215868930139 85.4349118240805
1.0055616841331 85.0863222381672
1.02790749933605 84.7911345003222
1.05025331453901 84.5562532444078
1.07259912974197 84.3885831043078
1.09494494494493 84.2950287138898
};
\addplot [semithick, white!20!black]
table {%
0 97.7609439682926
0.15551020408164 97.427669135481
0.311020408163279 97.1079973661957
0.466530612244919 96.8030205967455
0.622040816326559 96.5138307634381
0.777551020408199 96.2415198025838
0.933061224489838 95.9871796504905
1.08857142857148 95.7519022434669
1.24408163265312 95.5366889597893
1.39959183673476 95.340207752627
1.5551020408164 95.1586448105107
1.71061224489804 94.9880660607019
1.86612244897968 94.8245374304606
2.02163265306132 94.6641248470471
2.17714285714296 94.5028942377221
2.3326530612246 94.336911529746
2.48816326530624 94.1622426503786
2.64367346938787 93.9749535268815
2.79918367346951 93.7711100865147
2.95469387755115 93.5467782565386
3.11020408163279 93.2980239642137
3.26571428571443 93.0209131368005
3.42122448979607 92.7119626740439
3.57673469387771 92.3732433681458
3.73224489795935 92.0105627664901
3.88775510204099 91.6297979749164
4.04326530612263 91.2368260992643
4.19877551020427 90.8375242453735
4.35428571428591 90.4377695190824
4.50979591836755 90.0434390262329
4.66530612244919 89.6604098726637
4.82081632653083 89.2945591642141
4.97632653061247 88.951764006724
5.13183673469411 88.6379015060329
5.28734693877575 88.3588487679796
5.44285714285739 88.1204828984055
5.59836734693903 87.9270863125969
5.75387755102067 87.7720734933762
5.90938775510231 87.6443251991293
6.06489795918395 87.5327071124749
6.22040816326559 87.4260849160318
6.37591836734723 87.3133242924197
6.53142857142887 87.183290924257
6.68693877551051 87.024850494163
6.84244897959215 86.8268686847564
6.99795918367379 86.5782111786564
7.15346938775543 86.2677436584809
7.30897959183707 85.8843318068506
7.46448979591871 85.4168413063838
7.62000000000035 84.8541378396995
};
\addplot [semithick, white!20!black]
table {%
0 85.8777601836353
0.100129925844213 85.8501155795793
0.200259851688427 85.7730736290363
0.30038977753264 85.6529275704385
0.400519703376853 85.4959706422179
0.500649629221067 85.3084960828059
0.60077955506528 85.0967971306362
0.700909480909494 84.8670487700573
0.801039406753707 84.6243511636835
0.90116933259792 84.3732311245535
1.00129925844213 84.1182100990697
1.10142918428635 83.8638095336345
1.20155911013056 83.6145508746501
1.30168903597477 83.374955568518
1.40181896181899 83.1495450616427
1.5019488876632 82.9428408004254
1.60207881350741 82.7593642312685
1.70220873935163 82.6036368005744
1.80233866519584 82.4787442268727
1.90246859104005 82.3807072165768
2.00259851688427 82.3033769656485
2.10272844272848 82.240604088996
2.20285836857269 82.1862392015284
2.30298829441691 82.1341329181538
2.40311822026112 82.0781358537806
2.50324814610533 82.0120986233173
2.60337807194955 81.9298718416723
2.70350799779376 81.8253061237541
2.80363792363797 81.6922554129667
2.90376784948219 81.5269782630889
3.0038977753264 81.3323734010417
3.10402770117061 81.1124760578954
3.20415762701483 80.8713214647214
3.30428755285904 80.6129448525907
3.40441747870325 80.3413814525746
3.50454740454747 80.0606664957441
3.60467733039168 79.7748352131704
3.70480725623589 79.4879228359234
3.80493718208011 79.2039645950768
3.90506710792432 78.9269917349933
4.00519703376854 78.6607700357539
4.10532695961275 78.4086379517211
4.20545688545696 78.1738952602746
4.30558681130118 77.9598417387942
4.40571673714539 77.7697771646596
4.5058466629896 77.60700131525
4.60597658883382 77.4748139679467
4.70610651467803 77.3765149001285
4.80623644052224 77.3154038891754
4.90636636636646 77.2947807124671
};
\addplot [semithick, white!20!black]
table {%
0 64.700899676715
0.0334130048415773 64.6913849935712
0.0668260096831546 64.6629186815382
0.100239014524732 64.6169965498169
0.133652019366309 64.5551144076085
0.167065024207886 64.4787680641141
0.200478029049464 64.3894533285349
0.233891033891041 64.288666010072
0.267304038732618 64.1779019179264
0.300717043574195 64.0586568612993
0.334130048415773 63.9324266493919
0.36754305325735 63.8007070914051
0.400956058098927 63.6649939965402
0.434369062940505 63.5267831739982
0.467782067782082 63.3875704329803
0.501195072623659 63.2488515826877
0.534608077465236 63.1121224323213
0.568021082306814 62.9788787910823
0.601434087148391 62.8506164681719
0.634847091989968 62.728827714213
0.668260096831546 62.6144615394039
0.701673101673123 62.5073451189791
0.7350861065147 62.4071668997068
0.768499111356277 62.3136153283551
0.801912116197855 62.2263788516924
0.835325121039432 62.1451459164868
0.868738125881009 62.0696049695064
0.902151130722586 61.9994444575195
0.935564135564164 61.9343528272942
0.968977140405741 61.8740185255987
1.00239014524732 61.8181299992012
1.0358031500889 61.7663756948699
1.06921615493047 61.718444059373
1.10262915977205 61.6740235394787
1.13604216461363 61.6328025819551
1.1694551694552 61.5944696335704
1.20286817429678 61.5587131410929
1.23628117913836 61.5252215512907
1.26969418397994 61.493683310932
1.30310718882151 61.4637870321265
1.33652019366309 61.4352702210093
1.36993319850467 61.4079872117694
1.40334620334625 61.3818094196496
1.43675920818782 61.3566082598925
1.4701722130294 61.332255147741
1.50358521787098 61.3086214984378
1.53699822271255 61.2855787272258
1.57041122755413 61.2629982493476
1.60382423239571 61.2407514800461
1.63723723723729 61.2187098345641
};
\addplot [semithick, white!20!black]
table {%
0 88.6304851097234
0.464040571183427 88.5886858388449
0.928081142366854 88.4692366529232
1.39212171355028 88.2761179695471
1.85616228473371 88.0133102063053
2.32020285591714 87.6847937807867
2.78424342710056 87.29454911058
3.24828399828399 86.8465566132738
3.71232456946742 86.3447967064572
4.17636514065085 85.7932498077188
4.64040571183427 85.1958963346473
5.1044462830177 84.5567167048316
5.56848685420113 83.8796913358605
6.03252742538455 83.168826678084
6.49656799656798 82.4291820070468
6.96060856775141 81.6672083394187
7.42464913893484 80.8894521670656
7.88868971011826 80.1024599818536
8.35273028130169 79.3127782756488
8.81677085248512 78.5269535403171
9.28081142366855 77.7515322677247
9.74485199485197 76.9930609497374
10.2088925660354 76.2580860782215
10.6729331372188 75.5531541450429
11.1369737084023 74.8848116420676
11.6010142795857 74.2596050611621
12.0650548507691 73.6840808941916
12.5290954219525 73.1630233200535
12.993135993136 72.6819228593262
13.4571765643194 72.2143666332216
13.9212171355028 71.7337664463588
14.3852577066862 71.213534103357
14.8492982778697 70.6270814088354
15.3133388490531 69.9478201674131
15.7773794202365 69.1491621837091
16.24141999142 68.2045192623429
16.7054605626034 67.0873032079333
17.1695011337868 65.7709258250997
17.6335417049702 64.2287989184612
18.0975822761537 62.4343342926368
18.5616228473371 60.3609437522459
19.0256634185205 57.992627761302
19.4897039897039 55.3592677754582
19.9537445608874 52.5031692686181
20.4177851320708 49.4666379319347
20.8818257032542 46.2919794565609
21.3458662744377 43.0214995336498
21.8099068456211 39.6975038543544
22.2739474168045 36.3622981098277
22.7379879879879 33.0581879912227
};
\addplot [semithick, white!20!black]
table {%
0 92.0411875561487
0.0241506812935397 92.0370870728901
0.0483013625870793 92.023378566834
0.072452043880619 91.9995781464888
0.0966027251741586 91.9652019203631
0.120753406467698 91.9197659969646
0.144904087761238 91.8627864848021
0.169054769054778 91.7937794923837
0.193205450348317 91.7122611282179
0.217356131641857 91.617747500813
0.241506812935397 91.5097547186784
0.265657494228936 91.3877989986654
0.289808175522476 91.2515661085399
0.313958856816016 91.1012543030096
0.338109538109555 90.9371582889004
0.362260219403095 90.7595727730379
0.386410900696635 90.5687924622499
0.410561581990174 90.3651120633584
0.434712263283714 90.148826283191
0.458862944577254 89.9202298285735
0.483013625870793 89.6796174063316
0.507164307164333 89.4272837232911
0.531314988457873 89.1635234862802
0.555465669751412 88.8886314021198
0.579616351044952 88.6029021776381
0.603767032338491 88.3066305196609
0.627917713632031 88.0001111350138
0.652068394925571 87.6836387305226
0.67621907621911 87.3575080130164
0.70036975751265 87.0220136893146
0.72452043880619 86.677450466246
0.748671120099729 86.3241130506364
0.772821801393269 85.9622961493117
0.796972482686809 85.5922944690974
0.821123163980348 85.2144027168231
0.845273845273888 84.8289155993073
0.869424526567428 84.4361278233794
0.893575207860967 84.036334095865
0.917725889154507 83.62982912359
0.941876570448047 83.2169076133802
0.966027251741586 82.7978642720652
0.990177933035126 82.3729938064629
1.01432861432867 81.9425909234031
1.03847929562221 81.5069503297114
1.06262997691575 81.0663667322137
1.08678065820928 80.6211348377357
1.11093133950282 80.1715493531074
1.13508202079636 79.7179049851462
1.1592327020899 79.2604964406819
1.18338338338344 78.7996184265405
};
\end{axis}

\end{tikzpicture}

	\caption{Speed of the lead vehicle during 100 observed scenarios. For plotting purposes, the starting time of each scenario is set to 0.}
	\label{fig:speed lvd observed}
\end{figure}

\begin{figure}
	\centering
	% This file was created by tikzplotlib v0.9.8.
\begin{tikzpicture}

\begin{axis}[
height=\figureheight,
scaled y ticks=false,
tick align=outside,
tick pos=left,
width=\figurewidth,
x grid style={white!69.0196078431373!black},
xlabel={Time [s]},
xmin=-1.10516033155451, xmax=23.2083669626447,
xtick style={color=black},
xticklabel style={align=center},
y grid style={white!69.0196078431373!black},
ylabel={Speed [km/h]},
ymin=0.331988973036484, ymax=138.721869773196,
ytick style={color=black},
yticklabel style={/pgf/number format/fixed,/pgf/number format/precision=3}
]
\addplot [semithick, white!20!black]
table {%
0 71.8758761972237
0.179771058384672 71.6185506015928
0.359542116769344 71.3504460821283
0.539313175154016 71.0732956548755
0.719084233538688 70.7890690551451
0.89885529192336 70.4998779378663
1.07862635030803 70.2078047935033
1.2583974086927 69.9148160106654
1.43816846707738 69.6224380826378
1.61793952546205 69.3310342225422
1.79771058384672 69.0388894090505
1.97748164223139 68.7433321531129
2.15725270061606 68.4435607267486
2.33702375900074 68.1394846650989
2.51679481738541 67.8309900784395
2.69656587577008 67.5179342140965
2.87633693415475 67.2002243017413
3.05610799253942 66.8778163751384
3.2358790509241 66.5506821018729
3.41565010930877 66.2188860707468
3.59542116769344 65.8829689633948
3.77519222607811 65.5438314737623
3.95496328446278 65.2018701112109
4.13473434284746 64.8569877412137
4.31450540123213 64.5088350344503
4.4942764596168 64.1569789015875
4.67404751800147 63.8010011096591
4.85381857638614 63.4405272743329
5.03358963477082 63.0750667334663
5.21336069315549 62.7040351775294
5.39313175154016 62.3266449980825
5.57290280992483 61.9417390406592
5.7526738683095 61.5481214492626
5.93244492669418 61.1445636032946
6.11221598507885 60.7300483041831
6.29198704346352 60.3036041269306
6.47175810184819 59.8642239994611
6.65152916023287 59.4110478896754
6.83130021861754 58.94338362251
7.01107127700221 58.460364360413
7.19084233538688 57.9609162507227
7.37061339377155 57.444337085219
7.55038445215623 56.9112759097276
7.7301555105409 56.3633906362908
7.90992656892557 55.8030893680342
8.08969762731024 55.2337120875944
8.26946868569491 54.658842482574
8.44923974407959 54.0827034519898
8.62901080246426 53.5102180898147
8.80878186084893 52.9464507004551
};
\addplot [semithick, white!20!black]
table {%
0 120.1051467205
0.191237520579876 119.880826749629
0.382475041159752 119.640048905062
0.573712561739628 119.385759329861
0.764950082319504 119.121192537977
0.95618760289938 118.849676216006
1.14742512347926 118.574495439338
1.33866264405913 118.298789037936
1.52990016463901 118.025117812806
1.72113768521888 117.754397986781
1.91237520579876 117.484836777024
2.10361272637864 117.213442538523
2.29485024695851 116.93938634075
2.48608776753839 116.662678656881
2.67732528811826 116.383237402609
2.86856280869814 116.100846714777
3.05980032927802 115.815336746247
3.25103784985789 115.526693802957
3.44227537043777 115.234867328588
3.63351289101764 114.939784861433
3.82475041159752 114.641947340508
4.0159879321774 114.342266073692
4.20722545275727 114.041185942397
4.39846297333715 113.738666725618
4.58970049391702 113.434409106646
4.7809380144969 113.128028259565
4.97217553507678 112.819200655232
5.16341305565665 112.507744321775
5.35465057623653 112.193375474165
5.5458880968164 111.875804914395
5.73712561739628 111.554635800304
5.92836313797615 111.228914558965
6.11960065855603 110.897432841717
6.31083817913591 110.558896346069
6.50207569971578 110.212202014634
6.69331322029566 109.856286571429
6.88455074087554 109.490095549134
7.07578826145541 109.112715727825
7.26702578203529 108.723373534657
7.45826330261516 108.320917521905
7.64950082319504 107.903455711574
7.84073834377492 107.469051251331
8.03197586435479 107.01704625239
8.22321338493467 106.547648671711
8.41445090551454 106.062006785361
8.60568842609442 105.562226263198
8.79692594667429 105.050664159002
8.98816346725417 104.530351844032
9.17940098783405 104.004948158512
9.37063850841392 103.478204861471
};
\addplot [semithick, white!20!black]
table {%
0 82.1937224997087
0.0739999426885574 82.0852673701685
0.147999885377115 81.9671678598704
0.221999828065672 81.8407786991635
0.29599977075423 81.7074829241312
0.369999713442787 81.5686658149759
0.443999656131345 81.4256765560589
0.517999598819902 81.2797046668651
0.591999541508459 81.1315845961648
0.665999484197017 80.9815363932233
0.739999426885574 80.8291047119269
0.813999369574132 80.6736885554568
0.887999312262689 80.5153832390812
0.961999254951246 80.354451759084
1.0359991976398 80.1911055272667
1.10999914032836 80.0254929560347
1.18399908301692 79.8577472801385
1.25799902570548 79.6879574164202
1.33199896839403 79.5161855718475
1.40599891108259 79.3425125559524
1.47999885377115 79.1672106596698
1.55399879645971 78.9905444588422
1.62799873914826 78.8125698367173
1.70199868183682 78.6332993613368
1.77599862452538 78.4527362713726
1.84999856721394 78.2708659672193
1.92399850990249 78.087661460838
1.99799845259105 77.9030929912788
2.07199839527961 77.7171450676032
2.14599833796817 77.5298273346819
2.21999828065672 77.3411976587943
2.29399822334528 77.1513805851715
2.36799816603384 76.9605779871015
2.44199810872239 76.7689454159499
2.51599805141095 76.5766098133126
2.58999799409951 76.3836731463063
2.66399793678807 76.1902445371482
2.73799787947662 75.9964819877228
2.81199782216518 75.8026043090093
2.88599776485374 75.6088268762122
2.9599977075423 75.4150325310967
3.03399765023085 75.2206081058353
3.10799759291941 75.0250343848877
3.18199753560797 74.828073790443
3.25599747829653 74.6296404567614
3.32999742098508 74.429731139088
3.40399736367364 74.2283688877179
3.4779973063622 74.0255545610691
3.55199724905076 73.8212806168493
3.62599719173931 73.615553148301
};
\addplot [semithick, white!20!black]
table {%
0 91.7119074084564
0.0214398417622654 91.681802349843
0.0428796835245308 91.6421223173856
0.0643195252867962 91.5941383585888
0.0857593670490616 91.5390488468074
0.107199208811327 91.4779793066535
0.128639050573592 91.4120144591731
0.150078892335858 91.342054610648
0.171518734098123 91.2686756749387
0.192958575860389 91.1920743861704
0.214398417622654 91.1124265916928
0.235838259384919 91.030151654669
0.257278101147185 90.9457968844011
0.27871794290945 90.8598133112758
0.300157784671716 90.7725799649224
0.321597626433981 90.684386247594
0.343037468196246 90.5954730229091
0.364477309958512 90.505999215836
0.385917151720777 90.4160709203093
0.407356993483043 90.3257649679146
0.428796835245308 90.2352141547757
0.450236677007573 90.1443617632914
0.471676518769839 90.0530954301751
0.493116360532104 89.9614883596311
0.51455620229437 89.8697268406044
0.535996044056635 89.7780125620431
0.5574358858189 89.6865251553896
0.578875727581166 89.5954265563303
0.600315569343431 89.5049603891544
0.621755411105697 89.4154634370485
0.643195252867962 89.327456628512
0.664635094630227 89.2417318933069
0.686074936392493 89.1591980398104
0.707514778154758 89.0807059539333
0.728954619917024 89.0069545237261
0.750394461679289 88.9385812907823
0.771834303441554 88.876256589774
0.79327414520382 88.8206493946149
0.814713986966085 88.7724327509721
0.836153828728351 88.7323449249181
0.857593670490616 88.7006805927306
0.879033512252881 88.6767618320362
0.900473354015147 88.6593613298504
0.921913195777412 88.6471522932361
0.943353037539678 88.6386733880052
0.964792879301943 88.632117962969
0.986232721064208 88.6255962120873
1.00767256282647 88.6168642146302
1.02911240458874 88.6033042251519
1.050552246351 88.5822432111884
};
\addplot [semithick, white!20!black]
table {%
0 113.561953693669
0.155788794612518 113.377567055239
0.311577589225037 113.178228789346
0.467366383837555 112.966520708641
0.623155178450074 112.745237258715
0.778943973062592 112.517232194351
0.93473276767511 112.285317213606
1.09052156228763 112.052150235375
1.24631035690015 111.819865709665
1.40209915151267 111.589222947157
1.55788794612518 111.358839423289
1.7136767407377 111.12643680418
1.86946553535022 110.891472603735
2.02525432996274 110.654047533456
2.18104312457526 110.414174939379
2.33683191918778 110.171744961009
2.49262071380029 109.92667415124
2.64840950841281 109.678984296474
2.80419830302533 109.428657699103
2.95998709763785 109.175654752961
3.11577589225037 108.920400398371
3.27156468686289 108.663604575933
3.4273534814754 108.405595688108
3.58314227608792 108.146356335484
3.73893107070044 107.885687480867
3.89471986531296 107.623325577391
4.05050865992548 107.35905043064
4.206297454538 107.092749000451
4.36208624915052 106.824244386292
4.51787504376303 106.553373592354
4.67366383837555 106.279925362446
4.82945263298807 106.003306260472
4.98524142760059 105.722748203266
5.14103022221311 105.437403687724
5.29681901682563 105.146545373123
5.45260781143814 104.849464758048
5.60839660605066 104.545466544602
5.76418540066318 104.233967114907
5.9199741952757 103.914494690674
6.07576298988822 103.58629442708
6.23155178450074 103.247939765783
6.38734057911325 102.897771961314
6.54312937372577 102.535032127372
6.69891816833829 102.159628459275
6.85470696295081 101.772183510194
7.01049575756333 101.37400608979
7.16628455217585 100.966586748055
7.32207334678836 100.551876102751
7.47786214140088 100.132245065432
7.6336509360134 99.710127300327
};
\addplot [semithick, white!20!black]
table {%
0 47.5900990460404
0.0791957386322502 47.440374708229
0.1583914772645 47.2847891351973
0.237587215896751 47.1239057672188
0.316782954529001 46.9583065107131
0.395978693161251 46.7886265214558
0.475174431793501 46.6154750100651
0.554370170425752 46.43932775294
0.633565909058002 46.2603903783643
0.712761647690252 46.0785185908031
0.791957386322502 45.893154538948
0.871153124954753 45.7036616391829
0.950348863587003 45.5100427442944
1.02954460221925 45.3124472687119
1.1087403408515 45.1110243020068
1.18793607948375 44.9059366596631
1.267131818116 44.6973428016668
1.34632755674825 44.4852942472016
1.4255232953805 44.269857329427
1.50471903401275 44.0512055408595
1.583914772645 43.8296719203507
1.66311051127726 43.6055926062036
1.74230624990951 43.3790323060924
1.82150198854176 43.1499506576275
1.90069772717401 42.9182726152326
1.97989346580626 42.6838983275456
2.05908920443851 42.4466868459625
2.13828494307076 42.2064327197179
2.21748068170301 41.9629185822102
2.29667642033526 41.7158765048609
2.37587215896751 41.4649894925827
2.45506789759976 41.2100824603015
2.53426363623201 40.9511925562201
2.61345937486426 40.6883490390976
2.69265511349651 40.4215962696765
2.77185085212876 40.1509665492048
2.85104659076101 39.876463308767
2.93024232939326 39.5981550360687
3.00943806802551 39.3162037964156
3.08863380665776 39.0308873917505
3.16782954529001 38.7425352896303
3.24702528392226 38.4513763538744
3.32622102255451 38.1579392895746
3.40541676118676 37.8632234556164
3.48461249981901 37.5683227822007
3.56380823845126 37.2745015228803
3.64300397708351 36.9830718853066
3.72219971571576 36.6953816609785
3.80139545434801 36.4129083759267
3.88059119298026 36.137192381525
};
\addplot [semithick, white!20!black]
table {%
0 66.8708130351603
0.122552000447036 66.6834788978641
0.245104000894071 66.4871715248154
0.367656001341107 66.2831678331773
0.490208001788143 66.0728624799346
0.612760002235178 65.8577301876227
0.735312002682214 65.6392158106104
0.85786400312925 65.4186320150284
0.980416003576285 65.196925120691
1.10296800402332 64.9742655842556
1.22552000447036 64.7496072175596
1.34807200491739 64.5214191370963
1.47062400536443 64.2893698620628
1.59317600581146 64.0535302646846
1.7157280062585 63.8139480449099
1.83828000670553 63.5706468747215
1.96083200715257 63.3236675251454
2.08338400759961 63.0730283309333
2.20593600804664 62.818752600877
2.32848800849368 62.5609427232711
2.45104000894071 62.300008956365
2.57359200938775 62.0365203901793
2.69614400983478 61.7706868099775
2.81869601028182 61.5024556659204
2.94124801072886 61.23164942854
3.06380001117589 60.9580413121821
3.18635201162293 60.6813953195796
3.30890401206996 60.4014742038112
3.431456012517 60.1179889233665
3.55400801296403 59.8305993558484
3.67656001341107 59.5388717743346
3.79911201385811 59.2422689026386
3.92166401430514 58.9403211880817
4.04421601475218 58.632530705399
4.16676801519921 58.3184910122739
4.28932001564625 57.9978060982619
4.41187201609328 57.670056927897
4.53442401654032 57.3349219294765
4.65697601698736 56.9921980201312
4.77952801743439 56.641639696557
4.90208001788143 56.2828439142087
5.02463201832846 55.9154194336869
5.1471840187755 55.5396908562903
5.26973601922253 55.1566458361272
5.39228801966957 54.7676715891024
5.5148400201166 54.3746420644354
5.63739202056364 53.9795607128902
5.75994402101068 53.5847202478775
5.88249602145771 53.1927623303614
6.00504802190475 52.8064141446288
};
\addplot [semithick, white!20!black]
table {%
0 24.8443137459358
0.022377289918545 24.7463883360268
0.04475457983709 24.6463960690228
0.067131869755635 24.5440252665863
0.08950915967418 24.4388541030454
0.111886449592725 24.3304755947618
0.13426373951127 24.2184612285391
0.156641029429815 24.1022456424776
0.17901831934836 23.9811128058637
0.201395609266905 23.8545334050672
0.22377289918545 23.7225925594875
0.246150189103995 23.5858207768687
0.26852747902254 23.4446626338794
0.290904768941085 23.29937759047
0.31328205885963 23.1502510347166
0.335659348778175 22.9976260069337
0.35803663869672 22.8418125120951
0.380413928615265 22.6829078406224
0.40279121853381 22.5210340751056
0.425168508452355 22.3564509769753
0.4475457983709 22.1893868713739
0.469923088289445 22.0198696315703
0.49230037820799 21.8477744256776
0.514677668126535 21.6730803789009
0.53705495804508 21.4958512596211
0.559432247963625 21.3161583485165
0.58180953788217 21.133992264988
0.604186827800715 20.9492009683103
0.62656411771926 20.7616742253945
0.648941407637805 20.5712570878956
0.67131869755635 20.3778101892641
0.693695987474895 20.1816505992776
0.71607327739344 19.9834866028586
0.738450567311985 19.7840426456613
0.76082785723053 19.5839558172641
0.783205147149075 19.3838209746729
0.80558243706762 19.1841989534098
0.827959726986165 18.9856723362477
0.85033701690471 18.7888826028546
0.872714306823255 18.5947847992067
0.8950915967418 18.4046318454347
0.917468886660345 18.2194026558396
0.93984617657889 18.0398201592502
0.962223466497435 17.8668144623697
0.98460075641598 17.7010169527526
1.00697804633453 17.5428112848012
1.02935533625307 17.3925213869958
1.05173262617162 17.2501724552105
1.07410991609016 17.1156244993251
1.09648720600871 16.9887674344679
};
\addplot [semithick, white!20!black]
table {%
0 89.4566276708781
0.0892030793479995 89.3354843731375
0.178406158695999 89.2035489330851
0.267609238043999 89.0624381898017
0.356812317391998 88.9138323946203
0.446015396739998 88.7594227963245
0.535218476087997 88.6008630277948
0.624421555435997 88.4396471392594
0.713624634783996 88.2768790660877
0.802827714131996 88.1128946950902
0.892030793479995 87.9470680400052
0.981233872827995 87.7784835388114
1.07043695217599 87.6071199832192
1.15964003152399 87.4332142448373
1.24884311087199 87.2569428648777
1.33804619021999 87.0784050415944
1.42724926956799 86.8976922065438
1.51645234891599 86.7148820947436
1.60565542826399 86.5300217001744
1.69485850761199 86.3431655229412
1.78406158695999 86.1546122311504
1.87326466630799 85.964707493032
1.96246774565599 85.7735581806035
2.05167082500399 85.5811732882552
2.14087390435199 85.3875210156459
2.23007698369999 85.1925432053809
2.31928006304799 84.9961809349046
2.40848314239599 84.7983956311825
2.49768622174399 84.5991492985808
2.58688930109199 84.3984299239035
2.67609238043999 84.1962593482949
2.76529545978799 83.9926385969461
2.85449853913599 83.7875932797547
2.94370161848398 83.5810949461231
3.03290469783198 83.3731128491661
3.12210777717998 83.1635990152749
3.21131085652798 82.952515175018
3.30051393587598 82.7398831352129
3.38971701522398 82.5257938420189
3.47892009457198 82.3102774686735
3.56812317391998 82.0929527295506
3.65732625326798 81.8729765522625
3.74652933261598 81.6497435674652
3.83573241196398 81.4229952525095
3.92493549131198 81.192733017706
4.01413857065998 80.9591511684019
4.10334165000798 80.7224987384558
4.19254472935598 80.4830911515816
4.28174780870398 80.2413107778841
4.37095088805198 79.9975609502335
};
\addplot [semithick, white!20!black]
table {%
0 72.4146647503486
0.159397051188826 72.1845724109935
0.318794102377653 71.9440827901009
0.478191153566479 71.6948216278868
0.637588204755305 71.4386103854466
0.796985255944131 71.1773874142502
0.956382307132958 70.913060982963
1.11577935832178 70.647415890675
1.27517640951061 70.3818171655162
1.43457346069944 70.1165846651473
1.59397051188826 69.8502442852592
1.75336756307709 69.5805261716841
1.91276461426592 69.3067993318446
2.07216166545474 69.0290372977682
2.23155871664357 68.7471868715099
2.39095576783239 68.4611626052782
2.55035281902122 68.1709168060828
2.70974987021005 67.8764297925209
2.86914692139887 67.5776910364678
3.0285439725877 67.2747722229011
3.18794102377653 66.968164355931
3.34733807496535 66.6586473110202
3.50673512615418 66.3465516489269
3.666132177343 66.0317993465232
3.82552922853183 65.7141063971601
3.98492627972066 65.3931175738099
4.14432333090948 65.068485981207
4.30372038209831 64.7398968723153
4.46311743328714 64.4069435287785
4.62251448447596 64.0691454970289
4.78191153566479 63.7258638647939
4.94130858685362 63.376178307763
5.10070563804244 63.019158124015
5.26010268923127 62.6538388401552
5.41949974042009 62.2794225252547
5.57889679160892 61.8951437952805
5.73829384279775 61.500208422553
5.89769089398657 61.0939507938245
6.0570879451754 60.6758537017472
6.21648499636423 60.2452636794316
6.37588204755305 59.8013121286086
6.53527909874188 59.34334995361
6.6946761499307 58.8718432504034
6.85407320111953 58.3881334127865
7.01347025230836 57.894193067669
7.17286730349718 57.3927650880183
7.33226435468601 56.8867946399739
7.49166140587484 56.3797397735537
7.65105845706366 55.8756266315779
7.81045550825249 55.37859981704
};
\addplot [semithick, white!20!black]
table {%
0 59.7713929112965
0.0452544084566229 59.6782740286355
0.0905088169132458 59.5786592858902
0.135763225369869 59.4732014239412
0.181017633826492 59.3625099943423
0.226272042283115 59.2471914928014
0.271526450739738 59.1278216934861
0.31678085919636 59.0048226899719
0.362035267652983 58.878350470611
0.407289676109606 58.7483114141428
0.452544084566229 58.6145624606305
0.497798493022852 58.4771142334622
0.543052901479475 58.3362700319349
0.588307309936098 58.1923177256255
0.633561718392721 58.0455228475504
0.678816126849344 57.8961341784287
0.724070535305967 57.7443732513209
0.76932494376259 57.590342257724
0.814579352219213 57.4341343538705
0.859833760675836 57.2759039184249
0.905088169132458 57.1158854914153
0.950342577589081 56.9542005966873
0.995596986045704 56.7908066943198
1.04085139450233 56.6257109928694
1.08610580295895 56.4589674439593
1.13136021141557 56.290627317494
1.1766146198722 56.1206998602776
1.22186902832882 55.949131511181
1.26712343678544 55.7759038292473
1.31237784524206 55.601004103344
1.35763225369869 55.424473362609
1.40288666215531 55.2466093819675
1.44814107061193 55.0679228241283
1.49339547906856 54.888902286847
1.53864988752518 54.7099661408314
1.5839042959818 54.5314955389184
1.62915870443843 54.3538653238177
1.67441311289505 54.1774809417566
1.71966752135167 54.0027999087187
1.76492192980829 53.8304172318019
1.81017633826492 53.6608409593924
1.85543074672154 53.494108333444
1.90068515517816 53.3301154815718
1.94593956363479 53.1689559424499
1.99119397209141 53.0106482912115
2.03644838054803 52.85509694136
2.08170278900465 52.7021816386619
2.12695719746128 52.5515988941886
2.1722116059179 52.40292068862
2.21746601437452 52.2557297234376
};
\addplot [semithick, white!20!black]
table {%
0 88.2075711316939
0.072814032770686 88.106658997634
0.145628065541372 87.9954511426305
0.218442098312058 87.8754381499021
0.291256131082744 87.7481400447263
0.36407016385343 87.6150698942969
0.436884196624116 87.4777028952547
0.509698229394802 87.3373499236492
0.582512262165488 87.1949523249109
0.655326294936174 87.0507927851437
0.72814032770686 86.9044371093973
0.800954360477546 86.7552960773056
0.873768393248232 86.6034834336833
0.946582426018918 86.4492827820513
1.0193964587896 86.2929171979603
1.09221049156029 86.1345333927216
1.16502452433098 85.9742608510594
1.23783855710166 85.8121953196995
1.31065258987235 85.6483985376259
1.38346662264303 85.482935325819
1.45628065541372 85.3160667382836
1.52909468818441 85.1480432449694
1.60190872095509 84.9789182779878
1.67472275372578 84.8087138592445
1.74753678649646 84.6374477263367
1.82035081926715 84.4651211627953
1.89316485203784 84.2917279558962
1.96597888480852 84.1172696715591
2.03879291757921 83.9417670298067
2.11160695034989 83.7652793033513
2.18442098312058 83.5879314982488
2.25723501589127 83.4099034699813
2.33004904866195 83.2314293807461
2.40286308143264 83.0526903858491
2.47567711420332 82.8738308178456
2.54849114697401 82.6949676086871
2.6213051797447 82.5162311933753
2.69411921251538 82.3377978217753
2.76693324528607 82.1598985938602
2.83974727805675 81.9827410238331
2.91256131082744 81.8061332918073
2.98537534359813 81.6293168144717
3.05818937636881 81.4515879324398
3.1310034091395 81.2724899533533
3.20381744191018 81.0917261566805
3.27663147468087 80.9090649664633
3.34944550745156 80.7242967649662
3.42225954022224 80.5371779529165
3.49507357299293 80.3474307117302
3.56788760576361 80.1547820030415
};
\addplot [semithick, white!20!black]
table {%
0 43.4549485171263
0.0999231828880962 43.2739462648085
0.199846365776192 43.0870983682326
0.299769548664289 42.8949922050543
0.399692731552385 42.6982732165069
0.499615914440481 42.497670200605
0.599539097328577 42.2938879765189
0.699462280216673 42.087507959098
0.79938546310477 41.878830313904
0.899308645992866 41.6677157836916
0.999231828880962 41.4533571469252
1.09915501176906 41.2347161738691
1.19907819465715 41.0116169286681
1.29900137754525 40.784133596371
1.39892456043335 40.5523486466512
1.49884774332144 40.3163696159154
1.59877092620954 40.0763129265008
1.69869410909764 39.8322021439198
1.79861729198573 39.5840863901504
1.89854047487383 39.3321417726012
1.99846365776192 39.0767568731814
2.09838684065002 38.8183949856867
2.19831002353812 38.5571870784892
2.29823320642621 38.2930683448517
2.39815638931431 38.0258908697607
2.49807957220241 37.7554687829333
2.5980027550905 37.481578593644
2.6979259379786 37.2039374774729
2.79784912086669 36.922223885885
2.89777230375479 36.6360381011304
2.99769548664289 36.3448767859002
3.09761866953098 36.0482990187123
3.19754185241908 35.7460620878749
3.29746503530717 35.437920389665
3.39738821819527 35.1236924385353
3.49731140108337 34.8031991386304
3.59723458397146 34.4762220337757
3.69715776685956 34.1426276641591
3.79708094974766 33.80239883021
3.89700413263575 33.4556084217759
3.99692731552385 33.1024290208574
4.09685049841194 32.7431245000693
4.19677368130004 32.3785138930446
4.29669686418814 32.0100388991829
4.39662004707623 31.6393477067091
4.49654322996433 31.2684281021681
4.59646641285243 30.8993591114113
4.69638959574052 30.5343865232044
4.79631277862862 30.1760311247264
4.89623596151671 29.8269041889553
};
\addplot [semithick, white!20!black]
table {%
0 72.9310086488597
0.265969064037651 72.5617859379319
0.531938128075302 72.1797942190985
0.797907192112953 71.7872995738792
1.0638762561506 71.3869802218985
1.32984532018825 70.9817578784651
1.59581438422591 70.5745280202943
1.86178344826356 70.1680988374264
2.12775251230121 69.7647447699008
2.39372157633886 69.365048778724
2.65969064037651 68.9662805481065
2.92565970441416 68.5640628922568
3.19162876845181 68.1568764719088
3.45759783248946 67.744369475356
3.72356689652711 67.3261757713614
3.98953596056476 66.9019073553078
4.25550502460241 66.4712771916518
4.52147408864006 66.0341415893576
4.78744315267772 65.5903960514136
5.05341221671537 65.140066002016
5.31938128075302 64.6838976459431
5.58535034479067 64.2232988912287
5.85131940882832 63.7589460348817
6.11728847286597 63.290665735702
6.38325753690362 62.8178382606322
6.64922660094127 62.3397074288455
6.91519566497892 61.8555624586519
7.18116472901657 61.3647921799038
7.44713379305422 60.8665682429289
7.71310285709187 60.3598912390795
7.97907192112953 59.8433770579256
8.24504098516718 59.3148897139946
8.51101004920483 58.7721209369047
8.77697911324248 58.2127302580039
9.04294817728013 57.6347753845443
9.30891724131778 57.0364148333765
9.57488630535543 56.4157460080301
9.84085536939308 55.7710901959883
10.1068244334307 55.1010161448573
10.3727934974684 54.4037429876739
10.638762561506 53.6772802092816
10.9047316255437 52.9206181161788
11.1707006895813 52.1350829023402
11.436669753619 51.3235597349924
11.7026388176566 50.4901957213304
11.9686078816943 49.6407491869338
12.2345769457319 48.7813962929115
12.5005460097696 47.9194844032854
12.7665150738072 47.0636098673336
13.0324841378449 46.2226022983633
};
\addplot [semithick, white!20!black]
table {%
0 104.932930279256
0.0323535513947001 104.901691180649
0.0647071027894003 104.859151655927
0.0970606541841004 104.806961100038
0.129414205578801 104.746726150304
0.161767756973501 104.679976512207
0.194121308368201 104.608197692648
0.226474859762901 104.532685185334
0.258828411157601 104.454364013045
0.291181962552301 104.37360127914
0.323535513947001 104.290459876026
0.355889065341702 104.205117825299
0.388242616736402 104.118050154313
0.420596168131102 104.029711735615
0.452949719525802 103.940467449715
0.485303270920502 103.850564628733
0.517656822315202 103.760205596393
0.550010373709902 103.669548500204
0.582363925104603 103.578686499859
0.614717476499303 103.487655587708
0.647071027894003 103.396596325104
0.679424579288703 103.305500739921
0.711778130683403 103.214294928774
0.744131682078103 103.123060695128
0.776485233472803 103.031973491871
0.808838784867504 102.941218932389
0.841192336262204 102.8509759727
0.873545887656904 102.761437506025
0.905899439051604 102.672872928299
0.938252990446304 102.585661808677
0.970606541841004 102.500377493624
1.0029600932357 102.4177788223
1.0353136446304 102.338670630554
1.0676671960251 102.263785407485
1.1000207474198 102.193715037462
1.1323742988145 102.128993403394
1.16472785020921 102.070196977543
1.19708140160391 102.01790624237
1.22943495299861 101.972705271592
1.26178850439331 101.935172237641
1.29414205578801 101.90529628858
1.32649560718271 101.882035693254
1.35884915857741 101.86386867604
1.39120270997211 101.849184793051
1.42355626136681 101.836337069497
1.45590981276151 101.82340157589
1.48826336415621 101.808389060749
1.52061691555091 101.789013809042
1.55297046694561 101.762646104355
1.58532401834031 101.72659612316
};
\addplot [semithick, white!20!black]
table {%
0 106.250350168788
0.0955321854883562 106.137638152018
0.191064370976712 106.012105826706
0.286596556465069 105.875806219613
0.382128741953425 105.730878493087
0.477660927441781 105.579458623203
0.573193112930137 105.42364037543
0.668725298418493 105.26534823766
0.76425748390685 105.10606601474
0.859789669395206 104.946327799603
0.955321854883562 104.785452975116
1.05085404037192 104.622368321218
1.14638622586027 104.457023665932
1.24191841134863 104.289683863531
1.33745059683699 104.120529654135
1.43298278232534 103.94962815496
1.5285149678137 103.777038723825
1.62404715330206 103.602846907745
1.71957933879041 103.427089904939
1.81511152427877 103.249773510542
1.91064370976712 103.071188164685
2.00617589525548 102.891697013392
2.10170808074384 102.711431349188
2.19724026623219 102.530417878623
2.29277245172055 102.348634835559
2.38830463720891 102.166032056489
2.48383682269726 101.982575586238
2.57936900818562 101.798287299656
2.67490119367397 101.613191934124
2.77043337916233 101.427368776358
2.86596556465069 101.240959286776
2.96149775013904 101.054008648955
3.0570299356274 100.866507889971
3.15256212111575 100.678375502623
3.24809430660411 100.489525746475
3.34362649209247 100.299855067536
3.43915867758082 100.109284377029
3.53469086306918 99.917794737811
3.63022304855754 99.7254287099369
3.72575523404589 99.5320926856623
3.82128741953425 99.3370960709022
3.9168196050226 99.139158258245
4.01235179051096 98.9372379809194
4.10788397599932 98.7306060576044
4.20341616148767 98.5188732421346
4.29894834697603 98.3018705197999
4.39448053246439 98.0794910073139
4.49001271795274 97.8517208763394
4.5855449034411 97.6186028830449
4.68107708892945 97.3801864620633
};
\addplot [semithick, white!20!black]
table {%
0 92.3564185379319
0.163121828604012 92.1413050675858
0.326243657208023 91.9134682577714
0.489365485812035 91.6750294604847
0.652487314416047 91.4283250443986
0.815609143020058 91.1757868989633
0.97873097162407 90.9198112328108
1.14185280022808 90.6626579727395
1.30497462883209 90.4061112334642
1.46809645743611 90.1507170745996
1.63121828604012 89.8949812252399
1.79434011464413 89.6365215413292
1.95746194324814 89.3747042615962
2.12058377185215 89.1095477703813
2.28370560045616 88.8410150116597
2.44682742906018 88.5689931893217
2.60994925766419 88.2934050181082
2.7730710862682 88.014244520613
2.93619291487221 87.731492874535
3.09931474347622 87.4451655251272
3.26243657208023 87.1557346031275
3.42555840068424 86.8639783773908
3.58868022928826 86.5702441133919
3.75180205789227 86.2744782364012
3.91492388649628 85.9764206294527
4.07804571510029 85.6757398545485
4.2411675437043 85.3721316583849
4.40428937230832 85.0653637021445
4.56741120091233 84.7551188534495
4.73053302951634 84.441043665965
4.89365485812035 84.1226679557106
5.05677668672436 83.7991672026002
5.21989851532837 83.469618690725
5.38302034393238 83.1330439196587
5.5461421725364 82.7886203750713
5.70926400114041 82.4355549595253
5.87238582974442 82.0730444260687
6.03550765834843 81.7004108368676
6.19862948695244 81.3171120471537
6.36175131555646 80.9223876450116
6.52487314416047 80.5150413833509
6.68799497276448 80.0939169847402
6.85111680136849 79.6589318104971
7.0142386299725 79.2108128417091
7.17736045857651 78.7509909039531
7.34048228718053 78.2816695916442
7.50360411578454 77.8052559102818
7.66672594438855 77.3246779829343
7.82984777299256 76.8433954464049
7.99296960159657 76.3649653290731
};
\addplot [semithick, white!20!black]
table {%
0 92.3160219231584
0.0814387028508895 92.2079001354505
0.162877405701779 92.0888328311817
0.244316108552668 91.9604590072815
0.325754811403558 91.8244670115792
0.407193514254447 91.6825432147738
0.488632217105337 91.5363352580712
0.570070919956226 91.3873265170419
0.651509622807116 91.2366109869865
0.732948325658005 91.0845369400082
0.814387028508895 90.9305733637643
0.895825731359784 90.7739526092938
0.977264434210674 90.6147223014337
1.05870313706156 90.4531511901033
1.14014183991245 90.2894425529408
1.22158054276334 90.1237151921962
1.30301924561423 89.9560748903641
1.38445794846512 89.7866110401232
1.46589665131601 89.615376753157
1.5473353541669 89.4424219574751
1.62877405701779 89.2680226871393
1.71021275986868 89.092475429348
1.79165146271957 88.9158625391807
1.87309016557046 88.7382039948272
1.95452886842135 88.5594976353815
2.03596757127224 88.379720011922
2.11740627412313 88.1988467627366
2.19884497697402 88.0168743999226
2.28028367982491 87.833810815104
2.36172238267579 87.6497029087824
2.44316108552668 87.4646551351322
2.52459978837757 87.2787771956378
2.60603849122846 87.0922032259403
2.68747719407935 86.9050099777601
2.76891589693024 86.7172523185977
2.85035459978113 86.5289621138717
2.93179330263202 86.340186166429
3.01323200548291 86.1510234496664
3.0946707083338 85.96163252229
3.17610941118469 85.7721158741457
3.25754811403558 85.5821320173644
3.33898681688647 85.3907922368068
3.42042551973736 85.1973443401248
3.50186422258825 85.0013201476282
3.58330292543914 84.8024728148163
3.66474162829003 84.6006832155916
3.74618033114092 84.3958703160395
3.82761903399181 84.1879693614439
3.9090577368427 83.9769241574879
3.99049643969358 83.7626874615846
};
\addplot [semithick, white!20!black]
table {%
0 78.7060619332499
0.0821414169848659 78.5834654940527
0.164282833969732 78.4514405969118
0.246424250954598 78.3113071140449
0.328565667939463 78.1644276795111
0.410707084924329 78.0121814098144
0.492848501909195 77.8559126340985
0.574989918894061 77.6968112728296
0.657131335878927 77.5357126841971
0.739272752863793 77.3728186002565
0.821414169848659 77.2075735052031
0.903555586833525 77.0392222544141
0.98569700381839 76.8677872923218
1.06783842080326 76.6934968076124
1.14997983778812 76.5165335119806
1.23212125477299 76.3370256737326
1.31426267175785 76.1550920010958
1.39640408874272 75.9708087439869
1.47854550572759 75.7842318176075
1.56068692271245 75.5954481673595
1.64282833969732 75.4047545001898
1.72496975668218 75.2124675402154
1.80711117366705 75.0186687648471
1.88925259065191 74.8233585425044
1.97139400763678 74.6265081578722
2.05353542462165 74.4280656886026
2.13567684160651 74.2279665474462
2.21781825859138 74.0261419056178
2.29995967557624 73.8225255937819
2.38210109256111 73.6170619165646
2.46424250954598 73.4097173306089
2.54638392653084 73.2004990297737
2.62852534351571 72.9894937119437
2.71066676050057 72.7767457799817
2.79280817748544 72.5622920244119
2.87494959447031 72.3461504526206
2.95709101145517 72.1283402527838
3.03923242844004 71.9089379453834
3.1213738454249 71.6880912728982
3.20351526240977 71.4659414407165
3.28565667939463 71.2423366449263
3.3677980963795 71.0167237257008
3.44993951336437 70.7887516384954
3.53208093034923 70.5584190624473
3.6142223473341 70.3259156190295
3.69636376431896 70.0915825528256
3.77850518130383 69.8558050053697
3.8606465982887 69.6189988336599
3.94278801527356 69.3816341987022
4.02492943225843 69.1442081074953
};
\addplot [semithick, white!20!black]
table {%
0 81.5724280171385
0.0615750439632592 81.4796348809838
0.123150087926518 81.3775370487079
0.184725131889778 81.2674012680245
0.246300175853037 81.1504970305782
0.307875219816296 81.0280820380455
0.369450263779555 80.9013776095133
0.431025307742814 80.7714415771769
0.492600351706074 80.638991457031
0.554175395669333 80.5042105125662
0.615750439632592 80.3667891869062
0.677325483595851 80.2263732585583
0.73890052755911 80.0831608159418
0.800475571522369 79.9374512027791
0.862050615485629 79.7894915521787
0.923625659448888 79.6394660230878
0.985200703412147 79.4875363522599
1.04677574737541 79.333805406437
1.10835079133867 79.1783464892159
1.16992583530192 79.0212473418296
1.23150087926518 78.8627512848703
1.29307592322844 78.703050337234
1.3546509671917 78.5421599437205
1.41622601115496 78.3801030028666
1.47780105511822 78.2169208914007
1.53937609908148 78.0526446910636
1.60095114304474 77.8872882791382
1.662526187008 77.7208538411567
1.72410123097126 77.5533721049144
1.78567627493452 77.3849091310296
1.84725131889778 77.2156041808583
1.90882636286103 77.0457196351528
1.97040140682429 76.8756163927802
2.03197645078755 76.7056094687548
2.09355149475081 76.5359587881862
2.15512653871407 76.3668914985072
2.21670158267733 76.1986451062416
2.27827662664059 76.0314950772609
2.33985167060385 75.8657665908129
2.40142671456711 75.701808364279
2.46300175853037 75.5396421632368
2.52457680249363 75.3787107892603
2.58615184645689 75.2184110508298
2.64772689042015 75.0583442091443
2.7093019343834 74.898188281231
2.77087697834666 74.7376066154751
2.83245202230992 74.576263906574
2.89402706627318 74.4137263220832
2.95560211023644 74.2494739460638
3.0171771541997 74.0829878198265
};
\addplot [semithick, white!20!black]
table {%
0 85.2312506794003
0.0845853693752098 85.1119501479374
0.16917073875042 84.9824334463603
0.253756108125629 84.8441898376139
0.338341477500839 84.6987601447596
0.422926846876049 84.5476961440555
0.507512216251259 84.3925130385609
0.592097585626468 84.2345678504835
0.676682955001678 84.0748434076032
0.761268324376888 83.9136184180564
0.845853693752098 83.7503163187128
0.930439063127307 83.5841210077919
1.01502443250252 83.4150438028555
1.09960980187773 83.2433238118975
1.18419517125294 83.0691454228785
1.26878054062815 82.8926244821801
1.35336591000336 82.7138672698989
1.43795127937857 82.5329530890717
1.52253664875378 82.3499340621627
1.60712201812899 82.1648782166964
1.6917073875042 81.9780790261034
1.7762927568794 81.7898599157219
1.86087812625461 81.6003118021838
1.94546349562982 81.4094419496675
2.03004886500503 81.2172255897111
2.11463423438024 81.0236139633768
2.19921960375545 80.8285522366365
2.28380497313066 80.6319951103555
2.36839034250587 80.4339008506566
2.45297571188108 80.2342493113183
2.53756108125629 80.033053534092
2.6221464506315 79.8303380446398
2.70673182000671 79.626176300624
2.79131718938192 79.4205922864048
2.87590255875713 79.21360158826
2.96048792813234 79.0052007715496
3.04507329750755 78.7953933224946
3.12965866688276 78.5842400752843
3.21424403625797 78.3718701025526
3.29882940563318 78.158376675406
3.38341477500839 77.9434879674835
3.4680001443836 77.7264808646996
3.55258551375881 77.5068348299256
3.63717088313402 77.2843654453057
3.72175625250923 77.0591102302452
3.80634162188444 76.8312688286246
3.89092699125965 76.6010876268715
3.97551236063486 76.368853993525
4.06009773001007 76.1349054775821
4.14468309938528 75.8996009073926
};
\addplot [semithick, white!20!black]
table {%
0 79.9287366870355
0.192562079549556 79.6626702916919
0.385124159099112 79.3846406367962
0.577686238648668 79.0966473945944
0.770248318198223 78.800957636838
0.962810397747779 78.499985204827
1.15537247729734 78.1961129376804
1.34793455684689 77.8916062695432
1.54049663639645 77.5882561802928
1.733058715946 77.2865444969782
1.92562079549556 76.9846149459487
2.11818287504511 76.6795276150144
2.31074495459467 76.3703854970322
2.50330703414423 76.0570816283689
2.69586911369378 75.7394754210861
2.88843119324334 75.4173810814289
3.08099327279289 75.0906685989908
3.27355535234245 74.7592863582842
3.46611743189201 74.4231927391915
3.65867951144156 74.0824248703556
3.85124159099112 73.7375429161941
4.04380367054067 73.3895135005361
4.23636575009023 73.0387764821604
4.42892782963979 72.6852344280248
4.62148990918934 72.3285120018514
4.8140519887389 71.9681431525862
5.00661406828845 71.6036881762762
5.19917614783801 71.2347743182396
5.39173822738756 70.8609021521718
5.58430030693712 70.4814832963326
5.77686238648668 70.0957188205434
5.96942446603623 69.7023611294504
6.16198654558579 69.300071541643
6.35454862513534 68.8874699490079
6.5471107046849 68.4634079857468
6.73967278423446 68.0267890404227
6.93223486378401 67.576484931527
7.12479694333357 67.1115232908459
7.31735902288312 66.6311050487062
7.50992110243268 66.134200965216
7.70248318198223 65.6194868479864
7.89504526153179 65.0860188462287
8.08760734108135 64.5343170966045
8.2801694206309 63.9659567328506
8.47273150018046 63.3833590600862
8.66529357973001 62.78996742158
8.85785565927957 62.1894921623836
9.05041773882913 61.5863559260376
9.24297981837868 60.9857390577495
9.43554189792824 60.3929668328069
};
\addplot [semithick, white!20!black]
table {%
0 118.187377013266
0.0215712722274773 118.183466082391
0.0431425444549547 118.166996512424
0.064713816682432 118.139868975381
0.0862850889099094 118.103927325422
0.107856361137387 118.060908448021
0.129427633364864 118.012500812593
0.150998905592341 117.960189812905
0.172570177819819 117.905067135226
0.194141450047296 117.847618198397
0.215712722274773 117.788048845177
0.237283994502251 117.726724498656
0.258855266729728 117.66422895099
0.280426538957205 117.601087541501
0.301997811184683 117.537714897004
0.32356908341216 117.474377757536
0.345140355639637 117.411288484808
0.366711627867115 117.34862982025
0.388282900094592 117.286501084714
0.409854172322069 117.224906469285
0.431425444549547 117.163942153392
0.452996716777024 117.103520891366
0.474567989004501 117.043536886679
0.496139261231979 116.984100140131
0.517710533459456 116.925443863633
0.539281805686933 116.867819484268
0.560853077914411 116.811479747777
0.582424350141888 116.756709495973
0.603995622369365 116.703890457783
0.625566894596843 116.653551646774
0.64713816682432 116.6064718896
0.668709439051797 116.563625381418
0.690280711279275 116.525993822016
0.711851983506752 116.494471666428
0.73342325573423 116.469776872323
0.754994527961707 116.452558805592
0.776565800189184 116.443525925069
0.798137072416662 116.443376629584
0.819708344644139 116.452792151436
0.841279616871616 116.472419616481
0.862850889099094 116.502169039352
0.884422161326571 116.540704851716
0.905993433554048 116.586031691374
0.927564705781526 116.635936533682
0.949135978009003 116.688139785417
0.97070725023648 116.739981868513
0.992278522463958 116.78871179093
1.01384979469143 116.831205624521
1.03542106691891 116.8638862268
1.05699233914639 116.883087981833
};
\addplot [semithick, white!20!black]
table {%
0 130.985794219375
0.0333375149973956 130.979211284305
0.0666750299947911 130.958373330262
0.100012544992187 130.925554363264
0.133350059989582 130.883002967972
0.166687574986978 130.832858242534
0.200025089984373 130.777207627556
0.233362604981769 130.717930480518
0.266700119979164 130.656466576108
0.30003763497656 130.593469159168
0.333375149973956 130.529020559874
0.366712664971351 130.463228829494
0.400050179968747 130.396597766109
0.433387694966142 130.329652744362
0.466725209963538 130.262791158615
0.500062724960933 130.196235617704
0.533400239958329 130.130158483457
0.566737754955724 130.064740341461
0.60007526995312 130.000066933922
0.633412784950516 129.936102412558
0.666750299947911 129.872953358815
0.700087814945307 129.81058680106
0.733425329942702 129.748938069439
0.766762844940098 129.688124406506
0.800100359937493 129.628364739435
0.833437874934889 129.569890402599
0.866775389932284 129.512949369642
0.90011290492968 129.457853081681
0.933450419927075 129.405003431981
0.966787934924471 129.354964959852
1.00012544992187 129.308558776869
1.03346296491926 129.266713300766
1.06680047991666 129.230294036522
1.10013799491405 129.200065305328
1.13347550991145 129.176628576616
1.16681302490884 129.160520737251
1.20015053990624 129.152346880615
1.23348805490364 129.152708315273
1.26682556990103 129.162189839457
1.30016308489843 129.181270913274
1.33350059989582 129.209553226148
1.36683811489322 129.245345109468
1.40017562989061 129.286376015779
1.43351314488801 129.330175934518
1.4668506598854 129.37431025921
1.5001881748828 129.416040063791
1.5335256898802 129.452554702743
1.56686320487759 129.480733828327
1.60020071987499 129.497040441844
1.63353823487238 129.49784570664
};
\addplot [semithick, white!20!black]
table {%
0 46.5134504616868
0.170479275475211 46.2430422861676
0.340958550950422 45.9649124008827
0.511437826425634 45.6801366484813
0.681917101900845 45.3899938140711
0.852396377376056 45.0959263415314
1.02287565285127 44.7993544593351
1.19335492832648 44.5015970271306
1.36383420380169 44.2036090971683
1.5343134792769 43.9054551894426
1.70479275475211 43.6054996434123
1.87527203022732 43.3013038003685
2.04575130570253 42.9921074497934
2.21623058117774 42.6777770488548
2.38670985665296 42.3581915817986
2.55718913212817 42.0332559898801
2.72766840760338 41.7029253441227
2.89814768307859 41.3671435689207
3.0686269585538 41.0258968909567
3.23910623402901 40.6793233426605
3.40958550950422 40.3279766721267
3.58006478497943 39.9727327506082
3.75054406045465 39.6139520715397
3.92102333592986 39.2515105106354
4.09150261140507 38.8850427423065
4.26198188688028 38.5141025989705
4.43246116235549 38.1382332270924
4.6029404378307 37.7569682043269
4.77341971330591 37.3697210654212
4.94389898878112 36.9757683089338
5.11437826425633 36.5741397964243
5.28485753973155 36.1636086334914
5.45533681520676 35.7430277419798
5.62581609068197 35.3112451842023
5.79629536615718 34.8673239504901
5.96677464163239 34.4103738621461
6.1372539171076 33.9394467187736
6.30773319258281 33.4537414848976
6.47821246805802 32.9526367723633
6.64869174353324 32.4354498653516
6.81917101900845 31.9015712506021
6.98965029448366 31.3509586293815
7.16012956995887 30.7849215434433
7.33060884543408 30.2058326146565
7.50108812090929 29.6166954087807
7.6715673963845 29.0214061925603
7.84204667185971 28.4240942972807
8.01252594733493 27.8294896431968
8.18300522281014 27.2430397018557
8.35348449828535 26.6703550167152
};
\addplot [semithick, white!20!black]
table {%
0 97.0271733768824
0.0848998651789654 96.9192080550937
0.169799730357931 96.7996916600378
0.254699595536896 96.6703954869208
0.339599460715862 96.5331499747103
0.424499325894827 96.3897816193727
0.509399191073792 96.2420769615808
0.594299056252758 96.0916561347275
0.679198921431723 95.9397339245433
0.764098786610689 95.7867182785596
0.848998651789654 95.6320429795159
0.933898516968619 95.4748628346275
1.01879838214758 95.3152032809155
1.10369824732655 95.1533357492654
1.18859811250552 94.9894597344548
1.27349797768448 94.8236802603429
1.35839784286345 94.6560903347705
1.44329770804241 94.4867795730592
1.52819757322138 94.315796854625
1.61309743840034 94.1431777349298
1.69799730357931 93.9691999877516
1.78289716875827 93.7941749466231
1.86779703393724 93.6181972839552
1.9526968991162 93.4412904276997
2.03759676429517 93.263449707047
2.12249662947413 93.0846475582021
2.2073964946531 92.9048608452613
2.29229635983207 92.7240983112308
2.37719622501103 92.5423787345314
2.46209609019 92.3597663602392
2.54699595536896 92.1763873327052
2.63189582054793 91.9923444563252
2.71679568572689 91.807740353123
2.80169555090586 91.6226151126722
2.88659541608482 91.436990068357
2.97149528126379 91.2508644688315
3.05639514644275 91.0642561225629
3.14129501162172 90.8772365393185
3.22619487680068 90.6899362436556
3.31109474197965 90.5024051966203
3.39599460715862 90.3141975127034
3.48089447233758 90.1242962339852
3.56579433751655 89.931840727064
3.65069420269551 89.7362554700527
3.73559406787448 89.5372186745048
3.82049393305344 89.3345572517924
3.90539379823241 89.1281416940882
3.99029366341137 88.9178766718575
4.07519352859034 88.7036833036006
4.1600933937693 88.4854894785049
};
\addplot [semithick, white!20!black]
table {%
0 49.2903196781983
0.182913553566602 49.006383909279
0.365827107133205 48.7141438511564
0.548740660699807 48.4148145839753
0.73165421426641 48.1098409804482
0.914567767833012 47.800842826597
1.09748132139961 47.4894173599432
1.28039487496622 47.1770627514182
1.46330842853282 46.864892967956
1.64622198209942 46.5530328162104
1.82913553566602 46.2397033041465
2.01204908923263 45.9222146321643
2.19496264279923 45.5997067979726
2.37787619636583 45.2720159654234
2.56078974993243 44.9389882995895
2.74370330349904 44.6004911538224
2.92661685706564 44.2564487795405
3.10953041063224 43.9067930918658
3.29244396419884 43.5514986705658
3.47535751776545 43.1906907026948
3.65827107133205 42.8249489222659
3.84118462489865 42.4552192641554
4.02409817846525 42.0819031906117
4.20701173203186 41.7048692439426
4.38992528559846 41.3237177850424
4.57283883916506 40.9379610195359
4.75575239273166 40.5471071859022
4.93866594629826 40.1506679257664
5.12157949986487 39.7480218045808
5.30449305343147 39.3384045626136
5.48740660699807 38.9207856731535
5.67032016056467 38.4938152610447
5.85323371413128 38.0561931743104
6.03614726769788 37.60661157599
6.21906082126448 37.1440020397015
6.40197437483108 36.667350182486
6.58488792839769 36.1755825373789
6.76780148196429 35.6677830131989
6.95071503553089 35.1432245313965
7.13362858909749 34.6010836274959
7.3165421426641 34.0405805599342
7.4994556962307 33.4615635207957
7.6823692497973 32.8653634840779
7.8652828033639 32.2544420582395
8.04819635693051 31.631972278255
8.23110991049711 31.0021143366941
8.41402346406371 30.3692857642912
8.59693701763031 29.7385794754491
8.77985057119692 29.1158771464559
8.96276412476352 28.5072335003974
};
\addplot [semithick, white!20!black]
table {%
0 98.4716522010719
0.451085849614086 97.8853006428329
0.902171699228173 97.2792857785616
1.35325754884226 96.6575713124019
1.80434339845635 96.0249235169399
2.25542924807043 95.3865401698605
2.70651509768452 94.7475923972886
3.1576009472986 94.1132094770068
3.60868679691269 93.4877245637409
4.05977264652678 92.8724460482911
4.51085849614086 92.2624909243
4.96194434575495 91.649773756959
5.41303019536904 91.0312671181749
5.86411604498312 90.4061236134855
6.31520189459721 89.7734670374105
6.76628774421129 89.1323630932178
7.21737359382538 88.48208274879
7.66845944343947 87.822289218558
8.11954529305355 87.1527085837353
8.57063114266764 86.4732183496741
9.02171699228173 85.7849736304597
9.47280284189581 85.0904440961646
9.9238886915099 84.3909123072449
10.374974541124 83.6860737149817
10.8260603907381 82.9747694887942
11.2771462403522 82.2555937449561
11.7282320899662 81.5272721021709
12.1793179395803 80.7887932458352
12.6304037891944 80.0387254848314
13.0814896388085 79.2753475105453
13.5325754884226 78.4962217115106
13.9836613380367 77.6972715399999
14.4347471876508 76.8738653051133
14.8858330372648 76.0213145542807
15.3369188868789 75.1357082897801
15.788004736493 74.2133480739976
16.2390905861071 73.2504428211344
16.6901764357212 72.2435826066344
17.1412622853353 71.1897571412771
17.5923481349494 70.0851437090286
18.0434339845635 68.9254462431761
18.4945198341775 67.7084169406231
18.9456056837916 66.4361600794377
19.3966915334057 65.1134156693994
19.8477773830198 63.7473437526011
20.2988632326339 62.3481431324098
20.749949082248 60.9267969118936
21.2010349318621 59.4965846292129
21.6521207814761 58.0731428452207
22.1032066310902 56.67250977745
};
\addplot [semithick, white!20!black]
table {%
0 111.648046791416
0.119095343109752 111.509834693093
0.238190686219504 111.357683034889
0.357286029329257 111.193911918922
0.476381372439009 111.020978544064
0.595476715548761 110.841358075036
0.714572058658513 110.657482795757
0.833667401768265 110.471619979612
0.952762744878018 110.285557211059
1.07185808798777 110.099944302902
1.19095343109752 109.913829183945
1.31004877420727 109.725662455349
1.42914411731703 109.535205030421
1.54823946042678 109.342664739441
1.66733480353653 109.14816031381
1.78643014664628 108.951687526995
1.90552548975604 108.753247188819
2.02462083286579 108.552902211196
2.14371617597554 108.350667687183
2.26281151908529 108.146524698278
2.38190686219504 107.940812739349
2.5010022053048 107.734027567928
2.62009754841455 107.526378144527
2.7391928915243 107.317877467676
2.85828823463405 107.108438988821
2.9773835777438 106.897933918072
3.09647892085356 106.686262525517
3.21557426396331 106.473405747716
3.33466960707306 106.259322764084
3.45376495018281 106.044016610731
3.57286029329257 105.827515635819
3.69195563640232 105.609632912825
3.81105097951207 105.390069773179
3.93014632262182 105.16844950969
4.04924166573158 104.94443744994
4.16833700884133 104.71769473816
4.28743235195108 104.487905105501
4.40652769506083 104.254831739376
4.52562303817058 104.018316962499
4.64471838128034 103.777999986857
4.76381372439009 103.532865671564
4.88290906749984 103.28142252432
5.00200441060959 103.022665956893
5.12109975371934 102.75603087396
5.2401950968291 102.481444975097
5.35929043993885 102.19923500542
5.4783857830486 101.909835796745
5.59748112615835 101.613916872531
5.71657646926811 101.312338875388
5.83567181237786 101.005987858411
};
\addplot [semithick, white!20!black]
table {%
0 90.960735052076
0.168472988144968 90.7372198618801
0.336945976289937 90.5010222996419
0.505418964434905 90.2542621036422
0.673891952579874 89.9992840584126
0.842364940724842 89.7385366668374
1.01083792886981 89.4744333530273
1.17931091701478 89.2092540804866
1.34778390515975 88.944800991997
1.51625689330472 88.6816178740528
1.68472988144968 88.4181456854211
1.85320286959465 88.1518992262405
2.02167585773962 87.8821981120271
2.19014884588459 87.6090403793807
2.35862183402956 87.332371327198
2.52709482217453 87.0520641643332
2.69556781031949 86.7680311023433
2.86404079846446 86.4802586500142
3.03251378660943 86.1887236241782
3.2009867747544 85.8934430447455
3.36945976289937 85.5949038396014
3.53793275104434 85.2939174930143
3.7064057391893 84.9908483007383
3.87487872733427 84.6856359197975
4.04335171547924 84.3780008340007
4.21182470362421 84.0675887754641
4.38029769176918 83.7540732694455
4.54877067991415 83.4372004739483
4.71724366805911 83.1166246421009
4.88571665620408 82.7919559134685
5.05418964434905 82.4626727888045
5.22266263249402 82.1278798400915
5.39113562063899 81.7865811761474
5.55960860878396 81.4377267748788
5.72808159692893 81.080435561873
5.89655458507389 80.7138597007643
6.06502757321886 80.3371381710064
6.23350056136383 79.9495405182428
6.4019735495088 79.5504782018459
6.57044653765377 79.1391390205178
6.73891952579874 78.7142909745879
6.9073925139437 78.2747949699118
7.07586550208867 77.820652853402
7.24433849023364 77.3527168115097
7.41281147837861 76.8725709358858
7.58128446652358 76.3826162308887
7.74975745466855 75.885468392465
7.91823044281351 75.3842983436757
8.08670343095848 74.882847018116
8.25517641910345 74.3849602950541
};
\addplot [semithick, white!20!black]
table {%
0 92.9287197426693
0.0536943425204878 92.8575624163397
0.107388685040976 92.7759928988909
0.161083027561463 92.685501230108
0.214777370081951 92.5875708997623
0.268471712602439 92.4836494935069
0.322166055122927 92.3751446517628
0.375860397643415 92.2632898017931
0.429554740163902 92.1489567111502
0.48324908268439 92.0324332946147
0.536943425204878 91.9135164650877
0.590637767725366 91.7919857356594
0.644332110245854 91.6681210588263
0.698026452766342 91.5422779909926
0.751720795286829 91.4147422925388
0.805415137807317 91.2857109026646
0.859109480327805 91.1553511169687
0.912803822848293 91.023785288519
0.966498165368781 90.8910908044659
1.02019250788927 90.7573275877864
1.07388685040976 90.6227041928262
1.12758119293024 90.4873527249458
1.18127553545073 90.3512657054692
1.23496987797122 90.2144889688762
1.28866422049171 90.0771090912418
1.3423585630122 89.9392084232732
1.39605290553268 89.80085941136
1.44974724805317 89.6621392154582
1.50344159057366 89.5231694043116
1.55713593309415 89.3841374230813
1.61083027561463 89.2453490304736
1.66452461813512 89.1072355922224
1.71821896065561 88.9702919773574
1.7719133031761 88.8349544338986
1.82560764569659 88.7015763240579
1.87930198821707 88.5704700299973
1.93299633073756 88.4419720155187
1.98669067325805 88.3164458642323
2.04038501577854 88.1942883550447
2.09407935829902 88.0758929460751
2.14777370081951 87.9611994068992
2.20146804334 87.8493943521738
2.25516238586049 87.7394799721723
2.30885672838098 87.6305611266951
2.36255107090146 87.521801712285
2.41624541342195 87.4122734022718
2.46993975594244 87.3010296085063
2.52363409846293 87.186968084899
2.57732844098341 87.0688156219636
2.6310227835039 86.9452771340349
};
\addplot [semithick, white!20!black]
table {%
0 79.1997214064669
0.0408065693209175 79.1317789146247
0.0816131386418351 79.055249731046
0.122419707962753 78.971221848504
0.16322627728367 78.8807424909912
0.204032846604588 78.7848253616107
0.244839415925505 78.684447685441
0.285645985246423 78.5804177547555
0.32645255456734 78.4732316941538
0.367259123888258 78.3629967470213
0.408065693209175 78.2496454565079
0.448872262530093 78.1332386785123
0.48967883185101 78.0141443951318
0.530485401171928 77.8927189405607
0.571291970492846 77.7692673434396
0.612098539813763 77.6440346370575
0.652905109134681 77.5172316213331
0.693711678455598 77.3889832726156
0.734518247776516 77.259381778236
0.775324817097433 77.1285301237537
0.816131386418351 76.9966250663957
0.856937955739268 76.8637389221193
0.897744525060186 76.7298192325796
0.938551094381103 76.5949043036598
0.979357663702021 76.4590968750867
1.02016423302294 76.3225018544403
1.06097080234386 76.1851977190187
1.10177737166477 76.047233841278
1.14258394098569 75.9087112185881
1.18339051030661 75.769780491945
1.22419707962753 75.6307039331548
1.26500364894844 75.4919650570647
1.30581021826936 75.3541868180982
1.34661678759028 75.217948496864
1.3874233569112 75.0837312710233
1.42822992623211 74.9519708667256
1.46903649555303 74.8231173973586
1.50984306487395 74.6976411264398
1.55064963419487 74.5760445605988
1.59145620351578 74.4589038243642
1.6322627728367 74.3464922126423
1.67306934215762 74.238379200695
1.71387591147854 74.1338599887403
1.75468248079945 74.0323112116775
1.79548905012037 73.9330575830398
1.83629561944129 73.8352483254533
1.87710218876221 73.737992735141
1.91790875808312 73.6401749746335
1.95871532740404 73.5404673177503
1.99952189672496 73.4375234831113
};
\addplot [semithick, white!20!black]
table {%
0 86.2482157954172
0.1485208107374 86.0461514615604
0.297041621474801 85.8323684729955
0.445562432212201 85.6087570482367
0.594083242949602 85.377389216903
0.742604053687002 85.1404230001861
0.891124864424403 84.8999819488534
1.0396456751618 84.6580526402626
1.1881664858992 84.4161772304339
1.3366872966366 84.1747999730235
1.485208107374 83.9325915121797
1.6337289181114 83.6874701292904
1.78224972884881 83.4389152024372
1.93077053958621 83.1869726527425
2.07929135032361 82.9316400921785
2.22781216106101 82.6728512596931
2.37633297179841 82.4105680938341
2.52485378253581 82.1447961411665
2.67337459327321 81.8755309718432
2.82189540401061 81.6028109214402
2.97041621474801 81.3270815786782
3.11893702548541 81.0490423065152
3.26745783622281 80.7689916021601
3.41597864696021 80.4868805319459
3.56449945769761 80.2024841804986
3.71302026843501 79.9155145876351
3.86154107917241 79.6257005790089
4.01006188990981 79.3328223160563
4.15858270064721 79.0365888197812
4.30710351138461 78.7366737606909
4.45562432212201 78.4326500294791
4.60414513285941 78.1238177524258
4.75266594359681 77.8094259389644
4.90118675433421 77.4886742857237
5.04970756507161 77.160892392371
5.19822837580901 76.825431594517
5.34674918654641 76.4816314927658
5.49526999728382 76.1289459549476
5.64379080802122 75.7669559420714
5.79231161875862 75.3950759699579
5.94083242949602 75.0123509057398
6.08935324023342 74.6178237477333
6.23787405097082 74.2114694104511
6.38639486170822 73.7940059036935
6.53491567244562 73.366753347584
6.68343648318302 72.9316975550804
6.83195729392042 72.4910001636758
6.98047810465782 72.0472582119758
7.12899891539522 71.6035252618094
7.27751972613262 71.1629441380462
};
\addplot [semithick, white!20!black]
table {%
0 118.594103493837
0.0439351135739906 118.561293583664
0.0878702271479812 118.515393741055
0.131805340721972 118.458446098882
0.175740454295962 118.392481562411
0.219675567869953 118.319450249702
0.263610681443944 118.241254583604
0.307545795017934 118.159601293388
0.351480908591925 118.07577870304
0.395416022165915 117.990330692749
0.439351135739906 117.903199848287
0.483286249313897 117.814308796429
0.527221362887887 117.724055335754
0.571156476461878 117.63289737509
0.615091590035869 117.541184278605
0.659026703609859 117.44911904442
0.70296181718385 117.356863490988
0.74689693075784 117.264574606796
0.790832044331831 117.172331918559
0.834767157905822 117.080129091643
0.878702271479812 116.988115442413
0.922637385053803 116.89633516331
0.966572498627793 116.804755078801
1.01050761220178 116.713465603478
1.05444272577577 116.622630031489
1.09837783934977 116.532416216997
1.14231295292376 116.443001374248
1.18624806649775 116.354609530766
1.23018318007174 116.267535499627
1.27411829364573 116.182201566773
1.31805340721972 116.09923309798
1.36198852079371 116.01935125632
1.4059236343677 115.943249503299
1.44985874794169 115.871534055312
1.49379386151568 115.804682959696
1.53772897508967 115.743119914144
1.58166408866366 115.68732123444
1.62559920223765 115.637773069043
1.66953431581164 115.594964945403
1.71346942938563 115.559306396112
1.75740454295962 115.530467685081
1.80133965653362 115.507030166825
1.84527477010761 115.487170300402
1.8892098836816 115.468989008395
1.93314499725559 115.450653754182
1.97708011082958 115.430128089609
2.02101522440357 115.405329287439
2.06495033797756 115.373939609484
2.10888545155155 115.333330082417
2.15282056512554 115.280806975129
};
\addplot [semithick, white!20!black]
table {%
0 113.34919347893
0.0880571429117072 113.253345824289
0.176114285823414 113.144040895868
0.264171428735121 113.023456279582
0.352228571646829 112.893844718656
0.440285714558536 112.757437652325
0.528342857470243 112.616421986713
0.61640000038195 112.472807771271
0.704457143293657 112.328153724209
0.792514286205364 112.183053198282
0.880571429117072 112.036922189385
0.968628572028779 111.888821450677
1.05668571494049 111.738773247012
1.14474285785219 111.587085387173
1.2328000007639 111.433970258235
1.32085714367561 111.279510191995
1.40891428658731 111.123773732127
1.49697142949902 110.966861583615
1.58502857241073 110.808815741587
1.67308571532244 110.649625684453
1.76114285823414 110.489553999227
1.84920000114585 110.328911345088
1.93725714405756 110.16780641921
2.02531428696926 110.006282562316
2.11337142988097 109.844354313865
2.20142857279268 109.682013171421
2.28948571570439 109.519270528583
2.37754285861609 109.356202273697
2.4656000015278 109.192900077509
2.55365714443951 109.029531605883
2.64171428735121 108.866360340141
2.72977143026292 108.70356625647
2.81782857317463 108.541257108002
2.90588571608634 108.379460100263
2.99394285899804 108.218175278905
3.08200000190975 108.057378112704
3.17005714482146 107.897077897699
3.25811428773316 107.737335023141
3.34617143064487 107.578258671031
3.43422857355658 107.419810122429
3.52228571646829 107.26127418059
3.61034285937999 107.101218412675
3.6984000022917 106.938333946159
3.78645714520341 106.771543962629
3.87451428811511 106.60008544981
3.96257143102682 106.423347017824
4.05062857393853 106.240761992409
4.13868571685024 106.05180528801
4.22674285976194 105.855939023174
4.31480000267365 105.652614960465
};
\addplot [semithick, white!20!black]
table {%
0 88.6785831391575
0.160852158251774 88.4627810113136
0.321704316503549 88.2347188957095
0.482556474755323 87.9964174115665
0.643408633007098 87.750105426557
0.804260791258872 87.4981093801433
0.965112949510647 87.2427210041077
1.12596510776242 86.9860975685748
1.2868172660142 86.7299325398683
1.44766942426597 86.474726435571
1.60852158251774 86.2190073693651
1.76937374076952 85.9604451880587
1.93022589902129 85.6984198448358
2.09107805727307 85.432946305333
2.25193021552484 85.1639891943217
2.41278237377662 84.8914452463639
2.57363453202839 84.6152461802049
2.73448669028017 84.3353853396502
2.89533884853194 84.0518468256996
3.05619100678371 83.7646571218366
3.21704316503549 83.4742880388813
3.37789532328726 83.1815088059395
3.53874748153904 82.8866584748615
3.69959963979081 82.5896803881991
3.86045179804259 82.2903150226629
4.02130395629436 81.9882325220337
4.18215611454614 81.6831261940265
4.34300827279791 81.3747529392883
4.50386043104968 81.0627854050488
4.66471258930146 80.7468544840991
4.82556474755323 80.4264700160807
4.98641690580501 80.1008079189981
5.14726906405678 79.7689644841312
5.30812122230856 79.4299842462258
5.46897338056033 79.0830662258288
5.62982553881211 78.727438421742
5.79067769706388 78.3623157223843
5.95152985531565 77.9870375174757
6.11238201356743 77.6010798387195
6.2732341718192 77.2037187151108
6.43408633007098 76.7938348781743
6.59493848832275 76.3703707543727
6.75579064657453 75.9333319113551
6.9166428048263 75.4835354616507
7.07749496307808 75.0224796234024
7.23834712132985 74.5524224185936
7.39919927958162 74.0758218706028
7.5600514378334 73.5956458193847
7.72090359608517 73.1153902296116
7.88175575433695 72.6386506285707
};
\addplot [semithick, white!20!black]
table {%
0 93.838164716714
0.0540520957828643 93.7674443117741
0.108104191565729 93.6862015568703
0.162156287348593 93.5959502014839
0.216208383131457 93.4981987060714
0.270260478914322 93.3944188767913
0.324312574697186 93.2860423263111
0.37836467048005 93.1743259342697
0.432416766262914 93.0601621662717
0.486468862045779 92.9438497069837
0.540520957828643 92.8251823338451
0.594573053611507 92.7039307255361
0.648625149394372 92.5803731408551
0.702677245177236 92.4548666055551
0.7567293409601 92.3276970632369
0.810781436742965 92.1990596732423
0.864833532525829 92.0691199606262
0.918885628308693 91.938000683977
0.972937724091557 91.8057786867523
1.02698981987442 91.6725112480514
1.08104191565729 91.5384065126826
1.13509401144015 91.4035976210237
1.18914610722301 91.2680784655339
1.24319820300588 91.1318958263503
1.29725029878874 90.9951367755315
1.35130239457161 90.8578840400834
1.40535449035447 90.7202113670301
1.45940658613734 90.5821991477328
1.5134586819202 90.4439722884148
1.56751077770306 90.3057231048701
1.62156287348593 90.1677637279996
1.67561496926879 90.0305277431878
1.72966706505166 89.8945079524737
1.78371916083452 89.7601375367623
1.83777125661739 89.6277667189443
1.89182335240025 89.4977047197147
1.94587544818311 89.3702856361584
1.99992754396598 89.2458707026916
2.05397963974884 89.1248539491831
2.10803173553171 89.0076219720169
2.16208383131457 88.8940976335824
2.21613592709744 88.7834438053486
2.2701880228803 88.6746391952919
2.32424011866317 88.5667633950636
2.37829221444603 88.4589594541184
2.43234431022889 88.3502797975964
2.48639640601176 88.2397590204314
2.54044850179462 88.1262775963948
2.59450059757749 88.0085445708303
2.64855269336035 87.8852463556108
};
\addplot [semithick, white!20!black]
table {%
0 90.268277842297
0.100334426132767 90.133357103172
0.200668852265534 89.98731118736
0.301003278398301 89.8318418144521
0.401337704531069 89.6687371670414
0.501672130663836 89.4998086679848
0.602006556796603 89.3268303077524
0.70234098292937 89.1514197380742
0.802675409062137 88.9747906110201
0.903009835194905 88.7973145472645
1.00334426132767 88.6182352532529
1.10367868746044 88.43641558875
1.20401311359321 88.2517426339949
1.30434753972597 88.064421416978
1.40468196585874 87.8745968231713
1.50501639199151 87.6823358111308
1.60535081812427 87.4877040060218
1.70568524425704 87.2907668761841
1.80601967038981 87.0915614101051
1.90635409652258 86.8901351986321
2.00668852265534 86.6868125083522
2.10702294878811 86.482003709018
2.20735737492088 86.2758519602136
2.30769180105364 86.0683573569413
2.40802622718641 85.8594543873522
2.50836065331918 85.6490444524774
2.60869507945195 85.437032727565
2.70902950558471 85.2233532065415
2.80936393171748 85.0079278222604
2.90969835785025 84.7906958771686
3.01003278398301 84.571608784596
3.11036721011578 84.3505458544698
3.21070163624855 84.1273909457533
3.31103606238132 83.9019731819651
3.41137048851408 83.6741428748717
3.51170491464685 83.4437400297185
3.61203934077962 83.2106117392407
3.71237376691239 82.9746748708969
3.81270819304515 82.7359251662648
3.91304261917792 82.4942724740316
4.01337704531069 82.2492073558592
4.11371147144345 81.9998302659357
4.21404589757622 81.7456036591648
4.31438032370899 81.4864048398479
4.41471474984175 81.222438845125
4.51504917597452 80.954190424055
4.61538360210729 80.6822213234184
4.71571802824006 80.407227878746
4.81605245437282 80.1300421943466
4.91638688050559 79.851528374591
};
\addplot [semithick, white!20!black]
table {%
0 83.07432714363
0.0185026468579405 83.0394688032923
0.0370052937158811 82.9960717384619
0.0555079405738216 82.9451845251273
0.0740105874317622 82.8877720227669
0.0925132342897027 82.8247339171518
0.111015881147643 82.7569314656151
0.129518528005584 82.6850466062349
0.148021174863524 82.6094625574405
0.166523821721465 82.5302748777178
0.185026468579405 82.4476837617526
0.203529115437346 82.3621833749461
0.222031762295286 82.2743332700272
0.240534409153227 82.1845691095684
0.259037056011168 82.0932668274717
0.277539702869108 82.0007314246299
0.296042349727049 81.9072195485623
0.314544996584989 81.812885734743
0.33304764344293 81.7178408353239
0.35155029030087 81.622186598659
0.370052937158811 81.5260608166299
0.388555584016751 81.4293996584852
0.407058230874692 81.3320792321833
0.425560877732632 81.2341633453276
0.444063524590573 81.1358321211543
0.462566171448513 81.0372819374595
0.481068818306454 80.938678493765
0.499571465164395 80.8401517434727
0.518074112022335 80.7419117786711
0.536576758880276 80.6442468637905
0.555079405738216 80.5476141638754
0.573582052596157 80.4527792575303
0.592084699454097 80.3606646243655
0.610587346312038 80.2721443296186
0.629089993169978 80.1879421296901
0.647592640027919 80.1087209405156
0.666095286885859 80.0351687811619
0.6845979337438 79.967972556023
0.703100580601741 79.9078274687437
0.721603227459681 79.8555320904925
0.740105874317622 79.8115368591595
0.758608521175562 79.7753878534552
0.777111168033503 79.7460845690553
0.795613814891443 79.7225469480193
0.814116461749384 79.7035211218365
0.832619108607324 79.6873963570249
0.851121755465265 79.6724756014942
0.869624402323205 79.6566959396896
0.888127049181146 79.6376279976956
0.906629696039086 79.6127948060663
};
\addplot [semithick, white!20!black]
table {%
0 44.6619627003992
0.0416826118654716 44.5584768993385
0.0833652237309432 44.4502739148792
0.125047835596415 44.337626555892
0.166730447461886 44.2207483018045
0.208413059327358 44.099864872246
0.250095671192829 43.9751754787134
0.291778283058301 43.8467351026067
0.333460894923773 43.7143758734599
0.375143506789244 43.5778309706667
0.416826118654716 43.4369814544418
0.458508730520187 43.2919378434286
0.500191342385659 43.1430117883782
0.541873954251131 42.9904594739288
0.583556566116602 42.834536408787
0.625239177982074 42.6755142199835
0.666921789847545 42.5136385416291
0.708604401713017 42.3490020575592
0.750287013578488 42.1817048683745
0.79196962544396 42.0119442797827
0.833652237309432 41.8399673316367
0.875334849174903 41.6658923432765
0.917017461040375 41.4896616869593
0.958700072905846 41.3112647227452
1.00038268477132 41.1307396695193
1.04206529663679 40.9481225999094
1.08374790850226 40.7633930214623
1.12543052036773 40.5764370360602
1.1671131322332 40.3871713101307
1.20879574409868 40.1954905786663
1.25047835596415 40.0013133062088
1.29216096782962 39.8048732549393
1.33384357969509 39.6066847366177
1.37552619156056 39.4072566617638
1.41720880342603 39.2070340794988
1.45889141529151 39.0064267114507
1.50057402715698 38.8058240355808
1.54225663902245 38.6056479664074
1.58393925088792 38.406381345009
1.62562186275339 38.2087081260678
1.66730447461886 38.0133924491525
1.70898708648434 37.8208582829776
1.75066969834981 37.6314113640561
1.79235231021528 37.4455997775723
1.83403492208075 37.2638372020227
1.87571753394622 37.0864153226257
1.91740014581169 36.9135987925777
1.95908275767716 36.7454581586647
2.00076536954264 36.5819623521214
2.04244798140811 36.4231061761103
};
\addplot [semithick, white!20!black]
table {%
0 108.510876607409
0.175513760781426 108.295613480988
0.351027521562852 108.065539203113
0.526541282344278 107.823231860516
0.702055043125705 107.571519198078
0.877568803907131 107.313318365913
1.05308256468856 107.051506649682
1.22859632546998 106.78881788914
1.40411008625141 106.527454806605
1.57962384703284 106.268169352902
1.75513760781426 106.009340364362
1.93065136859569 105.748309450551
2.10616512937711 105.484362399863
2.28167889015854 105.217525183816
2.45719265093997 104.947746233905
2.63270641172139 104.674864023609
2.80822017250282 104.398756290236
2.98373393328424 104.119417208174
3.15924769406567 103.83681298165
3.3347614548471 103.550909552337
3.51027521562852 103.262186273957
3.68578897640995 102.97147541217
3.86130273719137 102.679168166635
4.0368164979728 102.385222319893
4.21233025875423 102.089367494358
4.38784401953565 101.791256167211
4.56335778031708 101.490586493337
4.73887154109851 101.187166605141
4.91438530187993 100.880714615822
5.08989906266136 100.570933984808
5.26541282344278 100.257425216865
5.44092658422421 99.9393341508708
5.61644034500564 99.6156232351208
5.79195410578706 99.2851814897115
5.96746786656849 98.9470657873576
6.14298162734991 98.6003658993773
6.31849538813134 98.2441737075809
6.49400914891277 97.8777121207862
6.66952290969419 97.5003383733484
6.84503667047562 97.1111064018599
7.02055043125705 96.7084565054409
7.19606419203847 96.290790955455
7.3715779528199 95.857659570685
7.54709171360132 95.4094289284451
7.72260547438275 94.9472836362689
7.89811923516418 94.473257151174
8.0736329959456 93.9896062215701
8.24914675672703 93.4991733258057
8.42466051750846 93.0053628559388
8.60017427828988 92.5116696841536
};
\addplot [semithick, white!20!black]
table {%
0 100.920841956054
0.0482155478918884 100.864822843807
0.0964310957837769 100.797610666064
0.144646643675665 100.720853003865
0.192862191567554 100.636184484075
0.241077739459442 100.545186939248
0.289293287351331 100.44939975198
0.337508835243219 100.350180566064
0.385724383135107 100.248510273676
0.433939931026996 100.14475060185
0.482155478918884 100.038772598671
0.530371026810773 99.9304485635485
0.578586574702661 99.8201154999852
0.62680212259455 99.7081683182147
0.675017670486438 99.5949197648293
0.723233218378326 99.4805755823754
0.771448766270215 99.3653068276661
0.819664314162103 99.2492494888427
0.867879862053992 99.1324837676094
0.91609540994588 99.0150499811855
0.964310957837769 98.8971323784776
1.01252650572966 98.7788212972262
1.06074205362155 98.6600933444802
1.10895760151343 98.5410104354711
1.15717314940532 98.4216907431179
1.20538869729721 98.3022524452852
1.2536042451891 98.1828089762131
1.30181979308099 98.0634900711567
1.35003534097288 97.9444809611048
1.39825088886476 97.8260541867066
1.44646643675665 97.7086322070488
1.49468198464854 97.5927645425943
1.54289753254043 97.4790394740557
1.59111308043232 97.3679776985696
1.63932862832421 97.2599976124441
1.68754417621609 97.1554709026461
1.73575972410798 97.0548029744705
1.78397527199987 96.9584187848759
1.83219081989176 96.8667649241805
1.88040636778365 96.7802656260439
1.92862191567554 96.6988022502111
1.97683746356743 96.6213808970654
2.02505301145931 96.5467265926888
2.0732685593512 96.4735958350536
2.12148410724309 96.4007920865639
2.16969965513498 96.326972538313
2.21791520302687 96.2507624570055
2.26613075091876 96.1705916611629
2.31434629881064 96.0846597019223
2.36256184670253 95.9911282579042
};
\addplot [semithick, white!20!black]
table {%
0 68.7117633669838
0.406383437093508 68.1543495282995
0.812766874187016 67.5815934340629
1.21915031128052 66.9964884748813
1.62553374837403 66.402722435783
2.03191718546754 65.8043997500936
2.43830062256105 65.2056050871815
2.84468405965456 64.6103869643915
3.25106749674806 64.0221227772928
3.65745093384157 63.4416887744429
4.06383437093508 62.8646936758965
4.47021780802859 62.2839932742671
4.8766012451221 61.6968908311915
5.28298468221561 61.1025920508401
5.68936811930911 60.5003118362875
6.09575155640262 59.8892676729083
6.50213499349613 59.2688623985109
6.90851843058964 58.6387844967936
7.31490186768315 57.9988068936805
7.72128530477665 57.348907494927
8.12766874187016 56.6901755806522
8.53405217896367 56.0248523186686
8.94043561605718 55.3540683957701
9.34681905315069 54.677518052552
9.7532024902442 53.994130457906
10.1595859273377 53.3026118564748
10.5659693644312 52.6017584126593
10.9723528015247 51.8905455896454
11.3787362386182 51.1675638612498
11.7851196757117 50.4310943740315
12.1915031128052 49.678723280894
12.5978865498988 48.9066788889001
13.0042699869923 48.1108241029151
13.4106534240858 47.286997129194
13.8170368611793 46.4317441262979
14.2234202982728 45.5418033586112
14.6298037353663 44.6138045856197
15.0361871724598 43.644728605162
15.4425706095533 42.6319381375419
15.8489540466468 41.5721835149376
16.2553374837403 40.4620669674262
16.6617209208338 39.3002253191494
17.0681043579273 38.0892603246583
17.4744877950208 36.8342555184259
17.8808712321144 35.54237492979
18.2872546692079 34.2235084405882
18.6936381063014 32.8882493345284
19.1000215433949 31.549233102484
19.5064049804884 30.2212561807215
19.9127884175819 28.9195056976428
};
\addplot [semithick, white!20!black]
table {%
0 87.6924944139729
0.0516462366044065 87.6188061407657
0.103292473208813 87.5353396736868
0.154938709813219 87.4434486103787
0.206584946417626 87.3444727619063
0.258231183022033 87.2397203684281
0.309877419626439 87.1304611546836
0.361523656230846 87.0177936294767
0.413169892835252 86.9024704878486
0.464816129439658 86.7847176640817
0.516462366044065 86.6643499858935
0.568108602648472 86.541197600254
0.619754839252878 86.4155501137274
0.671401075857284 86.287754586922
0.723047312461691 86.1580956908869
0.774693549066098 86.0267805793822
0.826339785670504 85.893986715894
0.877986022274911 85.7598341050002
0.929632258879317 85.6244032539226
0.981278495483724 85.4877693031198
1.03292473208813 85.350143192979
1.08457096869254 85.2116511386347
1.13621720529694 85.0722778048853
1.18786344190135 84.9320635709478
1.23950967850576 84.7910921202743
1.29115591511016 84.6494435928805
1.34280215171457 84.50718290652
1.39444838831898 84.3643685831979
1.44609462492338 84.2211029291975
1.49774086152779 84.0775452891292
1.5493870981322 83.9339646507781
1.6010333347366 83.7907794661946
1.65267957134101 83.6484963539776
1.70432580794541 83.507569070274
1.75597204454982 83.3683689266315
1.80761828115423 83.2312263895637
1.85926451775863 83.0964914277659
1.91091075436304 82.9645410393281
1.96255699096745 82.835787730697
2.01420322757185 82.7106643436153
2.06584946417626 82.5892078702585
2.11749570078067 82.4707415910209
2.16914193738507 82.3544029940183
2.22078817398948 82.2394425944134
2.27243441059389 82.1251445584618
2.32408064719829 82.0106917157531
2.3757268838027 81.8952461749724
2.42737312040711 81.7778056109971
2.47901935701151 81.6571994876719
2.53066559361592 81.5322397655202
};
\addplot [semithick, white!20!black]
table {%
0 70.2706138612587
0.213514091590202 69.9674750851822
0.427028183180404 69.6530055840303
0.640542274770606 69.3290991083175
0.854056366360808 68.9979536087502
1.06757045795101 68.6619513316053
1.28108454954121 68.3234470847542
1.49459864113141 67.9846922802014
1.70811273272162 67.6474669678232
1.92162682431182 67.3121984000612
2.13514091590202 66.9767717040664
2.34865500749222 66.6378515934827
2.56216909908242 66.2943525597492
2.77568319067263 65.94607613577
2.98919728226283 65.5928069463036
3.20271137385303 65.2343079316632
3.41622546544323 64.8704124248803
3.62973955703344 64.501035594191
3.84325364862364 64.1261197993305
4.05676774021384 63.745719831128
4.27028183180404 63.3604596406769
4.48379592339424 62.9714408620731
4.69731001498445 62.5791690621164
4.91082410657465 62.1835144603061
5.12433819816485 61.7840182664666
5.33785228975505 61.3801170845376
5.55136638134525 60.9712725524918
5.76488047293546 60.5570081963317
5.97839456452566 60.1366905948813
6.19190865611586 59.7095582406551
6.40542274770606 59.2745702879281
6.61893683929626 58.8301724002191
6.83245093088647 58.3747275699094
7.04596502247667 57.906568482002
7.25947911406687 57.4243142396093
7.47299320565707 56.9266518225594
7.68650729724727 56.4122205865983
7.90002138883748 55.879837680901
8.11353548042768 55.3285209014767
8.32704957201788 54.7570524472868
8.54056366360808 54.1640288485051
8.75407775519828 53.5486776304758
8.96759184678849 52.9119715480944
9.18110593837869 52.2561166410278
9.39462002996889 51.5842643464314
9.60813412155909 50.9007666366295
9.8216482131493 50.2102880038828
10.0351623047395 49.5183419017637
10.2486763963297 48.8313629242782
10.4621904879199 48.1559654518801
};
\addplot [semithick, white!20!black]
table {%
0 71.8805945650883
0.0817026348615181 71.7517757409073
0.163405269723036 71.6143064480011
0.245107904584554 71.4693418529653
0.326810539446072 71.3180749753509
0.40851317430759 71.1617236298412
0.490215809169108 71.0014727911168
0.571918444030626 70.8383576971655
0.653621078892144 70.6730773212783
0.735323713753663 70.5057579933546
0.817026348615181 70.3358412397294
0.898728983476699 70.1625938242946
0.980431618338217 69.9860321639555
1.06213425319973 69.8063665319129
1.14383688806125 69.6237716543768
1.22553952292277 69.4383827899038
1.30724215778429 69.2503269018545
1.38894479264581 69.0596745731059
1.47064742750733 68.8664838215056
1.55235006236884 68.6708604898079
1.63405269723036 68.4731097525495
1.71575533209188 68.2735538156855
1.7974579669534 68.0722711505939
1.87916060181492 67.8692530042821
1.96086323667643 67.6644598435645
2.04256587153795 67.4578284504964
2.12426850639947 67.2492767766738
2.20597114126099 67.0387054430225
2.28767377612251 66.8260142885301
2.36937641098402 66.6111000461246
2.45107904584554 66.3938655358233
2.53278168070706 66.1742755951144
2.61448431556858 65.9524033695462
2.6961869504301 65.7282872608825
2.77788958529161 65.5019634927239
2.85959222015313 65.2734511135082
2.94129485501465 65.0427636968211
3.02299748987617 64.8099740286392
3.10470012473769 64.5752312143198
3.1864027595992 64.338704232192
3.26810539446072 64.1003446795398
3.34980802932224 63.8597702612535
3.43151066418376 63.6168246137662
3.51321329904528 63.3717289587395
3.59491593390679 63.124875545317
3.67661856876831 62.876813985492
3.75832120362983 62.6281390978968
3.84002383849135 62.3794787662486
3.92172647335287 62.131533052333
4.00342910821438 61.8850369808597
};
\addplot [semithick, white!20!black]
table {%
0 85.4296976634304
0.0332554426947368 85.377853943162
0.0665108853894736 85.3168860413792
0.0997663280842104 85.2479854241284
0.133021770778947 85.1722911444916
0.166277213473684 85.090891650975
0.199532656168421 85.0048368386891
0.232788098863158 84.9150006786397
0.266043541557894 84.8219366765836
0.299298984252631 84.7258016883293
0.332554426947368 84.6266247320767
0.365809869642105 84.5246038256472
0.399065312336842 84.4201787894302
0.432320755031578 84.3137466935569
0.465576197726315 84.2056432566188
0.498831640421052 84.096129673673
0.532087083115789 83.9854270266794
0.565342525810526 83.8736747521782
0.598597968505262 83.7609701198869
0.631853411199999 83.6474022127408
0.665108853894736 83.5331409641799
0.698364296589473 83.4182067818236
0.73161973928421 83.302524163713
0.764875181978946 83.1861469069986
0.798130624673683 83.0692127438097
0.83138606736842 82.9518668811815
0.864641510063157 82.834230991762
0.897896952757894 82.7164046292373
0.93115239545263 82.5985514778158
0.964407838147367 82.4809046010847
0.997663280842104 82.3638403333522
1.03091872353684 82.2479718216236
1.06417416623158 82.1340372829418
1.09742960892631 82.0227242080553
1.13068505162105 81.9145997058309
1.16394049431579 81.8101788620059
1.19719593701052 81.7099996705796
1.23045137970526 81.6146114384782
1.2637068224 81.5245836412916
1.29696226509474 81.4405510414463
1.33021770778947 81.36277570546
1.36347315048421 81.2906970998824
1.39672859317895 81.2233674355434
1.42998403587368 81.1598437431183
1.46323947856842 81.0991023483215
1.49649492126316 81.0398760478024
1.52975036395789 80.9808404217784
1.56300580665263 80.9203946246081
1.59626124934737 80.8566586873274
1.6295166920421 80.7877180626809
};
\addplot [semithick, white!20!black]
table {%
0 95.198461244892
0.0843941568797348 95.0893373935547
0.16878831375947 94.9688793379461
0.253182470639205 94.8388119570184
0.337576627518939 94.7009171711631
0.421970784398674 94.5569747214
0.506364941278409 94.4087249006586
0.590759098158144 94.2577427132181
0.675153255037879 94.1052031282824
0.759547411917614 93.9514929535037
0.843941568797348 93.7960497566465
0.928335725677083 93.6380418862611
1.01272988255682 93.477496402576
1.09712403943655 93.3146811267198
1.18151819631629 93.1497945562563
1.26591235319602 92.9829446882702
1.35030651007576 92.8142276120671
1.43470066695549 92.6437318779881
1.51909482383523 92.4715072705731
1.60348898071496 92.297594571685
1.6878831375947 92.1222728916856
1.77227729447443 91.9458527432367
1.85667145135417 91.7684267345829
1.9410656082339 91.590016199605
2.02545976511364 91.4106147952345
2.10985392199337 91.2301933951941
2.19424807887311 91.0487255158457
2.27864223575284 90.8662128033783
2.36303639263258 90.6826664744762
2.44743054951231 90.4981399323714
2.53182470639205 90.3127450040666
2.61621886327178 90.126577586267
2.70061302015151 89.9397416925523
2.78500717703125 89.7522808194031
2.86940133391098 89.5642203176642
2.95379549079072 89.3755636351258
3.03818964767045 89.1863311155804
3.12258380455019 88.9965969602329
3.20697796142992 88.8064953885097
3.29137211830966 88.6160879022641
3.37576627518939 88.4249603786607
3.46016043206913 88.232142905335
3.54455458894886 88.0368238215427
3.6289487458286 87.8384815443733
3.71334290270833 87.6368405963389
3.79773705958807 87.4317726721335
3.8821312164678 87.2231926125135
3.96652537334754 87.0110476669809
4.05091953022727 86.7953038363882
4.13531368710701 86.5759356465065
};
\addplot [semithick, white!20!black]
table {%
0 65.4416308035636
0.233526576152312 65.1074599739188
0.467053152304624 64.7620676483616
0.700579728456936 64.4073508211776
0.934106304609248 64.0455484258828
1.16763288076156 63.6791138434286
1.40115945691387 63.3104751555143
1.63468603306618 62.9419673229032
1.8682126092185 62.5754454115317
2.10173918537081 62.2113324276499
2.33526576152312 61.8472717331575
2.56879233767543 61.4795418718288
2.80231891382774 61.1068835262028
3.03584548998005 60.729023234264
3.26937206613237 60.3456801619379
3.50289864228468 59.9565645814432
3.73642521843699 59.5614701293703
3.9699517945893 59.1602842271916
4.20347837074161 58.7529328274655
4.43700494689393 58.339475538979
4.67053152304624 57.9205911068568
4.90405809919855 57.4975049236646
5.13758467535086 57.07078633734
5.37111125150317 56.6402807999069
5.60463782765549 56.2054576696613
5.8381644038078 55.7656689116149
6.07169097996011 55.3202941445633
6.30521755611242 54.8687782949858
6.53874413226473 54.4103829735386
6.77227070841705 53.9442133357125
7.00579728456936 53.4690404438499
7.23932386072167 52.9830474886468
7.47285043687398 52.4843248992367
7.70637701302629 51.9709385489037
7.93990358917861 51.4412888375823
8.17343016533092 50.8938582242823
8.40695674148323 50.3270705834198
8.64048331763554 49.7395471133549
8.87400989378785 49.1301322344022
9.10753646994016 48.4974131371709
9.34106304609248 47.8398469873967
9.57458962224479 47.1567158758093
9.8081161983971 46.4492971713305
10.0416427745494 45.720252056813
10.2751693507017 44.9732930466845
10.508695926854 44.213497754736
10.7422225030063 43.4462983761937
10.9757490791587 42.6781033421628
11.209275655311 41.9163850122532
11.4428022314633 41.1688232918685
};
\addplot [semithick, white!20!black]
table {%
0 86.4182712140218
0.0656366845923578 86.3249818551757
0.131273369184716 86.2217540429699
0.196910053777073 86.1099935615071
0.262546738369431 85.9911200286788
0.328183422961789 85.8665398492798
0.393820107554147 85.7376218034592
0.459456792146504 85.6055691535918
0.525093476738862 85.4712278919202
0.59073016133122 85.3348438751709
0.656366845923578 85.1960654204643
0.722003530515935 85.0544486331994
0.787640215108293 84.9101645956564
0.853276899700651 84.7635138712606
0.918913584293009 84.614738226154
0.984550268885366 84.4640062251939
1.05018695347772 84.3114653415051
1.11582363807008 84.1572180850532
1.18146032266244 84.0013329645645
1.2470970072548 83.8438827267733
1.31273369184716 83.6851136860811
1.37837037643951 83.5252360905439
1.44400706103187 83.3642796809028
1.50964374562423 83.2022704521344
1.57528043021659 83.0392456150455
1.64091711480894 82.8752301273756
1.7065537994013 82.7102374080751
1.77219048399366 82.5442808169969
1.83782716858602 82.3774002830829
1.90346385317837 82.2096772475226
1.96910053777073 82.0412697513426
2.03473722236309 81.8724273432989
2.10037390695545 81.7034720180381
2.16601059154781 81.5346746007231
2.23164727614016 81.3662551364055
2.29728396073252 81.1984021579535
2.36292064532488 81.0313181289975
2.42855732991724 80.8652454961741
2.4941940145096 80.7004763007712
2.55983069910195 80.5372999323095
2.62546738369431 80.3756255294805
2.69110406828667 80.2147623479783
2.75674075287903 80.0539997730791
2.82237743747138 79.8928360013252
2.88801412206374 79.7308822715175
2.9536508066561 79.5677606959495
3.01928749124846 79.40310139388
3.08492417584081 79.2364574680803
3.15056086043317 79.0673072625877
3.21619754502553 78.895128378611
};
\addplot [semithick, white!20!black]
table {%
0 61.6635374585754
0.0885240800000397 61.5156053397065
0.177048160000079 61.3600249789142
0.265572240000119 61.1977490673448
0.354096320000159 61.0297759658451
0.442620400000198 60.8571502045971
0.531144480000238 60.6808866689607
0.619668560000278 60.5018600656952
0.708192640000318 60.3206283232025
0.796716720000357 60.1372229693021
0.885240800000397 59.9509929099819
0.973764880000437 59.7610899008449
1.06228896000048 59.5674589486218
1.15081304000052 59.3702605768603
1.23933712000056 59.1696355502111
1.3278612000006 58.9657083907565
1.41638528000064 58.7586016626928
1.50490936000067 58.5483687312163
1.59343344000071 58.3350641768551
1.68195752000075 58.1188189830234
1.77048160000079 57.8999689576268
1.85900568000083 57.6788886248763
1.94752976000087 57.4556762086489
2.03605384000091 57.2303025630117
2.12457792000095 57.0026883384614
2.21310200000099 56.7727252045961
2.30162608000103 56.5402793404402
2.39015016000107 56.3051846671304
2.47867424000111 56.0672603654535
2.56719832000115 55.8262952786346
2.65572240000119 55.5820443234194
2.74424648000123 55.3343233783643
2.83277056000127 55.0830885351305
2.92129464000131 54.8282725955844
3.00981872000135 54.5698306057156
3.09834280000139 54.3077076434214
3.18686688000143 54.0418310484617
3.27539096000147 53.7721968845093
3.36391504000151 53.4988921448638
3.45243912000155 53.2220483933279
3.54096320000159 52.9416938072312
3.62948728000163 52.6576774551604
3.71801136000167 52.3701957983954
3.80653544000171 52.0799127265004
3.89505952000175 51.7876775112699
3.98358360000179 51.4945644398655
4.07210768000183 51.2017101030412
4.16063176000187 50.9103338955287
4.24915584000191 50.6218020345142
4.33767992000195 50.3375359518683
};
\addplot [semithick, white!20!black]
table {%
0 76.8226808728367
0.0644496360177684 76.7213909002585
0.128899272035537 76.6112685895503
0.193348908053305 76.4934848128801
0.257798544071074 76.3692162187851
0.322248180088842 76.2396372481594
0.38669781610661 76.1058874335841
0.451147452124379 75.9689471624413
0.515597088142147 75.8294658782499
0.580046724159916 75.6875820559598
0.644496360177684 75.5429465787102
0.708945996195452 75.3951576213889
0.773395632213221 75.2443825492642
0.837845268230989 75.0908984877214
0.902294904248757 74.9349376545253
0.966744540266526 74.7766800582108
1.03119417628429 74.6162860549878
1.09564381230206 74.4538508537914
1.16009344831983 74.2894464214168
1.2245430843376 74.1231723113932
1.28899272035537 73.9552853730522
1.35344235637314 73.7860001977043
1.4178919923909 73.6153404509234
1.48234162840867 73.4433198181661
1.54679126442644 73.2699621030826
1.61124090044421 73.0952785339285
1.67569053646198 72.919259936429
1.74014017247975 72.7418783200808
1.80458980849751 72.5631280968871
1.86903944451528 72.3830266259783
1.93348908053305 72.201646498764
1.99793871655082 72.0191842405119
2.06238835256859 71.8359501714128
2.12683798858636 71.6522140466125
2.19128762460413 71.4682012389043
2.25573726062189 71.2841075034851
2.32018689663966 71.1001333448719
2.38463653265743 70.9165213830725
2.4490861686752 70.7335704667143
2.51353580469297 70.5516150524362
2.57798544071074 70.3707156081431
2.6424350767285 70.1904234605774
2.70688471274627 70.0102970054799
2.77133434876404 69.8301389487921
2.83578398478181 69.6498336739718
2.90023362079958 69.4692800040548
2.96468325681735 69.2883854561753
3.02913289283511 69.1069803099811
3.09358252885288 68.9248415633393
3.15803216487065 68.7417562647796
};
\addplot [semithick, white!20!black]
table {%
0 56.7004244188666
0.161419549488515 56.452032467139
0.322839098977029 56.1949685571071
0.484258648465544 55.9304971011293
0.645678197954059 55.6600730806416
0.807097747442573 55.3852911275258
0.968517296931088 55.1077203878735
1.1299368464196 54.828818334521
1.29135639590812 54.5496615906809
1.45277594539663 54.2704037430503
1.61419549488515 53.9895281121188
1.77561504437366 53.70475534405
1.93703459386218 53.4154152687127
2.09845414335069 53.1214308716756
2.25987369283921 52.822721644984
2.42129324232772 52.5192096370204
2.58271279181624 52.2108593652658
2.74413234130475 51.8976345588927
2.90555189079326 51.5795268603241
3.06697144028178 51.2566501856673
3.22839098977029 50.9295223166281
3.38981053925881 50.5989536386838
3.55123008874732 50.2652776269753
3.71264963823584 49.9283925252524
3.87406918772435 49.5879798409456
4.03548873721287 49.2436469034674
4.19690828670138 48.8949962209296
4.3583278361899 48.54163420683
4.51974738567841 48.1830638087015
4.68116693516693 47.8186801740156
4.84258648465544 47.4476765305669
5.00400603414396 47.0690003412966
5.16542558363247 46.6816504601425
5.32684513312098 46.2846094157895
5.4882646826095 45.8770453957791
5.64968423209801 45.4581647831908
5.81110378158653 45.0271288612948
5.97252333107504 44.5832345722672
6.13394288056356 44.1259418304923
6.29536243005207 43.6546290781391
6.45678197954059 43.1686340635031
6.6182015290291 42.6676911751895
6.77962107851762 42.1527396247735
6.94104062800613 41.6256775884757
7.10246017749465 41.0890065363967
7.26387972698316 40.5460353505122
7.42529927647168 40.0002843596476
7.58671882596019 39.4558109306135
7.74813837544871 38.9173010949695
7.90955792493722 38.3895813174743
};
\addplot [semithick, white!20!black]
table {%
0 98.3600066755455
0.026103537394536 98.3304113307216
0.052207074789072 98.290391272837
0.078310612183608 98.2414029414649
0.104414149578144 98.1848434740083
0.13051768697268 98.122034121683
0.156621224367216 98.054253594101
0.182724761761752 97.9825930883313
0.208828299156288 97.9077971098048
0.234931836550824 97.830146070381
0.26103537394536 97.7497687386214
0.287138911339896 97.6669794103172
0.313242448734432 97.5822959240977
0.339345986128968 97.4961737697071
0.365449523523504 97.4089872875216
0.39155306091804 97.3210070616182
0.417656598312576 97.2324564228275
0.443760135707112 97.1434948678357
0.469863673101648 97.0542227124
0.495967210496184 96.9646965991958
0.52207074789072 96.8750512498439
0.548174285285256 96.7852495963805
0.574277822679792 96.6951959419464
0.600381360074328 96.604968557545
0.626484897468864 96.5147508868063
0.6525884348634 96.4247396435933
0.678691972257936 96.3351169449704
0.704795509652472 96.2460626044211
0.730899047047008 96.1578364869712
0.757002584441544 96.0708009329247
0.78310612183608 95.9855090309051
0.809209659230616 95.9027455004725
0.835313196625152 95.8233775329807
0.861416734019688 95.748207127578
0.887520271414224 95.6778882194054
0.91362380880876 95.6130145501316
0.939727346203296 95.5542178340271
0.965830883597832 95.5021303872445
0.991934420992368 95.4573875868057
1.0180379583869 95.4206559021551
1.04414149578144 95.3920849955026
1.07024503317598 95.3708166953635
1.09634857057051 95.3554688525911
1.12245210796505 95.3445599694325
1.14855564535958 95.336518423193
1.17465918275412 95.3294551405407
1.20076272014866 95.321405192214
1.22686625754319 95.3100734295842
1.25296979493773 95.2928005979004
1.27907333233226 95.2668688387658
};
\addplot [semithick, white!20!black]
table {%
0 98.2038517559354
0.0365919895398496 98.1603575582836
0.0731839790796992 98.1062285328605
0.109775968619549 98.0429792234071
0.146367958159398 97.9720860486631
0.182959947699248 97.8949622692002
0.219551937239098 97.8129790418699
0.256143926778947 97.7273237182114
0.292735916318797 97.638826281483
0.329327905858646 97.5477907829662
0.365919895398496 97.4542221125893
0.402511884938346 97.3582275711586
0.439103874478195 97.2602372754921
0.475695864018045 97.1606741120586
0.512287853557895 97.0598813421085
0.548879843097744 96.958099947362
0.585471832637594 96.8555299238874
0.622063822177443 96.7523183763829
0.658655811717293 96.6485564245539
0.695247801257143 96.5442967256824
0.731839790796992 96.4396994060816
0.768431780336842 96.3347894513143
0.805023769876691 96.2295051426941
0.841615759416541 96.1239150817636
0.878207748956391 96.0181693010453
0.91479973849624 95.9124246612015
0.95139172803609 95.8068268917498
0.987983717575939 95.7015256695289
1.02457570711579 95.5967382803334
1.06116769665564 95.4927744828889
1.09775968619549 95.3901120026217
1.13435167573534 95.2894144812753
1.17094366527519 95.1914129515154
1.20753565481504 95.0967736378789
1.24412764435489 95.0060376912265
1.28071963389474 94.9196928444274
1.31731162343459 94.8382614211675
1.35390361297444 94.7622757920403
1.39049560251429 94.6922813040751
1.42708759205413 94.6288341735012
1.46367958159398 94.5719766436592
1.50027157113383 94.5208201920493
1.53686356067368 94.4740733300762
1.57345555021353 94.4304134360328
1.61004753975338 94.3884893099696
1.64663952929323 94.3467153460433
1.68323151883308 94.3034513620656
1.71982350837293 94.2567918994707
1.75641549791278 94.2045350318786
1.79300748745263 94.1444317884129
};
\addplot [semithick, white!20!black]
table {%
0 92.8419082409422
0.105279118224213 92.7030713422543
0.210558236448425 92.5527121624994
0.315837354672638 92.392622691887
0.421116472896851 92.224692926227
0.526395591121064 92.0508386748068
0.631674709345276 91.8729377593072
0.736953827569489 91.6927115400272
0.842232945793702 91.5114654273553
0.947512064017914 91.3296110200365
1.05279118224213 91.1463367771659
1.15807030046634 90.9604028388088
1.26334941869055 90.7716586870053
1.36862853691477 90.5803014495613
1.47390765513898 90.3864649579494
1.57918677336319 90.1901999862338
1.6844658915874 89.9915583381802
1.78974500981162 89.7906020367573
1.89502412803583 89.5873630693914
2.00030324626004 89.3818798903485
2.10558236448425 89.1744850502295
2.21086148270847 88.9656150434155
2.31614060093268 88.7554296621964
2.42141971915689 88.5439281340759
2.52669883738111 88.3310339291014
2.63197795560532 88.1166346826645
2.73725707382953 87.9006257710657
2.84253619205374 87.6829393066402
2.94781531027796 87.4634910098543
3.05309442850217 87.242214679341
3.15837354672638 87.0190520785084
3.26365266495059 86.7938438110011
3.36893178317481 86.5664170032888
3.47421090139902 86.3365412953028
3.57949001962323 86.1040158755236
3.68476913784744 85.8686320872679
3.79004825607166 85.6301894007601
3.89532737429587 85.3885606321304
4.00060649252008 85.1437000103834
4.1058856107443 84.895456432072
4.21116472896851 84.6432314927385
4.31644384719272 84.3860456235494
4.42172296541693 84.123327116391
4.52700208364115 83.8549393706584
4.63228120186536 83.581109253145
4.73756032008957 83.3023789231245
4.84283943831378 83.019376717254
4.948118556538 82.7328942146093
5.05339767476221 82.4438826283308
5.15867679298642 82.1533272885795
};
\addplot [semithick, white!20!black]
table {%
0 72.4918592313813
0.0909505006312329 72.3515313893652
0.181901001262466 72.2022834940432
0.272851501893699 72.0453397327253
0.363802002524932 71.8819812589956
0.454752503156165 71.7135242744655
0.545703003787398 71.5412523113489
0.63665350441863 71.3663019047397
0.727604005049863 71.1894619543531
0.818554505681096 71.0108877908659
0.909505006312329 70.8299126068592
1.00045550694356 70.6456192024521
1.0914060075748 70.4579477116365
1.18235650820603 70.2670817741754
1.27330700883726 70.0731697311585
1.36425750946849 69.8763201096486
1.45520801009973 69.6766385003634
1.54615851073096 69.4741852380036
1.63710901136219 69.2690100437589
1.72805951199343 69.0612131930656
1.81901001262466 68.8511212154997
1.90996051325589 68.6391101452566
2.00091101388712 68.425288586615
2.09186151451836 68.2096403051048
2.18281201514959 67.9920976474593
2.27376251578082 67.7725636774522
2.36471301641206 67.5509263441275
2.45566351704329 67.3270631836102
2.54661401767452 67.1008404135444
2.63756451830575 66.8721138608214
2.72851501893699 66.6407272180153
2.81946551956822 66.4065437645604
2.91041602019945 66.1695187024186
3.00136652083069 65.9295719988589
3.09231702146192 65.6866410128448
3.18326752209315 65.4406516949009
3.27421802272438 65.191522284041
3.36516852335562 64.9392383101097
3.45611902398685 64.6838697608056
3.54706952461808 64.4254858631792
3.63802002524932 64.1639326586807
3.72897052588055 63.8987824995019
3.81992102651178 63.6299372017571
3.91087152714302 63.3577329465468
4.00182202777425 63.0827331102194
4.09277252840548 62.8057306440247
4.18372302903671 62.5275822272998
4.27467352966795 62.249234310515
4.36562403029918 61.9717628905621
4.45657453093041 61.696288176358
};
\addplot [semithick, white!20!black]
table {%
0 52.9070425400879
0.114755818365252 52.7160177642951
0.229511636730504 52.5177611502534
0.344267455095756 52.3131719988115
0.459023273461009 52.1032433186671
0.573779091826261 51.8890569518899
0.688534910191513 51.6716685978973
0.803290728556765 51.4520089602028
0.918046546922017 51.230687120449
1.03280236528727 51.0077026081442
1.14755818365252 50.7820845394134
1.26231400201777 50.5524831966717
1.37706982038303 50.3186123187593
1.49182563874828 50.0805272799245
1.60658145711353 49.8382797450375
1.72133727547878 49.5919272436192
1.83609309384403 49.3415428925837
1.95084891220929 49.0871414903836
2.06560473057454 48.8287567217647
2.18036054893979 48.5665325387303
2.29511636730504 48.3008799568051
2.4098721856703 48.03233858828
2.52462800403555 47.7610896964938
2.6393838224008 47.4870682873472
2.75413964076605 47.2100964914573
2.8688954591313 46.9299504317928
2.98365127749656 46.6463820360511
3.09840709586181 46.3591108998766
3.21316291422706 46.0678059080445
3.32791873259231 45.7720634503222
3.44267455095756 45.471368173966
3.55743036932282 45.1651750715915
3.67218618768807 44.8530761301673
3.78694200605332 44.5346502737956
3.90169782441857 44.2095640018041
4.01645364278382 43.87749310708
4.13120946114908 43.5380788632526
4.24596527951433 43.1910576760389
4.36072109787958 42.8362884295477
4.47547691624483 42.4736553036551
4.59023273461009 42.1030389343218
4.70498855297534 41.724420155075
4.81974437134059 41.3384652061737
4.93450018970584 40.9465159755554
5.04925600807109 40.5502324290578
5.16401182643635 40.1517185263771
5.2787676448016 39.7531964451361
5.39352346316685 39.3571397941399
5.5082792815321 38.9663635210035
5.62303509989736 38.5837772291189
};
\addplot [semithick, white!20!black]
table {%
0 81.5171729285774
0.0711211241902549 81.4118162424626
0.14224224838051 81.2969538435709
0.213363372570765 81.1739074265509
0.28448449676102 81.0440207555648
0.355605620951275 80.9086372767664
0.42672674514153 80.7690644272659
0.497847869331785 80.6264494700466
0.56896899352204 80.4815894075529
0.640090117712295 80.3346899649014
0.711211241902549 80.1853289403993
0.782332366092804 80.0329635473012
0.853453490283059 79.8777121571924
0.924574614473314 79.7198446850164
0.995695738663569 79.5595800952374
1.06681686285382 79.3970755310387
1.13793798704408 79.2324713984529
1.20905911123433 79.0658593642847
1.28018023542459 78.8973043404869
1.35130135961484 78.726890202819
1.4224224838051 78.5548832728123
1.49354360799535 78.3815319447607
1.56466473218561 78.2068826088964
1.63578585637586 78.0309494842052
1.70690698056612 77.8537436935695
1.77802810475637 77.6752602111042
1.84914922894663 77.4954800372888
1.92027035313688 77.3143783332243
1.99139147732714 77.1319475280125
2.06251260151739 76.948206457919
2.13363372570765 76.7632266416085
2.2047548498979 76.5771608636503
2.27587597408816 76.3902463426268
2.34699709827841 76.2026746666163
2.41811822246867 76.0146031771515
2.48923934665892 75.8261625807522
2.56036047084918 75.637490951518
2.63148159503943 75.4487728903715
2.70260271922969 75.2602516661899
2.77372384341994 75.0721753547401
2.8448449676102 74.8844666965402
2.91596609180045 74.6965387127264
2.98708721599071 74.5078682030007
3.05820834018096 74.3181981156693
3.12932946437122 74.127404392166
3.20045058856147 73.9354237714229
3.27157171275173 73.7422136783585
3.34269283694198 73.5476920542853
3.41381396113224 73.3517520548786
3.48493508532249 73.1542982042839
};
\addplot [semithick, white!20!black]
table {%
0 92.1484727299508
0.0518771026398279 92.0789195578936
0.103754205279656 91.9990814917384
0.155631307919484 91.9104193288928
0.207508410559312 91.8143831722022
0.259385513199139 91.7123860603116
0.311262615838967 91.6058012430399
0.363139718478795 91.4958276851204
0.415016821118623 91.3833066507744
0.466893923758451 91.2685131659633
0.518771026398279 91.151264645029
0.570648129038107 91.0313780307811
0.622525231677935 90.9091473285998
0.674402334317763 90.7849314656133
0.726279436957591 90.6590204897627
0.778156539597418 90.5316171488477
0.830033642237246 90.4028936415855
0.881910744877074 90.2729737371887
0.933787847516902 90.1419366191243
0.98566495015673 90.0098451141847
1.03754205279656 89.8769045012302
1.08941915543639 89.7432370886809
1.14129625807621 89.6088293235281
1.19317336071604 89.4737275899613
1.24505046335587 89.3380228094411
1.2969275659957 89.2018027059273
1.34880466863553 89.0651437871789
1.40068177127535 88.9281246757085
1.45255887391518 88.7908700904714
1.50443597655501 88.65357070573
1.55631307919484 88.5165374455287
1.60819018183466 88.3802170500946
1.66006728447449 88.2451257049523
1.71194438711432 88.1117218022613
1.76382148975415 87.980377620824
1.81569859239398 87.8514235089662
1.8675756950338 87.7252136794046
1.91945279767363 87.6021280970598
1.97132990031346 87.4825788555165
2.02320700295329 87.3669812522385
2.07508410559312 87.2553053556809
2.12696120823294 87.1467630793221
2.17883831087277 87.0403643581253
2.2307154135126 86.935213606968
2.28259251615243 86.83046156484
2.33446961879226 86.7251534478608
2.38634672143208 86.6183127837008
2.43822382407191 86.5087967886843
2.49010092671174 86.3952824431064
2.54197802935157 86.2764238646111
};
\addplot [semithick, white!20!black]
table {%
0 89.3536692221506
0.0297093305523389 89.3103798797675
0.0594186611046778 89.2576015482047
0.0891279916570167 89.1965979975187
0.118837322209356 89.1285757043655
0.148546652761694 89.0546812256407
0.178255983314033 88.9760213029977
0.207965313866372 88.8935227654445
0.237674644418711 88.8077854717142
0.26738397497105 88.7190004495203
0.297093305523389 88.6272432125206
0.326802636075728 88.5327742557191
0.356511966628067 88.4360685527219
0.386221297180406 88.3375451209166
0.415930627732744 88.2375554481347
0.445639958285083 88.1363674787629
0.475349288837422 88.0342060597529
0.505058619389761 87.9312183221653
0.5347679499421 87.8275036714517
0.564477280494439 87.7231419247728
0.594186611046778 87.6182890261744
0.623895941599117 87.5129398260815
0.653605272151456 87.4070082324753
0.683314602703795 87.3005567093429
0.713023933256133 87.1937412066655
0.742733263808472 87.0867277507801
0.772442594360811 86.9796610733112
0.80215192491315 86.8726689419563
0.831861255465489 86.7659497066535
0.861570586017828 86.6597824103507
0.891279916570167 86.5546067119137
0.920989247122506 86.4511035646132
0.950698577674845 86.3500681269771
0.980407908227183 86.252240413566
1.01011723877952 86.1582286599897
1.03982656933186 86.0685857162327
1.0695358998842 85.9838923353238
1.09924523043654 85.9047361006929
1.12895456098888 85.831718288668
1.15866389154122 85.7654978146244
1.18837322209356 85.706317019765
1.21808255264589 85.6535295356081
1.24779188319823 85.6060444609301
1.27750121375057 85.5627352468666
1.30721054430291 85.5223836561469
1.33691987485525 85.4834946244564
1.36662920540759 85.4445074299637
1.39633853595993 85.4035600971619
1.42604786651227 85.358476927713
1.45575719706461 85.3070388636223
};
\addplot [semithick, white!20!black]
table {%
0 89.6550910226402
0.0632277568797961 89.5681816344316
0.126455513759592 89.4710216626707
0.189683270639388 89.3650795760646
0.252911027519184 89.2518346981038
0.316138784398981 89.1327460717133
0.379366541278777 89.0092340936363
0.442594298158573 88.8825505297856
0.505822055038369 88.7535839645078
0.569049811918165 88.6226096895249
0.632277568797961 88.4893082876159
0.695505325677757 88.3532771981673
0.758733082557553 88.2147121943432
0.82196083943735 88.0739303605787
0.885188596317146 87.9311849516509
0.948416353196942 87.786648636442
1.01164411007674 87.6404708371199
1.07487186695653 87.4927598078048
1.13809962383633 87.3435853644278
1.20132738071613 87.1930123933212
1.26455513759592 87.0412769088671
1.32778289447572 86.8885711223643
1.39101065135551 86.7349177130474
1.45423840823531 86.5803493598458
1.51746616511511 86.4249166705463
1.5806939219949 86.2686598295466
1.6439216788747 86.1116095072661
1.7071494357545 85.9538008894051
1.77037719263429 85.79530044475
1.83360494951409 85.636225010232
1.89683270639388 85.4767812283068
1.96006046327368 85.3172686743586
2.02328822015348 85.1580496335668
2.08651597703327 84.9994315879519
2.14974373391307 84.84166296247
2.21297149079286 84.6849582275624
2.27619924767266 84.5295497457533
2.33942700455246 84.3757066273009
2.40265476143225 84.2237427171674
2.46588251831205 84.0739618759445
2.52911027519184 83.9262515732216
2.59233803207164 83.7798485762418
2.65556578895144 83.6339284582006
2.71879354583123 83.4878455453514
2.78202130271103 83.3410611947446
2.84524905959083 83.1930242467501
2.90847681647062 83.0431856306173
2.97170457335042 82.8909019695989
3.03493233023021 82.7354298932586
3.09816008711001 82.5760186039506
};
\addplot [semithick, white!20!black]
table {%
0 101.811363480377
0.0836300373091731 101.709826524123
0.167260074618346 101.596227420622
0.250890111927519 101.472443616321
0.334520149236692 101.34041194351
0.418150186545865 101.202057581052
0.501780223855038 101.059264328455
0.585410261164211 100.913745762254
0.669040298473384 100.766798953006
0.752670335782557 100.618880889064
0.836300373091731 100.46944602886
0.919930410400904 100.317664999258
1.00356044771008 100.163580621535
1.08719048501925 100.007481721134
1.17082052232842 99.849578032264
1.2544505596376 99.6899741508358
1.33808059694677 99.5287608327952
1.42171063425594 99.3660335080743
1.50534067156511 99.2018409711992
1.58897070887429 99.0362061956267
1.67260074618346 98.8693972320136
1.75623078349263 98.7017122668779
1.83986082080181 98.5332429641473
1.92349085811098 98.3640205667351
2.00712089542015 98.1940529692148
2.09075093272933 98.0233264743023
2.1743809700385 97.85183559234
2.25801100734767 97.6796149029926
2.34164104465685 97.5067132911728
2.42527108196602 97.3332360832816
2.50890111927519 97.1593651331072
2.59253115658436 96.9852509801762
2.67616119389354 96.8110261568604
2.75979123120271 96.6367553384253
2.84342126851188 96.4624771993708
2.92705130582106 96.2882061899056
3.01068134313023 96.1139804704358
3.0943113804394 95.9398891722152
3.17794141774858 95.7660753960884
3.26157145505775 95.5925862970199
3.34520149236692 95.4189200072469
3.4288315296761 95.2439449517974
3.51246156698527 95.0666510866707
3.59609160429444 94.8862837968081
3.67972164160361 94.7023468437883
3.76335167891279 94.5144762082073
3.84698171622196 94.3223473407307
3.93061175353113 94.1256584738701
4.01424179084031 93.9241016585409
4.09787182814948 93.7173683180811
};
\addplot [semithick, white!20!black]
table {%
0 66.2937382824831
0.158921216419741 66.0581737463454
0.317842432839482 65.8129113996386
0.476763649259223 65.5594287929656
0.635684865678964 65.299394628646
0.794606082098705 65.034601855666
0.953527298518446 64.7668150834686
1.11244851493819 64.4976796134136
1.27136973135793 64.2284374717868
1.43029094777767 63.9593406983356
1.58921216419741 63.688913507318
1.74813338061715 63.4149073191256
1.90705459703689 63.1366864227932
2.06597581345663 62.85420853665
2.22489702987637 62.567413543883
2.38381824629612 62.2762224988406
2.54273946271586 61.9805952960588
2.7016606791356 61.6805072682396
2.86058189555534 61.375949845371
3.01950311197508 61.0670116938036
3.17842432839482 60.7541912134509
3.33734554481456 60.4382727100333
3.4962667612343 60.1195837815191
3.65518797765404 59.7980382991075
3.81410919407378 59.4733428150434
3.97303041049353 59.1451322832221
4.13195162691327 58.8130444318767
4.29087284333301 58.4767373480498
4.44979405975275 58.1357741582791
4.60871527617249 57.789632151649
4.76763649259223 57.437615927675
4.92655770901197 57.0787681198726
5.08547892543171 56.7121469384151
5.24440014185145 56.3367835921262
5.40332135827119 55.951880526193
5.56224257469094 55.5566741218395
5.72116379111068 55.1503659733432
5.88008500753042 54.7322879016873
6.03900622395016 54.3019246281163
6.1979274403699 53.8586484616464
6.35684865678964 53.4016843929816
6.51576987320938 52.9305368228289
6.67469108962912 52.4458457565419
6.83361230604886 51.9491509528686
6.99253352246861 51.4426050581591
7.15145473888835 50.9291354617604
7.31037595530809 50.4118728540199
7.46929717172783 49.8944624072932
7.62821838814757 49.3811328884827
7.78713960456731 48.8762387145393
};
\addplot [semithick, white!20!black]
table {%
0 98.1860779362647
0.124697179144374 98.027119097456
0.249394358288749 97.8556147394194
0.374091537433123 97.6735982274118
0.498788716577498 97.4832434562625
0.623485895721872 97.2867664054125
0.748183074866246 97.0863442565996
0.872880254010621 96.8840006135277
0.997577433154995 96.6813086446817
1.12227461229937 96.4787850710615
1.24697179144374 96.2753957179406
1.37166897058812 96.0695065756935
1.49636614973249 95.8608127162404
1.62106332887687 95.6494668031952
1.74576050802124 95.4355527853324
1.87045768716562 95.21906188901
1.99515486630999 94.9999967148094
2.11985204545437 94.7784014436475
2.24454922459874 94.5542896211007
2.36924640374311 94.3276768787899
2.49394358288749 94.0989349427294
2.61864076203186 93.8686085311503
2.74333794117624 93.6369216492311
2.86803512032061 93.4038634772059
2.99273229946499 93.1693057009129
3.11742947860936 92.9330728609252
3.24212665775373 92.6950086191664
3.36682383689811 92.4550155232787
3.49152101604248 92.2129599599092
3.61621819518686 91.9687194180975
3.74091537433123 91.7221512063816
3.86561255347561 91.4729108690612
3.99030973261998 91.2205893229793
4.11500691176436 90.9647144699495
4.23970409090873 90.7048810503191
4.3644012700531 90.4406869436689
4.48909844919748 90.1717375000427
4.61379562834185 89.8977270256668
4.73849280748623 89.618445052036
4.8631899866306 89.3335172588051
4.98788716577498 89.0420648156119
5.11258434491935 88.7429121243125
5.23728152406373 88.4354916735572
5.3619787032081 88.1197720259091
5.48667588235247 87.7962134549742
5.61137306149685 87.465737676577
5.73607024064122 87.1293901665006
5.8607674197856 86.7884956992513
5.98546459892997 86.4446468362257
6.11016177807435 86.0994845646855
};
\addplot [semithick, white!20!black]
table {%
0 101.662169023422
0.0446095763706294 101.611613198126
0.0892191527412589 101.549859125761
0.133828729111888 101.47855074388
0.178438305482518 101.399312159971
0.223047881853147 101.313709421002
0.267657458223777 101.223265775593
0.312267034594406 101.129320973157
0.356876610965036 101.032839885463
0.401486187335665 100.934183601906
0.446095763706294 100.833266579855
0.490705340076924 100.730030950916
0.535314916447553 100.624844853736
0.579924492818183 100.518116348158
0.624534069188812 100.410169800504
0.669143645559442 100.301220551806
0.713753221930071 100.191446948837
0.758362798300701 100.080989862718
0.80297237467133 99.9699324838843
0.847581951041959 99.8583146145121
0.892191527412589 99.7463108100409
0.936801103783218 99.6339892609288
0.981410680153848 99.5213150518798
1.02602025652448 99.4083543813092
1.07062983289511 99.2952381385085
1.11523940926574 99.1820995077407
1.15984898563637 99.0690663441093
1.20445856200699 98.9562819457475
1.24906813837762 98.8439497831003
1.29367771474825 98.7323654840645
1.33828729111888 98.6219841402454
1.38289686748951 98.5134016721247
1.42750644386014 98.4072551096742
1.47211602023077 98.3041130055933
1.5167255966014 98.2044330638319
1.56133517297203 98.1086237591893
1.60594474934266 98.0171291315939
1.65055432571329 97.9304092963998
1.69516390208392 97.8489420385074
1.73977347845455 97.7731871659643
1.78438305482518 97.7030530145312
1.82899263119581 97.6375390496656
1.87360220756644 97.5753191516219
1.91821178393707 97.5150721282909
1.9628213603077 97.4555043372713
2.00743093667832 97.3951463863606
2.05204051304895 97.3324894215397
2.09665008941958 97.2658063030065
2.14125966579021 97.1931142854297
2.18586924216084 97.1123879112394
};
\addplot [semithick, white!20!black]
table {%
0 57.9784447108587
0.146768036781423 57.7505238396514
0.293536073562846 57.5141051933036
0.440304110344268 57.2703971919496
0.587072147125691 57.0207698661238
0.733840183907114 56.7667137446031
0.880608220688537 56.5096929537414
1.02737625746996 56.2510540190878
1.17414429425138 55.9917747616584
1.3209123310328 55.731987289922
1.46768036781423 55.4703492355688
1.61444840459565 55.2048683445071
1.76121644137707 54.9349984912278
1.9079844781585 54.6607111963388
2.05475251493992 54.3819708074273
2.20152055172134 54.098739832641
2.34828858850276 53.8110142570031
2.49505662528419 53.5187760798657
2.64182466206561 53.222029523275
2.78859269884703 52.9208911648265
2.93536073562846 52.6158418194408
3.08212877240988 52.3076039357198
3.2288968091913 51.9964637698881
3.37566484597272 51.6823345509526
3.52243288275415 51.3649463540116
3.66920091953557 51.0439641911459
3.81596895631699 50.7190443459032
3.96273699309841 50.3898402652352
4.10950502987984 50.0559199260416
4.25627306666126 49.7167596492814
4.40304110344268 49.3716683003927
4.5498091402241 49.019769834928
4.69657717700553 48.6602562955373
4.84334521378695 48.2923016891374
4.99011325056837 47.9152325736416
5.1368812873498 47.5284039442799
5.28364932413122 47.1311314750249
5.43041736091264 46.7228529518575
5.57718539769406 46.3031544265921
5.72395343447549 45.8715647631366
5.87072147125691 45.4275564211665
6.01748950803833 44.9708798770061
6.16425754481976 44.502316906541
6.31102558160118 44.0235081937831
6.4577936183826 43.5366144309593
6.60456165516402 43.044486286033
6.75132969194545 42.5501558066868
6.89809772872687 42.057100572534
7.04486576550829 41.5693291308587
7.19163380228972 41.0909729807124
};
\addplot [semithick, white!20!black]
table {%
0 58.6498383386683
0.030986522409538 58.5742981742244
0.061973044819076 58.4926981672289
0.092959567228614 58.4055803321063
0.123946089638152 58.3134139002567
0.15493261204769 58.2166494598269
0.185919134457228 58.1157066655594
0.216905656866766 58.01084739101
0.247892179276304 57.9020854138735
0.278878701685842 57.789280425669
0.30986522409538 57.6724562943858
0.340851746504918 57.5519078143764
0.371838268914456 57.4280557231851
0.402824791323994 57.3012284776094
0.433811313733532 57.1717320758487
0.46479783614307 57.0398566910359
0.495784358552608 56.9058570207412
0.526770880962146 56.7698509060703
0.557757403371684 56.6319443484684
0.588743925781222 56.4923008053385
0.61973044819076 56.3511221184794
0.650716970600298 56.2084469749284
0.681703493009836 56.064186303422
0.712690015419374 55.9183586063131
0.743676537828912 55.7710609006169
0.77466306023845 55.6223961391818
0.805649582647988 55.4724193617364
0.836636105057526 55.3211117920564
0.867622627467064 55.168505926685
0.898609149876602 55.0146508653427
0.92959567228614 54.8596771159831
0.960582194695678 54.70403790153
0.991568717105216 54.5484253487701
1.02255523951475 54.3935104746115
1.05354176192429 54.2398640430343
1.08452828433383 54.0880107524831
1.11551480674337 53.9384722644943
1.14650132915291 53.7917884398981
1.17748785156244 53.648538796998
1.20847437397198 53.5094728472097
1.23946089638152 53.3752640774657
1.27044741879106 53.2460241316762
1.3014339412006 53.1215645609558
1.33242046361013 53.0018075283549
1.36340698601967 52.8865131230741
1.39439350842921 52.7752161024549
1.42538003083875 52.667398038124
1.45636655324829 52.5622699115674
1.48735307565782 52.4588302051119
1.51833959806736 52.3560740767722
};
\addplot [semithick, white!20!black]
table {%
0 89.446026736806
0.0873844337738527 89.327255909074
0.174768867547705 89.1977336138927
0.262153301321558 89.0590657204478
0.349537735095411 88.9129178159753
0.436922168869264 88.760964326966
0.524306602643116 88.6048419989211
0.611691036416969 88.4460273784949
0.699075470190822 88.2856088472046
0.786459903964675 88.123917782525
0.873844337738527 87.9603496332424
0.96122877151238 87.7940248023046
1.04861320528623 87.62493723847
1.13599763906009 87.4533293602701
1.22338207283394 87.2793830466742
1.31076650660779 87.1032026605586
1.39815094038164 86.924883719754
1.4855353741555 86.7445060738754
1.57291980792935 86.5621183209273
1.6603042417032 86.3777757553587
1.74768867547705 86.1917726927869
1.83507310925091 86.0044440862163
1.92245754302476 85.8158909072792
2.00984197679861 85.6261237744682
2.09722641057247 85.4351166133954
2.18461084434632 85.2428181055251
2.27199527812017 85.0491755330313
2.35937971189402 84.8541553729913
2.44676414566788 84.6577268155163
2.53414857944173 84.4598866903132
2.62153301321558 84.2606695377979
2.70891744698944 84.060097115931
2.79630188076329 83.8582185376989
2.88368631453714 83.6550288307278
2.97107074831099 83.4505167752203
3.05845518208485 83.2446527602511
3.1458396158587 83.0374174279723
3.23322404963255 82.8288498716114
3.3206084834064 82.6190566348271
3.40799291718026 82.4080871359439
3.49537735095411 82.1955792606938
3.58276178472796 81.9806961346946
3.67014621850182 81.7628177630642
3.75753065227567 81.5416593387775
3.84491508604952 81.317185220249
3.93229951982337 81.0895383121964
4.01968395359723 80.8589125702065
4.10706838737108 80.6255571068703
4.19445282114493 80.3897763672001
4.28183725491878 80.151893802105
};
\addplot [semithick, white!20!black]
table {%
0 108.105704149203
0.13850604024521 107.938529875135
0.277012080490421 107.757393474795
0.415518120735631 107.564645307971
0.554024160980841 107.362810107529
0.692530201226051 107.15445833959
0.831036241471262 106.942119335896
0.969542281716472 106.728166577145
1.10804832196168 106.514482596618
1.24655436220689 106.301725520914
1.3850604024521 106.088710143619
1.52356644269731 105.87351029487
1.66207248294252 105.655719925762
1.80057852318773 105.435477347834
1.93908456343294 105.212839353409
2.07759060367815 104.987749670652
2.21609664392336 104.760169390543
2.35460268416857 104.530135546005
2.49310872441378 104.297647029276
2.631614764659 104.062686465585
2.77012080490421 103.825644903891
2.90862684514942 103.587136789129
3.04713288539463 103.347433209747
3.18563892563984 103.106524733082
3.32414496588505 102.864257160545
3.46265100613026 102.620421774181
3.60115704637547 102.374842462848
3.73966308662068 102.127429241825
3.87816912686589 101.878045455445
4.0166751671111 101.62657312769
4.15518120735631 101.372869329412
4.29368724760152 101.116500476768
4.43219328784673 100.856906749949
4.57069932809194 100.593454604824
4.70920536833715 100.325598125345
4.84771140858236 100.052800651994
4.98621744882757 99.7745386168531
5.12472348907278 99.4903864713654
5.26322952931799 99.200018777619
5.4017355695632 98.9028817338117
5.54024160980841 98.5978085008282
5.67874765005362 98.2833321765146
5.81725369029883 97.958710467503
5.95575973054404 97.6237814443516
6.09426577078925 97.2789816181247
6.23277181103446 96.9253046216977
6.37127785127967 96.5638926037936
6.50978389152488 96.1962450056513
6.64828993177009 95.8241874230246
6.7867959720153 95.4495963498574
};
\addplot [semithick, white!20!black]
table {%
0 36.4869577168039
0.022426070627996 36.4005630442778
0.0448521412559921 36.3107896372616
0.0672782118839881 36.2176026429977
0.0897042825119842 36.1208640159486
0.11213035313998 36.020435537527
0.134556423767976 35.9161541116383
0.156982494395972 35.8077115842299
0.179408565023968 35.694618660512
0.201834635651964 35.5764729547878
0.22426070627996 35.4533728833306
0.246686776907956 35.3258254118211
0.269112847535953 35.1942912355575
0.291538918163949 35.0590625196497
0.313964988791945 34.9204403326463
0.336391059419941 34.7787577746631
0.358817130047937 34.6343123371028
0.381243200675933 34.4872117942565
0.403669271303929 34.3375752581764
0.426095341931925 34.1856305502719
0.448521412559921 34.0315898717656
0.470947483187917 33.8754676179597
0.493373553815913 33.7171418154736
0.515799624443909 33.5566077790182
0.538225695071905 33.3939499399302
0.560651765699901 33.229261485341
0.583077836327897 33.0625652177272
0.605503906955893 32.8937631698743
0.627929977583889 32.7228058764006
0.650356048211885 32.5496229669012
0.672782118839881 32.3741885495596
0.695208189467877 32.1968999488971
0.717634260095873 32.0184976206562
0.740060330723869 31.8397252865907
0.762486401351866 31.6612285190321
0.784912471979861 31.4836074658078
0.807338542607858 31.3074398214552
0.829764613235854 31.1333211941766
0.85219068386385 30.9618967410735
0.874616754491846 30.79408152425
0.897042825119842 30.6309594023437
0.919468895747838 30.4732201178727
0.941894966375834 30.3212487743576
0.96432103700383 30.1755856241567
0.986747107631826 30.0365020383589
1.00917317825982 29.9040063384365
1.03159924888782 29.7780434697248
1.05402531951581 29.6582513124672
1.07645139014381 29.5440675914209
1.09887746077181 29.4349453330216
};
\addplot [semithick, white!20!black]
table {%
0 98.0540727283909
0.161248892444419 97.8470875939705
0.322497784888838 97.6267782942834
0.483746677333257 97.3953904749343
0.644995569777676 97.1553843551484
0.806244462222095 96.9093060616095
0.967493354666514 96.6596643307058
1.12874224711093 96.4088269490727
1.28999113955535 96.1586729836018
1.45124003199977 95.9098060269356
1.61248892444419 95.6607606848362
1.77373781688861 95.4091807815298
1.93498670933303 95.1544562680182
2.09623560177745 94.8966273503507
2.25748449422187 94.6356702283755
2.41873338666629 94.3714726187499
2.5799822791107 94.1039553624645
2.74123117155512 93.8331198281787
2.90248006399954 93.5589474100577
3.06372895644396 93.2814387207808
3.22497784888838 93.0010534454175
3.3862267413328 92.7185520681785
3.54747563377722 92.4342771020176
3.70872452622164 92.1481846043141
3.86997341866606 91.8600305619141
4.03122231111047 91.5695014111188
4.19247120355489 91.2763151504362
4.35372009599931 90.9802712220767
4.51496898844373 90.6810897845151
4.67621788088815 90.378468085508
4.83746677333257 90.0720048262727
4.99871566577699 89.7609364622109
5.15996455822141 89.4443805654753
5.32121345066583 89.1213925751277
5.48246234311025 88.7911745029987
5.64371123555467 88.4529549982405
5.80496012799908 88.1059588052016
5.9662090204435 87.7495324025728
6.12745791288792 87.3831517169884
6.28870680533234 87.0060568040359
6.44995569777676 86.6169885168935
6.61120459022118 86.2146552537352
6.7724534826656 85.7987934762722
6.93370237511002 85.3699115944914
7.09495126755444 84.9292252626415
7.25620015999886 84.4787004279354
7.41744905244327 84.0205008230727
7.57869794488769 83.5572954967646
7.73994683733211 83.0922557515282
7.90119572977653 82.6286410799003
};
\addplot [semithick, white!20!black]
table {%
0 53.3370716654262
0.177094194537121 53.0647912540893
0.354188389074242 52.7838772726743
0.531282583611363 52.4956066243674
0.708376778148484 52.2014765869561
0.885470972685605 51.9031471625647
1.06256516722273 51.6022544224759
1.23965936175985 51.3003305165072
1.41675355629697 50.9985189671131
1.59384775083409 50.696974597965
1.77094194537121 50.3939918676686
1.94803613990833 50.086988145032
2.12513033444545 49.7751577198722
2.30222452898257 49.4583660206395
2.47931872351969 49.1364818385526
2.65641291805682 48.8093856173755
2.83350711259394 48.4770103362966
3.01060130713106 48.1392983712927
3.18769550166818 47.7962284086372
3.3647896962053 47.4479169895696
3.54188389074242 47.0949242284829
3.71897808527954 46.7381569485438
3.89607227981666 46.3779986782042
4.07316647435378 46.0143288463531
4.2502606688909 45.6467734322128
4.42735486342803 45.2748742671593
4.60444905796515 44.8981707823326
4.78154325250227 44.5162097792283
4.95863744703939 44.1284141832442
5.13573164157651 43.7340777583399
5.31282583611363 43.3322505058687
5.48992003065075 42.9216772164247
5.66701422518787 42.5011444408553
5.84410841972499 42.0694264455996
6.02120261426211 41.6255204444669
6.19829680879924 41.1684728390669
6.37539100333636 40.6972767632294
6.55248519787348 40.2110761673274
6.7295793924106 39.709195317764
6.90667358694772 39.190858497589
7.08376778148484 38.6552882341544
7.26086197602196 38.102251255027
7.43795617055908 37.5329131889219
7.6150503650962 36.949514646796
7.79214455963332 36.3549835688893
7.96923875417045 35.7531836361578
8.14633294870757 35.1482227859043
8.32342714324469 34.5448453697115
8.50052133778181 33.948534972232
8.67761553231893 33.3649368262852
};
\addplot [semithick, white!20!black]
table {%
0 46.4902998354245
0.0779757903438168 46.3410787672428
0.155951580687634 46.1861467717877
0.23392737103145 46.0260339770053
0.311903161375267 45.861285855499
0.389878951719084 45.6925010961758
0.467854742062901 45.5202527135096
0.545830532406718 45.344980646438
0.623806322750534 45.1668588486774
0.701782113094351 44.9857280984199
0.779757903438168 44.8010436210644
0.857733693781985 44.6121950773029
0.935709484125802 44.4191939363138
1.01368527446962 44.2221902499139
1.09166106481344 44.0213352031241
1.16963685515725 43.8167959957905
1.24761264550107 43.6087349931419
1.32558843584489 43.3972041435532
1.4035642261887 43.1822711309243
1.48154001653252 42.9641129763337
1.55951580687634 42.7430613232411
1.63749159722015 42.519446414114
1.71546738756397 42.2933287452155
1.79344317790779 42.0646675200579
1.8714189682516 41.8333895803102
1.94939475859542 41.5993975888845
2.02737054893924 41.3625517231649
2.10534633928305 41.1226448322466
2.18332212962687 40.8794586483156
2.26129791997069 40.6327232143333
2.3392737103145 40.3821193718031
2.41724950065832 40.1274783587388
2.49522529100214 39.868850010682
2.57320108134595 39.6062774673415
2.65117687168977 39.3398173429367
2.72915266203359 39.069513713618
2.8071284523774 38.7953810664658
2.88510424272122 38.517498218905
2.96308003306504 38.2360373146417
3.04105582340885 37.9512927694514
3.11903161375267 37.66362272997
3.19700740409649 37.3732873421185
3.27498319444031 37.0808372434224
3.35295898478412 36.7872908169577
3.43093477512794 36.4937510209449
3.50891056547176 36.2014829854603
3.58688635581557 35.9117976795624
3.66486214615939 35.6260349314744
3.74283793650321 35.3456598245774
3.82081372684702 35.0722003304647
};
\addplot [semithick, white!20!black]
table {%
0 66.1405049375262
0.233122805184588 65.807559174973
0.466245610369175 65.4633219974241
0.699368415553763 65.1097046193396
0.932491220738351 64.7489597871427
1.16561402592294 64.383553307679
1.39873683110753 64.0159254723371
1.63185963629211 63.6484228193189
1.8649824414767 63.2829105926753
2.09810524666129 62.9198184351816
2.33122805184588 62.5567953032005
2.56435085703046 62.190126365503
2.79747366221505 61.81855666508
3.03059646739964 61.441815950246
3.26371927258423 61.0596255250711
3.49684207776882 60.6716962180483
3.7299648829534 60.2778218258923
3.96308768813799 59.8778908744198
4.19621049332258 59.4718294959846
4.42933329850717 59.0596955532444
4.66245610369175 58.6421658468085
4.89557890887634 58.2204625611777
5.12870171406093 57.7951538969389
5.36182451924552 57.3660866444244
5.59494732443011 56.9327326869868
5.82807012961469 56.4944468384734
6.06119293479928 56.0506120436717
6.29431573998387 55.6006776153187
6.52743854516846 55.1439104311232
6.76056135035304 54.6794227183306
6.99368415553763 54.2059952139239
7.22680696072222 53.7218205789305
7.45992976590681 53.2249964362207
7.69305257109139 52.7135950705963
7.92617537627598 52.1860217598142
8.15929818146057 51.6407633880553
8.39242098664516 51.0762490744192
8.62554379182975 50.4911046676082
8.85866659701433 49.884178297546
9.09178940219892 49.2540590504195
9.32491220738351 48.5991981966161
9.5580350125681 47.9188617959186
9.79115781775268 47.2143035990341
10.0242806229373 46.4881554295545
10.2574034281219 45.7440998318311
10.4905262333064 44.9871803182076
10.723649038491 44.2227939401654
10.9567718436756 43.4573109711581
11.1898946488602 42.6981609096399
11.4230174540448 41.9529794826546
};
\addplot [semithick, white!20!black]
table {%
0 109.883371246311
0.0468538699021428 109.838061970878
0.0937077398042857 109.780580120398
0.140561609706428 109.712778130473
0.187415479608571 109.636497998719
0.234269349510714 109.553515698766
0.281123219412857 109.465562124607
0.327977089315 109.374179632082
0.374830959217143 109.280511846835
0.421684829119285 109.185014824708
0.468538699021428 109.087586351157
0.515392568923571 108.988108257778
0.562246438825714 108.88694162933
0.609100308727857 108.784510843248
0.65595417863 108.681144834831
0.702808048532142 108.577045661331
0.749661918434285 108.472377881799
0.796515788336428 108.367287189191
0.843369658238571 108.26185273054
0.890223528140714 108.156090824368
0.937077398042857 108.050169947261
0.983931267944999 107.944161807873
1.03078513784714 107.838040675989
1.07763900774929 107.731882190478
1.12449287765143 107.625824850657
1.17134674755357 107.520008975176
1.21820061745571 107.414577569326
1.26505448735786 107.309705917683
1.31190835726 107.205631601607
1.35876222716214 107.102699132974
1.40561609706428 107.001428174234
1.45246996696643 106.902446034094
1.49932383686857 106.806383864208
1.54617770677071 106.713795278112
1.59303157667286 106.625120232212
1.639885446575 106.540748618293
1.68673931647714 106.46111338211
1.73359318637928 106.38666281689
1.78044705628143 106.317858347119
1.82730092618357 106.255108205932
1.87415479608571 106.198178241199
1.92100866598786 106.145856514476
1.96786253589 106.096596503747
2.01471640579214 106.048834100095
2.06157027569428 106.001066864672
2.10842414559643 105.951623246053
2.15527801549857 105.898794126347
2.20213188540071 105.840659876743
2.24898575530286 105.775034721186
2.295839625205 105.699682253412
};
\addplot [semithick, white!20!black]
table {%
0 101.795549733972
0.0762035191958744 101.703728224321
0.152407038391749 101.600007571571
0.228610557587623 101.486221078733
0.304814076783497 101.364246365405
0.381017595979372 101.2359405491
0.457221115175246 101.103119094287
0.533424634371121 100.967424706793
0.609628153566995 100.830091476451
0.685831672762869 100.691558276806
0.762035191958744 100.551367129975
0.838238711154618 100.408835474054
0.914442230350492 100.26406806378
0.990645749546367 100.11737645323
1.06684926874224 99.9689922109214
1.14305278793812 99.8190409995922
1.21925630713399 99.6676302323932
1.29545982632986 99.5148639991822
1.37166334552574 99.3607976376777
1.44786686472161 99.2054572897472
1.52407038391749 99.0490931941419
1.60027390311336 98.8919597488055
1.67647742230924 98.7341245295586
1.75268094150511 98.575625449717
1.82888446070098 98.4164938351589
1.90508797989686 98.2567439675097
1.98129149909273 98.0963957707378
2.05749501828861 97.9355045747899
2.13369853748448 97.7741487517454
2.20990205668036 97.6124699385449
2.28610557587623 97.4507021156025
2.36230909507211 97.2890806763212
2.43851261426798 97.1278342063648
2.51471613346385 96.9671233023229
2.59091965265973 96.8070663996596
2.6671231718556 96.6477529470475
2.74332669105148 96.4892983721482
2.81953021024735 96.3318624278557
2.89573372944323 96.1756519177453
2.9719372486391 96.0207924917875
3.04814076783497 95.8668601715247
3.12434428703085 95.7127480097648
3.20054780622672 95.5573854403049
3.2767513254226 95.3999095929874
3.35295484461847 95.239672074392
3.42915836381435 95.0760980793853
3.50536188301022 94.9086372462452
3.58156540220609 94.7367161048839
3.65776892140197 94.5597074942076
3.73397244059784 94.3769755999984
};
\addplot [semithick, white!20!black]
table {%
0 90.2827797768497
0.0998765383743438 90.1484734643037
0.199753076748688 90.003050282705
0.299629615123031 89.8482095975579
0.399506153497375 89.6857363173753
0.499382691871719 89.5174380253255
0.599259230246063 89.3450848508626
0.699135768620407 89.1702904174269
0.799012306994751 88.9942647948227
0.898888845369094 88.8173786554147
0.998765383743438 88.6388811228961
1.09864192211778 88.4576440787792
1.19851846049213 88.2735584438264
1.29839499886647 88.0868306910549
1.39827153724081 87.8976070731833
1.49814807561516 87.7059558346293
1.5980246139895 87.5119436110816
1.69790115236384 87.315636418827
1.79777769073819 87.1170716458865
1.89765422911253 86.9162970360267
1.99753076748688 86.7136357360383
2.09740730586122 86.5094953975555
2.19728384423556 86.3040176985815
2.29716038260991 86.0972031669668
2.39703692098425 85.8889877632109
2.4969134593586 85.6792746427717
2.59678999773294 85.4679705907986
2.69666653610728 85.2550109526562
2.79654307448163 85.040319560291
2.89641961285597 84.8238380676549
2.99629615123031 84.6055212532874
3.09617268960466 84.3852537662749
3.196049227979 84.1629254314259
3.29592576635335 83.938371312825
3.39580230472769 83.7114466519361
3.49567884310203 83.4819960857816
3.59555538147638 83.2498714941555
3.69543191985072 83.0149941166804
3.79530845822507 82.7773636280207
3.89518499659941 82.536894664067
3.99506153497375 82.2930823647464
4.0949380733481 82.0450283184982
4.19481461172244 81.7921907990678
4.29469115009678 81.5344399170084
4.39456768847113 81.2719708483105
4.49444422684547 81.0052548450877
4.59432076521982 80.7348392257111
4.69419730359416 80.4614030564276
4.7940738419685 80.1857581961317
4.89395038034285 79.9087479894749
};
\addplot [semithick, white!20!black]
table {%
0 45.3888809379262
0.109811637773021 45.196847500235
0.219623275546043 44.9985360109233
0.329434913319064 44.7946380598757
0.439246551092085 44.5859244917356
0.549058188865106 44.3732588134337
0.658869826638128 44.1574803671089
0.768681464411149 43.9393071153187
0.87849310218417 43.7191602677289
0.988304739957192 43.4969455218587
1.09811637773021 43.2717413437659
1.20792801550323 43.0423103141143
1.31773965327626 42.8083967474017
1.42755129104928 42.5700499633642
1.5373629288223 42.3273259590351
1.64717456659532 42.0803026097644
1.75698620436834 41.8290721207068
1.86679784214136 41.5736482584816
1.97660947991438 41.3140709517659
2.0864211176874 41.0505068448647
2.19623275546043 40.7833655679572
2.30604439323345 40.513166457903
2.41585603100647 40.240073000196
2.52566766877949 39.9640141735609
2.63547930655251 39.6848142842805
2.74529094432553 39.4022538374182
2.85510258209855 39.1160808189732
2.96491421987157 38.8259936821006
3.0747258576446 38.5316416319135
3.18453749541762 38.2325905383459
3.29434913319064 37.9282863573855
3.40416077096366 37.6181884615647
3.51397240873668 37.3019316329841
3.6237840465097 36.9791458186762
3.73359568428272 36.6495448106753
3.84340732205575 36.3128506024299
3.95321895982877 35.9687447092691
4.06303059760179 35.6170018545302
4.17284223537481 35.2575206910145
4.28265387314783 34.89026310732
4.39246551092085 34.5152702825086
4.50227714869387 34.1327258126342
4.6120887864669 33.7434737468011
4.72190042423992 33.349035756612
4.83171206201294 32.9512033104892
4.94152369978596 32.5521829625764
5.05133533755898 32.1542918984645
5.161146975332 31.7600737512015
5.27095861310502 31.3724046145421
5.38077025087804 30.9942593460684
};
\addplot [semithick, white!20!black]
table {%
0 69.0807596159498
0.0851644744664487 68.9446163420148
0.170328948932897 68.8000626725632
0.255493423399346 68.6482076742113
0.340657897865795 68.4902036380091
0.425822372332244 68.3272354092984
0.510986846798692 68.1604559286758
0.596151321265141 67.9908715654385
0.68131579573159 67.8191560810864
0.766480270198039 67.6454136739518
0.851644744664487 67.4690417081595
0.936809219130936 67.2892439560369
1.02197369359738 67.1060039799198
1.10713816806383 66.9195135364082
1.19230264253028 66.7299333583456
1.27746711699673 66.5373912474564
1.36263159146318 66.3420093932615
1.44779606592963 66.1438518003036
1.53296054039608 65.9429741422315
1.61812501486253 65.7394884811006
1.70328948932897 65.5337122103412
1.78845396379542 65.3259912668733
1.87361843826187 65.1164146995601
1.95878291272832 64.9049667319107
2.04394738719477 64.6915918890521
2.12911186166122 64.4762085844752
2.21427633612767 64.2587151232092
2.29944081059412 64.0389893826554
2.38460528506057 63.816902757698
2.46976975952701 63.5923146046336
2.55493423399346 63.3650760017096
2.64009870845991 63.1350927470263
2.72526318292636 62.9023853022661
2.81042765739281 62.6669425561799
2.89559213185926 62.4287613804841
2.98075660632571 62.1878244689207
3.06592108079215 61.9441051899186
3.1510855552586 61.697640156988
3.23625002972505 61.4485477931521
3.3214145041915 61.1969700030762
3.40657897865795 60.9428627164058
3.4917434531244 60.6859018429297
3.57690792759085 60.426040782295
3.6620724020573 60.1636453905103
3.74723687652374 59.8992658899437
3.83240135099019 59.6336409821831
3.91756582545664 59.3675624839935
4.00273029992309 59.1018787505273
4.08789477438954 58.837540968272
4.17305924885599 58.5755425615832
};
\addplot [semithick, white!20!black]
table {%
0 46.5488845099816
0.0683256490329029 46.4123666620622
0.136651298065806 46.2703407877367
0.204976947098709 46.1232815338227
0.273302596131612 45.9716593571506
0.341628245164515 45.8159863303097
0.409953894197418 45.6567484730532
0.47827954323032 45.4942953787943
0.546605192263223 45.3287207013398
0.614930841296126 45.1598425421081
0.683256490329029 44.9872300024966
0.751582139361932 44.8104633586967
0.819907788394835 44.6296346695786
0.888233437427738 44.4449237442351
0.956559086460641 44.2565102471024
1.02488473549354 44.0645886859912
1.09321038452645 43.8693429864699
1.16153603355935 43.6708364214098
1.22986168259225 43.4691451571455
1.29818733162516 43.2644501148891
1.36651298065806 43.0570596826256
1.43483862969096 42.8472471096559
1.50316427872386 42.6350416101096
1.57148992775677 42.4204111654676
1.63981557678967 42.2033132059164
1.70814122582257 41.9836869007746
1.77646687485548 41.7614256701956
1.84479252388838 41.5363496949953
1.91311817292128 41.3082794378246
1.98144382195418 41.0769927006549
2.04976947098709 40.8422388328001
2.11809512001999 40.6039598820869
2.18642076905289 40.3623307202545
2.2547464180858 40.1175192640782
2.3230720671187 39.8696858317198
2.3913977161516 39.6189719928925
2.45972336518451 39.3654927550739
2.52804901421741 39.1094187943097
2.59637466325031 38.8510050585346
2.66470031228321 38.5906476929796
2.73302596131612 38.3288049093221
2.80135161034902 38.0657668908548
2.86967725938192 37.802003332675
2.93800290841483 37.5383892947416
3.00632855744773 37.2758275687421
3.07465420648063 37.015306825751
3.14297985551353 36.7578420244524
3.21130550454644 36.5044172935072
3.27963115357934 36.2560800424097
3.34795680261224 36.0139300407203
};
\addplot [semithick, white!20!black]
table {%
0 110.562359276518
0.147728106631972 110.385547574618
0.295456213263945 110.194296953696
0.443184319895917 109.991070474244
0.59091242652789 109.778525717462
0.738640533159862 109.559373830732
0.886368639791835 109.336284465739
1.03409674642381 109.111772850015
1.18182485305578 108.887847098546
1.32955295968775 108.665214376115
1.47728106631972 108.442583645587
1.6250091729517 108.21784170457
1.77273727958367 107.990509055826
1.92046538621564 107.760702664137
2.06819349284761 107.528455520781
2.21592159947959 107.293683141941
2.36364970611156 107.056323337652
2.51137781274353 106.816404593545
2.6591059193755 106.573917063889
2.80683402600748 106.328832758124
2.95456213263945 106.081561444042
3.10229023927142 105.832769060732
3.25001834590339 105.58275717388
3.39774645253537 105.331511465643
3.54547455916734 105.078852998408
3.69320266579931 104.824542934631
3.84093077243128 104.568380376664
3.98865887906326 104.310260919806
4.13638698569523 104.050024052403
4.2841150923272 103.787524462694
4.43184319895917 103.522578316194
4.57957130559115 103.254663575216
4.72729941222312 102.983108009415
4.87502751885509 102.707163128693
5.02275562548706 102.426185857084
5.17048373211904 102.139547614427
5.31821183875101 101.846632519406
5.46593994538298 101.54693015084
5.61366805201495 101.24003682369
5.76139615864693 100.925293001491
5.9091242652789 100.601399843236
6.05685237191087 100.266799368987
6.20458047854284 99.9207536434995
6.35230858517482 99.5631534126312
6.50003669180679 99.1945485618658
6.64776479843876 98.8161156180313
6.79549290507073 98.4291974988126
6.9432210117027 98.0355495924872
7.09094911833468 97.6373050527174
7.23867722496665 97.2366548294247
};
\addplot [semithick, white!20!black]
table {%
0 104.171771157113
0.0822213066984657 104.074430782656
0.164442613396931 103.964793090505
0.246663920095397 103.844783289634
0.328885226793863 103.716384508047
0.411106533492328 103.581563482002
0.493327840190794 103.442244836497
0.57554914688926 103.300180822725
0.657770453587725 103.156702507343
0.739991760286191 103.012289184428
0.822213066984657 102.86641473829
0.904434373683123 102.718273061078
0.986655680381588 102.567922076805
1.06887698708005 102.415661591663
1.15109829377852 102.261708693806
1.23331960047699 102.106169985551
1.31554090717545 101.949136864521
1.39776221387392 101.79070854309
1.47998352057238 101.63093446267
1.56220482727085 101.46983171821
1.64442613396931 101.307661684771
1.72664744066778 101.144711448271
1.80886874736625 100.981068651532
1.89109005406471 100.816769098096
1.97331136076318 100.651829352977
2.05553266746164 100.48624551045
2.13775397416011 100.320023468523
2.21997528085857 100.153212747425
2.30219658755704 99.985880200481
2.3844178942555 99.8181552593438
2.46663920095397 99.6502527702615
2.54886050765244 99.4823557732891
2.6310818143509 99.3146216797484
2.71330312104937 99.1471374010495
2.79552442774783 98.9799585698191
2.8777457344463 98.8131150366933
2.95996704114476 98.6466631442112
3.04218834784323 98.4807081580882
3.1244096545417 98.3154060943418
3.20663096124016 98.1508109856293
3.28885226793863 97.9864015237351
3.37107357463709 97.8209921369801
3.45329488133556 97.6534926425364
3.53551618803402 97.4830486066869
3.61773749473249 97.3090616434946
3.69995880143096 97.1310512770387
3.78218010812942 96.9485728900519
3.86440141482789 96.7611941895696
3.94662272152635 96.5684605269367
4.02884402822482 96.369912128371
};
\addplot [semithick, white!20!black]
table {%
0 57.2886679409007
0.34026776751729 56.8065114855663
0.68053553503458 56.3117341924555
1.02080330255187 55.8066684793298
1.36107107006916 55.2942003586573
1.70133883758645 54.7775683215528
2.04160660510374 54.2599919047283
2.38187437262103 53.7446398900829
2.72214214013832 53.2341096583663
3.06240990765561 52.7289933030141
3.4026776751729 52.2256656498484
3.74294544269019 51.7183113792236
4.08321321020748 51.2047691103884
4.42348097772477 50.6844150546208
4.76374874524206 50.1566431638345
5.10401651275935 49.6208679724819
5.44428428027664 49.0766526319538
5.78455204779393 48.5237525075978
6.12481981531122 47.9620016140487
6.46508758282851 47.3914369959101
6.8053553503458 46.8130050986031
7.14562311786309 46.2285705953874
7.48589088538037 45.6390471084082
7.82615865289766 45.0441725314519
8.16642642041495 44.4430644612682
8.50669418793224 43.8346568221161
8.84696195544953 43.2179406778679
9.18722972296682 42.5920236916293
9.52749749048411 41.9556997758534
9.8677652580014 41.307491144859
10.2080330255187 40.6453377209638
10.548300793036 39.9661449236275
10.8885685605533 39.266598999441
11.2288363280706 38.5433728181099
11.5691040955879 37.7937138939216
11.9093718631051 37.0150225716134
12.2496396306224 36.2045996645676
12.5899073981397 35.3600416182242
12.930175165657 34.4792743066815
13.2704429331743 33.5597851225568
13.6107107006916 32.5990330517406
13.9509784682089 31.5961542386382
14.2912462357262 30.5535393879742
14.6315140032435 29.4756867583132
14.9717817707607 28.3687556425251
15.312049538278 27.2411244330444
15.6523173057953 26.1017447800564
15.9925850733126 24.9612103633343
16.3328528408299 23.8318870384506
16.6731206083472 22.7264741073163
};
\addplot [semithick, white!20!black]
table {%
0 107.115322701629
0.117357872813485 106.974873222622
0.234715745626969 106.821032231753
0.352073618440454 106.656001929396
0.469431491253938 106.482115420289
0.586789364067423 106.301727568411
0.704147236880907 106.117151595979
0.821505109694392 105.930538321379
0.938862982507876 105.743572553584
1.05622085532136 105.556850537839
1.17357872813485 105.369435292591
1.29093660094833 105.179820549096
1.40829447376181 104.987775281795
1.5256523465753 104.793499857848
1.64301021938878 104.5971119613
1.76036809220227 104.39861610728
1.87772596501575 104.198021827944
1.99508383782924 103.995389960817
2.11244171064272 103.790738268608
2.22979958345621 103.584060990861
2.34715745626969 103.375699773086
2.46451532908318 103.166145483165
2.58187320189666 102.955600395653
2.69923107471014 102.744072754787
2.81658894752363 102.531473395982
2.93394682033711 102.317671481908
3.0513046931506 102.102560643933
3.16866256596408 101.886105585275
3.28602043877757 101.668248669635
3.40337831159105 101.448968433999
3.52073618440454 101.228261146203
3.63809405721802 101.005928295266
3.75545193003151 100.781680924664
3.87280980284499 100.555157024845
3.99016767565847 100.32603708021
4.10752554847196 100.093997531451
4.22488342128544 99.8587334310726
4.34224129409893 99.6200192413384
4.45959916691241 99.3777105952125
4.5769570397259 99.1314804223928
4.69431491253938 98.8803973715857
4.81167278535287 98.6230881390724
4.92903065816635 98.3586655480559
5.04638853097984 98.0866911089092
5.16374640379332 97.8071973613473
5.28110427660681 97.520608281692
5.39846214942029 97.2274538757639
5.51582002223377 96.9284914634292
5.63317789504726 96.6246720934076
5.75053576786074 96.3169760486746
};
\addplot [semithick, white!20!black]
table {%
0 58.786824457171
0.0240869446253497 58.7204614432712
0.0481738892506995 58.6481729481349
0.0722608338760492 58.5704635775763
0.0963477785013989 58.4877512436654
0.120434723126749 58.4004267980359
0.144521667752098 58.3088498610079
0.168608612377448 58.2132198087932
0.192695557002798 58.1134948555926
0.216782501628148 58.0095195154874
0.240869446253497 57.9013991878921
0.264956390878847 57.789564767422
0.289043335504197 57.674494745954
0.313130280129546 57.5565391242601
0.337217224754896 57.4360243917696
0.361304169380246 57.3132601659725
0.385391114005596 57.1885164581506
0.409478058630945 57.0619192922035
0.433565003256295 56.9335807103643
0.457651947881645 56.8036666977952
0.481738892506995 56.6723623366287
0.505825837132344 56.53966545296
0.529912781757694 56.4054646317428
0.553999726383044 56.2697847839278
0.578086671008394 56.1327449664921
0.602173615633743 55.994474413771
0.626260560259093 55.8550521973758
0.650347504884443 55.714479525291
0.674434449509792 55.5728170831958
0.698521394135142 55.4301488099569
0.722608338760492 55.2866551243644
0.746695283385842 55.1428691353181
0.770782228011191 54.9995726348798
0.794869172636541 54.8575260146264
0.818956117261891 54.7173742574894
0.843043061887241 54.5797118142911
0.86713000651259 54.4451323576606
0.89121695113794 54.3142415349258
0.91530389576329 54.1876781005248
0.939390840388639 54.066263977165
0.963477785013989 53.9507428186844
0.987564729639339 53.8412453841435
1.01165167426469 53.7375225904078
1.03573861889004 53.6393909383027
1.05982556351539 53.5464644547404
1.08391250814074 53.4580771595638
1.10799945276609 53.3734958776008
1.13208639739144 53.2916740969357
1.15617334201679 53.2113082084152
1.18026028664214 53.1310836315315
};
\addplot [semithick, white!20!black]
table {%
0 53.1261797887156
0.117903430039602 52.9312488345322
0.235806860079203 52.7289930378335
0.353710290118805 52.5203354547515
0.471613720158407 52.3062993612212
0.589517150198009 52.0880003418805
0.70742058023761 51.8665278935923
0.825324010277212 51.6428474419261
0.943227440316814 51.4175988914773
1.06113087035642 51.1907917632386
1.17903430039602 50.9614183139738
1.29693773043562 50.7280661905275
1.41484116047522 50.4904231837574
1.53274459051482 50.2485356342175
1.65064802055442 50.0024462413836
1.76855145059403 49.7522034270884
1.88645488063363 49.4978730220226
2.00435831067323 49.239466346598
2.12226174071283 48.9770142589149
2.24016517075243 48.7106587856841
2.35806860079203 48.4408181953847
2.47597203083164 48.1680504082262
2.59387546087124 47.892546946179
2.71177889091084 47.6142402841999
2.82968232095044 47.3329430017537
2.94758575099004 47.0484197664608
3.06548918102964 46.7604123251293
3.18339261106925 46.4686324678621
3.30129604110885 46.1727376940955
3.41919947114845 45.8723105515493
3.53710290118805 45.5668156711945
3.65500633122765 45.2556735562128
3.77290976126725 44.938436081426
3.89081319130686 44.6146418794241
4.00871662134646 44.283923808457
4.12662005138606 43.9459259800322
4.24452348142566 43.6002572361583
4.36242691146526 43.2466242968241
4.48033034150486 42.8848591214833
4.59823377154447 42.5148119009957
4.71613720158407 42.1363271837437
4.83404063162367 41.749370091759
4.95194406166327 41.3546263453132
5.06984749170287 40.9534765881208
5.18775092174247 40.547638687245
5.30565435178208 40.139299066813
5.42355778182168 39.7307686692844
5.54146121186128 39.3246291606
5.65936464190088 38.9238230378561
5.77726807194048 38.5313905880015
};
\addplot [semithick, white!20!black]
table {%
0 103.150539379898
0.150363652379559 102.962892970872
0.300727304759119 102.761584906943
0.451090957138678 102.548917734237
0.601454609518237 102.327389499744
0.751818261897797 102.099564559974
0.902181914277356 101.867968006983
1.05254556665692 101.634976649249
1.20290921903647 101.402476763243
1.35327287141603 101.171101152738
1.50363652379559 100.939518867892
1.65400017617515 100.705578989709
1.80436382855471 100.4687695135
1.95472748093427 100.229178406553
2.10509113331383 99.9868208301617
2.25545478569339 99.7416110689624
2.40581843807295 99.4934889242866
2.55618209045251 99.2424730930204
2.70654574283207 98.9885532725023
2.85690939521163 98.7317206834042
3.00727304759119 98.4724017617068
3.15763669997075 98.2112867563453
3.3080003523503 97.9486840611237
3.45836400472986 97.6845666487802
3.60872765710942 97.4187340039574
3.75909130948898 97.1509232862417
3.90945496186854 96.8809039690165
4.0598186142481 96.6085297467277
4.21018226662766 96.3335908048801
4.36054591900722 96.0558749093822
4.51090957138678 95.7751072142971
4.66127322376634 95.4906840835514
4.8116368761459 95.201878301207
4.96200052852546 94.9078946541512
5.11236418090502 94.6080560224801
5.26272783328458 94.3017035257277
5.41309148566414 93.9881827975567
5.56345513804369 93.6669497603603
5.71381879042325 93.337575512614
5.86418244280281 92.999397806442
6.01454609518237 92.6511975419265
6.16490974756193 92.291592126677
6.31527339994149 91.9200805846469
6.46563705232105 91.5368409078046
6.61600070470061 91.1427069811518
6.76636435708017 90.7391701415233
6.91672800945973 90.3278958377434
7.06709166183929 89.9109837755327
7.21745531421885 89.4909507640701
7.36781896659841 89.0703835944054
};
\addplot [semithick, white!20!black]
table {%
0 132.431420645916
0.0422505568770256 132.41459834581
0.0845011137540511 132.383164772042
0.126751670631077 132.339480795982
0.169002227508102 132.285900793286
0.211252784385128 132.224678425914
0.253503341262153 132.158015616279
0.295753898139179 132.08790823891
0.338004455016204 132.015899388667
0.38025501189323 131.942679589983
0.422505568770256 131.868227782068
0.464756125647281 131.792473051973
0.507006682524307 131.715846892436
0.549257239401332 131.638851425907
0.591507796278358 131.561859781009
0.633758353155383 131.485068081835
0.676008910032409 131.408627178855
0.718259466909434 131.33270855596
0.76051002378646 131.257389743714
0.802760580663486 131.182627179924
0.845011137540511 131.108546833127
0.887261694417537 131.035166598372
0.929512251294562 130.96245104638
0.971762808171588 130.890511403313
1.01401336504861 130.819541025778
1.05626392192564 130.749740375739
1.09851447880266 130.6813308904
1.14076503567969 130.614605749865
1.18301559255672 130.549938928008
1.22526614943374 130.487861778717
1.26751670631077 130.429146799914
1.30976726318779 130.374630473249
1.35201782006482 130.325067043414
1.39426837694184 130.2811081495
1.43651893381887 130.243260638331
1.47876949069589 130.211972094981
1.52102004757292 130.187757007484
1.56327060444995 130.171133579779
1.60552116132697 130.162610658251
1.647771718204 130.16256865976
1.69002227508102 130.170495140092
1.73227283195805 130.184633458702
1.77452338883507 130.202744259748
1.8167739457121 130.222439597678
1.85902450258912 130.241423239044
1.90127505946615 130.257163073283
1.94352561634318 130.26707284546
1.9857761732202 130.268310616229
2.02802673009723 130.257670491074
2.07027728697425 130.23186257355
};
\addplot [semithick, white!20!black]
table {%
0 78.8262042892168
0.125542171754205 78.6468581529343
0.251084343508411 78.4571278367935
0.376626515262616 78.2585918469081
0.502168687016821 78.0529595326562
0.627710858771027 77.84200840174
0.753253030525232 77.6274827659642
0.878795202279437 77.41098732739
1.00433737403364 77.1937260440924
1.12987954578785 76.976006813501
1.25542171754205 76.7567624285112
1.38096388929626 76.5343797222826
1.50650606105046 76.3085192379339
1.63204823280467 76.0792764156893
1.75759040455887 75.8467064047014
1.88313257631308 75.6108143348739
2.00867474806728 75.3716215397775
2.13421691982149 75.1291536856122
2.2597590915757 74.8834284420301
2.3853012633299 74.6345141621478
2.51084343508411 74.3828116074497
2.63638560683831 74.1288933707787
2.76192777859252 73.872981707466
2.88746995034672 73.615038042593
3.01301212210093 73.3548967773004
3.13855429385513 73.0923425324092
3.26409646560934 72.8271622798548
3.38963863736354 72.5591660547931
3.51518080911775 72.2881155093381
3.64072298087195 72.0137429504157
3.76626515262616 71.7357104764363
3.89180732438036 71.4535295855676
4.01734949613457 71.1667257160065
4.14289166788877 70.8747827514414
4.26843383964298 70.5772713788062
4.39397601139718 70.27377132639
4.51951818315139 69.9638502719823
4.6450603549056 69.6471720535853
4.7706025266598 69.3235121431051
4.89614469841401 68.9925529251875
5.02168687016821 68.6536870075061
5.14722904192242 68.306216616611
5.27277121367662 67.9501435957282
5.39831338543083 67.5860983393364
5.52385555718503 67.215158806721
5.64939772893924 66.8388971823139
5.77493990069344 66.459017354736
5.90048207244765 66.077522091033
6.02602424420185 65.6967460244008
6.15156641595606 65.3190975006139
};
\addplot [semithick, white!20!black]
table {%
0 90.387078285279
0.104414310325283 90.246929859155
0.208828620650565 90.0955543098829
0.313242930975848 89.9346802234229
0.417657241301131 89.7661309921279
0.522071551626414 89.5917579672162
0.626485861951696 89.4133751861931
0.730900172276979 89.2326416630389
0.835314482602262 89.0508077898687
0.939728792927545 88.8682563383695
1.04414310325283 88.684183041584
1.14855741357811 88.4973698924912
1.25297172390339 88.3076701034234
1.35738603422868 88.1152765267392
1.46180034455396 87.9203222025216
1.56621465487924 87.7228624232026
1.67062896520452 87.5229535457118
1.77504327552981 87.3206563806039
1.87945758585509 87.1160042940639
1.98387189618037 86.9090428344327
2.08828620650565 86.700105899968
2.19270051683094 86.4896277862289
2.29711482715622 86.2777649115501
2.4015291374815 86.0645138599663
2.50594344780679 85.8497964441243
2.61035775813207 85.6334989032899
2.71477206845735 85.415512757263
2.81918637878263 85.1957611131552
2.92360068910792 84.974150281104
3.0280149994332 84.7506004179206
3.13242930975848 84.5250353767453
3.23684362008376 84.2972886158889
3.34125793040905 84.0671915360354
3.44567224073433 83.8345207529591
3.55008655105961 83.5990828455659
3.6545008613849 83.3606766721165
3.75891517171018 83.1191070415129
3.86332948203546 82.8742521515923
3.96774379236074 82.6260727864905
4.07215810268603 82.3744352990426
4.17657241301131 82.1187858669826
4.28098672333659 81.8582086814075
4.38540103366187 81.5921962472769
4.48981534398716 81.3206816467822
4.59422965431244 81.0439500915849
4.69864396463772 80.7625985665527
4.80305827496301 80.4773092739695
4.90747258528829 80.1889241337803
5.01188689561357 79.8984466011259
5.11630120593885 79.606916401563
};
\addplot [semithick, white!20!black]
table {%
0 107.682335737436
0.145593017210605 107.505453556006
0.291186034421209 107.314503061824
0.436779051631814 107.111866325517
0.582372068842419 106.900113912892
0.727965086053023 106.681871134768
0.873558103263628 106.459722554911
1.01915112047423 106.236099615572
1.16474413768484 106.012936427956
1.31033715489544 105.790903629618
1.45593017210605 105.568731931461
1.60152318931665 105.344355914893
1.74711620652726 105.117309806803
1.89270922373786 104.88770899195
2.03830224094847 104.655588830684
2.18389525815907 104.420873316662
2.32948827536968 104.183508112368
2.47508129258028 103.943521589765
2.62067430979088 103.700906507308
2.76626732700149 103.455643668755
2.91186034421209 103.208141732242
3.0574533614227 102.959057447675
3.2030463786333 102.708684788605
3.34863939584391 102.45700734559
3.49423241305451 102.203847776052
3.63982543026512 101.948969830407
3.78541844747572 101.692171927493
3.93101146468633 101.433342239468
4.07660448189693 101.172313675842
4.22219749910754 100.908930415983
4.36779051631814 100.64299550099
4.51338353352875 100.373991329278
4.65897655073935 100.101265183996
4.80456956794995 99.8240912355545
4.95016258516056 99.5418471144272
5.09575560237116 99.2539243737655
5.24134861958177 98.9597250570377
5.38694163679237 98.6587557148656
5.53253465400298 98.3506299648517
5.67812767121358 98.0347205853661
5.82372068842419 97.7097927566998
5.96931370563479 97.3743669590478
6.1149067228454 97.0277713916674
6.260499740056 96.6699621395116
6.40609275726661 96.3015345079946
6.55168577447721 95.9236974657914
6.69727879168782 95.5378229782478
6.84287180889842 95.1456844204855
6.98846482610903 94.7494279881232
7.13405784331963 94.3512589855441
};
\addplot [semithick, white!20!black]
table {%
0 95.3902728868427
0.119713913768825 95.2350594924489
0.23942782753765 95.0677235347329
0.359141741306476 94.890202603873
0.478855655075301 94.7045630679748
0.598569568844126 94.512911048118
0.718283482612951 94.3173144853768
0.837997396381776 94.1196880091806
0.957711310150602 93.9215083852619
1.07742522391943 93.7232498434611
1.19713913768825 93.5239336465848
1.31685305145708 93.3220296979356
1.4365669652259 93.1172706740759
1.55628087899473 92.909816630127
1.67599479276355 92.6997623804141
1.79570870653238 92.4871156321869
1.9154226203012 92.2718931305651
2.03513653407003 92.0541423457413
2.15485044783885 91.8338819141846
2.27456436160768 91.6111372294427
2.3942782753765 91.3862719508125
2.51399218914533 91.1598047070505
2.63370610291415 90.9319426940055
2.75342001668298 90.7026756843901
2.8731339304518 90.4718861067471
2.99284784422063 90.2394119912191
3.11256175798946 90.0051063135682
3.23227567175828 89.7688725767014
3.35198958552711 89.5305823699126
3.47170349929593 89.2901172566977
3.59141741306476 89.0473422950921
3.71113132683358 88.8019506348407
3.83084524060241 88.5535897984562
3.95055915437123 88.3018472898252
4.07027306814006 88.0463692150484
4.18998698190888 87.7868023948521
4.30970089567771 87.5227998718332
4.42941480944653 87.2541001130332
4.54912872321536 86.9805344152596
4.66884263698418 86.7017905708983
4.78855655075301 86.4170823355131
4.90827046452183 86.1253197750383
5.02798437829066 85.8259756787122
5.14769829205948 85.519039402134
5.26741220582831 85.2049554856855
5.38712611959713 84.8845943264004
5.50684003336596 84.558940895075
5.62655394713478 84.2292307339068
5.74626786090361 83.8969437329627
5.86598177467244 83.5636063291147
};
\addplot [semithick, white!20!black]
table {%
0 65.1051444140979
0.0758138739300174 64.9772942909002
0.151627747860035 64.8416843715153
0.227441621790052 64.6992741857465
0.30325549572007 64.5510451851225
0.379069369650087 64.3980053555895
0.454883243580104 64.2411315718718
0.530697117510122 64.0812537396354
0.606510991440139 63.9188894265156
0.682324865370157 63.7540769378152
0.758138739300174 63.5863191743785
0.833952613230192 63.4150124270961
0.909766487160209 63.2402126229209
0.985580361090226 63.0621289729576
1.06139423502024 62.8809443248753
1.13720810895026 62.6968163355733
1.21302198288028 62.5098923818458
1.2888358568103 62.3202437904407
1.36464973074031 62.1279354681035
1.44046360467033 61.9330943111473
1.51627747860035 61.7360208044027
1.59209135253037 61.5370104283908
1.66790522646038 61.3361209855989
1.7437191003904 61.1333395826434
1.81953297432042 60.9286331738012
1.89534684825044 60.7219478948817
1.97116072218045 60.5132030574364
2.04697459611047 60.3022842467008
2.12278847004049 60.0890792719149
2.19860234397051 59.8734643898584
2.27441621790052 59.6553176396641
2.35023009183054 59.4346242005181
2.42604396576056 59.2115143283236
2.50185783969057 58.9860909213377
2.57767171362059 58.7584482167977
2.65348558755061 58.5286613727142
2.72929946148063 58.2967951428731
2.80511333541064 58.0629704761041
2.88092720934066 57.8273846108147
2.95674108327068 57.5902916932369
3.0325549572007 57.3518029891073
3.10836883113071 57.1117236115376
3.18418270506073 56.8700456733262
3.25999657899075 56.6271314280365
3.33581045292077 56.3834621821504
3.41162432685078 56.1396392352467
3.4874382007808 55.8962993560176
3.56325207471082 55.6540810054563
3.63906594864084 55.4136776449813
3.71487982257085 55.175819848716
};
\addplot [semithick, white!20!black]
table {%
0 66.3814703476331
0.230570527156844 66.0521088857214
0.461141054313689 65.7114842869713
0.691711581470533 65.3614984475676
0.922282108627378 65.0043897618837
1.15285263578422 64.6426063246837
1.38342316294107 64.2785705503226
1.61399369009791 63.9146100544409
1.84456421725476 63.5525732262822
2.0751347444116 63.1928861668462
2.30570527156844 62.8332282191693
2.53627579872529 62.4699345289374
2.76684632588213 62.1017717736089
2.99741685303898 61.7284782079928
3.22798738019582 61.3497829748681
3.45855790735267 60.9654039353392
3.68912843450951 60.5751403481983
3.91969896166636 60.1788839385088
4.1502694888232 59.7765630253706
4.38084001598004 59.3682358848011
4.61141054313689 58.9545728522573
4.84198107029373 58.5367807720497
5.07255159745058 58.1154196235328
5.30312212460742 57.6903388330203
5.53369265176427 57.2610187770852
5.76426317892111 56.8268243526243
5.99483370607796 56.3871479229967
6.2254042332348 55.9414470794914
6.45597476039165 55.4890001238709
6.68654528754849 55.0289335572924
6.91711581470533 54.5600484399427
7.14768634186218 54.0805683049451
7.37825686901902 53.5886244800982
7.60882739617587 53.0823226368592
7.83939792333271 52.5600956551209
8.06996845048956 52.0204563162519
8.3005389776464 51.4618606613993
8.53110950480325 50.8829590953514
8.76168003196009 50.2826217274893
8.99225055911693 49.65946378515
9.22282108627378 49.0119597398549
9.45339161343062 48.3393779986936
9.68396214058747 47.6429443545815
9.91453266774431 46.9252451611454
10.1451031949012 46.189903031034
10.375673722058 45.4418810629086
10.6062442492148 44.6864906510022
10.8368147763717 43.9300004576642
11.0673853035285 43.1797212959528
11.2979558306854 42.4431671391871
};
\addplot [semithick, white!20!black]
table {%
0 17.6600744783211
0.0518357469130427 17.5163941180473
0.103671493826085 17.3708162382613
0.155507240739128 17.2230321313269
0.207342987652171 17.0726784495614
0.259178734565213 16.9194511693202
0.311014481478256 16.7630279080971
0.362850228391299 16.6029646130884
0.414685975304341 16.4386542009728
0.466521722217384 16.269560280397
0.518357469130427 16.0954120008446
0.570193216043469 15.9161715737729
0.622028962956512 15.7320276206435
0.673864709869555 15.543128994241
0.725700456782597 15.3496646237426
0.77753620369564 15.1519000857651
0.829371950608683 14.9500870328894
0.881207697521725 14.7442818434212
0.933043444434768 14.5345824676206
0.984879191347811 14.3212559672153
1.03671493826085 14.1046114303368
1.0885506851739 13.8848589930227
1.14038643208694 13.6619676918583
1.19222217899998 13.4358800883182
1.24405792591302 13.2065540492367
1.29589367282607 12.9739361223339
1.34772941973911 12.7378959834012
1.39956516665215 12.4981655392322
1.45140091356519 12.2544796454596
1.50323666047824 12.0064865277581
1.55507240739128 11.753769204951
1.60690815430432 11.4962578585501
1.65874390121737 11.2342593452898
1.71057964813041 10.9681052330416
1.76241539504345 10.6981106206746
1.81425114195649 10.424569272004
1.86608688886954 10.1477247198745
1.91792263578258 9.86787101867403
1.96975838269562 9.58539441163587
2.02159412960866 9.30096314915894
2.07342987652171 9.0156261060006
2.12526562343475 8.73044440245425
2.17710137034779 8.44659156700419
2.22893711726083 8.16567022702008
2.28077286417388 7.88914003333607
2.33260861108692 7.61845524522276
2.38444435799996 7.35507233517987
2.43628010491301 7.10033644810024
2.48811585182605 6.85563795767154
2.53995159873909 6.62243810031648
};
\addplot [semithick, white!20!black]
table {%
0 122.280435938025
0.0642707751683547 122.224650723227
0.128541550336709 122.154919069531
0.192812325505064 122.073490515454
0.257083100673419 121.982646780448
0.321353875841773 121.884608310086
0.385624651010128 121.781547555996
0.449895426178483 121.675445645586
0.514166201346837 121.567834177576
0.578436976515192 121.459346314407
0.642707751683547 121.349689211264
0.706978526851901 121.238376424371
0.771249302020256 121.125641333943
0.835520077188611 121.011889892513
0.899790852356965 120.897416627863
0.96406162752532 120.782363784824
1.02833240269367 120.666843688913
1.09260317786203 120.55099291385
1.15687395303038 120.434872153403
1.22114472819874 120.318456409299
1.28541550336709 120.201938689455
1.34968627853545 120.085478783019
1.4139570537038 119.96911031865
1.47822782887216 119.852910525227
1.54249860404051 119.736985049875
1.60676937920887 119.621432045695
1.67104015437722 119.506369260618
1.73531092954558 119.391980865331
1.79958170471393 119.278500004919
1.86385247988229 119.166276077643
1.92812325505064 119.055827327713
1.99239403021899 118.947667853754
2.05666480538735 118.842238363254
2.1209355805557 118.739888644836
2.18520635572406 118.640881158639
2.24947713089241 118.545436047509
2.31374790606077 118.453823484293
2.37801868122912 118.366340458656
2.44228945639748 118.283303345682
2.50656023156583 118.204894389133
2.57083100673419 118.130517832141
2.63510178190254 118.058597158991
2.6993725570709 117.987368721409
2.76364333223925 117.915107660886
2.82791410740761 117.840285114854
2.89218488257596 117.761323941662
2.95645565774431 117.676640694261
3.02072643291267 117.5845398719
3.08499720808102 117.483133791484
3.14926798324938 117.37048740203
};
\addplot [semithick, white!20!black]
table {%
0 44.2627205049209
0.0603277150218114 44.1344058549257
0.120655430043623 44.0010142011451
0.180983145065434 43.8629187535958
0.241310860087245 43.7204709674314
0.301638575109057 43.5740573218384
0.361966290130868 43.424038601287
0.42229400515268 43.2706380363666
0.482621720174491 43.1138373440049
0.542949435196302 42.9534104019214
0.603277150218114 42.7890179089182
0.663604865239925 42.6204026488675
0.723932580261736 42.4477201586129
0.784260295283548 42.2711683213442
0.844588010305359 42.0909472317137
0.904915725327171 41.9072759956746
0.965243440348982 41.7203588952228
1.02557115537079 41.530266475078
1.0858988703926 41.3370825187997
1.14622658541442 41.1409975808572
1.20655430043623 40.942304032193
1.26688201545804 40.7412306472413
1.32720973047985 40.5377801723362
1.38753744550166 40.3319246087222
1.44786516052347 40.1236425635455
1.50819287554528 39.9128990324149
1.5685205905671 39.6996084994787
1.62884830558891 39.4836029269133
1.68917602061072 39.2647226519683
1.74950373563253 39.042768050241
1.80983145065434 38.81752242545
1.87015916567615 38.5890034578837
1.93048688069796 38.3574830701009
1.99081459571978 38.123228600927
2.05114231074159 37.8864845111075
2.1114700257634 37.6474720122043
2.17179774078521 37.4063859393547
2.23212545580702 37.1634703949916
2.29245317082883 36.9190481329521
2.35278088585064 36.6736075254628
2.41310860087245 36.4277235169219
2.47343631589427 36.1817691414412
2.53376403091608 35.9362148348814
2.59409174593789 35.6918951939507
2.6544194609597 35.4496195102024
2.71474717598151 35.2102228728849
2.77507489100332 34.9745512066709
2.83540260602513 34.7433718678368
2.89573032104695 34.5174712049648
2.95605803606876 34.2976821401636
};
\addplot [semithick, white!20!black]
table {%
0 73.7914845941155
0.141582293515962 73.5861066248078
0.283164587031925 73.3705631050086
0.424746880547887 73.1464074869824
0.566329174063849 72.9153536287525
0.707911467579812 72.6792092208909
0.849493761095774 72.439750807593
0.991076054611737 72.1986243813237
1.1326583481277 71.9570713935844
1.27424064164366 71.7153836574773
1.41582293515962 71.4722988024419
1.55740522867559 71.2258964020505
1.69898752219155 70.9756960380314
1.84056981570751 70.7217297562807
1.98215210922347 70.4639986579192
2.12373440273944 70.2024666392581
2.2653166962554 69.9371244573661
2.40689898977136 69.6679744851697
2.54848128328732 69.3950215346322
2.69006357680328 69.1183410124338
2.83164587031925 68.8383792218934
2.97322816383521 68.5558093255496
3.11481045735117 68.2709044296527
3.25639275086713 67.9836044846671
3.3979750443831 67.6936842206185
3.53955733789906 67.4008582107009
3.68113963141502 67.1048444529841
3.82272192493098 66.8053845636352
3.96430421844695 66.5021499659961
4.10588651196291 66.1947576148968
4.24746880547887 65.8827074701897
4.38905109899483 65.5652925752964
4.5306333925108 65.2418166272029
4.67221568602676 64.9115476678374
4.81379797954272 64.5738801916038
4.95538027305868 64.2282294238279
5.09696256657464 63.8739885961044
5.23854486009061 63.5106631417885
5.38012715360657 63.1378891615784
5.52170944712253 62.7551966866562
5.66329174063849 62.3618834783974
5.80487403415446 61.9573243990221
5.94645632767042 61.5417992641688
6.08803862118638 61.116342325011
6.22962091470234 60.6825173493374
6.37120320821831 60.2425157994021
6.51278550173427 59.7986949290763
6.65436779525023 59.3538137453528
6.79595008876619 58.911081095016
6.93753238228216 58.4738032523436
};
\addplot [semithick, white!20!black]
table {%
0 91.0741389916412
0.207666567700975 90.799380184838
0.415333135401951 90.5110753771196
0.622999703102926 90.2115779858267
0.830666270803901 89.9035460204412
1.03833283850488 89.5897878689749
1.24599940620585 89.273078276444
1.45366597390683 88.9560718881477
1.6613325416078 88.6409038108849
1.86899910930878 88.3282137585387
2.07666567700975 88.015980606671
2.28433224471073 87.7009442985015
2.4919988124117 87.382097637129
2.69966538011268 87.0593187928894
2.90733194781365 86.7324378744449
3.11499851551463 86.401216884885
3.3226650832156 86.0654800951699
3.53033165091658 85.7251683382878
3.73799821861756 85.3802238917284
3.94566478631853 85.030646971957
4.15333135401951 84.6770184677282
4.36099792172048 84.3203809228732
4.56866448942146 83.9612257654979
4.77633105712243 83.5994574912142
4.98399762482341 83.2346729782568
5.19166419252438 82.8663703563741
5.39933076022536 82.4940891131773
5.60699732792633 82.1174660361303
5.81466389562731 81.7359999161254
6.02233046332828 81.3491094813801
6.22999703102926 80.955998426223
6.43766359873023 80.5553237126792
6.64533016643121 80.1455826110058
6.85299673413219 79.7252189238414
7.06066330183316 79.292930660053
7.26832986953414 78.8474741924426
7.47599643723511 78.3875807658376
7.68366300493609 77.9121472536348
7.89132957263706 77.4202489334635
8.09899614033804 76.9106592437351
8.30666270803901 76.3817346890667
8.51432927573999 75.8322054583904
8.72199584344096 75.2623919356342
8.92966241114194 74.67371662083
9.13732897884291 74.0685656777101
9.34499554654389 73.4504515670315
9.55266211424486 72.8231807349537
9.76032868194584 72.1913570189352
9.96799524964682 71.5604049079646
10.1756618173478 70.9358961539659
};
\end{axis}

\end{tikzpicture}

	\caption{Speed of the lead vehicle during 100 generated scenarios scenarios.}
	\label{fig:speed lvd generated}
\end{figure}

\begin{table}
	\centering
	\caption{Explained variance according to \cref{eq:explained variance}.}
	\label{tab:explained variance}
	\begin{tabular}{lcc}
		\toprule
		& \multicolumn{2}{c}{Explained variance [\%]} \\
		$\dimension$ & Lead vehicle decelerating & Cut in \\ \otoprule
		1 & 77.5 & 40.0 \\
		2 & 99.8 & 73.6 \\
		3 & 100.0 & 99.3 \\
		4 & 100.0 & 99.8 \\
		5 & 100.0 & 100.0 \\
		\bottomrule
	\end{tabular}
\end{table}

Another way to determine $\dimension$ is to use the metric defined in \cref{eq:proposed metric}.


\setlength{\figureheight}{.8\figurewidth}
\begin{figure}
	\centering
	% This file was created by tikzplotlib v0.9.8.
\begin{tikzpicture}

\definecolor{color0}{rgb}{0.12156862745098,0.466666666666667,0.705882352941177}

\begin{axis}[
height=\figureheight,
scaled y ticks=false,
tick align=outside,
tick pos=left,
width=\figurewidth,
x grid style={white!69.0196078431373!black},
xmin=-0.2, xmax=4.2,
xtick style={color=black},
xtick={0,1,2,3,4},
xticklabel style={align=center},
xticklabels={
  Training\\set,
  $\dimension=2$,
  $\dimension=3$,
  $\dimension=4$,
  $\dimension=5$
},
y grid style={white!69.0196078431373!black},
ylabel={Metric},
ymin=0.025144751084137, ymax=0.11225480135924,
ytick style={color=black},
yticklabel style={/pgf/number format/fixed,/pgf/number format/precision=3}
]
\addplot [semithick, black, mark=*, mark size=3, mark options={solid}, only marks]
table {%
0 0.0996920534467681
1 0.0910142737582867
2 0.0968997198896531
3 0.104467780771374
4 0.108295253619462
};
\addplot [semithick, color0, mark=square*, mark size=3, mark options={solid,fill=white,draw=black}, only marks]
table {%
0 0.0816408708306334
1 0.0812073458096283
2 0.087547416511482
3 0.0970130610643488
4 0.100538689760016
};
\addplot [semithick, black, mark=square*, mark size=3, mark options={solid}, only marks]
table {%
0 0.0720144356477625
1 0.039957883857726
2 0.0358945906846928
3 0.0296270642036683
4 0.0291042988239144
};
\end{axis}

\end{tikzpicture}

	\caption{Medians of the metrics for the set of generated lead-vehicle-decelerating scenarios. Circle: proposed metric $\proposedmetric{\scenariosetgenerated}{\scenariosettest}{\scenarioset}$ of \cref{eq:proposed metric}, open square: empirical Wasserstein metric $\wasserstein{\scenariosettest}{\scenariosetgenerated}$ of \cref{eq:empirical wasserstein}, filled square: the ``penalty'' $\wasserstein{\scenariosettest}{\scenariosetgenerated} - \wasserstein{\scenarioset}{\scenariosetgenerated}$.}
	\label{fig:scores lvd}
\end{figure}

\begin{figure}
	\centering
	% This file was created by tikzplotlib v0.9.8.
\begin{tikzpicture}

\begin{axis}[
height=\figureheight,
scaled y ticks=false,
tick align=outside,
tick pos=left,
width=\figurewidth,
x grid style={white!69.0196078431373!black},
xmin=-0.3, xmax=6.3,
xtick style={color=black},
xtick={0,1,2,3,4,5,6},
xticklabel style={align=center},
xticklabels={
  Training\\set,
  $\dimension=2$,
  $\dimension=3$,
  $\dimension=4$,
  $\dimension=5$,
  $\dimension=6$,
  $\dimension=7$
},
y grid style={white!69.0196078431373!black},
ylabel={$\proposedmetric{\scenariosetgenerated}{\scenariosettest}{\scenarioset}$},
ymin=0.808099487917019, ymax=0.998798688044162,
ytick style={color=black},
yticklabel style={/pgf/number format/fixed,/pgf/number format/precision=3}
]
\addplot [semithick, black, mark=*, mark size=3, mark options={solid}, only marks]
table {%
0 0.874759048955906
1 0.990130542583838
2 0.816767633377344
3 0.829169755389455
4 0.839784622142605
5 0.847407179929393
6 0.854073070799821
};
\end{axis}

\end{tikzpicture}

	\caption{Medians of the metrics for the set of generated cut-in scenarios. For the meaning of the symbols, see \cref{fig:scores lvd}.}
	\label{fig:scores ci}
\end{figure}



\subsection{Evaluating the scenario comparison metric}
\label{sec:case study comparison metric}

\setlength{\figureheight}{.8\figurewidth}
\begin{figure}
	\centering
	% This file was created by tikzplotlib v0.9.8.
\begin{tikzpicture}

\definecolor{color0}{rgb}{0.12156862745098,0.466666666666667,0.705882352941177}
\definecolor{color1}{rgb}{1,0.498039215686275,0.0549019607843137}

\begin{axis}[
height=\figureheight,
scaled y ticks=false,
tick align=outside,
tick pos=left,
width=\figurewidth,
x grid style={white!69.0196078431373!black},
xmin=-0.2, xmax=4.2,
xtick style={color=black},
xtick={0,1,2,3,4},
xticklabel style={align=center},
xticklabels={
  Training\\set,
  $\dimension=2$,
  $\dimension=3$,
  $\dimension=4$,
  $\dimension=5$
},
y grid style={white!69.0196078431373!black},
ylabel={Wasserstein metric},
ymin=0.0318224305315917, ymax=0.110580293241016,
ytick style={color=black},
yticklabel style={/pgf/number format/fixed,/pgf/number format/precision=3}
]
\addplot [semithick, black, mark=*, mark size=3, mark options={solid}, only marks]
table {%
0 0.0969048169877733
1 0.0916116581940098
2 0.0989467999493009
3 0.103054597911698
4 0.107000390390587
};
\addplot [semithick, color0, mark=square*, mark size=3, mark options={solid,fill=white,draw=black}, only marks]
table {%
0 0.0797630187992332
1 0.0807005529247066
2 0.089092019800953
3 0.0933136154566217
4 0.0988564660597343
};
\addplot [semithick, black, mark=square*, mark size=3, mark options={solid}, only marks]
table {%
0 0.0685427529438103
1 0.0444340662903759
2 0.0388628737084231
3 0.03779375492379
4 0.0354023333820201
};
\addplot [semithick, color1, mark=*, mark size=3, mark options={solid,fill=white,draw=black}, only marks]
table {%
0 0.0540553661074323
1 0.046175432081863
2 0.059522840899614
3 0.0642109817985778
4 0.068917480270215
};
\end{axis}

\end{tikzpicture}

\end{figure}

\begin{figure}
	\centering
	% This file was created by tikzplotlib v0.9.8.
\begin{tikzpicture}

\begin{axis}[
height=\figureheight,
scaled y ticks=false,
tick align=outside,
tick pos=left,
width=\figurewidth,
x grid style={white!69.0196078431373!black},
xmin=-0.3, xmax=6.3,
xtick style={color=black},
xtick={0,1,2,3,4,5,6},
xticklabel style={align=center},
xticklabels={
  Training\\set,
  $\dimension=2$,
  $\dimension=3$,
  $\dimension=4$,
  $\dimension=5$,
  $\dimension=6$,
  $\dimension=7$
},
y grid style={white!69.0196078431373!black},
ylabel={Wasserstein metric and penalty},
ymin=0.105006846252839, ymax=0.996287170175304,
ytick style={color=black},
yticklabel style={/pgf/number format/fixed,/pgf/number format/precision=3}
]
\addplot [semithick, black, mark=square*, mark size=3, mark options={solid}, only marks]
table {%
0 0.473857831804062
1 0.837166874552501
2 0.35371704549914
3 0.365409466187815
4 0.380965292821083
5 0.38777659301658
6 0.401378680073143
};
\addplot [semithick, black, mark=diamond*, mark size=3, mark options={solid}, only marks]
table {%
0 0.713695231043646
1 0.955774428178828
2 0.734131233466173
3 0.751358946371325
4 0.762894243482017
5 0.772922836639589
6 0.777198876885373
};
\addplot [semithick, black, mark=x, mark size=3, mark options={solid}, only marks]
table {%
0 0.646781107728405
1 0.145519588249315
2 0.323354492526922
3 0.313200607671112
4 0.313135234591292
5 0.305503034903593
6 0.301066785587691
};
\end{axis}

\end{tikzpicture}

\end{figure}

\begin{figure}
	% This file was created by tikzplotlib v0.9.8.
\begin{tikzpicture}

\begin{axis}[
height=\figureheight,
scaled y ticks=false,
tick align=outside,
tick pos=left,
width=\figurewidth,
x grid style={white!69.0196078431373!black},
xlabel={$\penaltyweight$},
xmin=0, xmax=1,
xtick style={color=black},
xticklabel style={align=center},
y grid style={white!69.0196078431373!black},
ylabel={Correlation},
ymin=0.07121287239394, ymax=1.03906942913872,
ytick style={color=black},
yticklabel style={/pgf/number format/fixed,/pgf/number format/precision=3}
]
\addplot [semithick, black]
table {%
0 0.934792984742484
0.0101010101010101 0.938428542687498
0.0202020202020202 0.942053598555844
0.0303030303030303 0.945661255700661
0.0404040404040404 0.949243894833895
0.0505050505050505 0.952793115853462
0.0606060606060606 0.956299676814818
0.0707070707070707 0.959753430292117
0.0808080808080808 0.963143257483689
0.0909090909090909 0.966457000547606
0.101010101010101 0.969681393807899
0.111111111111111 0.972801994652858
0.121212121212121 0.975803115155616
0.131313131313131 0.978667755685162
0.141414141414141 0.981377542043467
0.151515151515152 0.983894805656626
0.161616161616162 0.986076121132653
0.171717171717172 0.988005046196441
0.181818181818182 0.989695144903219
0.191919191919192 0.991033056136645
0.202020202020202 0.99212203081362
0.212121212121212 0.992881485047917
0.222222222222222 0.993283962678707
0.232323232323232 0.993303927095304
0.242424242424242 0.992915476987417
0.252525252525253 0.992092532669579
0.262626262626263 0.990809048801921
0.272727272727273 0.989039253136146
0.282828282828283 0.986757910118305
0.292929292929293 0.98394060727367
0.303030303030303 0.980564061304281
0.313131313131313 0.976606439777323
0.323232323232323 0.972047693213612
0.333333333333333 0.966869891350704
0.343434343434343 0.961057556413751
0.353535353535354 0.954597985442972
0.363636363636364 0.947481553165324
0.373737373737374 0.939701986621214
0.383838383838384 0.931256602817795
0.393939393939394 0.922146501116318
0.404040404040404 0.912376702889965
0.414141414141414 0.902114476729616
0.424242424242424 0.891377069212476
0.434343434343434 0.880048659285284
0.444444444444444 0.867879909175544
0.454545454545455 0.855134914795377
0.464646464646465 0.841979316166824
0.474747474747475 0.828332686830153
0.484848484848485 0.814218110759703
0.494949494949495 0.79966950845137
0.505050505050505 0.784722678454929
0.515151515151515 0.769414901333072
0.525252525252525 0.753784534361088
0.535353535353535 0.737870606247983
0.545454545454546 0.721712420640917
0.555555555555556 0.705349176357211
0.565656565656566 0.688819611235194
0.575757575757576 0.672161675277314
0.585858585858586 0.655412237449148
0.595959595959596 0.638606829165718
0.606060606060606 0.621779426203862
0.616161616161616 0.604962269578107
0.626262626262626 0.58818572484685
0.636363636363636 0.571478178402414
0.646464646464647 0.554865968557156
0.656565656565657 0.538373348671608
0.666666666666667 0.522022479174014
0.676767676767677 0.505833445080714
0.686868686868687 0.489824295525824
0.696969696969697 0.474011101825925
0.707070707070707 0.458408030718839
0.717171717171717 0.443027429603461
0.727272727272727 0.42787992084939
0.737373737373737 0.412974502522265
0.747474747474748 0.398318653167438
0.757575757575758 0.383918438597234
0.767676767676768 0.369778618925276
0.777777777777778 0.355902754376694
0.787878787878788 0.342293308669852
0.797979797979798 0.328951749009517
0.808080808080808 0.315878641950995
0.818181818181818 0.303073744588701
0.828282828282828 0.290536090691289
0.838383838383838 0.278264071549456
0.848484848484849 0.266255511423858
0.858585858585859 0.254507737580455
0.868686868686869 0.243017644981806
0.878787878787879 0.231781755766838
0.888888888888889 0.220796273701101
0.898989898989899 0.210057133816063
0.909090909090909 0.199560047481876
0.919191919191919 0.189300543174745
0.929292929292929 0.179274003209313
0.939393939393939 0.169475696709583
0.94949494949495 0.159900809090202
0.95959595959596 0.1505444683144
0.96969696969697 0.141401768186421
0.97979797979798 0.132467788925717
0.98989898989899 0.123737615258105
1 0.115206352245976
};
\addplot [semithick, black, dashed]
table {%
0 0.899803548415668
0.0101010101010101 0.905210972557616
0.0202020202020202 0.910688449707165
0.0303030303030303 0.916200569112459
0.0404040404040404 0.921657229145063
0.0505050505050505 0.927067562752939
0.0606060606060606 0.932444376895512
0.0707070707070707 0.93779366977688
0.0808080808080808 0.9430972282808
0.0909090909090909 0.948382323355878
0.101010101010101 0.953539124712399
0.111111111111111 0.958543484705127
0.121212121212121 0.963369256641884
0.131313131313131 0.967968633617284
0.141414141414141 0.972252588983394
0.151515151515152 0.976273286992994
0.161616161616162 0.980008123324168
0.171717171717172 0.983414827254407
0.181818181818182 0.986455916859611
0.191919191919192 0.989093941025387
0.202020202020202 0.991292177857918
0.212121212121212 0.993018691448206
0.222222222222222 0.994241843037398
0.232323232323232 0.994940209579304
0.242424242424242 0.995075949286687
0.252525252525253 0.994639438760628
0.262626262626263 0.993578882133864
0.272727272727273 0.991865736137652
0.282828282828283 0.989537056488287
0.292929292929293 0.986517194781438
0.303030303030303 0.982779679376141
0.313131313131313 0.978307115263628
0.323232323232323 0.973086227824455
0.333333333333333 0.967108235245414
0.343434343434343 0.960369157713912
0.353535353535354 0.952920399058555
0.363636363636364 0.94481149972334
0.373737373737374 0.935980206625256
0.383838383838384 0.926441420960957
0.393939393939394 0.916226514308715
0.404040404040404 0.905362212735189
0.414141414141414 0.893865207375431
0.424242424242424 0.881768777181753
0.434343434343434 0.86910964698867
0.444444444444444 0.855794101349232
0.454545454545455 0.841106982545635
0.464646464646465 0.825856991699722
0.474747474747475 0.810100933516507
0.484848484848485 0.793886330384478
0.494949494949495 0.777250484331603
0.505050505050505 0.760263011307821
0.515151515151515 0.742979577811719
0.525252525252525 0.727088337365719
0.535353535353535 0.711139957486473
0.545454545454546 0.695144944354586
0.555555555555556 0.67914061363099
0.565656565656566 0.663161765007685
0.575757575757576 0.647240591431714
0.585858585858586 0.631303734575407
0.595959595959596 0.615281328416067
0.606060606060606 0.599382116694491
0.616161616161616 0.583635696267025
0.626262626262626 0.568062896704238
0.636363636363636 0.552682100931137
0.646464646464647 0.537486417615389
0.656565656565657 0.522469961947125
0.666666666666667 0.507676341996199
0.676767676767677 0.493120311299577
0.686868686868687 0.478810556711802
0.696969696969697 0.464754107302406
0.707070707070707 0.450956471662261
0.717171717171717 0.437421772101866
0.727272727272727 0.424152874312264
0.737373737373737 0.411151511405275
0.747474747474748 0.398418401551493
0.757575757575758 0.3859533586929
0.767676767676768 0.373755396024376
0.777777777777778 0.36182282211787
0.787878787878788 0.350153329708351
0.797979797979798 0.338744077275568
0.808080808080808 0.327591763644214
0.818181818181818 0.316692695890928
0.828282828282828 0.306042850893239
0.838383838383838 0.295637930886373
0.848484848484849 0.285473413411502
0.858585858585859 0.275544596046235
0.868686868686869 0.265846636307063
0.878787878787879 0.256374587105909
0.888888888888889 0.247123428130614
0.898989898989899 0.238088093503332
0.909090909090909 0.229263496052536
0.919191919191919 0.220644548514564
0.929292929292929 0.212226181960057
0.939393939393939 0.204003361719755
0.94949494949495 0.195971101063504
0.95959595959596 0.188124472866103
0.96969696969697 0.18045861947424
0.97979797979798 0.172968760970222
0.98989898989899 0.165650202010737
1 0.158498337402432
};
\end{axis}

\end{tikzpicture}

\end{figure}
