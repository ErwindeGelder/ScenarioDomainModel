\section{Introduction}
\label{sec:introduction}

% Introduce scenario-based testing
An essential facet in the development of \acp{av} is the assessment of quality and performance aspects of the \acp{av}, such as safety, comfort, and efficiency \autocite{bengler2014threedecades, stellet2015taxonomy, koopman2016challenges}. 
Because public road tests are expensive and time consuming \autocite{kalra2016driving, zhao2018evaluation}, a scenario-based approach has been proposed \autocite{riedmaier2020survey, elrofai2018scenario, putz2017pegasus, krajewski2018highD, deGelder2017assessment, stellet2015taxonomy, jacobo2019development}.
As a source of information for the scenarios for the assessment, real-world driving data has been proposed, such that the scenarios relate to real-world driving conditions \autocite{elrofai2018scenario, putz2017pegasus, krajewski2018highD}.

% Scenarios need to be representative and cover the same variety.
For the scenario-based assessment approach, it is important that the generated scenarios are representative of scenarios that could happen in real life. 
In other words ``scenarios should reflect the real world as closely as possible'' \autocite{riedmaier2020survey}.
Only then, the results of the assessment can be generalized to the performance of the system-under-test when operating in real life \autocite{deGelder2017assessment}.
Furthermore, it is essential that the generated scenarios cover the same variety that is found in real life. 
\textcite{riedmaier2020survey} argue that ``since infinite situations occur in the real world, the scenario methods must cover a large number of permutations within the physically possible parameter space.''

% Existing methods.
In literature, several methods are proposed to generate scenarios for the assessment based on real-world driving data. 
\textcite{lages2013automatic} proposed a method to create virtual scenarios using data from laser scanners. 
In \autocite{zofka2015datadrivetrafficscenarios}, the authors presented how recorded sensor data can be exploited ``to create critical traffic scenarios for open-loop testing'' by modifying few parameters of the recorded parameterized scenarios. 
In \autocite{feng2020testing, deGelder2017assessment, thal2020incorporating}, also parameterized scenarios were generated and, in addition, importance sampling techniques were presented that automatically generate scenarios in which the system-under-test shows (safety-)critical behavior. 
\textcite{schuldt2018method} provided a method to generate scenario-based test cases using combinatorial algorithms that should ensure that the test cases cover the variety of possible situation the system-under-test can encounter in real life.
More recently, \autocite{spooner2021generation} presented a generative adversarial network to generate pedestrian crossing scenarios.

% Problems of existing methods.
In the existing literature, the scenario generation methods for the assessment of \acp{av} have either one or more of the following shortcomings:
\begin{itemize}
	\item Observed scenarios are replayed without adding more variations. 
	In this case, the total variety of scenarios that is found in real life will not be covered unless unrealistic amounts of data are gathered.
	
	\item The scenarios are oversimplified.
	E.g., a vehicle's speed profile follows predetermined functional form.
	
	\item Too many assumptions are made regarding the distributions of the parameters.
	E.g., the parameters are assumed to come from a Gaussian distribution and/or it is assumed that the parameters are uncorrelated.
	
	\item No \ac{pdf} of the scenario parameters is known. 
	As a result, no evaluation can be made of the performance of the system once deployed on the road, because it is unknown how realistic and likely the scenario-based test cases are.
	
	\item The generated scenarios are not compared with the observed scenarios. 
	I.e., it might be argued that the parameters of the generated scenarios come from the estimated \ac{pdf} and, therefore, they follow the same trend, but this ignores the potentially disrupting oversimplifications done by the assumptions on the parameters and their distributions.
\end{itemize}

% What we propose.
We propose a method that overcomes the aforementioned shortcomings.
Our data-driven approach uses observed scenarios to generate new scenarios.
Instead of relying on a predetermined functional form of the signals, such as a vehicle's speed, and fitting parameters to this functional form, we employ \iac{svd} \autocite{golub2013matrix} to determine in a data-driven manner the parameters that describe the scenarios best.
Next, we use \ac{kde} \autocite{rosenblatt1956remarks, parzen1962estimation} to estimate the \ac{pdf} of the parameters without assuming a particular shape of the \ac{pdf}. 
Furthermore, with \ac{kde}, we model the correlations that might exists among the parameters.
We will also propose a metric that quantifies to what extend the generated scenarios are representative and cover the actual variety of real-life scenarios. 
This metric uses the Wasserstein metric \autocite{ruschendorf1985wasserstein} to compare a set of generated scenarios with a set of observed scenarios.

% Structure of the paper
This article is organized as follows.
In \cref{sec:generation}, we will explain our approach for generating scenarios for the assessment of \acp{av}. 
Next, in \cref{sec:comparison}, we will propose a metric for quantifying the performance of the scenario-generation method.
A case study is performed in \cref{sec:case study}.
In \cref{sec:discussion}, we will discuss few implications of our approach and some other applications.
Conclusions of the paper are provided in \cref{sec:conclusions}.
