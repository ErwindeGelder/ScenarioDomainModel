\section{Introduction}
\label{sec:introduction}

% Introduce scenario-based testing
An essential facet in the development of \acp{av} is the assessment of quality and performance aspects of the \acp{av}, such as safety, comfort, and efficiency \autocite{bengler2014threedecades, stellet2015taxonomy, koopman2016challenges}. 
Because public road tests are expensive and time consuming \autocite{kalra2016driving, zhao2018evaluation}, a scenario-based approach has been proposed \autocite{riedmaier2020survey, elrofai2018scenario, putz2017pegasus, krajewski2018highD, deGelder2017assessment, stellet2015taxonomy, jacobo2019development}.
As a source of information for the scenarios for the assessment, real-world driving data has been proposed, such that the scenarios relate to real-world driving conditions \autocite{elrofai2018scenario, putz2017pegasus, krajewski2018highD}.

% Scenarios need to be representative and cover the same variety.
For the scenario-based assessment approach, it is important that the generated scenarios are representative of scenarios that could happen in real life. 
In other words ``scenarios should reflect the real world as closely as possible'' \autocite{riedmaier2020survey}.
Only then, the results of the assessment can be generalized to the performance of the system-under-test when operating in real life \autocite{deGelder2017assessment}.
Furthermore, it is essential that the generated scenarios cover the same variety that is found in real life. 
\textcite{riedmaier2020survey} argue that ``since infinite situation occur in the real world, the scenario methods must cover a large number of permutations within the physically possible parameter space.''

% Existing methods.

% Structure of the paper
This paper is organized as follows.
