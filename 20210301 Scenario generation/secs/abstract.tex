\begin{abstract}

The development of assessment methods for the performance of \acp{av} is essential to enable to deployment of automated driving technologies, due to the complex operational domain of \acp{av}. 
One candidate is scenario-based assessment in which test cases are derived from real-world road traffic scenarios obtained from driving data.

Our contribution is twofold. 
First, we propose a method to generate scenarios for use in test case descriptions for the assessment of \acp{av}. 
The generated scenarios are based on real-world scenarios. 
Using a singular value decomposition, a reduction of parameters is obtained, such that the remaining parameters describe as much of the variations found in the real-world scenario as possible. 
Second, we propose a metric based on the Wasserstein distance that quantifies to what extend the generated scenarios are representative of real-world scenarios while covering the actual variety found in the real-world scenarios. 

We illustrate our proposed method by generating scenarios with a lead vehicle braking and cut-in scenarios. 
Our proposed metric is used to determine the appropriate number of parameters to be selected.  

\end{abstract}
