\documentclass[10pt,final,a4paper,oneside,onecolumn]{article}

%%==========================================================================
%% Packages
%%==========================================================================
\usepackage[a4paper,left=3.5cm,right=3.5cm,top=3cm,bottom=3cm]{geometry} %% change page layout; remove for IEEE paper format
\usepackage[T1]{fontenc}                        %% output font encoding for international characters (e.g., accented)
\usepackage[cmex10]{amsmath}                    %% math typesetting; consider using the [cmex10] option
\usepackage{amssymb}                            %% special (symbol) fonts for math typesetting
\usepackage{amsthm}                             %% theorem styles
\usepackage{dsfont}                             %% double stroke roman fonts: the real numbers R: $\mathds{R}$
\usepackage{mathrsfs}                           %% formal script fonts: the Laplace transform L: $\mathscr{L}$
\usepackage[pdftex]{graphicx}                   %% graphics control; use dvips for TeXify; use pdftex for PDFTeXify
\usepackage{array}                              %% array functionality (array, tabular)
\usepackage{upgreek}                            %% upright Greek letters; add the prefix 'up', e.g. \upphi
\usepackage[noadjust]{cite}                     %% citations; noadjust removes leading spaces
%\usepackage[round]{natbib}                     %% Author-year citations (remove package cite)
\usepackage{stfloats}                           %% improved handling of floats
\usepackage{multirow}                           %% cells spanning multiple rows in tables
%\usepackage{subfigure}                         %% subfigures and corresponding captions (for use with IEEEconf.cls)
\usepackage{subfig}                             %% subfigures (IEEEtran.cls: set caption=false)
\usepackage{fancyhdr}                           %% page headers and footers
\usepackage[official,left]{eurosym}             %% the euro symbol; command: \euro
\usepackage{appendix}                           %% appendix layout
\usepackage{xspace}                             %% add space after macro depending on context
\usepackage{verbatim}                           %% provides the comment environment
\usepackage[dutch,USenglish]{babel}             %% language support
\usepackage{wrapfig}                            %% wrapping text around figures
\usepackage{longtable}                          %% tables spanning multiple pages
\usepackage{pgfplots}                           %% support for TikZ figures (Matlab)
\usepackage[breaklinks=true,hidelinks,          %% implement hyperlinks (dvips yields minor problems with breaklinks;
bookmarksnumbered=true]{hyperref}   %% IEEEtran: set bookmarks=false)
%\usepackage[hyphenbreaks]{breakurl}            %% allow line breaks in URLs (don't use with PDFTeX)
\usepackage[final]{pdfpages}                    %% Include other pdfs

%%==========================================================================
%% Define header/title stuff
%%==========================================================================
\newcommand{\progressreportnumber}{5}
\renewcommand{\author}{Erwin de Gelder}
\renewcommand{\date}{22 February 2018}
\renewcommand{\title}{Performance assessment of automated vehicles using real-life driving scenarios}

%%==========================================================================
%% Fancy headers and footers
%%==========================================================================
\pagestyle{fancy}                                       %% set page style
\fancyhf{}                                              %% clear all header & footer fields
\fancyhead[L]{Progress report \progressreportnumber}    %% define headers (LE: left field/even pages, etc.)
\fancyhead[R]{\author, \date}                           %% similar
\fancyfoot[C]{\thepage}                                 %% define footer

\begin{document}
	
\begin{center}
	\begin{tabular}{c}
		\title \\ \\
		\textbf{\huge Progress report \progressreportnumber} \\ \\
		\author \\ 
		\date
	\end{tabular}
\end{center}

\section{Previous meeting minutes}

\begin{itemize}
	\item The paper ``Ontology of scenario for the assessment of automated vehicles'' for the Intelligent Vehicles Symposium 2018 is almost finished. Has to be reviewed by Jeroen and Bart. Deadline is the 29th of January.
	\item For the graduate school, I need to obtain 45 Graduate School Credits (GSCs) are divided into three times 15 GSCs:
	\begin{itemize}
		\item \emph{Research skills}: These are Learning on-the-Job Activities, e.g., paper review, presentation, poster, paper. I assume that these GSCs will be automatically achieved during the PhD, so no special planning regarding this.
		\item \emph{Discipline-related skills}: This relates to skills that improve knowledge related to the research. I will look if I can find some courses in Sinagpore.
		\item \emph{Transferable skills}: These skills will improve myself on a personal level. When I return from Singapore, I will look for useful courses.
	\end{itemize}
	\item We discussed about the report I wrote regarding the generation of test cases. A couple of remarks regarding the report:
	\begin{itemize}
		\item The problem formulation was at best very vague.
		\item Possibilities for the use of copulas should be shortly summarized.
		\item Most parts where not properly explained and notation of variables were mixed up.
		\item For a similarity measure that quantifies the similarity between two profiles, it would make sense to normalize it, such that it is between $0$ and $1$. 
		\item 
	\end{itemize}
\end{itemize}

\section{Summary of work}

\begin{itemize}
	\item 
\end{itemize}

\section{Future plans}

\begin{itemize}
	\item 
\end{itemize}

\section{Questions}

\begin{itemize}
	\item 
\end{itemize}

\bibliographystyle{ieeetr}
\bibliography{../../bib}

\end{document}