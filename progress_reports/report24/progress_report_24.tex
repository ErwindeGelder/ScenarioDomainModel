\documentclass[10pt,final,a4paper,oneside,onecolumn]{article}

%%==========================================================================
%% Packages
%%==========================================================================
\usepackage[a4paper,left=3.5cm,right=3.5cm,top=3cm,bottom=3cm]{geometry} %% change page layout; remove for IEEE paper format
\usepackage[T1]{fontenc}                        %% output font encoding for international characters (e.g., accented)
\usepackage[cmex10]{amsmath}                    %% math typesetting; consider using the [cmex10] option
\usepackage{amssymb}                            %% special (symbol) fonts for math typesetting
\usepackage{amsthm}                             %% theorem styles
\usepackage{dsfont}                             %% double stroke roman fonts: the real numbers R: $\mathds{R}$
\usepackage{mathrsfs}                           %% formal script fonts: the Laplace transform L: $\mathscr{L}$
\usepackage[pdftex]{graphicx}                   %% graphics control; use dvips for TeXify; use pdftex for PDFTeXify
\usepackage{array}                              %% array functionality (array, tabular)
\usepackage{upgreek}                            %% upright Greek letters; add the prefix 'up', e.g. \upphi
\usepackage{stfloats}                           %% improved handling of floats
\usepackage{multirow}                           %% cells spanning multiple rows in tables
%\usepackage{subfigure}                         %% subfigures and corresponding captions (for use with IEEEconf.cls)
\usepackage{subfig}                             %% subfigures (IEEEtran.cls: set caption=false)
\usepackage{fancyhdr}                           %% page headers and footers
\usepackage[official,left]{eurosym}             %% the euro symbol; command: \euro
\usepackage{appendix}                           %% appendix layout
\usepackage{xspace}                             %% add space after macro depending on context
\usepackage{verbatim}                           %% provides the comment environment
\usepackage[dutch,USenglish]{babel}             %% language support
\usepackage{wrapfig}                            %% wrapping text around figures
\usepackage{longtable}                          %% tables spanning multiple pages
\usepackage{pgfplots}                           %% support for TikZ figures (Matlab/Python)
\pgfplotsset{compat=1.14}						%% Run in backwards compatibility mode
\usepackage[breaklinks=true,hidelinks,          %% implement hyperlinks (dvips yields minor problems with breaklinks;
bookmarksnumbered=true]{hyperref}   %% IEEEtran: set bookmarks=false)
%\usepackage[hyphenbreaks]{breakurl}            %% allow line breaks in URLs (don't use with PDFTeX)
\usepackage[final]{pdfpages}                    %% Include other pdfs
\usepackage[capitalize]{cleveref}				%% Referensing to figures, equations, etc.
\usepackage{units}								%% Appropriate behavior of units
\usepackage[utf8]{inputenc}   				 	%% utf8 support (required for biblatex)
\usepackage{csquotes}							%% Quoted texts are typeset according to rules of main language

%%==========================================================================
%% Define reference stuff
%%==========================================================================
\crefname{figure}{Figure}{Figures}
\crefname{equation}{}{}

%%==========================================================================
%% Define header/title stuff
%%==========================================================================
\newcommand{\progressreportnumber}{24}
\renewcommand{\author}{Erwin de Gelder}
\renewcommand{\date}{Novermber 11, 2019}
\renewcommand{\title}{Performance assessment of automated vehicles using real-world driving scenarios}

%%==========================================================================
%% Fancy headers and footers
%%==========================================================================
\pagestyle{fancy}                                       %% set page style
\fancyhf{}                                              %% clear all header & footer fields
\fancyhead[L]{Progress report \progressreportnumber}    %% define headers (LE: left field/even pages, etc.)
\fancyhead[R]{\author, \date}                           %% similar
\fancyfoot[C]{\thepage}                                 %% define footer

\begin{document}
	
\begin{center}
	\begin{tabular}{c}
		\title \\ \\
		\textbf{\huge Progress report \progressreportnumber} \\ \\
		\author \\ 
		\date
	\end{tabular}
\end{center}

\section{Previous meeting minutes}

\begin{itemize}
	\item During the previous meeting, we had a yearly review. The details regarding the review are in a separate report, so I will not elaborate in this progress report on this matter.
	\item I received feedback on the response letter for the ontology paper. 
	\item We discussed the outlook for publications and concluded the following:
	\begin{itemize}
		\item We aim for a conference paper to describe the assessment framework for autonomous vehicles (FISITA 2020). After this conference paper, we will judge whether it is feasible to also write a journal article on this subject. 
		\item We will write a conference paper to describe the mining of traffic scenarios (IV 2020). After this conference paper, we continue with a journal article on the \emph{scenario risk quantification}.
	\end{itemize}
\end{itemize}

\section{Summary of work}

\begin{itemize}
	\item I wrote an abstract for FISITA 2020 to propose a framework for the safety assessment of an autonomous vehicle using real-world scenarios. The abstract is attached to this report.
	\item I processed all comments regarding the response letter of the ontology paper. Furthermore, I applied all changes to the manuscript. I sent the updated response letter and the revised manuscript in a separate mail. Since I think another review is not necessary, I do not attached the response letter and the manuscript to this progress report.
	\item I worked on the code for the scenario mining. I also started writing the paper. Unfortunately, while writing the paper and trying to explain the method for scenario mining, I discovered some bugs in the code. Therefore, it does not make much sense to discuss what I wrote, because I do not even know if it works. I propose to discuss this during the next meeting.
\end{itemize}

\section{Future plans}

\begin{itemize}
	\item If there are no comments regarding the ontology paper, I want to submit the revised manuscript after the meeting.
	\item I want to submit the abstract for FISITA 2020 (deadline November 30).
	\item My main focus will be to get the scenario mining code running correctly and to compare the results with the ground truth. Here, the ``ground truth'' refers to manually annotated scenarios. Furthermore, I want to make some good progress with writing the corresponding conference paper.
\end{itemize}


\clearpage
\includepdf[pages=-,pagecommand={},width=\paperwidth]{"ABS191107 Abstract FISITA 2020".pdf}

\end{document}