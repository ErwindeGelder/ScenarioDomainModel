\documentclass[10pt,final,a4paper,oneside,onecolumn]{article}

%%==========================================================================
%% Packages
%%==========================================================================
\usepackage[a4paper,left=3.5cm,right=3.5cm,top=3cm,bottom=3cm]{geometry} %% change page layout; remove for IEEE paper format
\usepackage[T1]{fontenc}                        %% output font encoding for international characters (e.g., accented)
\usepackage[cmex10]{amsmath}                    %% math typesetting; consider using the [cmex10] option
\usepackage{amssymb}                            %% special (symbol) fonts for math typesetting
\usepackage{amsthm}                             %% theorem styles
\usepackage{dsfont}                             %% double stroke roman fonts: the real numbers R: $\mathds{R}$
\usepackage{mathrsfs}                           %% formal script fonts: the Laplace transform L: $\mathscr{L}$
\usepackage[pdftex]{graphicx}                   %% graphics control; use dvips for TeXify; use pdftex for PDFTeXify
\usepackage{array}                              %% array functionality (array, tabular)
\usepackage{upgreek}                            %% upright Greek letters; add the prefix 'up', e.g. \upphi
\usepackage{stfloats}                           %% improved handling of floats
\usepackage{multirow}                           %% cells spanning multiple rows in tables
%\usepackage{subfigure}                         %% subfigures and corresponding captions (for use with IEEEconf.cls)
\usepackage{subfig}                             %% subfigures (IEEEtran.cls: set caption=false)
\usepackage{fancyhdr}                           %% page headers and footers
\usepackage[official,left]{eurosym}             %% the euro symbol; command: \euro
\usepackage{appendix}                           %% appendix layout
\usepackage{xspace}                             %% add space after macro depending on context
\usepackage{verbatim}                           %% provides the comment environment
\usepackage[dutch,USenglish]{babel}             %% language support
\usepackage{wrapfig}                            %% wrapping text around figures
\usepackage{longtable}                          %% tables spanning multiple pages
\usepackage{pgfplots}                           %% support for TikZ figures (Matlab/Python)
\pgfplotsset{compat=1.14}						%% Run in backwards compatibility mode
\usepackage[breaklinks=true,hidelinks,          %% implement hyperlinks (dvips yields minor problems with breaklinks;
bookmarksnumbered=true]{hyperref}   %% IEEEtran: set bookmarks=false)
%\usepackage[hyphenbreaks]{breakurl}            %% allow line breaks in URLs (don't use with PDFTeX)
\usepackage[final]{pdfpages}                    %% Include other pdfs
\usepackage[capitalize]{cleveref}				%% Referensing to figures, equations, etc.
\usepackage{units}								%% Appropriate behavior of units
\usepackage[utf8]{inputenc}   				 	%% utf8 support (required for biblatex)
\usepackage{csquotes}							%% Quoted texts are typeset according to rules of main language
\usepackage[style=ieee,doi=false,isbn=false,url=false,date=year,minbibnames=15,maxbibnames=15,backend=biber]{biblatex}
%\renewcommand*{\bibfont}{\footnotesize}		%% Use this for papers
\setlength{\biblabelsep}{\labelsep}
\bibliography{../../bib}

%%==========================================================================
%% Define reference stuff
%%==========================================================================
\crefname{figure}{Figure}{Figures}
\crefname{equation}{}{}

%%==========================================================================
%% Define header/title stuff
%%==========================================================================
\newcommand{\progressreportnumber}{11}
\renewcommand{\author}{Erwin de Gelder}
\renewcommand{\date}{20 September 2018}
\renewcommand{\title}{Performance assessment of automated vehicles using real-world driving scenarios}

%%==========================================================================
%% Fancy headers and footers
%%==========================================================================
\pagestyle{fancy}                                       %% set page style
\fancyhf{}                                              %% clear all header & footer fields
\fancyhead[L]{Progress report \progressreportnumber}    %% define headers (LE: left field/even pages, etc.)
\fancyhead[R]{\author, \date}                           %% similar
\fancyfoot[C]{\thepage}                                 %% define footer

\begin{document}
	
\begin{center}
	\begin{tabular}{c}
		\title \\ \\
		\textbf{\huge Progress report \progressreportnumber} \\ \\
		\author \\ 
		\date
	\end{tabular}
\end{center}

\section{Previous meeting minutes}

\begin{itemize}
	\item Regarding the ontology, we discussed two different approaches. 
	\begin{itemize}
		\item For the first approach, the qualitative scenarios (e.g., describing scenarios in words) are considered as classes, whereas the quantitative scenarios (e.g., the actual real-world scenarios) are instances of these classes. A potential problem with this approach is that the number of classes is virtually infinite, so it will not be feasible to define all classes.
		\item For the second approach, two classes are defined: \emph{Qualitative scenario} and \emph{Scenario}. An instance of \emph{qualitative scenario} could than be a scenario description (e.g., ``lead vehicle braking''). The actual real-world scenarios are instances of \emph{Scenario}. A method needs to be defined to capture the relations between instances of \emph{Qualitative scenario} and instances of \emph{Scenario}. 
	\end{itemize}
	During the meeting, it is noted that the first approach makes more sense. The mentioned problem (that the number of classes is virtually infinite) should actually not be a problem. The first approach is also preferred because it is more intuitive.
	
	\item Regarding the quantification of completeness of a dataset, the discussion is simplified by only quantifying the uncertainty of the probability density functions (pdfs). It should be mentioned that due to this simplification, only a part of the original problem (i.e., quantification of completeness) is addressed.
	
	\item Some smaller comments, like why a certain assumption is made and whether it will be restrictive. Also, it would be good to repeat an experiment in order to see whether the results are consistent.
	
	\item The Akaike information criterion might be helpful.
	
	\item The Go/No Go meeting is planned for Tuesday the 2nd of October, from 9.00 till 10.30 (Dutch time). A first version of the Go/No Go report is to be provided by the next progress meeting, i.e., the 20th of September.
\end{itemize}

\section{Summary of work}

\begin{itemize}
	\item Due to a two-day workshop and a one-day symposium, I had no time to continue my work on the ontology and the quantification of completeness, so no further results are presented in this progress report.
	\item I wrote a report for the Go/No Go meeting which is planned on the 2nd of October. The report is attached.
\end{itemize}

%\section{Future plans}
%
%\begin{itemize}
%	\item 
%\end{itemize}

\section{Questions}

Regarding the Go/No Go report:
\begin{itemize}
	\item Are there items missing?
	\item How to proceed? E.g., should I send it directly to the committee after processing comments, or will there be another review?
\end{itemize}


%\printbibliography

\newpage
\includepdf[pages=-,pagecommand={},width=\paperwidth]{../../"20180917 GoNoGo"/GoNoGo.pdf}

\end{document}