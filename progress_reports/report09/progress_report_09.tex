\documentclass[10pt,final,a4paper,oneside,onecolumn]{article}

%%==========================================================================
%% Packages
%%==========================================================================
\usepackage[a4paper,left=3.5cm,right=3.5cm,top=3cm,bottom=3cm]{geometry} %% change page layout; remove for IEEE paper format
\usepackage[T1]{fontenc}                        %% output font encoding for international characters (e.g., accented)
\usepackage[cmex10]{amsmath}                    %% math typesetting; consider using the [cmex10] option
\usepackage{amssymb}                            %% special (symbol) fonts for math typesetting
\usepackage{amsthm}                             %% theorem styles
\usepackage{dsfont}                             %% double stroke roman fonts: the real numbers R: $\mathds{R}$
\usepackage{mathrsfs}                           %% formal script fonts: the Laplace transform L: $\mathscr{L}$
\usepackage[pdftex]{graphicx}                   %% graphics control; use dvips for TeXify; use pdftex for PDFTeXify
\usepackage{array}                              %% array functionality (array, tabular)
\usepackage{upgreek}                            %% upright Greek letters; add the prefix 'up', e.g. \upphi
\usepackage{stfloats}                           %% improved handling of floats
\usepackage{multirow}                           %% cells spanning multiple rows in tables
%\usepackage{subfigure}                         %% subfigures and corresponding captions (for use with IEEEconf.cls)
\usepackage{subfig}                             %% subfigures (IEEEtran.cls: set caption=false)
\usepackage{fancyhdr}                           %% page headers and footers
\usepackage[official,left]{eurosym}             %% the euro symbol; command: \euro
\usepackage{appendix}                           %% appendix layout
\usepackage{xspace}                             %% add space after macro depending on context
\usepackage{verbatim}                           %% provides the comment environment
\usepackage[dutch,USenglish]{babel}             %% language support
\usepackage{wrapfig}                            %% wrapping text around figures
\usepackage{longtable}                          %% tables spanning multiple pages
\usepackage{pgfplots}                           %% support for TikZ figures (Matlab/Python)
\pgfplotsset{compat=1.14}						%% Run in backwards compatibility mode
\usepackage[breaklinks=true,hidelinks,          %% implement hyperlinks (dvips yields minor problems with breaklinks;
bookmarksnumbered=true]{hyperref}   %% IEEEtran: set bookmarks=false)
%\usepackage[hyphenbreaks]{breakurl}            %% allow line breaks in URLs (don't use with PDFTeX)
\usepackage[final]{pdfpages}                    %% Include other pdfs
\usepackage[capitalize]{cleveref}				%% Referensing to figures, equations, etc.
\usepackage{units}								%% Appropriate behavior of units
\usepackage[utf8]{inputenc}   				 	%% utf8 support (required for biblatex)
\usepackage{csquotes}							%% Quoted texts are typeset according to rules of main language
\usepackage[style=ieee,doi=false,isbn=false,url=false,date=year,minbibnames=15,maxbibnames=15,backend=biber]{biblatex}
%\renewcommand*{\bibfont}{\footnotesize}		%% Use this for papers
\setlength{\biblabelsep}{\labelsep}
\bibliography{../../bib}

\usepackage{tabularx}
\newcolumntype{b}{X}
\newcolumntype{s}{>{\hsize=.2\hsize}X}
\newcommand{\heading}[1]{\multicolumn{1}{c}{#1}}

%%==========================================================================
%% Define reference stuff
%%==========================================================================
\crefname{figure}{Figure}{Figures}
\crefname{equation}{}{}

%%==========================================================================
%% Define header/title stuff
%%==========================================================================
\newcommand{\progressreportnumber}{9}
\renewcommand{\author}{Erwin de Gelder}
\renewcommand{\date}{3 July, 2018}
\renewcommand{\title}{Performance assessment of automated vehicles using real-world driving scenarios}

%%==========================================================================
%% Fancy headers and footers
%%==========================================================================
\pagestyle{fancy}                                       %% set page style
\fancyhf{}                                              %% clear all header & footer fields
\fancyhead[L]{Progress report \progressreportnumber}    %% define headers (LE: left field/even pages, etc.)
\fancyhead[R]{\author, \date}                           %% similar
\fancyfoot[C]{\thepage}                                 %% define footer

\begin{document}
	
\begin{center}
	\begin{tabular}{c}
		\title \\ \\
		\textbf{\huge Progress report \progressreportnumber} \\ \\
		\author \\ 
		\date
	\end{tabular}
\end{center}

\section{Previous meeting minutes}

\begin{itemize}
	\item We agreed on the scope of the first journal paper, which will be an extension of the submitted conference paper about the ontology regarding the scenarios for the assessment of automated vehicles. Some comments regarding this journal paper:
	\begin{itemize}
		\item It is fine to refer to other work like the TNO white paper and the report on the definition of several ``scenario classes''. However, the journal paper should be understandable without needing to read the other material, because the reviewers will most likely not look at these references. Hence, the paper should be stand-alone.
		\item My main focus will be on writing this journal paper. As a result, I will not work further on the quantification of the completeness.
	\end{itemize}
	\item Bart suggested Bernd Heidengott from the Vrije Universiteit Amsterdam for his expertise with statistics.
	\item For the mathematical notation (e.g., of the entropy of a stochastic variable), it is best to adopt the notation of a well-known reference.
	\item The Go/No Go meeting will be in or around September. A date will be set in August.
\end{itemize}

\section{Summary of work}

\begin{itemize}
	\item I looked at different journals that might be suitable for submitting the journal paper I want to write. \Cref{tab:journals} lists some suitable journals in alphabetical order.
\end{itemize}

\begin{table}[b]
	\caption{Possible journals for submitting the journal paper in alphabetical order.}
	\label{tab:journals}
	\begin{tabularx}{\textwidth}{bs}
		\hline
		\heading{Journal name} & \heading{Relevant references} \\ \hline
		IEEE Intelligent Transportation Systems Magazine & \cite{bengler2014threedecades, bertozzi2013vaic, kasper2012oobayesnetworks} \\
		IEEE Transactions on Intelligent Transportation Systems & \cite{Bonnin2014, morales2017cooperativeintersection, ploeg2017GCDC, zhao2018evaluation} \\
		IEEE Transactions on Intelligent Vehicles & \cite{wang2017much} \\
		IET Intelligent Transport Systems & \cite{geyer2014, wang2018compression} \\
		International Journal of Intelligent Transportation Systems Research & \cite{ebner2011identifying} \\
		Safety Science & \cite{khastgir2017towards, rae2017forecast} \\
		Transportation Research Part A: Policy and Practice & \cite{kalra2016driving, fagnant2015preparing} \\ \hline
	\end{tabularx}
\end{table}

\section{Future plans}

\begin{itemize}
	\item 
\end{itemize}

\section{Questions}

\begin{itemize}
	\item 
\end{itemize}


\printbibliography

\end{document}