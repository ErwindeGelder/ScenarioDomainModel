\documentclass[10pt,final,a4paper,oneside,onecolumn]{article}

%%==========================================================================
%% Packages
%%==========================================================================
\usepackage[a4paper,left=3.5cm,right=3.5cm,top=3cm,bottom=3cm]{geometry} %% change page layout; remove for IEEE paper format
\usepackage[T1]{fontenc}                        %% output font encoding for international characters (e.g., accented)
\usepackage[cmex10]{amsmath}                    %% math typesetting; consider using the [cmex10] option
\usepackage{amssymb}                            %% special (symbol) fonts for math typesetting
\usepackage{amsthm}                             %% theorem styles
\usepackage{dsfont}                             %% double stroke roman fonts: the real numbers R: $\mathds{R}$
\usepackage{mathrsfs}                           %% formal script fonts: the Laplace transform L: $\mathscr{L}$
\usepackage[pdftex]{graphicx}                   %% graphics control; use dvips for TeXify; use pdftex for PDFTeXify
\usepackage{array}                              %% array functionality (array, tabular)
\usepackage{upgreek}                            %% upright Greek letters; add the prefix 'up', e.g. \upphi
\usepackage[noadjust]{cite}                     %% citations; noadjust removes leading spaces
%\usepackage[round]{natbib}                     %% Author-year citations (remove package cite)
\usepackage{stfloats}                           %% improved handling of floats
\usepackage{multirow}                           %% cells spanning multiple rows in tables
%\usepackage{subfigure}                         %% subfigures and corresponding captions (for use with IEEEconf.cls)
\usepackage{subfig}                             %% subfigures (IEEEtran.cls: set caption=false)
\usepackage{fancyhdr}                           %% page headers and footers
\usepackage[official,left]{eurosym}             %% the euro symbol; command: \euro
\usepackage{appendix}                           %% appendix layout
\usepackage{xspace}                             %% add space after macro depending on context
\usepackage{verbatim}                           %% provides the comment environment
\usepackage[dutch,USenglish]{babel}             %% language support
\usepackage{wrapfig}                            %% wrapping text around figures
\usepackage{longtable}                          %% tables spanning multiple pages
\usepackage{pgfplots}                           %% support for TikZ figures (Matlab)
\usepackage[breaklinks=true,hidelinks,          %% implement hyperlinks (dvips yields minor problems with breaklinks;
bookmarksnumbered=true]{hyperref}   %% IEEEtran: set bookmarks=false)
%\usepackage[hyphenbreaks]{breakurl}            %% allow line breaks in URLs (don't use with PDFTeX)

%%==========================================================================
%% Define header/title stuff
%%==========================================================================
\newcommand{\progressreportnumber}{1}
\renewcommand{\author}{Erwin de Gelder}
\renewcommand{\date}{12 Oktober 2017}

%%==========================================================================
%% Fancy headers and footers
%%==========================================================================
\pagestyle{fancy}                                       %% set page style
\fancyhf{}                                              %% clear all header & footer fields
\fancyhead[L]{Progress report \progressreportnumber}    %% define headers (LE: left field/even pages, etc.)
\fancyhead[R]{\author, \date}                           %% similar
\fancyfoot[C]{\thepage}                                 %% define footer

\begin{document}
	
\begin{center}
	\begin{tabular}{c}
		\textbf{\huge Progress report \progressreportnumber} \\ \\
		\author \\ 
		\date
	\end{tabular}
\end{center}

\section*{Previous meeting minutes}

This is the first progress meeting. There was another meeting though which I did not attend. Jeroen Ploeg (TNO), Bart Vuijk (TNO) and Niels de Boer (NTU) were present. It has been discussed that the first topics to focus on are the following:
\begin{itemize}
	\item It should be clarified what the requirements are of the to-be-provided data [by BMW].
	\item The notions of events and scenarios need to be clarified.
\end{itemize}

\section*{Summary of work}

\begin{itemize}
	\item A summary is written to describe the topic of the PhD research.
	\item A document is written that states the requirements of the data \cite{data_requirements}.
	\item I thought about a good definition of an event. Important considerations for an `event' are:
	\begin{itemize}
		\item An event is an action by a single actor in a traffic scenario.
		\item We should be able to describe a full drive with `events'. As such, each time instance should belong to an event. For example, (steady-state) cruising is considered as event.
		\item The term \emph{action} is also used in literature when describing a traffic scenario \cite{scenario_ontology,ulbrich2015}. However, no further description of \emph{action} is given. Nevertheless, I think it would be good to have a formal definition.
		\item The different types of events should make sense. For example, braking or accelerating are regarded as events.
		\item I came up with the following definition:\\
		\emph{The action starts and ends with a change of a state of an actor.} \\
		For example: with a braking event, at the start of the action, the state longitudinal acceleration \emph{changes} from approximately 0 to negative. At the end of the action, the state longitudinal acceleration \emph{changes} from negative to approximately 0.
	\end{itemize}
	\item We had a discussion about the definition of the type of scenarios. More on this in a report \cite{categorizationScenarios}. I want to detail more about this is the meeting.
\end{itemize}

\section*{Future plans}
\begin{itemize}
	\item Hopefully we soon get access to data of CETRAN (Singapore). First step would be to transform the data to a format/structure we can work with. Next step is to extract events from the data.
	\item I want to finalize our definition of events and scenarios. This enables us to explain the notion of completeness on different levels (i.e. completeness of type of events/scenarios and completeness regarding the variability of a specific event/scenario). We (Jan-Pieter and I) plan to write a paper about this. I want to make a start on this (have scope and structure clear).
	\item I am working on a small example scenario where I want to show the steps from extracting events, parametrize them and be able to `reconstruct' the scenario.
\end{itemize}

\section*{Questions}
\begin{itemize}
	\item What would be a good formal description of an event? Is the current definition sufficient?
	\item I remember that I need to obtain some ECTS. We can use this meeting to detail about this.
\end{itemize}

\bibliographystyle{ieeetr}
\bibliography{../progress_reports_bib}

\end{document}