\documentclass[10pt,final,a4paper,oneside,onecolumn]{article}

%%==========================================================================
%% Packages
%%==========================================================================
\usepackage[a4paper,left=3.5cm,right=3.5cm,top=3cm,bottom=3cm]{geometry} %% change page layout; remove for IEEE paper format
\usepackage[T1]{fontenc}                        %% output font encoding for international characters (e.g., accented)
\usepackage[cmex10]{amsmath}                    %% math typesetting; consider using the [cmex10] option
\usepackage{amssymb}                            %% special (symbol) fonts for math typesetting
\usepackage{amsthm}                             %% theorem styles
\usepackage{dsfont}                             %% double stroke roman fonts: the real numbers R: $\mathds{R}$
\usepackage{mathrsfs}                           %% formal script fonts: the Laplace transform L: $\mathscr{L}$
\usepackage[pdftex]{graphicx}                   %% graphics control; use dvips for TeXify; use pdftex for PDFTeXify
\usepackage{array}                              %% array functionality (array, tabular)
\usepackage{upgreek}                            %% upright Greek letters; add the prefix 'up', e.g. \upphi
\usepackage{stfloats}                           %% improved handling of floats
\usepackage{multirow}                           %% cells spanning multiple rows in tables
%\usepackage{subfigure}                         %% subfigures and corresponding captions (for use with IEEEconf.cls)
\usepackage{subfig}                             %% subfigures (IEEEtran.cls: set caption=false)
\usepackage{fancyhdr}                           %% page headers and footers
\usepackage[official,left]{eurosym}             %% the euro symbol; command: \euro
\usepackage{appendix}                           %% appendix layout
\usepackage{xspace}                             %% add space after macro depending on context
\usepackage{verbatim}                           %% provides the comment environment
\usepackage[dutch,USenglish]{babel}             %% language support
\usepackage{wrapfig}                            %% wrapping text around figures
\usepackage{longtable}                          %% tables spanning multiple pages
\usepackage{pgfplots}                           %% support for TikZ figures (Matlab/Python)
\pgfplotsset{compat=1.14}						%% Run in backwards compatibility mode
\usepackage[breaklinks=true,hidelinks,          %% implement hyperlinks (dvips yields minor problems with breaklinks;
bookmarksnumbered=true]{hyperref}   %% IEEEtran: set bookmarks=false)
%\usepackage[hyphenbreaks]{breakurl}            %% allow line breaks in URLs (don't use with PDFTeX)
\usepackage[final]{pdfpages}                    %% Include other pdfs
\usepackage[capitalize]{cleveref}				%% Referensing to figures, equations, etc.
\usepackage{units}								%% Appropriate behavior of units
\usepackage[utf8]{inputenc}   				 	%% utf8 support (required for biblatex)
\usepackage{csquotes}							%% Quoted texts are typeset according to rules of main language
\usepackage[style=ieee,doi=false,isbn=false,url=false,date=year,minbibnames=15,maxbibnames=15,backend=biber]{biblatex}
%\renewcommand*{\bibfont}{\footnotesize}		%% Use this for papers
\setlength{\biblabelsep}{\labelsep}
\bibliography{../../bib}

%%==========================================================================
%% Define reference stuff
%%==========================================================================
\crefname{figure}{Figure}{Figures}
\crefname{equation}{}{}

%%==========================================================================
%% Define header/title stuff
%%==========================================================================
\newcommand{\progressreportnumber}{18}
\renewcommand{\author}{Erwin de Gelder}
\renewcommand{\date}{May 9, 2019}
\renewcommand{\title}{Performance assessment of automated vehicles using real-world driving scenarios}

%%==========================================================================
%% Fancy headers and footers
%%==========================================================================
\pagestyle{fancy}                                       %% set page style
\fancyhf{}                                              %% clear all header & footer fields
\fancyhead[L]{Progress report \progressreportnumber}    %% define headers (LE: left field/even pages, etc.)
\fancyhead[R]{\author, \date}                           %% similar
\fancyfoot[C]{\thepage}                                 %% define footer

\begin{document}
	
\begin{center}
	\begin{tabular}{c}
		\title \\ \\
		\textbf{\huge Progress report \progressreportnumber} \\ \\
		\author \\ 
		\date
	\end{tabular}
\end{center}

\section{Previous meeting minutes}

\begin{itemize}
	\item Main discussion was the ontology paper. Bart gave feedback on the whole paper.
\end{itemize}

\section{Summary of work}

\begin{itemize}
	\item I worked mainly on a new potential paper on the overall strategy for the assessment of ``autonomous'' vehicles. The objective of the paper is to present the overall assessment strategy that we apply in Singapore. I want to have the following outline:
	\begin{itemize}
		\item Introduction: Level 4 autonomous vehicles are to be deployed. Hence, safety needs to be assessed and current methods are not complete enough.
		\item Background on safety of AVs: Explain the different aspects of safety, including a literature review on existing assessment methods. 
		\item Nomenclature: explain certain terms in order to avoid ambiguity. 
		\item Proposed assessment strategy: explain the assessment strategy shortly. The different components of the assessment should be linked to the different aspects of safety (as explained in section 2). 
		\item Case study: Apply part of the method on a simple case of a hypothetical deployment of an AV. It will be too much to focus on all aspects of safety, so the focus will be on behavioral safety.
		\item Discussion and conclusions: Possible shortcomings and improvements.
	\end{itemize}
	\item I think it can potentially be a good paper, because it is the first (serious) attempt for safety validation of an SAE Level 4 automated vehicle, meaning that there is no safety driver as a backup. Possible journals are ``safety science'', ``transportation research part a: policy and practice'', and ``journal of safety research''. 
	\item I attached an initial version of the document. It is mainly a copy of part of the report that we are writing in Singapore (therefore UK English is used). For now, detailed feedback is not necessary. Instead, I would like to get feedback on whether the paper might be suitable at all for a conference/journal and if the outline makes sense.
	
	\item I received feedback from Bart on the whole paper. This is enough to have a complete draft of the paper in one/two weeks. Arash (a colleague at TNO and one of the co-authors) is a PhD candidate at the TU Eindhoven at the department of computer science. He will ask a (associate/assistant-)professor from that department to also review the paper.
\end{itemize}

\section{Future plans}

\begin{itemize}
	\item For the ontology paper, I have the following planning:
	\begin{itemize}
		\item May 24: Finish first draft and send paper for feedback to Arash and computer science professor from the TU Eindhoven (TUE).
		\item June 13: Get feedback from Arash and TUE prof.
		\item June 28: Finish second draft and send paper for feedback to Jan-Pieter, Olaf, Hala, and Bart.
		\item July 12: Get feedback from Jan-Pieter, Olaf, Hala, and Bart.
		\item July 25: Finish and submit paper.
	\end{itemize}
\end{itemize}

\section{Questions}

\begin{itemize}
	\item Would the paper on the assessment strategy be suitable for a conference/journal? Does the outline make sense?
	\item In the assessment paper, four different aspects of safety are mentioned, i.e., functional safety, cybersecurity, safety of the intended functionality, and behavioral safety. Is this too much? I.e., would it be better if we address on only one aspect of safety?
	\item Is the planning for the ontology paper OK?
\end{itemize}


%\printbibliography

\newpage
\includepdf[pages=-,pagecommand={},width=\paperwidth]{../../"20190505 Assessment Strategy"/assessment_strategy.pdf}

\end{document}