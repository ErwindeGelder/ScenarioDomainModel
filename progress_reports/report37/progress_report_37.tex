\documentclass[10pt,final,a4paper,oneside,onecolumn]{article}

%%==========================================================================
%% Packages
%%==========================================================================
\usepackage[a4paper,left=3.5cm,right=3.5cm,top=3cm,bottom=3cm]{geometry} %% change page layout; remove for IEEE paper format
\usepackage[T1]{fontenc}                        %% output font encoding for international characters (e.g., accented)
\usepackage[cmex10]{amsmath}                    %% math typesetting; consider using the [cmex10] option
\usepackage{amssymb}                            %% special (symbol) fonts for math typesetting
\usepackage{amsthm}                             %% theorem styles
\usepackage{dsfont}                             %% double stroke roman fonts: the real numbers R: $\mathds{R}$
\usepackage{mathrsfs}                           %% formal script fonts: the Laplace transform L: $\mathscr{L}$
\usepackage[pdftex]{graphicx}                   %% graphics control; use dvips for TeXify; use pdftex for PDFTeXify
\usepackage{array}                              %% array functionality (array, tabular)
\usepackage{upgreek}                            %% upright Greek letters; add the prefix 'up', e.g. \upphi
\usepackage{stfloats}                           %% improved handling of floats
\usepackage{multirow}                           %% cells spanning multiple rows in tables
%\usepackage{subfigure}                         %% subfigures and corresponding captions (for use with IEEEconf.cls)
\usepackage{subfig}                             %% subfigures (IEEEtran.cls: set caption=false)
\usepackage{fancyhdr}                           %% page headers and footers
\usepackage[official,left]{eurosym}             %% the euro symbol; command: \euro
\usepackage{appendix}                           %% appendix layout
\usepackage{xspace}                             %% add space after macro depending on context
\usepackage{verbatim}                           %% provides the comment environment
\usepackage[dutch,USenglish]{babel}             %% language support
\usepackage{wrapfig}                            %% wrapping text around figures
\usepackage{longtable}                          %% tables spanning multiple pages
\usepackage{pgfplots}                           %% support for TikZ figures (Matlab/Python)
\pgfplotsset{compat=1.14}						%% Run in backwards compatibility mode
\usepackage[breaklinks=true,hidelinks,          %% implement hyperlinks (dvips yields minor problems with breaklinks;
bookmarksnumbered=true]{hyperref}   %% IEEEtran: set bookmarks=false)
%\usepackage[hyphenbreaks]{breakurl}            %% allow line breaks in URLs (don't use with PDFTeX)
\usepackage[final]{pdfpages}                    %% Include other pdfs
\usepackage[capitalize]{cleveref}				%% Referensing to figures, equations, etc.
\usepackage{units}								%% Appropriate behavior of units
\usepackage[utf8]{inputenc}   				 	%% utf8 support (required for biblatex)
\usepackage{csquotes}							%% Quoted texts are typeset according to rules of main language
\usepackage[style=ieee,doi=false,isbn=false,url=false,date=year,minbibnames=15,maxbibnames=15,backend=biber]{biblatex}
%\renewcommand*{\bibfont}{\footnotesize}		%% Use this for papers
\setlength{\biblabelsep}{\labelsep}
\bibliography{../../bib}

%%==========================================================================
%% Define reference stuff
%%==========================================================================
\crefname{figure}{Figure}{Figures}
\crefname{equation}{}{}

%%==========================================================================
%% Define header/title stuff
%%==========================================================================
\newcommand{\progressreportnumber}{37}
\renewcommand{\author}{Erwin de Gelder}
\renewcommand{\date}{February 22, 2021}
\renewcommand{\title}{Performance assessment of automated vehicles using real-world driving scenarios}

%%==========================================================================
%% Fancy headers and footers
%%==========================================================================
\pagestyle{fancy}                                       %% set page style
\fancyhf{}                                              %% clear all header & footer fields
\fancyhead[L]{Progress report \progressreportnumber}    %% define headers (LE: left field/even pages, etc.)
\fancyhead[R]{\author, \date}                           %% similar
\fancyfoot[C]{\thepage}                                 %% define footer

\begin{document}
	
\begin{center}
	\begin{tabular}{c}
		\title \\ \\
		\textbf{\huge Progress report \progressreportnumber} \\ \\
		\author \\ 
		\date
	\end{tabular}
\end{center}

\section{Previous meeting minutes}

\begin{itemize}
	\item I received feedback from all of you on the paper ``Conditional Sampling from a Kernel Density Estimator to Generate Scenarios for the Assessment of Automated Vehicles''.
	\item I discussed with Jan-Pieter the latest results for the paper on ``test scenario generation''.
	\item I discussed with Olaf an initial draft of the paper on ``a surrogate safety metric for risk evaluation''. 
\end{itemize}

\section{Summary of work}

\begin{itemize}
	\item I updated the paper ``Conditional Sampling from a Kernel Density Estimator to Generate Scenarios for the Assessment of Automated Vehicles'' based on the comments and I submitted the resulting paper to the Intelligent Vehicles Symposium 2021. 
	
	\item Due to other TNO work, I have not been able to work much on the papers, so any update will be discussed during the next progress meeting. 
\end{itemize}

\section{Future plans}

In \cref{fig:planning}, the updated planning is shown. There are no changes compared to the planning shown in the previous progress report.

\begin{figure}[b]
	\centering
	\includegraphics[width=\linewidth]{planning.png}
	\caption{Proposed planning at the time of this report. The red line indicated the time when writing this report.}
	\label{fig:planning}
\end{figure}



\section{Questions}

\begin{itemize}
	\item With the current material I have for the paper I want to write on ``test scenario generation'', I think the paper will be considerably larger than 10 pages. The journal that I have in mind (Transactions on Intelligent Transportation Systems) advises to not have more than 10 pages. I see three options:
	\begin{enumerate}
		\item Try to shorten the paper by leaving out some results.
		\item Simply go for a long paper and pay the extra charge.
		\item Split the paper in two. The first part would be dedicated to the method, whereas the second part would be dedicated to the case study. Similarly as \autocite{feng2020testing, feng2020testing2}.
	\end{enumerate}
	What would be the preferred option?
\end{itemize}


\printbibliography

%\clearpage
%\includepdf[pages=-,pagecommand={},width=\paperwidth]{../../""/.pdf}

\end{document}