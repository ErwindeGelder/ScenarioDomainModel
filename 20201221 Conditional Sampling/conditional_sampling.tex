\documentclass{article}

\usepackage[utf8]{inputenc}
\usepackage{graphicx}
\usepackage{amsmath} % assumes amsmath package installed
\usepackage{amsthm}  % For special theorem style
\usepackage{amsfonts}
\usepackage{breqn}
\usepackage{dsfont}
\usepackage[dvipsnames]{xcolor}
\usepackage{tikz}
\usepackage[nolist,nohyperlinks]{acronym}
\usepackage[linesnumbered,ruled,vlined]{algorithm2e}
\usepackage[capitalize]{cleveref}

\usepackage[utf8]{inputenc}   				 	%% utf8 support (required for biblatex)
\usepackage{silence}  							%% For filtering warnings
\usepackage[style=ieee,doi=false,isbn=false,url=false,date=year,backend=biber,maxbibnames=15,maxcitenames=2,mincitenames=1,uniquelist=false,uniquename=false,giveninits=true]{biblatex}
% Filter warnings issued by package biblatex starting with "Patching footnotes failed"
\WarningFilter{biblatex}{Patching footnotes failed}
\renewcommand*{\bibfont}{\footnotesize}		%% Use this for papers
\renewcommand*{\bibfont}{\small}
\setlength{\biblabelsep}{\labelsep}
\bibliography{../bib}

\title{Conditional Sampling from a Kernel Density Estimator with a Gaussian Kernel}
\author{Erwin de Gelder}
\date{December 2020}

% Variables
\newcommand{\bandwidth}{h}
\newcommand{\bandwidthmatrix}{H}
  \newcommand{\bandwidthmatrixinverse}{\Lambda}
  \newcommand{\bwiul}{\bandwidthmatrixinverse_{11}}
  \newcommand{\bwiur}{\bandwidthmatrixinverse_{12}}
  \newcommand{\bwibl}{\bandwidthmatrixinverse_{21}}
  \newcommand{\bwibr}{\bandwidthmatrixinverse_{22}}
  \newcommand{\bandwidthmatrixrotated}{\tilde{\bandwidthmatrix}}
  \newcommand{\bandwidthmatrrotatedixinverse}{\tilde{\Lambda}}
  \newcommand{\bwriul}{\bandwidthmatrrotatedixinverse_{11}}
  \newcommand{\bwriur}{\bandwidthmatrrotatedixinverse_{12}}
  \newcommand{\bwribl}{\bandwidthmatrrotatedixinverse_{21}}
  \newcommand{\bwribr}{\bandwidthmatrrotatedixinverse_{22}}
\newcommand{\constraintmatrix}{A}
\newcommand{\constraintvector}{b}
\newcommand{\density}[1]{f\left( #1 \right)}
\newcommand{\densityest}[1]{\hat{f}\left( #1 \right)}
\newcommand{\densitycond}[2]{\density{ #1 | #2 }}
\newcommand{\densityestcond}[2]{\densityest{ #1 | #2 }}
\newcommand{\determinant}[1]{\left| #1 \right|}
\newcommand{\dimension}{d}
  \newcommand{\dimensionparta}{\dimension_\mathrm{c}}
  \newcommand{\dimensionpartb}{\dimension_\mathrm{u}}
\newcommand{\dummyvar}{u}
\newcommand{\e}[1]{\exp\left\{ #1 \right\}}
\newcommand{\identitymatrix}[1]{I_{#1}}
\newcommand{\indexdata}{i}
\newcommand{\indexsampling}{j}
\newcommand{\kernelfunc}[1]{K \left( #1 \right)}
\newcommand{\kernelfuncnormalized}[2]{K_{#1} \left( #2 \right)}
\newcommand{\normtwo}[1]{\left\Vert #1 \right\Vert}
\newcommand{\numberofconstraints}{n_\mathrm{c}}
\newcommand{\numberofsamples}{N}
\newcommand{\realnumbers}{\mathds{R}}
\newcommand{\superquad}{\phantom{=}\quad\quad\quad\quad}
\newcommand{\svdu}{U}
  \newcommand{\svds}{\Sigma}
  \newcommand{\svdv}{V}
  \newcommand{\svdva}{V_1}
  \newcommand{\svdvb}{V_2}
\newcommand{\ud}{\,\mathrm{d}}
\newcommand{\variable}{x}
  \newcommand{\variableparta}{\breve{\variable}}
  \newcommand{\variablepartb}{\hat{\variable}}
  \newcommand{\variableconstrained}{\bar{\variable}}
  \newcommand{\variableunconstrained}{\tilde{\variable}}
  \newcommand{\datapoint}[1]{\variable_{#1}}
  \newcommand{\datapointparta}[1]{\variableparta_{#1}}
  \newcommand{\datapointpartb}[1]{\variablepartb_{#1}}
  \newcommand{\datapointpartbtranslated}[1]{\datapointpartb{#1}'}
  \newcommand{\datapointconstrained}[1]{\variableconstrained_{#1}}
  \newcommand{\datapointunconstrained}[1]{\variableunconstrained_{#1}}
  \newcommand{\datapointunconstrainedtranslated}[1]{\datapointunconstrained{#1}'}
\newcommand{\weight}[1]{w_{#1}}

\begin{document}

% Acronyms
\begin{acronym}[AAAAAAAA]
	\acro{kde}[KDE]{Kernel Density Estimation}
	\acro{pdf}[pdf]{probability density function}
	\acro{svd}[SVD]{Singular Value Decomposition}\acroindefinite{svd}{an}{a}
\end{acronym}

\maketitle

\begin{abstract}

% Introduce surrogate safety metrics
\acp{ssm} are used to express road safety in terms of the safety risk in traffic conflicts.
% What is lacking?
\cstarta Typically, \acp{ssm} rely on assumptions on the future evolution of traffic participants to generate a measure of risk. \cenda
\cstartb As a result, they are only applicable in scenarios where those assumptions hold. \cendb

% Our approach
\cstarta To address this issues, we present a novel data-driven \ac{ourmethod} method. 
The \ac{ourmethod} method is used to derive \acp{ssm} that provide a probability of a specific event (e.g., collision) in the near future. 
Because we adopt a data-driven approach to predict the possible future evolutions of traffic participants, less assumptions are needed.
To do so, the \ac{ourmethod} method uses Monte Carlo simulations to estimate the occurrence probability of the specified event.
We further introduce a statistical method that requires fewer simulations to estimate this probability. 
Combined with a regression model, this enables our derived \acp{ssm} to make real-time risk estimations.

% Results
%Results show that the \ac{ourmethod} method is a generalization of existing probabilistic \acp{ssm}.
To illustrate the \ac{ourmethod} method, \iac{ssm} is derived for risk evaluation during longitudinal traffic interactions. \cenda
\cstartb Since there is no known method to objectively estimate risk from first principles, i.e., there is no known risk ground truth, it is very difficult, if not impossible, to objectively compare the relative merits of two \acp{ssm}. \cendb
\cstarta Instead, we provide a method for benchmarking our derived \ac{ssm} with respect to expected risk tendencies.
The application of the benchmarking illustrates that the \ac{ssm} matches the expected risk tendencies. \cenda

% Conclusions
Whereas the derived \ac{ssm} shows the potential of the \cstarta\ac{ourmethod} method\cenda, future work involves applying the approach for other types of traffic conflicts, such as lateral traffic conflicts or interactions with vulnerable road users.

\end{abstract}

\section{Introduction}
\label{sec:introduction}

% Safety is important
Road safety is an important key performance indicator in transportation. 
Due to the enormous societal and economy losses incurred in road accidents, road safety research is an important research topic.
For example, in 2018\cstarta\footnote{\cstarta At the time of writing, more recent results were not yet available.\cenda}\cenda, there were over 6.7 million accidents in the U.S.A.\ \autocite{nhtsa2020summary}, which is about 1.3 accidents per 1 million vehicle kilometers driven.
These accidents in 2018 led to 2.7 million injured people and 37 thousand fatalities \autocite{nhtsa2020summary}.
Furthermore, apart from these societal losses, the economic costs of all accidents in the U.S.A.\ in 2018 was 242 billion dollars \autocite{nhtsa2020summary}.
Similarly, the \textcite{eu2020roadsafety} reported over 22 thousand fatalities in 2019.

% Quantify safety with surrogate metrics
Road safety can be expressed in terms of injuries or fatalities per kilometer of driving, but a more proactive method for expressing road safety is the use of safety indicators that directly measure the safety risk in traffic conflicts.
%Particularly challenging is that ``there is no yet consensus on what `driving safely' means'' \autocite{tejada2020safe}.
Traffic conflicts are far more frequent than traffic accidents and the frequency of traffic conflicts can be used to predict the frequency of crashes.
\cstarta Therefore, these traffic conflicts are used to define so-called \acp{ssm} that characterize the risk of a collision or harm given an initial condition \autocite{arun2021systematic}. \cenda
% Examples
\acp{ssm} vary from measures that estimate the remaining time until a collision, such as the well-known \ac{ttc} \autocite{hayward1972near}, to metrics that estimate the probability that a human driver cannot avoid a collision, see, e.g., \autocite{wang2014evaluation}.

% Common approach
Typically, \acp{ssm} are only applicable in certain types of scenarios.
For example, the \ac{ttc} \autocite{hayward1972near}, which is the ratio of the distance toward and the speed difference with an approaching object, is only applicable when approaching an object.
Additionally, \acp{ssm} typically assume certain models for what the driver or system controlling the vehicle(s) is capable of and how the future --- given an initial condition --- will develop. 
For example, with the \ac{ttc} \autocite{hayward1972near}, it is assumed that the driver of a vehicle will not accelerate and that the surrounding traffic continues with the same speed. 
More complex \acp{ssm} consider, e.g., a human model that can react to a risky situation by braking \autocite{wang2014evaluation} or the uncertainty over the future ambient traffic state \autocite{mullakkal2020probabilistic}.
Regardless of the complexity of these models, however, these \acp{ssm} consider neither the specific capabilities of the driver or of the system controlling the vehicle, nor the local context for predicted the future of the vehicle's environment.  

% Our proposal
\cstarta In this paper, we present the \ac{ourmetric}: A novel data-driven approach for deriving \acp{ssm} that are not limited to certain types of scenarios.
Because our method does not rely on a predetermined model of the driver or of the system that controls the vehicle, the derived \acp{ssm} can be adapted to the situations in which the \acp{ssm} are applied. 
In addition, to not rely on assumptions on how the ambient traffic evolves over time, the \ac{ourmetric}  includes a data-driven approach for modeling the variations of how the ambient traffic progresses over time. 
Monte Carlo simulations are employed to accurately predict the safety risk given these variations.
To enable the real-time evaluation of the derived \acp{ssm}, we use the \ac{nw} kernel estimator \autocite{wasserman2006nonparametric} for local regression.
% Advantages of our method
The \ac{ourmetric} provides the following benefits: \cenda
\begin{itemize}
	\item The derived \acp{ssm} gives a probability that, e.g., a collision will happen in the near future. 
	A probability is easier to interpret than, e.g., a value ranging from 0 to infinity.
	
	\item \cstarta Next to deriving new \acp{ssm}, \cenda it is possible to reproduce already existing metrics that provide a probability. 
	Therefore, our approach can be seen as a generalization for deriving such existing \acp{ssm}.
	
	\item Any driver behavior model can be used.
	It is also possible to use a model of \iac{ads}, such that the derived \ac{ssm} estimates the safety risk if this \ac{ads} controls the vehicle.
	
	\item Because we use a data-driven approach, our \ac{ssm} adapts to the recorded data. 
	In this way, it is possible to \cstarta adapt the \ac{ssm} to, e.g., the local traffic behavior\cenda.
	
	\item Our approach can be applied to various scenarios.
\end{itemize}

% Explain case study.
To illustrate our method and the aforementioned benefits, the presented approach is applied in a case study.
The case study demonstrates that our method can be used to create other already available metrics, such as the \ac{ssm} presented in \autocite{wang2014evaluation}.
Furthermore, the \ac{ngsim} data set \autocite{kovvali2007video} is used to apply our data-driven approach to derive \iac{ssm} for longitudinal traffic conflicts.
The derived \ac{ssm} is applied in 3 different scenarios ranging from a safe scenario to a scenario with a collision.
These 3 scenarios show that the derived \ac{ssm} provides the risk of a collision.
The case study also presents a method to benchmark a \ac{ssm} using expected risk tendencies.
\cstarta We used this method to benchmark \iac{ssm} derived using the \ac{ourmetric} and the result shows that the derived \ac{ssm} complies with the expected risk tendencies. \cenda

% Structure
This article is organized as follows.
\cref{sec:literature review} provides an overview of \acp{ssm} described in the literature.
The proposed \ac{ourmetric} is presented in \cref{sec:method}.
In \cref{sec:case study}, we illustrate the method in a case study.
The article is concluded in \cref{sec:conclusions}.

\section{\acl{ourmetric}}
\label{sec:method}

In this section, we propose \cstarta\ac{ourmetric}: \cenda a method for deriving a metric that quantifies the risk of a certain event, such as a collision, in a particular situation in which a vehicle - hereafter, the \textit{ego vehicle} - is in and that is applicable for real-time use.
\cstarta The \ac{ourmetric} \cenda consists of four steps. 
The first step is the parameterization of the current situation and the possible future situations.
Second, based on the current situation, we estimate the probability (density) for the possible future situations. 
The third step includes determining the probability of the specified event based on the current and the future situations.
Finally, local regression is used to speed up the calculations and to make it possible to use the \ac{ssm} in real time. 
These four steps are described in the following subsections.

In the remainder of this article, the following notation is used. 
To denote a probability, $\probability{\cdot}$ is used. 
\Iac{pdf} is denoted by $\density{\cdot}$. 
The conditional probability $\probabilitycond{\dummyvara}{\dummyvarb}$ is verbalized as \textit{the probability of $\dummyvara$ given $\dummyvarb$}. 
Similarly, the conditional \ac{pdf} is denoted by $\densitycond{\cdot}{\cdot}$. 
To denote the estimation of any of the aforementioned quantities, a circumflex is used, e.g, $\probabilityest{\dummyvara}$ denotes the estimated probability of $\dummyvara$.



\subsection{Parameterize current and future situations}
\label{sec:parametrization}

The first step is to parameterize the current situation the ego vehicle is in. 
In other words, the current situation needs to be described using $\situationcurrentdim$ numbers that are stacked into one vector $\situationcurrent \in \situationcurrentspace \subseteq \realnumbers^{\situationcurrentdim}$. 
As an example, $\situationcurrent$ could contain the speed of the ego vehicle and the distance toward its preceding vehicle. 
In \cref{sec:case study}, we will consider more examples.

Next to describing the current situation, the future situation is described using $\situationfuturedim$ numbers stacked into one vector $\situationfuture \in \situationfuturespace \subseteq \realnumbers^{\situationfuturedim}$. 
Together with $\situationcurrent$, $\situationfuture$ contains enough information to describe how the relevant future, e.g., the next 5 seconds, around the ego vehicle develops over time. 
As an example, $\situationfuture$ could contain the speed for the next 5 seconds of the leading vehicle (if any) that is in front of the ego vehicle.

Let $\collision$ denote an event, e.g., a collision, such that the probability of this event is $\probability{\collision}$.
The goal of our \ac{ssm} is to estimate the probability of the event $\collision$ given a particular situation $\situationcurrent$, i.e., $\probabilitycond{\collision}{\situationcurrent}$.
We do this by considering all future situations, $\situationfuturespace$, and calculating the probability of a collision given each possible value of $\situationfuture$. 
Using integration, we obtain $\probabilitycond{\collision}{\situationcurrent}$:
\begin{equation}
	\label{eq:probability collision expectation}
	\probabilitycond{\collision}{\situationcurrent} 
	= \int_{\situationfuturespace} 
	\probabilitycond{\collision}{\situationcurrent, \situationfuture} 
	\densitycond{\situationfuture}{\situationcurrent} 
	\ud \situationfuture.
\end{equation}
%In \cref{sec:estimate future}, we propose a method to estimate $\densitycond{\situationfuture}{\situationcurrent}$ and in \cref{sec:estimate collision}, we propose a method to estimate $\probabilitycond{\collision}{\situationcurrent, \situationfuture}$.



\subsection{Estimate $\densitycond{\situationfuture}{\situationcurrent}$}
\label{sec:estimate future}

In this section, we propose a method to estimate $\densitycond{\situationfuture}{\situationcurrent}$, i.e., the \ac{pdf} of $\situationfuture$ given $\situationcurrent$.
Using the product rule for probability, we can write:
\begin{equation}
	\densitycond{\situationfuture}{\situationcurrent} 
	= \frac{\density{\situationcurrent, \situationfuture}}{\density{\situationcurrent}}
	= \frac{\density{\situationcurrent, \situationfuture}}{
		\int_{\situationfuturespace} \density{\situationcurrent, \situationfuture} \ud\situationfuture
	}.
\end{equation}
Thus, it suffices to estimate $\density{\situationcurrent, \situationfuture}$. 

Our proposal is to estimate $\density{\situationcurrent, \situationfuture}$ in a data-driven manner. 
A data-driven approach brings several benefits.
First, the estimate automatically adapts to local driving styles and behaviors, which can change from region to region, provided that the data are obtained from the same local traffic.
Second, assumptions such as a constant speed of other vehicles, are not needed.
For our data-driven approach, let us assume that we have obtained $\situationnumberof$ situations, denoted by $\situationcurrentinstance{\situationindex}\in\situationcurrentspace, \situationindex\in\{1,\ldots,\situationnumberof\}$, and their corresponding future situations described by $\situationfutureinstance{\situationindex}\in\situationfuturespace$.



\subsubsection{Special case: all parameters from one distribution}
\label{sec:one kde}

We first explain how to estimate $\density{\situationcurrent, \situationfuture}$ if we assume that all parameters depend on each other and that no further simplifications are possible. 
The shape of the \ac{pdf} $\density{\situationcurrent, \situationfuture}$ is unknown beforehand. 
Furthermore, the shape of the estimated \ac{pdf} might change as more data are acquired. 
Assuming a functional form of the \ac{pdf} and fitting the parameters of the \ac{pdf} to the data may therefore lead to inaccurate fits unless a lot of hand-tuning is applied.
We employ a non-parametric approach using \ac{kde} \autocite{rosenblatt1956remarks, parzen1962estimation} because the shape of the \ac{pdf} is then automatically computed and \ac{kde} is highly flexible regarding the shape of the \ac{pdf}. 
Using the \ac{kde}, the estimated \ac{pdf} becomes:
\begin{equation}
	\label{eq:kde estimate}
	\densityest{\situationcurrent,\situationfuture}
	= \frac{1}{\situationnumberof} \sum_{\situationindex=1}^{\situationnumberof}
	\kernelfuncnormalized{\bandwidthmatrix}{
		\begin{bmatrix}
			\situationcurrent \\
			\situationfuture
		\end{bmatrix} -
		\begin{bmatrix}
			\situationcurrentinstance{\situationindex} \\
			\situationfutureinstance{\situationindex}
		\end{bmatrix}
	},
\end{equation}
where $\kernelfuncnormalized{\bandwidthmatrix}{\cdot}$ is an appropriate kernel function with positive definite bandwidth matrix $\bandwidthmatrix$. 
The choice of the kernel $\kernelfuncnormalized{\bandwidthmatrix}{\cdot}$ is not as important as the choice of the bandwidth matrix $\bandwidthmatrix$ \autocite{turlach1993bandwidthselection}.
For example, the Gaussian kernel is given by \autocite{duong2007ks}
\begin{equation}
	\label{eq:kernel current future}
	\kernelfuncnormalized{\bandwidthmatrix}{\dummyvarkernel}
	= \frac{1}{\left( 2 \pi \right)^{\left( \situationcurrentdim + \situationfuturedim \right) / 2} 
	\left|\bandwidthmatrix\right|^{1/2} }
	\e{ -\frac{1}{2} \dummyvarkernel\transpose \bandwidthmatrix^{-1} \dummyvarkernel }.
\end{equation}

The bandwidth matrix $\bandwidthmatrix$ controls the width of the kernel, or, in other words, the influence of each sampling point on nearby regions. 
There are many different ways of estimating the bandwidth, ranging from simple reference rules like, e.g., Silverman's rule of thumb \autocite{silverman1986density} to more elaborate methods; see \autocite{turlach1993bandwidthselection, chiu1996comparative, jones1996brief, bashtannyk2001bandwidth, zambom2013review} for reviews of different bandwidth selection methods.
Here, we use one-leave-out cross-validation to compute the bandwidth because this minimizes the Kullback-Leibler divergence between the real \ac{pdf} $\density{\situationcurrent, \situationfuture}$ and the estimated \ac{pdf} $\densityest{\situationcurrent, \situationfuture}$ \autocite{turlach1993bandwidthselection, zambom2013review}.

Drawing samples from the \ac{kde} in \cref{eq:kde estimate} is straightforward: two random numbers are drawn, one to choose a random generator kernel out of the $\situationnumberof$ kernels that are used to construct the \ac{kde}, and one random number from that kernel.
Sampling from $\densityestcond{\situationfuture}{\situationcurrent}$ works similarly, but instead of using an equal probability for each random generator kernel to be selected, different probabilities are used based on $\situationcurrent$.
For more information on sampling from a conditional \ac{pdf} obtained using \ac{kde}, see \autocite{holmes2012fast, degelder2021conditional}.



\subsubsection{Not all parameters from one distribution}
\label{sec:no special case}

Due to the curse of dimensionality \autocite{scott2015multivariate}, estimating $\density{\situationcurrent, \situationfuture}$ with one \ac{kde} according to \cref{eq:kde estimate} becomes inaccurate if $\situationcurrentdim + \situationfuturedim$ becomes large.
There are a few ways to avoid this curse of dimensionality.
Without going into much detail, we list a few options.

One option is to assume that one or more parameters are independent of the other parameters. 
E.g., suppose that $\situationfuture=\begin{bmatrix}\situationfutureparta & \situationfuturepartb\end{bmatrix}\transpose$, such that $\situationfuturepartb$ is independent of $\situationcurrent$ and $\situationfutureparta$.
Then we can write
\begin{equation}
	\density{\situationcurrent, \situationfuture}
	= \density{\situationcurrent, \situationfutureparta, \situationfuturepartb}
	= \density{\situationcurrent, \situationfutureparta} \cdot \density{\situationfuturepartb}.
\end{equation}
In this case, we would need to estimate $\density{\situationcurrent, \situationfutureparta}$ and $\density{\situationfuturepartb}$, which can be done in a similar manner as presented in \cref{sec:one kde}.
Because these two \acp{pdf} have less variables than $\density{\situationcurrent, \situationfuture}$, the two estimated \acp{pdf} will suffer less from the curse of dimensionality \autocite{scott2015multivariate}.

Another option is to model $\densitycond{\situationfuture}{\situationcurrent}$ as a cascade of conditional probabilities. 
For example, using the partitioning $\situationfuture=\begin{bmatrix}\situationfutureparta & \situationfuturepartb\end{bmatrix}\transpose$, $\densitycond{\situationcurrent}{\situationfuture}$ can be approximated using two conditional densities:
\begin{equation}
	\densitycond{\situationfuture}{\situationcurrent}
	= \densitycond{\situationfutureparta, \situationfuturepartb}{\situationcurrent}
	= \densitycond{\situationfutureparta}{\situationfuturepartb, \situationcurrent} \cdot \densitycond{\situationfuturepartb}{\situationcurrent}
	\approx \densitycond{\situationfutureparta}{\situationfuturepartb} \cdot \densitycond{\situationfuturepartb}{\situationcurrent}.
\end{equation}
The same partitioning can be applied to $\densitycond{\situationfutureparta}{\situationfuturepartb}$ and $\densitycond{\situationfuturepartb}{\situationcurrent}$ until only two-dimensional \acp{pdf} need to be estimated.
\cstarta Although this will lead to larger approximation errors, the lower-dimensional \acp{pdf} can be better estimated. \cenda
For more information on this approach, we refer the reader to \autocite{aas2009paircopula, nagler2016evading}.



\subsubsection{Reduce number of parameters using \acl{svd}}
\label{sec:parameter reduction}

One way to avoid the curse of dimensionality is to use \iac{svd} \autocite{golub2013matrix} to reduce the number of parameters.
With \iac{svd}, the parameters $\situationcurrent$ and $\situationfuture$ are transformed into a lower-dimensional vector of parameters in such a way that the reduced vector of parameters describes as much of the variation as possible.
To do this, \iac{svd} is made of the matrix that contains all $\situationnumberof$ observed situations:
\begin{equation}
	\begin{bmatrix}
		\situationcurrentinstance{1}-\situationcurrentmean & \cdots & \situationcurrentinstance{\situationnumberof}-\situationcurrentmean \\
		\situationfutureinstance{1}-\situationfuturemean & \cdots & \situationfutureinstance{\situationnumberof}-\situationfuturemean
	\end{bmatrix} = \svdu \svds \svdv\transpose.
\end{equation}
Here, $\situationcurrentmean=\sum_{\situationindex=1}^{\situationnumberof}\situationcurrentinstance{\situationindex}$ and $\situationfuturemean=\sum_{\situationindex=1}^{\situationnumberof}\situationfutureinstance{\situationindex}$.
The matrices $\svdu \in \realnumbers^{\left(\situationcurrentdim+\situationfuturedim\right)\times\left(\situationcurrentdim+\situationfuturedim\right)}$ and $\svdv \in \realnumbers^{\situationnumberof \times \situationnumberof}$ are orthonormal, i.e., $\svdu^{-1}=\svdu\transpose$ and $\svdv^{-1}=\svdv\transpose$.
Moreover, $\svds\in\realnumbers^{\left(\situationcurrentdim+\situationfuturedim\right)\times\situationnumberof}$ has only zeros except at the diagonal: the $(\svdindex,\svdindex)$-th element is $\svdsv{\svdindex}$, $\svdindex\in\{1,\ldots,\svdrank\}$ with  $\svdrank=\min(\situationcurrentdim+\situationfuturedim, \situationnumberof)$, such that
\begin{equation}
	\svdsv{1} \geq \svdsv{2} \geq \ldots \geq \svdsv{\svdrank} \geq 0.
\end{equation}
Because these so-called singular values are in decreasing order, we can approximate $\situationcurrent$ and $\situationfuture$ by setting $\svdsv{\svdindex}=0$ for $\svdindex > \dimension$ with $\situationcurrentdim < \dimension < \situationcurrentdim+\situationfuturedim$:
\begin{equation}
	\label{eq:svd approximation}
	\begin{bmatrix}
		\situationcurrentinstance{\situationindex} - \situationcurrentmean \\
		\situationfutureinstance{\situationindex} - \situationfuturemean
	\end{bmatrix}
	= \sum_{\svdindex=1}^{\svdrank} \svdsv{\svdindex} \svdventry{\situationindex}{\svdindex} \svduvec{\svdindex}
	\approx \sum_{\svdindex=1}^{\dimension} \svdsv{\svdindex} \svdventry{\situationindex}{\svdindex} \svduvec{\svdindex},
	= \begin{bmatrix} \svduupperleft \\ \svdulowerleft \end{bmatrix} \svdsupperleft \svdvvecd{\situationindex},
\end{equation}
where $\svdventry{\situationindex}{\svdindex}$ is the $(\situationindex,\svdindex)$-th element of $\svdv$ and $\svduvec{\svdindex}$ is the $\svdindex$-th column of $\svdu$.
Moreover, $\svduupperleft$ is the $\situationcurrentdim$-by-$\dimension$ upper left submatrix of $\svdu$, $\svdulowerleft$ is the $\situationfuturedim$-by-$\dimension$ lower left submatrix $\svdu$, $\svdsupperleft\in\realnumbers^{\dimension\times\dimension}$ is the diagonal matrix with the first $\dimension$ singular values on its diagonal and $\svdvvecd{\situationindex}\transpose = \begin{bmatrix} \svdventry{\situationindex}{1} & \cdots & \svdventry{\situationindex}{\dimension}	\end{bmatrix}$.

We can estimate the probability density of $\svdvvecd{\situationindex}$ in a similar way as described in \cref{sec:one kde}.
To sample from $\densityestcond{\situationfuture}{\situationcurrent}$, we can sample from the estimated distribution of $\svdvvecd{\situationindex}$.
Because \cref{eq:svd approximation} is a linear mapping, the sample $\svdvvecsymbol$ that is drawn from the estimated distribution of $\svdvvecd{\situationindex}$ is subject to a linear constraint:
\begin{equation}
	\label{eq:linear constraint}
	\svduupperleft \svdsupperleft \svdvvecsymbol = \situationcurrent - \situationcurrentmean.
\end{equation}
In \autocite{degelder2021conditional}, an algorithm is provided for sampling from \iac{kde} with a Gaussian kernel of \cref{eq:kernel current future} such that the resulting samples are subject to a linear constraint like \cref{eq:linear constraint}.



\subsection{Estimate $\probabilitycond{\collision}{\situationcurrent}$ using a Monte Carlo simulation}
\label{sec:estimate collision}

\cstarta Monte Carlo simulations are used to estimate $\probabilitycond{\collision}{\situationcurrent}$, i.e., the probability of an event $\collision$ given the current situation $\situationcurrent$. \cenda
The details of the simulation depends on the actual application. 
For example, if the goal of our \ac{ssm} is to evaluate the risk that a human-driven vehicle collides, the simulation should involve human driver behavior models. 
On the other hand, if the goal is to evaluate the risk of collision when \iac{ads} is controlling the vehicle, the simulation should include the model of this \ac{ads}.

A straightforward way to compute $\probabilitycond{\collision}{\situationcurrent}$ is to repeat a certain number of simulations with the same $\situationcurrent$ and count the number of simulations that result in the event $\collision$.
If $\numberofsimulations$ denotes the number of simulations and $\numberofcollisions$ is the number of events $\collision$, then $\probabilitycond{\collision}{\situationcurrent}$ could be estimated using
\begin{equation}
	\label{eq:binomial estimation}
	\probabilityestcond{\collision}{\situationcurrent}
	= \frac{\numberofcollisions}{\numberofsimulations}.
\end{equation}

An important choice for estimating $\probabilitycond{\collision}{\situationcurrent}$ is the number of simulations, $\numberofsimulations$.
One approach is to keep increasing $\numberofsimulations$ until there is enough confidence in the estimation of \cref{eq:binomial estimation}.
E.g., the Clopper-Pearson interval \autocite{clopper1934use} or the Wilson score interval \autocite{wilson1927probable} can be used to determine the confidence of the estimation of \cref{eq:binomial estimation}.
A disadvantage of this approach is that only the fact whether the event $\collision$ occurred or not is used while the simulation provides more information. 
Therefore, we provide an alternative approach to estimate $\probabilitycond{\collision}{\situationcurrent}$.

\cstarta For the alternative approach, let us assume that one simulation run provides more information that just the fact that the event $\collision$ occurred or not.
Let $\simulationresult \in \realnumbers^{\dimsimulationresult}$ be a continuous variable representing the result of a simulation run and let $\spacecollision$ denote the set of possible simulation results in which the event $\collision$ occurred. 
Thus, \cenda $\simulationresult \in \spacecollision$ if and only if the simulation results in the event $\collision$.
Therefore, we have
\begin{equation}
	\probabilitycond{\collision}{\situationcurrent}
	= \probabilitycond{\simulationresult \in \spacecollision}{\situationcurrent}
	= \int_{\spacecollision} \densitycond{\simulationresult}{\situationcurrent} \ud \simulationresult.
\end{equation}
Similar as with the estimation of $\density{\situationcurrent, \situationfuture}$ in \cref{sec:estimate future}, we employ \ac{kde} to estimate $\densitycond{\simulationresult}{\situationcurrent}$:
\begin{equation}
	\label{eq:kde simulation result}
	\densityestcond{\simulationresult}{\situationcurrent}
	= \frac{1}{\numberofsimulations} 
	\sum_{\simulationindex=1}^{\numberofsimulations} \kernelfuncnormalized{\simulationbandwidth}{\simulationinstance{\simulationindex} - \simulationresult},
\end{equation}
where $\simulationinstance{\simulationindex}$ denotes the result of the $\simulationindex$-th simulation and $\simulationbandwidth$ denotes an appropriate bandwidth matrix.
The kernel function $\kernelfuncnormalized{\simulationbandwidth}{\cdot}$ is similarly defined as \cref{eq:kernel current future}.
We can now estimate $\probabilitycond{\collision}{\situationcurrent}$ by substituting $\densityestcond{\simulationresult}{\situationcurrent}$ of \cref{eq:kde simulation result} for $\densitycond{\simulationresult}{\situationcurrent}$:
\begin{equation}
	\label{eq:estimate probability of collision}
	\probabilityestcond{\collision}{\situationcurrent}
	= \probabilityestcond{\simulationresult \in \spacecollision}{\situationcurrent}
	= \int_{\spacecollision} \densityestcond{\simulationresult}{\situationcurrent} \ud \simulationresult
	=\frac{1}{\numberofsimulations}
	\sum_{\simulationindex=1}^{\numberofsimulations} \int_{\spacecollision}
	\kernelfuncnormalized{\simulationbandwidth}{\simulationinstance{\simulationindex} - \simulationresult} \ud \simulationresult.
\end{equation}

Similar as with \cref{eq:binomial estimation}, we need to choose the number of simulations $\numberofsimulations$.
Our proposal is to keep increasing $\numberofsimulations$ until the variance of $\probabilityestcond{\simulationresult \in \spacecollision}{\situationcurrent}$ is below a threshold $\simulationthreshold$.
The variance follows from \autocite{nadaraya1964some}:
\begin{equation}
	\variance{\probabilityestcond{\simulationresult \in \spacecollision}{\situationcurrent}}
	= \frac{\probabilitycond{\simulationresult \in \spacecollision}{\situationcurrent}
		\left( 1-\probabilitycond{\simulationresult \in \spacecollision}{\situationcurrent} \right)}{\numberofsimulations}.
\end{equation}
Because $\probabilitycond{\simulationresult \in \spacecollision}{\situationcurrent}$ is unknown, we use the estimated counterpart of \cref{eq:estimate probability of collision}.
Thus, $\numberofsimulations$ is increased until the following condition is met:
\begin{equation}
	\label{eq:condition stop simulations}
	\frac{\probabilityestcond{\simulationresult \in \spacecollision}{\situationcurrent}
		\left( 1-\probabilityestcond{\simulationresult \in \spacecollision}{\situationcurrent} \right)}{\numberofsimulations}
	< \simulationthreshold.
\end{equation}



\subsection{Regression for real-time estimation of $\probabilitycond{\collision}{\situationcurrent}$}
\label{sec:final metric calculation}

To evaluate the risk metric during real-time operation of the ego vehicle, the expression of \cref{eq:estimate probability of collision} is problematic, because it would require multiple $\numberofsimulations$ simulation.
Even if the calculation is accelerated using a technique like importance sampling, it might take too long.
Therefore, we propose to evaluate \cref{eq:estimate probability of collision} only for some fixed $\situationcurrentinstance{\situationindexdesign}$, $\situationindexdesign\in\{1,\ldots,\numberofdesignpoints\}$.
Next, local regression is used to estimate \cref{eq:estimate probability of collision}.
More specifically, we use the \ac{nw} kernel estimator \autocite{wasserman2006nonparametric}:
\begin{equation}
	\label{eq:nadaraya watson}
	\probabilityestcond{\collision}{\situationcurrent}
	\approx \frac{ \sum_{\situationindex=1}^{\numberofdesignpoints}
		\kernelfuncnormalized{\bandwidthnw}{\situationcurrent - \situationcurrentinstance{\situationindexdesign}}
		\probabilityestcond{\collision}{\situationcurrentinstance{\situationindexdesign}}
	}{\sum_{\situationindex=1}^{\numberofdesignpoints}
		\kernelfuncnormalized{\bandwidthnw}{\situationcurrent - \situationcurrentinstance{\situationindexdesign}}}.
\end{equation}
Here, $\probabilityestcond{\collision}{\situationcurrentinstance{\situationindexdesign}}$ is based \cref{eq:estimate probability of collision} and $\kernelfuncnormalized{\bandwidthnw}{\cdot}$ represents the Gaussian kernel given by \cref{eq:kernel current future}.
Two important choices have to be made: The choice of the $\situationcurrentinstance{\situationindexdesign}$, $\situationindexdesign\in\{1,\ldots,\numberofdesignpoints\}$ for which to evaluate \cref{eq:estimate probability of collision} and the choice of the bandwidth matrix $\bandwidthnw$.
We suggest to base the design points $\situationcurrentinstance{\situationindexdesign}$, $\situationindexdesign\in\{1,\ldots,\numberofdesignpoints\}$ on the data that is used to estimate $\densitycond{\situationfuture}{\situationcurrent}$ in \cref{sec:estimate future}, i.e., $\situationcurrentinstance{\situationindex}$, $\situationindex \in \{1,\ldots,\situationnumberof\}$, such that all $\situationcurrentinstance{\situationindex}$ have at least one design point $\situationcurrentinstance{\situationindexdesign}$  nearby.
In other words, $\situationcurrentinstance{\situationindexdesign}$, $\situationindexdesign\in\{1,\ldots,\numberofdesignpoints\}$ is chosen such that
\begin{equation}
	\label{eq:design points distance}
	\min_{\situationindexdesign} 
	\left( \situationcurrentinstance{\situationindex} - \situationcurrentinstance{\situationindexdesign} \right)\transpose
	\weightmatrix 
	\left( \situationcurrentinstance{\situationindex} - \situationcurrentinstance{\situationindexdesign} \right)
	\leq \distancedesignpoints^2,
	\quad \forall \situationindex \in \{1, \ldots, \situationnumberof\},
\end{equation}
where $\weightmatrix$ denotes a weighting matrix and $\distancedesignpoints$ denotes the maximum ``distance''. 
Note that if $\weightmatrix$ is the identity matrix, then \cref{eq:design points distance} calculates the minimum squared Euclidean distance.
\cstarta Choosing $\weightmatrix$ is a trade-off: If $\weightmatrix$ is too large, then too many details are lost in the approximation of \cref{eq:nadaraya watson}.
If $\weightmatrix$ is too small, it takes too long to evaluate \cref{eq:estimate probability of collision} $\numberofdesignpoints$ times, as $\numberofdesignpoints$ increases with decreasing $\weightmatrix$. \cenda
The bandwidth matrix $\bandwidthnw$ might be based on $\weightmatrix$, e.g., $\bandwidthnw=\weightmatrix^{-1}$.
Alternatively, $\bandwidthnw$ might be based on the measurement uncertainty of $\situationcurrent$, where a larger $\bandwidthnw$ applies in case of a larger measurement uncertainty of $\situationcurrent$.



\section{Application example}
\label{sec:example}

To illustrate the ontology, a real-world example is presented. A schematic overview of the example is shown in Fig.~\ref{fig:example schematic}. Two vehicles are considered: a pickup truck and a sedan. The example can be described in words as follows. The ego vehicle (i.e., the sedan) accelerates towards its predecessor (i.e., the pickup truck), such that the distance between the ego vehicle and its predecessor becomes smaller. At some point, the ego vehicle brakes, such that it reaches approximately the same speed as the pickup truck's speed. From then on, the ego vehicle cruises, such that the distance between the two vehicles remains approximately constant. The pickup truck drives at a constant speed.

\begin{figure}
	\centering
	\setlength\figureheight{121pt}
	\setlength\figurewidth{260pt}
	% This file was created by matplotlib2tikz v0.6.14.
\begin{tikzpicture}

\begin{axis}[
xmin=-15, xmax=15,
ymin=-5, ymax=5.3,
width=\figurewidth,
height=\figureheight,
tick align=outside,
tick pos=left,
x grid style={white!69.019607843137251!black},
y grid style={white!69.019607843137251!black},
axis background/.style={fill=white!90.0!black},
ticks=none,
hide axis
]
\path [draw=white, fill=white] (axis cs:-20,3.5)
--(axis cs:20,3.5)
--(axis cs:20,-3.5)
--(axis cs:-20,-3.5)
--cycle;

\addplot [semithick, black, forget plot]
table {%
-20 3.5
20 3.5
};
\addplot [semithick, black, forget plot]
table {%
-20 -3.5
20 -3.5
};
\addplot [semithick, black, forget plot]
table {%
-20 0
-15.5555555555556 0
};
\addplot [semithick, black, forget plot]
table {%
-6.66666666666666 0
-2.22222222222222 0
};
\addplot [semithick, black, forget plot]
table {%
6.66666666666667 0
11.1111111111111 0
};
\addplot [red, forget plot]
table {%
12.25 1.77571428571429
12.25 1.36428571428571
12.1323529411765 1.08142857142857
11.4852941176471 0.927142857142857
9.33823529411765 0.927142857142857
9.48529411764706 0.798571428571428
9.33823529411765 0.927142857142857
8.30882352941176 0.927142857142857
7.89705882352941 1.08142857142857
7.75 1.44142857142857
7.75 1.75
7.75 1.46714285714286
8.72058823529412 1.13285714285714
};
\addplot [red, forget plot]
table {%
12.25 1.72428571428571
12.25 2.13571428571429
12.1323529411765 2.41857142857143
11.4852941176471 2.57285714285714
9.33823529411765 2.57285714285714
9.48529411764706 2.70142857142857
9.33823529411765 2.57285714285714
8.30882352941176 2.57285714285714
7.89705882352941 2.41857142857143
7.75 2.05857142857143
7.75 1.75
7.75 2.03285714285714
8.72058823529412 2.36714285714286
};
\addplot [red, forget plot]
table {%
11.1617647058824 1.03
10.7794117647059 1.13285714285714
9.89705882352941 1.13285714285714
9.48529411764706 1.03
};
\addplot [red, forget plot]
table {%
11.1617647058824 2.47
10.7794117647059 2.36714285714286
9.89705882352941 2.36714285714286
9.48529411764706 2.47
};
\addplot [red, forget plot]
table {%
11.1617647058824 1.75
11.1617647058824 1.57
11.0441176470588 1.21
11.3970588235294 1.21
11.6323529411765 1.28714285714286
11.75 1.39
11.8382352941176 1.59571428571429
11.8382352941176 1.75
};
\addplot [red, forget plot]
table {%
11.1617647058824 1.75
11.1617647058824 1.93
11.0441176470588 2.29
11.3970588235294 2.29
11.6323529411765 2.21285714285714
11.75 2.11
11.8382352941176 1.90428571428571
11.8382352941176 1.75
};
\addplot [red, forget plot]
table {%
9.69117647058824 1.75
9.69117647058824 1.41571428571429
9.72058823529412 1.21
9.01470588235294 1.03
8.86764705882353 1.62142857142857
8.86764705882353 1.75
};
\addplot [red, forget plot]
table {%
9.69117647058824 1.75
9.69117647058824 2.08428571428571
9.72058823529412 2.29
9.01470588235294 2.47
8.86764705882353 1.87857142857143
8.86764705882353 1.75
};
\addplot [blue, forget plot]
table {%
-3.22093023255814 1.75
-3.22093023255814 1.73767123287671
-4.50290697674419 1.73767123287671
-4.50290697674419 1.75
};
\addplot [blue, forget plot]
table {%
-3.22093023255814 1.75
-3.22093023255814 1.76232876712329
-4.50290697674419 1.76232876712329
-4.50290697674419 1.75
};
\addplot [blue, forget plot]
table {%
-3.22093023255814 1.49109589041096
-4.50290697674419 1.49109589041096
-4.50290697674419 1.51575342465753
-3.22093023255814 1.51575342465753
-3.22093023255814 1.49109589041096
};
\addplot [blue, forget plot]
table {%
-3.22093023255814 2.00890410958904
-4.50290697674419 2.00890410958904
-4.50290697674419 1.98424657534247
-3.22093023255814 1.98424657534247
-3.22093023255814 2.00890410958904
};
\addplot [blue, forget plot]
table {%
-3.22093023255814 1.21986301369863
-4.50290697674419 1.21986301369863
-4.50290697674419 1.24452054794521
-3.22093023255814 1.24452054794521
-3.22093023255814 1.21986301369863
};
\addplot [blue, forget plot]
table {%
-3.22093023255814 2.28013698630137
-4.50290697674419 2.28013698630137
-4.50290697674419 2.25547945205479
-3.22093023255814 2.25547945205479
-3.22093023255814 2.28013698630137
};
\addplot [blue, forget plot]
table {%
-2.75 1.75
-2.75 1.07191780821918
-2.82848837209302 0.997945205479452
-2.9593023255814 0.997945205479452
-2.9593023255814 1.75
-2.9593023255814 0.923972602739726
-3.03779069767442 0.85
-5.99418604651163 0.874657534246575
-5.83720930232558 0.726712328767123
-5.99418604651163 0.874657534246575
-6.75290697674419 0.899315068493151
-7.09302325581395 1.02260273972603
-7.25 1.63904109589041
-7.25 1.75
};
\addplot [blue, forget plot]
table {%
-2.75 1.75
-2.75 2.42808219178082
-2.82848837209302 2.50205479452055
-2.9593023255814 2.50205479452055
-2.9593023255814 1.75
-2.9593023255814 2.57602739726027
-3.03779069767442 2.65
-5.99418604651163 2.62534246575343
-5.83720930232558 2.77328767123288
-5.99418604651163 2.62534246575343
-6.75290697674419 2.60068493150685
-7.09302325581395 2.47739726027397
-7.25 1.86095890410959
-7.25 1.75
};
\addplot [blue, forget plot]
table {%
-4.65988372093023 1.75
-4.65988372093023 0.997945205479452
-4.8953488372093 1.09657534246575
-4.8953488372093 1.75
};
\addplot [blue, forget plot]
table {%
-4.65988372093023 1.75
-4.65988372093023 2.50205479452055
-4.8953488372093 2.40342465753425
-4.8953488372093 1.75
};
\addplot [blue, forget plot]
table {%
-5.83720930232558 1.75
-5.83720930232558 1.44178082191781
-5.75872093023256 1.07191780821918
-6.28197674418605 0.973287671232877
-6.36046511627907 1.46643835616438
-6.36046511627907 1.75
};
\addplot [blue, forget plot]
table {%
-5.83720930232558 1.75
-5.83720930232558 2.05821917808219
-5.75872093023256 2.42808219178082
-6.28197674418605 2.52671232876712
-6.36046511627907 2.03356164383562
-6.36046511627907 1.75
};
\addplot [blue, forget plot]
table {%
-3.11627906976744 1.75
-3.11627906976744 1.02260273972603
-4.52906976744186 1.02260273972603
};
\addplot [blue, forget plot]
table {%
-3.11627906976744 1.75
-3.11627906976744 2.47739726027397
-4.52906976744186 2.47739726027397
};
\addplot [semithick, red, dashed, forget plot]
table {%
7.75 1.75
1.75 1.75
};
\addplot [semithick, red, forget plot]
table {%
2.5 2.5
1.75 1.75
2.5 1
};
\addplot [semithick, blue, dashed, forget plot]
table {%
-7.25 1.75
-11.25 1.75
};
\addplot [semithick, blue, forget plot]
table {%
-10.5 2.5
-11.25 1.75
-10.5 1
};
\node at (axis cs:10,5.3)[
  scale=0.8,
  anchor=north,
  text=black,
  rotate=0.0
]{ Sedan};
\node at (axis cs:-5,5.3)[
  scale=0.8,
  anchor=north,
  text=black,
  rotate=0.0
]{ Pickup truck};
\end{axis}

\end{tikzpicture}%
	\caption{Schematic overview of the traffic scenarios. The sedan (red, right vehicle) is defined as the ego vehicle. Initially, the ego vehicle accelerates towards the pickup truck (blue, left vehicle).}
	\label{fig:example schematic}
	\spaceaftercaption
\end{figure}

In the presented example, two scenarios are identified. The scenarios can be qualitatively described as `gap-closing on rural road with clear weather' and `cruising behind target vehicle on rural road with clear weather', respectively. The first scenario ends when the ego vehicle starts cruising. 
% Note that, when testing an AV, the first scenario might end later in case it takes longer for the AV to start the cruising activity.

The activities of the ego vehicle are shown in Figure~\ref{fig:example ego states}, in which the speed is shown with respect to time $t$. Regarding the speed, two events are identified. At the first event, the ego vehicle's longitudinal acceleration transits from being positive to negative, i.e., the ego vehicle starts braking. As there are two events identified regarding the speed, three different activities are to be modeled, qualitatively described as `accelerating', `braking', and `cruising', respectively. 

\cbstart
Whereas the definitions from Geyer~et~al.~\cite{geyer2014}, Ulbrich~et~al.~\cite{ulbrich2015}, and Elrofai~et~al.~\cite{elrofai2016scenario} do not state when a scenario ends, \cref{def:scenario} states that a scenario contains all the relevant events.
\cbend

\begin{figure}
	\centering
	\setlength\figureheight{130pt}
	\setlength\figurewidth{248pt}
	% This file was created by matplotlib2tikz v0.6.14.
\begin{tikzpicture}

\definecolor{color0}{rgb}{0.12156862745098,0.466666666666667,0.705882352941177}

\begin{axis}[
xlabel={$t$ [s]},
ylabel={Speed [km/h]},
xmin=0, xmax=17.5,
ymin=35, ymax=70,
width=\figurewidth,
height=\figureheight,
tick align=outside,
tick pos=left,
xmajorgrids,
x grid style={lightgray!92.026143790849673!black},
ymajorgrids,
y grid style={lightgray!92.026143790849673!black},
clip marker paths
]
\addplot [line width=2.4000000000000004pt, color0, forget plot]
table {%
0 37.6080419972848
0.025 37.7152844154791
0.05 37.822339777017
0.075 37.9292080818984
0.1 38.0358893301232
0.125 38.1423835216916
0.15 38.2486906566035
0.175 38.3548107348589
0.2 38.4607437564578
0.225 38.5664897214002
0.25 38.6720486296862
0.275 38.7774204813156
0.3 38.8826052762886
0.325 38.987603014605
0.35 39.092413696265
0.375 39.1970373212685
0.4 39.3014738896155
0.425 39.405723401306
0.45 39.50978585634
0.475 39.6136612547175
0.5 39.7173495964385
0.525 39.8208508815031
0.55 39.9241651099112
0.575 40.0272922816627
0.6 40.1302323967578
0.625 40.2329854551964
0.65 40.3355514569785
0.675 40.4379304021041
0.7 40.5401222905732
0.725 40.6421271223858
0.75 40.743944897542
0.775 40.8455756160416
0.8 40.9470192778848
0.825 41.0482758830714
0.85 41.1493454316016
0.875 41.2502279234753
0.9 41.3509233586925
0.925 41.4514317372532
0.95 41.5517530591574
0.975 41.6518873244051
1 41.7518345329964
1.025 41.8515946849311
1.05 41.9511677802094
1.075 42.0505538188312
1.1 42.1497528007964
1.125 42.2487647261052
1.15 42.3475895947575
1.175 42.4462274067534
1.2 42.5446781620927
1.225 42.6429418607755
1.25 42.7410185028019
1.275 42.8389080881717
1.3 42.9366106168851
1.325 43.0341260889419
1.35 43.1314545043423
1.375 43.2285958630862
1.4 43.3255501651736
1.425 43.4223174106045
1.45 43.518897599379
1.475 43.6152907314969
1.5 43.7114968069583
1.525 43.8075158257633
1.55 43.9033477879118
1.575 43.9989926934037
1.6 44.0944505422392
1.625 44.1897213344182
1.65 44.2848050699407
1.675 44.3797017488067
1.7 44.4744113710163
1.725 44.5689339365693
1.75 44.6632694454659
1.775 44.7574178977059
1.8 44.8513792932895
1.825 44.9451536322166
1.85 45.0387409144872
1.875 45.1321411401013
1.9 45.2253543090589
1.925 45.31838042136
1.95 45.4112194770046
1.975 45.5038714759928
2 45.5963364183244
2.025 45.6886143039996
2.05 45.7807051330182
2.075 45.8726089053804
2.1 45.9643256210861
2.125 46.0558552801353
2.15 46.147197882528
2.175 46.2383534282642
2.2 46.329321917344
2.225 46.4201033497672
2.25 46.510697725534
2.275 46.6011050446442
2.3 46.691325307098
2.325 46.7813585128953
2.35 46.8712046620361
2.375 46.9608637545204
2.4 47.0503357903482
2.425 47.1396207695195
2.45 47.2287186920344
2.475 47.3176295578927
2.5 47.4063533670946
2.525 47.4948901196399
2.55 47.5832398155288
2.575 47.6714024547612
2.6 47.7593780373371
2.625 47.8471665632565
2.65 47.9347680325194
2.675 48.0221824451259
2.7 48.1094098010758
2.725 48.1964501003692
2.75 48.2833033430062
2.775 48.3699695289867
2.8 48.4564486583106
2.825 48.5427407309781
2.85 48.6288457469891
2.875 48.7147637063436
2.9 48.8004946090417
2.925 48.8860384550832
2.95 48.9713952444682
2.975 49.0565649771968
3 49.1415476532689
3.025 49.2263432726844
3.05 49.3109518354435
3.075 49.3953733415461
3.1 49.4796077909922
3.125 49.5636551837818
3.15 49.6475155199149
3.175 49.7311887993916
3.2 49.8146750222117
3.225 49.8979741883754
3.25 49.9810862978825
3.275 50.0640113507332
3.3 50.1467493469274
3.325 50.2293002864651
3.35 50.3116641693463
3.375 50.393840995571
3.4 50.4758307651392
3.425 50.557633478051
3.45 50.6392491343062
3.475 50.720677733905
3.5 50.8019192768472
3.525 50.882973763133
3.55 50.9638411927623
3.575 51.0445215657351
3.6 51.1250148820514
3.625 51.2053211417112
3.65 51.2854403447146
3.675 51.3653724910614
3.7 51.4451175807517
3.725 51.5246756137856
3.75 51.604046590163
3.775 51.6832305098839
3.8 51.7622273729483
3.825 51.8410371793561
3.85 51.9196599291076
3.875 51.9980956222025
3.9 52.0763442586409
3.925 52.1544058384228
3.95 52.2322803615483
3.975 52.3099678280173
4 52.3874682378297
4.025 52.4647815909857
4.05 52.5419078874852
4.075 52.6188471273282
4.1 52.6955993105147
4.125 52.7721644370448
4.15 52.8485425069183
4.175 52.9247335201353
4.2 53.0007374766959
4.225 53.0765543766
4.25 53.1521842198475
4.275 53.2276270064386
4.3 53.3028827363732
4.325 53.3779514096513
4.35 53.4528330262729
4.375 53.5275275862381
4.4 53.6020350895467
4.425 53.6763555361988
4.45 53.7504889261945
4.475 53.8244352595337
4.5 53.8982111571136
4.525 53.9719163443169
4.55 54.0455674420409
4.575 54.1191644502856
4.6 54.1927073690509
4.625 54.2661961983369
4.65 54.3396309381435
4.675 54.4130115884708
4.7 54.4863381493188
4.725 54.5596106206874
4.75 54.6328290025767
4.775 54.7059932949866
4.8 54.7791034979172
4.825 54.8521596113684
4.85 54.9251616353404
4.875 54.9981095698329
4.9 55.0710034148462
4.925 55.1438431703801
4.95 55.2166288364346
4.975 55.2893604130098
5 55.3620379001057
5.025 55.4346612977223
5.05 55.5072306058594
5.075 55.5797458245173
5.1 55.6522069536958
5.125 55.724613993395
5.15 55.7969669436148
5.175 55.8692658043553
5.2 55.9415105756165
5.225 56.0137012573983
5.25 56.0858378497007
5.275 56.1579203525239
5.3 56.2299487658677
5.325 56.3019230897321
5.35 56.3738433241172
5.375 56.445709469023
5.4 56.5175215244494
5.425 56.5892794903965
5.45 56.6609833668643
5.475 56.7326331538527
5.5 56.8042288513618
5.525 56.8757704593915
5.55 56.9472579779419
5.575 57.0186914070129
5.6 57.0900707466046
5.625 57.161395996717
5.65 57.23266715735
5.675 57.3038842285037
5.7 57.3750472101781
5.725 57.4461561023731
5.75 57.5172109050887
5.775 57.5882116183251
5.8 57.6591582420821
5.825 57.7300507763597
5.85 57.800889221158
5.875 57.871673576477
5.9 57.9424038423166
5.925 58.0130800186769
5.95 58.0837021055578
5.975 58.1542701029594
6 58.2247840108817
6.025 58.2952438293246
6.05 58.3656495582882
6.075 58.4360011977725
6.1 58.5062987477774
6.125 58.5765422083029
6.15 58.6467315793492
6.175 58.716866860916
6.2 58.7869480530036
6.225 58.8569751556118
6.25 58.9269481687407
6.275 58.9968670923902
6.3 59.0667319265604
6.325 59.1365426712512
6.35 59.2062993264627
6.375 59.2760018921949
6.4 59.3456503684477
6.425 59.4152447552212
6.45 59.4847850525153
6.475 59.5542712603301
6.5 59.6237033786656
6.525 59.6930814075217
6.55 59.7624053468985
6.575 59.8316751967959
6.6 59.900890957214
6.625 59.9700526281528
6.65 60.0391602096122
6.675 60.1082137015923
6.7 60.177213104093
6.725 60.2461584171144
6.75 60.3150496406565
6.775 60.3838867747192
6.8 60.4526698193026
6.825 60.5213987744066
6.85 60.5900736400313
6.875 60.6586944161767
6.9 60.7272611028427
6.925 60.7957737000294
6.95 60.8642322077367
6.975 60.9326366259647
7 61.0009869547134
7.025 61.0692831939827
7.05 61.1375253437727
7.075 61.2057134040833
7.1 61.2738473749146
7.125 61.3419272562666
7.15 61.4099530481392
7.175 61.4779247505325
7.2 61.5458423634464
7.225 61.613705886881
7.25 61.6815153208363
7.275 61.7492706653122
7.3 61.8169719203087
7.325 61.884619085826
7.35 61.9522121618639
7.375 62.0197511484224
7.4 62.0872360455017
7.425 62.1546668531015
7.45 62.2220435712221
7.475 62.2893661998633
7.5 62.3566347390251
7.525 62.4238491887076
7.55 62.4910095489108
7.575 62.5581158196346
7.6 62.6251680008791
7.625 62.6921660926443
7.65 62.7591100949301
7.675 62.8260000077366
7.7 62.8928358310637
7.725 62.9596175649115
7.75 63.02634520928
7.775 63.0930187641691
7.8 63.1596382295788
7.825 63.2262036055093
7.85 63.2927148919603
7.875 63.3591720889321
7.9 63.4255751964245
7.925 63.4919242144376
7.95 63.5582191429713
7.975 63.6244599820257
8 63.6906467316008
8.025 63.7567793916965
8.05 63.8228579623129
8.075 63.8888824434499
8.1 63.9548528351076
8.125 64.0207691372859
8.15 64.0866313499849
8.175 64.1524394732046
8.2 64.2181935069449
8.225 64.2838934512059
8.25 64.3495393059876
8.275 64.4151310712899
8.3 64.4806687471128
8.325 64.5461523334565
8.35 64.6115818303207
8.375 64.6769572377057
8.4 64.7422785556113
8.425 64.8075457840376
8.45 64.8727589229845
8.475 64.9379179724521
8.5 65.0030229324403
8.525 65.0680738029492
8.55 65.1330705839788
8.575 65.198013275529
8.6 65.2629018775999
8.625 65.3277363901914
8.65 65.3925168133037
8.675 65.4572431469365
8.7 65.52191539109
8.725 65.5865335457642
8.75 65.6510976109591
8.775 65.7156075866746
8.8 65.7800634729108
8.825 65.8444652696676
8.85 65.9088129769451
8.875 65.9731065947432
8.9 66.037346123062
8.925 66.1015315619015
8.95 66.1656629112616
8.975 66.2297401711424
9 65.9163525027803
9.025 65.816737139492
9.05 65.7180682141708
9.075 65.6203457268167
9.1 65.5235696774297
9.125 65.4277400660097
9.15 65.3328568925569
9.175 65.2389201570712
9.2 65.1459298595526
9.225 65.053886000001
9.25 64.9627885784166
9.275 64.8726375947992
9.3 64.783433049149
9.325 64.6951749414658
9.35 64.6078632717497
9.375 64.5214980400008
9.4 64.4360792462189
9.425 64.3516068904041
9.45 64.2680809725564
9.475 64.1855014926758
9.5 64.1038684507623
9.525 64.0231818468159
9.55 63.9434416808365
9.575 63.8646479528243
9.6 63.7868006627792
9.625 63.7098998107011
9.65 63.6339453965902
9.675 63.5589374204463
9.7 63.4848758822695
9.725 63.4117607820599
9.75 63.3395921198173
9.775 63.2683698955418
9.8 63.1980941092334
9.825 63.1287647608922
9.85 63.0603818505179
9.875 62.9929453781108
9.9 62.9264553436708
9.925 62.8609117471979
9.95 62.7963145886921
9.975 62.7326638681533
10 62.6700920870255
10.025 62.6093942539702
10.05 62.5507028704314
10.075 62.4940179364089
10.1 62.4393394519028
10.125 62.3866674169131
10.15 62.3360018314397
10.175 62.2873426954828
10.2 62.2406900090422
10.225 62.196043772118
10.25 62.1534039847103
10.275 62.1127706468189
10.3 62.0741437584438
10.325 62.0375233195852
10.35 62.002909330243
10.375 61.9703017904171
10.4 61.9397007001076
10.425 61.9111060593145
10.45 61.8845178680378
10.475 61.8599361262775
10.5 61.8373608340335
10.525 61.816791991306
10.55 61.7982295980948
10.575 61.7816736544
10.6 61.7671241602216
10.625 61.7545811155596
10.65 61.744044520414
10.675 61.7355143747847
10.7 61.7289906786719
10.725 61.7244734320754
10.75 61.7219626349953
10.775 61.7214582874316
10.8 61.7229603893843
10.825 61.7264689408534
10.85 61.7319839418388
10.875 61.7395053923406
10.9 61.7490332923588
10.925 61.7605676418935
10.95 61.7741084409445
10.975 61.7896556895118
11 61.392702927607
11.025 61.392702927607
11.05 61.392702927607
11.075 61.392702927607
11.1 61.392702927607
11.125 61.392702927607
11.15 61.392702927607
11.175 61.392702927607
11.2 61.392702927607
11.225 61.392702927607
11.25 61.392702927607
11.275 61.392702927607
11.3 61.392702927607
11.325 61.392702927607
11.35 61.392702927607
11.375 61.392702927607
11.4 61.392702927607
11.425 61.392702927607
11.45 61.392702927607
11.475 61.392702927607
11.5 61.392702927607
11.525 61.392702927607
11.55 61.392702927607
11.575 61.392702927607
11.6 61.392702927607
11.625 61.392702927607
11.65 61.392702927607
11.675 61.392702927607
11.7 61.392702927607
11.725 61.392702927607
11.75 61.392702927607
11.775 61.392702927607
11.8 61.392702927607
11.825 61.392702927607
11.85 61.392702927607
11.875 61.392702927607
11.9 61.392702927607
11.925 61.392702927607
11.95 61.392702927607
11.975 61.392702927607
12 61.392702927607
12.025 61.392702927607
12.05 61.392702927607
12.075 61.392702927607
12.1 61.392702927607
12.125 61.392702927607
12.15 61.392702927607
12.175 61.392702927607
12.2 61.392702927607
12.225 61.392702927607
12.25 61.392702927607
12.275 61.392702927607
12.3 61.392702927607
12.325 61.392702927607
12.35 61.392702927607
12.375 61.392702927607
12.4 61.392702927607
12.425 61.392702927607
12.45 61.392702927607
12.475 61.392702927607
12.5 61.392702927607
12.525 61.392702927607
12.55 61.392702927607
12.575 61.392702927607
12.6 61.392702927607
12.625 61.392702927607
12.65 61.392702927607
12.675 61.392702927607
12.7 61.392702927607
12.725 61.392702927607
12.75 61.392702927607
12.775 61.392702927607
12.8 61.392702927607
12.825 61.392702927607
12.85 61.392702927607
12.875 61.392702927607
12.9 61.392702927607
12.925 61.392702927607
12.95 61.392702927607
12.975 61.392702927607
13 61.392702927607
13.025 61.392702927607
13.05 61.392702927607
13.075 61.392702927607
13.1 61.392702927607
13.125 61.392702927607
13.15 61.392702927607
13.175 61.392702927607
13.2 61.392702927607
13.225 61.392702927607
13.25 61.392702927607
13.275 61.392702927607
13.3 61.392702927607
13.325 61.392702927607
13.35 61.392702927607
13.375 61.392702927607
13.4 61.392702927607
13.425 61.392702927607
13.45 61.392702927607
13.475 61.392702927607
13.5 61.392702927607
13.525 61.392702927607
13.55 61.392702927607
13.575 61.392702927607
13.6 61.392702927607
13.625 61.392702927607
13.65 61.392702927607
13.675 61.392702927607
13.7 61.392702927607
13.725 61.392702927607
13.75 61.392702927607
13.775 61.392702927607
13.8 61.392702927607
13.825 61.392702927607
13.85 61.392702927607
13.875 61.392702927607
13.9 61.392702927607
13.925 61.392702927607
13.95 61.392702927607
13.975 61.392702927607
14 61.392702927607
14.025 61.392702927607
14.05 61.392702927607
14.075 61.392702927607
14.1 61.392702927607
14.125 61.392702927607
14.15 61.392702927607
14.175 61.392702927607
14.2 61.392702927607
14.225 61.392702927607
14.25 61.392702927607
14.275 61.392702927607
14.3 61.392702927607
14.325 61.392702927607
14.35 61.392702927607
14.375 61.392702927607
14.4 61.392702927607
14.425 61.392702927607
14.45 61.392702927607
14.475 61.392702927607
14.5 61.392702927607
14.525 61.392702927607
14.55 61.392702927607
14.575 61.392702927607
14.6 61.392702927607
14.625 61.392702927607
14.65 61.392702927607
14.675 61.392702927607
14.7 61.392702927607
14.725 61.392702927607
14.75 61.392702927607
14.775 61.392702927607
14.8 61.392702927607
14.825 61.392702927607
14.85 61.392702927607
14.875 61.392702927607
14.9 61.392702927607
14.925 61.392702927607
14.95 61.392702927607
14.975 61.392702927607
15 61.392702927607
15.025 61.392702927607
15.05 61.392702927607
15.075 61.392702927607
15.1 61.392702927607
15.125 61.392702927607
15.15 61.392702927607
15.175 61.392702927607
15.2 61.392702927607
15.225 61.392702927607
15.25 61.392702927607
15.275 61.392702927607
15.3 61.392702927607
15.325 61.392702927607
15.35 61.392702927607
15.375 61.392702927607
15.4 61.392702927607
15.425 61.392702927607
15.45 61.392702927607
15.475 61.392702927607
15.5 61.392702927607
15.525 61.392702927607
15.55 61.392702927607
15.575 61.392702927607
15.6 61.392702927607
15.625 61.392702927607
15.65 61.392702927607
15.675 61.392702927607
15.7 61.392702927607
15.725 61.392702927607
15.75 61.392702927607
15.775 61.392702927607
15.8 61.392702927607
15.825 61.392702927607
15.85 61.392702927607
15.875 61.392702927607
15.9 61.392702927607
15.925 61.392702927607
15.95 61.392702927607
15.975 61.392702927607
16 61.392702927607
16.025 61.392702927607
16.05 61.392702927607
16.075 61.392702927607
16.1 61.392702927607
16.125 61.392702927607
16.15 61.392702927607
16.175 61.392702927607
16.2 61.392702927607
16.225 61.392702927607
16.25 61.392702927607
16.275 61.392702927607
16.3 61.392702927607
16.325 61.392702927607
16.35 61.392702927607
16.375 61.392702927607
16.4 61.392702927607
16.425 61.392702927607
16.45 61.392702927607
16.475 61.392702927607
16.5 61.392702927607
16.525 61.392702927607
16.55 61.392702927607
16.575 61.392702927607
16.6 61.392702927607
16.625 61.392702927607
16.65 61.392702927607
16.675 61.392702927607
16.7 61.392702927607
16.725 61.392702927607
16.75 61.392702927607
16.775 61.392702927607
16.8 61.392702927607
16.825 61.392702927607
16.85 61.392702927607
16.875 61.392702927607
16.9 61.392702927607
16.925 61.392702927607
16.95 61.392702927607
16.975 61.392702927607
17 61.392702927607
17.025 61.392702927607
17.05 61.392702927607
17.075 61.392702927607
17.1 61.392702927607
17.125 61.392702927607
17.15 61.392702927607
17.175 61.392702927607
17.2 61.392702927607
17.225 61.392702927607
17.25 61.392702927607
17.275 61.392702927607
17.3 61.392702927607
17.325 61.392702927607
17.35 61.392702927607
17.375 61.392702927607
17.4 61.392702927607
17.425 61.392702927607
17.45 61.392702927607
17.475 61.392702927607
17.5 61.392702927607
17.525 61.392702927607
17.55 61.392702927607
17.575 61.392702927607
17.6 61.392702927607
17.625 61.392702927607
17.65 61.392702927607
17.675 61.392702927607
17.7 61.392702927607
17.725 61.392702927607
17.75 61.392702927607
17.775 61.392702927607
17.8 61.392702927607
17.825 61.392702927607
17.85 61.392702927607
17.875 61.392702927607
17.9 61.392702927607
17.925 61.392702927607
17.95 61.392702927607
17.975 61.392702927607
18 61.7326653372238
18.025 61.8311539114564
18.05 61.9289491671942
18.075 62.026051104437
18.1 62.1224597231851
18.125 62.2181750234382
18.15 62.3131970051965
18.175 62.4075256684599
18.2 62.5011610132285
18.225 62.5941030395022
18.25 62.686351747281
18.275 62.7779071365649
18.3 62.868769207354
18.325 62.9589379596482
18.35 63.0484133934476
18.375 63.1371955087521
18.4 63.2252843055617
18.425 63.3126797838764
18.45 63.3993819436963
18.475 63.4853907850213
18.5 63.5707063078515
18.525 63.6553285121868
18.55 63.7392573980272
18.575 63.8224929653727
18.6 63.9050352142234
18.625 63.9868841445793
18.65 64.0680397564402
18.675 64.1485020498063
18.7 64.2282710246775
18.725 64.3073466810539
18.75 64.3857290189354
18.775 64.463418038322
18.8 64.5404137392137
18.825 64.6167161216106
18.85 64.6923251855127
18.875 64.7672409309198
18.9 64.8414633578321
18.925 64.9149924662495
18.95 64.9878282561721
18.975 65.0599707275998
19 65.1314198805326
19.025 65.2021757149706
19.05 65.2722382309137
19.075 65.3416074283619
19.1 65.4102833073152
19.125 65.4782658677737
19.15 65.5455551097374
19.175 65.6121510332061
19.2 65.67805363818
19.225 65.743262924659
19.25 65.8077788926432
19.275 65.8716015421325
19.3 65.9347308731269
19.325 65.9971668856265
19.35 66.0589095796312
19.375 66.119958955141
19.4 66.180315012156
19.425 66.239977750676
19.45 66.2989471707013
19.475 66.3572232722316
19.5 66.4148060552671
19.525 66.4716955198078
19.55 66.5278916658535
19.575 66.5833944934044
19.6 66.6382040024605
19.625 66.6923201930216
19.65 66.7457430650879
19.675 66.7984726186594
19.7 66.8505088537359
19.725 66.9018517703176
19.75 66.9525013684045
19.775 67.0024576479964
19.8 67.0517206090935
19.825 67.1002902516958
19.85 67.1481665758031
19.875 67.1953495814156
19.9 67.2418392685333
19.925 67.287635637156
19.95 67.3327386872839
19.975 67.3771484189169
20 67.4208648320551
20.025 67.4638879266984
20.05 67.5062177028469
20.075 67.5478541605004
20.1 67.5887972996591
20.125 67.629047120323
20.15 67.6686036224919
20.175 67.707466806166
20.2 67.7456366713453
20.225 67.7831132180296
20.25 67.8198964462192
20.275 67.8559863559138
20.3 67.8913829471136
20.325 67.9260862198185
20.35 67.9600961740285
20.375 67.9934128097437
20.4 68.026036126964
20.425 68.0579661256894
20.45 68.08920280592
20.475 68.1197461676557
20.5 68.1496431270512
20.525 68.1791751810337
20.55 68.2083892457577
20.575 68.2372853212234
20.6 68.2658634074307
20.625 68.2941235043795
20.65 68.32206561207
20.675 68.3496897305021
20.7 68.3769958596758
20.725 68.4039839995912
20.75 68.4306541502481
20.775 68.4570063116466
20.8 68.4830404837868
20.825 68.5087566666685
20.85 68.5341548602919
20.875 68.5592350646568
20.9 68.5839972797634
20.925 68.6084415056116
20.95 68.6325677422014
20.975 68.6563759895328
21 68.6798662476058
21.025 68.7030385164204
21.05 68.7258927959766
21.075 68.7484290862745
21.1 68.7706473873139
21.125 68.792547699095
21.15 68.8141300216176
21.175 68.8353943548819
21.2 68.8563406988878
21.225 68.8769690536352
21.25 68.8972794191243
21.275 68.917271795355
21.3 68.9369461823274
21.325 68.9563025800413
21.35 68.9753409884968
21.375 68.9940614076939
21.4 69.0124638376327
21.425 69.030548278313
21.45 69.048314729735
21.475 69.0657631918986
21.5 69.0828936648037
21.525 69.0997061484505
21.55 69.1162006428389
21.575 69.1323771479689
21.6 69.1482356638405
21.625 69.1637761904538
21.65 69.1789987278086
21.675 69.193903275905
21.7 69.2084898347431
21.725 69.2227584043227
21.75 69.236708984644
21.775 69.2503415757069
21.8 69.2636561775114
21.825 69.2766527900574
21.85 69.2893314133451
21.875 69.3016920473744
21.9 69.3137346921453
21.925 69.3254593476579
21.95 69.336866013912
21.975 69.3479546909077
22 69.3587253786451
22.025 69.369178077124
22.05 69.3793127863446
22.075 69.3891295063068
22.1 69.3986282370105
22.125 69.4078089784559
22.15 69.4166717306429
22.175 69.4252164935716
22.2 69.4334432672418
22.225 69.4413520516536
22.25 69.448942846807
22.275 69.4562156527021
22.3 69.4631704693387
22.325 69.469807296717
22.35 69.4761261348368
22.375 69.4821269836983
22.4 69.4878098433014
22.425 69.4931747136461
22.45 69.4982215947324
22.475 69.5029504865603
22.5 69.5073613891298
22.525 69.5114543024409
22.55 69.5152292264937
22.575 69.518686161288
22.6 69.521825106824
22.625 69.5246460631015
22.65 69.5271490301207
22.675 69.5293340078815
22.7 69.5312009963838
22.725 69.5327499956278
22.75 69.5339810056134
22.775 69.5348940263407
22.8 69.5354890578095
22.825 69.5357661000199
22.85 69.5357251529719
22.875 69.5353662166656
22.9 69.5346892911008
22.925 69.5336943762777
22.95 69.5323814721961
22.975 69.5307505788562
23 69.935051266673
23.025 69.935051266673
23.05 69.935051266673
23.075 69.935051266673
23.1 69.935051266673
23.125 69.935051266673
23.15 69.935051266673
23.175 69.935051266673
23.2 69.935051266673
23.225 69.935051266673
23.25 69.935051266673
23.275 69.935051266673
23.3 69.935051266673
23.325 69.935051266673
23.35 69.935051266673
23.375 69.935051266673
23.4 69.935051266673
23.425 69.935051266673
23.45 69.935051266673
23.475 69.935051266673
23.5 69.935051266673
23.525 69.935051266673
23.55 69.935051266673
23.575 69.935051266673
23.6 69.935051266673
23.625 69.935051266673
23.65 69.935051266673
23.675 69.935051266673
23.7 69.935051266673
23.725 69.935051266673
23.75 69.935051266673
23.775 69.935051266673
23.8 69.935051266673
23.825 69.935051266673
23.85 69.935051266673
23.875 69.935051266673
23.9 69.935051266673
23.925 69.935051266673
23.95 69.935051266673
23.975 69.935051266673
24 69.935051266673
24.025 69.935051266673
24.05 69.935051266673
24.075 69.935051266673
24.1 69.935051266673
24.125 69.935051266673
24.15 69.935051266673
24.175 69.935051266673
24.2 69.935051266673
24.225 69.935051266673
24.25 69.935051266673
24.275 69.935051266673
24.3 69.935051266673
24.325 69.935051266673
24.35 69.935051266673
24.375 69.935051266673
24.4 69.935051266673
24.425 69.935051266673
24.45 69.935051266673
24.475 69.935051266673
24.5 69.935051266673
24.525 69.935051266673
24.55 69.935051266673
24.575 69.935051266673
24.6 69.935051266673
24.625 69.935051266673
24.65 69.935051266673
24.675 69.935051266673
24.7 69.935051266673
24.725 69.935051266673
24.75 69.935051266673
24.775 69.935051266673
24.8 69.935051266673
24.825 69.935051266673
24.85 69.935051266673
24.875 69.935051266673
24.9 69.935051266673
24.925 69.935051266673
24.95 69.935051266673
24.975 69.935051266673
25 69.935051266673
25.025 69.935051266673
25.05 69.935051266673
25.075 69.935051266673
25.1 69.935051266673
25.125 69.935051266673
25.15 69.935051266673
25.175 69.935051266673
25.2 69.935051266673
25.225 69.935051266673
25.25 69.935051266673
25.275 69.935051266673
25.3 69.935051266673
25.325 69.935051266673
25.35 69.935051266673
25.375 69.935051266673
25.4 69.935051266673
25.425 69.935051266673
25.45 69.935051266673
25.475 69.935051266673
25.5 69.935051266673
25.525 69.935051266673
25.55 69.935051266673
25.575 69.935051266673
25.6 69.935051266673
25.625 69.935051266673
25.65 69.935051266673
25.675 69.935051266673
25.7 69.935051266673
25.725 69.935051266673
25.75 69.935051266673
25.775 69.935051266673
25.8 69.935051266673
25.825 69.935051266673
25.85 69.935051266673
25.875 69.935051266673
25.9 69.935051266673
25.925 69.935051266673
25.95 69.935051266673
25.975 69.935051266673
26 69.935051266673
26.025 69.935051266673
26.05 69.935051266673
26.075 69.935051266673
26.1 69.935051266673
26.125 69.935051266673
26.15 69.935051266673
26.175 69.935051266673
26.2 69.935051266673
26.225 69.935051266673
26.25 69.935051266673
26.275 69.935051266673
26.3 69.935051266673
26.325 69.935051266673
26.35 69.935051266673
26.375 69.935051266673
26.4 69.935051266673
26.425 69.935051266673
26.45 69.935051266673
26.475 69.935051266673
26.5 69.935051266673
26.525 69.935051266673
26.55 69.935051266673
26.575 69.935051266673
26.6 69.935051266673
26.625 69.935051266673
26.65 69.935051266673
26.675 69.935051266673
26.7 69.935051266673
26.725 69.935051266673
26.75 69.935051266673
26.775 69.935051266673
26.8 69.935051266673
26.825 69.935051266673
26.85 69.935051266673
26.875 69.935051266673
26.9 69.935051266673
26.925 69.935051266673
26.95 69.935051266673
26.975 69.935051266673
27 69.935051266673
27.025 69.935051266673
27.05 69.935051266673
27.075 69.935051266673
27.1 69.935051266673
27.125 69.935051266673
27.15 69.935051266673
27.175 69.935051266673
27.2 69.935051266673
27.225 69.935051266673
27.25 69.935051266673
27.275 69.935051266673
27.3 69.935051266673
27.325 69.935051266673
27.35 69.935051266673
27.375 69.935051266673
27.4 69.935051266673
27.425 69.935051266673
27.45 69.935051266673
27.475 69.935051266673
27.5 69.935051266673
27.525 69.935051266673
27.55 69.935051266673
27.575 69.935051266673
27.6 69.935051266673
27.625 69.935051266673
27.65 69.935051266673
27.675 69.935051266673
27.7 69.935051266673
27.725 69.935051266673
27.75 69.935051266673
27.775 69.935051266673
27.8 69.935051266673
27.825 69.935051266673
27.85 69.935051266673
27.875 69.935051266673
27.9 69.935051266673
27.925 69.935051266673
27.95 69.935051266673
27.975 69.935051266673
28 69.935051266673
28.025 69.935051266673
28.05 69.935051266673
28.075 69.935051266673
28.1 69.935051266673
28.125 69.935051266673
28.15 69.935051266673
28.175 69.935051266673
28.2 69.935051266673
28.225 69.935051266673
28.25 69.935051266673
28.275 69.935051266673
28.3 69.935051266673
28.325 69.935051266673
28.35 69.935051266673
28.375 69.935051266673
28.4 69.935051266673
28.425 69.935051266673
28.45 69.935051266673
28.475 69.935051266673
28.5 69.935051266673
28.525 69.935051266673
28.55 69.935051266673
28.575 69.935051266673
28.6 69.935051266673
28.625 69.935051266673
28.65 69.935051266673
28.675 69.935051266673
28.7 69.935051266673
28.725 69.935051266673
28.75 69.935051266673
28.775 69.935051266673
28.8 69.935051266673
28.825 69.935051266673
28.85 69.935051266673
28.875 69.935051266673
28.9 69.935051266673
28.925 69.935051266673
28.95 69.935051266673
28.975 69.935051266673
29 69.935051266673
29.025 69.935051266673
29.05 69.935051266673
29.075 69.935051266673
29.1 69.935051266673
29.125 69.935051266673
29.15 69.935051266673
29.175 69.935051266673
29.2 69.935051266673
29.225 69.935051266673
29.25 69.935051266673
29.275 69.935051266673
29.3 69.935051266673
29.325 69.935051266673
29.35 69.935051266673
29.375 69.935051266673
29.4 69.935051266673
29.425 69.935051266673
29.45 69.935051266673
29.475 69.935051266673
29.5 69.935051266673
29.525 69.935051266673
29.55 69.935051266673
29.575 69.935051266673
29.6 69.935051266673
29.625 69.935051266673
29.65 69.935051266673
29.675 69.935051266673
29.7 69.935051266673
29.725 69.935051266673
29.75 69.935051266673
29.775 69.935051266673
29.8 69.935051266673
29.825 69.935051266673
29.85 69.935051266673
29.875 69.935051266673
29.9 69.935051266673
29.925 69.935051266673
29.95 69.935051266673
29.975 69.935051266673
30 69.935051266673
30.025 69.935051266673
30.05 69.935051266673
30.075 69.935051266673
30.1 69.935051266673
30.125 69.935051266673
30.15 69.935051266673
30.175 69.935051266673
30.2 69.935051266673
30.225 69.935051266673
30.25 69.935051266673
30.275 69.935051266673
30.3 69.935051266673
30.325 69.935051266673
30.35 69.935051266673
30.375 69.935051266673
30.4 69.935051266673
30.425 69.935051266673
30.45 69.935051266673
30.475 69.935051266673
30.5 69.935051266673
30.525 69.935051266673
30.55 69.935051266673
30.575 69.935051266673
30.6 69.935051266673
30.625 69.935051266673
30.65 69.935051266673
30.675 69.935051266673
30.7 69.935051266673
30.725 69.935051266673
30.75 69.935051266673
30.775 69.935051266673
30.8 69.935051266673
30.825 69.935051266673
30.85 69.935051266673
30.875 69.935051266673
30.9 69.935051266673
30.925 69.935051266673
30.95 69.935051266673
30.975 69.935051266673
31 69.935051266673
31.025 69.935051266673
31.05 69.935051266673
31.075 69.935051266673
31.1 69.935051266673
31.125 69.935051266673
31.15 69.935051266673
31.175 69.935051266673
31.2 69.935051266673
31.225 69.935051266673
31.25 69.935051266673
31.275 69.935051266673
31.3 69.935051266673
31.325 69.935051266673
31.35 69.935051266673
31.375 69.935051266673
31.4 69.935051266673
31.425 69.935051266673
31.45 69.935051266673
31.475 69.935051266673
31.5 69.935051266673
31.525 69.935051266673
31.55 69.935051266673
31.575 69.935051266673
31.6 69.935051266673
31.625 69.935051266673
31.65 69.935051266673
31.675 69.935051266673
31.7 69.935051266673
31.725 69.935051266673
31.75 69.935051266673
31.775 69.935051266673
31.8 69.935051266673
31.825 69.935051266673
31.85 69.935051266673
31.875 69.935051266673
31.9 69.935051266673
31.925 69.935051266673
31.95 69.935051266673
31.975 69.935051266673
32 69.935051266673
32.025 69.935051266673
32.05 69.935051266673
32.075 69.935051266673
32.1 69.935051266673
32.125 69.935051266673
32.15 69.935051266673
32.175 69.935051266673
32.2 69.935051266673
32.225 69.935051266673
32.25 69.935051266673
32.275 69.935051266673
32.3 69.935051266673
32.325 69.935051266673
32.35 69.935051266673
32.375 69.935051266673
32.4 69.935051266673
32.425 69.935051266673
32.45 69.935051266673
32.475 69.935051266673
32.5 69.935051266673
32.525 69.935051266673
32.55 69.935051266673
32.575 69.935051266673
32.6 69.935051266673
32.625 69.935051266673
32.65 69.935051266673
32.675 69.935051266673
32.7 69.935051266673
32.725 69.935051266673
32.75 69.935051266673
32.775 69.935051266673
32.8 69.935051266673
32.825 69.935051266673
32.85 69.935051266673
32.875 69.935051266673
32.9 69.935051266673
32.925 69.935051266673
32.95 69.935051266673
32.975 69.935051266673
33 69.935051266673
33.025 69.935051266673
33.05 69.935051266673
33.075 69.935051266673
33.1 69.935051266673
33.125 69.935051266673
33.15 69.935051266673
33.175 69.935051266673
33.2 69.935051266673
33.225 69.935051266673
33.25 69.935051266673
33.275 69.935051266673
33.3 69.935051266673
33.325 69.935051266673
33.35 69.935051266673
33.375 69.935051266673
33.4 69.935051266673
33.425 69.935051266673
33.45 69.935051266673
33.475 69.935051266673
33.5 69.935051266673
33.525 69.935051266673
33.55 69.935051266673
33.575 69.935051266673
33.6 69.935051266673
33.625 69.935051266673
33.65 69.935051266673
33.675 69.935051266673
33.7 69.935051266673
33.725 69.935051266673
33.75 69.935051266673
33.775 69.935051266673
33.8 69.935051266673
33.825 69.935051266673
33.85 69.935051266673
33.875 69.935051266673
33.9 69.935051266673
33.925 69.935051266673
33.95 69.935051266673
33.975 69.935051266673
34 69.935051266673
34.025 69.935051266673
34.05 69.935051266673
34.075 69.935051266673
34.1 69.935051266673
34.125 69.935051266673
34.15 69.935051266673
34.175 69.935051266673
34.2 69.935051266673
34.225 69.935051266673
34.25 69.935051266673
34.275 69.935051266673
34.3 69.935051266673
34.325 69.935051266673
34.35 69.935051266673
34.375 69.935051266673
34.4 69.935051266673
34.425 69.935051266673
34.45 69.935051266673
34.475 69.935051266673
34.5 69.935051266673
34.525 69.935051266673
34.55 69.935051266673
34.575 69.935051266673
34.6 69.935051266673
34.625 69.935051266673
34.65 69.935051266673
34.675 69.935051266673
34.7 69.935051266673
34.725 69.935051266673
34.75 69.935051266673
34.775 69.935051266673
34.8 69.935051266673
34.825 69.935051266673
34.85 69.935051266673
34.875 69.935051266673
34.9 69.935051266673
34.925 69.935051266673
34.95 69.935051266673
34.975 69.935051266673
35 69.935051266673
35.025 69.935051266673
35.05 69.935051266673
35.075 69.935051266673
35.1 69.935051266673
35.125 69.935051266673
35.15 69.935051266673
35.175 69.935051266673
35.2 69.935051266673
35.225 69.935051266673
35.25 69.935051266673
35.275 69.935051266673
35.3 69.935051266673
35.325 69.935051266673
35.35 69.935051266673
35.375 69.935051266673
35.4 69.935051266673
35.425 69.935051266673
35.45 69.935051266673
35.475 69.935051266673
35.5 69.935051266673
35.525 69.935051266673
35.55 69.935051266673
35.575 69.935051266673
35.6 69.935051266673
35.625 69.935051266673
35.65 69.935051266673
35.675 69.935051266673
35.7 69.935051266673
35.725 69.935051266673
35.75 69.935051266673
35.775 69.935051266673
35.8 69.935051266673
35.825 69.935051266673
35.85 69.935051266673
35.875 69.935051266673
35.9 69.935051266673
35.925 69.935051266673
35.95 69.935051266673
35.975 69.935051266673
36 69.935051266673
36.025 69.935051266673
36.05 69.935051266673
36.075 69.935051266673
36.1 69.935051266673
36.125 69.935051266673
36.15 69.935051266673
36.175 69.935051266673
36.2 69.935051266673
36.225 69.935051266673
36.25 69.935051266673
36.275 69.935051266673
36.3 69.935051266673
36.325 69.935051266673
36.35 69.935051266673
36.375 69.935051266673
36.4 69.935051266673
36.425 69.935051266673
36.45 69.935051266673
36.475 69.935051266673
36.5 69.935051266673
36.525 69.935051266673
36.55 69.935051266673
36.575 69.935051266673
36.6 69.935051266673
36.625 69.935051266673
36.65 69.935051266673
36.675 69.935051266673
36.7 69.935051266673
36.725 69.935051266673
36.75 69.935051266673
36.775 69.935051266673
36.8 69.935051266673
36.825 69.935051266673
36.85 69.935051266673
36.875 69.935051266673
36.9 69.935051266673
36.925 69.935051266673
36.95 69.935051266673
36.975 69.935051266673
37 69.935051266673
37.025 69.935051266673
37.05 69.935051266673
37.075 69.935051266673
37.1 69.935051266673
37.125 69.935051266673
37.15 69.935051266673
37.175 69.935051266673
37.2 69.935051266673
37.225 69.935051266673
37.25 69.935051266673
37.275 69.935051266673
37.3 69.935051266673
37.325 69.935051266673
37.35 69.935051266673
37.375 69.935051266673
37.4 69.935051266673
37.425 69.935051266673
37.45 69.935051266673
37.475 69.935051266673
37.5 69.935051266673
37.525 69.935051266673
37.55 69.935051266673
37.575 69.935051266673
37.6 69.935051266673
37.625 69.935051266673
37.65 69.935051266673
37.675 69.935051266673
37.7 69.935051266673
37.725 69.935051266673
37.75 69.935051266673
37.775 69.935051266673
37.8 69.935051266673
37.825 69.935051266673
37.85 69.935051266673
37.875 69.935051266673
37.9 69.935051266673
37.925 69.935051266673
37.95 69.935051266673
37.975 69.935051266673
38 69.935051266673
38.025 69.935051266673
38.05 69.935051266673
38.075 69.935051266673
38.1 69.935051266673
38.125 69.935051266673
38.15 69.935051266673
38.175 69.935051266673
38.2 69.935051266673
38.225 69.935051266673
38.25 69.935051266673
38.275 69.935051266673
38.3 69.935051266673
38.325 69.935051266673
38.35 69.935051266673
38.375 69.935051266673
38.4 69.935051266673
38.425 69.935051266673
38.45 69.935051266673
38.475 69.935051266673
38.5 69.935051266673
38.525 69.935051266673
38.55 69.935051266673
38.575 69.935051266673
38.6 69.935051266673
38.625 69.935051266673
38.65 69.935051266673
38.675 69.935051266673
38.7 69.935051266673
38.725 69.935051266673
38.75 69.935051266673
38.775 69.935051266673
38.8 69.935051266673
38.825 69.935051266673
38.85 69.935051266673
38.875 69.935051266673
38.9 69.935051266673
38.925 69.935051266673
38.95 69.935051266673
38.975 69.935051266673
39 69.935051266673
39.025 69.935051266673
39.05 69.935051266673
39.075 69.935051266673
39.1 69.935051266673
39.125 69.935051266673
39.15 69.935051266673
39.175 69.935051266673
39.2 69.935051266673
39.225 69.935051266673
39.25 69.935051266673
39.275 69.935051266673
39.3 69.935051266673
39.325 69.935051266673
39.35 69.935051266673
39.375 69.935051266673
39.4 69.935051266673
39.425 69.935051266673
39.45 69.935051266673
39.475 69.935051266673
39.5 69.935051266673
39.525 69.935051266673
39.55 69.935051266673
39.575 69.935051266673
39.6 69.935051266673
39.625 69.935051266673
39.65 69.935051266673
39.675 69.935051266673
39.7 69.935051266673
39.725 69.935051266673
39.75 69.935051266673
39.775 69.935051266673
39.8 69.935051266673
39.825 69.935051266673
39.85 69.935051266673
39.875 69.935051266673
39.9 69.935051266673
39.925 69.935051266673
39.95 69.935051266673
39.975 69.935051266673
40 69.935051266673
40.025 69.935051266673
40.05 69.935051266673
40.075 69.935051266673
40.1 69.935051266673
40.125 69.935051266673
40.15 69.935051266673
40.175 69.935051266673
40.2 69.935051266673
40.225 69.935051266673
40.25 69.935051266673
40.275 69.935051266673
40.3 69.935051266673
40.325 69.935051266673
40.35 69.935051266673
40.375 69.935051266673
40.4 69.935051266673
40.425 69.935051266673
40.45 69.935051266673
40.475 69.935051266673
40.5 69.935051266673
40.525 69.935051266673
40.55 69.935051266673
40.575 69.935051266673
40.6 69.935051266673
40.625 69.935051266673
40.65 69.935051266673
40.675 69.935051266673
40.7 69.935051266673
40.725 69.935051266673
40.75 69.935051266673
40.775 69.935051266673
40.8 69.935051266673
40.825 69.935051266673
40.85 69.935051266673
40.875 69.935051266673
40.9 69.935051266673
40.925 69.935051266673
40.95 69.935051266673
40.975 69.935051266673
41 69.935051266673
41.025 69.935051266673
41.05 69.935051266673
41.075 69.935051266673
41.1 69.935051266673
41.125 69.935051266673
41.15 69.935051266673
41.175 69.935051266673
41.2 69.935051266673
41.225 69.935051266673
41.25 69.935051266673
41.275 69.935051266673
41.3 69.935051266673
41.325 69.935051266673
41.35 69.935051266673
41.375 69.935051266673
41.4 69.935051266673
41.425 69.935051266673
41.45 69.935051266673
41.475 69.935051266673
41.5 69.935051266673
41.525 69.935051266673
41.55 69.935051266673
41.575 69.935051266673
41.6 69.935051266673
41.625 69.935051266673
41.65 69.935051266673
41.675 69.935051266673
41.7 69.935051266673
41.725 69.935051266673
41.75 69.935051266673
41.775 69.935051266673
41.8 69.935051266673
41.825 69.935051266673
41.85 69.935051266673
41.875 69.935051266673
41.9 69.935051266673
41.925 69.935051266673
41.95 69.935051266673
41.975 69.935051266673
42 69.935051266673
42.025 69.935051266673
42.05 69.935051266673
42.075 69.935051266673
42.1 69.935051266673
42.125 69.935051266673
42.15 69.935051266673
42.175 69.935051266673
42.2 69.935051266673
42.225 69.935051266673
42.25 69.935051266673
42.275 69.935051266673
42.3 69.935051266673
42.325 69.935051266673
};
\addplot [ultra thick, black, forget plot]
table {%
9 35
9 70
};
\addplot [ultra thick, black, forget plot]
table {%
11 35
11 70
};
\addplot [ultra thick, black, forget plot]
table {%
18 35
18 70
};
\node at (axis cs:4.5,42)[
  anchor=base,
  text=black,
  rotate=0.0
]{ Accelerating};
\node at (axis cs:10,42)[
  anchor=base,
  text=black,
  rotate=0.0,
  align=center
]{ Bra-\\
king};
\node at (axis cs:14.5,42)[
  anchor=base,
  text=black,
  rotate=0.0
]{ Cruising};
\end{axis}

\end{tikzpicture}%
	\caption{Activities of the ego vehicle considering its speed. The black vertical lines indicate the events. The three different activities that can be qualitatively described as `accelerating', `braking', and `cruising', respectively.}
	\label{fig:example ego states}
	\spaceaftercaption
\end{figure}

%Figure~\ref{fig:example dynamic environment} shows the longitudinal and lateral position of the pickup truck and the station wagon with respect to the ego vehicle. The station wagon accelerates and performs a lane change at approximately $t\approx10$ s. At $t\approx12$ s, the station wagon overtakes the ego vehicle. The ego vehicle overtakes the pickup truck at $t\approx25$ s, as can be seen by the relative longitudinal distance of the pickup truck in Figure~\ref{fig:example dynamic environment}a, which becomes negative at $t\approx25$ s. The events related to the pickup truck and the station wagon are shown in Figure~\ref{fig:example events}. Here, it can be seen that the pickup truck is in the field of view of the ego vehicle's sensors from $t=1$ s until $t=34$ s. The station wagon is in the field of view of the ego vehicle's sensors from $t=0$ s until $t=18$ s.

%\begin{figure}
%	\centering
%	\setlength\figureheight{150pt}
%	\setlength\figurewidth{248pt}
%	\subfloat[Longitudinal distance of other vehicles with respect to ego vehicle.]{% This file was created by matplotlib2tikz v0.6.14.
\begin{tikzpicture}

\definecolor{color1}{rgb}{0.83921568627451,0.152941176470588,0.156862745098039}
\definecolor{color0}{rgb}{0.172549019607843,0.627450980392157,0.172549019607843}
\definecolor{color3}{rgb}{0.549019607843137,0.337254901960784,0.294117647058824}
\definecolor{color2}{rgb}{0.580392156862745,0.403921568627451,0.741176470588235}

\begin{axis}[
xlabel={Time [s]},
ylabel={Relative longitudinal distance [m]},
xmin=0, xmax=34.1600000858307,
ymin=-58.0809092368509, ymax=108.028903581739,
width=\figurewidth,
height=\figureheight,
tick align=outside,
tick pos=left,
xmajorgrids,
x grid style={lightgray!92.026143790849673!black},
ymajorgrids,
y grid style={lightgray!92.026143790849673!black},
clip marker paths
]
\addplot [semithick, color0, mark=*, mark size=1, mark options={solid}, only marks, forget plot]
table {%
0 -26.7074271603815
0.0400002002716064 -26.7193473728894
0.0800001621246338 -26.7383384251862
0.120000123977661 -26.7460490026369
0.160000085830688 -26.7664243002673
0.200000047683716 -26.779038823357
0.240000009536743 -26.7981872784185
0.28000020980835 -26.8295946814196
0.320000171661377 -26.8472175355928
0.360000133514404 -26.8765110027307
0.400000095367432 -26.8976779480472
0.440000057220459 -26.9161635570545
0.480000019073486 -26.9500390967969
0.519999980926514 -26.9626746751619
0.56000018119812 -26.9931910108844
0.600000143051147 -27.0158738229457
0.640000104904175 -27.0470681455827
0.680000066757202 -27.0739041547095
0.720000028610229 -27.1103165653785
0.759999990463257 -27.1499582959623
0.800000190734863 -27.1749450570478
0.840000152587891 -27.2110858703618
0.880000114440918 -27.2457953815992
0.920000076293945 -27.263344395089
0.960000038146973 -27.2944718244144
1 -27.3275244008983
1.04000020027161 -27.3461533459049
1.08000016212463 -27.404412105634
1.12000012397766 -27.4299916279488
1.16000008583069 -27.469466645427
1.20000004768372 -27.5036813379666
1.24000000953674 -27.5424287676051
1.28000020980835 -27.5992346562234
1.32000017166138 -27.6254737172967
1.3600001335144 -27.6676030564413
1.40000009536743 -27.7052177632922
1.44000005722046 -27.7366472687481
1.48000001907349 -27.7958631867077
1.51999998092651 -27.8290239459493
1.56000018119812 -27.8624309720108
1.60000014305115 -27.9005839226102
1.64000010490417 -27.936000716034
1.6800000667572 -27.9846167561773
1.72000002861023 -28.0065385651487
1.75999999046326 -28.0374855482733
1.80000019073486 -28.0730068300436
1.84000015258789 -28.1073708202912
1.88000011444092 -28.1585357137046
1.92000007629395 -28.1704042716519
1.96000003814697 -28.1984069697191
2 -28.2331737318946
2.04000020027161 -28.2569818936063
2.08000016212463 -28.3134767408137
2.12000012397766 -28.33457098291
2.16000008583069 -28.3772719417411
2.20000004768372 -28.4094436982869
2.24000000953674 -28.4405307212692
2.28000020980835 -28.4909659151326
2.32000017166138 -28.5102945930848
2.3600001335144 -28.5454310228088
2.40000009536743 -28.5738279901016
2.44000005722046 -28.6058762154462
2.48000001907349 -28.6507218298466
2.51999998092651 -28.6632214247602
2.56000018119812 -28.6973162838203
2.60000014305115 -28.7268398443266
2.64000010490417 -28.7571284448313
2.6800000667572 -28.808816608831
2.72000002861023 -28.8266725322683
2.75999999046326 -28.8658204385556
2.80000019073486 -28.8940627079883
2.84000015258789 -28.924219830471
2.88000011444092 -28.9840174570636
2.92000007629395 -29.0051623854943
2.96000003814697 -29.0588416641222
3 -29.0921224670674
3.04000020027161 -29.1367741825361
3.08000016212463 -29.196525812773
3.12000012397766 -29.2202876176016
3.16000008583069 -29.2727025826334
3.20000004768372 -29.3074194994151
3.24000000953674 -29.3521726259696
3.28000020980835 -29.4058262929539
3.32000017166138 -29.4296208923552
3.3600001335144 -29.4786247010088
3.40000009536743 -29.5130321161469
3.44000005722046 -29.5584644286209
3.48000001907349 -29.6141678829208
3.51999998092651 -29.6406701762025
3.56000018119812 -29.6823762718086
3.60000014305115 -29.7264484987882
3.64000010490417 -29.7713429856267
3.6800000667572 -29.8255580331006
3.72000002861023 -29.856951402975
3.75999999046326 -29.910074862717
3.80000019073486 -29.9485975998832
3.84000015258789 -29.9895311112195
3.88000011444092 -30.0398992823993
3.92000007629395 -30.0691890743856
3.96000003814697 -30.0977602159146
4 -30.159594566323
4.04000020027161 -30.1731866178852
4.08000016212463 -30.2148813720869
4.12000012397766 -30.2496221080928
4.16000008583069 -30.2803923345346
4.20000004768372 -30.3199595746482
4.24000000953674 -30.3390741809362
4.28000020980835 -30.3788283571612
4.32000017166138 -30.4023925829551
4.3600001335144 -30.4338549808817
4.40000009536743 -30.471955628771
4.44000005722046 -30.4928417260071
4.48000001907349 -30.5457550454139
4.51999998092651 -30.567108358624
4.56000018119812 -30.6140201087146
4.60000014305115 -30.6598147006916
4.64000010490417 -30.6878860990491
4.6800000667572 -30.7355324311775
4.72000002861023 -30.7683442772741
4.75999999046326 -30.8261440587339
4.80000019073486 -30.8687390755385
4.84000015258789 -30.9099162060938
4.88000011444092 -30.9738719872839
4.92000007629395 -30.9980793437735
4.96000003814697 -31.0767215251435
5 -31.1119637704469
5.04000020027161 -31.1593912526369
5.08000016212463 -31.214012101389
5.12000012397766 -31.2728830271117
5.16000008583069 -31.3172987296166
5.20000004768372 -31.383580664924
5.24000000953674 -31.4472430902279
5.28000020980835 -31.4886011693106
5.32000017166138 -31.5505265284828
5.3600001335144 -31.5957848910039
5.40000009536743 -31.6659730994397
5.44000005722046 -31.714066504248
5.48000001907349 -31.7630324400961
5.51999998092651 -31.8323117799755
5.56000018119812 -31.8743756620715
5.60000014305115 -31.9418214232555
5.64000010490417 -31.9858726325892
5.6800000667572 -32.0257822271687
5.72000002861023 -32.0993768488479
5.75999999046326 -32.1364252221247
5.80000019073486 -32.2288924718887
5.84000015258789 -32.2767301638487
5.88000011444092 -32.320289510335
5.92000007629395 -32.3877188390488
5.96000003814697 -32.4358495067754
6 -32.5199657559551
6.04000020027161 -32.60914191743
6.08000016212463 -32.6483753038101
6.12000012397766 -32.7333624830098
6.16000008583069 -32.7971334856502
6.20000004768372 -32.8847336545605
6.24000000953674 -32.9162921739353
6.28000020980835 -32.9859252078058
6.32000017166138 -33.0500405582061
6.3600001335144 -33.0750500777667
6.40000009536743 -33.1402647152117
6.44000005722046 -33.2106582522938
6.48000001907349 -33.2388524432699
6.51999998092651 -33.3051267675819
6.56000018119812 -33.3660438160696
6.60000014305115 -33.4123215717282
6.64000010490417 -33.4695835284092
6.6800000667572 -33.5204593584458
6.72000002861023 -33.5878880129167
6.75999999046326 -33.6522965958156
6.80000019073486 -33.7036051560499
6.84000015258789 -33.7577059540854
6.88000011444092 -33.8058453285303
6.92000007629395 -33.8648800318606
6.96000003814697 -33.9153492368514
7 -33.9676276877672
7.04000020027161 -34.0200439129221
7.08000016212463 -34.0696535416864
7.12000012397766 -34.1215291295975
7.16000008583069 -34.1879415296498
7.20000004768372 -34.2439704962035
7.24000000953674 -34.3016181066505
7.28000020980835 -34.3486455192688
7.32000017166138 -34.4160809300884
7.3600001335144 -34.4559836321223
7.40000009536743 -34.5210997451741
7.44000005722046 -34.5828120242459
7.48000001907349 -34.6020625552355
7.51999998092651 -34.6591875650829
7.56000018119812 -34.6927522353089
7.60000014305115 -34.7192409635682
7.64000010490417 -34.787252095015
7.6800000667572 -34.8113144948238
7.72000002861023 -34.848399161061
7.75999999046326 -34.8717833188257
7.80000019073486 -34.8933181017856
7.84000015258789 -34.9485304696154
7.88000011444092 -34.9624663773448
7.92000007629395 -35.0146831032653
7.96000003814697 -35.0409629800179
8 -35.0730814660183
8.04000020027161 -35.0862488593975
8.08000016212463 -35.099714781576
8.12000012397766 -35.1352397092269
8.16000008583069 -35.1369820298005
8.20000004768372 -35.147280783267
8.24000000953674 -35.1719449717675
8.28000020980835 -35.1563860035712
8.32000017166138 -35.1864213254939
8.3600001335144 -35.1782663117428
8.40000009536743 -35.1817153648408
8.44000005722046 -35.1742719525719
8.48000001907349 -35.161119165954
8.51999998092651 -35.1688152827337
8.56000018119812 -35.1701919391071
8.60000014305115 -35.1656855714955
8.64000010490417 -35.1476529812517
8.6800000667572 -35.1142517326825
8.72000002861023 -35.0963731100655
8.75999999046326 -35.0762722647596
8.80000019073486 -35.0255003242382
8.84000015258789 -35.0151752920829
8.88000011444092 -34.9693337524677
8.92000007629395 -34.9217376187516
8.96000003814697 -34.8897895912669
9 -34.812877766115
9.04000020027161 -34.7924157517209
9.08000016212463 -34.7151076064274
9.12000012397766 -34.640522485126
9.16000008583069 -34.5990262113537
9.20000004768372 -34.4940963246008
9.24000000953674 -34.4606192931769
9.28000020980835 -34.3371298010206
9.32000017166138 -34.23826584788
9.3600001335144 -34.1515125689275
9.40000009536743 -33.9918119513386
9.44000005722046 -33.9285336874655
9.48000001907349 -33.7962827252013
9.51999998092651 -33.6931405823179
9.56000018119812 -33.5694319435843
9.60000014305115 -33.4851381752378
9.64000010490417 -33.2599097684451
9.6800000667572 -33.1117586445798
9.72000002861023 -32.9809530615439
9.75999999046326 -32.8077820722556
9.80000019073486 -32.7035651437564
9.84000015258789 -32.4261585182394
9.88000011444092 -32.2410680008943
9.92000007629395 -32.0992599954934
9.96000003814697 -31.8726392893768
10 -31.7454597982378
10.0400002002716 -31.4411446770082
10.0800001621246 -31.233127927575
10.1200001239777 -31.0313373131121
10.1600000858307 -30.8354171908941
10.2000000476837 -30.5654751741386
10.2400000095367 -30.3117328715616
10.2800002098083 -30.0746374579539
10.3200001716614 -29.7310973859676
10.3600001335144 -29.5877009884753
10.4000000953674 -29.2912747910868
10.4400000572205 -29.0059770649514
10.4800000190735 -28.74191626227
10.5199999809265 -28.4421755285694
10.5600001811981 -28.2038473013163
10.6000001430511 -27.8992015238127
10.6400001049042 -27.5676398735959
10.6800000667572 -27.2787464374487
10.7200000286102 -26.9570306091809
10.7599999904633 -26.6933969814163
10.8000001907349 -26.3992908669352
10.8400001525879 -26.0010395113823
10.8800001144409 -25.6816084341845
10.9200000762939 -25.2293537422502
10.960000038147 -25.0533175649962
11 -24.7481268427546
11.0400002002716 -24.1939342429196
11.0800001621246 -23.9676291924116
11.1200001239777 -23.4778014131789
11.1600000858307 -23.2981569862113
11.2000000476837 -22.7823998891727
11.2400000095367 -22.4005123981951
11.2800002098083 -22.1704446725089
11.3200001716614 -21.6509466389034
11.3600001335144 -21.4665139782737
11.4000000953674 -20.9171570582475
11.4400000572205 -20.5083246894446
11.4800000190735 -20.2748806240961
11.5199999809265 -19.9254589927186
11.5600001811981 -19.5327539030077
11.6000001430511 -19.1384555607128
11.6400001049042 -18.7594986876302
11.6800000667572 -18.3516546520277
11.7200000286102 -17.9537587907071
11.7599999904633 -17.5428801713952
11.8000001907349 -17.0988092930475
11.8400001525879 -16.7071988682437
11.8800001144409 -16.2228427529953
11.9200000762939 -15.6044093580276
11.960000038147 -15.4035304260287
12 -14.9787021192078
12.0400002002716 -14.5429201174975
12.0800001621246 -14.079700822349
12.1200001239777 -13.6419597842887
12.1600000858307 -13.1375779392056
12.2000000476837 -12.7486996354583
12.2400000095367 -12.4589798666839
12.2800002098083 -11.8264947746738
12.3200001716614 -11.1703211419299
12.3600001335144 -10.8968332323839
12.4000000953674 -10.7114346185263
12.4400000572205 -9.96265820771623
12.4800000190735 -9.46858046699708
12.5199999809265 -9.27248414215137
12.5600001811981 -8.50994569608702
12.6000001430511 -8.02686366380658
12.6400001049042 -7.70846811756928
12.6800000667572 -7.02048047556673
12.7200000286102 -6.3287742265511
12.7599999904633 -5.91321427856565
12.8000001907349 -5.21998669377172
12.8400001525879 -4.80847770170476
12.8800001144409 -4.268529564657
12.9200000762939 -3.72207962568973
12.960000038147 -3.23143474607969
13 -2.61423576998277
13.0400002002716 -1.9705408693535
13.0800001621246 -1.72826559370515
13.1200001239777 -0.92838600739924
13.1600000858307 -0.693815903741779
13.2000000476837 -1.00334331995691
13.2400000095367 -0.484043699425456
13.2800002098083 0.0319108034018427
13.3200001716614 0.396130436654857
13.3600001335144 0.623004507833684
13.4000000953674 0.803756873021484
13.4400000572205 1.62131004846742
13.4800000190735 1.61357000075441
13.5199999809265 1.81224042025497
13.5600001811981 2.11852726216966
13.6000001430511 2.40308223495413
13.6400001049042 2.75513684568978
13.6800000667572 3.32946984285263
13.7200000286102 3.59663515279499
13.7599999904633 3.91507421624192
13.8000001907349 4.36358850169745
13.8400001525879 4.51360365835171
13.8800001144409 4.94167528904836
13.9200000762939 5.37009874303658
13.960000038147 5.68058540081438
14 5.91829565265107
14.0400002002716 6.67761481379603
14.0800001621246 7.30132191232951
14.1200001239777 7.63652949019706
14.1600000858307 8.15941498259781
14.2000000476837 8.72547375913564
14.2400000095367 9.1136385812315
14.2800002098083 9.57565185532076
14.3200001716614 10.0625135830778
14.3600001335144 10.4497263736266
14.4000000953674 10.9732587770159
14.4400000572205 11.3082878775567
14.4800000190735 11.9950598698833
14.5199999809265 12.652072886156
14.5600001811981 12.9875154633537
14.6000001430511 13.3510255465499
14.6400001049042 13.9892150072556
14.6800000667572 14.459709168912
14.7200000286102 15.0131879948458
14.7599999904633 15.4210395850077
14.8000001907349 15.8464314271132
14.8400001525879 16.3309100337337
14.8800001144409 17.0135110500887
14.9200000762939 17.2123697165607
14.960000038147 18.0479691289111
15 18.3648939073973
15.2000000476837 21.3686649939609
15.2400000095367 21.4334943564572
15.2800002098083 21.5685119330574
15.3200001716614 21.6704316239793
15.3600001335144 22.498243723534
15.4000000953674 22.5499004733701
15.4400000572205 23.2889026424637
15.4800000190735 23.4228098014428
15.5199999809265 23.5489864090632
15.5600001811981 23.6389803841303
15.6000001430511 23.9724396863803
15.6400001049042 24.1662413898084
15.6800000667572 24.2506874450501
15.7200000286102 25.6347522694105
15.7599999904633 26.8581646424063
15.8000001907349 27.2339136086321
15.8400001525879 27.3084825929363
15.8800001144409 28.4282980904154
15.9200000762939 29.3824495788267
15.960000038147 29.700553968125
16 30.6328035879069
16.0400002002716 30.8673750668913
16.0800001621246 31.381195706681
16.1200001239777 32.4193410273729
16.1600000858307 33.0757201501856
16.2000000476837 33.8633581596714
16.2400000095367 34.1473289805945
16.2800002098083 35.0630464370497
16.3200001716614 35.761957506611
16.3600001335144 36.1961462417075
16.4000000953674 36.937083341023
16.4400000572205 37.2649042567045
16.4800000190735 37.9478108444055
16.5199999809265 38.6263547744275
16.5600001811981 38.8948139790791
16.6000001430511 39.1654710781986
16.6400001049042 39.7387004257816
16.6800000667572 40.8411866273163
16.7200000286102 41.6673733483331
16.7599999904633 42.1244529573905
16.8000001907349 42.9967583665875
16.8400001525879 43.3654057657714
16.8800001144409 44.1395076419012
16.9200000762939 44.8641584779743
16.960000038147 45.2118127102403
17 45.9268941407809
17.0400002002716 46.2571265880215
17.0800001621246 46.8908053678169
17.1200001239777 47.1287872850517
17.1600000858307 48.0050614973097
17.2000000476837 48.6235918735074
17.2400000095367 49.3617076193077
17.2800002098083 49.7698168235875
17.3200001716614 50.5343237969464
17.3600001335144 50.9851337655946
17.4000000953674 51.1722581298818
17.4400000572205 52.2373630401016
17.4800000190735 52.8044304237519
17.5199999809265 53.0278847427908
17.5600001811981 54.0066121870695
17.6000001430511 54.6472276676432
17.6400001049042 55.1377461491975
17.6800000667572 55.8690445852444
17.7200000286102 56.737045744434
};
\addplot [semithick, color2, mark=*, mark size=1, mark options={solid}, only marks, forget plot]
table {%
0.800000190734863 98.4813046802665
0.840000152587891 98.5269853668706
0.880000114440918 98.5651418252746
0.920000076293945 98.6033709009916
0.960000038146973 98.6469268777219
1 98.6673857671885
1.04000020027161 98.6935346560804
1.08000016212463 98.7694577574075
1.12000012397766 98.8130497280763
1.16000008583069 98.8815859331189
1.20000004768372 98.9437801876738
1.24000000953674 99.0007340018838
1.28000020980835 99.0772220442977
1.32000017166138 99.1266808993332
1.3600001335144 99.1103223365662
1.40000009536743 99.0795647581508
1.44000005722046 99.0556357989281
1.48000001907349 99.0967343568227
1.51999998092651 99.1265321413612
1.56000018119812 99.1612503925808
1.60000014305115 99.18813955109
1.64000010490417 99.2165418463937
1.6800000667572 99.2523647192647
1.72000002861023 99.2730452563192
1.75999999046326 99.3363993665789
1.80000019073486 99.3872457384587
1.84000015258789 99.4351073665821
1.88000011444092 99.4567756132365
1.92000007629395 99.5599056210103
1.96000003814697 99.5970460406188
2 99.62334959074
2.04000020027161 99.6377982228696
2.08000016212463 99.7262297860507
2.12000012397766 99.7872557148912
2.16000008583069 99.8056618992468
2.20000004768372 99.9925754232299
2.24000000953674 100.01288461376
2.28000020980835 100.111777668471
2.32000017166138 100.194413163299
2.3600001335144 100.278176584017
2.40000009536743 100.38958907356
2.44000005722046 100.412386269027
2.48000001907349 100.47845754453
2.51999998092651 100.412045813178
2.56000018119812 100.347810031864
2.60000014305115 100.362673922806
2.64000010490417 100.289602193405
2.6800000667572 100.138496493411
2.72000002861023 100.065028071531
2.75999999046326 99.9870446548848
2.80000019073486 99.9097542781219
2.84000015258789 99.8325806074827
2.88000011444092 99.8159898597314
2.92000007629395 99.7307144526385
2.96000003814697 99.5815602040675
3 99.5118403643919
3.04000020027161 99.436903193262
3.08000016212463 99.3552290034149
3.12000012397766 99.3327217749993
3.16000008583069 99.2535749293002
3.20000004768372 99.1574147120245
3.24000000953674 99.0869582875548
3.28000020980835 98.9997106937844
3.32000017166138 98.9408059743982
3.3600001335144 98.8326512986769
3.40000009536743 98.7483295914244
3.44000005722046 98.6564206652492
3.48000001907349 98.5847908558499
3.51999998092651 98.5499685980594
3.56000018119812 98.4897440526111
3.60000014305115 98.4327039871951
3.64000010490417 98.3701243373562
3.6800000667572 98.2960420116415
3.72000002861023 98.2492274515735
3.75999999046326 98.1678258821194
3.80000019073486 98.1072076112578
3.84000015258789 98.0351887810648
3.88000011444092 97.9324927988655
3.92000007629395 97.884001395445
3.96000003814697 97.8052328055783
4 97.7013002816366
4.04000020027161 97.6467732030414
4.08000016212463 97.5375788497349
4.12000012397766 97.4794276997764
4.16000008583069 97.3789058751081
4.20000004768372 97.3115610222849
4.24000000953674 97.2318460306196
4.28000020980835 97.1074439109252
4.32000017166138 97.0867923079659
4.3600001335144 96.9634554600161
4.40000009536743 96.9118715765653
4.44000005722046 96.8326502830478
4.48000001907349 96.6991523945235
4.51999998092651 96.6752130427885
4.56000018119812 96.5582558588048
4.60000014305115 96.4875620480634
4.64000010490417 96.4227320332084
4.6800000667572 96.2846585254902
4.72000002861023 96.2659606480502
4.75999999046326 96.1239863613027
4.80000019073486 96.0389680428561
4.84000015258789 95.9748164943285
4.88000011444092 95.8400444662293
4.92000007629395 95.8021551418897
4.96000003814697 95.6503618025472
5 95.5737063134002
5.04000020027161 95.4724011675244
5.08000016212463 95.3350817488554
5.12000012397766 95.2544304569456
5.16000008583069 95.1617105926671
5.20000004768372 95.0490401065435
5.24000000953674 94.9434214909088
5.28000020980835 94.7951467589428
5.32000017166138 94.7160388562133
5.3600001335144 94.5758704656073
5.40000009536743 94.4677594501027
5.44000005722046 94.3651035750117
5.48000001907349 94.2280500001016
5.51999998092651 94.1267791856626
5.56000018119812 93.9988390636681
5.60000014305115 93.8766536203057
5.64000010490417 93.7502318796742
5.6800000667572 93.595016244979
5.72000002861023 93.4909765345074
5.75999999046326 93.3506675396111
5.80000019073486 93.2231530993176
5.84000015258789 93.0940445282467
5.88000011444092 92.9318985765749
5.92000007629395 92.8201664087719
5.96000003814697 92.6880559596902
6 92.5617145112883
6.04000020027161 92.4319133641111
6.08000016212463 92.290892333087
6.12000012397766 92.1832825687161
6.16000008583069 92.0511965021797
6.20000004768372 91.9054138146948
6.24000000953674 91.7717042434579
6.28000020980835 91.6209399415311
6.32000017166138 91.4793964985602
6.3600001335144 91.3239479667263
6.40000009536743 91.1716351935465
6.44000005722046 91.0197089062658
6.48000001907349 90.8496162846404
6.51999998092651 90.7080691100982
6.56000018119812 90.5053054968048
6.60000014305115 90.3972468432639
6.64000010490417 90.2392201299881
6.6800000667572 90.081127059726
6.72000002861023 89.9370776744418
6.75999999046326 89.7424178325964
6.80000019073486 89.6112349464693
6.84000015258789 89.4501744763838
6.88000011444092 89.2602529025735
6.92000007629395 89.1146002937967
6.96000003814697 88.901477039919
7 88.7567898048219
7.04000020027161 88.5933802124764
7.08000016212463 88.3856459812141
7.12000012397766 88.2318763106305
7.16000008583069 88.05425331474
7.20000004768372 87.8780010400769
7.24000000953674 87.6965430504169
7.28000020980835 87.4899699575835
7.32000017166138 87.3190719911581
7.3600001335144 87.1314329415472
7.40000009536743 86.8678534456667
7.44000005722046 86.7105932357827
7.48000001907349 86.5598101428404
7.51999998092651 86.385180428204
7.56000018119812 86.1950403277624
7.60000014305115 86.0904584861419
7.64000010490417 85.7740559649865
7.6800000667572 85.6523217211015
7.72000002861023 85.483337743306
7.75999999046326 85.2812435431842
7.80000019073486 85.179917384301
7.84000015258789 84.8431384234918
7.88000011444092 84.6960592417145
7.92000007629395 84.5249195184424
7.96000003814697 84.3316586626315
8 84.2234866174949
8.04000020027161 83.7690075841983
8.08000016212463 83.5583631163954
8.12000012397766 83.3758981975843
8.16000008583069 82.9997479503018
8.20000004768372 82.8797037677396
8.24000000953674 82.4146080198789
8.28000020980835 82.19328984811
8.32000017166138 81.9937010739686
8.3600001335144 81.7025286749194
8.40000009536743 81.5753696146949
8.44000005722046 81.1079965556528
8.48000001907349 80.8759391888416
8.51999998092651 80.6741330908808
8.56000018119812 80.3917022214828
8.60000014305115 80.2648741314861
8.64000010490417 79.9434520948016
8.6800000667572 79.6705405555949
8.72000002861023 79.3275562403887
8.75999999046326 79.1909267317787
8.80000019073486 78.8554675869018
8.84000015258789 78.7712238912536
8.88000011444092 78.4894668793258
8.92000007629395 78.252011492903
8.96000003814697 78.111811736153
9 77.7889219056851
9.04000020027161 77.6952749607754
9.08000016212463 77.4256673427735
9.12000012397766 77.1731179189155
9.16000008583069 77.023768318968
9.20000004768372 76.8262967206283
9.24000000953674 76.8892582716871
9.28000020980835 76.7639551922948
9.32000017166138 76.6147651942138
9.3600001335144 76.4572805592179
9.40000009536743 76.2871550613963
9.44000005722046 76.1624307891125
9.48000001907349 75.9950910817261
9.51999998092651 75.7732583720481
9.56000018119812 75.6711034929067
9.60000014305115 75.6076642039334
9.64000010490417 75.2965774110562
9.6800000667572 75.1482090560094
9.72000002861023 74.968358724529
9.75999999046326 74.8510652732966
9.80000019073486 74.7550080436522
9.84000015258789 74.483606921287
9.88000011444092 74.3648147486856
9.92000007629395 74.199954486081
9.96000003814697 74.0297278903963
10 73.9939979722676
10.0400002002716 73.6693078498174
10.0800001621246 73.5444326317156
10.1200001239777 73.3546367141134
10.1600000858307 73.1859152552315
10.2000000476837 72.9753319519295
10.2400000095367 72.8666503380427
10.2800002098083 72.7161943208976
10.3200001716614 72.4708107303068
10.3600001335144 72.3935057592043
10.4000000953674 72.163933548456
10.4400000572205 72.0476571020008
10.4800000190735 71.8952086077334
10.5199999809265 71.7158731643012
10.5600001811981 71.5807673227282
10.6000001430511 71.3446630600574
10.6400001049042 71.200188338189
10.6800000667572 71.0300311362025
10.7200000286102 70.8134511400185
10.7599999904633 70.6937341530866
10.8000001907349 70.5295471981026
10.8400001525879 70.3416382460964
10.8800001144409 70.1947327940961
10.9200000762939 69.9524152560025
10.960000038147 69.8766942431157
11 69.7442821386794
11.0400002002716 69.5081499953412
11.0800001621246 69.4115475297112
11.1200001239777 69.2131571529972
11.1600000858307 69.1400693315518
11.2000000476837 68.8960517075357
11.2400000095367 68.7618605381394
11.2800002098083 68.6672934896087
11.3200001716614 68.4594231316205
11.3600001335144 68.387142577576
11.4000000953674 68.1243966418333
11.4400000572205 68.0271575902225
11.4800000190735 67.8976694302601
11.5199999809265 67.8004433525639
11.5600001811981 67.6441063854381
11.6000001430511 67.4869642190315
11.6400001049042 67.3567480318907
11.6800000667572 67.1536921317584
11.7200000286102 67.0321304683785
11.7599999904633 66.9182711179819
11.8000001907349 66.7433549425223
11.8400001525879 66.596418219513
11.8800001144409 66.3925740738487
11.9200000762939 66.2369178161025
11.960000038147 66.1652093230332
12 65.9793267816203
12.0400002002716 65.8227919031087
12.0800001621246 65.6743533082499
12.1200001239777 65.5240155722659
12.1600000858307 65.3239246499761
12.2000000476837 65.1867124200489
12.2400000095367 65.1207254970323
12.2800002098083 64.8795839918312
12.3200001716614 64.7031552670123
12.3600001335144 64.5720209652227
12.4000000953674 64.5090173814588
12.4400000572205 64.2942849645933
12.4800000190735 64.0767863790734
12.5199999809265 64.0145269256827
12.5600001811981 63.7290996267184
12.6000001430511 63.6042031637789
12.6400001049042 63.5182594476209
12.6800000667572 63.2929317666913
12.7200000286102 63.0934819637423
12.7599999904633 62.9537550670393
12.8000001907349 62.7573855118571
12.8400001525879 62.6800711266769
12.8800001144409 62.5067331991486
12.9200000762939 62.3642591699263
12.960000038147 62.1874737130911
13 62.0314839285384
13.0400002002716 61.8260016394543
13.0800001621246 61.7241622982256
13.1200001239777 61.5517780405971
13.1600000858307 61.4496662423462
13.2000000476837 61.2750228351524
13.2400000095367 61.1197544686565
13.2800002098083 60.9956006658267
13.3200001716614 60.8239235176934
13.3600001335144 60.684230734556
13.4000000953674 60.646412660466
13.4400000572205 60.4113680738192
13.4800000190735 60.2575593062375
13.5199999809265 60.2219683607582
13.5600001811981 59.9787905826124
13.6000001430511 59.8097530966952
13.6400001049042 59.6886928021486
13.6800000667572 59.5056263512561
13.7200000286102 59.2962698809915
13.7599999904633 59.1990800204276
13.8000001907349 58.9741356745308
13.8400001525879 58.8384067664811
13.8800001144409 58.6900228698796
13.9200000762939 58.520578837406
13.960000038147 58.3712835800306
14 58.2882388707039
14.0400002002716 58.0777614707949
14.0800001621246 57.8721208536554
14.1200001239777 57.7426215103533
14.1600000858307 57.5652529110394
14.2000000476837 57.3577575218915
14.2400000095367 57.2275598712367
14.2800002098083 57.0633195360733
14.3200001716614 56.8775263221596
14.3600001335144 56.7498937537566
14.4000000953674 56.5479537440933
14.4400000572205 56.4550682828612
14.4800000190735 56.2441864623197
14.5199999809265 56.0239782900699
14.5600001811981 55.9227898398112
14.6000001430511 55.8075831533715
14.6400001049042 55.6272747536641
14.6800000667572 55.4883576822285
14.7200000286102 55.3151069097894
14.7599999904633 55.1817803648228
14.8000001907349 55.0446374164621
14.8400001525879 54.90368694676
14.8800001144409 54.7011295642951
14.9200000762939 54.6491269185244
14.960000038147 54.3918662189517
15 54.2738482121367
15.0400002002716 54.140767299783
15.0800001621246 53.9352796307194
15.1200001239777 53.8749033971799
15.1600000858307 53.6815295200704
15.2000000476837 53.4715277780888
15.2400000095367 53.3701543543903
15.2800002098083 53.1547464667055
15.3200001716614 52.9108216813875
15.3600001335144 52.8163490902298
15.4000000953674 52.701743822161
15.4400000572205 52.4955314710987
15.4800000190735 52.3445586924627
15.5199999809265 52.1796220770011
15.5600001811981 52.0720333679255
15.6000001430511 51.8234931573988
15.6400001049042 51.6227902485371
15.6800000667572 51.5294824312205
15.7200000286102 51.4552960028795
15.7599999904633 51.1995251637345
15.8000001907349 50.9896440220637
15.8400001525879 50.9449902595388
15.8800001144409 50.7395372224728
15.9200000762939 50.557330572432
15.960000038147 50.4161600219286
16 49.8933454087928
16.0400002002716 49.8287269836746
16.0800001621246 49.5958988120601
16.1200001239777 49.4423079533371
16.1600000858307 49.3535474912387
16.2000000476837 49.2108871194068
16.2400000095367 49.1207778991502
16.2800002098083 48.903704573846
16.3200001716614 48.760798637948
16.3600001335144 48.6290850177247
16.4000000953674 48.4615979602077
16.4400000572205 48.3711298913495
16.4800000190735 48.1557696345062
16.5199999809265 47.9565505869596
16.5600001811981 47.8678627867193
16.6000001430511 47.7931877469491
16.6400001049042 47.60271399456
16.6800000667572 47.400335798693
16.7200000286102 47.2500810495967
16.7599999904633 47.1143226950062
16.8000001907349 46.6605169290233
16.8400001525879 46.5427721819597
16.8800001144409 46.1769244162297
16.9200000762939 45.9661154405767
16.960000038147 45.7757367114227
17 45.5644383420695
17.0400002002716 45.3650144847143
17.0800001621246 45.2285070788039
17.1200001239777 45.1587131239721
17.1600000858307 44.8532749736496
17.2000000476837 44.5918577888442
17.2400000095367 44.440558965025
17.2800002098083 44.2987053716188
17.3200001716614 44.0434672121464
17.3600001335144 43.9339045326087
17.4000000953674 43.8686612129131
17.4400000572205 43.6488654722471
17.4800000190735 43.4288592162411
17.5199999809265 43.3696913238691
17.5600001811981 43.1644681866837
17.6000001430511 42.9707507779003
17.6400001049042 42.9121507764248
17.6800000667572 42.7037861452136
17.7200000286102 42.5738659826839
17.7599999904633 42.4974162556209
17.8000001907349 42.2500980906207
17.8400001525879 42.2068670938843
17.8800001144409 42.0974219083437
17.9200000762939 41.962168946462
17.960000038147 41.902468921613
18 41.8578663595617
18.0400002002716 41.7483312594868
18.0800001621246 41.5303715610207
18.1200001239777 41.4072990742334
18.1600000858307 41.2425721938253
18.2000000476837 41.0736742230511
18.2400000095367 40.9250716992774
18.2800002098083 40.4125830179619
18.3200001716614 40.2627285149829
18.3600001335144 40.1015450170635
18.4000000953674 39.9462275140231
18.4400000572205 39.7644965609688
18.4800000190735 39.5838978472748
18.5199999809265 39.5514266770879
18.5600001811981 39.268589014504
18.6000001430511 39.0533921655169
18.6400001049042 38.9679917284866
18.6800000667572 38.8392801879272
18.7200000286102 38.6548477970991
18.7599999904633 38.108161887707
18.8000001907349 37.7470343574023
18.8400001525879 37.4287551458619
18.8800001144409 37.1496807864478
18.9200000762939 36.8330183470425
18.960000038147 36.6826766453596
19 36.5896465481746
19.0400002002716 36.180948770947
19.0800001621246 35.9435419878337
19.1200001239777 35.6127631311138
19.1600000858307 35.5069438402097
19.2000000476837 35.2187979798509
19.2400000095367 34.9880523154225
19.2800002098083 34.8191972757377
19.3200001716614 34.4393978520202
19.3600001335144 34.3447040925294
19.4000000953674 34.2440134860899
19.4400000572205 33.8250454999161
19.4800000190735 33.6382115582674
19.5199999809265 33.3718098585359
19.5600001811981 33.2261817778926
19.6000001430511 32.9622111300469
19.6400001049042 32.8186304817591
19.6800000667572 32.502190601821
19.7200000286102 32.2647618670781
19.7599999904633 32.0647019009266
19.8000001907349 31.8113134998785
19.8400001525879 31.6043074554782
19.8800001144409 31.3549088083109
19.9200000762939 31.1001584362384
19.960000038147 30.9066141441508
20 30.6516875993439
20.0400002002716 30.4624807351775
20.0800001621246 30.2125688848901
20.1200001239777 29.9692411026317
20.1600000858307 29.7548883619074
20.2000000476837 29.5032443090495
20.2400000095367 29.3004009337401
20.2800002098083 29.0532929702767
20.3200001716614 28.8401740098016
20.3600001335144 28.6151487420575
20.4000000953674 28.3665451064917
20.4400000572205 28.1662384056654
20.4800000190735 27.9163665707802
20.5199999809265 27.7153229648211
20.5600001811981 27.4622102289341
20.6000001430511 27.2183949631981
20.6400001049042 27.0250388625391
20.6800000667572 26.7687860190053
20.7200000286102 26.5547277751211
20.7599999904633 26.298235489634
20.8000001907349 26.0485994120736
20.8400001525879 25.8450235084147
20.8800001144409 25.569980872051
20.9200000762939 25.4427271924342
20.960000038147 25.1593901164015
21 25.0129054721838
21.0400002002716 24.7393458578481
21.0800001621246 24.3870792166035
21.1200001239777 24.2283612260144
21.1600000858307 23.9626020073993
21.2000000476837 23.8179353694704
21.2400000095367 23.56275790158
21.2800002098083 23.1927447305952
21.3200001716614 23.012915401594
21.3600001335144 22.7670369710286
21.4000000953674 22.604121147735
21.4400000572205 22.3481422794357
21.4800000190735 21.985332189417
21.5199999809265 21.7180189203482
21.5600001811981 21.6019234601226
21.6000001430511 21.2369073490081
21.6400001049042 21.0194671260888
21.6800000667572 20.7517484831496
21.7200000286102 20.4756911986788
21.7599999904633 20.3234344469056
21.8000001907349 20.1603400349213
21.8400001525879 19.8276357945306
21.8800001144409 19.4754804450949
21.9200000762939 19.2051117929677
21.960000038147 18.9938474312949
22 18.721095581921
22.0400002002716 18.4512012303585
22.0800001621246 18.4938177093354
22.1200001239777 17.9611231369909
22.1600000858307 17.754933347127
22.2000000476837 17.4812174621875
22.2400000095367 17.214181801628
22.2800002098083 17.009337681513
22.3200001716614 16.7405844179157
22.3600001335144 16.5288031747223
22.4000000953674 16.2550208639459
22.4400000572205 15.9829085847778
22.4800000190735 15.8021288543441
22.5199999809265 15.4949580946304
22.5600001811981 15.2908743615117
22.6000001430511 15.0139553841254
22.6400001049042 14.7405577121663
22.6800000667572 14.5543233667195
22.7200000286102 14.2531851706717
22.7599999904633 13.996866964746
22.8000001907349 13.783870561294
22.8400001525879 13.5042519236868
22.8800001144409 13.3530786450374
22.9200000762939 13.0192735586043
22.960000038147 12.8489053919311
23 12.5407619015896
23.0400002002716 12.2617572443505
23.0800001621246 12.1525235334029
23.1200001239777 11.9577532499134
23.1600000858307 11.4972395834266
23.2000000476837 11.377653780386
23.2400000095367 11.1342848793338
23.2800002098083 10.764351130525
23.3200001716614 10.70190527052
23.3600001335144 10.3203086557933
23.4000000953674 10.0949248431934
23.4400000572205 9.94701176399212
23.4800000190735 9.52337369089037
23.5199999809265 9.43607880381023
23.5600001811981 9.12278826884722
23.6000001430511 8.83677904091019
23.6400001049042 8.54867760191519
23.6800000667572 8.45163303987647
23.7200000286102 8.21153066469378
23.7599999904633 7.92316163603755
23.8000001907349 7.53254458390074
23.8400001525879 7.3937985609482
23.8800001144409 7.04001922885436
23.9200000762939 6.85277089686679
23.960000038147 6.54638944603357
24 6.41215320500305
24.0400002002716 6.01416069670813
24.0800001621246 5.91639514231247
24.1200001239777 5.56138548014133
24.1600000858307 5.30973078305942
24.2000000476837 5.19010770008754
24.2400000095367 4.83452820983257
24.2800002098083 4.73887402655782
24.3200001716614 4.44058631563712
24.3600001335144 4.25586748671594
24.4000000953674 4.19955539676812
24.4400000572205 4.04521880163156
24.4800000190735 3.95760035088824
24.5199999809265 3.72991129099319
24.5600001811981 3.57797092521832
24.6000001430511 3.54692020296898
24.6400001049042 3.05203488708139
24.6800000667572 2.96768736947342
24.7200000286102 2.55828956140977
24.7599999904633 2.25677871855078
24.8000001907349 2.06316063526356
24.8400001525879 1.52622432436328
24.8800001144409 1.42450394870502
24.9200000762939 1.09040791514053
24.960000038147 0.867810532912699
25 0.787722407485489
25.0400002002716 0.6973735436477
25.0800001621246 0.602877947292654
25.1200001239777 0.301773411913018
25.1600000858307 0.175121084845159
25.2000000476837 0.193748657839024
25.2400000095367 0.202185050175103
25.2800002098083 0.123075007362786
25.3200001716614 0.0156457769044209
25.3600001335144 0.0262138280122599
25.4000000953674 0.038151033104441
25.4400000572205 -0.0170885274874308
25.4800000190735 -0.0727373062454717
25.5199999809265 0.0140880338931311
25.5600001811981 -0.184653622014594
25.6000001430511 -0.236955668766313
25.6400001049042 -0.313192715680998
25.6800000667572 -0.539133643209425
25.7200000286102 -0.62519133702699
25.7599999904633 -0.785571496413468
25.8000001907349 -0.965875200237861
25.8400001525879 -1.20504169246669
25.8800001144409 -1.35574678109515
25.9200000762939 -1.47761137256202
25.960000038147 -1.89415930579526
26 -2.18874969756507
26.0400002002716 -2.53524233798635
26.0800001621246 -2.88899348344057
26.1200001239777 -3.18319836720366
26.1600000858307 -3.34269787828816
26.2000000476837 -3.4551393305901
26.2400000095367 -3.75037590070315
26.2800002098083 -3.90046827996957
26.3200001716614 -4.19344752401412
26.3600001335144 -4.34442746254354
26.4000000953674 -4.41793076970134
26.4400000572205 -4.62233715359798
26.4800000190735 -4.84928645275249
26.5199999809265 -5.04018979420289
26.5600001811981 -5.18588997182269
26.6000001430511 -5.29410446614202
26.6400001049042 -5.65567044169256
26.6800000667572 -5.81020059913317
26.7200000286102 -6.44169753260576
26.7599999904633 -6.57321300304284
26.8000001907349 -6.66905491330908
26.8400001525879 -6.92303286529568
26.8800001144409 -7.1837382522117
26.9200000762939 -7.43764217338503
26.960000038147 -7.59013303258143
27 -7.71430464165496
27.0400002002716 -8.11511324451203
27.0800001621246 -8.18078908899952
27.1200001239777 -8.59291285789186
27.1600000858307 -8.75312718403075
27.2000000476837 -8.89209180473335
27.2400000095367 -9.34487696104406
27.2800002098083 -9.40477989673309
27.3200001716614 -9.84328285912125
27.3600001335144 -10.0121988515566
27.4000000953674 -10.1575324731275
27.4400000572205 -10.5198671809085
27.4800000190735 -10.6593364139644
27.5199999809265 -11.1228665389553
27.5600001811981 -11.3009511535074
27.6000001430511 -11.4450129154975
27.6400001049042 -11.8259638117888
27.6800000667572 -11.9741990941238
27.7200000286102 -12.4201923588407
27.7599999904633 -12.6063091848428
27.8000001907349 -12.744839267667
27.8400001525879 -13.1226214781709
27.8800001144409 -13.26811134932
27.9200000762939 -13.7081152942865
27.960000038147 -13.9013549478186
28 -14.031722869433
28.0400002002716 -14.4857008491854
28.0800001621246 -14.5482300134117
28.1200001239777 -14.989597254742
28.1600000858307 -15.1812828324637
28.2000000476837 -15.3161401467132
28.2400000095367 -15.7027211175937
28.2800002098083 -15.8243539524337
28.3200001716614 -16.2614023603674
28.3600001335144 -16.4496491320879
28.4000000953674 -16.5795754933843
28.4400000572205 -16.9885586535365
28.4800000190735 -17.0871460088183
28.5199999809265 -17.5337730135634
28.5600001811981 -17.7258404268814
28.6000001430511 -17.8391770099352
28.6400001049042 -18.2792300030942
28.6800000667572 -18.3372227602304
28.7200000286102 -18.7875850943765
28.7599999904633 -18.966646818948
28.8000001907349 -19.0789943004893
28.8400001525879 -19.5296290901115
28.8800001144409 -19.5779580321941
28.9200000762939 -20.0307706867843
28.960000038147 -20.2094415304346
29 -20.329274779735
29.0400002002716 -20.7829151760325
29.0800001621246 -20.8514600270719
29.1200001239777 -21.29606071323
29.1600000858307 -21.4764859031711
29.2000000476837 -21.6006154368442
29.2400000095367 -22.04397066196
29.2800002098083 -22.1032559630439
29.3200001716614 -22.5465036126388
29.3600001335144 -22.7271658477785
29.4000000953674 -22.8584042771254
29.4400000572205 -23.3298960640313
29.4800000190735 -23.3803117962343
29.5199999809265 -23.8196697242493
29.5600001811981 -24.000737984521
29.6000001430511 -24.120171080629
29.6400001049042 -24.5609671461189
29.6800000667572 -24.613976538194
29.7200000286102 -25.0602445297445
29.7599999904633 -25.2402723911637
29.8000001907349 -25.350380686883
29.8400001525879 -25.7974325652121
29.8800001144409 -25.8449194284676
29.9200000762939 -26.2919462002501
29.960000038147 -26.4713486230939
30 -26.5891572474447
30.0400002002716 -27.0384627457734
30.0800001621246 -27.085822151912
30.1200001239777 -27.5276554665434
30.1600000858307 -27.7051624945307
30.2000000476837 -27.8220865840685
30.2400000095367 -28.2604995585916
30.2800002098083 -28.4663039977349
30.3200001716614 -28.50723830469
30.3600001335144 -28.8893539981473
30.4000000953674 -29.2303377878834
30.4400000572205 -29.4776080970605
30.4800000190735 -29.5241237622886
30.5199999809265 -29.9659232401609
30.5600001811981 -30.1411500857121
30.6000001430511 -30.2644369610498
30.6400001049042 -30.696888092114
30.6800000667572 -30.9012092473549
30.7200000286102 -31.179165681313
30.7599999904633 -31.3537292364963
30.8000001907349 -31.6670679562812
30.8400001525879 -31.7327897953492
30.8800001144409 -32.1076711571714
30.9200000762939 -32.4016810404628
30.960000038147 -32.5708997145193
31 -32.68662248896
31.0400002002716 -33.1101107898867
31.0800001621246 -33.3245547379538
31.1200001239777 -33.3940119977815
31.1600000858307 -33.7792782806591
31.2000000476837 -33.8941622725961
31.2400000095367 -34.2641448524082
31.2800002098083 -34.5294697985701
31.3200001716614 -34.6197988614749
31.3600001335144 -34.9410825236919
31.4000000953674 -35.3351562655953
31.4400000572205 -35.460242380108
31.4800000190735 -35.7578077735398
31.5199999809265 -36.0553079623751
31.5600001811981 -36.2124785794003
31.6000001430511 -36.339861883207
31.6400001049042 -36.6392979916636
31.6800000667572 -36.9714506718101
31.7200000286102 -37.2120387440227
31.7599999904633 -37.3800448817892
31.8000001907349 -37.7808287118187
31.8400001525879 -37.8529843019987
31.8800001144409 -38.1867113863664
31.9200000762939 -38.3927126456638
31.960000038147 -38.6663102513758
32 -38.9992190241483
32.0400002002716 -39.2862690839338
32.0800001621246 -39.2896692464965
32.1200001239777 -39.5738616783146
32.1600000858307 -39.9095866755324
32.2000000476837 -40.1674840022533
32.2400000095367 -40.2791464588245
32.2800002098083 -40.6763818520012
32.3200001716614 -41.0221587463711
32.3600001335144 -41.0924282124433
32.4000000953674 -41.364746152427
32.4400000572205 -41.5912162997502
32.4800000190735 -41.8292777579081
32.5199999809265 -42.0631080417843
32.5600001811981 -42.3207879184338
32.6000001430511 -42.5526869969035
32.6400001049042 -42.8052508141391
32.6800000667572 -43.083107164588
32.7200000286102 -43.1749218554123
32.7599999904633 -43.5542458971668
32.8000001907349 -43.6748024754779
32.8400001525879 -43.9905359527656
32.8800001144409 -44.3234813646541
32.9200000762939 -44.5254904218091
32.960000038147 -44.8058956548521
33 -45.0321085357282
33.0400002002716 -45.2845930197091
33.0800001621246 -45.5443714751727
33.1200001239777 -45.7856099574819
33.1600000858307 -46.0503223747673
33.2000000476837 -46.2860321341523
33.2400000095367 -46.5399599587145
33.2800002098083 -46.8401721027712
33.3200001716614 -46.9311127161327
33.3600001335144 -47.3093517807902
33.4000000953674 -47.4274854639716
33.4400000572205 -47.7881927215003
33.4800000190735 -48.0582099315816
33.5199999809265 -48.3752998567343
33.5600001811981 -48.5594607958938
33.6000001430511 -48.9085602364794
33.6400001049042 -49.0470561753627
33.6800000667572 -49.2931560794677
33.7200000286102 -49.5886182672257
33.7599999904633 -49.7895713949219
33.8000001907349 -50.1437683703589
33.8400001525879 -50.2811793235214
33.8800001144409 -50.5304631996423
33.9200000762939 -50.5304631996423
33.960000038147 -50.5304631996423
34 -50.5304631996423
34.0400002002716 -50.5304631996423
34.0800001621246 -50.5304631996423
34.1200001239777 -50.5304631996423
34.1600000858307 -50.5304631996423
};
\path [draw=black, fill=black] (axis cs:6,90)
--(axis cs:6.5,85)
--(axis cs:6.125,85)
--(axis cs:6.125,55)
--(axis cs:5.875,55)
--(axis cs:5.875,85)
--(axis cs:5.5,85)
--cycle;

\path [draw=black, fill=black] (axis cs:6,-30)
--(axis cs:5.5,-25)
--(axis cs:5.875,-25)
--(axis cs:5.875,0)
--(axis cs:6.125,0)
--(axis cs:6.125,-25)
--(axis cs:6.5,-25)
--cycle;

\addplot [line width=2.4000000000000004pt, color1, dashed, forget plot]
table {%
0 -26.7175784709144
0.0400002002716064 -26.7243654111636
0.0800001621246338 -26.7314819097196
0.120000123977661 -26.7396296546924
0.160000085830688 -26.7492617059173
0.200000047683716 -26.7596715582452
0.240000009536743 -26.7706567422049
0.28000020980835 -26.7825349440174
0.320000171661377 -26.7951511711675
0.360000133514404 -26.8089794848718
0.400000095367432 -26.8236069011274
0.440000057220459 -26.8388033854232
0.480000019073486 -26.855165851341
0.519999980926514 -26.8723440812869
0.56000018119812 -26.8895414941097
0.600000143051147 -26.9079118209845
0.640000104904175 -26.9280129088892
0.680000066757202 -26.9485278205402
0.720000028610229 -26.969098921781
0.759999990463257 -26.9901609043653
0.800000190734863 -27.0120413573074
0.840000152587891 -27.0343262985807
0.880000114440918 -27.0565240104
0.920000076293945 -27.0796372099052
0.960000038146973 -27.1038730281359
1 -27.1281820677234
1.04000020027161 -27.1537033707246
1.08000016212463 -27.1807314001008
1.12000012397766 -27.2082531779934
1.16000008583069 -27.2365396211998
1.20000004768372 -27.2651381415992
1.24000000953674 -27.2938742479172
1.28000020980835 -27.3230542043041
1.32000017166138 -27.3538782623908
1.3600001335144 -27.3848737659473
1.40000009536743 -27.4165966796883
1.44000005722046 -27.4484888402148
1.48000001907349 -27.4804090353351
1.51999998092651 -27.5130134987761
1.56000018119812 -27.5471488938731
1.60000014305115 -27.5818174223027
1.64000010490417 -27.6163206897717
1.6800000667572 -27.6507988524164
1.72000002861023 -27.6859239473408
1.75999999046326 -27.7222763434822
1.80000019073486 -27.7588030132149
1.84000015258789 -27.795280492802
1.88000011444092 -27.832156006054
1.92000007629395 -27.8703110392255
1.96000003814697 -27.9088722314961
2 -27.9476465745615
2.04000020027161 -27.9866356037659
2.08000016212463 -28.0259866874843
2.12000012397766 -28.0660989096316
2.16000008583069 -28.1060256169405
2.20000004768372 -28.1460512887932
2.24000000953674 -28.1871289558444
2.28000020980835 -28.2291543211637
2.32000017166138 -28.2706902289134
2.3600001335144 -28.3133015581842
2.40000009536743 -28.3562319003116
2.44000005722046 -28.3989189689248
2.48000001907349 -28.4423334011659
2.51999998092651 -28.4862761486587
2.56000018119812 -28.5296921362162
2.60000014305115 -28.573725529963
2.64000010490417 -28.6186927322598
2.6800000667572 -28.6631335481848
2.72000002861023 -28.7080418093041
2.75999999046326 -28.753405109255
2.80000019073486 -28.7985197262632
2.84000015258789 -28.8445866310379
2.88000011444092 -28.8910881919919
2.92000007629395 -28.9373496131688
2.96000003814697 -28.9842954516088
3 -29.0314746689432
3.04000020027161 -29.0786196870135
3.08000016212463 -29.1260262817257
3.12000012397766 -29.1730287659194
3.16000008583069 -29.2209547131265
3.20000004768372 -29.2687897823454
3.24000000953674 -29.3166431937737
3.28000020980835 -29.3649037912752
3.32000017166138 -29.4131615476884
3.3600001335144 -29.4543796584022
3.40000009536743 -29.5025891976201
3.44000005722046 -29.551406530664
3.48000001907349 -29.600235753921
3.51999998092651 -29.6483781659065
3.56000018119812 -29.6964589707737
3.60000014305115 -29.7438278290835
3.64000010490417 -29.7913891986586
3.6800000667572 -29.8391581981174
3.72000002861023 -29.8876280522222
3.75999999046326 -29.9366023347902
3.80000019073486 -29.9845253328877
3.84000015258789 -30.0447163463327
3.88000011444092 -30.0979140942435
3.92000007629395 -30.148712161975
3.96000003814697 -30.1996864593675
4 -30.2505206238766
4.04000020027161 -30.3012621398625
4.08000016212463 -30.3521311240913
4.12000012397766 -30.4025377770549
4.16000008583069 -30.4537460934843
4.20000004768372 -30.5051980261012
4.24000000953674 -30.5561673141323
4.28000020980835 -30.6072812105103
4.32000017166138 -30.6579393943304
4.3600001335144 -30.7094840677042
4.40000009536743 -30.7619495178642
4.44000005722046 -30.8140643056875
4.48000001907349 -30.8665931764311
4.51999998092651 -30.9194581863486
4.56000018119812 -30.9716001643005
4.60000014305115 -31.023226508554
4.64000010490417 -31.075111287077
4.6800000667572 -31.1261876440695
4.72000002861023 -31.1778425744644
4.75999999046326 -31.2289875960429
4.80000019073486 -31.2804444880027
4.84000015258789 -31.3321596301762
4.88000011444092 -31.3844463757854
4.92000007629395 -31.4358255308473
4.96000003814697 -31.4869021137902
5 -31.5387333021302
5.04000020027161 -31.5904573320213
5.08000016212463 -31.6413302909581
5.12000012397766 -31.6928656281997
5.16000008583069 -31.7438184730825
5.20000004768372 -31.7946022793142
5.24000000953674 -31.8454822043439
5.28000020980835 -31.8961485721698
5.32000017166138 -31.9476574228866
5.3600001335144 -31.9995819930828
5.40000009536743 -32.0513316295873
5.44000005722046 -32.1028003166617
5.48000001907349 -32.1541960382393
5.51999998092651 -32.2052423965311
5.56000018119812 -32.255519110957
5.60000014305115 -32.3056695085805
5.64000010490417 -32.3561526932468
5.6800000667572 -32.4056528191903
5.72000002861023 -32.4549889081071
5.75999999046326 -32.5050663938634
5.80000019073486 -32.5550115067317
5.84000015258789 -32.605484811527
5.88000011444092 -32.6564479603496
5.92000007629395 -32.7097608645677
5.96000003814697 -32.7601417538596
6 -32.8102613062449
6.04000020027161 -32.8604002328006
6.08000016212463 -32.9100463838552
6.12000012397766 -32.9584054948648
6.16000008583069 -33.0070599985852
6.20000004768372 -33.0554352734915
6.24000000953674 -33.1035913305636
6.28000020980835 -33.1517611737972
6.32000017166138 -33.1999653084559
6.3600001335144 -33.2485409212917
6.40000009536743 -33.296876641123
6.44000005722046 -33.344601821637
6.48000001907349 -33.391868881476
6.51999998092651 -33.439015511902
6.56000018119812 -33.4861574528945
6.60000014305115 -33.5332483209386
6.64000010490417 -33.5799430717033
6.6800000667572 -33.6258868896065
6.72000002861023 -33.6710781557121
6.75999999046326 -33.7155102456575
6.80000019073486 -33.7601340584818
6.84000015258789 -33.8043516016298
6.88000011444092 -33.8485590639651
6.92000007629395 -33.8931031511469
6.96000003814697 -33.9374640670378
7 -33.9807361190318
7.04000020027161 -34.0233188763494
7.08000016212463 -34.0655558002966
7.12000012397766 -34.106839446268
7.16000008583069 -34.1468173313807
7.20000004768372 -34.1864717683407
7.24000000953674 -34.2253422956951
7.28000020980835 -34.2631064651755
7.32000017166138 -34.300300190107
7.3600001335144 -34.3363280993882
7.40000009536743 -34.3724187204731
7.44000005722046 -34.4084271995274
7.48000001907349 -34.4431106610997
7.51999998092651 -34.4766956918303
7.56000018119812 -34.5093938848562
7.60000014305115 -34.5406944503131
7.64000010490417 -34.5702667278278
7.6800000667572 -34.5985707434129
7.72000002861023 -34.6262606769161
7.75999999046326 -34.6523912136979
7.80000019073486 -34.6769567954875
7.84000015258789 -34.7005756325328
7.88000011444092 -34.7233659845545
7.92000007629395 -34.7451899877822
7.96000003814697 -34.7658030944846
8 -34.7853549479639
8.04000020027161 -34.8033319743071
8.08000016212463 -34.8192808029148
8.12000012397766 -34.8335706129358
8.16000008583069 -34.8469139592736
8.20000004768372 -34.8587108913481
8.24000000953674 -34.867586575622
8.28000020980835 -34.8748569422997
8.32000017166138 -34.8805497281733
8.3600001335144 -34.8824542952168
8.40000009536743 -34.8798455382715
8.44000005722046 -34.873666412372
8.48000001907349 -34.8628158953046
8.51999998092651 -34.8466863821228
8.56000018119812 -34.8275411426009
8.60000014305115 -34.8053648371975
8.64000010490417 -34.7791264964841
8.6800000667572 -34.7492068263622
8.72000002861023 -34.7161259823991
8.75999999046326 -34.6782454945154
8.80000019073486 -34.6361135090503
8.84000015258789 -34.5888792364651
8.88000011444092 -34.5364818580104
8.92000007629395 -34.4790036827147
8.96000003814697 -34.4147249228028
9 -34.3455024488127
9.04000020027161 -34.2710478500255
9.08000016212463 -34.1915866134504
9.12000012397766 -34.1076928276288
9.16000008583069 -34.0194423498562
9.20000004768372 -33.9261176367307
9.24000000953674 -33.8286552026135
9.28000020980835 -33.7260992103973
9.32000017166138 -33.6183726355431
9.3600001335144 -33.5055776924763
9.40000009536743 -33.389199558747
9.44000005722046 -33.2630290904509
9.48000001907349 -33.1340475901303
9.51999998092651 -33.0042606404805
9.56000018119812 -32.8698292856764
9.60000014305115 -32.7308726052715
9.64000010490417 -32.5872802419654
9.6800000667572 -32.4393218609968
9.72000002861023 -32.2878577105017
9.75999999046326 -32.1314953083368
9.80000019073486 -31.9712223007574
9.84000015258789 -31.8070259010082
9.88000011444092 -31.637812982183
9.92000007629395 -31.4635577050028
9.96000003814697 -31.2847233424162
10 -31.1012464370412
10.0400002002716 -30.9129084975665
10.0800001621246 -30.7206633874621
10.1200001239777 -30.5080851557315
10.1600000858307 -30.3026355220615
10.2000000476837 -30.0952073284805
10.2400000095367 -29.883083479919
10.2800002098083 -29.6661798292695
10.3200001716614 -29.4448510742777
10.3600001335144 -29.2186696969833
10.4000000953674 -28.9885889505336
10.4400000572205 -28.7527615779134
10.4800000190735 -28.5118978725388
10.5199999809265 -28.2658791236296
10.5600001811981 -28.0153315898673
10.6000001430511 -27.7763634059292
10.6400001049042 -27.5356806876965
10.6800000667572 -27.2731808875858
10.7200000286102 -27.0058550283188
10.7599999904633 -26.733252490944
10.8000001907349 -26.456593929253
10.8400001525879 -26.1748016254791
10.8800001144409 -25.8878069286538
10.9200000762939 -25.5959393676679
10.960000038147 -25.2987013081406
11 -24.996724021636
11.0400002002716 -24.6891997240644
11.0800001621246 -24.3759738521063
11.1200001239777 -24.0586057887277
11.1600000858307 -23.7371646739414
11.2000000476837 -23.4104823437431
11.2400000095367 -23.0787437115736
11.2800002098083 -22.7417706339265
11.3200001716614 -22.3994447528069
11.3600001335144 -22.0520808524588
11.4000000953674 -21.699383901879
11.4400000572205 -21.3415550215697
11.4800000190735 -20.9792017832518
11.5199999809265 -20.6116059174002
11.5600001811981 -20.2383120643975
11.6000001430511 -19.8589073313936
11.6400001049042 -19.473864948804
11.6800000667572 -19.0838958079899
11.7200000286102 -18.6890108894782
11.7599999904633 -18.3236105651357
11.8000001907349 -17.9168212486995
11.8400001525879 -17.5049482807633
11.8800001144409 -17.0874868962303
11.9200000762939 -16.6637963913717
11.960000038147 -16.2350335467941
12 -15.8013691082273
12.0400002002716 -15.3623497444628
12.0800001621246 -14.9179127721508
12.1200001239777 -14.468370282117
12.1600000858307 -14.0138620187464
12.2000000476837 -13.5540306968045
12.2400000095367 -13.0874726981346
12.2800002098083 -12.6148790125299
12.3200001716614 -12.1357105529123
12.3600001335144 -11.651978880032
12.4000000953674 -11.1612168650178
12.4400000572205 -10.6643743121166
12.4800000190735 -10.161138721148
12.5199999809265 -9.65164671759392
12.5600001811981 -9.13592841995691
12.6000001430511 -8.61487364642503
12.6400001049042 -8.08845840778686
12.6800000667572 -7.55615312596888
12.7200000286102 -7.01792091268072
12.7599999904633 -6.47393601384779
12.8000001907349 -5.92328944757777
12.8400001525879 -5.36568652184827
12.8800001144409 -4.80216298679045
12.9200000762939 -4.23284219297128
12.960000038147 -3.65636325993022
13 -3.0730742599296
13.0400002002716 -2.48387735877781
13.0800001621246 -1.88709491644232
13.1200001239777 -1.28322392873598
13.1600000858307 -0.672801261541959
13.2000000476837 -0.981820537729377
13.2400000095367 -0.468522907250198
13.2800002098083 0.0446833584366141
13.3200001716614 0.557849732536402
13.3600001335144 1.06967309875755
13.4000000953674 1.58154752071255
13.4400000572205 2.0923537472183
13.4800000190735 2.60270794857555
13.5199999809265 3.11289063153797
13.5600001811981 3.62207867240508
13.6000001430511 4.13190154529372
13.6400001049042 4.64196460456484
13.6800000667572 5.15203964767281
13.7200000286102 5.66181045819619
13.7599999904633 6.17148873023899
13.8000001907349 6.68137740469866
13.8400001525879 7.19140232840999
13.8800001144409 7.70084992257853
13.9200000762939 8.21012704018239
13.960000038147 8.72019913656431
14 9.22962413778846
14.0400002002716 9.73902227233729
14.0800001621246 10.2484609302266
14.1200001239777 10.7576769346328
14.1600000858307 11.2678786986544
14.2000000476837 11.7770772179533
14.2400000095367 12.2866417357034
14.2800002098083 12.7968880927004
14.3200001716614 13.3067907877759
14.3600001335144 13.8161381052737
14.4000000953674 14.3252107369533
14.4400000572205 14.8347909771295
14.4800000190735 15.3446154806592
14.5199999809265 15.8549520637096
14.5600001811981 16.3649879538904
14.6000001430511 16.8747218303066
14.6400001049042 17.3847069569655
14.6800000667572 17.895684847981
14.7200000286102 18.4068534575592
14.7599999904633 18.9176717396362
14.8000001907349 19.4287604379424
14.8400001525879 19.9399021588383
14.8800001144409 20.4514173411147
14.9200000762939 20.9632937469633
14.960000038147 21.4750348337541
15 21.9867619200529
15.2000000476837 24.5449711406067
15.2400000095367 25.0570218488948
15.2800002098083 25.5691261145978
15.3200001716614 26.0808551908654
15.3600001335144 26.5920419427237
15.4000000953674 27.1037574969091
15.4400000572205 27.6153499514776
15.4800000190735 28.1267311989635
15.5199999809265 28.6374870229249
15.5600001811981 29.1491106356454
15.6000001430511 29.6608947026609
15.6400001049042 30.1720400793853
15.6800000667572 30.6829803473988
15.7200000286102 31.1941064111879
15.7599999904633 31.7056245879031
15.8000001907349 32.2172509859221
15.8400001525879 32.7289004065284
15.8800001144409 33.2407355463509
15.9200000762939 33.7519445825988
15.960000038147 34.2629420980836
16 34.7749320287061
16.0400002002716 35.2876253452188
16.0800001621246 35.7997808496254
16.1200001239777 36.3133418946666
16.1600000858307 36.8267206447312
16.2000000476837 37.3405706096815
16.2400000095367 37.8554402880879
16.2800002098083 38.3720861217492
16.3200001716614 38.8886768887058
16.3600001335144 39.4048743391815
16.4000000953674 39.9216817120439
16.4400000572205 40.4382215021454
16.4800000190735 40.9553988563104
16.5199999809265 41.4717039415344
16.5600001811981 41.9889230478244
16.6000001430511 42.5051050464569
16.6400001049042 43.0233282872727
16.6800000667572 43.5406649139557
16.7200000286102 44.0572084853081
16.7599999904633 44.573313679281
16.8000001907349 45.0903697521
16.8400001525879 45.6074624698763
16.8800001144409 46.1236257137711
16.9200000762939 46.6391175447807
16.960000038147 47.1560855076011
17 47.6714883392016
17.0400002002716 48.1859845883444
17.0800001621246 48.697077230755
17.1200001239777 49.2095369730903
17.1600000858307 49.7214166934099
17.2000000476837 50.2324364773054
17.2400000095367 50.7427929221044
17.2800002098083 51.2516060450493
17.3200001716614 51.7596903035496
17.3600001335144 52.2665872159685
17.4000000953674 52.7704038953291
17.4400000572205 53.2736143384863
17.4800000190735 53.7759850683972
17.5199999809265 54.2761956571994
17.5600001811981 54.7744491186442
17.6000001430511 55.2723050524881
17.6400001049042 55.7694647277533
17.6800000667572 56.2641732957044
17.7200000286102 56.7577889035433
};
\addplot [line width=2.4000000000000004pt, color3, dashed, forget plot]
table {%
0.800000190734863 98.4670024716349
0.840000152587891 98.5169748869348
0.880000114440918 98.5661548942805
0.920000076293945 98.613630211953
0.960000038146973 98.6588800329536
1 98.7030637911806
1.04000020027161 98.7455444295213
1.08000016212463 98.785676140083
1.12000012397766 98.8243346166123
1.16000008583069 98.8609503282723
1.20000004768372 98.8961412006541
1.24000000953674 98.9303025480536
1.28000020980835 98.9629949841733
1.32000017166138 98.9927879419047
1.3600001335144 99.0213014956195
1.40000009536743 99.0479407530052
1.44000005722046 99.0730986015163
1.48000001907349 99.0972646065405
1.51999998092651 99.1196303103129
1.56000018119812 99.1391741621427
1.60000014305115 99.1570876040575
1.64000010490417 99.1740598807678
1.6800000667572 99.190113754077
1.72000002861023 99.2044340941243
1.75999999046326 99.2161690822956
1.80000019073486 99.2267858895617
1.84000015258789 99.236316482539
1.88000011444092 99.244388503858
1.92000007629395 99.2503205402758
1.96000003814697 99.2548829550534
2 99.2580665982567
2.04000020027161 99.2598714979143
2.08000016212463 99.2603047146009
2.12000012397766 99.2590128357564
2.16000008583069 99.2568243783294
2.20000004768372 99.2533642873841
2.24000000953674 99.2478442791067
2.28000020980835 99.2402824901061
2.32000017166138 99.2318441175966
2.3600001335144 99.2197192972612
2.40000009536743 99.2003066013635
2.44000005722046 99.1898444730218
2.48000001907349 99.1774171945759
2.51999998092651 99.163247909125
2.56000018119812 99.1475199015381
2.60000014305115 99.128598799319
2.64000010490417 99.1075906357488
2.6800000667572 99.085974238725
2.72000002861023 99.0628181576057
2.75999999046326 99.0380499194235
2.80000019073486 99.0126361427323
2.84000015258789 98.9852418035048
2.88000011444092 98.9610938897675
2.92000007629395 98.9322208449223
2.96000003814697 98.8988129590737
3 98.8638732841332
3.04000020027161 98.8281081543998
3.08000016212463 98.7919809982292
3.12000012397766 98.755472004108
3.16000008583069 98.7167661378882
3.20000004768372 98.6768913731598
3.24000000953674 98.6357561558711
3.28000020980835 98.592949158699
3.32000017166138 98.5541864373908
3.3600001335144 98.5201831307642
3.40000009536743 98.4727776502759
3.44000005722046 98.4235040884178
3.48000001907349 98.3729170667096
3.51999998092651 98.3217188388474
3.56000018119812 98.2692976906824
3.60000014305115 98.2162279277546
3.64000010490417 98.1616351318129
3.6800000667572 98.1054725284021
3.72000002861023 98.0472461729178
3.75999999046326 97.9871878500372
3.80000019073486 97.9267484085787
3.84000015258789 97.8699113129988
3.88000011444092 97.8045022065906
3.92000007629395 97.7394355198724
3.96000003814697 97.672948296758
4 97.605356078995
4.04000020027161 97.5366080730579
4.08000016212463 97.4664817203106
4.12000012397766 97.3955555067826
4.16000008583069 97.3225725425259
4.20000004768372 97.248085455138
4.24000000953674 97.1728248082475
4.28000020980835 97.0961645830203
4.32000017166138 97.0187030574996
4.3600001335144 96.9390879078834
4.40000009536743 96.8572862898682
4.44000005722046 96.7745612238201
4.48000001907349 96.6901483912466
4.51999998092651 96.6041267744225
4.56000018119812 96.5175509829555
4.60000014305115 96.4302149199894
4.64000010490417 96.3413416207194
4.6800000667572 96.2519864336774
4.72000002861023 96.1607573538018
4.75999999046326 96.0687420167527
4.80000019073486 95.9750985559895
4.84000015258789 95.8798933108926
4.88000011444092 95.7828046709905
4.92000007629395 95.6853009684441
4.96000003814697 95.5867913716209
5 95.4861815774712
5.04000020027161 95.3843546989679
5.08000016212463 95.282053203149
5.12000012397766 95.1777576952191
5.16000008583069 95.0639648450405
5.20000004768372 94.9576778009706
5.24000000953674 94.8499733686834
5.28000020980835 94.7411893085438
5.32000017166138 94.6302324423731
5.3600001335144 94.5175228310947
5.40000009536743 94.4036955241319
5.44000005722046 94.288848144467
5.48000001907349 94.1727598536971
5.51999998092651 94.0557340493483
5.56000018119812 93.9380092174235
5.60000014305115 93.8190470062695
5.64000010490417 93.6984139837879
5.6800000667572 93.5774317470792
5.72000002861023 93.4552341514443
5.75999999046326 93.3309203321973
5.80000019073486 93.2053554249511
5.84000015258789 93.0779055152263
5.88000011444092 92.9485974633043
5.92000007629395 92.8179642101187
5.96000003814697 92.6865352022059
6 92.554081445325
6.04000020027161 92.4202802702756
6.08000016212463 92.2857277603797
6.12000012397766 92.1510610967377
6.16000008583069 92.0147380421569
6.20000004768372 91.8773410379419
6.24000000953674 91.7387436982055
6.28000020980835 91.5987624686678
6.32000017166138 91.4573284880189
6.3600001335144 91.3141033442817
6.40000009536743 91.1696573518965
6.44000005722046 91.0243925347928
6.48000001907349 90.8781396060415
6.51999998092651 90.7305633770463
6.56000018119812 90.5815824396431
6.60000014305115 90.4312036431199
6.64000010490417 90.2797736277853
6.6800000667572 90.1275918851744
6.72000002861023 89.9746724102878
6.75999999046326 89.8210080983227
6.80000019073486 89.665627304324
6.84000015258789 89.5090724908122
6.88000011444092 89.3509630703687
6.92000007629395 89.1908758437491
6.96000003814697 89.0292746647332
7 88.8670236740101
7.04000020027161 88.7036952997758
7.08000016212463 88.5389301515697
7.12000012397766 88.3733043647187
7.16000008583069 88.2071579592573
7.20000004768372 88.0394538754078
7.24000000953674 87.8706229516511
7.28000020980835 87.7009029860303
7.32000017166138 87.5297946350003
7.3600001335144 87.3578410031378
7.40000009536743 87.1837948581215
7.44000005722046 87.0077722207348
7.48000001907349 86.831029525282
7.51999998092651 86.6532577564752
7.56000018119812 86.4742367149087
7.60000014305115 86.2944186203099
7.64000010490417 86.1140697337577
7.6800000667572 85.9327182712404
7.72000002861023 85.7497082973844
7.75999999046326 85.5659332994442
7.80000019073486 85.3813725872864
7.84000015258789 85.1953622515263
7.88000011444092 85.0077473288753
7.92000007629395 84.8186522312267
7.96000003814697 84.6282051018267
8 84.4363244756515
8.04000020027161 84.2434864991791
8.08000016212463 84.0501112738793
8.12000012397766 83.8558380664234
8.16000008583069 83.6599181542029
8.20000004768372 83.4628962395856
8.24000000953674 83.2660423047039
8.28000020980835 83.0680871561554
8.32000017166138 82.8689304550081
8.3600001335144 82.6707777737
8.40000009536743 82.4743124070684
8.44000005722046 82.2785561552435
8.48000001907349 82.0846674602935
8.51999998092651 81.893173257272
8.56000018119812 81.7016932261708
8.60000014305115 81.5102935349486
8.64000010490417 81.3199131685562
8.6800000667572 81.1302067987531
8.72000002861023 80.9406465873573
8.75999999046326 80.7529353241116
8.80000019073486 80.5663982845304
8.84000015258789 80.3819675956145
8.88000011444092 80.1995183719054
8.92000007629395 80.0159243942424
8.96000003814697 79.8185691758818
9 79.6459485390124
9.04000020027161 79.4752780620898
9.08000016212463 79.3061366480732
9.12000012397766 79.1378981422122
9.16000008583069 78.9704871436514
9.20000004768372 78.8047069716418
9.24000000953674 78.6392208433471
9.28000020980835 78.4751198010658
9.32000017166138 78.3123855690568
9.3600001335144 78.1510043399563
9.40000009536743 77.989458591504
9.44000005722046 77.8386203529117
9.48000001907349 77.6806116950832
9.51999998092651 77.5202782915455
9.56000018119812 77.3608046831388
9.60000014305115 77.2020321180262
9.64000010490417 77.0440442134533
9.6800000667572 76.8865400252419
9.72000002861023 76.7286397521739
9.75999999046326 76.5717044974751
9.80000019073486 76.4147141529648
9.84000015258789 76.2576472819034
9.88000011444092 76.1015740792245
9.92000007629395 75.9464874939063
9.96000003814697 75.7921016096959
10 75.6385552138407
10.0400002002716 75.4858546427161
10.0800001621246 75.3329201654464
10.1200001239777 75.1924631967391
10.1600000858307 75.0406770361307
10.2000000476837 74.8884593127462
10.2400000095367 74.7365396265843
10.2800002098083 74.5849463965605
10.3200001716614 74.4333114650282
10.3600001335144 74.2819864370284
10.4000000953674 74.1300641338912
10.4400000572205 73.979281452157
10.4800000190735 73.8288256508433
10.5199999809265 73.6787487940711
10.5600001811981 73.5284299967886
10.6000001430511 73.3789897727469
10.6400001049042 73.2267551850043
10.6800000667572 73.0778068910665
10.7200000286102 72.9290270437336
10.7599999904633 72.7808463761096
10.8000001907349 72.6319497334398
10.8400001525879 72.4834519860795
10.8800001144409 72.3353376693574
10.9200000762939 72.1872866553379
10.960000038147 72.039802949774
11 71.8922597999429
11.0400002002716 71.7454207904944
11.0800001621246 71.5994095229652
11.1200001239777 71.4527566892723
11.1600000858307 71.3052082219505
11.2000000476837 71.1579215931025
11.2400000095367 71.0107045146287
11.2800002098083 70.8638152619535
11.3200001716614 70.717289730343
11.3600001335144 70.5707591259613
11.4000000953674 70.4245539252859
11.4400000572205 70.2783926485193
11.4800000190735 70.1316701794605
11.5199999809265 69.9849762440033
11.5600001811981 69.8387073080091
11.6000001430511 69.6931381029706
11.6400001049042 69.5478623652246
11.6800000667572 69.4022134217382
11.7200000286102 69.2561483377044
11.7599999904633 69.1102724301831
11.8000001907349 68.9649222384192
11.8400001525879 68.8192263816022
11.8800001144409 68.6735422662001
11.9200000762939 68.5284844528542
11.960000038147 68.3827930350985
12 68.2362383864101
12.0400002002716 68.0894205256793
12.0800001621246 67.9423669310506
12.1200001239777 67.794734312554
12.1600000858307 67.6462795602815
12.2000000476837 67.4973693623721
12.2400000095367 67.3493441470332
12.2800002098083 67.2014003181668
12.3200001716614 67.0541908630437
12.3600001335144 66.9054701243777
12.4000000953674 66.7577632793846
12.4400000572205 66.6101986779947
12.4800000190735 66.463149810695
12.5199999809265 66.3163454757845
12.5600001811981 66.1695678643646
12.6000001430511 66.0220915141858
12.6400001049042 65.8738982599664
12.6800000667572 65.7254811960109
12.7200000286102 65.5768230881815
12.7599999904633 65.4276667709508
12.8000001907349 65.2788806466902
12.8400001525879 65.1305157776363
12.8800001144409 64.9817584279882
12.9200000762939 64.8323102462638
12.960000038147 64.6834108515181
13 64.5349600607666
13.0400002002716 64.3860417403307
13.0800001621246 64.2380136290039
13.1200001239777 64.0906838257893
13.1600000858307 63.9433392101888
13.2000000476837 63.7956089437667
13.2400000095367 63.6475936222517
13.2800002098083 63.4995658167098
13.3200001716614 63.3514865520987
13.3600001335144 63.2026009178869
13.4000000953674 63.0537198513713
13.4400000572205 62.9041047232998
13.4800000190735 62.7540761381627
13.5199999809265 62.6039487562406
13.5600001811981 62.4532070061192
13.6000001430511 62.3030951118317
13.6400001049042 62.1533079538585
13.6800000667572 62.003517797381
13.7200000286102 61.8533550807895
13.7599999904633 61.7029337804976
13.8000001907349 61.5528526640041
13.8400001525879 61.402916172863
13.8800001144409 61.2523296233677
13.9200000762939 61.1017147982333
13.960000038147 60.951861792942
14 60.8013467342219
14.0400002002716 60.6507350562406
14.0800001621246 60.5001274098655
14.1200001239777 60.3492474137607
14.1600000858307 60.1993763110148
14.2000000476837 60.048823438423
14.2400000095367 59.8984447075001
14.2800002098083 59.7486147295642
14.3200001716614 59.5984759464258
14.3600001335144 59.4477354994688
14.4000000953674 59.2967576155703
14.4400000572205 59.1461643122355
14.4800000190735 58.9959115595067
14.5199999809265 58.8460662966637
14.5600001811981 58.6958921420776
14.6000001430511 58.5453702916626
14.6400001049042 58.395050801376
14.6800000667572 58.2452625824402
14.7200000286102 58.0953989520966
14.7599999904633 57.9451836501322
14.8000001907349 57.7948931876925
14.8400001525879 57.6446118090856
14.8800001144409 57.4944958659649
14.9200000762939 57.3446740680802
14.960000038147 57.1947936559765
15 57.044604720098
15.0400002002716 56.8939980573229
15.0800001621246 56.7437428418436
15.1200001239777 56.5935502640086
15.1600000858307 56.4430460092374
15.2000000476837 56.2929598666517
15.2400000095367 56.1433238626947
15.2800002098083 55.9939941679808
15.3200001716614 55.844270757937
15.3600001335144 55.6941740132381
15.4000000953674 55.544583970567
15.4400000572205 55.3949072091932
15.4800000190735 55.2448947647196
15.5199999809265 55.0944763914408
15.5600001811981 54.9448423144326
15.6000001430511 54.7955530360132
15.6400001049042 54.6456912644268
15.6800000667572 54.4956420518728
15.7200000286102 54.3456112188415
15.7599999904633 54.1958769863608
15.8000001907349 54.0462500671953
15.8400001525879 53.8968354463794
15.8800001144409 53.7474103591022
15.9200000762939 53.5976072275366
15.960000038147 53.4476476668314
16 53.2981687676654
16.0400002002716 53.1492925706838
16.0800001621246 52.999678113339
16.1200001239777 52.8512196472918
16.1600000858307 52.7024550605536
16.2000000476837 52.5539341288829
16.2400000095367 52.4060414597341
16.2800002098083 52.2583446882627
16.3200001716614 52.1111194161779
16.3600001335144 51.9641010766068
16.4000000953674 51.8169784441936
16.4400000572205 51.6699685370563
16.4800000190735 51.5230771822245
16.5199999809265 51.3761218892204
16.5600001811981 51.2285229299956
16.6000001430511 51.0804731648315
16.6400001049042 50.9337741152705
16.6800000667572 50.7862195242433
16.7200000286102 50.6382046970789
16.7599999904633 50.4907222458212
16.8000001907349 50.3446697290132
16.8400001525879 50.1984131125929
16.8800001144409 50.051054224066
16.9200000762939 49.9037519040805
16.960000038147 49.755397923853
17 49.6068105858828
17.0400002002716 49.4573050487354
17.0800001621246 49.3061502206948
17.1200001239777 49.1550027902058
17.1600000858307 49.0048709163076
17.2000000476837 48.852641478701
17.2400000095367 48.6986306217989
17.2800002098083 48.5440370882232
17.3200001716614 48.3886240018323
17.3600001335144 48.23193953349
17.4000000953674 48.0734106643344
17.4400000572205 47.9144478514588
17.4800000190735 47.7562073094867
17.5199999809265 47.5946606257051
17.5600001811981 47.431790170853
17.6000001430511 47.2670330121479
17.6400001049042 47.1018664200556
17.6800000667572 46.9332430545997
17.7200000286102 46.7660174240315
17.7599999904633 46.5975353058177
17.8000001907349 46.4267638068555
17.8400001525879 46.2531607964303
17.8800001144409 46.077267510025
17.9200000762939 45.8991875189626
17.960000038147 45.722294668217
18 45.543622607034
18.0400002002716 45.3650505999639
18.0800001621246 45.1842251075944
18.1200001239777 45.0051440581585
18.1600000858307 44.8200213045979
18.2000000476837 44.6327130327441
18.2400000095367 44.4447555663842
18.2800002098083 44.254379168751
18.3200001716614 44.0622959194873
18.3600001335144 43.8689918678519
18.4000000953674 43.6760996333684
18.4400000572205 43.4833980832013
18.4800000190735 43.2893416364304
18.5199999809265 43.095664825583
18.5600001811981 42.8995947437503
18.6000001430511 42.7041676806963
18.6400001049042 42.5084005910851
18.6800000667572 42.314991858598
18.7200000286102 42.1193528951251
18.7599999904633 41.9214745628355
18.8000001907349 41.7240426170165
18.8400001525879 41.5243858350389
18.8800001144409 41.3229937204059
18.9200000762939 41.1198373728949
18.960000038147 40.9112298467049
19 40.7019909212328
19.0400002002716 40.4942200217757
19.0800001621246 40.2835412111592
19.1200001239777 40.072381044106
19.1600000858307 39.8590164680949
19.2000000476837 39.6478638405936
19.2400000095367 39.4323664821268
19.2800002098083 39.2176911736039
19.3200001716614 39.0016570376651
19.3600001335144 38.7845863900593
19.4000000953674 38.5676758153006
19.4400000572205 38.3506661522198
19.4800000190735 38.1341282459435
19.5199999809265 37.9164397756176
19.5600001811981 37.6980924387244
19.6000001430511 37.479199501517
19.6400001049042 37.2575057106451
19.6800000667572 37.0364933234289
19.7200000286102 36.8146236481114
19.7599999904633 36.5956681590111
19.8000001907349 36.3751568928819
19.8400001525879 36.1549223409246
19.8800001144409 35.9339375489866
19.9200000762939 35.7129048694288
19.960000038147 35.4897330502412
20 35.2646249460833
20.0400002002716 35.0387595313637
20.0800001621246 34.8139125952622
20.1200001239777 34.5894034406006
20.1600000858307 34.3655352549276
20.2000000476837 34.1420788756263
20.2400000095367 33.9170337776565
20.2800002098083 33.6903057439378
20.3200001716614 33.4638657364178
20.3600001335144 33.2372871876101
20.4000000953674 33.0090270144152
20.4400000572205 32.7824364196696
20.4800000190735 32.5559479236025
20.5199999809265 32.3303061881762
20.5600001811981 32.1034282225988
20.6000001430511 31.876611251424
20.6400001049042 31.6494114588178
20.6800000667572 31.4208073064913
20.7200000286102 31.192461218549
20.7599999904633 30.9647712163794
20.8000001907349 30.7362623122821
20.8400001525879 30.5070405822846
20.8800001144409 30.2778827732976
20.9200000762939 30.0490625062096
20.960000038147 29.8192703009792
21 29.5896738245832
21.0400002002716 29.3599984727721
21.0800001621246 29.1294709172505
21.1200001239777 28.8985321584902
21.1600000858307 28.6668588155729
21.2000000476837 28.433978287122
21.2400000095367 28.2007632955362
21.2800002098083 27.9674484738761
21.3200001716614 27.7333026924032
21.3600001335144 27.499591823704
21.4000000953674 27.2652653920801
21.4400000572205 27.0310258047683
21.4800000190735 26.7960456536657
21.5199999809265 26.5624941556389
21.5600001811981 26.3289779508727
21.6000001430511 26.0946248552391
21.6400001049042 25.8599727904865
21.6800000667572 25.624759919285
21.7200000286102 25.3888514343276
21.7599999904633 25.1528444850872
21.8000001907349 24.9158050077957
21.8400001525879 24.6783792618272
21.8800001144409 24.4408909105768
21.9200000762939 24.2032299915826
21.960000038147 23.9654766669277
22 23.7277617975347
22.0400002002716 23.4892237719966
22.0800001621246 23.2512023712409
22.1200001239777 23.013354556714
22.1600000858307 22.7752016894609
22.2000000476837 22.5375388317801
22.2400000095367 22.2986936032705
22.2800002098083 22.0594737987944
22.3200001716614 21.8195865181485
22.3600001335144 21.5791571807305
22.4000000953674 21.3386188644697
22.4400000572205 21.0981567855121
22.4800000190735 20.8580991948833
22.5199999809265 20.6169331468284
22.5600001811981 20.3759994989055
22.6000001430511 20.1352263028836
22.6400001049042 19.8941725935706
22.6800000667572 19.6523137269056
22.7200000286102 19.4107871709917
22.7599999904633 19.1692058617431
22.8000001907349 18.9276216459298
22.8400001525879 18.6859244654426
22.8800001144409 18.4439297242193
22.9200000762939 18.2016466933751
22.960000038147 17.960093360464
23 17.7177542080551
23.0400002002716 17.4755357397136
23.0800001621246 17.2332222021739
23.1200001239777 16.9902839787805
23.1600000858307 16.7474573589705
23.2000000476837 16.5047817751036
23.2400000095367 16.2621666136114
23.2800002098083 16.0193926336377
23.3200001716614 15.7765379930693
23.3600001335144 15.5339027566252
23.4000000953674 15.2907396299897
23.4400000572205 15.0471475940689
23.4800000190735 14.8041685650283
23.5199999809265 14.5589955819545
23.5600001811981 14.3138971755172
23.6000001430511 14.068146889515
23.6400001049042 13.8222597809408
23.6800000667572 13.5764334879395
23.7200000286102 13.330290156528
23.7599999904633 13.0835966938341
23.8000001907349 12.8371590176302
23.8400001525879 12.5906051946063
23.8800001144409 12.3445507942396
23.9200000762939 12.0984322257961
23.960000038147 11.8524843941314
24 11.6074917342831
24.0400002002716 11.3636319043129
24.0800001621246 11.1206324432499
24.1200001239777 10.8775115130967
24.1600000858307 10.6350083978169
24.2000000476837 10.3917792268632
24.2400000095367 10.1482747411077
24.2800002098083 9.90460165948608
24.3200001716614 9.66133147654945
24.3600001335144 9.41817752271743
24.4000000953674 9.17464150807697
24.4400000572205 8.93092919987336
24.4800000190735 8.68775264480631
24.5199999809265 8.44470575502963
24.5600001811981 8.20199260678111
24.6000001430511 7.95888183828268
24.6400001049042 7.71612471367198
24.6800000667572 7.47370509958245
24.7200000286102 7.23171100614825
24.7599999904633 6.98996608117357
24.8000001907349 6.74789298459445
24.8400001525879 6.50612555655721
24.8800001144409 6.26400946992883
24.9200000762939 6.02186507016333
24.960000038147 5.77993889129898
25 5.53758745554304
25.0400002002716 5.2923292598475
25.0800001621246 5.05002070109587
25.1200001239777 4.80701453939024
25.1600000858307 4.56439410468632
25.2000000476837 4.32180295606798
25.2400000095367 4.07871253660878
25.2800002098083 3.83594516362961
25.3200001716614 3.59278689965685
25.3600001335144 3.35010068238296
25.4000000953674 3.10724186135652
25.4400000572205 2.86491904207999
25.4800000190735 2.62245380005268
25.5199999809265 2.37990273797251
25.5600001811981 2.13782890384934
25.6000001430511 1.89498975229364
25.6400001049042 1.65160842260097
25.6800000667572 1.408256156337
25.7200000286102 1.16463578392898
25.7599999904633 0.920993583457479
25.8000001907349 0.6770381676511
25.8400001525879 0.43342182127281
25.8800001144409 0.189484709356958
25.9200000762939 -0.0544583573771354
25.960000038147 -0.298338481654085
26 -0.542456463314144
26.0400002002716 -0.786389706079018
26.0800001621246 -1.02993609504475
26.1200001239777 -1.27303516145216
26.1600000858307 -1.51594567836876
26.2000000476837 -1.75957576911505
26.2400000095367 -2.00374549573292
26.2800002098083 -2.24848050975066
26.3200001716614 -2.49331721407037
26.3600001335144 -2.73838887357165
26.4000000953674 -2.98343255211792
26.4400000572205 -3.22852086766941
26.4800000190735 -3.47386514749146
26.5199999809265 -3.7191280648463
26.5600001811981 -3.96402393705579
26.6000001430511 -4.2087186082854
26.6400001049042 -4.45319730591268
26.6800000667572 -4.6978665934501
26.7200000286102 -4.94249917702789
26.7599999904633 -5.18662081257298
26.8000001907349 -5.43068479859753
26.8400001525879 -5.67468826277356
26.8800001144409 -5.91873280828941
26.9200000762939 -6.16275801735898
26.960000038147 -6.40750594386061
27 -6.6519937395158
27.0400002002716 -6.89640208545963
27.0800001621246 -7.14152812727675
27.1200001239777 -7.3864304645406
27.1600000858307 -7.6308522443735
27.2000000476837 -7.87527378685528
27.2400000095367 -8.11984707059406
27.2800002098083 -8.36399522705321
27.3200001716614 -8.60842286648267
27.3600001335144 -8.85176928476948
27.4000000953674 -9.09501125286461
27.4400000572205 -9.33768553941695
27.4800000190735 -9.58044101210887
27.5199999809265 -9.82319441752095
27.5600001811981 -10.0658391233192
27.6000001430511 -10.3087192475476
27.6400001049042 -10.5518329334993
27.6800000667572 -10.7946822197454
27.7200000286102 -11.0375560206052
27.7599999904633 -11.2802666373469
27.8000001907349 -11.5234004618883
27.8400001525879 -11.766724038657
27.8800001144409 -12.0099769329805
27.9200000762939 -12.2545573106493
27.960000038147 -12.4981980544888
28 -12.7411811807465
28.0400002002716 -12.9835465768964
28.0800001621246 -13.2257955912809
28.1200001239777 -13.4680671107148
28.1600000858307 -13.7102413917981
28.2000000476837 -13.9530181470262
28.2400000095367 -14.1959566225908
28.2800002098083 -14.4386199968462
28.3200001716614 -14.6817133685392
28.3600001335144 -14.9246095727717
28.4000000953674 -15.1678505370992
28.4400000572205 -15.411390564798
28.4800000190735 -15.655039940699
28.5199999809265 -15.8987277542211
28.5600001811981 -16.1422228155214
28.6000001430511 -16.3855647791658
28.6400001049042 -16.6289367145533
28.6800000667572 -16.8725254440558
28.7200000286102 -17.1159595366849
28.7599999904633 -17.3587434985618
28.8000001907349 -17.6019182080773
28.8400001525879 -17.8453989289813
28.8800001144409 -18.0878594146654
28.9200000762939 -18.3311422700513
28.960000038147 -18.5742501875053
29 -18.8179991980167
29.0400002002716 -19.0624980813195
29.0800001621246 -19.3071046954011
29.1200001239777 -19.5513082113908
29.1600000858307 -19.7957978928756
29.2000000476837 -20.0402082550657
29.2400000095367 -20.28412587847
29.2800002098083 -20.5278092734818
29.3200001716614 -20.772047904557
29.3600001335144 -21.0160944159372
29.4000000953674 -21.2601001159584
29.4400000572205 -21.5040399162646
29.4800000190735 -21.7474927538666
29.5199999809265 -21.9908435045334
29.5600001811981 -22.2341643709792
29.6000001430511 -22.4775208167634
29.6400001049042 -22.721156732233
29.6800000667572 -22.9642241590358
29.7200000286102 -23.2070037453867
29.7599999904633 -23.4495959251097
29.8000001907349 -23.6928035507227
29.8400001525879 -23.936044888634
29.8800001144409 -24.1784171284787
29.9200000762939 -24.4217035321896
29.960000038147 -24.6636739176137
30 -24.9060261525888
30.0400002002716 -25.1479776370229
30.0800001621246 -25.3900369021994
30.1200001239777 -25.632144695199
30.1600000858307 -25.874783213838
30.2000000476837 -26.1171899673699
30.2400000095367 -26.3597907316162
30.2800002098083 -26.6010871799182
30.3200001716614 -26.8420878580333
30.3600001335144 -27.0822387088534
30.4000000953674 -27.3237817475092
30.4400000572205 -27.5643967461771
30.4800000190735 -27.8065731553455
30.5199999809265 -28.0477107990619
30.5600001811981 -28.2877459846953
30.6000001430511 -28.5284235745679
30.6400001049042 -28.7688278653692
30.6800000667572 -29.0089552642984
30.7200000286102 -29.2486855945563
30.7599999904633 -29.4890859198745
30.8000001907349 -29.7304171214448
30.8400001525879 -29.9688239202864
30.8800001144409 -30.2083400924655
30.9200000762939 -30.4487736035979
30.960000038147 -30.6886774948751
31 -30.9289319432337
31.0400002002716 -31.1709677004766
31.0800001621246 -31.4125487318841
31.1200001239777 -31.6526561904779
31.1600000858307 -31.8917877805096
31.2000000476837 -32.1309711103648
31.2400000095367 -32.3685229884222
31.2800002098083 -32.6075751827031
31.3200001716614 -32.8497682832067
31.3600001335144 -33.0923957512802
31.4000000953674 -33.3351609961337
31.4400000572205 -33.5762443502626
31.4800000190735 -33.8197257463605
31.5199999809265 -34.060210558826
31.5600001811981 -34.3012985761355
31.6000001430511 -34.5520029528566
31.6400001049042 -34.8054622854104
31.6800000667572 -35.0475726785361
31.7200000286102 -35.2909225298842
31.7599999904633 -35.5341771029155
31.8000001907349 -35.7746312367219
31.8400001525879 -36.0118226028305
31.8800001144409 -36.2511924998276
31.9200000762939 -36.4901957154969
31.960000038147 -36.7303360701302
32 -36.9707114963677
32.0400002002716 -37.2154715410083
32.0800001621246 -37.4581390674334
32.1200001239777 -37.6825393191545
32.1600000858307 -37.9267975741903
32.2000000476837 -38.1721463369024
32.2400000095367 -38.4159307883754
32.2800002098083 -38.6550347168116
32.3200001716614 -38.8985164060852
32.3600001335144 -39.1421729117431
32.4000000953674 -39.3868594061593
32.4400000572205 -39.6343776473904
32.4800000190735 -39.8775511220815
32.5199999809265 -40.1241195329776
32.5600001811981 -40.368728264099
32.6000001430511 -40.613573175108
32.6400001049042 -40.8568248273204
32.6800000667572 -41.1052487916441
32.7200000286102 -41.3514590530005
32.7599999904633 -41.6032370774662
32.8000001907349 -41.8558292642011
32.8400001525879 -42.0976116178112
32.8800001144409 -42.3120624423304
32.9200000762939 -42.5575209489888
32.960000038147 -42.8027438312865
33 -43.0500312048559
33.0400002002716 -43.2944165354469
33.0800001621246 -43.5407857329085
33.1200001239777 -43.7852006311762
33.1600000858307 -44.029530757291
33.2000000476837 -44.2728883578855
33.2400000095367 -44.5177984460008
33.2800002098083 -44.7637968207379
33.3200001716614 -45.0090647814993
33.3600001335144 -45.2886419738545
33.4000000953674 -45.549767391269
33.4400000572205 -45.7976405483218
33.4800000190735 -46.0461022285263
33.5199999809265 -46.2919240384677
33.5600001811981 -46.5386572766462
33.6000001430511 -46.7844099033875
33.6400001049042 -47.0334302838414
33.6800000667572 -47.2811911067108
33.7200000286102 -47.5299511083685
33.7599999904633 -47.7767512942
33.8000001907349 -48.0230985834538
33.8400001525879 -48.2697248482917
33.8800001144409 -48.5164541403062
33.9200000762939 -48.7623624048637
33.960000038147 -49.0084275858052
34 -49.256233183204
34.0400002002716 -49.5035902428145
34.0800001621246 -49.7515083257997
34.1200001239777 -49.9988180432542
34.1600000858307 -50.2449911153204
};
\node at (axis cs:6,55)[
  anchor=north,
  text=black,
  rotate=0.0
]{ Pickup truck};
\node at (axis cs:6,0)[
  anchor=south,
  text=black,
  rotate=0.0
]{ Station wagon};
\end{axis}

\end{tikzpicture}}\\
%	\subfloat[Lateral distance of other vehicles with respect to ego vehicle.]{% This file was created by matplotlib2tikz v0.6.14.
\begin{tikzpicture}

\definecolor{color1}{rgb}{0.83921568627451,0.152941176470588,0.156862745098039}
\definecolor{color0}{rgb}{0.172549019607843,0.627450980392157,0.172549019607843}
\definecolor{color3}{rgb}{0.549019607843137,0.337254901960784,0.294117647058824}
\definecolor{color2}{rgb}{0.580392156862745,0.403921568627451,0.741176470588235}

\begin{axis}[
xlabel={Time [s]},
ylabel={Relative lateral distance [m]},
xmin=0, xmax=34.1600000858307,
ymin=-0.173290601769933, ymax=3.64210220513419,
width=\figurewidth,
height=\figureheight,
tick align=outside,
tick pos=left,
xmajorgrids,
x grid style={lightgray!92.026143790849673!black},
ymajorgrids,
y grid style={lightgray!92.026143790849673!black},
clip marker paths
]
\addplot [semithick, color0, mark=*, mark size=1, mark options={solid}, only marks, forget plot]
table {%
0 0.0381368308942715
0.0400002002716064 0.02113685530519
0.0800001621246338 0.0391010935685372
0.120000123977661 0.0167204241158708
0.160000085830688 0.0221102971568354
0.200000047683716 0.0371810088295469
0.240000009536743 0.0350331130112317
0.28000020980835 0.00338142268235999
0.320000171661377 0.0357759749536047
0.360000133514404 0.0434837984305201
0.400000095367432 0.040388045409689
0.440000057220459 0.0324422897119403
0.480000019073486 0.00869010725140057
0.519999980926514 0.0400389850973934
0.56000018119812 0.0513388147283746
0.600000143051147 0.056642619208894
0.640000104904175 0.0707127113560991
0.680000066757202 0.0729581499457811
0.720000028610229 0.083510252345897
0.759999990463257 0.0937699759551173
0.800000190734863 0.102649999558732
0.840000152587891 0.106834772183385
0.880000114440918 0.12309483773967
0.920000076293945 0.131749202753455
0.960000038146973 0.125239754073978
1 0.117490166662961
1.04000020027161 0.107150301723575
1.08000016212463 0.0853712896616162
1.12000012397766 0.0989462054921034
1.16000008583069 0.0809999891363644
1.20000004768372 0.0963594796869076
1.24000000953674 0.0721337799846891
1.28000020980835 0.0378834284309467
1.32000017166138 0.0516034937470624
1.3600001335144 0.0413781985486888
1.40000009536743 0.0452413135034469
1.44000005722046 0.039054345362423
1.48000001907349 0.0171489648683924
1.51999998092651 0.0118968251668364
1.56000018119812 0.00724660155053917
1.60000014305115 0.00303050984339228
1.64000010490417 0.00470757975882657
1.6800000667572 0.0188723602537929
1.72000002861023 0.0103054748517441
1.75999999046326 0.0115996037285363
1.80000019073486 0.00363111710992475
1.84000015258789 0.000994896573605503
1.88000011444092 0.00760431297703624
1.92000007629395 0.0106438583849226
1.96000003814697 0.0097567034020915
2 0.0174589407677711
2.04000020027161 0.0231076402552772
2.08000016212463 0.030971922243608
2.12000012397766 0.0265008615913468
2.16000008583069 0.0513680670644315
2.20000004768372 0.0573883985274429
2.24000000953674 0.0741871900441446
2.28000020980835 0.0811665913930259
2.32000017166138 0.0649581867138497
2.3600001335144 0.0678215210672767
2.40000009536743 0.0270355539670269
2.44000005722046 0.0212123710780815
2.48000001907349 0.0557109903259803
2.51999998092651 0.0844498331672178
2.56000018119812 0.129960611386368
2.60000014305115 0.133542259351323
2.64000010490417 0.142971831220934
2.6800000667572 0.161033209252872
2.72000002861023 0.171232803801011
2.75999999046326 0.176720492995873
2.80000019073486 0.184989538556193
2.84000015258789 0.188972637256443
2.88000011444092 0.285250910688982
2.92000007629395 0.329705483304926
2.96000003814697 0.300258874743721
3 0.265186751058268
3.04000020027161 0.253399363733641
3.08000016212463 0.253555777957019
3.12000012397766 0.247126701503728
3.16000008583069 0.246873773980347
3.20000004768372 0.236923675247426
3.24000000953674 0.26285822920758
3.28000020980835 0.262643412507302
3.32000017166138 0.352906903258998
3.3600001335144 0.850008637476276
3.40000009536743 0.867149702423307
3.44000005722046 0.901693788670374
3.48000001907349 0.923200910256654
3.51999998092651 0.937779528099259
3.56000018119812 0.951134412868823
3.60000014305115 0.960803254197163
3.64000010490417 0.965178555687328
3.6800000667572 0.992860299039072
3.72000002861023 1.00948530231622
3.75999999046326 1.04657679302002
3.80000019073486 1.06486542737576
3.84000015258789 0.619778490482562
3.88000011444092 0.407590262788984
3.92000007629395 0.392139217885747
3.96000003814697 0.420834489436337
4 0.417584687981949
4.04000020027161 0.42304886546305
4.08000016212463 0.452549643309931
4.12000012397766 0.420823513875285
4.16000008583069 0.432315884342853
4.20000004768372 0.410410513678159
4.24000000953674 0.417494713655485
4.28000020980835 0.433781744355344
4.32000017166138 0.415479606259259
4.3600001335144 0.283359475924521
4.40000009536743 0.270350412183972
4.44000005722046 0.287235490455843
4.48000001907349 0.298643102561026
4.51999998092651 0.289660673169994
4.56000018119812 0.292899067750897
4.60000014305115 0.290184070297981
4.64000010490417 0.302704664411031
4.6800000667572 0.322220504534402
4.72000002861023 0.299287728603343
4.75999999046326 0.302480904256152
4.80000019073486 0.287748772964173
4.84000015258789 0.263406310959874
4.88000011444092 0.264441305017787
4.92000007629395 0.236154442628217
4.96000003814697 0.235250404030711
5 0.220006901913876
5.04000020027161 0.232175489009746
5.08000016212463 0.259265424147744
5.12000012397766 0.226095143364803
5.16000008583069 0.239319109984045
5.20000004768372 0.186899611270509
5.24000000953674 0.156289115410779
5.28000020980835 0.175985005931449
5.32000017166138 0.122900966400057
5.3600001335144 0.135298972096452
5.40000009536743 0.115195194136861
5.44000005722046 0.112372446731432
5.48000001907349 0.145115316845279
5.51999998092651 0.116709271500684
5.56000018119812 0.159954393439151
5.60000014305115 0.118922426253256
5.64000010490417 0.165801327115087
5.6800000667572 0.209757829449264
5.72000002861023 0.175374092604902
5.75999999046326 0.214746814239462
5.80000019073486 0.184947044044712
5.84000015258789 0.198066446025995
5.88000011444092 0.226900728295784
5.92000007629395 0.330112421633569
5.96000003814697 0.347542522410502
6 0.344363570168427
6.04000020027161 0.325582296375428
6.08000016212463 0.382446474662502
6.12000012397766 0.358450480935073
6.16000008583069 0.366514547541245
6.20000004768372 0.355413272193381
6.24000000953674 0.401930438405605
6.28000020980835 0.418293076250252
6.32000017166138 0.423867373103444
6.3600001335144 0.486368003998145
6.40000009536743 0.499850093321639
6.44000005722046 0.489963723010275
6.48000001907349 0.558690034909302
6.51999998092651 0.535025944559044
6.56000018119812 0.556901671330504
6.60000014305115 0.556949468478594
6.64000010490417 0.562768906057496
6.6800000667572 0.582905352124388
6.72000002861023 0.541537924806151
6.75999999046326 0.522629561487791
6.80000019073486 0.519192186092225
6.84000015258789 0.48775860834054
6.88000011444092 0.49352707304628
6.92000007629395 0.462680621277502
6.96000003814697 0.468604198649577
7 0.464229275171788
7.04000020027161 0.436675179829648
7.08000016212463 0.450833879197452
7.12000012397766 0.444163141443064
7.16000008583069 0.435168879307809
7.20000004768372 0.429090584033307
7.24000000953674 0.411226149415275
7.28000020980835 0.419837915534217
7.32000017166138 0.368849174910318
7.3600001335144 0.387556341682623
7.40000009536743 0.400342711502304
7.44000005722046 0.38247722426308
7.48000001907349 0.399123443918386
7.51999998092651 0.403911125312095
7.56000018119812 0.418106841704959
7.60000014305115 0.408543696906752
7.64000010490417 0.405272836512228
7.6800000667572 0.424002722740362
7.72000002861023 0.413370620924076
7.75999999046326 0.431904400920678
7.80000019073486 0.423889188138996
7.84000015258789 0.430991015856339
7.88000011444092 0.471400908670376
7.92000007629395 0.421113489316624
7.96000003814697 0.368561200989445
8 0.356465920253978
8.04000020027161 0.345349814277919
8.08000016212463 0.395563984562723
8.12000012397766 0.339181590764903
8.16000008583069 0.324202424770464
8.20000004768372 0.312487617607969
8.24000000953674 0.293719189711732
8.28000020980835 0.335845408641955
8.32000017166138 0.262432057701375
8.3600001335144 0.272301449535368
8.40000009536743 0.260572649083723
8.44000005722046 0.256227682192313
8.48000001907349 0.31990035031589
8.51999998092651 0.244340220037733
8.56000018119812 0.265252367085148
8.60000014305115 0.256818583355582
8.64000010490417 0.278051708284813
8.6800000667572 0.351167994737714
8.72000002861023 0.264834889182869
8.75999999046326 0.271198795479087
8.80000019073486 0.336807394614122
8.84000015258789 0.253936360996017
8.88000011444092 0.324953741252065
8.92000007629395 0.144236236539778
8.96000003814697 0.208770271117139
9 0.201904937851891
9.04000020027161 0.157041847230222
9.08000016212463 0.120741711393716
9.12000012397766 0.263655902197788
9.16000008583069 0.236632861888699
9.20000004768372 0.272968044703493
9.24000000953674 0.295020938287213
9.28000020980835 0.212118550602269
9.32000017166138 0.25351635424825
9.3600001335144 0.309977897156007
9.40000009536743 0.265581991356659
9.44000005722046 0.097669690405285
9.48000001907349 0.109834857886905
9.51999998092651 0.0641834187643611
9.56000018119812 0.177445673772016
9.60000014305115 0.263010201811044
9.64000010490417 0.353703827719019
9.6800000667572 0.400278656721455
9.72000002861023 0.609579959043417
9.75999999046326 0.668295128762963
9.80000019073486 0.771999741989727
9.84000015258789 0.93010536095884
9.88000011444092 0.916723984013404
9.92000007629395 1.07077058083962
9.96000003814697 1.09434113375602
10 1.15265930952471
10.0400002002716 1.16750564983283
10.0800001621246 1.10652150123824
10.1200001239777 0.695523336384983
10.1600000858307 0.603335165712303
10.2000000476837 0.64119564488129
10.2400000095367 0.701054087993718
10.2800002098083 0.671819028616685
10.3200001716614 0.693899377631263
10.3600001335144 0.713256436535206
10.4000000953674 0.729935882394785
10.4400000572205 0.877707720198635
10.4800000190735 0.879440257556437
10.5199999809265 1.02901165390859
10.5600001811981 1.04006227817663
10.6000001430511 1.32427905708316
10.6400001049042 1.79363281068199
10.6800000667572 1.85455471154084
10.7200000286102 1.97496357276248
10.7599999904633 2.03143995433146
10.8000001907349 2.09590442495046
10.8400001525879 2.17585708631795
10.8800001144409 2.24222398749117
10.9200000762939 2.32775753684118
10.960000038147 2.34446117161317
11 2.36780419317215
11.0400002002716 2.46685050342601
11.0800001621246 2.4922220328291
11.1200001239777 2.54237299454481
11.1600000858307 2.53393769798958
11.2000000476837 2.55710651218478
11.2400000095367 2.55151504142248
11.2800002098083 2.6085780914353
11.3200001716614 2.66923119881023
11.3600001335144 2.632146309978
11.4000000953674 2.70811512935098
11.4400000572205 2.74971549331209
11.4800000190735 2.75376540230869
11.5199999809265 2.77266108834581
11.5600001811981 2.81884306831138
11.6000001430511 2.8551698098277
11.6400001049042 2.88241593354433
11.6800000667572 2.91483252677163
11.7200000286102 2.8891443957292
11.7599999904633 2.85834195498671
11.8000001907349 2.85666361761894
11.8400001525879 2.80801159919429
11.8800001144409 2.84903544152595
11.9200000762939 2.86874839089442
11.960000038147 2.89320091992745
12 2.89530369931046
12.0400002002716 2.87638673567583
12.0800001621246 2.88749643170761
12.1200001239777 2.98589340183008
12.1600000858307 3.00793843312561
12.2000000476837 2.9862076800158
12.2400000095367 2.99665902367942
12.2800002098083 3.02270202232385
12.3200001716614 3.07628246729457
12.3600001335144 3.0796880874079
12.4000000953674 3.08454955579072
12.4400000572205 3.09974419938209
12.4800000190735 3.11716774312233
12.5199999809265 3.12144920187392
12.5600001811981 3.14116041167727
12.6000001430511 3.13783521402095
12.6400001049042 3.14754789007854
12.6800000667572 3.16604844313826
12.7200000286102 3.0979505408759
12.7599999904633 3.08826160311041
12.8000001907349 3.0946761144545
12.8400001525879 3.06169144463503
12.8800001144409 3.0639310114521
12.9200000762939 2.99042769903511
12.960000038147 2.98474987973219
13 2.91418390375484
13.0400002002716 2.90144157554646
13.0800001621246 2.85703090438434
13.1200001239777 2.83612255898468
13.1600000858307 2.83047800689174
13.2000000476837 2.70924432423486
13.2400000095367 2.68546002770376
13.2800002098083 2.6624271302076
13.3200001716614 2.61513069521449
13.3600001335144 2.60182450451831
13.4000000953674 2.59519637377109
13.4400000572205 2.55886363896048
13.4800000190735 2.55770646594
13.5199999809265 2.55141288431576
13.5600001811981 2.58049323538798
13.6000001430511 2.61715559025556
13.6400001049042 2.61022747380488
13.6800000667572 2.59698408859404
13.7200000286102 2.66278028112081
13.7599999904633 2.6597901891299
13.8000001907349 2.71880618824384
13.8400001525879 2.76922713808551
13.8800001144409 2.77306122983895
13.9200000762939 2.77662056386169
13.960000038147 2.79334902166357
14 2.7955393008786
14.0400002002716 2.83318693139692
14.0800001621246 2.91683944769744
14.1200001239777 2.92564592859913
14.1600000858307 2.93925408147821
14.2000000476837 2.95475522476333
14.2400000095367 2.96027599862503
14.2800002098083 2.97033219447196
14.3200001716614 3.03611151096895
14.3600001335144 3.04884747986789
14.4000000953674 3.06526980108736
14.4400000572205 3.0410476931707
14.4800000190735 3.03673605446401
14.5199999809265 3.05019566164339
14.5600001811981 3.03158736458916
14.6000001430511 3.03245834465956
14.6400001049042 3.01306616367621
14.6800000667572 3.01168536918049
14.7200000286102 2.97350008188327
14.7599999904633 2.9695991363219
14.8000001907349 2.97053018352922
14.8400001525879 3.01139676220304
14.8800001144409 3.01212631643736
14.9200000762939 3.01786409872198
14.960000038147 3.02150319760802
15 3.02663156697457
15.2000000476837 3.36885601344697
15.2400000095367 3.36923767309898
15.2800002098083 3.36122612360904
15.3200001716614 3.46867525936582
15.3600001335144 3.30142528443136
15.4000000953674 3.40008726312075
15.4400000572205 3.26790644798411
15.4800000190735 3.25228484324031
15.5199999809265 3.34317805813538
15.5600001811981 3.34141173696059
15.6000001430511 3.42871388092302
15.6400001049042 3.43495122066327
15.6800000667572 3.43995134792983
15.7200000286102 3.34943497274909
15.7599999904633 3.28342022558534
15.8000001907349 3.36013051830697
15.8400001525879 3.36380278132245
15.8800001144409 3.24568818654841
15.9200000762939 3.21964124465118
15.960000038147 3.1949872701469
16 3.29409639578867
16.0400002002716 3.28689164924304
16.0800001621246 3.25766218295856
16.1200001239777 3.21815093381332
16.1600000858307 3.23775530606568
16.2000000476837 3.18979064959155
16.2400000095367 3.18056180237619
16.2800002098083 3.14080209848205
16.3200001716614 3.0354304586768
16.3600001335144 3.00298317987784
16.4000000953674 3.07758922243363
16.4400000572205 3.07111244018489
16.4800000190735 3.15934360705326
16.5199999809265 3.11213168171007
16.5600001811981 3.14021387193637
16.6000001430511 3.29317447985628
16.6400001049042 3.29614497847614
16.6800000667572 3.15900799676591
16.7200000286102 3.09934733177267
16.7599999904633 3.08079231812819
16.8000001907349 3.05552026359351
16.8400001525879 3.03883688410232
16.8800001144409 3.0804266906036
16.9200000762939 3.01118057807018
16.960000038147 3.14582314926644
17 3.10972974676133
17.0400002002716 3.15005946028387
17.0800001621246 3.12402976532463
17.1200001239777 3.1708184526824
17.1600000858307 3.10368177539309
17.2000000476837 3.1053573031198
17.2400000095367 3.09375465285855
17.2800002098083 3.13088130935675
17.3200001716614 3.1268690969873
17.3600001335144 3.15230607057874
17.4000000953674 3.20315913203944
17.4400000572205 3.09071581970171
17.4800000190735 3.07772384667936
17.5199999809265 3.1214161459182
17.5600001811981 3.0274406113622
17.6000001430511 3.05173165284688
17.6400001049042 3.08002107245363
17.6800000667572 3.11634013163619
17.7200000286102 3.03594275663301
};
\addplot [semithick, color2, mark=*, mark size=1, mark options={solid}, only marks, forget plot]
table {%
0.800000190734863 0.621780894541407
0.840000152587891 0.667228643531078
0.880000114440918 0.703931256601435
0.920000076293945 0.734100324128962
0.960000038146973 0.736254281278027
1 0.711349356136605
1.04000020027161 0.677915373690927
1.08000016212463 0.658645718119136
1.12000012397766 0.645746178976589
1.16000008583069 0.637156971378115
1.20000004768372 0.60970180909758
1.24000000953674 0.585981901321645
1.28000020980835 0.553362883479742
1.32000017166138 0.538066360577512
1.3600001335144 0.549180523163616
1.40000009536743 0.565690841267087
1.44000005722046 0.592133119777859
1.48000001907349 0.611832855182124
1.51999998092651 0.634410413681086
1.56000018119812 0.666099692220559
1.60000014305115 0.697466832557911
1.64000010490417 0.728840019987937
1.6800000667572 0.82253019583235
1.72000002861023 0.905312451703777
1.75999999046326 0.967942166761131
1.80000019073486 1.01666099043302
1.84000015258789 1.06484734610574
1.88000011444092 1.10114205273518
1.92000007629395 1.03110704850989
1.96000003814697 0.936859025909506
2 0.85558562916167
2.04000020027161 0.85931011746561
2.08000016212463 0.674752552758707
2.12000012397766 0.585023937402385
2.16000008583069 0.554808606126438
2.20000004768372 0.352883599014832
2.24000000953674 0.322628624191163
2.28000020980835 0.202770637212911
2.32000017166138 0.121911066719998
2.3600001335144 0.0997670688596479
2.40000009536743 0.0452575142227856
2.44000005722046 0.0896243586057727
2.48000001907349 0.0411385392143181
2.51999998092651 0.012268286282695
2.56000018119812 0.0348481555937973
2.60000014305115 0.0571527979769944
2.64000010490417 0.00297619831334797
2.6800000667572 0.0879693512905296
2.72000002861023 0.115198760156291
2.75999999046326 0.0742148151556042
2.80000019073486 0.0408990149515641
2.84000015258789 0.0147685561399521
2.88000011444092 0.0299590241877634
2.92000007629395 0.0653050805864573
2.96000003814697 0.0628954681623737
3 0.0577424071097065
3.04000020027161 0.0155928789030811
3.08000016212463 0.0372663020696852
3.12000012397766 0.0247204226322216
3.16000008583069 0.0508525763709966
3.20000004768372 0.0690241114686972
3.24000000953674 0.0843954069625317
3.28000020980835 0.107217435099316
3.32000017166138 0.0785262910363228
3.3600001335144 1.70200866725813
3.40000009536743 1.79498869787246
3.44000005722046 1.88140559259185
3.48000001907349 1.93818411921195
3.51999998092651 2.00261007195238
3.56000018119812 2.06496033976859
3.60000014305115 2.13344111395693
3.64000010490417 2.20753564389867
3.6800000667572 2.26483198259269
3.72000002861023 2.3352306005904
3.75999999046326 2.4086457427422
3.80000019073486 2.46855126613592
3.84000015258789 1.37595675691967
3.88000011444092 0.838953331775156
3.92000007629395 0.845434671076559
3.96000003814697 0.833348629987186
4 0.821542679996477
4.04000020027161 0.820438089472733
4.08000016212463 0.790130435352635
4.12000012397766 0.767786592643065
4.16000008583069 0.73254779248225
4.20000004768372 0.705584198451115
4.24000000953674 0.681695163797445
4.28000020980835 0.695308189336034
4.32000017166138 0.668692867212533
4.3600001335144 0.660143261311504
4.40000009536743 0.617554756439765
4.44000005722046 0.57961680913751
4.48000001907349 0.57055120403499
4.51999998092651 0.530677636815153
4.56000018119812 0.487668931761905
4.60000014305115 0.489932113697967
4.64000010490417 0.467691960674357
4.6800000667572 0.457152391236219
4.72000002861023 0.448558991258536
4.75999999046326 0.431169694733837
4.80000019073486 0.428170314717866
4.84000015258789 0.413458218598482
4.88000011444092 0.372380531461593
4.92000007629395 0.344653873513428
4.96000003814697 0.308955672454582
5 0.271574082903553
5.04000020027161 0.234374268666664
5.08000016212463 0.20329956549974
5.12000012397766 0.184200816524698
5.16000008583069 0.176812393256492
5.20000004768372 0.145526456784332
5.24000000953674 0.123534794276004
5.28000020980835 0.10780625976147
5.32000017166138 0.0942934485020026
5.3600001335144 0.0910923761141962
5.40000009536743 0.0612185765086253
5.44000005722046 0.0535760384698931
5.48000001907349 0.0672446810344412
5.51999998092651 0.0776819919664344
5.56000018119812 0.0738334441383116
5.60000014305115 0.0677011714305781
5.64000010490417 0.0624247532694325
5.6800000667572 0.0996246517290538
5.72000002861023 0.127814548154203
5.75999999046326 0.117697008853272
5.80000019073486 0.159646478268813
5.84000015258789 0.16761583149645
5.88000011444092 0.17063192514451
5.92000007629395 0.141460613967864
5.96000003814697 0.177144248749035
6 0.185117040527483
6.04000020027161 0.175447508869797
6.08000016212463 0.203932174252298
6.12000012397766 0.231666374692676
6.16000008583069 0.249752933114062
6.20000004768372 0.261257640144873
6.24000000953674 0.263129471747779
6.28000020980835 0.261430288082534
6.32000017166138 0.265747706920059
6.3600001335144 0.263977249021343
6.40000009536743 0.256807848451168
6.44000005722046 0.250441089355871
6.48000001907349 0.220851977105753
6.51999998092651 0.194199628878898
6.56000018119812 0.153433850658369
6.60000014305115 0.121122851327442
6.64000010490417 0.121745215920324
6.6800000667572 0.0826897064031779
6.72000002861023 0.0622861530284031
6.75999999046326 0.060396507998844
6.80000019073486 0.0523067745113687
6.84000015258789 0.0470072716733371
6.88000011444092 0.0430134646391351
6.92000007629395 0.0565703396721105
6.96000003814697 0.0161798518078868
7 0.0220095225221495
7.04000020027161 0.00787134057374015
7.08000016212463 0.058120042189996
7.12000012397766 0.0608935919898734
7.16000008583069 0.085058386871908
7.20000004768372 0.119083213486177
7.24000000953674 0.1355147701758
7.28000020980835 0.171641382362876
7.32000017166138 0.202313893080596
7.3600001335144 0.232243105467075
7.40000009536743 0.264094617297433
7.44000005722046 0.264654921285325
7.48000001907349 0.311506562881569
7.51999998092651 0.343761126937422
7.56000018119812 0.349903824185768
7.60000014305115 0.368473577646571
7.64000010490417 0.410713578084173
7.6800000667572 0.447589349057611
7.72000002861023 0.477496024197941
7.75999999046326 0.489103712693528
7.80000019073486 0.495620358576148
7.84000015258789 0.52386940479388
7.88000011444092 0.532961234543162
7.92000007629395 0.496104818197295
7.96000003814697 0.532633547562445
8 0.564926396335646
8.04000020027161 0.531783801526278
8.08000016212463 0.539435144649708
8.12000012397766 0.573417726345222
8.16000008583069 0.51771291188833
8.20000004768372 0.536751662155298
8.24000000953674 0.480114940794846
8.28000020980835 0.497583059058367
8.32000017166138 0.51764956692923
8.3600001335144 0.479921255157875
8.40000009536743 0.501235528109217
8.44000005722046 0.458899498699662
8.48000001907349 0.468813320492855
8.51999998092651 0.485225978800553
8.56000018119812 0.472820934005038
8.60000014305115 0.495183224636732
8.64000010490417 0.492175079200897
8.6800000667572 0.512290205144629
8.72000002861023 0.509631269318428
8.75999999046326 0.520631911221449
8.80000019073486 0.537625034411401
8.84000015258789 0.525146420429587
8.88000011444092 0.557548445349939
8.92000007629395 0.731996313599361
8.96000003814697 1.49279526405524
9 1.44228255082623
9.04000020027161 1.37590708439686
9.08000016212463 1.29695096576545
9.12000012397766 1.23974675789227
9.16000008583069 1.16185887436896
9.20000004768372 1.01512478472019
9.24000000953674 0.940284536374514
9.28000020980835 0.824018306757474
9.32000017166138 0.769678236925772
9.3600001335144 0.69284107271026
9.40000009536743 0.622179570433674
9.44000005722046 0.0266525098300333
9.48000001907349 0.217194673077815
9.51999998092651 0.130655145433253
9.56000018119812 0.0394684191601839
9.60000014305115 0.0355517350391979
9.64000010490417 0.136960268341282
9.6800000667572 0.210872953623
9.72000002861023 0.310468385630127
9.75999999046326 0.396331529423356
9.80000019073486 0.462980360020937
9.84000015258789 0.565742825470723
9.88000011444092 0.63437355156416
9.92000007629395 0.725151057463978
9.96000003814697 0.789513557038284
10 0.818825728677042
10.0400002002716 0.828741219657281
10.0800001621246 0.844435077565438
10.1200001239777 0.206447746273092
10.1600000858307 0.406270046557841
10.2000000476837 0.464667973518317
10.2400000095367 0.523664088557207
10.2800002098083 0.581434519356024
10.3200001716614 0.598128106205852
10.3600001335144 0.648586228954781
10.4000000953674 0.698943159455128
10.4400000572205 0.723094229098046
10.4800000190735 0.756390195792693
10.5199999809265 0.802780695380516
10.5600001811981 0.842425019693637
10.6000001430511 0.42083127306057
10.6400001049042 0.144429304393653
10.6800000667572 0.173382989338568
10.7200000286102 0.139930955219012
10.7599999904633 0.161614006685713
10.8000001907349 0.151563781416845
10.8400001525879 0.162657984915037
10.8800001144409 0.175433806466977
10.9200000762939 0.142536299308423
10.960000038147 0.143126346778155
11 0.0961589178916442
11.0400002002716 0.0954002530236981
11.0800001621246 0.0965203125741231
11.1200001239777 0.0439736979513271
11.1600000858307 0.0461292459357782
11.2000000476837 0.00841181896919599
11.2400000095367 0.0141444034055472
11.2800002098083 0.0118462076119628
11.3200001716614 0.0510562482861891
11.3600001335144 0.0428973202452942
11.4000000953674 0.0738726283368093
11.4400000572205 0.0756833286545554
11.4800000190735 0.0874906484996138
11.5199999809265 0.0987301794726312
11.5600001811981 0.103387036754249
11.6000001430511 0.107849821776834
11.6400001049042 0.132174631275044
11.6800000667572 0.135087656211702
11.7200000286102 0.140686128106692
11.7599999904633 0.139291506245917
11.8000001907349 0.132753177205873
11.8400001525879 0.131823031549001
11.8800001144409 0.137518490619854
11.9200000762939 0.157026130593935
11.960000038147 0.164723307729189
12 0.176308209094178
12.0400002002716 0.193141934442109
12.0800001621246 0.187594694840706
12.1200001239777 0.200775397619241
12.1600000858307 0.179515517778414
12.2000000476837 0.173200499897326
12.2400000095367 0.163924277429197
12.2800002098083 0.126312419333323
12.3200001716614 0.0736073679362119
12.3600001335144 0.074403460649033
12.4000000953674 0.0790638705828976
12.4400000572205 0.0481929176261871
12.4800000190735 0.0401924769652898
12.5199999809265 0.0504283178406302
12.5600001811981 0.046348212140439
12.6000001430511 0.0673523367476783
12.6400001049042 0.0680237666674475
12.6800000667572 0.0874025258856866
12.7200000286102 0.114004576946424
12.7599999904633 0.0982366152312537
12.8000001907349 0.131135808485439
12.8400001525879 0.188352775676997
12.8800001144409 0.207912447369169
12.9200000762939 0.249880941043445
12.960000038147 0.249812134426902
13 0.291660513679775
13.0400002002716 0.301424545998567
13.0800001621246 0.349486685559043
13.1200001239777 0.401227232211693
13.1600000858307 0.396406818362359
13.2000000476837 0.41077916548122
13.2400000095367 0.407681164748659
13.2800002098083 0.438980253128986
13.3200001716614 0.489302494212933
13.3600001335144 0.464973917744299
13.4000000953674 0.49045157122265
13.4400000572205 0.497963602921794
13.4800000190735 0.509200289361905
13.5199999809265 0.517863824033615
13.5600001811981 0.480890191268729
13.6000001430511 0.497975823721007
13.6400001049042 0.453487761938065
13.6800000667572 0.441869718285306
13.7200000286102 0.402727969539386
13.7599999904633 0.394687698067158
13.8000001907349 0.397299126675974
13.8400001525879 0.365948071942483
13.8800001144409 0.353739627734933
13.9200000762939 0.347941746485463
13.960000038147 0.343529746702566
14 0.328603604693036
14.0400002002716 0.307654347305141
14.0800001621246 0.306709280336313
14.1200001239777 0.298248826545759
14.1600000858307 0.287708778065702
14.2000000476837 0.277488400005343
14.2400000095367 0.271485272022714
14.2800002098083 0.273887734509736
14.3200001716614 0.230308511107575
14.3600001335144 0.239479991778814
14.4000000953674 0.226453453187606
14.4400000572205 0.232170462616738
14.4800000190735 0.239405478273298
14.5199999809265 0.24306726017475
14.5600001811981 0.23485696003874
14.6000001430511 0.19732651518618
14.6400001049042 0.194861581401044
14.6800000667572 0.187798450064056
14.7200000286102 0.185994413604425
14.7599999904633 0.192761698985771
14.8000001907349 0.197384006692541
14.8400001525879 0.191652887111309
14.8800001144409 0.211036992096431
14.9200000762939 0.225907294838038
14.960000038147 0.282585869442023
15 0.261629462311808
15.0400002002716 0.311236259153442
15.0800001621246 0.380036354367733
15.1200001239777 0.354660729243435
15.1600000858307 0.399325703634067
15.2000000476837 0.387316306664285
15.2400000095367 0.406496058349613
15.2800002098083 0.446278490811678
15.3200001716614 0.440044220623218
15.3600001335144 0.456759483868232
15.4000000953674 0.438290035864812
15.4400000572205 0.482478482972585
15.4800000190735 0.53714561099533
15.5199999809265 0.501549908420876
15.5600001811981 0.532116767794257
15.6000001430511 0.491607148458034
15.6400001049042 0.516180602157746
15.6800000667572 0.525293774114985
15.7200000286102 0.50642885102998
15.7599999904633 0.521231358828431
15.8000001907349 0.569761145235892
15.8400001525879 0.579181188789721
15.8800001144409 0.607581804706419
15.9200000762939 0.552661991001875
15.960000038147 0.553267534029087
16 0.365615653145265
16.0400002002716 0.401332523388542
16.0800001621246 0.359754073851637
16.1200001239777 0.376951673314373
16.1600000858307 0.379601471857892
16.2000000476837 0.41462038903375
16.2400000095367 0.451592776274832
16.2800002098083 0.419996646820623
16.3200001716614 0.461380224473872
16.3600001335144 0.509497692392404
16.4000000953674 0.46590171386417
16.4400000572205 0.49910295459866
16.4800000190735 0.460182231925046
16.5199999809265 0.461365744528246
16.5600001811981 0.500481197214694
16.6000001430511 0.483752179051981
16.6400001049042 0.475620553850882
16.6800000667572 0.433775343251846
16.7200000286102 0.424146321920919
16.7599999904633 0.4180893332191
16.8000001907349 0.209092554244105
16.8400001525879 0.188686834817205
16.8800001144409 0.00013634399843655
16.9200000762939 0.0025850661097468
16.960000038147 0.0745291954672623
17 0.0799456778099812
17.0400002002716 0.128473379604302
17.0800001621246 0.10048621873211
17.1200001239777 0.0293047333017541
17.1600000858307 0.194594627942457
17.2000000476837 0.334986817469675
17.2400000095367 0.276600884060138
17.2800002098083 0.260272970407489
17.3200001716614 0.253344246054211
17.3600001335144 0.330393348630384
17.4000000953674 0.293878978347185
17.4400000572205 0.273034328802317
17.4800000190735 0.353704570314134
17.5199999809265 0.287662481760449
17.5600001811981 0.326474189930113
17.6000001430511 0.291838684981508
17.6400001049042 0.27455317312446
17.6800000667572 0.215512080071668
17.7200000286102 0.251733939789449
17.7599999904633 0.23430824101798
17.8000001907349 0.268560403282869
17.8400001525879 0.276048856383893
17.8800001144409 0.259853623079938
17.9200000762939 0.227958529581907
17.960000038147 0.260994450552771
18 0.233100629120662
18.0400002002716 0.264157634991637
18.0800001621246 0.282801467473936
18.1200001239777 0.32760498660685
18.1600000858307 0.312647665938944
18.2000000476837 0.282262408341053
18.2400000095367 0.266138267814961
18.2800002098083 0.331732793870103
18.3200001716614 0.302405795128371
18.3600001335144 0.310731547452844
18.4000000953674 0.298223590696064
18.4400000572205 0.305324944169221
18.4800000190735 0.328682458059662
18.5199999809265 0.272222003648666
18.5600001811981 0.309990926918121
18.6000001430511 0.311578919126468
18.6400001049042 0.319468521375502
18.6800000667572 0.336465744568389
18.7200000286102 0.391127665791252
18.7599999904633 0.46391263807286
18.8000001907349 0.483142592515203
18.8400001525879 0.518703999457798
18.8800001144409 0.519025505640833
18.9200000762939 0.569957434352852
18.960000038147 0.519293424105593
19 0.489291243473054
19.0400002002716 0.586702701905003
19.0800001621246 0.533510124158728
19.1200001239777 0.566029066082637
19.1600000858307 0.484918223556727
19.2000000476837 0.499537731084118
19.2400000095367 0.460253285992675
19.2800002098083 0.444523567300749
19.3200001716614 0.525376367517177
19.3600001335144 0.458011103960263
19.4000000953674 0.434124299913115
19.4400000572205 0.446846195501011
19.4800000190735 0.407541723549119
19.5199999809265 0.39089233729046
19.5600001811981 0.357971040279523
19.6000001430511 0.280716449877248
19.6400001049042 0.23702360905534
19.6800000667572 0.161256625834216
19.7200000286102 0.170063885273325
19.7599999904633 0.153364190987747
19.8000001907349 0.118982913836433
19.8400001525879 0.174507319566782
19.8800001144409 0.100878741467281
19.9200000762939 0.111860770599194
19.960000038147 0.115804136567188
20 0.132500291130613
20.0400002002716 0.103055511590517
20.0800001621246 0.080178016128675
20.1200001239777 0.058869214740979
20.1600000858307 0.0765786667350544
20.2000000476837 0.0625957181753782
20.2400000095367 0.0404800474838357
20.2800002098083 0.00378935650153357
20.3200001716614 0.0140178244448197
20.3600001335144 0.0399869008579174
20.4000000953674 0.0727868070503225
20.4400000572205 0.0868396650850215
20.4800000190735 0.107064261738325
20.5199999809265 0.120540603196731
20.5600001811981 0.151135418825004
20.6000001430511 0.214306935671204
20.6400001049042 0.244836214588161
20.6800000667572 0.272070688847286
20.7200000286102 0.272532454552194
20.7599999904633 0.293134520544023
20.8000001907349 0.313433272604645
20.8400001525879 0.304442212555397
20.8800001144409 0.321128237100606
20.9200000762939 0.299318223989583
20.960000038147 0.307952004278627
21 0.312508408144233
21.0400002002716 0.309598216799631
21.0800001621246 0.336645404516135
21.1200001239777 0.311515122814787
21.1600000858307 0.300694484038369
21.2000000476837 0.289387129238116
21.2400000095367 0.277390570836323
21.2800002098083 0.292375762421693
21.3200001716614 0.285234062334272
21.3600001335144 0.296208915943725
21.4000000953674 0.304444221937433
21.4400000572205 0.308343274207445
21.4800000190735 0.350206315021647
21.5199999809265 0.376217679876752
21.5600001811981 0.359353196469773
21.6000001430511 0.413355685342322
21.6400001049042 0.419488396392495
21.6800000667572 0.426059147745544
21.7200000286102 0.438628563052379
21.7599999904633 0.411465128936165
21.8000001907349 0.42228845450073
21.8400001525879 0.411288834177299
21.8800001144409 0.435335087546701
21.9200000762939 0.442362684370192
21.960000038147 0.425994559091791
22 0.465457446096395
22.0400002002716 0.485126271139445
22.0800001621246 0.441758446544142
22.1200001239777 0.496125275603297
22.1600000858307 0.482524407294949
22.2000000476837 0.488852730970544
22.2400000095367 0.479582864787968
22.2800002098083 0.465554229974978
22.3200001716614 0.490687917324542
22.3600001335144 0.444105355534152
22.4000000953674 0.477543795264516
22.4400000572205 0.474717153496746
22.4800000190735 0.434782582857806
22.5199999809265 0.436977927161987
22.5600001811981 0.370756478196088
22.6000001430511 0.361851306279801
22.6400001049042 0.362174933042488
22.6800000667572 0.326574131601732
22.7200000286102 0.343482954526857
22.7599999904633 0.404438122192113
22.8000001907349 0.35416956717401
22.8400001525879 0.34229893606695
22.8800001144409 0.311442493809119
22.9200000762939 0.350749008920192
22.960000038147 0.346339684029994
23 0.335358973903887
23.0400002002716 0.335512040245873
23.0800001621246 0.331709570346287
23.1200001239777 0.32120300144517
23.1600000858307 0.327052423315267
23.2000000476837 0.298750641238595
23.2400000095367 0.344846661698518
23.2800002098083 0.344917176212343
23.3200001716614 0.328768254052933
23.3600001335144 0.351894151668737
23.4000000953674 0.341693756985598
23.4400000572205 0.360720423511159
23.4800000190735 0.389433296336091
23.5199999809265 0.379997703693941
23.5600001811981 0.403646458689526
23.6000001430511 0.397651571740727
23.6400001049042 0.468892731261373
23.6800000667572 0.471455301490921
23.7200000286102 0.451396022198251
23.7599999904633 0.488251397497882
23.8000001907349 0.521874463798703
23.8400001525879 0.557897129220281
23.8800001144409 0.59052328796361
23.9200000762939 0.618009402545095
23.960000038147 0.652365229631006
24 0.676452392595554
24.0400002002716 0.724648920732077
24.0800001621246 0.733191597135662
24.1200001239777 0.759492228721452
24.1600000858307 0.803027182466171
24.2000000476837 0.830810999895736
24.2400000095367 0.879543891973312
24.2800002098083 0.890434175883387
24.3200001716614 0.940589346584622
24.3600001335144 0.868945318590419
24.4000000953674 0.865845000067434
24.4400000572205 0.890338062335698
24.4800000190735 0.899428686068025
24.5199999809265 0.924537698357935
24.5600001811981 0.953498080029797
24.6000001430511 0.977711788208381
24.6400001049042 1.02309208893961
24.6800000667572 1.03244203500956
24.7200000286102 1.05373211596241
24.7599999904633 1.06818765592556
24.8000001907349 1.07193168433519
24.8400001525879 1.00667318347247
24.8800001144409 1.0135093129229
24.9200000762939 1.06014095354918
24.960000038147 1.08227991062457
25 1.09762317889069
25.0400002002716 1.14645181048556
25.0800001621246 1.1549058098166
25.1200001239777 1.18094310046245
25.1600000858307 1.19932580601511
25.2000000476837 1.2029790506364
25.2400000095367 1.21764467799694
25.2800002098083 1.22414588087019
25.3200001716614 1.24459662209246
25.3600001335144 1.25161317224383
25.4000000953674 1.25368829522452
25.4400000572205 1.26605400122151
25.4800000190735 1.2687005302304
25.5199999809265 1.26294541046953
25.5600001811981 1.28475830406148
25.6000001430511 1.29641596629058
25.6400001049042 1.30193639216602
25.6800000667572 1.31006753000589
25.7200000286102 1.31005217643373
25.7599999904633 1.29404117374669
25.8000001907349 1.29716498478734
25.8400001525879 1.30391391355735
25.8800001144409 1.316804459337
25.9200000762939 1.32121571485366
25.960000038147 1.31181146062277
26 1.30211707139493
26.0400002002716 1.27579011427456
26.0800001621246 1.25784560188674
26.1200001239777 1.24621901502802
26.1600000858307 1.22817882881089
26.2000000476837 1.18592074456898
26.2400000095367 1.10068460469787
26.2800002098083 1.08893346205365
26.3200001716614 1.03945230936395
26.3600001335144 1.07499854823709
26.4000000953674 1.06881152288062
26.4400000572205 1.0540541448882
26.4800000190735 1.04252060879542
26.5199999809265 1.0330400569292
26.5600001811981 1.01751568475828
26.6000001430511 1.00750103903527
26.6400001049042 1.01611940318834
26.6800000667572 1.00489809799477
26.7200000286102 1.05437956089349
26.7599999904633 1.04962483913171
26.8000001907349 1.02485614687702
26.8400001525879 0.995293188401687
26.8800001144409 0.974964295315154
26.9200000762939 0.957580459892909
26.960000038147 0.949215347301679
27 0.921619294422794
27.0400002002716 0.8712841427799
27.0800001621246 0.865582629745157
27.1200001239777 0.846861018799135
27.1600000858307 0.834693277341888
27.2000000476837 0.813130120003853
27.2400000095367 0.829569657121442
27.2800002098083 0.825765048896103
27.3200001716614 0.774733331603395
27.3600001335144 0.76722811475868
27.4000000953674 0.747325741726035
27.4400000572205 0.723436952308069
27.4800000190735 0.71094404557987
27.5199999809265 0.735615644553743
27.5600001811981 0.73151414010295
27.6000001430511 0.723362848046036
27.6400001049042 0.748736988871685
27.6800000667572 0.775342890325627
27.7200000286102 0.754046989290428
27.7599999904633 0.750935219382309
27.8000001907349 0.772594171025342
27.8400001525879 0.769264295290918
27.8800001144409 0.748634157110951
27.9200000762939 0.686512789697967
27.960000038147 0.680481733556663
28 0.673092564535698
28.0400002002716 0.631512366006671
28.0800001621246 0.632874918267727
28.1200001239777 0.613223702888927
28.1600000858307 0.608664624065168
28.2000000476837 0.620396671920389
28.2400000095367 0.586853965621933
28.2800002098083 0.636829186215751
28.3200001716614 0.644050985322809
28.3600001335144 0.639759528834164
28.4000000953674 0.685457077862031
28.4400000572205 0.712704041109344
28.4800000190735 0.708106064432922
28.5199999809265 0.745509620804733
28.5600001811981 0.746388602429261
28.6000001430511 0.775597057090858
28.6400001049042 0.760758321377397
28.6800000667572 0.77193400640396
28.7200000286102 0.770466852608329
28.7599999904633 0.77332490744634
28.8000001907349 0.766444278884544
28.8400001525879 0.658979039578151
28.8800001144409 0.657378492846069
28.9200000762939 0.628159564747654
28.960000038147 0.620226847162164
29 0.59845225653863
29.0400002002716 0.588912452800154
29.0800001621246 0.554826931868338
29.1200001239777 0.516047470986799
29.1600000858307 0.507924405328001
29.2000000476837 0.535231637023761
29.2400000095367 0.50703434141637
29.2800002098083 0.521414699990158
29.3200001716614 0.467646599941609
29.3600001335144 0.459152436401133
29.4000000953674 0.456640640126182
29.4400000572205 0.486196664344293
29.4800000190735 0.485563524457599
29.5199999809265 0.478673092895954
29.5600001811981 0.472959974386668
29.6000001430511 0.528643628347308
29.6400001049042 0.515306046202154
29.6800000667572 0.555210666720541
29.7200000286102 0.520055980331315
29.7599999904633 0.518880124445493
29.8000001907349 0.503461184303436
29.8400001525879 0.48342330055467
29.8800001144409 0.484039681220135
29.9200000762939 0.461437060887748
29.960000038147 0.460617677315596
30 0.478610794074599
30.0400002002716 0.439327509868841
30.0800001621246 0.448298874792385
30.1200001239777 0.40342639717108
30.1600000858307 0.395873517019548
30.2000000476837 0.406102051907925
30.2400000095367 0.341704326187608
30.2800002098083 0.338699499349744
30.3200001716614 0.362775321499224
30.3600001335144 0.308158623749152
30.4000000953674 0.190165820833951
30.4400000572205 0.211298753785216
30.4800000190735 0.21436633383723
30.5199999809265 0.198040557795417
30.5600001811981 0.201553598560005
30.6000001430511 0.205880773786072
30.6400001049042 0.190909807013472
30.6800000667572 0.199010277454244
30.7200000286102 0.205660956993144
30.7599999904633 0.20500511060999
30.8000001907349 0.244483386133404
30.8400001525879 0.274227406860817
30.8800001144409 0.287940605923074
30.9200000762939 0.252746134409878
30.960000038147 0.260471752862544
31 0.240695948874707
31.0400002002716 0.239291830025392
31.0800001621246 0.198151223143207
31.1200001239777 0.212758699712277
31.1600000858307 0.181378543024731
31.2000000476837 0.237535352869602
31.2400000095367 0.225682374984193
31.2800002098083 0.270088671800154
31.3200001716614 0.284303313736791
31.3600001335144 0.286124951453097
31.4000000953674 0.303052104706051
31.4400000572205 0.314679580948081
31.4800000190735 0.320161844182962
31.5199999809265 0.350513268056124
31.5600001811981 0.294433326818256
31.6000001430511 0.21959404342779
31.6400001049042 0.0999495032299266
31.6800000667572 0.0858554281115678
31.7200000286102 0.126833874527097
31.7599999904633 0.145517608338847
31.8000001907349 0.139688382326635
31.8400001525879 0.206791419574616
31.8800001144409 0.262962184427832
31.9200000762939 0.307241193855083
31.960000038147 0.338226376289059
32 0.441588424090603
32.0400002002716 0.474941502526265
32.0800001621246 0.433480015746912
32.1200001239777 0.614622695051674
32.1600000858307 0.643031896445602
32.2000000476837 0.582861581659344
32.2400000095367 0.58724618846102
32.2800002098083 0.658036390448944
32.3200001716614 0.740159777694843
32.3600001335144 0.705793509954762
32.4000000953674 0.693916416980759
32.4400000572205 0.702958781486473
32.4800000190735 0.743914447637634
32.5199999809265 0.739764335872602
32.5600001811981 0.816044675381436
32.6000001430511 0.875164416000407
32.6400001049042 0.874584371670034
32.6800000667572 0.819136720803466
32.7200000286102 0.794705897554688
32.7599999904633 0.786185095204859
32.8000001907349 0.693546071015795
32.8400001525879 0.746387037799311
32.8800001144409 1.14395146049978
32.9200000762939 1.18497166455873
32.960000038147 1.22894113490388
33 1.22728445772491
33.0400002002716 1.25797815724363
33.0800001621246 1.26795787099968
33.1200001239777 1.27320635944933
33.1600000858307 1.27141488185912
33.2000000476837 1.341196347059
33.2400000095367 1.38416733614214
33.2800002098083 1.4836683189435
33.3200001716614 1.47477115567618
33.3600001335144 1.11267810369337
33.4000000953674 0.876470885557915
33.4400000572205 0.916595960523551
33.4800000190735 0.90314067731879
33.5199999809265 0.931439230192232
33.5600001811981 0.912571817986715
33.6000001430511 0.963495801337397
33.6400001049042 0.843929866451444
33.6800000667572 0.764551749839646
33.7200000286102 0.748537425801999
33.7599999904633 0.720913486900789
33.8000001907349 0.74529040987813
33.8400001525879 0.720154074416756
33.8800001144409 0.720566983653109
33.9200000762939 0.672647956404273
33.960000038147 0.625130604835647
34 0.539892951020988
34.0400002002716 0.458189353516504
34.0800001621246 0.369195279047211
34.1200001239777 0.297305799178222
34.1600000858307 0.239364693709839
};
\path [draw=black, fill=black] (axis cs:17,0.49)
--(axis cs:16.5,0.6)
--(axis cs:16.875,0.6)
--(axis cs:16.875,1)
--(axis cs:17.125,1)
--(axis cs:17.125,0.6)
--(axis cs:17.5,0.6)
--cycle;

\path [draw=black, fill=black] (axis cs:17,2.96)
--(axis cs:17.5,2.85)
--(axis cs:17.125,2.85)
--(axis cs:17.125,2.5)
--(axis cs:16.875,2.5)
--(axis cs:16.875,2.85)
--(axis cs:16.5,2.85)
--cycle;

\addplot [line width=2.4000000000000004pt, color1, dashed, forget plot]
table {%
0 0.0381368308942715
0.0400002002716064 0.0390036301211818
0.0800001621246338 0.0398704241815919
0.120000123977661 0.040737218242002
0.160000085830688 0.0416040123024122
0.200000047683716 0.0424708063628223
0.240000009536743 0.0433376004232325
0.28000020980835 0.0442043996501428
0.320000171661377 0.0450711937105529
0.360000133514404 0.045937987770963
0.400000095367432 0.0468047818313732
0.440000057220459 0.0476715758917833
0.480000019073486 0.0485383699521935
0.519999980926514 0.0494051640126036
0.56000018119812 0.0502719632395139
0.600000143051147 0.051138757299924
0.640000104904175 0.0520055513603342
0.680000066757202 0.0528723454207443
0.720000028610229 0.0537391394811545
0.759999990463257 0.0546059335415646
0.800000190734863 0.0554727327684749
0.840000152587891 0.056339526828885
0.880000114440918 0.0572063208892952
0.920000076293945 0.0580731149497053
0.960000038146973 0.0589399090101155
1 0.0598067030705256
1.04000020027161 0.0606735022974359
1.08000016212463 0.061540296357846
1.12000012397766 0.0624070904182562
1.16000008583069 0.0632738844786663
1.20000004768372 0.0641406785390765
1.24000000953674 0.0650074725994866
1.28000020980835 0.0658742718263969
1.32000017166138 0.066741065886807
1.3600001335144 0.0676078599472172
1.40000009536743 0.0684746540076273
1.44000005722046 0.0693414480680375
1.48000001907349 0.0702082421284476
1.51999998092651 0.0710750361888577
1.56000018119812 0.071941835415768
1.60000014305115 0.0728086294761782
1.64000010490417 0.0736754235365883
1.6800000667572 0.0745422175969984
1.72000002861023 0.0754090116574086
1.75999999046326 0.0762758057178187
1.80000019073486 0.077142604944729
1.84000015258789 0.0780093990051391
1.88000011444092 0.0788761930655493
1.92000007629395 0.0797429871259594
1.96000003814697 0.0806097811863696
2 0.0814765752467797
2.04000020027161 0.08234337447369
2.08000016212463 0.0832101685341002
2.12000012397766 0.0840769625945103
2.16000008583069 0.0849437566549204
2.20000004768372 0.0858105507153306
2.24000000953674 0.0866773447757407
2.28000020980835 0.087544144002651
2.32000017166138 0.0884109380630611
2.3600001335144 0.0892777321234713
2.40000009536743 0.0901445261838814
2.44000005722046 0.0910113202442916
2.48000001907349 0.0918781143047017
2.51999998092651 0.0927449083651119
2.56000018119812 0.0936117075920221
2.60000014305115 0.0944785016524323
2.64000010490417 0.0953452957128424
2.6800000667572 0.0962120897732526
2.72000002861023 0.0970788838336627
2.75999999046326 0.0979456778940729
2.80000019073486 0.0988124771209831
2.84000015258789 0.0996792711813933
2.88000011444092 0.100546065241803
2.92000007629395 0.101412859302214
2.96000003814697 0.102279653362624
3 0.103146447423034
3.04000020027161 0.104013246649944
3.08000016212463 0.104880040710354
3.12000012397766 0.105746834770764
3.16000008583069 0.106613628831175
3.20000004768372 0.107480422891585
3.24000000953674 0.108347216951995
3.28000020980835 0.109214016178905
3.32000017166138 0.110080810239315
3.3600001335144 0.110947604299725
3.40000009536743 0.111814398360136
3.44000005722046 0.112681192420546
3.48000001907349 0.113547986480956
3.51999998092651 0.114414780541366
3.56000018119812 0.115281579768276
3.60000014305115 0.116148373828686
3.64000010490417 0.117015167889097
3.6800000667572 0.117881961949507
3.72000002861023 0.118748756009917
3.75999999046326 0.119615550070327
3.80000019073486 0.120482349297237
3.84000015258789 0.121349143357647
3.88000011444092 0.122215937418058
3.92000007629395 0.123082731478468
3.96000003814697 0.123949525538878
4 0.124816319599288
4.04000020027161 0.125683118826198
4.08000016212463 0.126549912886608
4.12000012397766 0.127416706947019
4.16000008583069 0.128283501007429
4.20000004768372 0.129150295067839
4.24000000953674 0.130017089128249
4.28000020980835 0.130883888355159
4.32000017166138 0.131750682415569
4.3600001335144 0.13261747647598
4.40000009536743 0.13348427053639
4.44000005722046 0.1343510645968
4.48000001907349 0.13521785865721
4.51999998092651 0.13608465271762
4.56000018119812 0.13695145194453
4.60000014305115 0.137818246004941
4.64000010490417 0.138685040065351
4.6800000667572 0.139551834125761
4.72000002861023 0.140418628186171
4.75999999046326 0.141285422246581
4.80000019073486 0.142152221473491
4.84000015258789 0.143019015533902
4.88000011444092 0.143885809594312
4.92000007629395 0.144752603654722
4.96000003814697 0.145619397715132
5 0.146486191775542
5.04000020027161 0.147352991002452
5.08000016212463 0.148219785062863
5.12000012397766 0.149086579123273
5.16000008583069 0.149953373183683
5.20000004768372 0.150820167244093
5.24000000953674 0.151686961304503
5.28000020980835 0.152553760531413
5.32000017166138 0.153420554591823
5.3600001335144 0.154287348652234
5.40000009536743 0.155154142712644
5.44000005722046 0.156020936773054
5.48000001907349 0.156887730833464
5.51999998092651 0.157754524893874
5.56000018119812 0.158621324120785
5.60000014305115 0.159488118181195
5.64000010490417 0.160354912241605
5.6800000667572 0.161221706302015
5.72000002861023 0.162088500362425
5.75999999046326 0.162955294422835
5.80000019073486 0.163822093649745
5.84000015258789 0.164688887710156
5.88000011444092 0.165555681770566
5.92000007629395 0.166422475830976
5.96000003814697 0.167289269891386
6 0.168156063951796
6.04000020027161 0.169022863178706
6.08000016212463 0.169889657239117
6.12000012397766 0.170756451299527
6.16000008583069 0.171623245359937
6.20000004768372 0.172490039420347
6.24000000953674 0.173356833480757
6.28000020980835 0.174223632707667
6.32000017166138 0.175090426768078
6.3600001335144 0.175957220828488
6.40000009536743 0.176824014888898
6.44000005722046 0.177690808949308
6.48000001907349 0.178557603009718
6.51999998092651 0.179424397070128
6.56000018119812 0.180291196297039
6.60000014305115 0.181157990357449
6.64000010490417 0.182024784417859
6.6800000667572 0.182891578478269
6.72000002861023 0.183758372538679
6.75999999046326 0.184625166599089
6.80000019073486 0.185491965826
6.84000015258789 0.18635875988641
6.88000011444092 0.18722555394682
6.92000007629395 0.18809234800723
6.96000003814697 0.18895914206764
7 0.18982593612805
7.04000020027161 0.190692735354961
7.08000016212463 0.191559529415371
7.12000012397766 0.192426323475781
7.16000008583069 0.193293117536191
7.20000004768372 0.194159911596601
7.24000000953674 0.195026705657011
7.28000020980835 0.195893504883922
7.32000017166138 0.196760298944332
7.3600001335144 0.197627093004742
7.40000009536743 0.198493887065152
7.44000005722046 0.199360681125562
7.48000001907349 0.200227475185972
7.51999998092651 0.201094269246382
7.56000018119812 0.201961068473293
7.60000014305115 0.202827862533703
7.64000010490417 0.203694656594113
7.6800000667572 0.204561450654523
7.72000002861023 0.205428244714933
7.75999999046326 0.206295038775343
7.80000019073486 0.207161838002254
7.84000015258789 0.208028632062664
7.88000011444092 0.208895426123074
7.92000007629395 0.209762220183484
7.96000003814697 0.210629014243894
8 0.211495808304304
8.04000020027161 0.212362607531215
8.08000016212463 0.213229401591625
8.12000012397766 0.214096195652035
8.16000008583069 0.214962989712445
8.20000004768372 0.215829783772855
8.24000000953674 0.216696577833265
8.28000020980835 0.217563377060176
8.32000017166138 0.218430171120586
8.3600001335144 0.219296965180996
8.40000009536743 0.220163759241406
8.44000005722046 0.221030553301816
8.48000001907349 0.221897347362226
8.51999998092651 0.222764141422637
8.56000018119812 0.223630940649547
8.60000014305115 0.224497734709957
8.64000010490417 0.225364528770367
8.6800000667572 0.226231322830777
8.72000002861023 0.227098116891187
8.75999999046326 0.227964910951598
8.80000019073486 0.228831710178508
8.84000015258789 0.229698504238918
8.88000011444092 0.230565298299328
8.92000007629395 0.231432092359738
8.96000003814697 0.232298886420148
9 0.233165680480559
9.04000020027161 0.234032479707469
9.08000016212463 0.234899273767879
9.12000012397766 0.235766067828289
9.16000008583069 0.236632861888699
9.20000004768372 0.272968044703495
9.24000000953674 0.292814970753036
9.28000020980835 0.311952915996219
9.32000017166138 0.330522149807395
9.3600001335144 0.348663285048164
9.40000009536743 0.366516818535964
9.44000005722046 0.384223247088231
9.48000001907349 0.401923067522402
9.51999998092651 0.419756776655915
9.56000018119812 0.437864980335848
9.60000014305115 0.456387960211966
9.64000010490417 0.475466320077163
9.6800000667572 0.495240556748879
9.72000002861023 0.515851167044549
9.75999999046326 0.53743864778161
9.80000019073486 0.560143634718252
9.84000015258789 0.584106354706585
9.88000011444092 0.60946743642605
9.92000007629395 0.636367376694081
9.96000003814697 0.664946672328117
10 0.695345820145595
10.0400002002716 0.727705515963248
10.0800001621246 0.762165871540666
10.1200001239777 0.798867570591264
10.1600000858307 0.83795110993248
10.2000000476837 0.879556986381751
10.2400000095367 0.923825696756513
10.2800002098083 0.97089802707948
10.3200001716614 1.02091391372285
10.3600001335144 1.07401412558145
10.4000000953674 1.13033915947272
10.4400000572205 1.19002951221409
10.4800000190735 1.253225680623
10.5199999809265 1.31997328347691
10.5600001811981 1.38993885315558
10.6000001430511 1.46269278205633
10.6400001049042 1.53780670661876
10.6800000667572 1.61485185542031
10.7200000286102 1.69339945703845
10.7599999904633 1.77302074005062
10.8000001907349 1.85328741252864
10.8400001525879 1.93376974407171
10.8800001144409 2.01403944018551
10.9200000762939 2.0936677294475
10.960000038147 2.17222584043512
11 2.24928500172586
11.0400002002716 2.32441688311886
11.0800001621246 2.39719181542452
11.1200001239777 2.46718148120997
11.1600000858307 2.53395710905266
11.2000000476837 2.59708992753007
11.2400000095367 2.65615116521964
11.2800002098083 2.71071236164326
11.3200001716614 2.76034409283178
11.3600001335144 2.80461792640916
11.4000000953674 2.84310509095285
11.4400000572205 2.8753768150403
11.4800000190735 2.90100432724897
11.5199999809265 2.91955885615633
11.5600001811981 2.93061167301053
11.6000001430511 2.93373387250011
11.6400001049042 2.92849677186505
11.6800000667572 2.91447159968283
11.7200000286102 2.8912295845309
11.7599999904633 2.8583419549867
11.8000001907349 2.85666361761894
11.8400001525879 2.85787496203076
11.8800001144409 2.85908630644258
11.9200000762939 2.86029765085441
11.960000038147 2.86150899526623
12 2.86272033967805
12.0400002002716 2.86393169131006
12.0800001621246 2.86514303572188
12.1200001239777 2.86635438013371
12.1600000858307 2.86756572454553
12.2000000476837 2.86877706895735
12.2400000095367 2.86998841336918
12.2800002098083 2.87119976500118
12.3200001716614 2.87241110941301
12.3600001335144 2.87362245382483
12.4000000953674 2.87483379823665
12.4400000572205 2.87604514264848
12.4800000190735 2.8772564870603
12.5199999809265 2.87846783147212
12.5600001811981 2.87967918310413
12.6000001430511 2.88089052751595
12.6400001049042 2.88210187192778
12.6800000667572 2.8833132163396
12.7200000286102 2.88452456075142
12.7599999904633 2.88573590516325
12.8000001907349 2.88694725679525
12.8400001525879 2.88815860120708
12.8800001144409 2.8893699456189
12.9200000762939 2.89058129003072
12.960000038147 2.89179263444255
13 2.89300397885437
13.0400002002716 2.89421533048638
13.0800001621246 2.8954266748982
13.1200001239777 2.89663801931002
13.1600000858307 2.89784936372185
13.2000000476837 2.89906070813367
13.2400000095367 2.90027205254549
13.2800002098083 2.9014834041775
13.3200001716614 2.90269474858932
13.3600001335144 2.90390609300115
13.4000000953674 2.90511743741297
13.4400000572205 2.90632878182479
13.4800000190735 2.90754012623662
13.5199999809265 2.90875147064844
13.5600001811981 2.90996282228045
13.6000001430511 2.91117416669227
13.6400001049042 2.91238551110409
13.6800000667572 2.91359685551592
13.7200000286102 2.91480819992774
13.7599999904633 2.91601954433956
13.8000001907349 2.91723089597157
13.8400001525879 2.91844224038339
13.8800001144409 2.91965358479522
13.9200000762939 2.92086492920704
13.960000038147 2.92207627361886
14 2.92328761803069
14.0400002002716 2.92449896966269
14.0800001621246 2.92571031407452
14.1200001239777 2.92692165848634
14.1600000858307 2.92813300289816
14.2000000476837 2.92934434730999
14.2400000095367 2.93055569172181
14.2800002098083 2.93176704335382
14.3200001716614 2.93297838776564
14.3600001335144 2.93418973217746
14.4000000953674 2.93540107658929
14.4400000572205 2.93661242100111
14.4800000190735 2.93782376541293
14.5199999809265 2.93903510982476
14.5600001811981 2.94024646145676
14.6000001430511 2.94145780586859
14.6400001049042 2.94266915028041
14.6800000667572 2.94388049469223
14.7200000286102 2.94509183910406
14.7599999904633 2.94630318351588
14.8000001907349 2.94751453514789
14.8400001525879 2.94872587955971
14.8800001144409 2.94993722397153
14.9200000762939 2.95114856838336
14.960000038147 2.95235991279518
15 2.953571257207
15.2000000476837 2.9596279864863
15.2400000095367 2.96083933089813
15.2800002098083 2.96205068253013
15.3200001716614 2.96326202694196
15.3600001335144 2.96447337135378
15.4000000953674 2.9656847157656
15.4400000572205 2.96689606017743
15.4800000190735 2.96810740458925
15.5199999809265 2.96931874900107
15.5600001811981 2.97053010063308
15.6000001430511 2.9717414450449
15.6400001049042 2.97295278945673
15.6800000667572 2.97416413386855
15.7200000286102 2.97537547828037
15.7599999904633 2.9765868226922
15.8000001907349 2.9777981743242
15.8400001525879 2.97900951873603
15.8800001144409 2.98022086314785
15.9200000762939 2.98143220755967
15.960000038147 2.9826435519715
16 2.98385489638332
16.0400002002716 2.98506624801533
16.0800001621246 2.98627759242715
16.1200001239777 2.98748893683897
16.1600000858307 2.9887002812508
16.2000000476837 2.98991162566262
16.2400000095367 2.99112297007444
16.2800002098083 2.99233432170645
16.3200001716614 2.99354566611827
16.3600001335144 2.9947570105301
16.4000000953674 2.99596835494192
16.4400000572205 2.99717969935374
16.4800000190735 2.99839104376557
16.5199999809265 2.99960238817739
16.5600001811981 3.0008137398094
16.6000001430511 3.00202508422122
16.6400001049042 3.00323642863304
16.6800000667572 3.00444777304487
16.7200000286102 3.00565911745669
16.7599999904633 3.00687046186851
16.8000001907349 3.00808181350052
16.8400001525879 3.00929315791234
16.8800001144409 3.01050450232417
16.9200000762939 3.01171584673599
16.960000038147 3.01292719114781
17 3.01413853555964
17.0400002002716 3.01534988719164
17.0800001621246 3.01656123160347
17.1200001239777 3.01777257601529
17.1600000858307 3.01898392042711
17.2000000476837 3.02019526483894
17.2400000095367 3.02140660925076
17.2800002098083 3.02261796088277
17.3200001716614 3.02382930529459
17.3600001335144 3.02504064970641
17.4000000953674 3.02625199411824
17.4400000572205 3.02746333853006
17.4800000190735 3.02867468294188
17.5199999809265 3.02988602735371
17.5600001811981 3.03109737898571
17.6000001430511 3.03230872339754
17.6400001049042 3.03352006780936
17.6800000667572 3.03473141222118
17.7200000286102 3.03594275663301
};
\addplot [line width=2.4000000000000004pt, color3, dashed, forget plot]
table {%
0.800000190734863 0.621780894541407
0.840000152587891 0.62132236236235
0.880000114440918 0.620863830183293
0.920000076293945 0.620405298004236
0.960000038146973 0.619946765825179
1 0.619488233646122
1.04000020027161 0.619029698733997
1.08000016212463 0.61857116655494
1.12000012397766 0.618112634375883
1.16000008583069 0.617654102196826
1.20000004768372 0.617195570017769
1.24000000953674 0.616737037838712
1.28000020980835 0.616278502926587
1.32000017166138 0.61581997074753
1.3600001335144 0.615361438568473
1.40000009536743 0.614902906389416
1.44000005722046 0.614444374210359
1.48000001907349 0.613985842031302
1.51999998092651 0.613527309852245
1.56000018119812 0.61306877494012
1.60000014305115 0.612610242761063
1.64000010490417 0.612151710582006
1.6800000667572 0.611693178402949
1.72000002861023 0.611234646223892
1.75999999046326 0.610776114044835
1.80000019073486 0.61031757913271
1.84000015258789 0.609859046953653
1.88000011444092 0.609400514774596
1.92000007629395 0.608941982595539
1.96000003814697 0.608483450416482
2 0.608024918237425
2.04000020027161 0.607566383325301
2.08000016212463 0.607107851146244
2.12000012397766 0.606649318967186
2.16000008583069 0.606190786788129
2.20000004768372 0.605732254609072
2.24000000953674 0.605273722430015
2.28000020980835 0.604815187517891
2.32000017166138 0.604356655338834
2.3600001335144 0.603898123159777
2.40000009536743 0.603439590980719
2.44000005722046 0.602981058801662
2.48000001907349 0.602522526622605
2.51999998092651 0.602063994443548
2.56000018119812 0.601605459531424
2.60000014305115 0.601146927352367
2.64000010490417 0.60068839517331
2.6800000667572 0.600229862994253
2.72000002861023 0.599771330815195
2.75999999046326 0.599312798636138
2.80000019073486 0.598854263724014
2.84000015258789 0.598395731544957
2.88000011444092 0.5979371993659
2.92000007629395 0.597478667186843
2.96000003814697 0.597020135007786
3 0.596561602828728
3.04000020027161 0.596103067916604
3.08000016212463 0.595644535737547
3.12000012397766 0.59518600355849
3.16000008583069 0.594727471379433
3.20000004768372 0.594268939200376
3.24000000953674 0.593810407021319
3.28000020980835 0.593351872109194
3.32000017166138 0.592893339930137
3.3600001335144 0.59243480775108
3.40000009536743 0.591976275572023
3.44000005722046 0.591517743392966
3.48000001907349 0.591059211213909
3.51999998092651 0.590600679034852
3.56000018119812 0.590142144122727
3.60000014305115 0.58968361194367
3.64000010490417 0.589225079764613
3.6800000667572 0.588766547585556
3.72000002861023 0.588308015406499
3.75999999046326 0.587849483227442
3.80000019073486 0.587390948315317
3.84000015258789 0.58693241613626
3.88000011444092 0.586473883957203
3.92000007629395 0.586015351778146
3.96000003814697 0.585556819599089
4 0.585098287420032
4.04000020027161 0.584639752507907
4.08000016212463 0.58418122032885
4.12000012397766 0.583722688149793
4.16000008583069 0.583264155970736
4.20000004768372 0.582805623791679
4.24000000953674 0.582347091612622
4.28000020980835 0.581888556700498
4.32000017166138 0.58143002452144
4.3600001335144 0.580971492342383
4.40000009536743 0.580512960163326
4.44000005722046 0.580054427984269
4.48000001907349 0.579595895805212
4.51999998092651 0.579137363626155
4.56000018119812 0.578678828714031
4.60000014305115 0.578220296534974
4.64000010490417 0.577761764355916
4.6800000667572 0.577303232176859
4.72000002861023 0.576844699997802
4.75999999046326 0.576386167818745
4.80000019073486 0.575927632906621
4.84000015258789 0.575469100727564
4.88000011444092 0.575010568548507
4.92000007629395 0.57455203636945
4.96000003814697 0.574093504190392
5 0.573634972011335
5.04000020027161 0.573176437099211
5.08000016212463 0.572717904920154
5.12000012397766 0.572259372741097
5.16000008583069 0.57180084056204
5.20000004768372 0.571342308382982
5.24000000953674 0.570883776203925
5.28000020980835 0.570425241291801
5.32000017166138 0.569966709112744
5.3600001335144 0.569508176933687
5.40000009536743 0.56904964475463
5.44000005722046 0.568591112575573
5.48000001907349 0.568132580396516
5.51999998092651 0.567674048217458
5.56000018119812 0.567215513305334
5.60000014305115 0.566756981126277
5.64000010490417 0.56629844894722
5.6800000667572 0.565839916768163
5.72000002861023 0.565381384589106
5.75999999046326 0.564922852410049
5.80000019073486 0.564464317497924
5.84000015258789 0.564005785318867
5.88000011444092 0.56354725313981
5.92000007629395 0.563088720960753
5.96000003814697 0.562630188781696
6 0.562171656602639
6.04000020027161 0.561713121690514
6.08000016212463 0.561254589511457
6.12000012397766 0.5607960573324
6.16000008583069 0.560337525153343
6.20000004768372 0.559878992974286
6.24000000953674 0.559420460795229
6.28000020980835 0.558961925883104
6.32000017166138 0.558503393704047
6.3600001335144 0.55804486152499
6.40000009536743 0.557586329345933
6.44000005722046 0.557127797166876
6.48000001907349 0.556669264987819
6.51999998092651 0.556210732808762
6.56000018119812 0.555752197896637
6.60000014305115 0.55529366571758
6.64000010490417 0.554835133538523
6.6800000667572 0.554376601359466
6.72000002861023 0.553918069180409
6.75999999046326 0.553459537001352
6.80000019073486 0.553001002089228
6.84000015258789 0.55254246991017
6.88000011444092 0.552083937731113
6.92000007629395 0.551625405552056
6.96000003814697 0.551166873372999
7 0.550708341193942
7.04000020027161 0.550249806281818
7.08000016212463 0.549791274102761
7.12000012397766 0.549332741923704
7.16000008583069 0.548874209744646
7.20000004768372 0.548415677565589
7.24000000953674 0.547957145386532
7.28000020980835 0.547498610474408
7.32000017166138 0.547040078295351
7.3600001335144 0.546581546116294
7.40000009536743 0.546123013937237
7.44000005722046 0.545664481758179
7.48000001907349 0.545205949579122
7.51999998092651 0.544747417400065
7.56000018119812 0.544288882487941
7.60000014305115 0.543830350308884
7.64000010490417 0.543371818129827
7.6800000667572 0.54291328595077
7.72000002861023 0.542454753771712
7.75999999046326 0.541996221592655
7.80000019073486 0.541537686680531
7.84000015258789 0.541079154501474
7.88000011444092 0.540620622322417
7.92000007629395 0.54016209014336
7.96000003814697 0.539703557964303
8 0.539245025785246
8.04000020027161 0.538786490873121
8.08000016212463 0.538327958694064
8.12000012397766 0.537869426515007
8.16000008583069 0.53741089433595
8.20000004768372 0.536952362156893
8.24000000953674 0.536493829977836
8.28000020980835 0.536035295065711
8.32000017166138 0.535576762886654
8.3600001335144 0.535118230707597
8.40000009536743 0.53465969852854
8.44000005722046 0.534201166349483
8.48000001907349 0.533742634170426
8.51999998092651 0.533284101991369
8.56000018119812 0.532825567079244
8.60000014305115 0.532367034900187
8.64000010490417 0.53190850272113
8.6800000667572 0.531449970542073
8.72000002861023 0.530991438363016
8.75999999046326 0.530532906183959
8.80000019073486 0.530074371271834
8.84000015258789 0.529615839092777
8.88000011444092 0.52915730691372
8.92000007629395 0.528698774734663
8.96000003814697 0.528240242555606
9 0.527781710376549
9.04000020027161 0.527323175464425
9.08000016212463 0.526864643285367
9.12000012397766 0.52640611110631
9.16000008583069 0.525947578927253
9.20000004768372 0.525489046748196
9.24000000953674 0.525030514569139
9.28000020980835 0.524571979657015
9.32000017166138 0.524113447477958
9.3600001335144 0.523654915298901
9.40000009536743 0.523196383119843
9.44000005722046 0.522737850940786
9.48000001907349 0.522279318761729
9.51999998092651 0.521820786582672
9.56000018119812 0.521362251670548
9.60000014305115 0.520903719491491
9.64000010490417 0.520445187312433
9.6800000667572 0.519986655133376
9.72000002861023 0.519528122954319
9.75999999046326 0.519069590775262
9.80000019073486 0.518611055863138
9.84000015258789 0.518152523684081
9.88000011444092 0.517693991505024
9.92000007629395 0.517235459325966
9.96000003814697 0.516776927146909
10 0.516318394967852
10.0400002002716 0.515859860055728
10.0800001621246 0.515401327876671
10.1200001239777 0.514942795697614
10.1600000858307 0.514484263518557
10.2000000476837 0.5140257313395
10.2400000095367 0.513567199160442
10.2800002098083 0.513108664248318
10.3200001716614 0.512650132069261
10.3600001335144 0.512191599890204
10.4000000953674 0.511733067711147
10.4400000572205 0.51127453553209
10.4800000190735 0.510816003353033
10.5199999809265 0.510357471173976
10.5600001811981 0.509898936261851
10.6000001430511 0.509440404082794
10.6400001049042 0.508981871903737
10.6800000667572 0.50852333972468
10.7200000286102 0.508064807545623
10.7599999904633 0.507606275366566
10.8000001907349 0.507147740454441
10.8400001525879 0.506689208275384
10.8800001144409 0.506230676096327
10.9200000762939 0.50577214391727
10.960000038147 0.505313611738213
11 0.504855079559156
11.0400002002716 0.504396544647031
11.0800001621246 0.503938012467974
11.1200001239777 0.503479480288917
11.1600000858307 0.50302094810986
11.2000000476837 0.502562415930803
11.2400000095367 0.502103883751746
11.2800002098083 0.501645348839621
11.3200001716614 0.501186816660564
11.3600001335144 0.500728284481507
11.4000000953674 0.50026975230245
11.4400000572205 0.499811220123393
11.4800000190735 0.499352687944336
11.5199999809265 0.498894155765279
11.5600001811981 0.498435620853155
11.6000001430511 0.497977088674097
11.6400001049042 0.49751855649504
11.6800000667572 0.497060024315983
11.7200000286102 0.496601492136926
11.7599999904633 0.496142959957869
11.8000001907349 0.495684425045745
11.8400001525879 0.495225892866688
11.8800001144409 0.49476736068763
11.9200000762939 0.494308828508573
11.960000038147 0.493850296329516
12 0.493391764150459
12.0400002002716 0.492933229238335
12.0800001621246 0.492474697059278
12.1200001239777 0.492016164880221
12.1600000858307 0.491557632701163
12.2000000476837 0.491099100522106
12.2400000095367 0.490640568343049
12.2800002098083 0.490182033430925
12.3200001716614 0.489723501251868
12.3600001335144 0.489264969072811
12.4000000953674 0.488806436893754
12.4400000572205 0.488347904714697
12.4800000190735 0.487889372535639
12.5199999809265 0.487430840356582
12.5600001811981 0.486972305444458
12.6000001430511 0.486513773265401
12.6400001049042 0.486055241086344
12.6800000667572 0.485596708907287
12.7200000286102 0.48513817672823
12.7599999904633 0.484679644549172
12.8000001907349 0.484221109637048
12.8400001525879 0.483762577457991
12.8800001144409 0.483304045278934
12.9200000762939 0.482845513099877
12.960000038147 0.48238698092082
13 0.481928448741763
13.0400002002716 0.481469913829638
13.0800001621246 0.481011381650581
13.1200001239777 0.480552849471524
13.1600000858307 0.480094317292467
13.2000000476837 0.47963578511341
13.2400000095367 0.479177252934353
13.2800002098083 0.478718718022228
13.3200001716614 0.478260185843171
13.3600001335144 0.477801653664114
13.4000000953674 0.477343121485057
13.4400000572205 0.476884589306
13.4800000190735 0.476426057126943
13.5199999809265 0.475967524947886
13.5600001811981 0.475508990035761
13.6000001430511 0.475050457856704
13.6400001049042 0.474591925677647
13.6800000667572 0.47413339349859
13.7200000286102 0.473674861319533
13.7599999904633 0.473216329140476
13.8000001907349 0.472757794228351
13.8400001525879 0.472299262049294
13.8800001144409 0.471840729870237
13.9200000762939 0.47138219769118
13.960000038147 0.470923665512123
14 0.470465133333066
14.0400002002716 0.470006598420942
14.0800001621246 0.469548066241884
14.1200001239777 0.469089534062827
14.1600000858307 0.46863100188377
14.2000000476837 0.468172469704713
14.2400000095367 0.467713937525656
14.2800002098083 0.467255402613532
14.3200001716614 0.466796870434475
14.3600001335144 0.466338338255418
14.4000000953674 0.46587980607636
14.4400000572205 0.465421273897303
14.4800000190735 0.464962741718246
14.5199999809265 0.464504209539189
14.5600001811981 0.464045674627065
14.6000001430511 0.463587142448008
14.6400001049042 0.463128610268951
14.6800000667572 0.462670078089893
14.7200000286102 0.462211545910836
14.7599999904633 0.461753013731779
14.8000001907349 0.461294478819655
14.8400001525879 0.460835946640598
14.8800001144409 0.460377414461541
14.9200000762939 0.459918882282484
14.960000038147 0.459460350103427
15 0.459001817924369
15.0400002002716 0.458543283012245
15.0800001621246 0.458084750833188
15.1200001239777 0.457626218654131
15.1600000858307 0.457167686475074
15.2000000476837 0.456709154296017
15.2400000095367 0.45625062211696
15.2800002098083 0.455792087204835
15.3200001716614 0.455333555025778
15.3600001335144 0.454875022846721
15.4000000953674 0.454416490667664
15.4400000572205 0.453957958488607
15.4800000190735 0.45349942630955
15.5199999809265 0.453040894130493
15.5600001811981 0.452582359218368
15.6000001430511 0.452123827039311
15.6400001049042 0.451665294860254
15.6800000667572 0.451206762681197
15.7200000286102 0.45074823050214
15.7599999904633 0.450289698323083
15.8000001907349 0.449831163410958
15.8400001525879 0.449372631231901
15.8800001144409 0.448914099052844
15.9200000762939 0.448455566873787
15.960000038147 0.44799703469473
16 0.447538502515673
16.0400002002716 0.447079967603548
16.0800001621246 0.446621435424491
16.1200001239777 0.446162903245434
16.1600000858307 0.445704371066377
16.2000000476837 0.44524583888732
16.2400000095367 0.444787306708263
16.2800002098083 0.444328771796138
16.3200001716614 0.443870239617081
16.3600001335144 0.443411707438024
16.4000000953674 0.442953175258967
16.4400000572205 0.44249464307991
16.4800000190735 0.442036110900853
16.5199999809265 0.441577578721796
16.5600001811981 0.441119043809672
16.6000001430511 0.440660511630614
16.6400001049042 0.440201979451557
16.6800000667572 0.4397434472725
16.7200000286102 0.439284915093443
16.7599999904633 0.438826382914386
16.8000001907349 0.438367848002262
16.8400001525879 0.437909315823205
16.8800001144409 0.437450783644148
16.9200000762939 0.43699225146509
16.960000038147 0.436533719286033
17 0.436075187106976
17.0400002002716 0.435616652194852
17.0800001621246 0.435158120015795
17.1200001239777 0.434699587836738
17.1600000858307 0.434241055657681
17.2000000476837 0.433782523478623
17.2400000095367 0.433323991299566
17.2800002098083 0.432865456387442
17.3200001716614 0.432406924208385
17.3600001335144 0.431948392029328
17.4000000953674 0.431489859850271
17.4400000572205 0.431031327671214
17.4800000190735 0.430572795492157
17.5199999809265 0.430114263313099
17.5600001811981 0.429655728400975
17.6000001430511 0.429197196221918
17.6400001049042 0.428738664042861
17.6800000667572 0.428280131863804
17.7200000286102 0.427821599684747
17.7599999904633 0.42736306750569
17.8000001907349 0.426904532593565
17.8400001525879 0.426446000414508
17.8800001144409 0.425987468235451
17.9200000762939 0.425528936056394
17.960000038147 0.425070403877337
18 0.42461187169828
18.0400002002716 0.424153336786155
18.0800001621246 0.423694804607098
18.1200001239777 0.423236272428041
18.1600000858307 0.422777740248984
18.2000000476837 0.422319208069927
18.2400000095367 0.42186067589087
18.2800002098083 0.421402140978745
18.3200001716614 0.420943608799688
18.3600001335144 0.420485076620631
18.4000000953674 0.420026544441574
18.4400000572205 0.419568012262517
18.4800000190735 0.41910948008346
18.5199999809265 0.418650947904403
18.5600001811981 0.418192412992278
18.6000001430511 0.417733880813221
18.6400001049042 0.417275348634164
18.6800000667572 0.416816816455107
18.7200000286102 0.41635828427605
18.7599999904633 0.415899752096993
18.8000001907349 0.415441217184869
18.8400001525879 0.414982685005811
18.8800001144409 0.414524152826754
18.9200000762939 0.414065620647697
18.960000038147 0.41360708846864
19 0.413148556289583
19.0400002002716 0.412690021377459
19.0800001621246 0.412231489198402
19.1200001239777 0.411772957019344
19.1600000858307 0.411314424840287
19.2000000476837 0.41085589266123
19.2400000095367 0.410397360482173
19.2800002098083 0.409938825570049
19.3200001716614 0.409480293390992
19.3600001335144 0.409021761211935
19.4000000953674 0.408563229032878
19.4400000572205 0.40810469685382
19.4800000190735 0.407646164674763
19.5199999809265 0.407187632495706
19.5600001811981 0.406729097583582
19.6000001430511 0.406270565404525
19.6400001049042 0.405812033225468
19.6800000667572 0.405353501046411
19.7200000286102 0.404894968867353
19.7599999904633 0.404436436688296
19.8000001907349 0.403977901776172
19.8400001525879 0.403519369597115
19.8800001144409 0.403060837418058
19.9200000762939 0.402602305239001
19.960000038147 0.402143773059944
20 0.401685240880887
20.0400002002716 0.401226705968762
20.0800001621246 0.400768173789705
20.1200001239777 0.400309641610648
20.1600000858307 0.399851109431591
20.2000000476837 0.399392577252534
20.2400000095367 0.398934045073477
20.2800002098083 0.398475510161352
20.3200001716614 0.398016977982295
20.3600001335144 0.397558445803238
20.4000000953674 0.397099913624181
20.4400000572205 0.396641381445124
20.4800000190735 0.396182849266067
20.5199999809265 0.39572431708701
20.5600001811981 0.395265782174885
20.6000001430511 0.394807249995828
20.6400001049042 0.394348717816771
20.6800000667572 0.393890185637714
20.7200000286102 0.393431653458657
20.7599999904633 0.3929731212796
20.8000001907349 0.392514586367475
20.8400001525879 0.392056054188418
20.8800001144409 0.391597522009361
20.9200000762939 0.391138989830304
20.960000038147 0.390680457651247
21 0.39022192547219
21.0400002002716 0.389763390560065
21.0800001621246 0.389304858381008
21.1200001239777 0.388846326201951
21.1600000858307 0.388387794022894
21.2000000476837 0.387929261843837
21.2400000095367 0.38747072966478
21.2800002098083 0.387012194752656
21.3200001716614 0.386553662573599
21.3600001335144 0.386095130394541
21.4000000953674 0.385636598215484
21.4400000572205 0.385178066036427
21.4800000190735 0.38471953385737
21.5199999809265 0.384261001678313
21.5600001811981 0.383802466766189
21.6000001430511 0.383343934587132
21.6400001049042 0.382885402408074
21.6800000667572 0.382426870229017
21.7200000286102 0.38196833804996
21.7599999904633 0.381509805870903
21.8000001907349 0.381051270958779
21.8400001525879 0.380592738779722
21.8800001144409 0.380134206600665
21.9200000762939 0.379675674421608
21.960000038147 0.37921714224255
22 0.378758610063493
22.0400002002716 0.378300075151369
22.0800001621246 0.377841542972312
22.1200001239777 0.377383010793255
22.1600000858307 0.376924478614198
22.2000000476837 0.376465946435141
22.2400000095367 0.376007414256083
22.2800002098083 0.375548879343959
22.3200001716614 0.375090347164902
22.3600001335144 0.374631814985845
22.4000000953674 0.374173282806788
22.4400000572205 0.373714750627731
22.4800000190735 0.373256218448674
22.5199999809265 0.372797686269617
22.5600001811981 0.372339151357492
22.6000001430511 0.371880619178435
22.6400001049042 0.371422086999378
22.6800000667572 0.370963554820321
22.7200000286102 0.370505022641264
22.7599999904633 0.370046490462207
22.8000001907349 0.369587955550082
22.8400001525879 0.369129423371025
22.8800001144409 0.368670891191968
22.9200000762939 0.368212359012911
22.960000038147 0.367753826833854
23 0.367295294654797
23.0400002002716 0.366836759742672
23.0800001621246 0.366378227563615
23.1200001239777 0.365919695384558
23.1600000858307 0.365461163205501
23.2000000476837 0.365002631026444
23.2400000095367 0.364544098847387
23.2800002098083 0.364085563935262
23.3200001716614 0.363627031756205
23.3600001335144 0.363168499577148
23.4000000953674 0.362709967398091
23.4400000572205 0.362251435219034
23.4800000190735 0.361792903039977
23.5199999809265 0.36133437086092
23.5600001811981 0.360875835948795
23.6000001430511 0.360417303769738
23.6400001049042 0.359958771590681
23.6800000667572 0.359500239411624
23.7200000286102 0.359041707232567
23.7599999904633 0.35858317505351
23.8000001907349 0.358124640141386
23.8400001525879 0.357666107962329
23.8800001144409 0.357207575783271
23.9200000762939 0.356749043604214
23.960000038147 0.356290511425157
24 0.3558319792461
24.0400002002716 0.355373444333976
24.0800001621246 0.354914912154919
24.1200001239777 0.354456379975862
24.1600000858307 0.353997847796804
24.2000000476837 0.353539315617747
24.2400000095367 0.35308078343869
24.2800002098083 0.352622248526566
24.3200001716614 0.352163716347509
24.3600001335144 0.351705184168452
24.4000000953674 0.351246651989395
24.4400000572205 0.350788119810338
24.4800000190735 0.35032958763128
24.5199999809265 0.349871055452223
24.5600001811981 0.349412520540099
24.6000001430511 0.348953988361042
24.6400001049042 0.348495456181985
24.6800000667572 0.348036924002928
24.7200000286102 0.347578391823871
24.7599999904633 0.347119859644813
24.8000001907349 0.346661324732689
24.8400001525879 0.346202792553632
24.8800001144409 0.345744260374575
24.9200000762939 0.345285728195518
24.960000038147 0.344827196016461
25 0.344368663837404
25.0400002002716 0.343910128925279
25.0800001621246 0.343451596746222
25.1200001239777 0.342993064567165
25.1600000858307 0.342534532388108
25.2000000476837 0.342076000209051
25.2400000095367 0.341617468029994
25.2800002098083 0.341158933117869
25.3200001716614 0.340700400938812
25.3600001335144 0.340241868759755
25.4000000953674 0.339783336580698
25.4400000572205 0.339324804401641
25.4800000190735 0.338866272222584
25.5199999809265 0.338407740043527
25.5600001811981 0.337949205131402
25.6000001430511 0.337490672952345
25.6400001049042 0.337032140773288
25.6800000667572 0.336573608594231
25.7200000286102 0.336115076415174
25.7599999904633 0.335656544236117
25.8000001907349 0.335198009323992
25.8400001525879 0.334739477144935
25.8800001144409 0.334280944965878
25.9200000762939 0.333822412786821
25.960000038147 0.333363880607764
26 0.332905348428707
26.0400002002716 0.332446813516583
26.0800001621246 0.331988281337525
26.1200001239777 0.331529749158468
26.1600000858307 0.331071216979411
26.2000000476837 0.330612684800354
26.2400000095367 0.330154152621297
26.2800002098083 0.329695617709173
26.3200001716614 0.329237085530116
26.3600001335144 0.328778553351058
26.4000000953674 0.328320021172001
26.4400000572205 0.327861488992944
26.4800000190735 0.327402956813887
26.5199999809265 0.32694442463483
26.5600001811981 0.326485889722706
26.6000001430511 0.326027357543649
26.6400001049042 0.325568825364592
26.6800000667572 0.325110293185534
26.7200000286102 0.324651761006477
26.7599999904633 0.32419322882742
26.8000001907349 0.323734693915296
26.8400001525879 0.323276161736239
26.8800001144409 0.322817629557182
26.9200000762939 0.322359097378125
26.960000038147 0.321900565199068
27 0.32144203302001
27.0400002002716 0.320983498107886
27.0800001621246 0.320524965928829
27.1200001239777 0.320066433749772
27.1600000858307 0.319607901570715
27.2000000476837 0.319149369391658
27.2400000095367 0.318690837212601
27.2800002098083 0.318232302300476
27.3200001716614 0.317773770121419
27.3600001335144 0.317315237942362
27.4000000953674 0.316856705763305
27.4400000572205 0.316398173584248
27.4800000190735 0.315939641405191
27.5199999809265 0.315481109226134
27.5600001811981 0.315022574314009
27.6000001430511 0.314564042134952
27.6400001049042 0.314105509955895
27.6800000667572 0.313646977776838
27.7200000286102 0.313188445597781
27.7599999904633 0.312729913418724
27.8000001907349 0.312271378506599
27.8400001525879 0.311812846327542
27.8800001144409 0.311354314148485
27.9200000762939 0.310895781969428
27.960000038147 0.310437249790371
28 0.309978717611314
28.0400002002716 0.309520182699189
28.0800001621246 0.309061650520132
28.1200001239777 0.308603118341075
28.1600000858307 0.308144586162018
28.2000000476837 0.307686053982961
28.2400000095367 0.307227521803904
28.2800002098083 0.30676898689178
28.3200001716614 0.306310454712722
28.3600001335144 0.305851922533665
28.4000000953674 0.305393390354608
28.4400000572205 0.304934858175551
28.4800000190735 0.304476325996494
28.5199999809265 0.304017793817437
28.5600001811981 0.303559258905313
28.6000001430511 0.303100726726255
28.6400001049042 0.302642194547198
28.6800000667572 0.302183662368141
28.7200000286102 0.301725130189084
28.7599999904633 0.301266598010027
28.8000001907349 0.300808063097903
28.8400001525879 0.300349530918846
28.8800001144409 0.299890998739789
28.9200000762939 0.299432466560731
28.960000038147 0.298973934381674
29 0.298515402202617
29.0400002002716 0.298056867290493
29.0800001621246 0.297598335111436
29.1200001239777 0.297139802932379
29.1600000858307 0.296681270753322
29.2000000476837 0.296222738574264
29.2400000095367 0.295764206395207
29.2800002098083 0.295305671483083
29.3200001716614 0.294847139304026
29.3600001335144 0.294388607124969
29.4000000953674 0.293930074945912
29.4400000572205 0.293471542766855
29.4800000190735 0.293013010587797
29.5199999809265 0.29255447840874
29.5600001811981 0.292095943496616
29.6000001430511 0.291637411317559
29.6400001049042 0.291178879138502
29.6800000667572 0.290720346959445
29.7200000286102 0.290261814780388
29.7599999904633 0.289803282601331
29.8000001907349 0.289344747689206
29.8400001525879 0.288886215510149
29.8800001144409 0.288427683331092
29.9200000762939 0.287969151152035
29.960000038147 0.287510618972978
30 0.287052086793921
30.0400002002716 0.286593551881796
30.0800001621246 0.286135019702739
30.1200001239777 0.285676487523682
30.1600000858307 0.285217955344625
30.2000000476837 0.284759423165568
30.2400000095367 0.284300890986511
30.2800002098083 0.283842356074386
30.3200001716614 0.283383823895329
30.3600001335144 0.282925291716272
30.4000000953674 0.282466759537215
30.4400000572205 0.282008227358158
30.4800000190735 0.281549695179101
30.5199999809265 0.281091163000044
30.5600001811981 0.280632628087919
30.6000001430511 0.280174095908862
30.6400001049042 0.279715563729805
30.6800000667572 0.279257031550748
30.7200000286102 0.278798499371691
30.7599999904633 0.278339967192634
30.8000001907349 0.277881432280509
30.8400001525879 0.277422900101452
30.8800001144409 0.276964367922395
30.9200000762939 0.276505835743338
30.960000038147 0.276047303564281
31 0.275588771385224
31.0400002002716 0.2751302364731
31.0800001621246 0.274671704294043
31.1200001239777 0.274213172114985
31.1600000858307 0.273754639935928
31.2000000476837 0.273296107756871
31.2400000095367 0.272837575577814
31.2800002098083 0.27237904066569
31.3200001716614 0.271920508486633
31.3600001335144 0.271461976307576
31.4000000953674 0.271003444128519
31.4400000572205 0.270544911949461
31.4800000190735 0.270086379770404
31.5199999809265 0.269627847591347
31.5600001811981 0.269169312679223
31.6000001430511 0.268710780500166
31.6400001049042 0.268252248321109
31.6800000667572 0.267793716142052
31.7200000286102 0.267335183962994
31.7599999904633 0.266876651783937
31.8000001907349 0.266418116871813
31.8400001525879 0.265959584692756
31.8800001144409 0.265501052513699
31.9200000762939 0.265042520334642
31.960000038147 0.264583988155585
32 0.264125455976527
32.0400002002716 0.263666921064403
32.0800001621246 0.263208388885346
32.1200001239777 0.262749856706289
32.1600000858307 0.262291324527232
32.2000000476837 0.261832792348175
32.2400000095367 0.261374260169118
32.2800002098083 0.260915725256993
32.3200001716614 0.260457193077936
32.3600001335144 0.259998660898879
32.4000000953674 0.259540128719822
32.4400000572205 0.259081596540765
32.4800000190735 0.258623064361708
32.5199999809265 0.258164532182651
32.5600001811981 0.257705997270526
32.6000001430511 0.257247465091469
32.6400001049042 0.256788932912412
32.6800000667572 0.256330400733355
32.7200000286102 0.255871868554298
32.7599999904633 0.255413336375241
32.8000001907349 0.254954801463116
32.8400001525879 0.254496269284059
32.8800001144409 0.254037737105002
32.9200000762939 0.253579204925945
32.960000038147 0.253120672746888
33 0.252662140567831
33.0400002002716 0.252203605655706
33.0800001621246 0.251745073476649
33.1200001239777 0.251286541297592
33.1600000858307 0.250828009118535
33.2000000476837 0.250369476939478
33.2400000095367 0.249910944760421
33.2800002098083 0.249452409848297
33.3200001716614 0.248993877669239
33.3600001335144 0.248535345490182
33.4000000953674 0.248076813311125
33.4400000572205 0.247618281132068
33.4800000190735 0.247159748953011
33.5199999809265 0.246701216773954
33.5600001811981 0.24624268186183
33.6000001430511 0.245784149682773
33.6400001049042 0.245325617503715
33.6800000667572 0.244867085324658
33.7200000286102 0.244408553145601
33.7599999904633 0.243950020966544
33.8000001907349 0.24349148605442
33.8400001525879 0.243032953875363
33.8800001144409 0.242574421696306
33.9200000762939 0.242115889517248
33.960000038147 0.241657357338191
34 0.241198825159134
34.0400002002716 0.24074029024701
34.0800001621246 0.240281758067953
34.1200001239777 0.239823225888896
34.1600000858307 0.239364693709839
};
\node at (axis cs:17,1)[
  anchor=south,
  text=black,
  rotate=0.0
]{ Pickup truck};
\node at (axis cs:17,2.5)[
  anchor=north,
  text=black,
  rotate=0.0
]{ Station wagon};
\end{axis}

\end{tikzpicture}}
%	\caption{States of the dynamic environment.}
%	\label{fig:example dynamic environment}
%\end{figure}

%\begin{figure*}
%	\centering
%	\setlength\figureheight{170pt}
%	\setlength\figurewidth{520pt}
%	% This file was created by matplotlib2tikz v0.6.14.
\begin{tikzpicture}

\begin{axis}[
xlabel={Time [s]},
xmin=-7, xmax=42.325,
ymin=0, ymax=6,
width=\figurewidth,
height=\figureheight,
tick align=outside,
tick pos=left,
x grid style={lightgray!92.026143790849673!black},
clip marker paths,
ytick=\empty,
xtick={0,5,10,15,20,25,30,35,40},
y axis line style={draw opacity=0}
]
\path [draw=white!80.0!black, fill=white!80.0!black] (axis cs:0,0)
--(axis cs:42.325,0)
--(axis cs:42.325,1)
--(axis cs:0,1)
--cycle;

\path [draw=lightgray!93.333333333333329!black, fill=lightgray!93.333333333333329!black] (axis cs:0,1)
--(axis cs:42.325,1)
--(axis cs:42.325,2)
--(axis cs:0,2)
--cycle;

\path [draw=white!80.0!black, fill=white!80.0!black] (axis cs:0,2)
--(axis cs:17.7200000286102,2)
--(axis cs:17.7200000286102,3)
--(axis cs:0,3)
--cycle;

\path [draw=lightgray!93.333333333333329!black, fill=lightgray!93.333333333333329!black] (axis cs:0,3)
--(axis cs:17.7200000286102,3)
--(axis cs:17.7200000286102,4)
--(axis cs:0,4)
--cycle;

\path [draw=white!80.0!black, fill=white!80.0!black] (axis cs:0.800000190734863,4)
--(axis cs:34.1600000858307,4)
--(axis cs:34.1600000858307,5)
--(axis cs:0.800000190734863,5)
--cycle;

\path [draw=lightgray!93.333333333333329!black, fill=lightgray!93.333333333333329!black] (axis cs:0.800000190734863,5)
--(axis cs:34.1600000858307,5)
--(axis cs:34.1600000858307,6)
--(axis cs:0.800000190734863,6)
--cycle;

\addplot [semithick, black, forget plot]
table {%
0 0
0 1
};
\addplot [semithick, black, forget plot]
table {%
9 0
9 1
};
\addplot [semithick, black, forget plot]
table {%
11 0
11 1
};
\addplot [semithick, black, forget plot]
table {%
18 0
18 1
};
\addplot [semithick, black, forget plot]
table {%
23 0
23 1
};
\addplot [semithick, black, forget plot]
table {%
0 1
0 2
};
\addplot [semithick, black, forget plot]
table {%
17 1
17 2
};
\addplot [semithick, black, forget plot]
table {%
23 1
23 2
};
\addplot [semithick, black, forget plot]
table {%
31.25 1
31.25 2
};
\addplot [semithick, black, forget plot]
table {%
36.25 1
36.25 2
};
\addplot [semithick, black, forget plot]
table {%
42.325 0
42.325 2
};
\addplot [semithick, black, forget plot]
table {%
0 2
0 3
};
\addplot [semithick, black, forget plot]
table {%
0 3
0 4
};
\addplot [semithick, black, forget plot]
table {%
17.7200000286102 2
17.7200000286102 3
};
\addplot [semithick, black, forget plot]
table {%
17.7200000286102 3
17.7200000286102 4
};
\addplot [semithick, black, forget plot]
table {%
13.2000000476837 2
13.2000000476837 3
};
\addplot [semithick, black, forget plot]
table {%
9.20000004768372 3
9.20000004768372 4
};
\addplot [semithick, black, forget plot]
table {%
11.8000001907349 3
11.8000001907349 4
};
\addplot [semithick, black, forget plot]
table {%
0.800000190734863 4
0.800000190734863 5
};
\addplot [semithick, black, forget plot]
table {%
0.800000190734863 5
0.800000190734863 6
};
\addplot [semithick, black, forget plot]
table {%
34.1600000858307 4
34.1600000858307 5
};
\addplot [semithick, black, forget plot]
table {%
34.1600000858307 5
34.1600000858307 6
};
\addplot [semithick, black, forget plot]
table {%
0 1
42.325 1
};
\addplot [semithick, black, forget plot]
table {%
0 2
42.325 2
};
\addplot [semithick, black, forget plot]
table {%
0 3
42.325 3
};
\addplot [semithick, black, forget plot]
table {%
0 4
42.325 4
};
\addplot [semithick, black, forget plot]
table {%
0 5
42.325 5
};
\node at (axis cs:4.4875,0.5)[
  scale=0.75,
  text=black,
  rotate=0.0
]{ Accelerating};
\node at (axis cs:9.9875,0.5)[
  scale=0.75,
  text=black,
  rotate=0.0,
  align=center
]{ Bra-\\
king};
\node at (axis cs:14.4875,0.5)[
  scale=0.75,
  text=black,
  rotate=0.0
]{ Cruising};
\node at (axis cs:20.4875,0.5)[
  scale=0.75,
  text=black,
  rotate=0.0
]{ Accelerating};
\node at (axis cs:32.6625,0.5)[
  scale=0.75,
  text=black,
  rotate=0.0
]{ Cruising};
\node at (axis cs:8.4875,1.5)[
  scale=0.75,
  text=black,
  rotate=0.0
]{ Straight};
\node at (axis cs:19.9875,1.5)[
  scale=0.75,
  text=black,
  rotate=0.0,
  align=center
]{ Lane\\
Change};
\node at (axis cs:27.1125,1.5)[
  scale=0.75,
  text=black,
  rotate=0.0
]{ Straight};
\node at (axis cs:33.7375,1.5)[
  scale=0.75,
  text=black,
  rotate=0.0,
  align=center
]{ Lane\\
Change};
\node at (axis cs:39.2875,1.5)[
  scale=0.75,
  text=black,
  rotate=0.0
]{ Straight};
\node at (axis cs:6.60000002384186,2.5)[
  scale=0.75,
  text=black,
  rotate=0.0
]{ Accelerating};
\node at (axis cs:15.460000038147,2.5)[
  scale=0.75,
  text=black,
  rotate=0.0
]{ Cruising};
\node at (axis cs:4.60000002384186,3.5)[
  scale=0.75,
  text=black,
  rotate=0.0
]{ Straight};
\node at (axis cs:10.5000001192093,3.5)[
  scale=0.75,
  text=black,
  rotate=0.0,
  align=center
]{ Lane\\
Change};
\node at (axis cs:14.7600001096725,3.5)[
  scale=0.75,
  text=black,
  rotate=0.0
]{ Straight};
\node at (axis cs:17.4800001382828,4.5)[
  scale=0.75,
  text=black,
  rotate=0.0
]{ Cruising};
\node at (axis cs:17.4800001382828,5.5)[
  scale=0.75,
  text=black,
  rotate=0.0
]{ Straight};
\node at (axis cs:-3.5,0.5)[
  scale=0.75,
  text=black,
  rotate=0.0,
  align=center
]{ Ego vehicle,\\
Longitudinal state};
\node at (axis cs:-3.5,1.5)[
  scale=0.75,
  text=black,
  rotate=0.0,
  align=center
]{ Ego vehicle,\\
Lateral state};
\node at (axis cs:-3.5,2.5)[
  scale=0.75,
  text=black,
  rotate=0.0,
  align=center
]{ Target 1,\\
Longitudinal state};
\node at (axis cs:-3.5,3.5)[
  scale=0.75,
  text=black,
  rotate=0.0,
  align=center
]{ Target 1,\\
Lateral state};
\node at (axis cs:-3.5,4.5)[
  scale=0.75,
  text=black,
  rotate=0.0,
  align=center
]{ Target 2,\\
Longitudinal state};
\node at (axis cs:-3.5,5.5)[
  scale=0.75,
  text=black,
  rotate=0.0,
  align=center
]{ Target 2,\\
Lateral state};
\end{axis}

\end{tikzpicture}
%	\caption{Complete overview of the events and activities of the example. Every vertical black line represents an event. The end of the first scenario and the start of the second scenario is at the moment the ego vehicle starts the first lane change, i.e., at $t=17\ \textup{s}$.}
%	\label{fig:example events}
%\end{figure*}

\section{Conclusions}
\label{sec:conclusions}

% Safety is important
Road safety is an important research topic because of the societal and economical losses caused by accidents.
% Quantify safety with surrogate metrics
To quantify the safety at a vehicle level, use is made of \acp{ssm} that characterize the risk of a collision. 
% What we did
We have proposed a method for deriving \iac{ssm} that calculates the probability that a certain event, e.g., a collision, will happen in the near future.
% Advantages of our method
With our data-driven approach, it is possible to adapt the \ac{ssm} to the local traffic context.
Besides, the presented method could be applied for various types of scenarios.
We have illustrated that our method is a generalization of already existing \acp{ssm}.
In an example, we have derived \iac{ssm} based on the \ac{ngsim} data set.
Through few explanatory scenarios, it has been shown that the derived \ac{ssm} provides a quantification of the collision risk.
We have also presented how the evaluation of the partial derivatives of the \ac{ssm} can be used to benchmark \iac{ssm} with few expected causal tendencies.

%Concluding remarks about our method
Our proposed method has the potential for deriving multiple \acp{ssm} for quantifying the safety of a --- possibly automated --- driver.
These metrics can be used to warn drivers for unsafe situations and ensuring that proper attention is being paid to the road situation.
Furthermore, the metrics can measure the impact of newly introduced systems on the traffic safety.
% Future work
A limitation of the current study is that the presented approach is only applied to longitudinal traffic interactions. 
Future work involves the consideration of lateral traffic interactions and interactions with vulnerable road users. 


%\section*{Acknowledgement}
%Tux also likes to thank the Free Software Foundation for their GNU software.



\printbibliography

\end{document}
