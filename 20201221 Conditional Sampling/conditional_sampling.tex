\documentclass{article}

\usepackage[utf8]{inputenc}
\usepackage{graphicx}
\usepackage{amsmath} % assumes amsmath package installed
\usepackage{amsthm}  % For special theorem style
\usepackage{amsfonts}
\usepackage{breqn}
\usepackage{dsfont}
\usepackage[dvipsnames]{xcolor}
\usepackage{tikz}
\usepackage[nolist,nohyperlinks]{acronym}
\usepackage[linesnumbered,ruled,vlined]{algorithm2e}
\usepackage[capitalize]{cleveref}
\Crefname{figure}{Figure}{Figures}
\usepackage{siunitx}

\usepackage[utf8]{inputenc}   				 	%% utf8 support (required for biblatex)
\usepackage{silence}  							%% For filtering warnings
\usepackage[style=ieee,doi=false,isbn=false,url=false,date=year,backend=biber,maxbibnames=15,maxcitenames=2,mincitenames=1,uniquelist=false,uniquename=false,giveninits=true]{biblatex}
% Filter warnings issued by package biblatex starting with "Patching footnotes failed"
\WarningFilter{biblatex}{Patching footnotes failed}
\renewcommand*{\bibfont}{\footnotesize}		%% Use this for papers
\renewcommand*{\bibfont}{\small}
\setlength{\biblabelsep}{\labelsep}
\bibliography{../bib}

\title{Conditional Sampling from a Kernel Density Estimator with a Gaussian Kernel}
\author{Erwin de Gelder}
\date{}

% Variables
\newcommand{\bandwidth}{h}
\newcommand{\bandwidthmatrix}{H}
  \newcommand{\bandwidthmatrixinverse}{\Lambda}
  \newcommand{\bwiul}{\bandwidthmatrixinverse_{11}}
  \newcommand{\bwiur}{\bandwidthmatrixinverse_{12}}
  \newcommand{\bwibl}{\bandwidthmatrixinverse_{21}}
  \newcommand{\bwibr}{\bandwidthmatrixinverse_{22}}
  \newcommand{\bandwidthmatrixrotated}{\tilde{\bandwidthmatrix}}
  \newcommand{\bandwidthmatrrotatedixinverse}{\tilde{\Lambda}}
  \newcommand{\bwriul}{\bandwidthmatrrotatedixinverse_{11}}
  \newcommand{\bwriur}{\bandwidthmatrrotatedixinverse_{12}}
  \newcommand{\bwribl}{\bandwidthmatrrotatedixinverse_{21}}
  \newcommand{\bwribr}{\bandwidthmatrrotatedixinverse_{22}}
\newcommand{\constraintmatrix}{A}
\newcommand{\constraintvector}{b}
\newcommand{\density}[1]{f\left( #1 \right)}
\newcommand{\densityest}[1]{\hat{f}\left( #1 \right)}
\newcommand{\densitycond}[2]{\density{ #1 | #2 }}
\newcommand{\densityestcond}[2]{\densityest{ #1 | #2 }}
\newcommand{\determinant}[1]{\left| #1 \right|}
\newcommand{\dimension}{d}
  \newcommand{\dimensionparta}{\dimension_\mathrm{c}}
  \newcommand{\dimensionpartb}{\dimension_\mathrm{u}}
\newcommand{\dummyvar}{u}
\newcommand{\e}[1]{\exp\left\{ #1 \right\}}
\newcommand{\identitymatrix}[1]{I_{#1}}
\newcommand{\indexdata}{i}
\newcommand{\indexsampling}{j}
\newcommand{\kernelfunc}[1]{K \left( #1 \right)}
\newcommand{\kernelfuncnormalized}[2]{K_{#1} \left( #2 \right)}
\newcommand{\normtwo}[1]{\left\Vert #1 \right\Vert}
\newcommand{\numberofconstraints}{n_\mathrm{c}}
\newcommand{\numberofsamples}{N}
\newcommand{\realnumbers}{\mathds{R}}
\newcommand{\superquad}{\phantom{=}\quad\quad\quad\quad}
\newcommand{\svdu}{U}
  \newcommand{\svds}{\Sigma}
  \newcommand{\svdv}{V}
  \newcommand{\svdva}{V_1}
  \newcommand{\svdvb}{V_2}
\newcommand{\ud}{\,\mathrm{d}}
\newcommand{\variable}{x}
  \newcommand{\variableparta}{\breve{\variable}}
  \newcommand{\variablepartb}{\hat{\variable}}
  \newcommand{\variableconstrained}{\bar{\variable}}
  \newcommand{\variableunconstrained}{\tilde{\variable}}
  \newcommand{\datapoint}[1]{\variable_{#1}}
  \newcommand{\datapointparta}[1]{\variableparta_{#1}}
  \newcommand{\datapointpartb}[1]{\variablepartb_{#1}}
  \newcommand{\datapointpartbtranslated}[1]{\datapointpartb{#1}'}
  \newcommand{\datapointconstrained}[1]{\variableconstrained_{#1}}
  \newcommand{\datapointunconstrained}[1]{\variableunconstrained_{#1}}
  \newcommand{\datapointunconstrainedtranslated}[1]{\datapointunconstrained{#1}'}
\newcommand{\weight}[1]{w_{#1}}

\begin{document}

% Acronyms
\begin{acronym}[AAAAAAAA]
	\acro{kde}[KDE]{Kernel Density Estimation}
	\acro{pdf}[pdf]{probability density function}
	\acro{svd}[SVD]{Singular Value Decomposition}\acroindefinite{svd}{an}{a}
\end{acronym}

\maketitle

\begin{abstract}

% Introduce surrogate safety metrics
\acp{ssm} are used to express road safety in terms of the safety risk in traffic conflicts.
As opposed to historical crash data, which takes a long time to obtain, \acp{ssm} can be used to determine  the risk of a collision or harm of a certain vehicle.
% What is lacking?
\cstarta Typically, \acp{ssm} rely on assumptions on the future evolution of traffic participants to generate a measure of risk. 
As a result, \acp{ssm} are only applicable in certain types of scenarios. 

% Our approach
We present a novel data-driven approach called \ac{psmda}. 
The \ac{psmda} is used to derive \acp{ssm} that provide a probability that a certain specified event, such as a collision, happens in the near future. 
These derived \acp{ssm} are not limited to certain types of scenarios.
Furthermore, because we adopt a data-driven approach to predict the possible future evolutions of traffic participants, less assumptions are needed.
The \ac{psmda} uses Monte Carlo simulations, such that the derived \acp{ssm} can accurately estimate the probability of the specified event.
We further introduce a statistical method that requires fewer simulations to estimate this probability. 
Combined with a regression model, this enables our derived \acp{ssm} to make real-time risk estimation.

% Results
Results show that the \ac{psmda} is a generalization of existing probabilistic \acp{ssm}.
To further illustrate the approach, \iac{ssm} is derived for risk evaluation during longitudinal traffic conflicts. 
Since the ground truth for the safety risk is generally lacking, it is very difficult, if not impossible, to argue that one \ac{ssm} it better than another \ac{ssm}.
As an alternative, we provide a method for benchmarking the derived \ac{ssm} with respect to expected risk tendencies.
The application of the benchmarking illustrates that the derived \ac{ssm} matches the expected risk tendencies. \cenda

% Conclusions
Whereas the derived \ac{ssm} shows the potential of the \cstarta\ac{psmda}\cenda, future work involves applying the approach for other types of traffic conflicts, such as lateral traffic conflicts or interactions with vulnerable road users.

\end{abstract}

\section{Introduction}
\label{sec:introduction}

Bladibla. 
This is still the same paragraph.

This is a new paragraph. Reference to equation: \ref{eq:emc2}.

\begin{equation} \label{eq:emc2}
    E = m c^2
\end{equation}


\section{Problem definitions}
\label{sec:problem} % Arash

This Section presents: 
\begin{itemize}
	\item Why risk assessment matters?
	\item What is currently the risk estimation method? 
	\item What is lacking in this approach? 
	\item What need to be done more? 
\end{itemize}

In this section we position our research in the automotive safety engineering domain. 
We present the current risk assessment methods and discuss their limitation and the impact of these limits.
The argue that advancement in the risk assessment methods are required for achieving higher levels of automation in the automotive domain. 

\subsection{Why risk assessment?}

Safety means avoiding risk. 
The risk associated with driving may come from multiple sources. 
It could be a traffic situation in which a sequence of uncorrelated actions performed by different actors lead to an accident. 
The risk could also be due to technical failure originating from a system fault. 
%This type of faults and failures should be avoided by the manufacturers of the automotive systems (OEMs and Tiers), 
%	while the later is a responsibility of the traffic participants. 
For the former, the manufacturers are deemed responsible; 
	therefore, they put a lot of effort on the quality  assurance of their products 
		and for understanding and mitigating technical safety issues.  
The later comes into the public domain and road authorities are responsible for minimizing the risk by good design of roads and traffic rules. 
The  risk, however, cannot be avoided fully. 
There is always a certain amount of \textit{residual risk} remaining after taking risk avoidance/mitigation measures. 

Understanding the risk and measuring it is crucial in both directing the effort on avoiding or mitigating the impacts. 
Moreover, formulating and opinion about when using a system is ``safe enough'' depends on the ability to measure the risk. 
	

\subsection{Risk assessment in ISO~26262}

Risk is defined in ISO~26262 as:
\begin{definition}
	``combination of the probability of occurrence of harm and the severity of that harm'' 
\end{definition}



Risk assessment is an integral part of the safety life-cycle of ISO~26262 and one of the earlier activities. 
During risk assessment, the identified hazardous events are analyzed and the associated severity, probability of exposure and controllability are estimated and assigned to some predefined levels. 
The combination of the estimated severity, exposure and controllability contribute to construct the Automotive Safety Integrity Level (ASIL). 


This is the domain of functional safety. 
The goal of functional safety is to avoid \emph{unreasonable} risk\footnote{Unreasonable risk is judged according to the the society's acceptance of level of risk.} that is due to some sort of malfunctioning.


\section{Method}
\label{sec:method}

In this section, we propose a method for deriving a metric that quantifies the risk of a collision in a particular situation in which a vehicle - hereafter, the \textit{ego vehicle} - is in and that is applicable for real-time use.
Our method for calculating this \ac{ssm} for quantifying the risk consists of four steps. 
The first step is the parameterization of the current situation and the possible future situation.
Second, based on the current situation, we estimate the probability (density) for the possible future situations. 
The third step is to determine the probability of a collision based on the current and the future situations.
Finally, local regression is used to speed up the calculations and to make it possible to use the \ac{ssm} in real time. 
These four steps are described in the following sections.

In the remainder of this article, the following notation is used. 
To denote a probability, $\probability{\cdot}$ is used. 
\Iac{pdf} is denoted by $\density{\cdot}$. 
The conditional probability $\probabilitycond{\dummyvara}{\dummyvarb}$ is verbalized as \textit{the probability of $\dummyvara$ given $\dummyvarb$}. 
Similarly, the conditional \ac{pdf} is denoted by $\densitycond{\cdot}{\cdot}$. 
To denote the estimation of any of the aforementioned quantities, a circumflex is used. 
E.g, $\probabilityest{\dummyvara}$ denotes the estimated probability of $\dummyvara$.



\subsection{Parameterize current and future situations}
\label{sec:parametrization}

The first step is to parameterize the current situation in which the ego vehicle is in. 
In other words, the current situation needs to be described using $\situationcurrentdim$ numbers that are stacked into one vector $\situationcurrent \in \situationcurrentspace \subseteq \realnumbers^{\situationcurrentdim}$. 
As an example, $\situationcurrent$ could contain the speed of the ego vehicle and distance towards its preceding vehicle. 
In \cref{sec:case study}, we will see more examples.

Next to describing the current situation, the future situation is described using $\situationfuturedim$ numbers stacked into one vector $\situationfuture \in \situationfuturespace \subseteq \realnumbers^{\situationfuturedim}$. 
Together with $\situationcurrent$, $\situationfuture$ contains enough information to describe how the relevant future, e.g., the next 5 seconds, around the ego vehicle develops over time. 
As an example, $\situationfuture$ could contain the speed for the next 5 seconds of the lead vehicle (if any) which is in front of the ego vehicle.

Let $\collision$ denote a collision, such that the probability of a collision is $\probability{\collision}$.
The goal of our \ac{ssm} is to estimate the probability of a collision given a particular situation $\situationcurrent$, i.e., $\probabilitycond{\collision}{\situationcurrent}$.
We do this by considering all future situations, $\situationfuturespace$, and calculating the probability of a collision given each possible value of $\situationfuture$. 
Using integration, we obtain $\probabilitycond{\collision}{\situationcurrent}$:
\begin{equation}
	\label{eq:probability collision expectation}
	\probabilitycond{\collision}{\situationcurrent} 
	= \int_{\situationfuturespace} 
	\probabilitycond{\collision}{\situationcurrent, \situationfuture} 
	\densitycond{\situationfuture}{\situationcurrent} 
	\ud \situationfuture.
\end{equation}
In \cref{sec:estimate future}, we propose a method to estimate $\densitycond{\situationfuture}{\situationcurrent}$ and in \cref{sec:estimate collision}, we propose a method to estimate $\probabilitycond{\collision}{\situationcurrent, \situationfuture}$.



\subsection{Estimate $\densitycond{\situationfuture}{\situationcurrent}$}
\label{sec:estimate future}

In this section, we propose a method to estimate $\densitycond{\situationfuture}{\situationcurrent}$, i.e., the \ac{pdf} of the $\situationfuture$ given $\situationcurrent$.
Using the product rule for probability, we can write:
\begin{equation}
	\densitycond{\situationfuture}{\situationcurrent} 
	= \frac{\density{\situationcurrent, \situationfuture}}{\density{\situationcurrent}}
	= \frac{\density{\situationcurrent, \situationfuture}}{
		\int_{\situationfuturespace} \density{\situationcurrent, \situationfuture} \ud\situationfuture
	}
\end{equation}
Thus, it suffices to estimate $\density{\situationcurrent, \situationfuture}$. 

Our proposal is to estimate $\density{\situationcurrent, \situationfuture}$ in a data-driven manner. 
A data-driven approach brings several benefits.
First, the estimate automatically adapts to local driving styles and behaviors, which can change from region to region, provided that the data are obtained from the same local traffic.
Second, strong assumptions such as assuming a constant speed of other vehicles, are not needed.
For our data-driven approach, let us assume that we have obtained $\situationnumberof$ situations, denoted by $\situationcurrentinstance{\situationindex}\in\situationcurrentspace, \situationindex\in\{1,\ldots,\situationnumberof\}$, and their corresponding future situations described by $\situationfutureinstance{\situationindex}\in\situationfuturespace$.



\subsubsection{Special case: all parameters from one distribution}
\label{sec:one kde}

We first explain how to estimate $\density{\situationcurrent, \situationfuture}$ if we assume that all parameters depend on each other and that no further simplifications are possible. 
The shape of the \ac{pdf} $\density{\situationcurrent, \situationfuture}$ is unknown beforehand. 
Furthermore, the shape of the estimated \ac{pdf} might change as more data are acquired. 
Assuming a functional form of the \ac{pdf} and fitting the parameters of the \ac{pdf} to the data may therefore lead to inaccurate fits unless a lot of hand-tuning is applied.
We employ a non-parametric approach using \ac{kde} \autocite{rosenblatt1956remarks, parzen1962estimation} because the shape of the \ac{pdf} is automatically computed and \ac{kde} is highly flexible regarding the shape of the \ac{pdf}. 
Using the \ac{kde}, the estimated \ac{pdf} becomes:
\begin{equation}
	\label{eq:kde estimate}
	\densityest{\situationcurrent,\situationfuture}
	= \frac{1}{\situationnumberof} \sum_{\situationindex=1}^{\situationnumberof}
	\kernelfuncnormalized{\bandwidthmatrix}{
		\begin{bmatrix}
			\situationcurrent \\
			\situationfuture
		\end{bmatrix} -
		\begin{bmatrix}
			\situationcurrentinstance{\situationindex} \\
			\situationfutureinstance{\situationindex}
		\end{bmatrix}
	},
\end{equation}
where $\kernelfuncnormalized{\bandwidthmatrix}{\cdot}$ is an appropriate kernel function with positive definite bandwidth matrix $\bandwidthmatrix$. 
The choice of the kernel $\kernelfuncnormalized{\bandwidthmatrix}{\cdot}$ is not as important as the choice of the bandwidth matrix $\bandwidthmatrix$ \cite{turlach1993bandwidthselection}.
We use a Gaussian kernel, but our method also applies with other kernels.
The Gaussian kernel is given by
\begin{equation}
	\label{eq:kernel current future}
	\kernelfuncnormalized{\bandwidthmatrix}{\dummyvarkernel}
	= \frac{1}{\left( 2 \pi \right)^{\left( \situationcurrentdim + \situationfuturedim \right) / 2} 
	\left|\bandwidthmatrix\right|^{1/2} }
	\e{ -\frac{1}{2} \dummyvarkernel\transpose \bandwidthmatrix^{-1} \dummyvarkernel }.
\end{equation}

The bandwidth matrix controls the width of the Gaussian kernel, or, in other words, the influence of each sampling point on nearby regions. 
There are many different ways of estimating the bandwidth, ranging from simple reference rules like, e.g., Silverman's rule of thumb \autocite{silverman1986density} to more elaborate methods; see \autocite{turlach1993bandwidthselection, chiu1996comparative, jones1996brief, bashtannyk2001bandwidth, zambom2013review} for reviews of different bandwidth selection methods.
Here, we use one-leave-out cross-validation to compute the bandwidth because this minimizes the Kullback-Leibler divergence between the real \ac{pdf} $\density{\situationcurrent, \situationfuture}$ and the estimated \ac{pdf} $\densityest{\situationcurrent, \situationfuture}$ \autocite{turlach1993bandwidthselection, zambom2013review}.

Drawing samples from the \ac{kde} in \cref{eq:kde estimate} is straightforward: two random numbers are drawn, one to choose a random generator Gaussian out of the $\situationnumberof$ Gaussians that are used to construct the \ac{kde}, and one random number from that Gaussian.
Sampling from $\densityestcond{\situationfuture}{\situationcurrent}$ works similarly, but instead of using an equal probability for each random generator Gaussian to be selected, different probabilities are used based on $\situationcurrent$.
For more information on sampling from a conditional \ac{pdf} obtained using \ac{kde}, see \autocite{holmes2012fast}.



\subsubsection{Not all parameters from one distribution}
\label{sec:no special case}

Estimating $\density{\situationcurrent, \situationfuture}$ with one \ac{kde} according to \cref{eq:kde estimate} becomes problematic if $\situationcurrentdim + \situationfuturedim$ becomes large, due to the curse of dimensionality \cite{scott2015multivariate}.
There are a few ways to avoid this curse of dimensionality.
Without going into much detail, we list a few options.

One option is to assume that one or more parameter are independent of the other parameters. 
E.g., suppose that $\situationfuture=\begin{bmatrix}\situationfutureparta & \situationfuturepartb\end{bmatrix}\transpose$, such that $\situationfuturepartb$ is independent of $\situationcurrent$ and $\situationfutureparta$.
Then we can write
\begin{equation}
	\density{\situationcurrent, \situationfuture}
	= \density{\situationcurrent, \situationfutureparta, \situationfuturepartb}
	= \density{\situationcurrent, \situationfutureparta} \cdot \density{\situationfuturepartb}.
\end{equation}
In this case, we would need to estimation $\density{\situationcurrent, \situationfutureparta}$ and $\density{\situationfuturepartb}$, which can be done in a similar manner as presented in \cref{sec:one kde}.
Because these two \acp{pdf} have less variables than $\density{\situationcurrent, \situationfuture}$, it will suffer less from the curse of dimensionality \cite{scott2015multivariate}.

Another option is to model $\densitycond{\situationfuture}{\situationcurrent}$ as a cascade of conditional probabilities. E.g., using the partitioning $\situationfuture=\begin{bmatrix}\situationfutureparta & \situationfuturepartb\end{bmatrix}\transpose$, $\densitycond{\situationcurrent}{\situationfuture}$ can be approximated using two conditional densities:
\begin{equation}
	\densitycond{\situationfuture}{\situationcurrent}
	= \densitycond{\situationfutureparta, \situationfuturepartb}{\situationcurrent}
	= \densitycond{\situationfutureparta}{\situationfuturepartb, \situationcurrent} \cdot \densitycond{\situationfuturepartb}{\situationcurrent}
	\approx \densitycond{\situationfutureparta}{\situationfuturepartb} \cdot \densitycond{\situationfuturepartb}{\situationcurrent}.
\end{equation}
The same partitioning can be applied to $\densitycond{\situationfutureparta}{\situationfuturepartb}$ and $\densitycond{\situationfuturepartb}{\situationcurrent}$ until only two-dimensional \acp{pdf} need to be estimated.
For more information on this approach, we refer the reader to \autocite{aas2009paircopula, nagler2016evading}.



\subsubsection{Reduce number of parameters using \acl{svd}}
\label{sec:parameter reduction}

One way to avoid the curse of dimensionality is to use \iac{svd} \autocite{golub2013matrix} to reduce the number of parameters.
With \iac{svd}, the parameters $\situationcurrent$ and $\situationfuture$ are transformed into a lower-dimensional vector of parameters in such a way that the reduced vector of parameters describes as much of the variation as possible.
To do this, \iac{svd} is made of the matrix that contains all $\situationnumberof$ observed situations:
\begin{equation}
	\begin{bmatrix}
		\situationcurrentinstance{1}-\situationcurrentmean & \ldots & \situationcurrentinstance{\situationnumberof}-\situationcurrentmean \\
		\situationfutureinstance{1}-\situationfuturemean & \ldots & \situationfutureinstance{\situationnumberof}-\situationfuturemean
	\end{bmatrix} = \svdu \svds \svdv\transpose.
\end{equation}
Here, $\situationcurrentmean=\sum_{\situationindex=1}^{\situationnumberof}\situationcurrentinstance{\situationindex}$ and $\situationfuturemean=\sum_{\situationindex=1}^{\situationnumberof}\situationfutureinstance{\situationindex}$.
The matrices $\svdu \in \realnumbers^{\left(\situationcurrentdim+\situationfuturedim\right)\times\left(\situationcurrentdim+\situationfuturedim\right)}$ and $\svdv \in \realnumbers^{\situationnumberof \times \situationnumberof}$ are orthonormal, i.e., $\svdu^{-1}=\svdu\transpose$ and $\svdv^{-1}=\svdv\transpose$.
Moreover, $\svds\in\realnumbers^{\left(\situationcurrentdim+\situationfuturedim\right)\times\situationnumberof}$ has only zeros except at the diagonal: the $(\svdindex,\svdindex)$-th element is $\svdsv{\svdindex}$, $\svdindex\in\{1,\ldots,\svdrank\}$, $\svdrank=\min(\situationcurrentdim+\situationfuturedim, \situationnumberof)$, such that
\begin{equation}
	\svdsv{1} \geq \svdsv{2} \geq \ldots \geq \svdsv{\svdrank} \geq 0.
\end{equation}
Because these so-called singular values are in decreasing order, we can approximate $\situationcurrent$ and $\situationfuture$ by setting $\svdsv{\svdindex}=0$ for $\svdindex > \dimension$:
\begin{equation}
	\label{eq:svd approximation}
	\begin{bmatrix}
		\situationcurrentinstance{\situationindex} - \situationcurrentmean \\
		\situationfutureinstance{\situationindex} - \situationfuturemean
	\end{bmatrix}
	= \sum_{\svdindex=1}^{\svdrank} \svdsv{\svdindex} \svdventry{\situationindex}{\svdindex} \svduvec{\svdindex}
	\approx \sum_{\svdindex=1}^{\dimension} \svdsv{\svdindex} \svdventry{\situationindex}{\svdindex} \svduvec{\svdindex},
\end{equation}
where $\svdventry{\situationindex}{\svdindex}$ is the $(\situationindex,\svdindex)$-th element of $\svdv$ and $\svduvec{\svdindex}$ is the $\svdindex$-th column of $\svdu$.

Using the approximation of \cref{eq:svd approximation} based on the \ac{svd}, $\situationcurrentinstance{\situationindex}$ and $\situationfutureinstance{\situationindex}$ are described by the vector $\svdvvecd{\situationindex}$:
\begin{equation}
	\label{eq:reduced parameter vector}
	\svdvvecd{\situationindex}\transpose = \begin{bmatrix}
		\svdventry{\situationindex}{1} & \ldots & \svdventry{\situationindex}{\dimension}
	\end{bmatrix}.
\end{equation}
We can estimate the probability density of $\svdvvecd{\situationindex}$ in a similar way as described in \cref{sec:one kde}.
To sample from $\densityestcond{\situationfuture}{\situationcurrent}$, we can sample from the estimated distribution of $\svdvvecd{\situationindex}$.
Because \cref{eq:svd approximation} is a linear mapping, the sample that is drawn from the estimated distribution of $\svdvvecd{\situationindex}$ is subject to a linear constraint.
In \autocite{degelder2021conditional}, algorithms are provided for sampling from \iac{kde} with a Gaussian kernel such that the resulting samples are subject to a linear constraint.



\subsection{Estimate $\probabilitycond{\collision}{\situationcurrent}$}
\label{sec:estimate collision}

To estimate $\probabilitycond{\collision}{\situationcurrent}$, i.e., the probability of a collision given the current situation $\situationcurrent$, we employ simulations. 
The details of the simulation depends on the actual application. 
For example, if the goal of our \ac{ssm} is to evaluate the risk that a human driven vehicle collides, the simulation should involve human driver behavior models. 
On the other hand, if the goal is to evaluate the risk of collision when \iac{ads} is controlling the vehicle, the simulation should include the model of this \ac{ads}.

A straightforward way to compute $\probabilitycond{\collision}{\situationcurrent}$ is to repeat a certain number of simulations with the same $\situationcurrent$ and count the number of simulations that result in a collision.
If $\numberofsimulations$ denotes the number of simulations and $\numberofcollisions$ is the number of collisions, then $\probabilitycond{\collision}{\situationcurrent}$ could be estimated using
\begin{equation}
	\label{eq:binomial estimation}
	\probabilityestcond{\collision}{\situationcurrent}
	= \frac{\numberofcollisions}{\numberofsimulations}.
\end{equation}

An important choice for estimating $\probabilitycond{\collision}{\situationcurrent}$ is the number of simulations, $\numberofsimulations$.
One approach is to keep increasing $\numberofsimulations$ until there is enough confidence in the estimation of \cref{eq:binomial estimation}.
E.g., the Clopper-Pearson interval \autocite{clopper1934use} or the Wilson score interval \autocite{wilson1927probable} can be used to determine the confidence of the estimation of \cref{eq:binomial estimation}.
A disadvantage of this approach is that only the fact that a simulation results in a collision is used while the simulation provides more information. 
Therefore, we provide an alternative approach to estimate $\probabilitycond{\collision}{\situationcurrent}$.

Let $\simulationresult \in \realnumbers^{\dimsimulationresult}$ be a continuous variable representing the result of a simulation, such that $\simulationresult \in \spacecollision$ if and only if the simulation results in a collision.
Here, $\spacecollision \subset \realnumbers^{\dimsimulationresult}$ represents the set of possible simulation results in which a collision occurs. 
Therefore, we have
\begin{equation}
	\probabilitycond{\collision}{\situationcurrent}
	= \probabilitycond{\simulationresult \in \spacecollision}{\situationcurrent}
	= \int_{\spacecollision} \densitycond{\simulationresult}{\situationcurrent} \ud \simulationresult.
\end{equation}
Similar as with the estimation of $\density{\situationcurrent, \situationfuture}$ in \cref{sec:estimate future}, we employ \ac{kde} to estimate $\densitycond{\simulationresult}{\situationcurrent}$:
\begin{equation}
	\label{eq:kde simulation result}
	\densityestcond{\simulationresult}{\situationcurrent}
	= \frac{1}{\numberofsimulations} 
	\sum_{\simulationindex=1}^{\numberofsimulations} \kernelfuncnormalized{\simulationbandwidth}{\simulationinstance{\simulationindex} - \simulationresult},
\end{equation}
where $\simulationinstance{\simulationindex}$ denotes the result of the $\simulationindex$-th simulation and $\simulationbandwidth$ denotes an appropriate bandwidth matrix.
The kernel function $\kernelfuncnormalized{\simulationbandwidth}{\cdot}$ is similarly defined as \cref{eq:kernel current future}.
We can now estimate $\probabilitycond{\collision}{\situationcurrent}$ by substituting $\densityestcond{\simulationresult}{\situationcurrent}$ of \cref{eq:kde simulation result} for $\densitycond{\simulationresult}{\situationcurrent}$:
\begin{equation}
	\label{eq:estimate probability of collision}
	\probabilityestcond{\collision}{\situationcurrent}
	= \probabilityestcond{\simulationresult \in \spacecollision}{\situationcurrent}
	= \int_{\spacecollision} \densityestcond{\simulationresult}{\situationcurrent} \ud \simulationresult
	=\frac{1}{\numberofsimulations}
	\sum_{\simulationindex=1}^{\numberofsimulations} \int_{\spacecollision}
	\kernelfuncnormalized{\simulationbandwidth}{\simulationinstance{\simulationindex} - \simulationresult} \ud \simulationresult.
\end{equation}

We are still left with choosing the number of simulations $\numberofsimulations$ in \cref{eq:kde simulation result}.
Our proposal is keep increasing $\numberofsimulations$ until the variance of $\probabilityestcond{\simulationresult \in \spacecollision}{\situationcurrent}$ is below a threshold $\simulationthreshold$.
The variance follows from \textcite{nadaraya1964some}:
\begin{equation}
	\variance{\probabilityestcond{\simulationresult \in \spacecollision}{\situationcurrent}}
	= \frac{\probabilitycond{\simulationresult \in \spacecollision}{\situationcurrent}
		\left( 1-\probabilitycond{\simulationresult \in \spacecollision}{\situationcurrent} \right)}{\numberofsimulations}.
\end{equation}
Because $\probabilitycond{\simulationresult \in \spacecollision}{\situationcurrent}$ is unknown, we use the estimated counterpart of \cref{eq:estimate probability of collision}.
Thus, $\numberofsimulations$ is increased until the following condition is met:
\begin{equation}
	\label{eq:condition stop simulations}
	\frac{\probabilityestcond{\simulationresult \in \spacecollision}{\situationcurrent}
		\left( 1-\probabilityestcond{\simulationresult \in \spacecollision}{\situationcurrent} \right)}{\numberofsimulations}
	< \simulationthreshold.
\end{equation}



\subsection{Regression for real-time estimation of $\probabilitycond{\collision}{\situationcurrent}$}
\label{sec:final metric calculation}

Since it is our objective to evaluate the risk metric during real-time operation of the ego vehicle, the expression of \cref{eq:estimate probability of collision} is problematic, even if the calculation is accelerated using a technique like importance sampling.
Therefore, we propose to evaluate \cref{eq:estimate probability of collision} only for some fixed $\situationcurrentinstance{\situationindexdesign}, \situationindexdesign\in\{1,\ldots,\numberofdesignpoints\}$.
Next, local regression is used to estimate \cref{eq:estimate probability of collision}.
More specifically, we use the \ac{nw} kernel estimator \autocite{wasserman2006nonparametric}:
\begin{equation}
	\label{eq:nadaraya watson}
	\probabilityestcond{\collision}{\situationcurrent}
	\approx \frac{ \sum_{\situationindex=1}^{\numberofdesignpoints}
		\kernelfuncnormalized{\bandwidthnw}{\situationcurrent - \situationcurrentinstance{\situationindexdesign}}
		\probabilityestcond{\collision}{\situationcurrentinstance{\situationindexdesign}}
	}{\sum_{\situationindex=1}^{\numberofdesignpoints}
		\kernelfuncnormalized{\bandwidthnw}{\situationcurrent - \situationcurrentinstance{\situationindexdesign}}}.
\end{equation}
Here, $\probabilityestcond{\collision}{\situationcurrentinstance{\situationindexdesign}}$ is based \cref{eq:estimate probability of collision} and $\kernelfuncnormalized{\bandwidthnw}{\cdot}$ represents the Gaussian kernel given by \cref{eq:kernel current future}.
Two important choices have to be made: The choice of the $\situationcurrentinstance{\situationindexdesign}, \situationindexdesign\in\{1,\ldots,\numberofdesignpoints\}$ for which to evaluate \cref{eq:estimate probability of collision} and the choice of the bandwidth matrix $\bandwidthnw$.
We suggest to base the design points $\situationcurrentinstance{\situationindexdesign}, \situationindexdesign\in\{1,\ldots,\numberofdesignpoints\}$ on the data that is used to estimate $\densitycond{\situationfuture}{\situationcurrent}$ in \cref{sec:estimate future}, i.e., $\situationcurrentinstance{\situationindex}, \situationindex \in \{1,\ldots,\situationnumberof\}$, such that all $\situationcurrentinstance{\situationindex}$ have at least one design point $\situationcurrentinstance{\situationindexdesign}$ within nearby::
\begin{equation}
	\label{eq:design points distance}
	\min_{\situationindexdesign} 
	\left( \situationcurrentinstance{\situationindex} - \situationcurrentinstance{\situationindexdesign} \right)\transpose
	\weightmatrix 
	\left( \situationcurrentinstance{\situationindex} - \situationcurrentinstance{\situationindexdesign} \right)
	\leq \distancedesignpoints^2,
	\quad \forall \situationindex \in \{1, \ldots, \situationnumberof\},
\end{equation}
where $\weightmatrix$ denotes a weighting matrix and $\distancedesignpoints$ denotes the maximum ``distance''. 
Note that if $\weightmatrix$ is the identity matrix, then \cref{eq:design points distance} calculates the minimum squared Euclidean distance.
The bandwidth matrix $\bandwidthnw$ might be based on $\weightmatrix$, e.g., $\bandwidthnw=\weightmatrix^{-1}$.
Alternatively, $\bandwidthnw$ might be based on the measurement uncertainty of $\situationcurrent$, where a larger $\bandwidthnw$ applies in case of a larger measurement uncertainty of $\situationcurrent$.



\section{Example}
\label{sec:example}

We will first explain the data that we use and how we use \ac{kde} to estimate the \ac{pdf}.
Next, we apply conditional sampling from the \ac{kde} according to \cref{alg:conditional simple,alg:conditional hard}.
Finally, we apply \cref{alg:constrained simple,alg:constrained hard} to sample from the \ac{kde} while the scenario parameters are subject to linear equality constraints.



\subsection{Data and \acp{kde}}
\label{sec:example data}

\cstarta We work with a data set obtained during an experiment in which 20 participants have driven a predefined route in The Netherlands.
The 50-kilometer route contained a variety of traffic elements, such as urban roads, motorways, junctions (with and without priority), pedestrian crossings, curvy road sections, traffic lights, speed bumps, etc.
The route took about 1 hour.
Because each participant drove the route thrice, the data set contains about 60 hours of driving.
Among other data, the speed of the participants was recorded.
As already done in \autocite{deGelder2017assessment, degelder2019completeness}, the speed was used to extract 2786 braking activities of the vehicle that is equipped with the sensors. \cenda
Each of the braking activities is described by $\dimension=3$ variables:
\begin{enumerate}
	\item the mean deceleration;
	\item the end speed; and
	\item the speed reduction.
\end{enumerate}

\cstarta To illustrate our method, we construct \iac{kde} of these parameters using the $\numberofsamples=2786$ data samples and sample a new set of parameters using the constructed \ac{kde}. \cenda
To apply \cref{alg:conditional simple,alg:constrained simple}, we employ \ac{kde} with the bandwidth matrix $\bandwidthmatrix=\bandwidth^2\identitymatrix{3}$. 
\cstarta The bandwidth $\bandwidth$ is determine using Silverman's rule of thumb \autocite{silverman1986density}, which results in $\bandwidth\approx 0.61$. \cenda

To illustrate \cref{alg:conditional hard,alg:constrained hard}, we use a full bandwidth matrix $\bandwidthmatrix$. 
The bandwidth matrix is estimated using the plug-in selector of \textcite{wand1994multivariate}. 
The resulting bandwidth matrix is:
\begin{equation*}
	\bandwidthmatrix \approx
	\begin{bmatrix}
		0.0058 & -0.032 & 0.021 \\
		-0.032 & 0.92 & -0.23 \\
		0.021 & -0.23 & 0.40
	\end{bmatrix}.
\end{equation*}



\subsection{Sampling with part of $\variable$ fixed}

In this first example, we want to sample the speed difference, while the average deceleration equals $\SI{1}{\meter\per\second\squared}$ and the end speed equals $\SI{10}{\meter\per\second}$. 
Thus, we have two variables fixed ($\dimensionparta=2$) and one variable that is unconstrained ($\dimensionpartb=1$).

In \cref{fig:conditional simple,fig:conditional hard}, the result of \cref{alg:conditional simple,alg:conditional hard} is shown, respectively. 
In total, $10^6$ samples are generated and shown by the histogram (the bin width is \SI{0.2}{\meter\per\second}). 
Because the histograms follow the same pattern as the actual density of the speed difference according to the \ac{kde}, it illustrates that the provided algorithms correctly sample from the \ac{kde}. 

\newcommand{\figurevspace}{\vspace{-1.2cm}}
\newcommand{\captionvspace}{\vspace{-1cm}}
\begin{figure}
	\figurevspace
	\centering
	\resizebox{\columnwidth}{!}{%
		% Created by tikzDevice version 0.12.3 on 2021-02-14 20:34:46
% !TEX encoding = UTF-8 Unicode
\begin{tikzpicture}[x=1pt,y=1pt]
\definecolor{fillColor}{RGB}{255,255,255}
\path[use as bounding box,fill=fillColor,fill opacity=0.00] (0,0) rectangle (303.53,216.81);
\begin{scope}
\path[clip] (  0.00,  0.00) rectangle (303.53,216.81);
\definecolor{drawColor}{RGB}{0,0,0}

\node[text=drawColor,anchor=base,inner sep=0pt, outer sep=0pt, scale=  1.00] at (163.77, 15.60) {Speed difference [m/s]};

\node[text=drawColor,rotate= 90.00,anchor=base,inner sep=0pt, outer sep=0pt, scale=  1.00] at ( 10.80,114.41) {Density};
\end{scope}
\begin{scope}
\path[clip] (  0.00,  0.00) rectangle (303.53,216.81);
\definecolor{drawColor}{RGB}{0,0,0}

\path[draw=drawColor,line width= 0.4pt,line join=round,line cap=round] ( 49.20, 61.20) -- (263.06, 61.20);

\path[draw=drawColor,line width= 0.4pt,line join=round,line cap=round] ( 49.20, 61.20) -- ( 49.20, 55.20);

\path[draw=drawColor,line width= 0.4pt,line join=round,line cap=round] ( 79.75, 61.20) -- ( 79.75, 55.20);

\path[draw=drawColor,line width= 0.4pt,line join=round,line cap=round] (110.30, 61.20) -- (110.30, 55.20);

\path[draw=drawColor,line width= 0.4pt,line join=round,line cap=round] (140.85, 61.20) -- (140.85, 55.20);

\path[draw=drawColor,line width= 0.4pt,line join=round,line cap=round] (171.40, 61.20) -- (171.40, 55.20);

\path[draw=drawColor,line width= 0.4pt,line join=round,line cap=round] (201.96, 61.20) -- (201.96, 55.20);

\path[draw=drawColor,line width= 0.4pt,line join=round,line cap=round] (232.51, 61.20) -- (232.51, 55.20);

\path[draw=drawColor,line width= 0.4pt,line join=round,line cap=round] (263.06, 61.20) -- (263.06, 55.20);

\node[text=drawColor,anchor=base,inner sep=0pt, outer sep=0pt, scale=  1.00] at ( 49.20, 39.60) {0};

\node[text=drawColor,anchor=base,inner sep=0pt, outer sep=0pt, scale=  1.00] at ( 79.75, 39.60) {2};

\node[text=drawColor,anchor=base,inner sep=0pt, outer sep=0pt, scale=  1.00] at (110.30, 39.60) {4};

\node[text=drawColor,anchor=base,inner sep=0pt, outer sep=0pt, scale=  1.00] at (140.85, 39.60) {6};

\node[text=drawColor,anchor=base,inner sep=0pt, outer sep=0pt, scale=  1.00] at (171.40, 39.60) {8};

\node[text=drawColor,anchor=base,inner sep=0pt, outer sep=0pt, scale=  1.00] at (201.96, 39.60) {10};

\node[text=drawColor,anchor=base,inner sep=0pt, outer sep=0pt, scale=  1.00] at (232.51, 39.60) {12};

\node[text=drawColor,anchor=base,inner sep=0pt, outer sep=0pt, scale=  1.00] at (263.06, 39.60) {14};

\path[draw=drawColor,line width= 0.4pt,line join=round,line cap=round] ( 49.20, 61.20) -- ( 49.20,164.97);

\path[draw=drawColor,line width= 0.4pt,line join=round,line cap=round] ( 49.20, 61.20) -- ( 43.20, 61.20);

\path[draw=drawColor,line width= 0.4pt,line join=round,line cap=round] ( 49.20, 76.02) -- ( 43.20, 76.02);

\path[draw=drawColor,line width= 0.4pt,line join=round,line cap=round] ( 49.20, 90.85) -- ( 43.20, 90.85);

\path[draw=drawColor,line width= 0.4pt,line join=round,line cap=round] ( 49.20,105.67) -- ( 43.20,105.67);

\path[draw=drawColor,line width= 0.4pt,line join=round,line cap=round] ( 49.20,120.50) -- ( 43.20,120.50);

\path[draw=drawColor,line width= 0.4pt,line join=round,line cap=round] ( 49.20,135.32) -- ( 43.20,135.32);

\path[draw=drawColor,line width= 0.4pt,line join=round,line cap=round] ( 49.20,150.14) -- ( 43.20,150.14);

\path[draw=drawColor,line width= 0.4pt,line join=round,line cap=round] ( 49.20,164.97) -- ( 43.20,164.97);

\node[text=drawColor,rotate= 90.00,anchor=base,inner sep=0pt, outer sep=0pt, scale=  1.00] at ( 34.80, 61.20) {0.00};

\node[text=drawColor,rotate= 90.00,anchor=base,inner sep=0pt, outer sep=0pt, scale=  1.00] at ( 34.80, 90.85) {0.10};

\node[text=drawColor,rotate= 90.00,anchor=base,inner sep=0pt, outer sep=0pt, scale=  1.00] at ( 34.80,120.50) {0.20};

\node[text=drawColor,rotate= 90.00,anchor=base,inner sep=0pt, outer sep=0pt, scale=  1.00] at ( 34.80,150.14) {0.30};
\end{scope}
\begin{scope}
\path[clip] ( 49.20, 61.20) rectangle (278.33,167.61);
\definecolor{drawColor}{RGB}{0,0,0}

\path[draw=drawColor,line width= 0.4pt,line join=round,line cap=round] ( -2.74, 61.20) rectangle (  0.32, 61.20);

\path[draw=drawColor,line width= 0.4pt,line join=round,line cap=round] (  0.32, 61.20) rectangle (  3.37, 61.20);

\path[draw=drawColor,line width= 0.4pt,line join=round,line cap=round] (  3.37, 61.20) rectangle (  6.43, 61.20);

\path[draw=drawColor,line width= 0.4pt,line join=round,line cap=round] (  6.43, 61.20) rectangle (  9.48, 61.20);

\path[draw=drawColor,line width= 0.4pt,line join=round,line cap=round] (  9.48, 61.20) rectangle ( 12.54, 61.20);

\path[draw=drawColor,line width= 0.4pt,line join=round,line cap=round] ( 12.54, 61.20) rectangle ( 15.59, 61.20);

\path[draw=drawColor,line width= 0.4pt,line join=round,line cap=round] ( 15.59, 61.20) rectangle ( 18.65, 61.20);

\path[draw=drawColor,line width= 0.4pt,line join=round,line cap=round] ( 18.65, 61.20) rectangle ( 21.70, 61.20);

\path[draw=drawColor,line width= 0.4pt,line join=round,line cap=round] ( 21.70, 61.20) rectangle ( 24.76, 61.20);

\path[draw=drawColor,line width= 0.4pt,line join=round,line cap=round] ( 24.76, 61.20) rectangle ( 27.81, 61.20);

\path[draw=drawColor,line width= 0.4pt,line join=round,line cap=round] ( 27.81, 61.20) rectangle ( 30.87, 61.20);

\path[draw=drawColor,line width= 0.4pt,line join=round,line cap=round] ( 30.87, 61.20) rectangle ( 33.92, 61.21);

\path[draw=drawColor,line width= 0.4pt,line join=round,line cap=round] ( 33.92, 61.20) rectangle ( 36.98, 61.21);

\path[draw=drawColor,line width= 0.4pt,line join=round,line cap=round] ( 36.98, 61.20) rectangle ( 40.03, 61.24);

\path[draw=drawColor,line width= 0.4pt,line join=round,line cap=round] ( 40.03, 61.20) rectangle ( 43.09, 61.38);

\path[draw=drawColor,line width= 0.4pt,line join=round,line cap=round] ( 43.09, 61.20) rectangle ( 46.14, 61.67);

\path[draw=drawColor,line width= 0.4pt,line join=round,line cap=round] ( 46.14, 61.20) rectangle ( 49.20, 62.43);

\path[draw=drawColor,line width= 0.4pt,line join=round,line cap=round] ( 49.20, 61.20) rectangle ( 52.26, 64.09);

\path[draw=drawColor,line width= 0.4pt,line join=round,line cap=round] ( 52.26, 61.20) rectangle ( 55.31, 67.48);

\path[draw=drawColor,line width= 0.4pt,line join=round,line cap=round] ( 55.31, 61.20) rectangle ( 58.37, 73.09);

\path[draw=drawColor,line width= 0.4pt,line join=round,line cap=round] ( 58.37, 61.20) rectangle ( 61.42, 82.26);

\path[draw=drawColor,line width= 0.4pt,line join=round,line cap=round] ( 61.42, 61.20) rectangle ( 64.48, 94.60);

\path[draw=drawColor,line width= 0.4pt,line join=round,line cap=round] ( 64.48, 61.20) rectangle ( 67.53,110.85);

\path[draw=drawColor,line width= 0.4pt,line join=round,line cap=round] ( 67.53, 61.20) rectangle ( 70.59,127.75);

\path[draw=drawColor,line width= 0.4pt,line join=round,line cap=round] ( 70.59, 61.20) rectangle ( 73.64,145.03);

\path[draw=drawColor,line width= 0.4pt,line join=round,line cap=round] ( 73.64, 61.20) rectangle ( 76.70,157.74);

\path[draw=drawColor,line width= 0.4pt,line join=round,line cap=round] ( 76.70, 61.20) rectangle ( 79.75,164.95);

\path[draw=drawColor,line width= 0.4pt,line join=round,line cap=round] ( 79.75, 61.20) rectangle ( 82.81,165.88);

\path[draw=drawColor,line width= 0.4pt,line join=round,line cap=round] ( 82.81, 61.20) rectangle ( 85.86,161.28);

\path[draw=drawColor,line width= 0.4pt,line join=round,line cap=round] ( 85.86, 61.20) rectangle ( 88.92,152.68);

\path[draw=drawColor,line width= 0.4pt,line join=round,line cap=round] ( 88.92, 61.20) rectangle ( 91.97,141.69);

\path[draw=drawColor,line width= 0.4pt,line join=round,line cap=round] ( 91.97, 61.20) rectangle ( 95.03,131.39);

\path[draw=drawColor,line width= 0.4pt,line join=round,line cap=round] ( 95.03, 61.20) rectangle ( 98.08,122.29);

\path[draw=drawColor,line width= 0.4pt,line join=round,line cap=round] ( 98.08, 61.20) rectangle (101.14,114.38);

\path[draw=drawColor,line width= 0.4pt,line join=round,line cap=round] (101.14, 61.20) rectangle (104.19,108.69);

\path[draw=drawColor,line width= 0.4pt,line join=round,line cap=round] (104.19, 61.20) rectangle (107.25,104.18);

\path[draw=drawColor,line width= 0.4pt,line join=round,line cap=round] (107.25, 61.20) rectangle (110.30,100.85);

\path[draw=drawColor,line width= 0.4pt,line join=round,line cap=round] (110.30, 61.20) rectangle (113.36, 96.80);

\path[draw=drawColor,line width= 0.4pt,line join=round,line cap=round] (113.36, 61.20) rectangle (116.41, 93.30);

\path[draw=drawColor,line width= 0.4pt,line join=round,line cap=round] (116.41, 61.20) rectangle (119.47, 89.43);

\path[draw=drawColor,line width= 0.4pt,line join=round,line cap=round] (119.47, 61.20) rectangle (122.52, 85.85);

\path[draw=drawColor,line width= 0.4pt,line join=round,line cap=round] (122.52, 61.20) rectangle (125.58, 81.92);

\path[draw=drawColor,line width= 0.4pt,line join=round,line cap=round] (125.58, 61.20) rectangle (128.63, 78.43);

\path[draw=drawColor,line width= 0.4pt,line join=round,line cap=round] (128.63, 61.20) rectangle (131.69, 76.20);

\path[draw=drawColor,line width= 0.4pt,line join=round,line cap=round] (131.69, 61.20) rectangle (134.74, 74.15);

\path[draw=drawColor,line width= 0.4pt,line join=round,line cap=round] (134.74, 61.20) rectangle (137.80, 72.88);

\path[draw=drawColor,line width= 0.4pt,line join=round,line cap=round] (137.80, 61.20) rectangle (140.85, 71.81);

\path[draw=drawColor,line width= 0.4pt,line join=round,line cap=round] (140.85, 61.20) rectangle (143.91, 71.15);

\path[draw=drawColor,line width= 0.4pt,line join=round,line cap=round] (143.91, 61.20) rectangle (146.96, 70.25);

\path[draw=drawColor,line width= 0.4pt,line join=round,line cap=round] (146.96, 61.20) rectangle (150.02, 69.32);

\path[draw=drawColor,line width= 0.4pt,line join=round,line cap=round] (150.02, 61.20) rectangle (153.07, 68.38);

\path[draw=drawColor,line width= 0.4pt,line join=round,line cap=round] (153.07, 61.20) rectangle (156.13, 67.38);

\path[draw=drawColor,line width= 0.4pt,line join=round,line cap=round] (156.13, 61.20) rectangle (159.18, 66.59);

\path[draw=drawColor,line width= 0.4pt,line join=round,line cap=round] (159.18, 61.20) rectangle (162.24, 65.73);

\path[draw=drawColor,line width= 0.4pt,line join=round,line cap=round] (162.24, 61.20) rectangle (165.29, 65.32);

\path[draw=drawColor,line width= 0.4pt,line join=round,line cap=round] (165.29, 61.20) rectangle (168.35, 64.99);

\path[draw=drawColor,line width= 0.4pt,line join=round,line cap=round] (168.35, 61.20) rectangle (171.40, 64.87);

\path[draw=drawColor,line width= 0.4pt,line join=round,line cap=round] (171.40, 61.20) rectangle (174.46, 64.71);

\path[draw=drawColor,line width= 0.4pt,line join=round,line cap=round] (174.46, 61.20) rectangle (177.52, 64.75);

\path[draw=drawColor,line width= 0.4pt,line join=round,line cap=round] (177.52, 61.20) rectangle (180.57, 64.66);

\path[draw=drawColor,line width= 0.4pt,line join=round,line cap=round] (180.57, 61.20) rectangle (183.63, 64.56);

\path[draw=drawColor,line width= 0.4pt,line join=round,line cap=round] (183.63, 61.20) rectangle (186.68, 64.14);

\path[draw=drawColor,line width= 0.4pt,line join=round,line cap=round] (186.68, 61.20) rectangle (189.74, 63.88);

\path[draw=drawColor,line width= 0.4pt,line join=round,line cap=round] (189.74, 61.20) rectangle (192.79, 63.36);

\path[draw=drawColor,line width= 0.4pt,line join=round,line cap=round] (192.79, 61.20) rectangle (195.85, 62.89);

\path[draw=drawColor,line width= 0.4pt,line join=round,line cap=round] (195.85, 61.20) rectangle (198.90, 62.51);

\path[draw=drawColor,line width= 0.4pt,line join=round,line cap=round] (198.90, 61.20) rectangle (201.96, 62.27);

\path[draw=drawColor,line width= 0.4pt,line join=round,line cap=round] (201.96, 61.20) rectangle (205.01, 62.11);

\path[draw=drawColor,line width= 0.4pt,line join=round,line cap=round] (205.01, 61.20) rectangle (208.07, 61.93);

\path[draw=drawColor,line width= 0.4pt,line join=round,line cap=round] (208.07, 61.20) rectangle (211.12, 61.95);

\path[draw=drawColor,line width= 0.4pt,line join=round,line cap=round] (211.12, 61.20) rectangle (214.18, 61.96);

\path[draw=drawColor,line width= 0.4pt,line join=round,line cap=round] (214.18, 61.20) rectangle (217.23, 62.09);

\path[draw=drawColor,line width= 0.4pt,line join=round,line cap=round] (217.23, 61.20) rectangle (220.29, 62.03);

\path[draw=drawColor,line width= 0.4pt,line join=round,line cap=round] (220.29, 61.20) rectangle (223.34, 62.05);

\path[draw=drawColor,line width= 0.4pt,line join=round,line cap=round] (223.34, 61.20) rectangle (226.40, 61.93);

\path[draw=drawColor,line width= 0.4pt,line join=round,line cap=round] (226.40, 61.20) rectangle (229.45, 61.80);

\path[draw=drawColor,line width= 0.4pt,line join=round,line cap=round] (229.45, 61.20) rectangle (232.51, 61.68);

\path[draw=drawColor,line width= 0.4pt,line join=round,line cap=round] (232.51, 61.20) rectangle (235.56, 61.60);

\path[draw=drawColor,line width= 0.4pt,line join=round,line cap=round] (235.56, 61.20) rectangle (238.62, 61.49);

\path[draw=drawColor,line width= 0.4pt,line join=round,line cap=round] (238.62, 61.20) rectangle (241.67, 61.40);

\path[draw=drawColor,line width= 0.4pt,line join=round,line cap=round] (241.67, 61.20) rectangle (244.73, 61.33);

\path[draw=drawColor,line width= 0.4pt,line join=round,line cap=round] (244.73, 61.20) rectangle (247.78, 61.30);

\path[draw=drawColor,line width= 0.4pt,line join=round,line cap=round] (247.78, 61.20) rectangle (250.84, 61.27);

\path[draw=drawColor,line width= 0.4pt,line join=round,line cap=round] (250.84, 61.20) rectangle (253.89, 61.25);

\path[draw=drawColor,line width= 0.4pt,line join=round,line cap=round] (253.89, 61.20) rectangle (256.95, 61.22);

\path[draw=drawColor,line width= 0.4pt,line join=round,line cap=round] (256.95, 61.20) rectangle (260.00, 61.21);

\path[draw=drawColor,line width= 0.4pt,line join=round,line cap=round] (260.00, 61.20) rectangle (263.06, 61.21);

\path[draw=drawColor,line width= 0.4pt,line join=round,line cap=round] (263.06, 61.20) rectangle (266.11, 61.20);

\path[draw=drawColor,line width= 0.4pt,line join=round,line cap=round] (266.11, 61.20) rectangle (269.17, 61.20);

\path[draw=drawColor,line width= 0.4pt,line join=round,line cap=round] (269.17, 61.20) rectangle (272.22, 61.20);

\path[draw=drawColor,line width= 0.4pt,line join=round,line cap=round] (272.22, 61.20) rectangle (275.28, 61.20);

\path[draw=drawColor,line width= 0.4pt,line join=round,line cap=round] (275.28, 61.20) rectangle (278.33, 61.20);

\path[draw=drawColor,line width= 0.4pt,line join=round,line cap=round] (278.33, 61.20) rectangle (281.39, 61.20);

\path[draw=drawColor,line width= 0.4pt,line join=round,line cap=round] (281.39, 61.20) rectangle (284.44, 61.20);

\path[draw=drawColor,line width= 0.4pt,line join=round,line cap=round] (284.44, 61.20) rectangle (287.50, 61.20);

\path[draw=drawColor,line width= 0.4pt,line join=round,line cap=round] (287.50, 61.20) rectangle (290.55, 61.20);

\path[draw=drawColor,line width= 0.4pt,line join=round,line cap=round] (290.55, 61.20) rectangle (293.61, 61.20);

\path[draw=drawColor,line width= 0.4pt,line join=round,line cap=round] (293.61, 61.20) rectangle (296.66, 61.20);

\path[draw=drawColor,line width= 0.4pt,line join=round,line cap=round] (296.66, 61.20) rectangle (299.72, 61.20);

\path[draw=drawColor,line width= 0.4pt,line join=round,line cap=round] (299.72, 61.20) rectangle (302.77, 61.20);

\path[draw=drawColor,line width= 0.4pt,line join=round,line cap=round] (302.77, 61.20) rectangle (305.83, 61.20);
\definecolor{drawColor}{RGB}{255,0,0}

\path[draw=drawColor,line width= 1.2pt,line join=round,line cap=round] (  0.00, 61.20) --
	(  0.32, 61.20) --
	(  1.85, 61.20) --
	(  3.37, 61.20) --
	(  4.90, 61.20) --
	(  6.43, 61.20) --
	(  7.96, 61.20) --
	(  9.48, 61.20) --
	( 11.01, 61.20) --
	( 12.54, 61.20) --
	( 14.07, 61.20) --
	( 15.59, 61.20) --
	( 17.12, 61.20) --
	( 18.65, 61.20) --
	( 20.18, 61.20) --
	( 21.70, 61.20) --
	( 23.23, 61.20) --
	( 24.76, 61.20) --
	( 26.29, 61.20) --
	( 27.81, 61.20) --
	( 29.34, 61.20) --
	( 30.87, 61.20) --
	( 32.40, 61.20) --
	( 33.92, 61.21) --
	( 35.45, 61.21) --
	( 36.98, 61.23) --
	( 38.51, 61.25) --
	( 40.03, 61.29) --
	( 41.56, 61.36) --
	( 43.09, 61.48) --
	( 44.62, 61.67) --
	( 46.14, 61.96) --
	( 47.67, 62.42) --
	( 49.20, 63.09) --
	( 50.73, 64.05) --
	( 52.26, 65.42) --
	( 53.78, 67.28) --
	( 55.31, 69.76) --
	( 56.84, 72.98) --
	( 58.37, 77.02) --
	( 59.89, 81.98) --
	( 61.42, 87.87) --
	( 62.95, 94.66) --
	( 64.48,102.28) --
	( 66.00,110.54) --
	( 67.53,119.22) --
	( 69.06,128.02) --
	( 70.59,136.61) --
	( 72.11,144.64) --
	( 73.64,151.78) --
	( 75.17,157.72) --
	( 76.70,162.25) --
	( 78.22,165.21) --
	( 79.75,166.56) --
	( 81.28,166.32) --
	( 82.81,164.63) --
	( 84.33,161.67) --
	( 85.86,157.68) --
	( 87.39,152.94) --
	( 88.92,147.71) --
	( 90.44,142.26) --
	( 91.97,136.81) --
	( 93.50,131.55) --
	( 95.03,126.62) --
	( 96.55,122.11) --
	( 98.08,118.08) --
	( 99.61,114.53) --
	(101.14,111.42) --
	(102.66,108.72) --
	(104.19,106.35) --
	(105.72,104.23) --
	(107.25,102.30) --
	(108.77,100.47) --
	(110.30, 98.69) --
	(111.83, 96.90) --
	(113.36, 95.08) --
	(114.89, 93.22) --
	(116.41, 91.30) --
	(117.94, 89.35) --
	(119.47, 87.39) --
	(121.00, 85.44) --
	(122.52, 83.55) --
	(124.05, 81.75) --
	(125.58, 80.08) --
	(127.11, 78.55) --
	(128.63, 77.18) --
	(130.16, 76.00) --
	(131.69, 74.99) --
	(133.22, 74.14) --
	(134.74, 73.44) --
	(136.27, 72.86) --
	(137.80, 72.37) --
	(139.33, 71.95) --
	(140.85, 71.56) --
	(142.38, 71.18) --
	(143.91, 70.80) --
	(145.44, 70.38) --
	(146.96, 69.94) --
	(148.49, 69.47) --
	(150.02, 68.97) --
	(151.55, 68.45) --
	(153.07, 67.93) --
	(154.60, 67.42) --
	(156.13, 66.93) --
	(157.66, 66.47) --
	(159.18, 66.07) --
	(160.71, 65.71) --
	(162.24, 65.42) --
	(163.77, 65.20) --
	(165.29, 65.03) --
	(166.82, 64.91) --
	(168.35, 64.85) --
	(169.88, 64.82) --
	(171.40, 64.81) --
	(172.93, 64.81) --
	(174.46, 64.81) --
	(175.99, 64.80) --
	(177.52, 64.77) --
	(179.04, 64.71) --
	(180.57, 64.62) --
	(182.10, 64.50) --
	(183.63, 64.35) --
	(185.15, 64.17) --
	(186.68, 63.97) --
	(188.21, 63.76) --
	(189.74, 63.53) --
	(191.26, 63.30) --
	(192.79, 63.08) --
	(194.32, 62.87) --
	(195.85, 62.68) --
	(197.37, 62.50) --
	(198.90, 62.35) --
	(200.43, 62.23) --
	(201.96, 62.13) --
	(203.48, 62.05) --
	(205.01, 62.01) --
	(206.54, 61.98) --
	(208.07, 61.98) --
	(209.59, 61.99) --
	(211.12, 62.01) --
	(212.65, 62.03) --
	(214.18, 62.05) --
	(215.70, 62.07) --
	(217.23, 62.07) --
	(218.76, 62.07) --
	(220.29, 62.05) --
	(221.81, 62.01) --
	(223.34, 61.97) --
	(224.87, 61.91) --
	(226.40, 61.86) --
	(227.92, 61.79) --
	(229.45, 61.73) --
	(230.98, 61.67) --
	(232.51, 61.62) --
	(234.03, 61.57) --
	(235.56, 61.52) --
	(237.09, 61.48) --
	(238.62, 61.44) --
	(240.14, 61.40) --
	(241.67, 61.37) --
	(243.20, 61.35) --
	(244.73, 61.32) --
	(246.26, 61.30) --
	(247.78, 61.28) --
	(249.31, 61.27) --
	(250.84, 61.25) --
	(252.37, 61.24) --
	(253.89, 61.23) --
	(255.42, 61.22) --
	(256.95, 61.22) --
	(258.48, 61.21) --
	(260.00, 61.21) --
	(261.53, 61.21) --
	(263.06, 61.20) --
	(264.59, 61.20) --
	(266.11, 61.20) --
	(267.64, 61.20) --
	(269.17, 61.20) --
	(270.70, 61.20) --
	(272.22, 61.20) --
	(273.75, 61.20) --
	(275.28, 61.20) --
	(276.81, 61.20) --
	(278.33, 61.20) --
	(279.86, 61.20) --
	(281.39, 61.20) --
	(282.92, 61.20) --
	(284.44, 61.20) --
	(285.97, 61.20) --
	(287.50, 61.20) --
	(289.03, 61.20) --
	(290.55, 61.20) --
	(292.08, 61.20) --
	(293.61, 61.20) --
	(295.14, 61.20) --
	(296.66, 61.20) --
	(298.19, 61.20) --
	(299.72, 61.20) --
	(301.25, 61.20) --
	(302.77, 61.20) --
	(303.53, 61.20);
\end{scope}
\end{tikzpicture}

	}
	\captionvspace
	\caption{The histogram shows the result of the conditional sampling according to \cref{alg:conditional simple}. The red line represents the true \ac{pdf}.}
	\label{fig:conditional simple}
\end{figure}

\begin{figure}
	\figurevspace
	\centering
	\resizebox{\columnwidth}{!}{%
		% Created by tikzDevice version 0.12.3 on 2021-01-24 20:23:36
% !TEX encoding = UTF-8 Unicode
\begin{tikzpicture}[x=1pt,y=1pt]
\definecolor{fillColor}{RGB}{255,255,255}
\path[use as bounding box,fill=fillColor,fill opacity=0.00] (0,0) rectangle (361.35,216.81);
\begin{scope}
\path[clip] (  0.00,  0.00) rectangle (361.35,216.81);
\definecolor{drawColor}{RGB}{0,0,0}

\node[text=drawColor,anchor=base,inner sep=0pt, outer sep=0pt, scale=  1.00] at (192.68, 15.60) {Speed difference [m/s]};

\node[text=drawColor,rotate= 90.00,anchor=base,inner sep=0pt, outer sep=0pt, scale=  1.00] at ( 10.80,114.41) {Density};
\end{scope}
\begin{scope}
\path[clip] (  0.00,  0.00) rectangle (361.35,216.81);
\definecolor{drawColor}{RGB}{0,0,0}

\path[draw=drawColor,line width= 0.4pt,line join=round,line cap=round] ( 49.20, 61.20) -- (317.02, 61.20);

\path[draw=drawColor,line width= 0.4pt,line join=round,line cap=round] ( 49.20, 61.20) -- ( 49.20, 55.20);

\path[draw=drawColor,line width= 0.4pt,line join=round,line cap=round] ( 87.46, 61.20) -- ( 87.46, 55.20);

\path[draw=drawColor,line width= 0.4pt,line join=round,line cap=round] (125.72, 61.20) -- (125.72, 55.20);

\path[draw=drawColor,line width= 0.4pt,line join=round,line cap=round] (163.98, 61.20) -- (163.98, 55.20);

\path[draw=drawColor,line width= 0.4pt,line join=round,line cap=round] (202.24, 61.20) -- (202.24, 55.20);

\path[draw=drawColor,line width= 0.4pt,line join=round,line cap=round] (240.50, 61.20) -- (240.50, 55.20);

\path[draw=drawColor,line width= 0.4pt,line join=round,line cap=round] (278.76, 61.20) -- (278.76, 55.20);

\path[draw=drawColor,line width= 0.4pt,line join=round,line cap=round] (317.02, 61.20) -- (317.02, 55.20);

\node[text=drawColor,anchor=base,inner sep=0pt, outer sep=0pt, scale=  1.00] at ( 49.20, 39.60) {0};

\node[text=drawColor,anchor=base,inner sep=0pt, outer sep=0pt, scale=  1.00] at ( 87.46, 39.60) {2};

\node[text=drawColor,anchor=base,inner sep=0pt, outer sep=0pt, scale=  1.00] at (125.72, 39.60) {4};

\node[text=drawColor,anchor=base,inner sep=0pt, outer sep=0pt, scale=  1.00] at (163.98, 39.60) {6};

\node[text=drawColor,anchor=base,inner sep=0pt, outer sep=0pt, scale=  1.00] at (202.24, 39.60) {8};

\node[text=drawColor,anchor=base,inner sep=0pt, outer sep=0pt, scale=  1.00] at (240.50, 39.60) {10};

\node[text=drawColor,anchor=base,inner sep=0pt, outer sep=0pt, scale=  1.00] at (278.76, 39.60) {12};

\node[text=drawColor,anchor=base,inner sep=0pt, outer sep=0pt, scale=  1.00] at (317.02, 39.60) {14};

\path[draw=drawColor,line width= 0.4pt,line join=round,line cap=round] ( 49.20, 61.20) -- ( 49.20,150.23);

\path[draw=drawColor,line width= 0.4pt,line join=round,line cap=round] ( 49.20, 61.20) -- ( 43.20, 61.20);

\path[draw=drawColor,line width= 0.4pt,line join=round,line cap=round] ( 49.20, 79.01) -- ( 43.20, 79.01);

\path[draw=drawColor,line width= 0.4pt,line join=round,line cap=round] ( 49.20, 96.81) -- ( 43.20, 96.81);

\path[draw=drawColor,line width= 0.4pt,line join=round,line cap=round] ( 49.20,114.62) -- ( 43.20,114.62);

\path[draw=drawColor,line width= 0.4pt,line join=round,line cap=round] ( 49.20,132.42) -- ( 43.20,132.42);

\path[draw=drawColor,line width= 0.4pt,line join=round,line cap=round] ( 49.20,150.23) -- ( 43.20,150.23);

\node[text=drawColor,rotate= 90.00,anchor=base,inner sep=0pt, outer sep=0pt, scale=  1.00] at ( 34.80, 61.20) {0.00};

\node[text=drawColor,rotate= 90.00,anchor=base,inner sep=0pt, outer sep=0pt, scale=  1.00] at ( 34.80, 96.81) {0.10};

\node[text=drawColor,rotate= 90.00,anchor=base,inner sep=0pt, outer sep=0pt, scale=  1.00] at ( 34.80,132.42) {0.20};
\end{scope}
\begin{scope}
\path[clip] ( 49.20, 61.20) rectangle (336.15,167.61);
\definecolor{drawColor}{RGB}{0,0,0}

\path[draw=drawColor,line width= 0.4pt,line join=round,line cap=round] ( 18.59, 61.20) rectangle ( 22.42, 61.20);

\path[draw=drawColor,line width= 0.4pt,line join=round,line cap=round] ( 22.42, 61.20) rectangle ( 26.24, 61.20);

\path[draw=drawColor,line width= 0.4pt,line join=round,line cap=round] ( 26.24, 61.20) rectangle ( 30.07, 61.20);

\path[draw=drawColor,line width= 0.4pt,line join=round,line cap=round] ( 30.07, 61.20) rectangle ( 33.90, 61.22);

\path[draw=drawColor,line width= 0.4pt,line join=round,line cap=round] ( 33.90, 61.20) rectangle ( 37.72, 61.27);

\path[draw=drawColor,line width= 0.4pt,line join=round,line cap=round] ( 37.72, 61.20) rectangle ( 41.55, 61.42);

\path[draw=drawColor,line width= 0.4pt,line join=round,line cap=round] ( 41.55, 61.20) rectangle ( 45.37, 61.83);

\path[draw=drawColor,line width= 0.4pt,line join=round,line cap=round] ( 45.37, 61.20) rectangle ( 49.20, 62.68);

\path[draw=drawColor,line width= 0.4pt,line join=round,line cap=round] ( 49.20, 61.20) rectangle ( 53.03, 64.50);

\path[draw=drawColor,line width= 0.4pt,line join=round,line cap=round] ( 53.03, 61.20) rectangle ( 56.85, 68.20);

\path[draw=drawColor,line width= 0.4pt,line join=round,line cap=round] ( 56.85, 61.20) rectangle ( 60.68, 73.79);

\path[draw=drawColor,line width= 0.4pt,line join=round,line cap=round] ( 60.68, 61.20) rectangle ( 64.50, 82.61);

\path[draw=drawColor,line width= 0.4pt,line join=round,line cap=round] ( 64.50, 61.20) rectangle ( 68.33, 94.85);

\path[draw=drawColor,line width= 0.4pt,line join=round,line cap=round] ( 68.33, 61.20) rectangle ( 72.16,109.14);

\path[draw=drawColor,line width= 0.4pt,line join=round,line cap=round] ( 72.16, 61.20) rectangle ( 75.98,124.40);

\path[draw=drawColor,line width= 0.4pt,line join=round,line cap=round] ( 75.98, 61.20) rectangle ( 79.81,139.05);

\path[draw=drawColor,line width= 0.4pt,line join=round,line cap=round] ( 79.81, 61.20) rectangle ( 83.63,151.82);

\path[draw=drawColor,line width= 0.4pt,line join=round,line cap=round] ( 83.63, 61.20) rectangle ( 87.46,160.93);

\path[draw=drawColor,line width= 0.4pt,line join=round,line cap=round] ( 87.46, 61.20) rectangle ( 91.29,165.03);

\path[draw=drawColor,line width= 0.4pt,line join=round,line cap=round] ( 91.29, 61.20) rectangle ( 95.11,166.59);

\path[draw=drawColor,line width= 0.4pt,line join=round,line cap=round] ( 95.11, 61.20) rectangle ( 98.94,164.04);

\path[draw=drawColor,line width= 0.4pt,line join=round,line cap=round] ( 98.94, 61.20) rectangle (102.76,158.87);

\path[draw=drawColor,line width= 0.4pt,line join=round,line cap=round] (102.76, 61.20) rectangle (106.59,152.94);

\path[draw=drawColor,line width= 0.4pt,line join=round,line cap=round] (106.59, 61.20) rectangle (110.42,145.27);

\path[draw=drawColor,line width= 0.4pt,line join=round,line cap=round] (110.42, 61.20) rectangle (114.24,139.10);

\path[draw=drawColor,line width= 0.4pt,line join=round,line cap=round] (114.24, 61.20) rectangle (118.07,133.33);

\path[draw=drawColor,line width= 0.4pt,line join=round,line cap=round] (118.07, 61.20) rectangle (121.89,128.50);

\path[draw=drawColor,line width= 0.4pt,line join=round,line cap=round] (121.89, 61.20) rectangle (125.72,123.78);

\path[draw=drawColor,line width= 0.4pt,line join=round,line cap=round] (125.72, 61.20) rectangle (129.55,117.53);

\path[draw=drawColor,line width= 0.4pt,line join=round,line cap=round] (129.55, 61.20) rectangle (133.37,111.14);

\path[draw=drawColor,line width= 0.4pt,line join=round,line cap=round] (133.37, 61.20) rectangle (137.20,103.41);

\path[draw=drawColor,line width= 0.4pt,line join=round,line cap=round] (137.20, 61.20) rectangle (141.02, 96.18);

\path[draw=drawColor,line width= 0.4pt,line join=round,line cap=round] (141.02, 61.20) rectangle (144.85, 89.35);

\path[draw=drawColor,line width= 0.4pt,line join=round,line cap=round] (144.85, 61.20) rectangle (148.68, 83.11);

\path[draw=drawColor,line width= 0.4pt,line join=round,line cap=round] (148.68, 61.20) rectangle (152.50, 78.09);

\path[draw=drawColor,line width= 0.4pt,line join=round,line cap=round] (152.50, 61.20) rectangle (156.33, 74.80);

\path[draw=drawColor,line width= 0.4pt,line join=round,line cap=round] (156.33, 61.20) rectangle (160.15, 73.11);

\path[draw=drawColor,line width= 0.4pt,line join=round,line cap=round] (160.15, 61.20) rectangle (163.98, 72.71);

\path[draw=drawColor,line width= 0.4pt,line join=round,line cap=round] (163.98, 61.20) rectangle (167.81, 72.71);

\path[draw=drawColor,line width= 0.4pt,line join=round,line cap=round] (167.81, 61.20) rectangle (171.63, 73.15);

\path[draw=drawColor,line width= 0.4pt,line join=round,line cap=round] (171.63, 61.20) rectangle (175.46, 73.04);

\path[draw=drawColor,line width= 0.4pt,line join=round,line cap=round] (175.46, 61.20) rectangle (179.28, 72.54);

\path[draw=drawColor,line width= 0.4pt,line join=round,line cap=round] (179.28, 61.20) rectangle (183.11, 72.20);

\path[draw=drawColor,line width= 0.4pt,line join=round,line cap=round] (183.11, 61.20) rectangle (186.94, 71.35);

\path[draw=drawColor,line width= 0.4pt,line join=round,line cap=round] (186.94, 61.20) rectangle (190.76, 71.37);

\path[draw=drawColor,line width= 0.4pt,line join=round,line cap=round] (190.76, 61.20) rectangle (194.59, 71.51);

\path[draw=drawColor,line width= 0.4pt,line join=round,line cap=round] (194.59, 61.20) rectangle (198.41, 71.71);

\path[draw=drawColor,line width= 0.4pt,line join=round,line cap=round] (198.41, 61.20) rectangle (202.24, 71.87);

\path[draw=drawColor,line width= 0.4pt,line join=round,line cap=round] (202.24, 61.20) rectangle (206.07, 71.77);

\path[draw=drawColor,line width= 0.4pt,line join=round,line cap=round] (206.07, 61.20) rectangle (209.89, 70.85);

\path[draw=drawColor,line width= 0.4pt,line join=round,line cap=round] (209.89, 61.20) rectangle (213.72, 69.18);

\path[draw=drawColor,line width= 0.4pt,line join=round,line cap=round] (213.72, 61.20) rectangle (217.54, 67.48);

\path[draw=drawColor,line width= 0.4pt,line join=round,line cap=round] (217.54, 61.20) rectangle (221.37, 65.73);

\path[draw=drawColor,line width= 0.4pt,line join=round,line cap=round] (221.37, 61.20) rectangle (225.20, 64.18);

\path[draw=drawColor,line width= 0.4pt,line join=round,line cap=round] (225.20, 61.20) rectangle (229.02, 62.97);

\path[draw=drawColor,line width= 0.4pt,line join=round,line cap=round] (229.02, 61.20) rectangle (232.85, 62.23);

\path[draw=drawColor,line width= 0.4pt,line join=round,line cap=round] (232.85, 61.20) rectangle (236.67, 61.77);

\path[draw=drawColor,line width= 0.4pt,line join=round,line cap=round] (236.67, 61.20) rectangle (240.50, 61.51);

\path[draw=drawColor,line width= 0.4pt,line join=round,line cap=round] (240.50, 61.20) rectangle (244.33, 61.37);

\path[draw=drawColor,line width= 0.4pt,line join=round,line cap=round] (244.33, 61.20) rectangle (248.15, 61.29);

\path[draw=drawColor,line width= 0.4pt,line join=round,line cap=round] (248.15, 61.20) rectangle (251.98, 61.33);

\path[draw=drawColor,line width= 0.4pt,line join=round,line cap=round] (251.98, 61.20) rectangle (255.80, 61.36);

\path[draw=drawColor,line width= 0.4pt,line join=round,line cap=round] (255.80, 61.20) rectangle (259.63, 61.43);

\path[draw=drawColor,line width= 0.4pt,line join=round,line cap=round] (259.63, 61.20) rectangle (263.46, 61.62);

\path[draw=drawColor,line width= 0.4pt,line join=round,line cap=round] (263.46, 61.20) rectangle (267.28, 61.88);

\path[draw=drawColor,line width= 0.4pt,line join=round,line cap=round] (267.28, 61.20) rectangle (271.11, 62.23);

\path[draw=drawColor,line width= 0.4pt,line join=round,line cap=round] (271.11, 61.20) rectangle (274.93, 62.57);

\path[draw=drawColor,line width= 0.4pt,line join=round,line cap=round] (274.93, 61.20) rectangle (278.76, 62.76);

\path[draw=drawColor,line width= 0.4pt,line join=round,line cap=round] (278.76, 61.20) rectangle (282.59, 62.73);

\path[draw=drawColor,line width= 0.4pt,line join=round,line cap=round] (282.59, 61.20) rectangle (286.41, 62.59);

\path[draw=drawColor,line width= 0.4pt,line join=round,line cap=round] (286.41, 61.20) rectangle (290.24, 62.26);

\path[draw=drawColor,line width= 0.4pt,line join=round,line cap=round] (290.24, 61.20) rectangle (294.06, 61.96);

\path[draw=drawColor,line width= 0.4pt,line join=round,line cap=round] (294.06, 61.20) rectangle (297.89, 61.71);

\path[draw=drawColor,line width= 0.4pt,line join=round,line cap=round] (297.89, 61.20) rectangle (301.72, 61.48);

\path[draw=drawColor,line width= 0.4pt,line join=round,line cap=round] (301.72, 61.20) rectangle (305.54, 61.33);

\path[draw=drawColor,line width= 0.4pt,line join=round,line cap=round] (305.54, 61.20) rectangle (309.37, 61.25);

\path[draw=drawColor,line width= 0.4pt,line join=round,line cap=round] (309.37, 61.20) rectangle (313.19, 61.23);

\path[draw=drawColor,line width= 0.4pt,line join=round,line cap=round] (313.19, 61.20) rectangle (317.02, 61.22);

\path[draw=drawColor,line width= 0.4pt,line join=round,line cap=round] (317.02, 61.20) rectangle (320.85, 61.20);

\path[draw=drawColor,line width= 0.4pt,line join=round,line cap=round] (320.85, 61.20) rectangle (324.67, 61.21);

\path[draw=drawColor,line width= 0.4pt,line join=round,line cap=round] (324.67, 61.20) rectangle (328.50, 61.22);

\path[draw=drawColor,line width= 0.4pt,line join=round,line cap=round] (328.50, 61.20) rectangle (332.32, 61.22);

\path[draw=drawColor,line width= 0.4pt,line join=round,line cap=round] (332.32, 61.20) rectangle (336.15, 61.23);

\path[draw=drawColor,line width= 0.4pt,line join=round,line cap=round] (336.15, 61.20) rectangle (339.98, 61.23);

\path[draw=drawColor,line width= 0.4pt,line join=round,line cap=round] (339.98, 61.20) rectangle (343.80, 61.23);

\path[draw=drawColor,line width= 0.4pt,line join=round,line cap=round] (343.80, 61.20) rectangle (347.63, 61.23);

\path[draw=drawColor,line width= 0.4pt,line join=round,line cap=round] (347.63, 61.20) rectangle (351.45, 61.21);

\path[draw=drawColor,line width= 0.4pt,line join=round,line cap=round] (351.45, 61.20) rectangle (355.28, 61.21);

\path[draw=drawColor,line width= 0.4pt,line join=round,line cap=round] (355.28, 61.20) rectangle (359.11, 61.21);

\path[draw=drawColor,line width= 0.4pt,line join=round,line cap=round] (359.11, 61.20) rectangle (362.93, 61.21);
\definecolor{drawColor}{RGB}{255,0,0}

\path[draw=drawColor,line width= 1.2pt,line join=round,line cap=round] (  0.00, 61.20) --
	(  0.41, 61.20) --
	(  3.10, 61.20) --
	(  5.79, 61.20) --
	(  8.48, 61.20) --
	( 11.17, 61.20) --
	( 13.86, 61.20) --
	( 16.55, 61.20) --
	( 19.24, 61.20) --
	( 21.93, 61.20) --
	( 24.62, 61.20) --
	( 27.30, 61.20) --
	( 29.99, 61.21) --
	( 32.68, 61.23) --
	( 35.37, 61.26) --
	( 38.06, 61.34) --
	( 40.75, 61.49) --
	( 43.44, 61.78) --
	( 46.13, 62.32) --
	( 48.82, 63.25) --
	( 51.50, 64.77) --
	( 54.19, 67.13) --
	( 56.88, 70.60) --
	( 59.57, 75.44) --
	( 62.26, 81.81) --
	( 64.95, 89.75) --
	( 67.64, 99.10) --
	( 70.33,109.50) --
	( 73.02,120.43) --
	( 75.71,131.25) --
	( 78.39,141.32) --
	( 81.08,150.09) --
	( 83.77,157.16) --
	( 86.46,162.29) --
	( 89.15,165.40) --
	( 91.84,166.56) --
	( 94.53,165.93) --
	( 97.22,163.75) --
	( 99.91,160.36) --
	(102.59,156.10) --
	(105.28,151.37) --
	(107.97,146.54) --
	(110.66,141.89) --
	(113.35,137.61) --
	(116.04,133.73) --
	(118.73,130.15) --
	(121.42,126.66) --
	(124.11,123.04) --
	(126.80,119.08) --
	(129.48,114.67) --
	(132.17,109.83) --
	(134.86,104.66) --
	(137.55, 99.34) --
	(140.24, 94.10) --
	(142.93, 89.13) --
	(145.62, 84.62) --
	(148.31, 80.73) --
	(151.00, 77.57) --
	(153.69, 75.20) --
	(156.37, 73.61) --
	(159.06, 72.73) --
	(161.75, 72.43) --
	(164.44, 72.51) --
	(167.13, 72.76) --
	(169.82, 72.98) --
	(172.51, 73.04) --
	(175.20, 72.89) --
	(177.89, 72.55) --
	(180.57, 72.11) --
	(183.26, 71.69) --
	(185.95, 71.38) --
	(188.64, 71.25) --
	(191.33, 71.31) --
	(194.02, 71.51) --
	(196.71, 71.75) --
	(199.40, 71.92) --
	(202.09, 71.91) --
	(204.78, 71.61) --
	(207.46, 70.99) --
	(210.15, 70.06) --
	(212.84, 68.90) --
	(215.53, 67.61) --
	(218.22, 66.30) --
	(220.91, 65.10) --
	(223.60, 64.06) --
	(226.29, 63.21) --
	(228.98, 62.57) --
	(231.66, 62.11) --
	(234.35, 61.79) --
	(237.04, 61.58) --
	(239.73, 61.44) --
	(242.42, 61.36) --
	(245.11, 61.32) --
	(247.80, 61.30) --
	(250.49, 61.31) --
	(253.18, 61.33) --
	(255.87, 61.39) --
	(258.55, 61.48) --
	(261.24, 61.61) --
	(263.93, 61.78) --
	(266.62, 61.99) --
	(269.31, 62.22) --
	(272.00, 62.45) --
	(274.69, 62.63) --
	(277.38, 62.75) --
	(280.07, 62.77) --
	(282.76, 62.69) --
	(285.44, 62.53) --
	(288.13, 62.32) --
	(290.82, 62.08) --
	(293.51, 61.85) --
	(296.20, 61.66) --
	(298.89, 61.50) --
	(301.58, 61.39) --
	(304.27, 61.31) --
	(306.96, 61.26) --
	(309.64, 61.23) --
	(312.33, 61.22) --
	(315.02, 61.21) --
	(317.71, 61.21) --
	(320.40, 61.21) --
	(323.09, 61.21) --
	(325.78, 61.21) --
	(328.47, 61.22) --
	(331.16, 61.22) --
	(333.85, 61.23) --
	(336.53, 61.23) --
	(339.22, 61.23) --
	(341.91, 61.23) --
	(344.60, 61.23) --
	(347.29, 61.23) --
	(349.98, 61.23) --
	(352.67, 61.22) --
	(355.36, 61.21) --
	(358.05, 61.21) --
	(360.73, 61.21) --
	(361.35, 61.21);
\end{scope}
\end{tikzpicture}

	}
	\captionvspace
	\caption{The histogram show the result of the conditional sampling according to \cref{alg:conditional hard}. The red line represents the true \ac{pdf}.}
	\label{fig:conditional hard}
\end{figure}



\subsection{Sampling with linear constraints}

In this second example, we want to sample the speed difference, while the average deceleration equals $\SI{1}{\meter\per\second\squared}$ and the start speed equals $\SI{15}{\meter\per\second}$. 
This means that we have
\begin{equation*}
	\constraintmatrix = \begin{bmatrix} 1 & 0 & 0 \\ 0 & 1 & 1 \end{bmatrix}, 
	\constraintvector = \begin{bmatrix} 1 \\ 15 \end{bmatrix}.
\end{equation*}
The \ac{svd} according to \cref{eq:svd constraint matrix} gives
\begin{align*}
	&\svdu = \begin{bmatrix} 0 & 1 \\ 1 & 0 \end{bmatrix},
	\svds = \begin{bmatrix} \sqrt{2} & 0 \\ 0 & 1 \end{bmatrix}, 
	\svdva^T = \begin{bmatrix} 0 & \frac{1}{2}\sqrt{2} & \frac{1}{2}\sqrt{2} \\ 1 & 0 & 0 \end{bmatrix},\\
	&\svdvb^T = \begin{bmatrix} 0 & -\frac{1}{2}\sqrt{2} & \frac{1}{2}\sqrt{2} \end{bmatrix}.
\end{align*}

In \cref{fig:constrained simple,fig:constrained hard}, the result of \cref{alg:constrained simple,alg:constrained hard} is shown, respectively. 
In total, $10^6$ samples are generated and shown by the histogram. 
Because the histograms follow the same pattern as the actual density of the speed difference according to the \ac{kde}, it illustrates that the provided algorithms correctly sample from the \ac{kde}. 

\begin{figure}
	\figurevspace
	\centering
	\resizebox{\columnwidth}{!}{%
		% Created by tikzDevice version 0.12.3 on 2021-02-01 12:22:11
% !TEX encoding = UTF-8 Unicode
\begin{tikzpicture}[x=1pt,y=1pt]
\definecolor{fillColor}{RGB}{255,255,255}
\path[use as bounding box,fill=fillColor,fill opacity=0.00] (0,0) rectangle (303.53,216.81);
\begin{scope}
\path[clip] (  0.00,  0.00) rectangle (303.53,216.81);
\definecolor{drawColor}{RGB}{0,0,0}

\node[text=drawColor,anchor=base,inner sep=0pt, outer sep=0pt, scale=  1.00] at (163.77, 15.60) {Speed difference [m/s]};

\node[text=drawColor,rotate= 90.00,anchor=base,inner sep=0pt, outer sep=0pt, scale=  1.00] at ( 10.80,114.41) {Density};
\end{scope}
\begin{scope}
\path[clip] (  0.00,  0.00) rectangle (303.53,216.81);
\definecolor{drawColor}{RGB}{0,0,0}

\path[draw=drawColor,line width= 0.4pt,line join=round,line cap=round] ( 49.20, 61.20) -- (263.06, 61.20);

\path[draw=drawColor,line width= 0.4pt,line join=round,line cap=round] ( 49.20, 61.20) -- ( 49.20, 55.20);

\path[draw=drawColor,line width= 0.4pt,line join=round,line cap=round] ( 79.75, 61.20) -- ( 79.75, 55.20);

\path[draw=drawColor,line width= 0.4pt,line join=round,line cap=round] (110.30, 61.20) -- (110.30, 55.20);

\path[draw=drawColor,line width= 0.4pt,line join=round,line cap=round] (140.85, 61.20) -- (140.85, 55.20);

\path[draw=drawColor,line width= 0.4pt,line join=round,line cap=round] (171.40, 61.20) -- (171.40, 55.20);

\path[draw=drawColor,line width= 0.4pt,line join=round,line cap=round] (201.96, 61.20) -- (201.96, 55.20);

\path[draw=drawColor,line width= 0.4pt,line join=round,line cap=round] (232.51, 61.20) -- (232.51, 55.20);

\path[draw=drawColor,line width= 0.4pt,line join=round,line cap=round] (263.06, 61.20) -- (263.06, 55.20);

\node[text=drawColor,anchor=base,inner sep=0pt, outer sep=0pt, scale=  1.00] at ( 49.20, 39.60) {0};

\node[text=drawColor,anchor=base,inner sep=0pt, outer sep=0pt, scale=  1.00] at ( 79.75, 39.60) {2};

\node[text=drawColor,anchor=base,inner sep=0pt, outer sep=0pt, scale=  1.00] at (110.30, 39.60) {4};

\node[text=drawColor,anchor=base,inner sep=0pt, outer sep=0pt, scale=  1.00] at (140.85, 39.60) {6};

\node[text=drawColor,anchor=base,inner sep=0pt, outer sep=0pt, scale=  1.00] at (171.40, 39.60) {8};

\node[text=drawColor,anchor=base,inner sep=0pt, outer sep=0pt, scale=  1.00] at (201.96, 39.60) {10};

\node[text=drawColor,anchor=base,inner sep=0pt, outer sep=0pt, scale=  1.00] at (232.51, 39.60) {12};

\node[text=drawColor,anchor=base,inner sep=0pt, outer sep=0pt, scale=  1.00] at (263.06, 39.60) {14};

\path[draw=drawColor,line width= 0.4pt,line join=round,line cap=round] ( 49.20, 61.20) -- ( 49.20,164.23);

\path[draw=drawColor,line width= 0.4pt,line join=round,line cap=round] ( 49.20, 61.20) -- ( 43.20, 61.20);

\path[draw=drawColor,line width= 0.4pt,line join=round,line cap=round] ( 49.20, 75.92) -- ( 43.20, 75.92);

\path[draw=drawColor,line width= 0.4pt,line join=round,line cap=round] ( 49.20, 90.64) -- ( 43.20, 90.64);

\path[draw=drawColor,line width= 0.4pt,line join=round,line cap=round] ( 49.20,105.36) -- ( 43.20,105.36);

\path[draw=drawColor,line width= 0.4pt,line join=round,line cap=round] ( 49.20,120.07) -- ( 43.20,120.07);

\path[draw=drawColor,line width= 0.4pt,line join=round,line cap=round] ( 49.20,134.79) -- ( 43.20,134.79);

\path[draw=drawColor,line width= 0.4pt,line join=round,line cap=round] ( 49.20,149.51) -- ( 43.20,149.51);

\path[draw=drawColor,line width= 0.4pt,line join=round,line cap=round] ( 49.20,164.23) -- ( 43.20,164.23);

\node[text=drawColor,rotate= 90.00,anchor=base,inner sep=0pt, outer sep=0pt, scale=  1.00] at ( 34.80, 61.20) {0.00};

\node[text=drawColor,rotate= 90.00,anchor=base,inner sep=0pt, outer sep=0pt, scale=  1.00] at ( 34.80, 90.64) {0.04};

\node[text=drawColor,rotate= 90.00,anchor=base,inner sep=0pt, outer sep=0pt, scale=  1.00] at ( 34.80,120.07) {0.08};

\node[text=drawColor,rotate= 90.00,anchor=base,inner sep=0pt, outer sep=0pt, scale=  1.00] at ( 34.80,149.51) {0.12};
\end{scope}
\begin{scope}
\path[clip] ( 49.20, 61.20) rectangle (278.33,167.61);
\definecolor{drawColor}{RGB}{0,0,0}

\path[draw=drawColor,line width= 0.4pt,line join=round,line cap=round] ( -2.74, 61.20) rectangle (  0.32, 61.20);

\path[draw=drawColor,line width= 0.4pt,line join=round,line cap=round] (  0.32, 61.20) rectangle (  3.37, 61.20);

\path[draw=drawColor,line width= 0.4pt,line join=round,line cap=round] (  3.37, 61.20) rectangle (  6.43, 61.20);

\path[draw=drawColor,line width= 0.4pt,line join=round,line cap=round] (  6.43, 61.20) rectangle (  9.48, 61.20);

\path[draw=drawColor,line width= 0.4pt,line join=round,line cap=round] (  9.48, 61.20) rectangle ( 12.54, 61.20);

\path[draw=drawColor,line width= 0.4pt,line join=round,line cap=round] ( 12.54, 61.20) rectangle ( 15.59, 61.20);

\path[draw=drawColor,line width= 0.4pt,line join=round,line cap=round] ( 15.59, 61.20) rectangle ( 18.65, 61.20);

\path[draw=drawColor,line width= 0.4pt,line join=round,line cap=round] ( 18.65, 61.20) rectangle ( 21.70, 61.20);

\path[draw=drawColor,line width= 0.4pt,line join=round,line cap=round] ( 21.70, 61.20) rectangle ( 24.76, 61.20);

\path[draw=drawColor,line width= 0.4pt,line join=round,line cap=round] ( 24.76, 61.20) rectangle ( 27.81, 61.20);

\path[draw=drawColor,line width= 0.4pt,line join=round,line cap=round] ( 27.81, 61.20) rectangle ( 30.87, 61.20);

\path[draw=drawColor,line width= 0.4pt,line join=round,line cap=round] ( 30.87, 61.20) rectangle ( 33.92, 61.20);

\path[draw=drawColor,line width= 0.4pt,line join=round,line cap=round] ( 33.92, 61.20) rectangle ( 36.98, 61.20);

\path[draw=drawColor,line width= 0.4pt,line join=round,line cap=round] ( 36.98, 61.20) rectangle ( 40.03, 61.20);

\path[draw=drawColor,line width= 0.4pt,line join=round,line cap=round] ( 40.03, 61.20) rectangle ( 43.09, 61.22);

\path[draw=drawColor,line width= 0.4pt,line join=round,line cap=round] ( 43.09, 61.20) rectangle ( 46.14, 61.27);

\path[draw=drawColor,line width= 0.4pt,line join=round,line cap=round] ( 46.14, 61.20) rectangle ( 49.20, 61.42);

\path[draw=drawColor,line width= 0.4pt,line join=round,line cap=round] ( 49.20, 61.20) rectangle ( 52.26, 61.94);

\path[draw=drawColor,line width= 0.4pt,line join=round,line cap=round] ( 52.26, 61.20) rectangle ( 55.31, 63.05);

\path[draw=drawColor,line width= 0.4pt,line join=round,line cap=round] ( 55.31, 61.20) rectangle ( 58.37, 66.04);

\path[draw=drawColor,line width= 0.4pt,line join=round,line cap=round] ( 58.37, 61.20) rectangle ( 61.42, 70.58);

\path[draw=drawColor,line width= 0.4pt,line join=round,line cap=round] ( 61.42, 61.20) rectangle ( 64.48, 78.77);

\path[draw=drawColor,line width= 0.4pt,line join=round,line cap=round] ( 64.48, 61.20) rectangle ( 67.53, 89.26);

\path[draw=drawColor,line width= 0.4pt,line join=round,line cap=round] ( 67.53, 61.20) rectangle ( 70.59,103.07);

\path[draw=drawColor,line width= 0.4pt,line join=round,line cap=round] ( 70.59, 61.20) rectangle ( 73.64,114.13);

\path[draw=drawColor,line width= 0.4pt,line join=round,line cap=round] ( 73.64, 61.20) rectangle ( 76.70,125.48);

\path[draw=drawColor,line width= 0.4pt,line join=round,line cap=round] ( 76.70, 61.20) rectangle ( 79.75,135.26);

\path[draw=drawColor,line width= 0.4pt,line join=round,line cap=round] ( 79.75, 61.20) rectangle ( 82.81,145.00);

\path[draw=drawColor,line width= 0.4pt,line join=round,line cap=round] ( 82.81, 61.20) rectangle ( 85.86,151.37);

\path[draw=drawColor,line width= 0.4pt,line join=round,line cap=round] ( 85.86, 61.20) rectangle ( 88.92,157.32);

\path[draw=drawColor,line width= 0.4pt,line join=round,line cap=round] ( 88.92, 61.20) rectangle ( 91.97,162.35);

\path[draw=drawColor,line width= 0.4pt,line join=round,line cap=round] ( 91.97, 61.20) rectangle ( 95.03,164.90);

\path[draw=drawColor,line width= 0.4pt,line join=round,line cap=round] ( 95.03, 61.20) rectangle ( 98.08,165.90);

\path[draw=drawColor,line width= 0.4pt,line join=round,line cap=round] ( 98.08, 61.20) rectangle (101.14,166.42);

\path[draw=drawColor,line width= 0.4pt,line join=round,line cap=round] (101.14, 61.20) rectangle (104.19,162.61);

\path[draw=drawColor,line width= 0.4pt,line join=round,line cap=round] (104.19, 61.20) rectangle (107.25,160.24);

\path[draw=drawColor,line width= 0.4pt,line join=round,line cap=round] (107.25, 61.20) rectangle (110.30,155.25);

\path[draw=drawColor,line width= 0.4pt,line join=round,line cap=round] (110.30, 61.20) rectangle (113.36,152.41);

\path[draw=drawColor,line width= 0.4pt,line join=round,line cap=round] (113.36, 61.20) rectangle (116.41,147.47);

\path[draw=drawColor,line width= 0.4pt,line join=round,line cap=round] (116.41, 61.20) rectangle (119.47,142.96);

\path[draw=drawColor,line width= 0.4pt,line join=round,line cap=round] (119.47, 61.20) rectangle (122.52,134.19);

\path[draw=drawColor,line width= 0.4pt,line join=round,line cap=round] (122.52, 61.20) rectangle (125.58,123.50);

\path[draw=drawColor,line width= 0.4pt,line join=round,line cap=round] (125.58, 61.20) rectangle (128.63,113.59);

\path[draw=drawColor,line width= 0.4pt,line join=round,line cap=round] (128.63, 61.20) rectangle (131.69,106.09);

\path[draw=drawColor,line width= 0.4pt,line join=round,line cap=round] (131.69, 61.20) rectangle (134.74,101.04);

\path[draw=drawColor,line width= 0.4pt,line join=round,line cap=round] (134.74, 61.20) rectangle (137.80,101.03);

\path[draw=drawColor,line width= 0.4pt,line join=round,line cap=round] (137.80, 61.20) rectangle (140.85,104.93);

\path[draw=drawColor,line width= 0.4pt,line join=round,line cap=round] (140.85, 61.20) rectangle (143.91,112.07);

\path[draw=drawColor,line width= 0.4pt,line join=round,line cap=round] (143.91, 61.20) rectangle (146.96,119.40);

\path[draw=drawColor,line width= 0.4pt,line join=round,line cap=round] (146.96, 61.20) rectangle (150.02,126.70);

\path[draw=drawColor,line width= 0.4pt,line join=round,line cap=round] (150.02, 61.20) rectangle (153.07,129.22);

\path[draw=drawColor,line width= 0.4pt,line join=round,line cap=round] (153.07, 61.20) rectangle (156.13,129.40);

\path[draw=drawColor,line width= 0.4pt,line join=round,line cap=round] (156.13, 61.20) rectangle (159.18,126.86);

\path[draw=drawColor,line width= 0.4pt,line join=round,line cap=round] (159.18, 61.20) rectangle (162.24,123.78);

\path[draw=drawColor,line width= 0.4pt,line join=round,line cap=round] (162.24, 61.20) rectangle (165.29,122.78);

\path[draw=drawColor,line width= 0.4pt,line join=round,line cap=round] (165.29, 61.20) rectangle (168.35,125.73);

\path[draw=drawColor,line width= 0.4pt,line join=round,line cap=round] (168.35, 61.20) rectangle (171.40,129.85);

\path[draw=drawColor,line width= 0.4pt,line join=round,line cap=round] (171.40, 61.20) rectangle (174.46,133.00);

\path[draw=drawColor,line width= 0.4pt,line join=round,line cap=round] (174.46, 61.20) rectangle (177.52,131.07);

\path[draw=drawColor,line width= 0.4pt,line join=round,line cap=round] (177.52, 61.20) rectangle (180.57,125.42);

\path[draw=drawColor,line width= 0.4pt,line join=round,line cap=round] (180.57, 61.20) rectangle (183.63,116.34);

\path[draw=drawColor,line width= 0.4pt,line join=round,line cap=round] (183.63, 61.20) rectangle (186.68,106.08);

\path[draw=drawColor,line width= 0.4pt,line join=round,line cap=round] (186.68, 61.20) rectangle (189.74, 98.16);

\path[draw=drawColor,line width= 0.4pt,line join=round,line cap=round] (189.74, 61.20) rectangle (192.79, 90.66);

\path[draw=drawColor,line width= 0.4pt,line join=round,line cap=round] (192.79, 61.20) rectangle (195.85, 86.08);

\path[draw=drawColor,line width= 0.4pt,line join=round,line cap=round] (195.85, 61.20) rectangle (198.90, 82.84);

\path[draw=drawColor,line width= 0.4pt,line join=round,line cap=round] (198.90, 61.20) rectangle (201.96, 80.90);

\path[draw=drawColor,line width= 0.4pt,line join=round,line cap=round] (201.96, 61.20) rectangle (205.01, 80.42);

\path[draw=drawColor,line width= 0.4pt,line join=round,line cap=round] (205.01, 61.20) rectangle (208.07, 78.61);

\path[draw=drawColor,line width= 0.4pt,line join=round,line cap=round] (208.07, 61.20) rectangle (211.12, 76.43);

\path[draw=drawColor,line width= 0.4pt,line join=round,line cap=round] (211.12, 61.20) rectangle (214.18, 73.14);

\path[draw=drawColor,line width= 0.4pt,line join=round,line cap=round] (214.18, 61.20) rectangle (217.23, 71.11);

\path[draw=drawColor,line width= 0.4pt,line join=round,line cap=round] (217.23, 61.20) rectangle (220.29, 70.08);

\path[draw=drawColor,line width= 0.4pt,line join=round,line cap=round] (220.29, 61.20) rectangle (223.34, 70.56);

\path[draw=drawColor,line width= 0.4pt,line join=round,line cap=round] (223.34, 61.20) rectangle (226.40, 73.22);

\path[draw=drawColor,line width= 0.4pt,line join=round,line cap=round] (226.40, 61.20) rectangle (229.45, 76.40);

\path[draw=drawColor,line width= 0.4pt,line join=round,line cap=round] (229.45, 61.20) rectangle (232.51, 81.25);

\path[draw=drawColor,line width= 0.4pt,line join=round,line cap=round] (232.51, 61.20) rectangle (235.56, 85.11);

\path[draw=drawColor,line width= 0.4pt,line join=round,line cap=round] (235.56, 61.20) rectangle (238.62, 90.01);

\path[draw=drawColor,line width= 0.4pt,line join=round,line cap=round] (238.62, 61.20) rectangle (241.67, 95.91);

\path[draw=drawColor,line width= 0.4pt,line join=round,line cap=round] (241.67, 61.20) rectangle (244.73,100.18);

\path[draw=drawColor,line width= 0.4pt,line join=round,line cap=round] (244.73, 61.20) rectangle (247.78,105.13);

\path[draw=drawColor,line width= 0.4pt,line join=round,line cap=round] (247.78, 61.20) rectangle (250.84,108.35);

\path[draw=drawColor,line width= 0.4pt,line join=round,line cap=round] (250.84, 61.20) rectangle (253.89,108.91);

\path[draw=drawColor,line width= 0.4pt,line join=round,line cap=round] (253.89, 61.20) rectangle (256.95,106.87);

\path[draw=drawColor,line width= 0.4pt,line join=round,line cap=round] (256.95, 61.20) rectangle (260.00,102.92);

\path[draw=drawColor,line width= 0.4pt,line join=round,line cap=round] (260.00, 61.20) rectangle (263.06, 97.42);

\path[draw=drawColor,line width= 0.4pt,line join=round,line cap=round] (263.06, 61.20) rectangle (266.11, 92.63);

\path[draw=drawColor,line width= 0.4pt,line join=round,line cap=round] (266.11, 61.20) rectangle (269.17, 88.73);

\path[draw=drawColor,line width= 0.4pt,line join=round,line cap=round] (269.17, 61.20) rectangle (272.22, 85.73);

\path[draw=drawColor,line width= 0.4pt,line join=round,line cap=round] (272.22, 61.20) rectangle (275.28, 83.68);

\path[draw=drawColor,line width= 0.4pt,line join=round,line cap=round] (275.28, 61.20) rectangle (278.33, 81.01);

\path[draw=drawColor,line width= 0.4pt,line join=round,line cap=round] (278.33, 61.20) rectangle (281.39, 78.20);

\path[draw=drawColor,line width= 0.4pt,line join=round,line cap=round] (281.39, 61.20) rectangle (284.44, 75.10);

\path[draw=drawColor,line width= 0.4pt,line join=round,line cap=round] (284.44, 61.20) rectangle (287.50, 71.65);

\path[draw=drawColor,line width= 0.4pt,line join=round,line cap=round] (287.50, 61.20) rectangle (290.55, 68.97);

\path[draw=drawColor,line width= 0.4pt,line join=round,line cap=round] (290.55, 61.20) rectangle (293.61, 66.55);

\path[draw=drawColor,line width= 0.4pt,line join=round,line cap=round] (293.61, 61.20) rectangle (296.66, 64.95);

\path[draw=drawColor,line width= 0.4pt,line join=round,line cap=round] (296.66, 61.20) rectangle (299.72, 63.55);

\path[draw=drawColor,line width= 0.4pt,line join=round,line cap=round] (299.72, 61.20) rectangle (302.77, 62.46);

\path[draw=drawColor,line width= 0.4pt,line join=round,line cap=round] (302.77, 61.20) rectangle (305.83, 61.82);
\definecolor{drawColor}{RGB}{255,0,0}

\path[draw=drawColor,line width= 1.2pt,line join=round,line cap=round] (  0.00, 61.20) --
	(  1.66, 61.20) --
	(  3.80, 61.20) --
	(  5.95, 61.20) --
	(  8.10, 61.20) --
	( 10.24, 61.20) --
	( 12.39, 61.20) --
	( 14.54, 61.20) --
	( 16.69, 61.20) --
	( 18.83, 61.20) --
	( 20.98, 61.20) --
	( 23.13, 61.20) --
	( 25.27, 61.20) --
	( 27.42, 61.20) --
	( 29.57, 61.20) --
	( 31.72, 61.20) --
	( 33.86, 61.20) --
	( 36.01, 61.20) --
	( 38.16, 61.20) --
	( 40.30, 61.21) --
	( 42.45, 61.22) --
	( 44.60, 61.26) --
	( 46.75, 61.35) --
	( 48.89, 61.55) --
	( 51.04, 61.96) --
	( 53.19, 62.75) --
	( 55.33, 64.17) --
	( 57.48, 66.53) --
	( 59.63, 70.13) --
	( 61.78, 75.24) --
	( 63.92, 81.88) --
	( 66.07, 89.84) --
	( 68.22, 98.61) --
	( 70.36,107.57) --
	( 72.51,116.19) --
	( 74.66,124.15) --
	( 76.81,131.42) --
	( 78.95,138.10) --
	( 81.10,144.22) --
	( 83.25,149.71) --
	( 85.40,154.42) --
	( 87.54,158.26) --
	( 89.69,161.28) --
	( 91.84,163.61) --
	( 93.98,165.31) --
	( 96.13,166.35) --
	( 98.28,166.56) --
	(100.43,165.73) --
	(102.57,163.86) --
	(104.72,161.16) --
	(106.87,158.08) --
	(109.01,155.10) --
	(111.16,152.45) --
	(113.31,149.98) --
	(115.46,147.15) --
	(117.60,143.34) --
	(119.75,138.12) --
	(121.90,131.54) --
	(124.04,124.12) --
	(126.19,116.66) --
	(128.34,110.02) --
	(130.49,104.89) --
	(132.63,101.68) --
	(134.78,100.57) --
	(136.93,101.50) --
	(139.07,104.28) --
	(141.22,108.58) --
	(143.37,113.89) --
	(145.52,119.54) --
	(147.66,124.69) --
	(149.81,128.48) --
	(151.96,130.30) --
	(154.10,130.00) --
	(156.25,128.06) --
	(158.40,125.45) --
	(160.55,123.32) --
	(162.69,122.61) --
	(164.84,123.71) --
	(166.99,126.30) --
	(169.13,129.47) --
	(171.28,132.05) --
	(173.43,132.98) --
	(175.58,131.65) --
	(177.72,128.04) --
	(179.87,122.62) --
	(182.02,116.17) --
	(184.17,109.44) --
	(186.31,103.02) --
	(188.46, 97.28) --
	(190.61, 92.38) --
	(192.75, 88.35) --
	(194.90, 85.22) --
	(197.05, 82.93) --
	(199.20, 81.40) --
	(201.34, 80.41) --
	(203.49, 79.66) --
	(205.64, 78.80) --
	(207.78, 77.59) --
	(209.93, 75.94) --
	(212.08, 74.02) --
	(214.23, 72.14) --
	(216.37, 70.69) --
	(218.52, 69.97) --
	(220.67, 70.14) --
	(222.81, 71.23) --
	(224.96, 73.10) --
	(227.11, 75.56) --
	(229.26, 78.39) --
	(231.40, 81.40) --
	(233.55, 84.50) --
	(235.70, 87.68) --
	(237.84, 91.02) --
	(239.99, 94.58) --
	(242.14, 98.34) --
	(244.29,102.09) --
	(246.43,105.48) --
	(248.58,108.07) --
	(250.73,109.44) --
	(252.87,109.35) --
	(255.02,107.76) --
	(257.17,104.92) --
	(259.32,101.28) --
	(261.46, 97.42) --
	(263.61, 93.82) --
	(265.76, 90.78) --
	(267.90, 88.34) --
	(270.05, 86.36) --
	(272.20, 84.61) --
	(274.35, 82.89) --
	(276.49, 81.08) --
	(278.64, 79.12) --
	(280.79, 77.04) --
	(282.94, 74.90) --
	(285.08, 72.77) --
	(287.23, 70.74) --
	(289.38, 68.87) --
	(291.52, 67.20) --
	(293.67, 65.76) --
	(295.82, 64.55) --
	(297.97, 63.59) --
	(300.11, 62.84) --
	(302.26, 62.29) --
	(303.53, 62.05);
\end{scope}
\end{tikzpicture}

	}
	\captionvspace
	\caption{The histogram shows the result of the conditional sampling according to \cref{alg:constrained simple}. The red line represents the true \ac{pdf}.}
	\label{fig:constrained simple}
\end{figure}

\begin{figure}
	\figurevspace
	\centering
	\resizebox{\columnwidth}{!}{%
		% Created by tikzDevice version 0.12.3 on 2021-02-01 12:22:52
% !TEX encoding = UTF-8 Unicode
\begin{tikzpicture}[x=1pt,y=1pt]
\definecolor{fillColor}{RGB}{255,255,255}
\path[use as bounding box,fill=fillColor,fill opacity=0.00] (0,0) rectangle (303.53,216.81);
\begin{scope}
\path[clip] (  0.00,  0.00) rectangle (303.53,216.81);
\definecolor{drawColor}{RGB}{0,0,0}

\node[text=drawColor,anchor=base,inner sep=0pt, outer sep=0pt, scale=  1.00] at (163.77, 15.60) {Speed difference [m/s]};

\node[text=drawColor,rotate= 90.00,anchor=base,inner sep=0pt, outer sep=0pt, scale=  1.00] at ( 10.80,114.41) {Density};
\end{scope}
\begin{scope}
\path[clip] (  0.00,  0.00) rectangle (303.53,216.81);
\definecolor{drawColor}{RGB}{0,0,0}

\path[draw=drawColor,line width= 0.4pt,line join=round,line cap=round] ( 49.20, 61.20) -- (263.06, 61.20);

\path[draw=drawColor,line width= 0.4pt,line join=round,line cap=round] ( 49.20, 61.20) -- ( 49.20, 55.20);

\path[draw=drawColor,line width= 0.4pt,line join=round,line cap=round] ( 79.75, 61.20) -- ( 79.75, 55.20);

\path[draw=drawColor,line width= 0.4pt,line join=round,line cap=round] (110.30, 61.20) -- (110.30, 55.20);

\path[draw=drawColor,line width= 0.4pt,line join=round,line cap=round] (140.85, 61.20) -- (140.85, 55.20);

\path[draw=drawColor,line width= 0.4pt,line join=round,line cap=round] (171.40, 61.20) -- (171.40, 55.20);

\path[draw=drawColor,line width= 0.4pt,line join=round,line cap=round] (201.96, 61.20) -- (201.96, 55.20);

\path[draw=drawColor,line width= 0.4pt,line join=round,line cap=round] (232.51, 61.20) -- (232.51, 55.20);

\path[draw=drawColor,line width= 0.4pt,line join=round,line cap=round] (263.06, 61.20) -- (263.06, 55.20);

\node[text=drawColor,anchor=base,inner sep=0pt, outer sep=0pt, scale=  1.00] at ( 49.20, 39.60) {0};

\node[text=drawColor,anchor=base,inner sep=0pt, outer sep=0pt, scale=  1.00] at ( 79.75, 39.60) {2};

\node[text=drawColor,anchor=base,inner sep=0pt, outer sep=0pt, scale=  1.00] at (110.30, 39.60) {4};

\node[text=drawColor,anchor=base,inner sep=0pt, outer sep=0pt, scale=  1.00] at (140.85, 39.60) {6};

\node[text=drawColor,anchor=base,inner sep=0pt, outer sep=0pt, scale=  1.00] at (171.40, 39.60) {8};

\node[text=drawColor,anchor=base,inner sep=0pt, outer sep=0pt, scale=  1.00] at (201.96, 39.60) {10};

\node[text=drawColor,anchor=base,inner sep=0pt, outer sep=0pt, scale=  1.00] at (232.51, 39.60) {12};

\node[text=drawColor,anchor=base,inner sep=0pt, outer sep=0pt, scale=  1.00] at (263.06, 39.60) {14};

\path[draw=drawColor,line width= 0.4pt,line join=round,line cap=round] ( 49.20, 61.20) -- ( 49.20,162.69);

\path[draw=drawColor,line width= 0.4pt,line join=round,line cap=round] ( 49.20, 61.20) -- ( 43.20, 61.20);

\path[draw=drawColor,line width= 0.4pt,line join=round,line cap=round] ( 49.20, 75.70) -- ( 43.20, 75.70);

\path[draw=drawColor,line width= 0.4pt,line join=round,line cap=round] ( 49.20, 90.20) -- ( 43.20, 90.20);

\path[draw=drawColor,line width= 0.4pt,line join=round,line cap=round] ( 49.20,104.70) -- ( 43.20,104.70);

\path[draw=drawColor,line width= 0.4pt,line join=round,line cap=round] ( 49.20,119.19) -- ( 43.20,119.19);

\path[draw=drawColor,line width= 0.4pt,line join=round,line cap=round] ( 49.20,133.69) -- ( 43.20,133.69);

\path[draw=drawColor,line width= 0.4pt,line join=round,line cap=round] ( 49.20,148.19) -- ( 43.20,148.19);

\path[draw=drawColor,line width= 0.4pt,line join=round,line cap=round] ( 49.20,162.69) -- ( 43.20,162.69);

\node[text=drawColor,rotate= 90.00,anchor=base,inner sep=0pt, outer sep=0pt, scale=  1.00] at ( 34.80, 61.20) {0.00};

\node[text=drawColor,rotate= 90.00,anchor=base,inner sep=0pt, outer sep=0pt, scale=  1.00] at ( 34.80, 90.20) {0.04};

\node[text=drawColor,rotate= 90.00,anchor=base,inner sep=0pt, outer sep=0pt, scale=  1.00] at ( 34.80,119.19) {0.08};

\node[text=drawColor,rotate= 90.00,anchor=base,inner sep=0pt, outer sep=0pt, scale=  1.00] at ( 34.80,148.19) {0.12};
\end{scope}
\begin{scope}
\path[clip] ( 49.20, 61.20) rectangle (278.33,167.61);
\definecolor{drawColor}{RGB}{0,0,0}

\path[draw=drawColor,line width= 0.4pt,line join=round,line cap=round] ( -2.74, 61.20) rectangle (  0.32, 61.20);

\path[draw=drawColor,line width= 0.4pt,line join=round,line cap=round] (  0.32, 61.20) rectangle (  3.37, 61.20);

\path[draw=drawColor,line width= 0.4pt,line join=round,line cap=round] (  3.37, 61.20) rectangle (  6.43, 61.20);

\path[draw=drawColor,line width= 0.4pt,line join=round,line cap=round] (  6.43, 61.20) rectangle (  9.48, 61.20);

\path[draw=drawColor,line width= 0.4pt,line join=round,line cap=round] (  9.48, 61.20) rectangle ( 12.54, 61.20);

\path[draw=drawColor,line width= 0.4pt,line join=round,line cap=round] ( 12.54, 61.20) rectangle ( 15.59, 61.20);

\path[draw=drawColor,line width= 0.4pt,line join=round,line cap=round] ( 15.59, 61.20) rectangle ( 18.65, 61.20);

\path[draw=drawColor,line width= 0.4pt,line join=round,line cap=round] ( 18.65, 61.20) rectangle ( 21.70, 61.20);

\path[draw=drawColor,line width= 0.4pt,line join=round,line cap=round] ( 21.70, 61.20) rectangle ( 24.76, 61.20);

\path[draw=drawColor,line width= 0.4pt,line join=round,line cap=round] ( 24.76, 61.20) rectangle ( 27.81, 61.20);

\path[draw=drawColor,line width= 0.4pt,line join=round,line cap=round] ( 27.81, 61.20) rectangle ( 30.87, 61.20);

\path[draw=drawColor,line width= 0.4pt,line join=round,line cap=round] ( 30.87, 61.20) rectangle ( 33.92, 61.20);

\path[draw=drawColor,line width= 0.4pt,line join=round,line cap=round] ( 33.92, 61.20) rectangle ( 36.98, 61.20);

\path[draw=drawColor,line width= 0.4pt,line join=round,line cap=round] ( 36.98, 61.20) rectangle ( 40.03, 61.20);

\path[draw=drawColor,line width= 0.4pt,line join=round,line cap=round] ( 40.03, 61.20) rectangle ( 43.09, 61.21);

\path[draw=drawColor,line width= 0.4pt,line join=round,line cap=round] ( 43.09, 61.20) rectangle ( 46.14, 61.27);

\path[draw=drawColor,line width= 0.4pt,line join=round,line cap=round] ( 46.14, 61.20) rectangle ( 49.20, 61.46);

\path[draw=drawColor,line width= 0.4pt,line join=round,line cap=round] ( 49.20, 61.20) rectangle ( 52.26, 61.83);

\path[draw=drawColor,line width= 0.4pt,line join=round,line cap=round] ( 52.26, 61.20) rectangle ( 55.31, 62.73);

\path[draw=drawColor,line width= 0.4pt,line join=round,line cap=round] ( 55.31, 61.20) rectangle ( 58.37, 64.70);

\path[draw=drawColor,line width= 0.4pt,line join=round,line cap=round] ( 58.37, 61.20) rectangle ( 61.42, 68.79);

\path[draw=drawColor,line width= 0.4pt,line join=round,line cap=round] ( 61.42, 61.20) rectangle ( 64.48, 74.61);

\path[draw=drawColor,line width= 0.4pt,line join=round,line cap=round] ( 64.48, 61.20) rectangle ( 67.53, 83.73);

\path[draw=drawColor,line width= 0.4pt,line join=round,line cap=round] ( 67.53, 61.20) rectangle ( 70.59, 94.80);

\path[draw=drawColor,line width= 0.4pt,line join=round,line cap=round] ( 70.59, 61.20) rectangle ( 73.64,108.87);

\path[draw=drawColor,line width= 0.4pt,line join=round,line cap=round] ( 73.64, 61.20) rectangle ( 76.70,122.74);

\path[draw=drawColor,line width= 0.4pt,line join=round,line cap=round] ( 76.70, 61.20) rectangle ( 79.75,135.21);

\path[draw=drawColor,line width= 0.4pt,line join=round,line cap=round] ( 79.75, 61.20) rectangle ( 82.81,146.85);

\path[draw=drawColor,line width= 0.4pt,line join=round,line cap=round] ( 82.81, 61.20) rectangle ( 85.86,155.48);

\path[draw=drawColor,line width= 0.4pt,line join=round,line cap=round] ( 85.86, 61.20) rectangle ( 88.92,161.34);

\path[draw=drawColor,line width= 0.4pt,line join=round,line cap=round] ( 88.92, 61.20) rectangle ( 91.97,163.85);

\path[draw=drawColor,line width= 0.4pt,line join=round,line cap=round] ( 91.97, 61.20) rectangle ( 95.03,165.21);

\path[draw=drawColor,line width= 0.4pt,line join=round,line cap=round] ( 95.03, 61.20) rectangle ( 98.08,166.24);

\path[draw=drawColor,line width= 0.4pt,line join=round,line cap=round] ( 98.08, 61.20) rectangle (101.14,165.85);

\path[draw=drawColor,line width= 0.4pt,line join=round,line cap=round] (101.14, 61.20) rectangle (104.19,164.74);

\path[draw=drawColor,line width= 0.4pt,line join=round,line cap=round] (104.19, 61.20) rectangle (107.25,160.42);

\path[draw=drawColor,line width= 0.4pt,line join=round,line cap=round] (107.25, 61.20) rectangle (110.30,155.81);

\path[draw=drawColor,line width= 0.4pt,line join=round,line cap=round] (110.30, 61.20) rectangle (113.36,149.35);

\path[draw=drawColor,line width= 0.4pt,line join=round,line cap=round] (113.36, 61.20) rectangle (116.41,145.23);

\path[draw=drawColor,line width= 0.4pt,line join=round,line cap=round] (116.41, 61.20) rectangle (119.47,137.98);

\path[draw=drawColor,line width= 0.4pt,line join=round,line cap=round] (119.47, 61.20) rectangle (122.52,131.73);

\path[draw=drawColor,line width= 0.4pt,line join=round,line cap=round] (122.52, 61.20) rectangle (125.58,123.54);

\path[draw=drawColor,line width= 0.4pt,line join=round,line cap=round] (125.58, 61.20) rectangle (128.63,117.26);

\path[draw=drawColor,line width= 0.4pt,line join=round,line cap=round] (128.63, 61.20) rectangle (131.69,111.30);

\path[draw=drawColor,line width= 0.4pt,line join=round,line cap=round] (131.69, 61.20) rectangle (134.74,108.69);

\path[draw=drawColor,line width= 0.4pt,line join=round,line cap=round] (134.74, 61.20) rectangle (137.80,109.98);

\path[draw=drawColor,line width= 0.4pt,line join=round,line cap=round] (137.80, 61.20) rectangle (140.85,112.01);

\path[draw=drawColor,line width= 0.4pt,line join=round,line cap=round] (140.85, 61.20) rectangle (143.91,119.27);

\path[draw=drawColor,line width= 0.4pt,line join=round,line cap=round] (143.91, 61.20) rectangle (146.96,123.61);

\path[draw=drawColor,line width= 0.4pt,line join=round,line cap=round] (146.96, 61.20) rectangle (150.02,128.48);

\path[draw=drawColor,line width= 0.4pt,line join=round,line cap=round] (150.02, 61.20) rectangle (153.07,128.65);

\path[draw=drawColor,line width= 0.4pt,line join=round,line cap=round] (153.07, 61.20) rectangle (156.13,127.60);

\path[draw=drawColor,line width= 0.4pt,line join=round,line cap=round] (156.13, 61.20) rectangle (159.18,125.14);

\path[draw=drawColor,line width= 0.4pt,line join=round,line cap=round] (159.18, 61.20) rectangle (162.24,124.11);

\path[draw=drawColor,line width= 0.4pt,line join=round,line cap=round] (162.24, 61.20) rectangle (165.29,124.22);

\path[draw=drawColor,line width= 0.4pt,line join=round,line cap=round] (165.29, 61.20) rectangle (168.35,125.20);

\path[draw=drawColor,line width= 0.4pt,line join=round,line cap=round] (168.35, 61.20) rectangle (171.40,124.60);

\path[draw=drawColor,line width= 0.4pt,line join=round,line cap=round] (171.40, 61.20) rectangle (174.46,122.71);

\path[draw=drawColor,line width= 0.4pt,line join=round,line cap=round] (174.46, 61.20) rectangle (177.52,117.32);

\path[draw=drawColor,line width= 0.4pt,line join=round,line cap=round] (177.52, 61.20) rectangle (180.57,110.99);

\path[draw=drawColor,line width= 0.4pt,line join=round,line cap=round] (180.57, 61.20) rectangle (183.63,104.66);

\path[draw=drawColor,line width= 0.4pt,line join=round,line cap=round] (183.63, 61.20) rectangle (186.68, 98.49);

\path[draw=drawColor,line width= 0.4pt,line join=round,line cap=round] (186.68, 61.20) rectangle (189.74, 93.24);

\path[draw=drawColor,line width= 0.4pt,line join=round,line cap=round] (189.74, 61.20) rectangle (192.79, 87.34);

\path[draw=drawColor,line width= 0.4pt,line join=round,line cap=round] (192.79, 61.20) rectangle (195.85, 83.47);

\path[draw=drawColor,line width= 0.4pt,line join=round,line cap=round] (195.85, 61.20) rectangle (198.90, 80.19);

\path[draw=drawColor,line width= 0.4pt,line join=round,line cap=round] (198.90, 61.20) rectangle (201.96, 79.05);

\path[draw=drawColor,line width= 0.4pt,line join=round,line cap=round] (201.96, 61.20) rectangle (205.01, 78.20);

\path[draw=drawColor,line width= 0.4pt,line join=round,line cap=round] (205.01, 61.20) rectangle (208.07, 78.55);

\path[draw=drawColor,line width= 0.4pt,line join=round,line cap=round] (208.07, 61.20) rectangle (211.12, 77.69);

\path[draw=drawColor,line width= 0.4pt,line join=round,line cap=round] (211.12, 61.20) rectangle (214.18, 76.92);

\path[draw=drawColor,line width= 0.4pt,line join=round,line cap=round] (214.18, 61.20) rectangle (217.23, 75.71);

\path[draw=drawColor,line width= 0.4pt,line join=round,line cap=round] (217.23, 61.20) rectangle (220.29, 75.74);

\path[draw=drawColor,line width= 0.4pt,line join=round,line cap=round] (220.29, 61.20) rectangle (223.34, 77.04);

\path[draw=drawColor,line width= 0.4pt,line join=round,line cap=round] (223.34, 61.20) rectangle (226.40, 78.57);

\path[draw=drawColor,line width= 0.4pt,line join=round,line cap=round] (226.40, 61.20) rectangle (229.45, 81.55);

\path[draw=drawColor,line width= 0.4pt,line join=round,line cap=round] (229.45, 61.20) rectangle (232.51, 83.64);

\path[draw=drawColor,line width= 0.4pt,line join=round,line cap=round] (232.51, 61.20) rectangle (235.56, 85.85);

\path[draw=drawColor,line width= 0.4pt,line join=round,line cap=round] (235.56, 61.20) rectangle (238.62, 89.66);

\path[draw=drawColor,line width= 0.4pt,line join=round,line cap=round] (238.62, 61.20) rectangle (241.67, 94.17);

\path[draw=drawColor,line width= 0.4pt,line join=round,line cap=round] (241.67, 61.20) rectangle (244.73, 99.62);

\path[draw=drawColor,line width= 0.4pt,line join=round,line cap=round] (244.73, 61.20) rectangle (247.78,105.33);

\path[draw=drawColor,line width= 0.4pt,line join=round,line cap=round] (247.78, 61.20) rectangle (250.84,108.83);

\path[draw=drawColor,line width= 0.4pt,line join=round,line cap=round] (250.84, 61.20) rectangle (253.89,109.39);

\path[draw=drawColor,line width= 0.4pt,line join=round,line cap=round] (253.89, 61.20) rectangle (256.95,107.23);

\path[draw=drawColor,line width= 0.4pt,line join=round,line cap=round] (256.95, 61.20) rectangle (260.00,103.29);

\path[draw=drawColor,line width= 0.4pt,line join=round,line cap=round] (260.00, 61.20) rectangle (263.06, 98.91);

\path[draw=drawColor,line width= 0.4pt,line join=round,line cap=round] (263.06, 61.20) rectangle (266.11, 93.99);

\path[draw=drawColor,line width= 0.4pt,line join=round,line cap=round] (266.11, 61.20) rectangle (269.17, 88.44);

\path[draw=drawColor,line width= 0.4pt,line join=round,line cap=round] (269.17, 61.20) rectangle (272.22, 84.54);

\path[draw=drawColor,line width= 0.4pt,line join=round,line cap=round] (272.22, 61.20) rectangle (275.28, 80.69);

\path[draw=drawColor,line width= 0.4pt,line join=round,line cap=round] (275.28, 61.20) rectangle (278.33, 77.20);

\path[draw=drawColor,line width= 0.4pt,line join=round,line cap=round] (278.33, 61.20) rectangle (281.39, 73.09);

\path[draw=drawColor,line width= 0.4pt,line join=round,line cap=round] (281.39, 61.20) rectangle (284.44, 70.50);

\path[draw=drawColor,line width= 0.4pt,line join=round,line cap=round] (284.44, 61.20) rectangle (287.50, 67.88);

\path[draw=drawColor,line width= 0.4pt,line join=round,line cap=round] (287.50, 61.20) rectangle (290.55, 66.40);

\path[draw=drawColor,line width= 0.4pt,line join=round,line cap=round] (290.55, 61.20) rectangle (293.61, 64.88);

\path[draw=drawColor,line width= 0.4pt,line join=round,line cap=round] (293.61, 61.20) rectangle (296.66, 63.91);

\path[draw=drawColor,line width= 0.4pt,line join=round,line cap=round] (296.66, 61.20) rectangle (299.72, 63.11);

\path[draw=drawColor,line width= 0.4pt,line join=round,line cap=round] (299.72, 61.20) rectangle (302.77, 62.45);

\path[draw=drawColor,line width= 0.4pt,line join=round,line cap=round] (302.77, 61.20) rectangle (305.83, 62.05);
\definecolor{drawColor}{RGB}{255,0,0}

\path[draw=drawColor,line width= 1.2pt,line join=round,line cap=round] (  0.00, 61.20) --
	(  1.66, 61.20) --
	(  3.80, 61.20) --
	(  5.95, 61.20) --
	(  8.10, 61.20) --
	( 10.24, 61.20) --
	( 12.39, 61.20) --
	( 14.54, 61.20) --
	( 16.69, 61.20) --
	( 18.83, 61.20) --
	( 20.98, 61.20) --
	( 23.13, 61.20) --
	( 25.27, 61.20) --
	( 27.42, 61.20) --
	( 29.57, 61.20) --
	( 31.72, 61.20) --
	( 33.86, 61.20) --
	( 36.01, 61.20) --
	( 38.16, 61.20) --
	( 40.30, 61.21) --
	( 42.45, 61.23) --
	( 44.60, 61.26) --
	( 46.75, 61.35) --
	( 48.89, 61.53) --
	( 51.04, 61.88) --
	( 53.19, 62.52) --
	( 55.33, 63.62) --
	( 57.48, 65.40) --
	( 59.63, 68.11) --
	( 61.78, 71.99) --
	( 63.92, 77.22) --
	( 66.07, 83.88) --
	( 68.22, 91.88) --
	( 70.36,100.95) --
	( 72.51,110.68) --
	( 74.66,120.53) --
	( 76.81,129.99) --
	( 78.95,138.62) --
	( 81.10,146.10) --
	( 83.25,152.29) --
	( 85.40,157.16) --
	( 87.54,160.81) --
	( 89.69,163.40) --
	( 91.84,165.12) --
	( 93.98,166.14) --
	( 96.13,166.56) --
	( 98.28,166.40) --
	(100.43,165.62) --
	(102.57,164.15) --
	(104.72,161.96) --
	(106.87,159.14) --
	(109.01,155.84) --
	(111.16,152.20) --
	(113.31,148.31) --
	(115.46,144.13) --
	(117.60,139.58) --
	(119.75,134.60) --
	(121.90,129.27) --
	(124.04,123.80) --
	(126.19,118.58) --
	(128.34,114.03) --
	(130.49,110.56) --
	(132.63,108.51) --
	(134.78,108.10) --
	(136.93,109.35) --
	(139.07,112.10) --
	(141.22,115.94) --
	(143.37,120.24) --
	(145.52,124.25) --
	(147.66,127.29) --
	(149.81,128.90) --
	(151.96,129.02) --
	(154.10,127.98) --
	(156.25,126.41) --
	(158.40,124.99) --
	(160.55,124.22) --
	(162.69,124.23) --
	(164.84,124.75) --
	(166.99,125.25) --
	(169.13,125.10) --
	(171.28,123.89) --
	(173.43,121.45) --
	(175.58,117.95) --
	(177.72,113.74) --
	(179.87,109.23) --
	(182.02,104.70) --
	(184.17,100.31) --
	(186.31, 96.11) --
	(188.46, 92.13) --
	(190.61, 88.44) --
	(192.75, 85.16) --
	(194.90, 82.47) --
	(197.05, 80.49) --
	(199.20, 79.26) --
	(201.34, 78.66) --
	(203.49, 78.45) --
	(205.64, 78.35) --
	(207.78, 78.12) --
	(209.93, 77.65) --
	(212.08, 76.99) --
	(214.23, 76.34) --
	(216.37, 75.93) --
	(218.52, 75.96) --
	(220.67, 76.51) --
	(222.81, 77.53) --
	(224.96, 78.90) --
	(227.11, 80.46) --
	(229.26, 82.11) --
	(231.40, 83.82) --
	(233.55, 85.70) --
	(235.70, 87.92) --
	(237.84, 90.65) --
	(239.99, 93.96) --
	(242.14, 97.71) --
	(244.29,101.60) --
	(246.43,105.14) --
	(248.58,107.84) --
	(250.73,109.27) --
	(252.87,109.26) --
	(255.02,107.85) --
	(257.17,105.33) --
	(259.32,102.11) --
	(261.46, 98.60) --
	(263.61, 95.11) --
	(265.76, 91.78) --
	(267.90, 88.65) --
	(270.05, 85.67) --
	(272.20, 82.78) --
	(274.35, 79.96) --
	(276.49, 77.24) --
	(278.64, 74.70) --
	(280.79, 72.40) --
	(282.94, 70.40) --
	(285.08, 68.70) --
	(287.23, 67.30) --
	(289.38, 66.15) --
	(291.52, 65.20) --
	(293.67, 64.42) --
	(295.82, 63.76) --
	(297.97, 63.20) --
	(300.11, 62.72) --
	(302.26, 62.32) --
	(303.53, 62.12);
\end{scope}
\end{tikzpicture}

	}
	\captionvspace
	\caption{The histogram shows the result of the conditional sampling according to \cref{alg:constrained hard}. The red line represents the true \ac{pdf}.}
	\label{fig:constrained hard}
\end{figure}

\section{Conclusions}
\label{sec:conclusions}

% Safety is important
Road safety is an important research topic because of the societal and economical losses caused by accidents.
% Quantify safety with surrogate metrics
To quantify the safety at a vehicle level, use is made of \acp{ssm} that characterize the risk of a collision. 
% What we did
We have proposed a method for deriving \iac{ssm} that calculates the probability that a certain event, e.g., a collision, will happen in the near future.
% Advantages of our method
With our data-driven approach, it is possible to adapt the \ac{ssm} to the local traffic context.
Besides, the presented method could be applied to various types of scenarios.
We have illustrated that our method is a generalization of already existing \acp{ssm}.
In an example, we have derived \iac{ssm} based on the \ac{ngsim} data set that calculates the risk of a collision in a longitudinal interaction between two vehicles.
Through few explanatory scenarios, it has been shown that the derived \ac{ssm} provides a quantification of the collision risk.
We have also presented how the evaluation of the partial derivatives of the \ac{ssm} can be used to benchmark \iac{ssm} with few expected causal tendencies.

%Concluding remarks about our method
Our proposed method has the potential for deriving multiple \acp{ssm} for quantifying the safety of a --- possibly automated --- driver.
These metrics can be used to warn drivers for unsafe situations and ensuring that proper attention is being paid to the road situation.
Furthermore, the metrics can measure the impact of newly introduced systems on the traffic safety.
% Future work
A limitation of the current study is that the presented approach is only applied to longitudinal traffic interactions. 
Future work involves the consideration of lateral traffic interactions and interactions with vulnerable road users. 


%\section*{Acknowledgement}
%Tux also likes to thank the Free Software Foundation for their GNU software.



\printbibliography

\end{document}
