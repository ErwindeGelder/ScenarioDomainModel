\section{Introduction}
\label{sec:introduction}

% Introduce scenario-based testing
An essential facet in the development of \acp{av} is the assessment of quality and performance aspects of the \acp{av}, such as safety, comfort, and efficiency \autocite{bengler2014threedecades, stellet2015taxonomy, koopman2016challenges}. 
Because naturalistic field operational tests are expensive and time consuming \autocite{kalra2016driving, zhao2018evaluation}, a scenario-based approach has been proposed \autocite{riedmaier2020survey}.
As a source of information for the scenarios for the assessment, real-world driving data has been proposed, such that the scenarios relate to real-world driving conditions \autocite{elrofai2018scenario, putz2017pegasus, krajewski2018highD}.

% List possible methods for generating scenarios
When using scenarios extracted from real-world driving data as a direct source for the test scenarios, two problems arise.
First, not all possible variations of the scenarios might be found in the data. 
Therefore, the failure modes of the \acp{av} might not be reflected in the existing scenarios \autocite{zhao2018evaluation}.
Second, this might not reduce the actual testing load, because the extracted scenarios is largely composed of non-safety critical scenarios \autocite{zhao2018evaluation}.
As a solution to this, so-called importance sampling has been introduced in order to put more emphasis on safety-critical scenarios \autocite{deGelder2017assessment, xu2018accelerated, zhao2018evaluation, jesenski2020scalable}.
These methods \autocite{deGelder2017assessment, xu2018accelerated, zhao2018evaluation, jesenski2020scalable} have in common that they describe scenarios using parameters for which \iac{pdf} is estimated.

% Out solution
As already proposed in \autocite{deGelder2017assessment}, we propose to estimate the \ac{pdf} using \ac{kde}.
\ac{kde} \autocite{parzen1962estimation, rosenblatt1956remarks} is often referred to as a non-parametric way to estimate the \ac{pdf}, because no use is made of a predefined \ac{pdf} for which certain parameters are fitted to the data. 
This makes \ac{kde} flexible regarding the actual underlying distribution of the parameters.
Sampling from \iac{kde} is straightforward.
However, in some cases, one wants to sample from the estimated \ac{pdf} while a part of the random sample is fixed.
This is known as conditional sampling. 
For example, suppose we have a two dimensional estimated distribution and we want to draw samples from this distribution while the first of these two variables is equal to a predefined value. 
One approach would be to evaluate the conditional \ac{pdf} and to use this for sampling.
This method, however, would be highly cumbersome, especially with higher dimensional \acp{pdf}. 

% Contribution of this paper (conditional sampling and sampling with linear constraints)
We will propose algorithms to sample parameters from \iac{kde} while either one or more parameters are fixed or the parameters are subjected to linear constraints.
Conditional sampling with one or more variables fixed easily follows from estimation techniques for conditional \acp{pdf} \autocite{hyndman1996estimating, holmes2007fast}
In this paper, we go one step further: We will also describe a method for sampling from \iac{kde} while the variables satisfy linear equality constraints. 
In our proposed method, the actual (conditional) \ac{pdf} does not need to be evaluated, which would make it inefficient.
In our case, the computational costs scales approximately quadratically with respect to the dimension and linearly the number of data points. 
If more samples are to be generated with the same linear constraints, most computations need to be done only once, so this reduces the computational load per generated sample.
We illustrate the proposed sampling techniques and its practical usefulness using an example.
Furthermore, we will explain the usefulness of sampling with linear constraints in case parameter reduction techniques are used to avoid the curse of dimensionality that \ac{pdf} estimation techniques, such as \ac{kde}, are subjected to.

% Structure of the paper
In \cref{sec:problem}, we first describe the problem in more detail, which results in four sub-problems.
In \cref{sec:method}, we provide four algorithms to solve each of these sub-problems.
Through an example, we illustrate the correct performance of these algorithms in \cref{sec:example}.
In \cref{sec:discussion}, we will discuss the scalability of the algorithms and a possible useful application of the algorithms. 
We conclude this paper in \cref{sec:conclusions}.
