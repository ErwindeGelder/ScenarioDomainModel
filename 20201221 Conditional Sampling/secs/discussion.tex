\section{Discussion}
\label{sec:discussion}

% Explain how it scales with d and N
In the introduction, we have mentioned that the computational costs scales quadratically with respect to the dimension and linearly with the number of data points.
Because the number of data points is generally much larger than the dimension of the data points, let us assume that $\numberofsamples \gg \dimension$.
Looking at \cref{alg:constrained hard} and considering $\numberofsamples \gg \dimension$, steps 2, 3, and 6 are most time consuming, because these steps contain a loop over the data points. 
It is easy to see that the computations of these steps scale linearly with $\numberofsamples$. 
Since these computations contain a matrix-vector multiplication, the computational cost scales quadratically with $\dimension$. 

% Explain how it scales if we don't change the constraint
Note that if we want to sample multiple times using the same linear constraint, it suffices to do the steps 1 till 6 of \cref{alg:constrained hard} only once.
Step 7 of \cref{alg:constrained hard} does not depend on $\dimension$ and scales only linearly with $\numberofsamples$ \autocite{vose1991linear}.
Steps 8 till 10 of \cref{alg:constrained hard} do not depend on $\numberofsamples$ and scale quadratically with $\dimension$.
Because these steps do not depend on $\numberofsamples$ and $\numberofsamples \gg \dimension$, the computational time of these steps are minor compared to step 7 of \cref{alg:constrained hard}. 

% Explain possible use of our method
Our proposed methods in \cref{alg:constrained simple,alg:constrained hard} for sampling with linear constraints become especially useful if a parameter reduction technique like \ac{svd} or \ac{pca} \autocite{abdi2010principal} is used.
To avoid the curse of dimensionality when estimating the \ac{pdf} \autocite{scott1992multivariate}, the number of parameters can be reduced using \iac{svd} or \iac{pca}.
This will result in a linear mapping of the parameters.
As a result, if parameters are to be sampled with one or more of the original parameters fixed at a predetermined value, this would give a linear constraint on the new parameters after the reduction.
Thus, whereas \cref{alg:conditional simple,alg:conditional hard} could be used if \iac{kde} is used to estimated the \ac{pdf} of the original parameters, \cref{alg:constrained simple,alg:constrained hard}, respectively, should be used if the \ac{kde} is constructed for the reduced set of parameters.
Since scenarios for the assessment of \acp{av} are likely to be more complex than the ones shown in the example in \cref{sec:example}, it is conceivable that more parameters are to be used and, consequently, that a reduction technique using \iac{svd} or \iac{pca} will become useful.
