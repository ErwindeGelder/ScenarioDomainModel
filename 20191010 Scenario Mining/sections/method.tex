\section{Method}
\label{sec:method}



\subsection{Activity detection}
\label{sec:activity detection}

Here, we explain which activities we detect and how we detect activities. We distinguish between longitudinal activities and lateral activities and we distinguish between activities of the ego vehicle and other vehicles.

\subsubsection{Longitudinal activities of ego vehicle}
\label{sec:longitudinal ego}

We distinguish between three different longitudinal activities: ``accelerating'', ``decelerating'', and ``cruising''. The ego vehicle is performing either one of these activities. According to \cref{def:activity}, an activity starts and ends with an event and, therefore, we call the event at the start of the activity ``accelerating'' an ``accelerating event'', and so on. To extract the activities, we first extract the events. Hence, the activity ``accelerating'' starts when an ``accelerating event'' occurs. The activity ``accelerating'' ends as soon as either a ``decelerating event'' or a ``cruising event'' occurs.

To extract the longitudinal events from the data, let $\speed{\time}$ denote the speed of the ego vehicle at time $\time$. Next, let us define the minimum and maximum speed between the current time $\time$ and $\time+\timehorizon$, where $\timehorizon>0$ is a parameter:
\begin{align}
	\speedmin{\time} &\defsym \min_{\time \leq \timedummy \leq \time+\timehorizon} \speed{\timedummy}, \\
	\speedmax{\time} &\defsym \max_{\time \leq \timedummy \leq \time+\timehorizon} \speed{\timedummy}.
\end{align}
For detecting accelerating and decelerating events, the maximum speed increase $\speedinc{\time}$ and the minimum speed decrease $\speedinc{\time}$ within the same time window are used:
\begin{align}
	\speedinc{\time} &\defsym \speed{\time + \timehorizon} - \speedmin{\time},\\
	\speeddec{\time} &\defsym \speed{\time + \timehorizon} - \speedmax{\time}.
\end{align}

First, we assume that the event at the start of the dataset is a cruising event at $\time=\timeinit$. Next, we go chronologically through the dataset. A accelerating event is happening at time $\time$ if any of the following conditions is true:
\begin{itemize}
	\item The vehicle is not performing an accelerating activity, i.e., the last event is not an accelerating event.
	\item There is a significant acceleration, i.e., 
	\begin{equation}
		\speedinc{\time} \geq \accelerationstart / \timehorizon,
	\end{equation}
	where $\accelerationstart>0$ is a parameter.
	\item There is no lower speed in the near future, i.e., $\speedmin{\time}=\speed{\time}$.
	\item There is a significant speed difference during the activity, i.e., 
	\begin{equation}
		\speed{\timeendinc{\time}} - \speed{\time} > \speeddiff,
	\end{equation}
	where $\timeendinc{\time}$ is controlled by the parameter $\accelerationcruise$ and defined as follows:
	\begin{equation}
		\timeendinc{\time} \defsym \arg \min_{\timedummy > \time} \left\{ \timedummy: \speedinc{\timedummy} < \accelerationcruise/\timehorizon \right\}.
	\end{equation}
\end{itemize}

A decelerating event is happening at time $\time$ if any of the following conditions is true:
\begin{itemize}
	\item The vehicle is not performing a decelerating activity, i.e., the last event is not an decelerating event.
	\item There is a significant deceleration, i.e., 
	\begin{equation}
		\speeddec{\time} \leq -\accelerationstart / \timehorizon.
	\end{equation}
	\item There is no lower speed in the near future, i.e., $\speedmax{\time}=\speed{\time}$.
	\item There is a significant speed difference during the activity, i.e., 
	\begin{equation}
		\speed{\timeenddec{\time}} - \speed{\time} < -\speeddiff,
	\end{equation}
	with
	\begin{equation}
		\timeenddec{\time} \defsym \arg \min_{\timedummy > \time} \left\{ \timedummy: \speeddec{\timedummy} > -\accelerationcruise/\timehorizon \right\}.
	\end{equation}
\end{itemize}

Let $\timeaccevent$ and $\timedecevent$ denote the time of the most recent accelerating and decelerating event, respectively. A cruising event happens at time $\time$ if the the vehicle is not performing a cruising event and if any of the following conditions is true:
\begin{itemize}
	\item $\time > \timeendinc{\timeaccevent}$ while the vehicle is performing an accelerating activity.
	\item $\time > \timeenddec{\timedecevent}$ while the vehicle is performing an decelerating activity.
\end{itemize}