\documentclass[journal]{IEEEtran}


% make changes take effect
\pagestyle{headings}
% adjust as needed
\addtolength{\footskip}{0\baselineskip}
\addtolength{\textheight}{-1\baselineskip}




%\documentclass[a4paper, 10pt, conference]{ieeeconf}      % Use this line for a4 paper

%\IEEEoverridecommandlockouts                              % This command is only needed if 
                                                          % you want to use the \thanks command

% See the \addtolength command later in the file to balance the column lengths
% on the last page of the document

% The following packages can be found on http:\\www.ctan.org
\usepackage{graphicx} % for pdf, bitmapped graphics files
\usepackage{epstopdf}
%\usepackage{epsfig} % for postscript graphics files
%\usepackage{mathptmx} % assumes new font selection scheme installed
%\usepackage{times} % assumes new font selection scheme installed
\let\proof\relax
\let\endproof\relax
\usepackage{amsmath} % assumes amsmath package installed
\usepackage{amsthm}  % For special theorem style
\usepackage{amsfonts}
%\usepackage{breqn}
%\usepackage{amssymb}  % assumes amsmath package installed
\usepackage[dvipsnames]{xcolor}
\usepackage{tikz}
\usepackage{pgfplots}
\usetikzlibrary{shapes,arrows}
\usetikzlibrary{backgrounds}
\usepackage{multirow}
\usepackage[keeplastbox]{flushend}
%\usepackage{cite}  % Make sure that citation appear as [1]-[3] instead of [1], [2], [3]
\usepackage[utf8]{inputenc}    % utf8 support
\usepackage[T1]{fontenc}       % code for pdf file
\usepackage[USenglish]{babel}  % language support
%\usepackage{authblk}


\usepackage[utf8]{inputenc}   				 	%% utf8 support (required for biblatex)
\usepackage{silence}  							%% For filtering warnings
\usepackage[style=ieee,doi=false,isbn=false,url=false,date=year,backend=biber,maxbibnames=15,maxcitenames=2,mincitenames=1,uniquelist=false,uniquename=false,giveninits=true]{biblatex}
% Filter warnings issued by package biblatex starting with "Patching footnotes failed"
\WarningFilter{biblatex}{Patching footnotes failed}
\renewcommand*{\bibfont}{\footnotesize}		%% Use this for papers
\renewcommand*{\bibfont}{\small}
\setlength{\biblabelsep}{\labelsep}
\bibliography{../bib}


%\usepackage{url}
%\usepackage{natbib}
%\usepackage{letltxmacro}
%\LetLtxMacro{\autocite}{\citep}
%\LetLtxMacro{\textcite}{\citet}


% Table stuff
\usepackage{booktabs}
\usepackage{tabularx}
\setlength{\heavyrulewidth}{0.1em}
\newcommand{\otoprule}{\midrule[\heavyrulewidth]}



\usepackage{pgfplots}
\pgfplotsset{compat=1.9}  % Prevent warning, pgf running in backwards compatibility mode anyway%\usetikzlibrary{external}                       %% Create pdf figures from TikZ. Use PDFTeXify ...
%\tikzexternalize[prefix=tikz/]                  %% ... with --tex-option=--shell-escape switch.
\usepackage[capitalize]{cleveref}
\Crefname{figure}{Figure}{Figures}
\crefname{equation}{}{}
\Crefname{equation}{Equation}{Equations}
\usepackage{subcaption}
\usepackage{xstring}
\usepackage{xparse}
\usepackage{siunitx}
\usepackage[nolist,nohyperlinks]{acronym}
\makeatletter
\newcommand*{\org@overidelabel}{}
\let\org@overridelabel\@verridelabel
\@ifpackagelater{acronym}{2015/03/21}{% v1.41
  \renewcommand*{\@verridelabel}[1]{%
    \@bsphack
    \protected@write\@auxout{}{\string\AC@undonewlabel{#1@cref}}%
    \org@overridelabel{#1}%
    \@esphack
  }%
}{% older versions
  \renewcommand*{\@verridelabel}[1]{%
    \@bsphack
    \protected@write\@auxout{}{\string\undonewlabel{#1@cref}}%
    \org@overridelabel{#1}%
    \@esphack
  }%
}
\makeatother

\theoremstyle{plain}
\newtheorem{definition}{Definition}
\theoremstyle{remark}\newtheorem{remarkenv}{Remark}        %% remarks
\newenvironment{remark}{\begin{remarkenv}}%
	{\hfill$\lozenge$\end{remarkenv}}            %% end remark with a lozenge

%\pgfplotsset{compat=newest} 
%\pgfplotsset{plot coordinates/math parser=false}

%Images path 
\graphicspath{ {figures/} }


\newlength\figurewidth
\newlength\figureheight
\newlength\venncircle
\newlength\objectwidth\setlength{\objectwidth}{16.2em}


\usetikzlibrary{arrows,positioning}
\usetikzlibrary{arrows.meta}
\definecolor{TNOlightgray}{RGB}{222,222,231}
\definecolor{abstractclass}{RGB}{255, 255, 200}
%\definecolor{scenariocategory}{RGB}{137, 197, 255}
\definecolor{scenariocategory}{RGB}{121, 221, 255}
\definecolor{category}{RGB}{121, 221, 255}
%\definecolor{scenario}{RGB}{255, 157, 121}
\definecolor{scenario}{RGB}{255, 188, 163}
\definecolor{otherclass}{RGB}{255, 188, 163}
\tikzstyle{tag}=[text height=.8em, text depth=.1em, font=\small\sffamily, rounded corners=0.2em, fill=TNOlightgray, node distance=9em, text width=7em, align=center]
\tikzstyle{tag wide}=[tag, text width=12em]
\tikzstyle{tagarrow}=[->, line width=0.75mm, color=TNOlightgray]
\tikzstyle{object}=[draw, text width=\objectwidth-.5em, align=left, line width=1pt, minimum width=\objectwidth, anchor=north west, node distance=3pt]
\newcounter{tagbcounter}
\newcounter{tagccounter}
\newcommand{\taga}[1]{
	\node[tag wide](taga){#1};
	\node[coordinate, below of=taga, node distance=1.2em](helper){};
}
\newcommand{\tagb}[3]{
	\node[tag, below of=taga, node distance=3em, xshift=#3](tagb#2){#1};
	\draw[tagarrow] (taga) -- (helper) -| (tagb#2);
	\node[coordinate, xshift=1em](tagb#2 helper) at (tagb#2.south west) {};
}
\newcommand{\tagc}[4]{
	\node[tag, below of=tagb#4, node distance=#3em, xshift=2em](tagc#2){#1};
	\draw[tagarrow] (tagb#4 helper) |- (tagc#2);
}
\ExplSyntaxOn
\NewDocumentCommand{\tree}{O{} m m}{%
	\begin{tikzpicture}
	\taga{#2}
	\setcounter{tagbcounter}{0}
	\seq_set_split:Nnn \arg { ; } { #3 }
	\seq_map_inline:Nn \arg {
		\seq_set_split:Nnn \argb {,} {##1}
		\seq_pop_left:NN \argb \argl
		\tagb{\argl}{\arabic{tagbcounter}}{\arabic{tagbcounter}*10em-\seq_count:N \arg *5em+5em}
		\setcounter{tagccounter}{0}
		\seq_map_inline:Nn \argb {
			\stepcounter{tagccounter}
			\tagc{####1}{\arabic{tagccounter}}{2*\arabic{tagccounter}}{\arabic{tagbcounter}}
		}
		\stepcounter{tagbcounter}
	}
	#1
	\end{tikzpicture}
}
\ExplSyntaxOff

% Needed for the class diagrams
\newlength\blockwidth
\newlength\blockheight
\newlength\blockx
\newlength\blocky
\newlength\legendwidth
\setlength{\blockwidth}{6.3em}
\setlength{\blockheight}{2.5em}
\setlength{\blockx}{8em}
\setlength{\blocky}{-5.5em}
\setlength{\legendwidth}{3.5em}
\tikzstyle{class}=[draw, text width=\blockwidth-.5em, align=center, minimum height=\blockheight, line width=1pt, minimum width=\blockwidth]
\tikzstyle{aggregation}=[-{Diamond[width=8pt, length=12pt, fill=white]}, line width=1pt]
\tikzstyle{falls into}=[->, line width=1pt]
\tikzstyle{superclass}=[-{Triangle[width=8pt, length=12pt, fill=white]}, line width=1pt]

% Notations
\newcommand{\amplitude}{A}
\newcommand{\attrtstart}{start event}
\newcommand{\attrtend}{end event}
\newcommand{\comprises}{\ni}
\newcommand{\dimensionstate}{n}
\newcommand{\distancecondition}{d_{\mathrm{v,p}}}
\newcommand{\duration}{T}
\newcommand{\east}{x}
\newcommand{\egosub}{ego}
\newcommand{\egoeast}{\east_{\mathrm{\egosub}}}
\newcommand{\egonorth}{\north_{\mathrm{\egosub}}}
\newcommand{\egospeed}{v_{\mathrm{\egosub}}}
\newcommand{\egoheading}{\head_{\mathrm{\egosub}}}
\newcommand{\function}{f}
\newcommand{\hasone}{$1$}
\newcommand{\hastwo}{$2$}
\newcommand{\hasn}{$0,1,\ldots$}
\newcommand{\head}{\phi}
\newcommand{\includes}{\supseteq}
\newcommand{\inputsystem}{u}
\newcommand{\north}{y}
\newcommand{\origin}{O}
\newcommand{\parameter}{\theta}
\newcommand{\parametera}{a}
\newcommand{\parameterb}{b}
\newcommand{\pedsub}{ped}
\newcommand{\pedeast}{\east_{\mathrm{\pedsub}}}
\newcommand{\pednorth}{\north_{\mathrm{\pedsub}}}
\newcommand{\pedheading}{\head_{\mathrm{\pedsub}}}
\newcommand{\scenario}{S}
\newcommand{\scenarioa}{\scenario_{1}}
\newcommand{\scenariob}{\scenario_{2}}
\newcommand{\scenarioc}{\scenario_{3}}
\newcommand{\scenariocategory}{\mathcal{C}}
\newcommand{\scenariocategorya}{\scenariocategory_{1}}
\newcommand{\scenariocategoryb}{\scenariocategory_{2}}
\newcommand{\scenariocategoryc}{\scenariocategory_{3}}
\newcommand{\slope}{s}
\newcommand{\state}{z}
\newcommand{\statedot}{\dot{\state}}
\newcommand{\stateinit}{\state_{0}}
\renewcommand{\time}{t}
\newcommand{\timeinit}{\time_{0}}

% Acronyms
\begin{acronym}[AAAAAAAA]
	\acro{av}[AV]{Automated Vehicle}
	\acroindefinite{av}{an}{an}
	\acro{odd}[ODD]{Operational Design Domain}
	\acroindefinite{odd}{an}{an}
	\acro{oof}[OOF]{Object-Oriented Framework}
	\acroindefinite{oof}{an}{an}
\end{acronym}

%\usepackage{changebar}
\newcommand{\cstart}{}  % Revision 2 changes (from 11 Jan 2020)
\newcommand{\cend}{}
\newcommand{\cstartb}{}  % Changes since July 14, 2020
\newcommand{\cendb}{}
\newcommand{\cstartc}{}  % Changes since August 15, 2020
\newcommand{\cendc}{}
\newcommand{\cstartd}{}  % Changes since September 22, 2020
\newcommand{\cendd}{}
\newcommand{\cstarte}{\color{red}}  % Changes since September 30, 2020
\newcommand{\cende}{\color{black}}
%\newcommand{\todo}[1]{\color{red}TODO: #1\color{black}}

%\usepackage{setspace}
%\doublespacing


\begin{document}
\date{}


%\thispagestyle{empty}
%\pagestyle{empty}
\title{\cstartd Towards an Ontology for the Scenario Definition for the Assessment of Automated Vehicles: An Object-Oriented Framework \cendd}

\author{Erwin~de~Gelder$^{1,2,*}$,
	    Jan-Pieter~Paardekooper$^{1,3}$,
	    Arash~Khabbaz~Saberi$^{1}$,
	    Hala~Elrofai$^{1}$,
	    Olaf~Op~den~Camp$^{1}$,
	    Steven~Kraines$^{4}$,
	    Jeroen~Ploeg$^{5,6}$,
	    Bart~De~Schutter$^{2}$%
\thanks{$^1$ TNO, Dept.\ of Integrated Vehicle Safety, Helmond, The Netherlands}%
\thanks{$^2$ Delft University of Technology, Delft Center for Systems and Control, Delft, The Netherlands}%
\thanks{$^3$ Radboud University, Donders Institute for Brain, Cognition and Behaviour, Nijmegen, The Netherlands}%
\thanks{$^4$ Symphony Co.\ Tokyo, Japan}%
\thanks{$^5$ 2getthere, Utrecht, The Netherlands}%
\thanks{$^6$ Eindhoven University of Technology, Dept.\ of Mechanical Engineering Dynamics and Control group, Eindhoven, The Netherlands}%
\thanks{$^*$ Corresponding author: \textit{erwin.degelder@tno.nl}}}%


%%%%%%%%%%%%%%%%%%%%%%%%%%%%%%%%%%%%%%%%%%%%%%%%%%%%%%%%%%%%%%%%%%%%%%%%%%%%%%%%
\maketitle
\begin{abstract}
	\todo{Write abstract}
\end{abstract}
\acresetall

%%%%%%%%%%%%%%%%%%%%%%%%%%%%%%%%%%%%%%%%%%%%%%%%%%%%%%%%%%%%%%%%%%%%%%%%%%%%%%%%
\section{Introduction}
\label{sec:introduction}

TODO

\color{red}
For Arash and Hala: please use \\{\tt \textbackslash cref\{sec:introduction\}}\\ to refer to sections, figures etc. It automatically adds things as `Section': e.g., \cref{sec:introduction}.
\color{black}

The introduction contains: 
\begin{itemize}
	\item Setting up the context of automated driving
	\item Giving some background information about the hazard analysis and risk assessment in the ISO~26262 standard
	\item The gap (the problem) We currently have this as a separate chapter, we could move it to introduction depending on the length. 
	\item our contribution. 
	\item paper structure
\end{itemize}
\section{Problem formulation}
\label{sec:problem}

To formulate the scenario mining problem, we distinguish quantitative scenarios from qualitative scenarios, using the definitions of \emph{scenario} and \emph{scenario category} of \autocite{degelder2018ontology}:

\begin{definition}[Scenario]
	\label{def:scenario}
	A scenario is a quantitative description of the relevant characteristics of the ego vehicle, its activities and/or goals, its static environment, and its dynamic environment. In addition, a scenario contains all events that are relevant to the ego vehicle.
\end{definition}

\begin{definition}[Scenario category \autocite{degelder2018ontology}]
	\label{def:scenario category}
	A scenario category is a qualitative description of the ego vehicle, its activities and/or goals, its static environment, and its dynamic environment.
\end{definition}

\cstarta
A scenario category is an abstraction of a scenario and, therefore, a scenario category comprises multiple scenarios \autocite{degelder2018ontology}.
For example, the scenario category ``cut in'' comprises all possible cut-in scenarios. 
Given such a scenario category, our goal is to find all corresponding scenarios in a given data set. 
Hence, we define the scenario mining problem as follows:
\begin{problem}[Scenario mining]
	Given a scenario category, how to find all scenarios that correspond to this scenario category in a given data set?
\end{problem}
\cenda


\section{Nomenclature} % Erwin (use the other paper) & Arash
\label{sec:definitions}

TODO: give more examples
First, we introduce the following notions:
\begin{enumerate}
\item Risk = Severity $\times$ Probability
\item Severity: a quantitative measure, depending on the impact velocity.
\item Condition:
\item Actor: the relevant actors here are the lead vehicle (denoted with L), the following vehicle (F) and the communication unit (V2V)
\item Activity: the relevant actions here are braking, and failure.
\item Scenario:
\end{enumerate}

\cstartb
\section{Object-oriented framework for scenarios}
\label{sec:oo framework}
\cendb

We have already explained the use of an \cstartb object-oriented framework \cendb in \cref{sec:why oo framework}. In this section, we present our \cstartb object-oriented framework \cendb for scenarios for the assessment of AVs. 
The \cstartb overview of the framework \cendb is formally represented through \cstartb class diagrams \cendb that are briefly presented in \cref{sec:class diagram}. Next, in \cref{sec:domain scenario category}, we explain how a scenario category is formally represented \cstartb in our framework\cendb. Similarly, in \cref{sec:domain scenario}, we describe how a scenario is formally represented. 



\cstartb
\subsection{Class diagrams}\cendb
\label{sec:class diagram}

In \cref{fig:class overview,fig:class relations}, the blue blocks represent the classes that are used to describe a scenario category according to \cref{def:scenario category} and the orange blocks represent the classes that are used to describe a scenario according to \cref{def:scenario}. \cstartb The yellow blocks represent so-called abstract classes. Abstract classes cannot be instantiated. \cendb

\cstartd \Cref{fig:class overview} shows the class-level relationships while \cref{fig:class relations} shows the instance-level relationships.
In \cref{fig:class overview}, the arrow from, e.g., \textit{scenario} to \textit{time interval}, denotes that the class \textit{scenario} is a subclass of the class \textit{time interval}. Therefore, all properties of the class \textit{time interval} are inherited by the class \textit{scenario}. 
The arrow with the diamond in \cref{fig:class relations} denotes an aggregation\footnote{This is typically implemented using pointers. For example, an aggregation arrow from \textit{B} to \textit{A} means that an object of class \textit{A} contains a pointer to an object of class \textit{B}.}.
This means that, e.g., an \textit{actor} object, which is an instance of the \textit{actor} class, has an \textit{actor category} object as an attribute. \cendd
Here, the ``\hasone'' at the start of the arrow from \textit{actor category} to \textit{actor} indicates that an \textit{actor} has \cstartd exactly \cendd one \textit{actor category}.
\cstartb Similarly, ``\hastwo'' at the aggregation arrow from \textit{event} to \textit{time interval} indicates that a \textit{time interval} contains two \textit{events}, i.e., the events that define the start and the end of the time interval. \cendb 
A ``\hasn'' at the start of an aggregation arrow indicates that an object has zero, one, or multiple objects of the corresponding class.
The arrow with the text ``comprises'' and ``includes'' represent methods that are explained in \cref{sec:scenario category}. Here, ``comprises'' can be denoted by $\comprises$ and ``includes'' can be denoted by $\includes$, see \cref{eq:scenario category include}. 

\begin{figure*}[t]
	\centering
	\begin{tikzpicture}
\tikzstyle{every node}=[font=\footnotesize]

% Top classes
\node[class, fill=abstractclass](thing) at (0, 0) {Thing};
\node[class, fill=abstractclass](qualitative) at (-\blockx, 0) {Qualitative thing};
\node[class, fill=abstractclass](quantitative) at (\blockx, 0) {Quantitative thing};
\node[class, fill=abstractclass](physicalcat) at (-1.5\blockx, 2\blocky) {Physical thing category};
\node[class, fill=abstractclass](timeinterval) at (.5\blockx, \blocky) {Time interval};
\node[class, fill=abstractclass](physical) at (1.5\blockx, 2\blocky) {Physical thing};

% Qualitative classes
\node[class, fill=scenariocategory](scencat) at (-1.5\blockx, \blocky) {Scenario category};
\node[class, fill=category](activitycat) at (-.5\blockx, \blocky) {Activity category};
\node[class, fill=category](dynamiccat) at (-1.5\blockx, 3\blocky) {Dynamic physical thing category};
\node[class, fill=category](staticcat) at (-.5\blockx, 3\blocky) {Static physical thing category};
\node[class, fill=category](actorcat) at (-1.5\blockx, 4\blocky) {Actor category};

% Quantitative classes
\node[class, fill=scenario](scenario) at (-.5\blockx, 2\blocky) {Scenario};
\node[class, fill=otherclass](activity) at (.5\blockx, 2\blocky) {Activity};
\node[class, fill=otherclass](event) at (1.5\blockx, \blocky) {Event};
\node[class, fill=otherclass](dynamic) at (1.5\blockx, 3\blocky) {Dynamic physical thing};
\node[class, fill=otherclass](actor) at (1.5\blockx, 4\blocky) {Actor};
\node[class, fill=otherclass](static) at (.5\blockx, 3\blocky) {Static physical thing};

% Superclass arrows
\foreach \fromclass/\toclass in {scencat/qualitative,
								 activitycat/qualitative,
								 physicalcat/qualitative,
								 staticcat/physicalcat,
								 timeinterval/quantitative,
								 scenario/timeinterval,
								 event/quantitative,
								 physical/quantitative,
								 static/physical} {
	\node[coordinate, above of=\fromclass, node distance=-.4\blocky](helper){};
	\draw[superclass] (\fromclass) -- (helper) -| (\toclass);
}
\foreach \fromclass/\toclass in {qualitative/thing, 
								 quantitative/thing,
								 dynamiccat/physicalcat,
								 actorcat/dynamiccat,
								 activity/timeinterval,
								 dynamic/physical,
								 actor/dynamic} {
	\draw[superclass] (\fromclass) -- (\toclass);
}

\end{tikzpicture}
	\caption{Schematic overview of most classes of our object-oriented framework.}
	\label{fig:class overview}
\end{figure*}

\begin{figure*}[t]
	\centering
	\setlength{\blockwidth}{7.1em}
\begin{tikzpicture}
\tikzstyle{every node}=[font=\footnotesize]

% Qualitative classes
\node[class, fill=scenariocategory](scenario category) at (.5\blockx,0) {Scenario category};
\node[class, fill=category](actorcategory) at (\blockx, \blocky) {Actor category};
\node[class, fill=category](activitycategory) at (2\blockx, \blocky) {Activity category};
\node[class, fill=abstractclass](model) at (1.5\blockx+0.25\blockwidth, 2\blocky) {\textit{Model}};
\node[class, fill=category](elementcategory) at (3\blockx, \blocky) {Physical element category};

% Quantitative classes
\node[class, fill=scenario](scenario) at (.5\blockx, 2\blocky) {Scenario};
\node[class, fill=otherclass](actor) at (\blockx, 3\blocky) {Actor};
\node[class, fill=otherclass](activity) at (2\blockx, 3\blocky) {Activity};
\node[class, fill=otherclass](element) at (3\blockx, 3\blocky) {Physical element};
\node[class, fill=otherclass](event) at (4\blockx, 3\blocky) {Event};
\node[class, fill=abstractclass](timeinterval) at (4\blockx, 2\blocky) {\textit{Time interval}};

% Aggregation arrows for the scenario category
\node[coordinate, below of=scenario category, node distance=-\blocky/2, xshift=\blockwidth/3](helper scenario category){};
\node[coordinate, below of=scenario category, node distance=\blockheight/2, xshift=\blockwidth/3](aggregation scenario category){};
\foreach \class in {actor, activity, element}
{
	\node[coordinate, above of=\class category, node distance=\blockheight/2](helper \class){};  % Needed for later
	\draw[aggregation] (\class category) |- (helper scenario category) -- (aggregation scenario category);
	\node[anchor=south east] at (helper \class) {\hasn};
}

% Aggregation arrow for the model
\foreach \fromclass/\toclass in {model/activitycategory} {
	\node[coordinate, above of=\fromclass, node distance=\blockheight/2, xshift=-\blockwidth/8+\blockx/4](aggregation \fromclass){};
	\node[coordinate, below of=\toclass, node distance=\blockheight/2, xshift=\blockwidth/8-\blockx/4](aggregation \toclass){};
	\draw[aggregation] (aggregation \fromclass) -- (aggregation \toclass);
}
\node[anchor=south east] at (aggregation model) {\hasone};

% Aggregation arrow for scenario
\node[coordinate, below of=scenario, node distance=-\blocky/2](helper scenario){};
\node[coordinate, below of=scenario, node distance=\blockheight/2](aggregation scenario){};
\foreach \class in {actor, activity, element}
{
	\node[coordinate, above of=\class, node distance=\blockheight/2](helper \class){};
	\draw[aggregation] (helper \class) |- (helper scenario) -- (aggregation scenario);
	\node[anchor=south east] at (helper \class) {\hasn};
}
\node[coordinate, above of=event, node distance=\blockheight/2](helper event){};
\draw[aggregation] (helper event) |- (helper scenario) -- (aggregation scenario);
\node[anchor=south east] at (helper event) {$2,3,\ldots$};

% Aggregations for static thing, activity, and actor
\foreach \class in {element, activity, actor}
{
	\node[coordinate, below of=\class category, node distance=\blockheight/2, xshift=\blockwidth/4](category helper){};
	\node[coordinate, above of=\class, node distance=\blockheight/2, xshift=\blockwidth/4](helper){};
	\draw[aggregation] (category helper) -- (helper);
	\node[anchor=north east] at (category helper) {\hasone};
}

% Aggregation for event -> time interval
\node[coordinate, above of=event, node distance=\blockheight/2, xshift=\blockwidth/4](helper event){};
\node[coordinate, below of=timeinterval, node distance=\blockheight/2, xshift=\blockwidth/4](helper timeinterval){};
\draw[aggregation] (helper event) -- (helper timeinterval);
\node[anchor=south east] at (helper event) {\hastwo};

% falls into arrows
\node[coordinate, right of=scenario category, node distance=\blockwidth/2+1pt, yshift=-\blockheight/3](helper1){};
\node[coordinate, right of=scenario category, node distance=\blockwidth/2+1pt, yshift=\blockheight/3](helper2){};
\node[coordinate, right of=helper1, node distance=\blockwidth/2](helper3){};
\node[coordinate, right of=helper2, node distance=\blockwidth/2](helper4){};
\draw[falls into] (helper1) -- (helper3) -- node[fill=white]{includes} (helper4) -- (helper2);
\node[coordinate, above of=scenario, node distance=\blockheight/2, xshift=-.28\blockwidth](helper1){};
\node[coordinate, below of=scenario category, node distance=\blockheight/2, xshift=-.28\blockwidth](helper2){};
\draw[falls into] (helper2) -- node[fill=white, align=center, text width=3.55em]{comprises} (helper1);

% Legend
\node[class, fill=TNOlightgray, minimum height=6.5em, text width=2\blockwidth+1em](legend) at (4.5\blockx, .4\blocky) {};
\node[yshift=2.65em] at (legend) {Legend};
\node[class, fill=abstractclass, minimum height=1.5em, text width=2\blockwidth, yshift=1.3em](abstract) at (legend) {\textit{Abstract class}};
\node[class, fill=category, minimum height=1.5em, text width=2\blockwidth, yshift=-.4em](category) at (legend) {Class for qualitative description};
\node[class, fill=otherclass, minimum height=1.5em, text width=2\blockwidth, yshift=-2.1em](category) at (legend) {Class for quantitative description};

\end{tikzpicture}
	\caption{Schematic overview of the relation between the classes for representing the scenarios for the assessment of automated vehicles.}
	\label{fig:class relations}
\end{figure*}



\subsection{Scenario category and its attributes}
\label{sec:domain scenario category}

\cstartb Because all other classes in \cref{fig:class overview} are subclasses of a \textit{thing}\cendb\cstarte\footnote{\cstarte It is common practice to call the most general class in an ontology `thing'.\cende}\cende\cstartb, these classes inherit the attributes and procedures of a \textit{thing}. In our framework, a \textit{thing} has a human-interpretable name, a unique ID, and possibly predefined tags that are also interpretable by a software agent. So, all other classes in \cref{fig:class overview} also have these attributes. \cendb

The static environment is qualitatively described by \cstarte one or more \cende\cstartc \textit{static physical thing categories}. \cendc
Because the \cstartc\textit{static physical thing categories} \cendc qualitatively describe the static environment, they contain a human-interpretable description of the \cstartc physical things they describe\cendc.

The ego vehicle\cstartd(s) \cendd and the dynamic environment are qualitatively described by \textit{activity categories}, \cstartb\textit{dynamic physical thing categories}\cendb, and \textit{actor categories}. 
In line with \cref{def:activity}, the \textit{activity category} includes the state variable(s).
The \textit{model} that is used to describe the time evolution of the state variable(s) is specified. 
\cstarte Note that a \textit{model} is an abstract class that serves as a template for different models, such as the three examples shown in \cref{fig:class overview}: \textit{Sinusoidal}, \textit{linear}, and \textit{constant}. 
Let $\state(\time)$ denote the state variable(s) at time $\time$, then the \textit{sinusoidal} model is defined as follows:
\begin{align}
	\statedot(\time) &= \frac{\pi \amplitude}{2\duration} \sin \left( \frac{\pi \left( \time - \timeinit\right)}{\duration} \right),\ \time \in [\timeinit, \timeinit+\duration], \label{eq:sinusoidala} \\
	\state(\timeinit) &= \stateinit. \label{eq:sinusoidalb}
\end{align}
Here, the amplitude ($\amplitude$), duration ($\duration$), initial time ($\timeinit$), and initial state ($\stateinit$) are parameters. 
The \textit{linear} and \textit{constant} model are described by the following equations, respectively:
\begin{align}
	\statedot(\time) &= \slope,\ \state(\timeinit) = \stateinit, \label{eq:linear} \\
	\state(\time) &= \stateinit. \label{eq:constant}
\end{align}
The \textit{linear} model contains three parameters, i.e., the slope ($\slope$), initial time ($\timeinit$), and initial state ($\stateinit$). The \textit{constant} model only has the parameter $\stateinit$.
Since the \textit{activity category} is a qualitative description, the values of the parameters of the \textit{model} are not part of the \textit{activity category}. \cende

The \cstartb \textit{dynamic physical thing category} and \cendb \textit{actor category} have an attribute that specifies the type of object.
\cstartb The difference between an \textit{actor category} and a \textit{dynamic physical thing category} is that an \textit{actor category} has \cendb\cstarte one or more \cende\cstartb (possibly unknown) intents. \cendb
To indicate that an actor is an ego vehicle, the tag ``Ego vehicle'' is added to the list of tags of the \textit{actor category}.

% Scenario category
The \textit{scenario category} has \cstartc \textit{static physical thing categories}\cendc, \textit{activity categories}, \cstartb \textit{dynamic physical thing categories}, \cendb and \textit{actor categories} as attributes. 
%As with the other classes, a \textit{scenario category} contains a name and may contain predefined tags that describe parts of the scenario that are not described by the other classes.
Another attribute of the scenario category is the list of acts. %\footnote{ In line with the definition of \emph{act} in \cref{sec:act}, for a \textit{scenario category}, an \emph{act} is a combination of \textit{activity categories} and \textit{actor categories}.}. 
These acts describe which \cstartb dynamic things and \cendb actors perform which activities. Naturally, it is possible that one \cstartb dynamic thing or \cendb actor performs multiple activities and that one activity is performed by multiple \cstartb dynamic things and \cendb actors.

% Explain why we have these different classes
The reader might wonder why we introduce the different classes for describing a scenario category, i.e., the blue blocks, instead of only one class for modeling a scenario category. 
The main advantage of the different classes is the re-usability of the instances of the classes, because these instances can be exchanged among different \textit{scenario categories}. For example, if two \textit{scenario categories} have the same \textit{actor categories}, we only need to define the \textit{actor categories} once, whereas if the \textit{actor categories} would not be a class on its own but only a property of the scenario category, we would need to define the \textit{actor categories} twice.



\subsection{Scenario and its attributes}
\label{sec:domain scenario}

\cstartb The class scenario is a subclass of \textit{time interval} and, therefore, it has events that define the start and the end of the scenario. \cendb
A \textit{scenario} has \cstartc\textit{static physical things}\cendc, \textit{activities}, \cstartb \textit{dynamic physical things}, \cendb \textit{actors}, and \textit{events} as attributes. 
%The ego vehicle and the dynamic environment are quantitatively described by activities and actors. 
It is similar to a \textit{scenario category}, as the \cstartc\textit{static physical things}\cendc, \textit{activities}, \cstartb \textit{dynamic physical things}, \cendb and \textit{actors} are the quantitative counterparts of the \cstartc\textit{static physical thing categories}\cendc, \textit{activity categories}, \cstartb \textit{dynamic physical thing categories}, \cendb and \textit{actor categories}, just as a \textit{scenario} is the quantitative counterpart of a \textit{scenario category}. 
As with a \textit{scenario category}, a \textit{scenario} contains a list of acts that describe which \cstartb \textit{dynamic physical things} and \cendb actors perform which activities.

% Static environment
%The \textit{static environment} ``has'' a \textit{static environment category}. The most notable difference between the \textit{static environment category} and the \textit{static environment}, is that the \textit{static environment} \cstartb ``has'' \textit{static physical things} instead of \textit{static physical thing categories}. \cendb
\cstartb A \textit{static physical thing} has a \textit{static physical thing category} and it may have multiple properties that quantitatively define the static object. \cendb These properties define, the road layout, static weather and lighting conditions, and infrastructural elements, etc.

% Activity
According to \cref{def:activity}, an activity quantitatively describes the evolution of one or more state variables in a time interval. The state variable(s) are defined by the \textit{activity category} that the \textit{activity} has as an attribute. Together with the \textit{model} that is contained by the \textit{activity category}, the time evolution of the state variable is described by a set of parameters. The values of the parameters are part of the \textit{activity}. 
%\cstartb Note that because the class \textit{activity} is a subclass of \textit{time interval}, it has two events defining the start and the end of the \textit{activity}. \cendb
%Note that the model can be a time interpolation of (measured) data points, in which case the parameters are a sequence of time and state values. The time interval is defined by a duration of the activity.

% Set and triggered activity
%Two different types of activities can be defined. A set activity describes an activity that happens at a certain fixed time. This is often used to describe real-world scenarios that are extracted from real-world data. On the other hand, the starting time of a triggered activity is in general not defined beforehand, as the activity is triggered by an event. This is often used to describe test cases for scenario-based testing, e.g., see the example presented in \cref{sec:example test case}. 
%%Here, the starting time of the activity of the pedestrian, i.e., walking on the pedestrian crossing, is not defined beforehand and depends on, e.g., the speed of the ego vehicle \autocite{seiniger2015test}. 
%Both the \textit{set activity} and the \textit{triggered activity} are subclasses of the \textit{activity}, as shown in \cref{fig:framework classes}. Additionally, the \textit{set activity} ``has'' a starting time and the \textit{triggered activity} ``has'' an \textit{event} that triggers the activity.

% Event
Following \cref{def:event}, an \textit{event} contains conditions that describe the threshold or mode transition at the time of the \textit{event}.

% Actor
Similar to the \textit{static physical thing} and the \textit{activity}, the \cstartb\textit{dynamic physical thing} and \textit{actor} have their qualitative counterparts as an attribute, the \textit{dynamic physical thing category} and \textit{actor category}, respectively. Additionally, because the \textit{dynamic physical thing} and \textit{actor} involve a quantitative description, they may have an initial state vector and multiple properties defined, such as their \cendb size, weight, color, radar cross section, etc. \cstartb The intent of the \textit{actor} can be used to specify the desired state vector. \cendb This is especially useful for defining a test case in terms of the objective of the ego vehicle rather than its activities. 
%Additionally, the goals can be formulated as text if they cannot be formulated using a desired state vector.
%Note that the goals are typically on a strategic level, such as destinations and waypoints, as opposed to the tactical and operational levels\footnote{ See \autocite[p.~7, Figure 1]{sae2018j3016} for an overview of the difference between the strategic, tactical, and operational functions.}.

%According to \cref{def:event}, an events marks the time instant at which the system reaches a specified boundary or at which a mode transition occurs. To describe this time instant, one or more conditions are specified. For example, a condition could be that the distance of the ego vehicle and a pedestrian crossing should be less than \SI{30}{\meter}. For this example, the moment at which this condition is met for the first time corresponds to the event.

% Another advantage of the blue blocks -> only one activity category for multiple activities
An advantage of having the qualitative counterparts of the \textit{static environment}, \cstartb \textit{static physical thing}\cendb, \textit{activity}, \cstartb \textit{dynamic physical thing}, \cendb and \textit{actor} is that the qualitative description can be reused and exchanged. For example, there can be many different braking activities, but there needs to be only one \textit{activity category} for qualitatively defining the braking activity. Here, it is assumed that all braking activities are modeled with the same model and that similar tags apply. If this is not the case, multiple \textit{activity categories} need to be defined, but the number of \textit{activity categories} will still be significantly lower than the number of activities.


\section{Case Study} % Erwin & Hala
%\label{sec:example}

In this section, we present a case study to illustrate the method of quantifying the risk for a cut-in scenario. We will first describe the cut-in scenario and the use case. The actual system for which the risk is computed is presented in next. After these two steps, we will go through the steps of our proposed method.



\subsection{The cut-in scenario and the use case}
%\label{sec:scenario class}

We want to quantify the risk for cut-in scenarios that are linguistically described as follows: while the ego vehicle drives at a moderate to high speed while staying in its lane, another vehicle cuts into the lane of the ego vehicle, such that this vehicle becomes the ego vehicle's lead vehicle. Furthermore, the ego vehicle needs to brake to prevent a collision.

For the quantification of the risk, 60 hours of data (see also \cite{deGelder2017assessment}) are collected by driving a specific route in and between Eindhoven and Helmond, The Netherlands, with twenty different drivers, each driving the route twice. Therefore, it is assumed that the use case of the AD system is driving this route. We will use the data for the estimation of the risk. Hence, we will make use of the following assumption:
\begin{assumption}
	The recorded naturalistic driving data is representative for what a vehicle with the AD system might encounter along the same route.
\end{assumption}



\subsection{System-under-test}
%\label{sec:system}

To reduce efforts for the assessment, often simulations are employed. However, even simulations can consume considerable time, as these simulations might run real-time \cite{shah2018airsim} or slower when a higher level of detail is used \cite{zofka2016testing}. For our method, we will simplify the simulations, such that the total required time on a common computer is in the order of minutes. Since we are interested in approximate results, a high level of detail is not required. 

To simplify the system-under-test, it is assumed that the system's desired acceleration is similar to the adaptive cruise control defined in \cite{deGelder2017assessment}, i.e.,
\begin{equation}
	\label{eq:desired acceleration} 
	u(t) = k_{\mathrm{d}}(v(t))(d(t) - \tau_{\mathrm{h}} v(t) - s_0) + k_{\mathrm{v}}\left(\dot{d}(t) - ha(t) \right),
\end{equation}
with
\begin{equation}
	\label{eq:gain}
	k_{\mathrm{d}}(v(t)) = k_{\mathrm{d1}} + \left( k_{\mathrm{d2}} - k_{\mathrm{d1}} \right) \exp \left\{ -\frac{v(t)^2}{2\sigma_{\mathrm{d}}} \right\}.
\end{equation}
Here, $v$ is the speed of the ego vehicle, $d$ denotes the clearance between the ego vehicle and its predecessor, i.e., the vehicle that performs the cut-in. The relative speed is denoted by $\dot{d}$ and $a$ refers to the acceleration of the ego vehicle. The ego vehicle is modeled using a first order model with a time delay, i.e.,
\begin{equation}
	\label{eq:vehicle model}
	\tau \dot{a}(t) + a(t) = u(t - \theta).
\end{equation}
Furthermore, the deceleration is limited at $\unit[-6]{ms^{-2}}$. A description of the constants of \cref{eq:desired acceleration,eq:gain,eq:vehicle model} are listed in \cref{tab:constants}. The controller runs at \unit[100]{Hz}.

\begin{table}
	\centering
	\caption{The constants used for the simple automation system of \cref{eq:desired acceleration,eq:gain,eq:vehicle model}.}
	\label{tab:constants}
	\begin{tabular}{clc}
		\toprule
		Parameter & Description & Value \\ \otoprule
		$\tau_{\mathrm{h}}$ & Desired headway time & \unit[2.0]{s} \\
		$s_0$ & Safety distance & \unit[1.5]{m} \\
		$k_{\mathrm{d1}}$ & Distance gain at high speed & $\unit[0.7]{s^{-2}}$ \\
		$k_{\mathrm{d2}}$ & Distance gain at low speed & $\unit[2.0]{s^{-2}}$ \\
		$\sigma_{\mathrm{d}}$ & Shaping coefficient of distance gain & $\unit[5]{ms^{-1}}$ \\
		$k_{\mathrm{v}}$ & Speed difference gain & $\unit[0.35]{s^{-1}}$ \\
		$\tau$ & Time constant of the vehicle model & \unit[0.1]{s} \\
		$\theta$ & Delay of the vehicle response & \unit[0.2]{s} \\
		\bottomrule
	\end{tabular}
\end{table}

Note that there is no intervention of a human:
\begin{assumption}
	The ego vehicle is fully controlled by the automation system as defined by \cref{eq:desired acceleration,eq:gain}. Hence, there is no intervention of a human.
\end{assumption}



\subsection{Calculate exposure}
%\label{sec:example exposure}

The cut-in scenarios are subject to the following conditions:
\begin{itemize}
	\item $C_1$: The speed of the ego vehicle is within the range of \unit[60]{km/h} and \unit[130]{km/h}.
	\item $C_2$: There are no restrictions on the weather conditions.
	\item $C_3$: There are no restrictions on the lighting conditions.
\end{itemize}

Obviously, because there are no restrictions to the weather and lighting conditions, we have $P(C_2,C_3)=1$. For the first condition, we can use the data to estimate the likelihood. The data, however, has been recorded during sunny weather at daylight. Therefore, we need to following assumption.

\begin{assumption} \label{asm:conditions}
	Let $C_2'$ and $C_3'$ denote the conditions of having sunny weather and daylight, respectively. Then we have $P(C_1|C_2,C_3)=P(C_1|C_2',C_3')$.
\end{assumption}

From the data, it appeared that $P(C_1|C_2',C_3')=0.20$. Using \cref{asm:conditions}, we have
\begin{equation}
	P(C) = P(C_1,C_2,C_3) = P(C_1|C_2',C_3')\cdot P(C_2,C_3) = 0.20.
\end{equation}

The cut-in scenarios consist of the following activities:
\begin{itemize}
	\item $A_1$: The ego vehicle is lane following.
	\item $A_2$: The target vehicle is driving in an adjacent lane in the same direction as the ego vehicle.
	\item $A_3$: After activity $A_2$, the target vehicle performs a lane change towards the lane of the ego vehicle, such that the ego vehicle needs to brake.
	\item $A_4$: The automation system detects the cut-in.
	\item $A_5$: After activity $A_4$, the automation system activates the brakes of the ego vehicle.
\end{itemize}

The likelihood of the activities $A_1$, $A_2$, and $A_3$ can be estimated using the data. It is assumed that the ego vehicle needs to brake if the target vehicle is driving slower and the headway time is less than three seconds. In case of a slower target vehicle with a larger headway time, the scenario is referred to as a gap closing scenario \cite{semsarkazerooni2016cacc, gelder2016pacc}.

For simplicity, we assume the following:
\begin{assumption}
	The automation system always detects the cut-in and activates the brakes after detecting the cut-in, such that $P(A_4,A_5|A_1,A_2,A_3,C) = 1$.
\end{assumption}

Using this assumption, we can compute $\lambda_{A|C}$ by detecting the number of occurrences of the activities $A_1$, $A_2$, and $A_3$ under the conditions $C$. Based on the dataset, we have $\lambda_{A|C}=\unit[9.9]{h^{-1}}$, i.e., in each hour that the ego vehicle is driving in a speed range of \unit[60]{km/h} and \unit[130]{km/h}, there are on average $9.9$ cut-ins with the target vehicle driving slower than the ego vehicle, such that the headway time after the cut-in is less than three seconds. From this, it simply follows that
\begin{equation}
	\lambda_{A,C} = \lambda_{A|C} \cdot P(C) = 2.0.
\end{equation}



\subsection{Calculating severity}
%\label{sec:example severity}

To limit the number of parameters, we assume the following:
\begin{assumption}
	The ego vehicle is driving at a constant speed at the moment of the cut-in of the target vehicle, i.e., the moment that the target vehicle enters the lane of the ego vehicle.
\end{assumption}
\begin{assumption}
	The target vehicle is driving at a constant speed.
\end{assumption}
Both assumptions can be justified using the data. In case of the ego vehicle, the average acceleration at the moment of the cut-in is $\unit[-0.29]{ms^{-2}}$ and the standard deviation equals $\unit[0.50]{ms^{-2}}$. In case of the target vehicle, the average deceleration at the moment of the cut-in is $\unit[0.05]{ms^{-2}}$ and the standard deviation equals $\unit[0.37]{ms^{-2}}$. As a result, the scenario is parametrized using $d=3$ parameters:
\begin{enumerate}
	\item The clearance between the target vehicle and the ego vehicle at the moment of the cut-in, i.e., the moment than the target vehicle enters the lane of the ego vehicle.
	\item The speed of the ego vehicle at the moment of the cut-in.
	\item The speed of the target vehicle throughout the whole scenario.
\end{enumerate}


A histogram of the data of the parameters is shown in \cref{fig:histogram}. The probability density function is estimated using the KDE of \cref{eq:kde} with the Gaussian kernel of \cref{eq:gaussian kernel}. Before applying KDE, the data is scaled, such that the standard deviation equals one for each parameter. We use leave-one-out cross validation to compute the bandwidth $h$ (see also \cite{duin1976parzen}) because this minimizes the Kullback-Leibler divergence between the real underlying pdf and the estimated pdf \cite{turlach1993bandwidthselection,zambom2013review}. The resulting bandwidth equals $h=0.198$. The marginal probability distributions coming from the resulting joint distribution, i.e. the KDE, are shown in \cref{fig:histogram} by the black lines.

\setlength\figurewidth{0.5\linewidth}
\setlength\figureheight{0.25\linewidth}
\begin{figure}
	\centering
	% This file was created by matplotlib2tikz v0.6.17.
\begin{tikzpicture}

\begin{groupplot}[group style={group size=1 by 3}]
\nextgroupplot[
xlabel={Clearance [m]},
ylabel={Density},
xmin=8.12085897227673, xmax=100.239482786353,
ymin=0, ymax=0.0358233749416452,
width=\figurewidth,
height=\figureheight,
tick align=outside,
tick pos=left,
x grid style={white!69.01960784313725!black},
y grid style={white!69.01960784313725!black},
axis x line*=bottom,
axis y line*=left,
scaled y ticks = false,
y tick label style={/pgf/number format/fixed}
]
\draw[draw=black,fill=gray] (axis cs:12.3080691456438,0) rectangle (axis cs:16.4952793190109,0.0120414705686202);
\draw[draw=black,fill=gray] (axis cs:16.4952793190109,0) rectangle (axis cs:20.682489492378,0.0100345588071835);
\draw[draw=black,fill=gray] (axis cs:20.682489492378,0) rectangle (axis cs:24.8696996657451,0.0260898528986771);
\draw[draw=black,fill=gray] (axis cs:24.8696996657451,0) rectangle (axis cs:29.0569098391121,0.034117499944424);
\draw[draw=black,fill=gray] (axis cs:29.0569098391122,0) rectangle (axis cs:33.2441200124792,0.0160552940914936);
\draw[draw=black,fill=gray] (axis cs:33.2441200124792,0) rectangle (axis cs:37.4313301858463,0.0180622058529303);
\draw[draw=black,fill=gray] (axis cs:37.4313301858463,0) rectangle (axis cs:41.6185403592134,0.0220760293758037);
\draw[draw=black,fill=gray] (axis cs:41.6185403592134,0) rectangle (axis cs:45.8057505325805,0.0100345588071835);
\draw[draw=black,fill=gray] (axis cs:45.8057505325805,0) rectangle (axis cs:49.9929607059476,0.0160552940914936);
\draw[draw=black,fill=gray] (axis cs:49.9929607059476,0) rectangle (axis cs:54.1801708793146,0.0220760293758038);
\draw[draw=black,fill=gray] (axis cs:54.1801708793146,0) rectangle (axis cs:58.3673810526817,0.0120414705686202);
\draw[draw=black,fill=gray] (axis cs:58.3673810526817,0) rectangle (axis cs:62.5545912260488,0.0080276470457468);
\draw[draw=black,fill=gray] (axis cs:62.5545912260488,0) rectangle (axis cs:66.7418013994159,0.00401382352287341);
\draw[draw=black,fill=gray] (axis cs:66.7418013994159,0) rectangle (axis cs:70.929011572783,0.00602073528431012);
\draw[draw=black,fill=gray] (axis cs:70.929011572783,0) rectangle (axis cs:75.1162217461501,0.0060207352843101);
\draw[draw=black,fill=gray] (axis cs:75.1162217461501,0) rectangle (axis cs:79.3034319195171,0.00401382352287341);
\draw[draw=black,fill=gray] (axis cs:79.3034319195171,0) rectangle (axis cs:83.4906420928842,0.00401382352287341);
\draw[draw=black,fill=gray] (axis cs:83.4906420928842,0) rectangle (axis cs:87.6778522662513,0.0020069117614367);
\draw[draw=black,fill=gray] (axis cs:87.6778522662513,0) rectangle (axis cs:91.8650624396184,0);
\draw[draw=black,fill=gray] (axis cs:91.8650624396184,0) rectangle (axis cs:96.0522726129855,0.0060207352843101);
\addplot [very thick, black, forget plot]
table {%
8.12085897227673 0.00183504532546135
10.0008308868497 0.00339657358685259
11.8808028014227 0.00528599275795651
13.7607747159957 0.00725825955475721
15.6407466305686 0.00939473355010381
17.5207185451416 0.0121118079549573
19.4006904597146 0.0156352071526778
21.2806623742876 0.019524214372825
23.1606342888605 0.0228741245174167
25.0406062034335 0.0249480173856481
26.9205781180065 0.0255164505880221
28.8005500325795 0.0247790331703545
30.6805219471524 0.0232196174226075
32.5604938617254 0.0214346789684533
34.4404657762984 0.0198361008585762
36.3204376908714 0.0184565224694155
38.2004096054443 0.0171178383723391
40.0803815200173 0.0158408817383565
41.9603534345903 0.0150085292584412
43.8403253491633 0.0149883504483441
45.7202972637363 0.0156889924355427
47.6002691783092 0.016633089010273
49.4802410928822 0.0172824782581887
51.3602130074552 0.0171709350329699
53.2401849220281 0.0160353433185709
55.1201568366011 0.0140440303580629
57.0001287511741 0.0117399946150153
58.8801006657471 0.00966924155768559
60.7600725803201 0.00812493023353695
62.640044494893 0.00714814073279193
64.520016409466 0.00659055446858629
66.399988324039 0.00619392002395415
68.279960238612 0.00576814790188572
70.1599321531849 0.00533478495791458
72.0399040677579 0.00502884268136011
73.9198759823309 0.00488620187167895
75.7998478969039 0.00480058510610992
77.6798198114768 0.00463086872766452
79.5597917260498 0.00426990390723946
81.4397636406228 0.00367352208580758
83.3197355551958 0.002911397952567
85.1997074697687 0.00217651417425062
87.0796793843417 0.00170433569140452
88.9596512989147 0.00163829842167866
90.8396232134877 0.00190499378008258
92.7195951280606 0.00221235703924666
94.5995670426336 0.00224116277006498
96.4795389572066 0.00188363475625321
98.3595108717796 0.00129365752109914
100.239482786353 0.000721425131598922
};
\nextgroupplot[
xlabel={Ego vehicle speed [km/h]},
ylabel={Density},
xmin=56.8026, xmax=132.1614,
ymin=0, ymax=0.0463664183487372,
width=\figurewidth,
height=\figureheight,
tick align=outside,
tick pos=left,
x grid style={white!69.01960784313725!black},
y grid style={white!69.01960784313725!black},
axis x line*=bottom,
axis y line*=left,
scaled y ticks = false,
y tick label style={/pgf/number format/fixed}
]
\draw[draw=black,fill=gray] (axis cs:60.228,0) rectangle (axis cs:63.6534,0.0294389957769761);
\draw[draw=black,fill=gray] (axis cs:63.6534,0) rectangle (axis cs:67.0788,0.044158493665464);
\draw[draw=black,fill=gray] (axis cs:67.0788,0) rectangle (axis cs:70.5042,0.0367987447212201);
\draw[draw=black,fill=gray] (axis cs:70.5042,0) rectangle (axis cs:73.9296,0.0220792468327321);
\draw[draw=black,fill=gray] (axis cs:73.9296,0) rectangle (axis cs:77.355,0.00981299859232536);
\draw[draw=black,fill=gray] (axis cs:77.355,0) rectangle (axis cs:80.7804,0.0122662482404067);
\draw[draw=black,fill=gray] (axis cs:80.7804,0) rectangle (axis cs:84.2058,0.00981299859232532);
\draw[draw=black,fill=gray] (axis cs:84.2058,0) rectangle (axis cs:87.6312,0.00735974894424402);
\draw[draw=black,fill=gray] (axis cs:87.6312,0) rectangle (axis cs:91.0566,0.0171727475365694);
\draw[draw=black,fill=gray] (axis cs:91.0566,0) rectangle (axis cs:94.482,0.022079246832732);
\draw[draw=black,fill=gray] (axis cs:94.482,0) rectangle (axis cs:97.9074,0.0196259971846507);
\draw[draw=black,fill=gray] (axis cs:97.9074,0) rectangle (axis cs:101.3328,0.00735974894424402);
\draw[draw=black,fill=gray] (axis cs:101.3328,0) rectangle (axis cs:104.7582,0.00735974894424402);
\draw[draw=black,fill=gray] (axis cs:104.7582,0) rectangle (axis cs:108.1836,0.014719497888488);
\draw[draw=black,fill=gray] (axis cs:108.1836,0) rectangle (axis cs:111.609,0.014719497888488);
\draw[draw=black,fill=gray] (axis cs:111.609,0) rectangle (axis cs:115.0344,0.00245324964808134);
\draw[draw=black,fill=gray] (axis cs:115.0344,0) rectangle (axis cs:118.4598,0.00735974894424402);
\draw[draw=black,fill=gray] (axis cs:118.4598,0) rectangle (axis cs:121.8852,0.00245324964808133);
\draw[draw=black,fill=gray] (axis cs:121.8852,0) rectangle (axis cs:125.3106,0.00245324964808133);
\draw[draw=black,fill=gray] (axis cs:125.3106,0) rectangle (axis cs:128.736,0.00245324964808134);
\addplot [very thick, black, forget plot]
table {%
56.8026 0.0051251569966451
58.3405346938776 0.00948273774153526
59.8784693877551 0.0152737911443766
61.4164040816326 0.0216226093112603
62.9543387755102 0.0272307114498158
64.4922734693878 0.0310165077928978
66.0302081632653 0.0326451670840618
67.5681428571429 0.0324529962834777
69.1060775510204 0.0308932664673115
70.644012244898 0.0281869811378229
72.1819469387755 0.0245386832434187
73.7198816326531 0.0204882159449357
75.2578163265306 0.0168287779986908
76.7957510204082 0.0141397740791014
78.3336857142857 0.0124556966394098
79.8716204081633 0.0114128586744221
81.4095551020408 0.0107144084500764
82.9474897959184 0.0104479311726113
84.4854244897959 0.0109388055897548
86.0233591836735 0.0123061387946482
87.561293877551 0.0142089192941123
89.0992285714286 0.0160347079489904
90.6371632653061 0.0172758605250283
92.1750979591837 0.0177022546970844
93.7130326530612 0.0172862137574363
95.2509673469388 0.0161263707827698
96.7889020408163 0.0144854204851373
98.3268367346939 0.0127945110615371
99.8647714285714 0.0115070745512157
101.402706122449 0.0109087496237425
102.940640816327 0.0110258700545376
104.478575510204 0.0116168743401693
106.016510204082 0.0122015258253593
107.554444897959 0.0122126107321703
109.092379591837 0.011316036665026
110.630314285714 0.00967556360656004
112.168248979592 0.00785465965638699
113.706183673469 0.00638972154271003
115.244118367347 0.00543292751134418
116.782053061224 0.00477235058212446
118.319987755102 0.0041313841930039
119.85792244898 0.00341979893967049
121.395857142857 0.00273966624795546
122.933791836735 0.00222136776855905
124.471726530612 0.00189080614806837
126.00966122449 0.00167050575312529
127.547595918367 0.00145897016799862
129.085530612245 0.00119601841250331
130.623465306122 0.000883660847297828
132.1614 0.000571013454877462
};
\nextgroupplot[
xlabel={Target vehicle speed [km/h]},
ylabel={Density},
xmin=48.2488141652528, xmax=129.081859577732,
ymin=0, ymax=0.0312190317572164,
width=\figurewidth,
height=\figureheight,
tick align=outside,
tick pos=left,
x grid style={white!69.01960784313725!black},
y grid style={white!69.01960784313725!black},
axis x line*=bottom,
axis y line*=left,
scaled y ticks = false,
y tick label style={/pgf/number format/fixed}
]
\draw[draw=black,fill=gray] (axis cs:51.9230435021837,0) rectangle (axis cs:55.5972728391146,0.0160097598754956);
\draw[draw=black,fill=gray] (axis cs:55.5972728391146,0) rectangle (axis cs:59.2715021760454,0.0274453026437067);
\draw[draw=black,fill=gray] (axis cs:59.2715021760454,0) rectangle (axis cs:62.9457315129763,0.029732411197349);
\draw[draw=black,fill=gray] (axis cs:62.9457315129763,0) rectangle (axis cs:66.6199608499072,0.0297324111973489);
\draw[draw=black,fill=gray] (axis cs:66.6199608499072,0) rectangle (axis cs:70.2941901868381,0.0251581940900645);
\draw[draw=black,fill=gray] (axis cs:70.2941901868381,0) rectangle (axis cs:73.968419523769,0.0114355427682111);
\draw[draw=black,fill=gray] (axis cs:73.9684195237689,0) rectangle (axis cs:77.6426488606998,0.0114355427682111);
\draw[draw=black,fill=gray] (axis cs:77.6426488606998,0) rectangle (axis cs:81.3168781976307,0.0091484342145689);
\draw[draw=black,fill=gray] (axis cs:81.3168781976307,0) rectangle (axis cs:84.9911075345616,0.0251581940900646);
\draw[draw=black,fill=gray] (axis cs:84.9911075345616,0) rectangle (axis cs:88.6653368714925,0.0137226513218533);
\draw[draw=black,fill=gray] (axis cs:88.6653368714925,0) rectangle (axis cs:92.3395662084233,0.0137226513218534);
\draw[draw=black,fill=gray] (axis cs:92.3395662084233,0) rectangle (axis cs:96.0137955453542,0.0137226513218533);
\draw[draw=black,fill=gray] (axis cs:96.0137955453542,0) rectangle (axis cs:99.6880248822851,0.0114355427682112);
\draw[draw=black,fill=gray] (axis cs:99.6880248822851,0) rectangle (axis cs:103.362254219216,0.0114355427682111);
\draw[draw=black,fill=gray] (axis cs:103.362254219216,0) rectangle (axis cs:107.036483556147,0.0091484342145689);
\draw[draw=black,fill=gray] (axis cs:107.036483556147,0) rectangle (axis cs:110.710712893078,0);
\draw[draw=black,fill=gray] (axis cs:110.710712893078,0) rectangle (axis cs:114.384942230009,0.0091484342145689);
\draw[draw=black,fill=gray] (axis cs:114.384942230009,0) rectangle (axis cs:118.059171566939,0.00228710855364223);
\draw[draw=black,fill=gray] (axis cs:118.059171566939,0) rectangle (axis cs:121.73340090387,0);
\draw[draw=black,fill=gray] (axis cs:121.73340090387,0) rectangle (axis cs:125.407630240801,0.00228710855364223);
\addplot [very thick, black, forget plot]
table {%
48.2488141652528 0.00168027351915601
49.8984681532626 0.00364136394523865
51.5481221412724 0.00688176107580989
53.1977761292821 0.0114489015470479
54.8474301172919 0.0168709546332398
56.4970841053017 0.0221330359214515
58.1467380933115 0.0260855665048589
59.7963920813213 0.0281384416069486
61.4460460693311 0.0285958499818234
63.0957000573408 0.0281924497114946
64.7453540453506 0.0273196705422846
66.3950080333604 0.0258020442276619
68.0446620213702 0.0233588397540983
69.6943160093799 0.0201327171673932
71.3439699973897 0.0167342119826331
72.9936239853995 0.013855278897072
74.6432779734093 0.0119505628514347
76.2929319614191 0.0112658440630234
77.9425859494289 0.0119182568524426
79.5922399374386 0.0137110725781274
81.2418939254484 0.0159345040339956
82.8915479134582 0.0176106969018885
84.541201901468 0.0181007199210612
86.1908558894778 0.0174565218739648
87.8405098774875 0.0161714322178921
89.4901638654973 0.0147087680617018
91.1398178535071 0.013322338242064
92.7894718415169 0.0121557097427847
94.4391258295267 0.0113256717531443
96.0887798175364 0.0108960916673038
97.7384338055462 0.0108124450380329
99.388087793556 0.0108638117514665
101.037741781566 0.0107358182498313
102.687395769576 0.0101646288594111
104.337049757585 0.00909804954681981
105.986703745595 0.00775174572590957
107.636357733605 0.00650285877305912
109.286011721615 0.00563802203977032
110.935665709624 0.00512673120774266
112.585319697634 0.00467017312959077
114.234973685644 0.00400580078752985
115.884627673654 0.00312601636465223
117.534281661664 0.00221697926245841
119.183935649673 0.00148313963112521
120.833589637683 0.00104082813978185
122.483243625693 0.000878756940049195
124.132897613703 0.000864231690922192
125.782551601712 0.000826393843663879
127.432205589722 0.000676661115011963
129.081859577732 0.000451217339567434
};
\end{groupplot}

\end{tikzpicture}
	\caption{Histogram of the data of the parameters (bars) and their estimated marginal probabilities (lines).}
	\label{fig:histogram}
\end{figure}


Let $R$ denote the result of having a collision. Given a certain parameter vector $\theta$, we have $P(R|\theta,A,C)=1$ if the outcome of the simulation is a collision and $P(R|\theta,A,C)=0$ otherwise. For the simulation, we used the forward Euler method with a step size of \unit[0.01]{s}, similar as the sample time of the controller. On a regular computer, approximately 2000 simulations are performed in a second. We performed a million simulations, i.e., $N=10^6$. In total, 28 simulations ended with a collision, thus, according to \cref{eq:monte carlo}, we have:
\begin{equation}
	P(R|A,C) = 2.8 \cdot 10^{-5}.
\end{equation}



\subsection{Calculating the risk}
%\label{sec:example risk}

Let $\lambda$ denote the average number of collisions with a cut-in scenario as described earlier along the specified route for a vehicle with the automation system as described above. Using \cref{eq:risk}, we have:
\begin{equation}
	\lambda = \lambda_{A,C} \cdot P(R|A,C) = \unit[5.5 \cdot 10^{-5}]{h^{-1}}.
\end{equation}

Using \cref{eq:no harm}, the probability of having no collision in a cut-in scenario as described above during an hour of driving is
\begin{equation}
	P(\text{no }R,A,C\text{ during an hour}) = 0.999945.
\end{equation}

By solving the Poisson distribution of \cref{eq:poisson risk} for $\lambda$ with $k=0$, we can also conclude that with \unit[95]{\%} certainty, there will be no collision in a cut-in scenario as described earlier when driving \unit[925]{h}.

%\section{Discussion and future outlook} % Hala & Erwin & Arash
\label{sec:discussion}

To be discussed:
\begin{itemize}
	\item Method gives only order of risk.
	\item ``Controllability'' not considered.
	\item A lot of assumptions: with this method, these assumptions are made explicit, whereas often people make these assumptions implicit (and implicit assumptions are the mother of all fuck-ups; should be rephrased :)).
\end{itemize}

\section{Conclusions}
\label{sec:conclusions}

\todo{Write the conclusions. Perhaps also a short discussion?}

\section*{Acknowledgement}

\color{red}
Thanks for CETRAN.

\color{black}



%\addtolength{\textheight}{-12cm}  % This command serves to balance the column lengths
                                  % on the last page of the document manually. It shortens
                                  % the textheight of the last page by a suitable amount.
                                  % This command does not take effect until the next page
                                  % so it should come on the page before the last. Make
                                  % sure that you do not shorten the textheight too much.

\printbibliography
%\bibliographystyle{abbrvnat}
%\bibliography{../bib}

\appendices
\subsection{Nomenclature}
\label{sec:nomenclature}

In the previous subsections, the definitions of a scenario and and event are presented. For this purpose, several notions are adopted. In this subsection, the notions of \emph{ego vehicle}, \emph{activity}, \emph{static environment}, \emph{dynamic environment}, \emph{model}, \emph{actor} and \emph{state} are explained. 

\subsubsection{Ego vehicle}
\label{sec:ego vehicle}
Literally, `ego' means `I'. Thus, the ego vehicle refers to the perspective from which the world is seen. Usually, the ego vehicle refers to the vehicle which is perceiving the world through its sensors (see e.g.~\cite{Bonnin2014}) or the vehicle which has to perform a specific task (see e.g.~\cite{althoff2017CommonRoad}). The ontology presented by Geyer~et~al.\ ``is described from the ego-vehicle’s point of view'' \cite{geyer2014}. In this paper, the ego vehicle determines which events are `relevant'. Thus, the ego vehicle is responsible for determining the start and end time of the scenario. Furthermore, the tags assigned to the scenario may depend on the ego vehicle. For example, the `target in front' refers to the vehicle in front of the ego vehicle. 

Note that in case a sensor-equipped vehicle is used to extract scenarios from real-life driving, the ego vehicle in an extracted scenario does not necessarily correspond to the sensor-equipped vehicle which is used to acquire the real-life data.

\subsubsection{Activity}
\label{sec:activity}
A scenario contains the quantitative description of the activity of the ego vehicle. Here, the description refers to the changing states of the ego vehicle which are relevant for the scenario, e.g., acceleration and velocity. The activity is described using the models (see section \ref{sec:model}) that describe the way the state evolves over time.

\subsubsection{Static environment}
\label{sec:static environment}
The static environment refers to the part of a scenario that does not change during a scenario. This includes geo-spatially stationary elements \cite{ulbrich2015}. Although one might argue whether light and weather conditions are dynamic or not \cite{geyer2014,bach2016modelbased}, we think it is reasonable to assume that these conditions are not subject to significant changes during the time frame of a scenario. Hence, light and weather conditions are considered to be part of the static environment.

\subsubsection{Dynamic environment}
\label{sec:dynamic environment}
As opposed to the static environment, the dynamic environment refers to the part of a scenario that changes during the time frame of a scenario. The dynamic environment is described using the models (see section \ref{sec:model}) that describe the way the state evolves over time. In practice, the dynamic environment consists of the moving actors (other than the ego vehicle) which are relevant for the ego vehicle.

\subsubsection{Model}
\label{sec:model}
Typically, a system is modelled using a differential equation of the form $\dot{x}=f(x,u,t)$, where $x$ represents the state, $u$ represents an external input and $t$ denotes the time \cite{norman2011control}. The input $u$ is a function of time, which needs to be quantified. For this purpose, a parametrized function can be used, i.e. $u=g_{\theta}(t)$ with parameter vector $\theta$, such that the differential equation can be rewritten to $\dot{x}=h_{\theta}(x,t)$. In the context of this paper, model refers to a parametrized function, such as $h_{\theta}(x,t)$. It might be more practical to directly model the state (i.e., the result of the differential equation) using a function $x=k_{\theta}(t)$, such that no explicit information is required about the system dynamics. For example, see \cite{deGelder2017assessment}.

\subsubsection{State}
\label{sec:state}
A state refers to a quantity. The only restriction is that all states must be linearly independent \cite{norman2011control}.

\subsubsection{Actor}
\label{sec:actor}
An actor is an element of a scenario acting on its own behalf \cite{ulbrich2015}. In practice, this can be a driver of e.g., a car or bike, an automation system or a combination of a driver and an automation system \cite{geyer2014}.



\end{document}
