\documentclass[journal]{IEEEtran}


% make changes take effect
\pagestyle{headings}
% adjust as needed
\addtolength{\footskip}{0\baselineskip}
\addtolength{\textheight}{-1\baselineskip}




%\documentclass[a4paper, 10pt, conference]{ieeeconf}      % Use this line for a4 paper

%\IEEEoverridecommandlockouts                              % This command is only needed if 
                                                          % you want to use the \thanks command

% See the \addtolength command later in the file to balance the column lengths
% on the last page of the document

% The following packages can be found on http:\\www.ctan.org
\usepackage{graphicx} % for pdf, bitmapped graphics files
\usepackage{epstopdf}
%\usepackage{epsfig} % for postscript graphics files
%\usepackage{mathptmx} % assumes new font selection scheme installed
%\usepackage{times} % assumes new font selection scheme installed
\let\proof\relax
\let\endproof\relax
\usepackage{amsmath} % assumes amsmath package installed
\usepackage{amsthm}  % For special theorem style
\usepackage{amsfonts}
%\usepackage{breqn}
%\usepackage{amssymb}  % assumes amsmath package installed
\usepackage[dvipsnames]{xcolor}
\usepackage{tikz}
\usepackage{pgfplots}
\usetikzlibrary{shapes,arrows}
\usetikzlibrary{backgrounds}
\usepackage{multirow}
\usepackage[keeplastbox]{flushend}
%\usepackage{cite}  % Make sure that citation appear as [1]-[3] instead of [1], [2], [3]
\usepackage[utf8]{inputenc}    % utf8 support
\usepackage[T1]{fontenc}       % code for pdf file
\usepackage[USenglish]{babel}  % language support
%\usepackage{authblk}


\usepackage[utf8]{inputenc}   				 	%% utf8 support (required for biblatex)
\usepackage{silence}  							%% For filtering warnings
\usepackage[style=ieee,doi=false,isbn=false,url=false,date=year,backend=biber,maxbibnames=15,maxcitenames=2,mincitenames=1,uniquelist=false,uniquename=false,giveninits=true]{biblatex}
% Filter warnings issued by package biblatex starting with "Patching footnotes failed"
\WarningFilter{biblatex}{Patching footnotes failed}
\renewcommand*{\bibfont}{\footnotesize}		%% Use this for papers
\renewcommand*{\bibfont}{\small}
\setlength{\biblabelsep}{\labelsep}
\bibliography{../bib}


%\usepackage{url}
%\usepackage{natbib}
%\usepackage{letltxmacro}
%\LetLtxMacro{\autocite}{\citep}
%\LetLtxMacro{\textcite}{\citet}


% Table stuff
\usepackage{booktabs}
\usepackage{tabularx}
\setlength{\heavyrulewidth}{0.1em}
\newcommand{\otoprule}{\midrule[\heavyrulewidth]}



\usepackage{pgfplots}
\pgfplotsset{compat=1.9}  % Prevent warning, pgf running in backwards compatibility mode anyway%\usetikzlibrary{external}                       %% Create pdf figures from TikZ. Use PDFTeXify ...
%\tikzexternalize[prefix=tikz/]                  %% ... with --tex-option=--shell-escape switch.
\usepackage[capitalize]{cleveref}
\Crefname{figure}{Figure}{Figures}
\crefname{equation}{}{}
\Crefname{equation}{Equation}{Equations}
\usepackage{subcaption}
\usepackage{xstring}
\usepackage{xparse}
\usepackage{siunitx}
\usepackage[nolist,nohyperlinks]{acronym}
\makeatletter
\newcommand*{\org@overidelabel}{}
\let\org@overridelabel\@verridelabel
\@ifpackagelater{acronym}{2015/03/21}{% v1.41
  \renewcommand*{\@verridelabel}[1]{%
    \@bsphack
    \protected@write\@auxout{}{\string\AC@undonewlabel{#1@cref}}%
    \org@overridelabel{#1}%
    \@esphack
  }%
}{% older versions
  \renewcommand*{\@verridelabel}[1]{%
    \@bsphack
    \protected@write\@auxout{}{\string\undonewlabel{#1@cref}}%
    \org@overridelabel{#1}%
    \@esphack
  }%
}
\makeatother

\theoremstyle{plain}
\newtheorem{definition}{Definition}
\theoremstyle{remark}\newtheorem{remarkenv}{Remark}        %% remarks
\newenvironment{remark}{\begin{remarkenv}}%
	{\hfill$\lozenge$\end{remarkenv}}            %% end remark with a lozenge

%\pgfplotsset{compat=newest} 
%\pgfplotsset{plot coordinates/math parser=false}

%Images path 
\graphicspath{ {figures/} }


\newlength\figurewidth
\newlength\figureheight
\newlength\venncircle
\newlength\objectwidth\setlength{\objectwidth}{16.2em}


\usetikzlibrary{arrows,positioning}
\usetikzlibrary{arrows.meta}
\definecolor{TNOlightgray}{RGB}{222,222,231}
\definecolor{abstractclass}{RGB}{255, 255, 200}
%\definecolor{scenariocategory}{RGB}{137, 197, 255}
\definecolor{scenariocategory}{RGB}{201, 231, 255}
\definecolor{category}{RGB}{201, 231, 255}
%\definecolor{scenario}{RGB}{255, 157, 121}
\definecolor{scenario}{RGB}{255, 210, 190}
\definecolor{otherclass}{RGB}{255, 210, 190}
\tikzstyle{tag}=[text height=.8em, text depth=.1em, font=\small\sffamily, rounded corners=0.2em, fill=TNOlightgray, node distance=9em, text width=7em, align=center]
\tikzstyle{tag wide}=[tag, text width=12em]
\tikzstyle{tagarrow}=[->, line width=0.75mm, color=TNOlightgray]
\tikzstyle{object}=[draw, text width=\objectwidth-.5em, align=left, line width=1pt, minimum width=\objectwidth, anchor=north west, node distance=3pt]
\newcounter{tagbcounter}
\newcounter{tagccounter}
\newcommand{\taga}[1]{
	\node[tag wide](taga){#1};
	\node[coordinate, below of=taga, node distance=1.2em](helper){};
}
\newcommand{\tagb}[3]{
	\node[tag, below of=taga, node distance=3em, xshift=#3](tagb#2){#1};
	\draw[tagarrow] (taga) -- (helper) -| (tagb#2);
	\node[coordinate, xshift=1em](tagb#2 helper) at (tagb#2.south west) {};
}
\newcommand{\tagc}[4]{
	\node[tag, below of=tagb#4, node distance=#3em, xshift=2em](tagc#2){#1};
	\draw[tagarrow] (tagb#4 helper) |- (tagc#2);
}
\ExplSyntaxOn
\NewDocumentCommand{\tree}{O{} m m}{%
	\begin{tikzpicture}
	\taga{#2}
	\setcounter{tagbcounter}{0}
	\seq_set_split:Nnn \arg { ; } { #3 }
	\seq_map_inline:Nn \arg {
		\seq_set_split:Nnn \argb {,} {##1}
		\seq_pop_left:NN \argb \argl
		\tagb{\argl}{\arabic{tagbcounter}}{\arabic{tagbcounter}*10em-\seq_count:N \arg *5em+5em}
		\setcounter{tagccounter}{0}
		\seq_map_inline:Nn \argb {
			\stepcounter{tagccounter}
			\tagc{####1}{\arabic{tagccounter}}{2*\arabic{tagccounter}}{\arabic{tagbcounter}}
		}
		\stepcounter{tagbcounter}
	}
	#1
	\end{tikzpicture}
}
\ExplSyntaxOff

% Needed for the class diagrams
\newlength\blockwidth
\newlength\blockheight
\newlength\blockx
\newlength\blocky
\newlength\legendwidth
\setlength{\blockwidth}{6.3em}
\setlength{\blockheight}{2.5em}
\setlength{\blockx}{8em}
\setlength{\blocky}{-5.5em}
\setlength{\legendwidth}{3.5em}
\tikzstyle{class}=[draw, text width=\blockwidth-.5em, align=center, minimum height=\blockheight, line width=1pt, minimum width=\blockwidth]
\tikzstyle{aggregation}=[-{Diamond[width=8pt, length=12pt, fill=white]}, line width=1pt]
\tikzstyle{falls into}=[->, line width=1pt]
\tikzstyle{superclass}=[-{Triangle[width=8pt, length=12pt, fill=white]}, line width=1pt]

% Notations
\newcommand{\amplitude}{A}
\newcommand{\attrtstart}{start event}
\newcommand{\attrtend}{end event}
\newcommand{\comprises}{\ni}
\newcommand{\dimensionstate}{n}
\newcommand{\distancecondition}{d_{\mathrm{v,p}}}
\newcommand{\duration}{T}
\newcommand{\east}{x}
\newcommand{\egosub}{ego}
\newcommand{\egoeast}{\east_{\mathrm{\egosub}}}
\newcommand{\egonorth}{\north_{\mathrm{\egosub}}}
\newcommand{\egospeed}{v_{\mathrm{\egosub}}}
\newcommand{\egoheading}{\head_{\mathrm{\egosub}}}
\newcommand{\function}{f}
\newcommand{\hasone}{$1$}
\newcommand{\hastwo}{$2$}
\newcommand{\hasn}{$0,1,\ldots$}
\newcommand{\head}{\phi}
\newcommand{\includes}{\supseteq}
\newcommand{\inputsystem}{u}
\newcommand{\north}{y}
\newcommand{\origin}{O}
\newcommand{\parameter}{\theta}
\newcommand{\parametera}{a}
\newcommand{\parameterb}{b}
\newcommand{\pedsub}{ped}
\newcommand{\pedeast}{\east_{\mathrm{\pedsub}}}
\newcommand{\pednorth}{\north_{\mathrm{\pedsub}}}
\newcommand{\pedheading}{\head_{\mathrm{\pedsub}}}
\newcommand{\scenario}{S}
\newcommand{\scenarioa}{\scenario_{1}}
\newcommand{\scenariob}{\scenario_{2}}
\newcommand{\scenarioc}{\scenario_{3}}
\newcommand{\scenariocategory}{\mathcal{C}}
\newcommand{\scenariocategorya}{\scenariocategory_{1}}
\newcommand{\scenariocategoryb}{\scenariocategory_{2}}
\newcommand{\scenariocategoryc}{\scenariocategory_{3}}
\newcommand{\slope}{s}
\newcommand{\state}{z}
\newcommand{\statedot}{\dot{\state}}
\newcommand{\stateinit}{\state_{0}}
\renewcommand{\time}{t}
\newcommand{\timeinit}{\time_{0}}

% Acronyms
\begin{acronym}[AAAAAAAA]
	\acro{av}[AV]{Automated Vehicle}
	\acroindefinite{av}{an}{an}
	\acro{odd}[ODD]{Operational Design Domain}
	\acroindefinite{odd}{an}{an}
	\acro{oof}[OOF]{Object-Oriented Framework}
	\acroindefinite{oof}{an}{an}
\end{acronym}

%\usepackage{changebar}
\newcommand{\cstart}{}  % Revision 2 changes (from 11 Jan 2020)
\newcommand{\cend}{}
\newcommand{\cstartb}{}  % Changes since July 14, 2020
\newcommand{\cendb}{}
\newcommand{\cstartc}{}  % Changes since August 15, 2020
\newcommand{\cendc}{}
\newcommand{\cstartd}{}  % Changes since September 22, 2020
\newcommand{\cendd}{}
\newcommand{\cstarte}{}  % Changes since September 30, 2020
\newcommand{\cende}{}
\newcommand{\cstartf}{}  % Changes since November 3, 2020
\newcommand{\cendf}{}
%\newcommand{\todo}[1]{\color{red}TODO: #1\color{black}}

%\usepackage{setspace}
%\doublespacing


\begin{document}
\date{}


%\thispagestyle{empty}
%\pagestyle{empty}
\title{\cstartd Towards an Ontology for Scenario Definition for the Assessment of Automated Vehicles: An Object-Oriented Framework \cendd}

\author{Erwin~de~Gelder$^{1,2,*}$,
	    Jan-Pieter~Paardekooper$^{1,3}$,
	    Arash~Khabbaz~Saberi$^{1}$,
	    Hala~Elrofai$^{1}$,
	    Olaf~Op~den~Camp$^{1}$,
	    Steven~Kraines$^{4}$,
	    Jeroen~Ploeg$^{5,6}$,
	    Bart~De~Schutter$^{2}$%
\thanks{$^1$ TNO, Dept.\ of Integrated Vehicle Safety, Helmond, The Netherlands}%
\thanks{$^2$ Delft University of Technology, Delft Center for Systems and Control, Delft, The Netherlands}%
\thanks{$^3$ Radboud University, Donders Institute for Brain, Cognition and Behaviour, Nijmegen, The Netherlands}%
\thanks{$^4$ Symphony Co.\ Tokyo, Japan}%
\thanks{$^5$ 2getthere, Utrecht, The Netherlands}%
\thanks{$^6$ Eindhoven University of Technology, Dept.\ of Mechanical Engineering Dynamics and Control group, Eindhoven, The Netherlands}%
\thanks{$^*$ Corresponding author: \textit{erwin.degelder@tno.nl}}}%


%%%%%%%%%%%%%%%%%%%%%%%%%%%%%%%%%%%%%%%%%%%%%%%%%%%%%%%%%%%%%%%%%%%%%%%%%%%%%%%%
\maketitle
\begin{abstract}
	\cstarte Scenario-based methods for the assessment of Automated Vehicles (AVs) are widely supported by many players in the automotive field.
	Scenarios captured from real-world data can be used to define the scenarios for the assessment and to estimate the relevance of the scenarios for the assessment.
	Therefore, different techniques are proposed for capturing scenarios from real-world data.
	In this paper, we propose a new method to capture scenarios from real-world data using a two-step approach.
	The first step is to automatically provide the data with tags.
	Next, by representing a scenario using a combination of tags, we are able to mine the scenarios based on the provided tags.
	One of the benefits of our approach is that the tags can be used to identify characteristics of a scenario that are shared among different type of scenarios. 
	In this way, these characteristics need to be identified only once, whereas these characteristics would be identified multiple times if each type of scenario would be identified completely independently.
	Furthermore, the method is not specific for one type of scenario and, therefore, it can be applied to a large variety of scenarios.
	We provide two examples to illustrate the method.
	This paper is concluded with some promising future possibilities for our approach, such as the generation of scenarios for the assessment of automated vehicles.
	\cende
\end{abstract}
\acresetall

%%%%%%%%%%%%%%%%%%%%%%%%%%%%%%%%%%%%%%%%%%%%%%%%%%%%%%%%%%%%%%%%%%%%%%%%%%%%%%%%
\section{Introduction}
\label{sec:introduction}

\cstartg
% Introduce scenario-based testing.
The development of Automated Vehicles (AVs) has made significant progress in the last years and it is expected that AVs will soon be introduced on our roads \autocite{madni2018autonomous,bimbraw2015autonomous} and become an integral part of intelligent transportation systems \autocite{eskandarian2012introduction,chanedmiston2020itsjpo}. \cendg
\cstarta An essential aspect in the development of AVs is the assessment of quality and performance aspects of the AVs, such as safety, comfort, and efficiency \autocite{bengler2014threedecades, stellet2015taxonomy}. 
Among other methods, a scenario-based approach has been proposed \autocite{elrofai2018scenario, putz2017pegasus}. 
% Explain that these scenarios may be based on real-world scenarios.
For scenario-based assessment, proper specification of scenarios is crucial since they are directly reflected in the test cases used for the assessment \autocite{stellet2015taxonomy}. 
One approach for specifying these test cases is to base them on captured scenarios from real-world data collected on the level of individual vehicles \autocite{elrofai2018scenario, putz2017pegasus, roesener2016scenariobased, deGelder2017assessment}. 

% Mention other literature that tries to extract scenarios.
Different techniques for capturing scenarios and driving maneuvers have been proposed in literature. 
\textcite{kasper2012oobayesnetworks} use object-oriented Bayesian networks for the recognition of 27 type of driving maneuvers. 
\textcite{krajewski2018highD} detect lane changes using lane crossings and \textcite{schlechtriemen2015lanechange} detect lane changes using a naive Bayes classifier and a hidden Markov model. 
%\textcite{paardekooper2019dataset6000km} present an approach for identification of scenarios and include results for scenarios labeled ``braking in front'' and ``cut in''. 
In \autocite{xie2017driving}, random forest classifiers are used for detecting accelerating, braking, and turning with features extracted using principal component analysis, stacked sparse auto-encoders, and statistical features.
In \autocite{cara2015carcyclist}, safety-critical car-cyclist scenarios are extracted from data collected by a vehicle using several machine-learning techniques, among which support vector machines and multiple instance learning.

% Contribution of this paper.
In this paper, we propose a new method for mining scenarios from real-world driving data using automated tagging and searching for combination of tags. 
Our method consists of two steps. 
First, the data is automatically tagged with relevant information. For example, a tag ``lane change'' is added to a vehicle at the time this vehicle is performing a lane change. 
Second, the scenarios are mined based on the aforementioned tags. \cenda
\cstartd To do this, we represent a scenario using a combination of tags and we search for this combination of tags in the tagged data from the previous step. \cendd

% Advantages of our method:
% 1. Tags are pretty basic --> easy.
% 2. Tagging can be very different, depending on the type of data --> scenario mining still the same!
% 3. Accuracy: by not only relying on past data, accuracy is improved.
% 4. Scalable: many more type of scenarios could be extracted.
\cstarta The proposed method brings several benefits. 
First, by tagging the data, characteristics that are shared among different type of scenarios need to be identified only once, whereas these characteristics would be identified multiple times if each type of scenarios would be identified completely independently. \cenda
\cstartf For example, a characteristic could be the presence of a lead vehicle, so if we independently identify two different types of scenarios that consider a lead vehicle, we would identify the lead vehicle two times. \cendf
\cstarta Second, by splitting the process in two parts, i.e., the tagging and the scenario mining, the scenario mining can be applied to different types of data (e.g., data from a vehicle \autocite{paardekooper2019dataset6000km} or a measurement unit above the road \autocite{kovvali2007video,krajewski2018highD}). 
It is also possible to have manually tagged data, e.g., see \autocite{fontana2018action}. 
%Thirdly, because the scenario mining is performed offline, we do not only rely on past data, which, in turn, increases the accuracy of the scenario mining. 
Third, our approach is easily scalable because additional types of scenarios can be mined by  describing them as a combination of (sequential) tags. \cenda
\cstartf Fourth, the approach reveals promising future possibilities, such as the generation of scenarios based on the mined scenarios. \cendf
\cstartg The generated scenarios can be used to define the test cases for the assessment of intelligent vehicles \autocite{elrofai2018scenario, putz2017pegasus, roesener2016scenariobased, deGelder2017assessment, stellet2015taxonomy, zhao2018evaluation}. \cendg

% Structure.
\cstarta In \cref{sec:problem}, we formulate the problem of scenario mining. \Cref{sec:tagging,sec:mining} describe the two steps of our proposed method, i.e., the tagging of the data and the scenario mining based on these tags. 
We illustrate the proposed scenario mining approach with few examples in \cref{sec:case study}. \cenda
\cstartf In \cref{sec:discussion}, we discuss the approach, results, and some possible future improvements. \cendf
We end this paper with conclusions and discuss next steps in \cref{sec:conclusions}. \cenda

\section{Problem formulation}
\label{sec:problem}

To formulate the risk quantification problem, we distinguish quantitative scenarios from qualitative scenarios, using the definitions of \emph{scenario} and \emph{scenario category} of \autocite{degelder2018ontology}:

\begin{definition}[Scenario]
	\label{def:scenario}
	A scenario is a quantitative description of the relevant characteristics of the ego vehicle, its activities and/or goals, its static environment, and its dynamic environment. In addition, a scenario contains all events that are relevant to the ego vehicle.
\end{definition}

\begin{definition}[Scenario category]
	\label{def:scenario category}
	A scenario category is a qualitative description of the ego vehicle, its activities and/or goals, its static environment, and its dynamic environment.
\end{definition}

A scenario category is an abstraction of a scenario and, therefore, a scenario category comprises multiple scenarios \autocite{degelder2018ontology}.
For example, the scenario category ``cut-in'' comprises all possible cut-in scenarios. 
Given such a scenario category, our goal is to find all corresponding scenarios in a given data set. 

The main objective of this paper is to quantify the risk of a system-under-test when operating in all possible scenarios comprised by a given scenario category. The focus is on the system-under-test itself, its occupants, and its direct surroundings. Hence, we define the risk quantification problem as follows:

\begin{problem}[Risk quantification]
	\label{problem:risk quantification}
	Given a scenario category, how to quantify the risk on vehicle level?
\end{problem}

Risk is typically a combination of the likelihood and the severity of any harm. In our case, the likelihood of any harm can be decomposed in two components: the likelihood of a scenario of the given scenario category and the likelihood of any harm if such a scenario occurs. Based on this observation, we can divide \cref{problem:risk quantification} into three sub-problems:

\begin{problem}[Exposure]
	\label{problem:exposure}
	Given a scenario category, how to find all scenarios that correspond to this scenario category in a given data set?
\end{problem}

\begin{problem}[Harm likelihood]
	\label{problem:controllability}
	Given a scenario category and a system-under-test, how to estimate the likelihood of any harm?
\end{problem}

\begin{problem}[Severity]
	\label{problem:severity}
	Given a scenario category, a system-under-test, and a possible harmful outcome, how to estimate the severity of that harm?
\end{problem}

\section{Nomenclature} % Erwin (use the other paper) & Arash
\label{sec:definitions}

TODO: give more examples
First, we introduce the following notions:
\begin{enumerate}
\item Risk = Severity $\times$ Probability
\item Severity: a quantitative measure, depending on the impact velocity.
\item Condition:
\item Actor: the relevant actors here are the lead vehicle (denoted with L), the following vehicle (F) and the communication unit (V2V)
\item Activity: the relevant actions here are braking, and failure.
\item Scenario:
\end{enumerate}

\cstartb
\section{Object-oriented framework for scenarios}
\label{sec:oo framework}
\cendb

We have already explained the use of \cstartb \iac{oof} \cendb in \cref{sec:why oo framework}. In this section, we present our \cstartb \ac{oof} \cendb for scenarios for the assessment of \acp{av}. 
The \cstartb overview of the framework \cendb is formally represented through \cstartb class diagrams \cendb that are briefly presented in \cref{sec:class diagram}. Next, in \cref{sec:domain scenario category}, we explain how a scenario category is formally represented \cstartb in our framework\cendb. Similarly, in \cref{sec:domain scenario}, we describe how a scenario is formally represented. 



\cstartb
\subsection{Class diagrams}\cendb
\label{sec:class diagram}

In \cref{fig:class overview,fig:class relations}, the blue blocks represent the classes\footnote{In the remainder of this paper, when referring to (an instance of) a class, italic font is used. \cstarte Additionally, class names start with capital letters and instance names with lowercase letters.\cende} that are used to describe a scenario category according to \cref{def:scenario category} and the orange blocks represent the classes that are used to describe a scenario according to \cref{def:scenario}. \cstartb The yellow blocks represent so-called abstract classes. Abstract classes cannot be instantiated. \cendb

\cstartd \Cref{fig:class overview} shows the class-level relationships while \cref{fig:class relations} shows the instance-level relationships.
In \cref{fig:class overview}, the arrow from, e.g., \textit{Scenario} to \textit{Time interval}, denotes that \textit{Scenario} is a subclass of \textit{Time interval}. Therefore, all properties of the \textit{Time interval} are inherited by \textit{Scenario}. 
The arrow with the diamond in \cref{fig:class relations} denotes an aggregation\footnote{This is typically implemented using pointers. For example, an aggregation arrow from \textit{B} to \textit{A} means that an object of class \textit{A} contains a pointer to an object of class \textit{B}.}.
This means that, e.g., an \textit{actor}, which is an instance of the \textit{Actor} class, has an \textit{actor category} as an attribute. \cendd
Here, the ``\hasone'' at the start of the arrow from \textit{Actor category} to \textit{Actor} indicates that an \textit{actor} has \cstartd exactly \cendd one \textit{actor category}.
\cstartb Similarly, ``\hastwo'' at the aggregation arrow from \textit{Event} to \textit{Time interval} indicates that a \textit{time interval} contains two \textit{events}, i.e., the events that define the start and the end of the time interval. \cendb 
A ``\hasn'' at the start of an aggregation arrow indicates that an object has zero, one, or multiple objects of the corresponding class.
The arrow with the text ``comprises'' and ``includes'' represent methods that are explained in \cref{sec:scenario category}. Here, ``comprises'' can be denoted by $\comprises$ and ``includes'' can be denoted by $\includes$, see \cref{eq:scenario category include}. 

\begin{figure*}[t]
	\centering
	\begin{tikzpicture}
\tikzstyle{every node}=[font=\footnotesize]

% Abstract classes
\node[class, fill=abstractclass](thing) at (0, 0) {\textit{Scenario element}};
\node[class, fill=abstractclass](qualitative) at (-1.5\blockx, 0) {\textit{Qualitative element}};
\node[class, fill=abstractclass](quantitative) at (1.5\blockx, 0) {\textit{Quantitative element}};
\node[class, fill=abstractclass](timeinterval) at (.5\blockx, \blocky) {\textit{Time interval}};
\node[class, fill=abstractclass](model) at (-\blockx, 2\blocky) {\textit{Model}};

% Qualitative classes
\node[class, fill=scenariocategory](scencat) at (-0.5\blockx, \blocky) {Scenario category};
\node[class, fill=category](physicalcat) at (-2.5\blockx, \blocky) {Physical element category};
\node[class, fill=category](activitycat) at (-1.5\blockx, \blocky) {Activity category};
\node[class, fill=category](actorcat) at (-2.5\blockx, 2\blocky) {Actor category};
\node[class, fill=category](sinusoidal) at (-2\blockx, 3\blocky) {Sinusoidal};
\node[class, fill=category](linear) at (-\blockx, 3\blocky) {Linear};
\node[class, fill=category](constant) at (0, 3\blocky) {Constant};

% Quantitative classes
\node[class, fill=scenario](scenario) at (.5\blockx, 2\blocky) {Scenario};
\node[class, fill=otherclass](physical) at (2.5\blockx, \blocky) {Physical element};
\node[class, fill=otherclass](activity) at (1.5\blockx, 2\blocky) {Activity};
\node[class, fill=otherclass](event) at (1.5\blockx, \blocky) {Event};
\node[class, fill=otherclass](actor) at (2.5\blockx, 2\blocky) {Actor};

% Superclass arrows
\foreach \fromclass/\toclass in {scencat/qualitative,
								 activitycat/qualitative,
								 physicalcat/qualitative,
								 timeinterval/quantitative,
								 sinusoidal/model,
								 linear/model,
								 constant/model,
								 scenario/timeinterval,
								 activity/timeinterval,
								 event/quantitative,
								 physical/quantitative} {
	\node[coordinate, above of=\fromclass, node distance=-.4\blocky](helper \fromclass){};
	\draw[superclass] (\fromclass) -- (helper \fromclass) -| (\toclass);
}
\foreach \fromclass/\toclass in {qualitative/thing, 
								 quantitative/thing,
								 actorcat/physicalcat,
								 actor/physical} {
	\draw[superclass] (\fromclass) -- (\toclass);
}
\draw[superclass] (model) |- (helper activitycat) -- (qualitative);

% Legend
\node[class, fill=TNOlightgray, minimum height=6.5em, text width=2\blockwidth+1em](legend) at (2\blockx, 2.95\blocky) {};
\node[yshift=2.65em] at (legend) {Legend};
\node[class, fill=abstractclass, minimum height=1.5em, text width=2\blockwidth, yshift=1.3em](abstract) at (legend) {\textit{Abstract class}};
\node[class, fill=category, minimum height=1.5em, text width=2\blockwidth, yshift=-.4em](category) at (legend) {Class for qualitative description};
\node[class, fill=otherclass, minimum height=1.5em, text width=2\blockwidth, yshift=-2.1em](category) at (legend) {Class for quantitative description};

\end{tikzpicture}
	\caption{Schematic overview of most classes of our \acf{oof}.}
	\label{fig:class overview}
\end{figure*}

\begin{figure*}[t]
	\centering
	\setlength{\blockwidth}{7.1em}
\begin{tikzpicture}
\tikzstyle{every node}=[font=\footnotesize]

% Qualitative classes
\node[class, fill=scenariocategory](scenario category) at (.5\blockx,0) {Scenario category};
\node[class, fill=category](dynamiccategory) at (\blockx, \blocky) {Dynamic physical thing category};
\node[class, fill=category](actorcategory) at (2\blockx, \blocky) {Actor category};
\node[class, fill=category](activitycategory) at (3\blockx, \blocky) {Activity category};
\node[class, fill=abstractclass](model) at (2.5\blockx+0.25\blockwidth, 2\blocky) {Model};
\node[class, fill=category](staticcategory) at (4\blockx, \blocky) {Static environment category};
\node[class, fill=category](staticthingcategory) at (3.5\blockx+0.25\blockwidth, 2\blocky) {Static physical thing category};

% Quantitative classes
\node[class, fill=scenario](scenario) at (.5\blockx, 2\blocky) {Scenario};
\node[class, fill=otherclass](dynamic) at (\blockx, 3\blocky) {Dynamic physical thing};
\node[class, fill=otherclass](actor) at (2\blockx, 3\blocky) {Actor};
\node[class, fill=otherclass](activity) at (3\blockx, 3\blocky) {Activity};
\node[class, fill=otherclass](static) at (4\blockx, 3\blocky) {Static environment};
\node[class, fill=otherclass](event) at (0.5\blockx, 4\blocky) {Event};
\node[class, fill=abstractclass](timeinterval) at (1.5\blockx+0.25\blockwidth, 4\blocky) {Time interval};
\node[class, fill=otherclass](staticthing) at (3.5\blockx+0.25\blockwidth, 4\blocky) {Static physical thing};

% Aggregation arrows for the scenario category
\node[coordinate, below of=scenario category, node distance=-\blocky/2, xshift=\blockwidth/3](helper scenario category){};
\node[coordinate, below of=scenario category, node distance=\blockheight/2, xshift=\blockwidth/3](aggregation scenario category){};
\foreach \class in {dynamic, actor, activity, static}
{
	\node[coordinate, above of=\class category, node distance=\blockheight/2](helper \class){};  % Needed for later
	\draw[aggregation] (\class category) |- (helper scenario category) -- (aggregation scenario category);
}
\node[anchor=south east] at (helper dynamic) {\hasn};
\node[anchor=south east] at (helper actor) {\hasn};
\node[anchor=south east] at (helper static) {\hasone};
\node[anchor=south east] at (helper activity) {\hasn};

% Aggregation arrow for the model and static thing category
\foreach \fromclass/\toclass in {model/activitycategory, 
								 staticthingcategory/staticcategory,
								 staticthing/static} {
	\node[coordinate, above of=\fromclass, node distance=\blockheight/2, xshift=-\blockwidth/8+\blockx/4](aggregation \fromclass){};
	\node[coordinate, below of=\toclass, node distance=\blockheight/2, xshift=\blockwidth/8-\blockx/4](aggregation \toclass){};
	\draw[aggregation] (aggregation \fromclass) -- (aggregation \toclass);
}
\node[anchor=south east] at (aggregation model) {\hasone};
\node[anchor=south east] at (aggregation staticthingcategory) {\hasn};
\node[anchor=south east] at (aggregation staticthing) {\hasn};

% Aggregation arrow for scenario
\node[coordinate, below of=scenario, node distance=-\blocky/2](helper scenario){};
\node[coordinate, below of=scenario, node distance=\blockheight/2](aggregation scenario){};
\foreach \class in {dynamic, actor, activity, static, event}
{
	\node[coordinate, above of=\class, node distance=\blockheight/2, xshift=-\blockwidth/4](helper \class){};
	\draw[aggregation] (helper \class) |- (helper scenario) -- (aggregation scenario);
}
\node[anchor=south east] at (helper dynamic) {\hasn};
\node[anchor=south east] at (helper actor) {\hasn};
\node[anchor=south east] at (helper activity) {\hasn};
\node[anchor=south east] at (helper static) {\hasone};
\node[anchor=south east] at (helper event) {\hasn};

% Aggregations for static environment, activity, and actor
\foreach \class in {dynamic, static, activity, actor}
{
	\node[coordinate, below of=\class category, node distance=\blockheight/2, xshift=\blockwidth/4](category helper){};
	\node[coordinate, above of=\class, node distance=\blockheight/2, xshift=\blockwidth/4](helper){};
	\draw[aggregation] (category helper) -- (helper);
	\node[anchor=north east] at (category helper) {\hasone};
}

% Aggregation for static physical thing category -> static physical thing
\node[coordinate, below of=staticthingcategory, node distance=\blockheight/2, xshift=-\blockwidth/4](category helper){};
\node[coordinate, above of=staticthing, node distance=\blockheight/2, xshift=-\blockwidth/4](helper){};
\draw[aggregation] (category helper) -- (helper);
\node[anchor=north east] at (category helper) {\hasone};

% Aggregation for event -> time interval
\draw[aggregation] (event) -- (timeinterval);
\node[coordinate, right of=event, node distance=\blockwidth/2](helper event) {};
\node[anchor=south west] at (helper event) {\hastwo};

% falls into arrows
\node[coordinate, right of=scenario category, node distance=\blockwidth/2+1pt, yshift=-\blockheight/3](helper1){};
\node[coordinate, right of=scenario category, node distance=\blockwidth/2+1pt, yshift=\blockheight/3](helper2){};
\node[coordinate, right of=helper1, node distance=\blockwidth/2](helper3){};
\node[coordinate, right of=helper2, node distance=\blockwidth/2](helper4){};
\draw[falls into] (helper1) -- (helper3) -- node[fill=white]{includes} (helper4) -- (helper2);
\node[coordinate, above of=scenario, node distance=\blockheight/2, xshift=-.28\blockwidth](helper1){};
\node[coordinate, below of=scenario category, node distance=\blockheight/2, xshift=-.28\blockwidth](helper2){};
\draw[falls into] (helper2) -- node[fill=white, align=center, text width=3.55em]{comprises} (helper1);

\end{tikzpicture}
	\caption{Schematic overview of the relation between the classes for representing the scenarios for the assessment of \acp{av}.}
	\label{fig:class relations}
\end{figure*}



\subsection{Scenario category and its attributes}
\label{sec:domain scenario category}

\cstartb Because all other classes in \cref{fig:class overview} are subclasses of \textit{Scenario element}, these classes inherit the attributes and procedures of \textit{Scenario element}. In our framework, a \textit{scenario element} has a human-interpretable name, a unique ID, and possibly predefined tags that are also interpretable by a software agent. So, all other classes in \cref{fig:class overview} also have these attributes. \cendb

The static environment is qualitatively described by \cstarte one or more \textit{physical element categories}. \cende
Because the \cstarte\textit{physical element categories} \cende qualitatively describe the static environment, they contain a human-interpretable description of the \cstartc physical things they describe\cendc.

The ego vehicle\cstartd(s) \cendd and the dynamic environment are qualitatively described by \textit{activity categories} and \textit{actor categories}. 
In line with \cref{def:activity}, \textit{Activity category} includes the state variable(s).
The \textit{Model} that is used to describe the time evolution of the state variable(s) is specified. 
\cstarte Note that a \textit{model} is an abstract class that serves as a template for different models, such as the three examples shown in \cref{fig:class overview}: \textit{Sinusoidal}, \textit{Linear}, and \textit{Constant}. 
Let $\state(\time)$ denote the state variable(s) at time $\time$, then the \textit{Sinusoidal} model is defined as follows:
\begin{align}
	\statedot(\time) &= \frac{\pi \amplitude}{2\duration} \sin \left( \frac{\pi \left( \time - \timeinit\right)}{\duration} \right),\ \time \in [\timeinit, \timeinit+\duration], \label{eq:sinusoidala} \\
	\state(\timeinit) &= \stateinit. \label{eq:sinusoidalb}
\end{align}
Here, the amplitude ($\amplitude$), duration ($\duration$), initial time ($\timeinit$), and initial state ($\stateinit$) are parameters. 
The \textit{Linear} and \textit{Constant} models are described by the following equations, respectively:
\begin{align}
	\statedot(\time) &= \slope,\ \state(\timeinit) = \stateinit, \label{eq:linear} \\
	\state(\time) &= \stateinit. \label{eq:constant}
\end{align}
The \textit{Linear} model contains three parameters, i.e., the slope ($\slope$), initial time ($\timeinit$), and initial state ($\stateinit$). The \textit{Constant} model only has the parameter $\stateinit$.
Since an \textit{activity category} is a qualitative description, the values of the parameters of its \textit{model} are not part of the \textit{activity category}. \cende

The \textit{Actor category} has an attribute that specifies the type of object.
To indicate that an actor is an ego vehicle, the tag ``Ego vehicle'' is added to the list of tags of the \textit{actor category}.

% Scenario category
The \textit{Scenario category} has \cstarte \textit{physical element categories}\cende, \textit{activity categories}, and \textit{actor categories} as attributes. 
%As with the other classes, a \textit{scenario category} contains a name and may contain predefined tags that describe parts of the scenario that are not described by the other classes.
Another attribute of the \textit{Scenario category} is the list of acts. %\footnote{ In line with the definition of \emph{act} in \cref{sec:act}, for a \textit{scenario category}, an \emph{act} is a combination of \textit{activity categories} and \textit{actor categories}.}. 
These acts describe which actors perform which activities. Naturally, it is possible that one actor performs multiple activities and that one activity is performed by multiple actors.

% Explain why we have these different classes
The reader might wonder why we introduce the different classes for describing a scenario category, i.e., the blue blocks, instead of only one class for modeling a scenario category. 
The main advantage of the different classes is the re-usability of the instances of the classes, because these instances can be exchanged among different \textit{scenario categories}. For example, if two \textit{scenario categories} have the same \textit{actor categories}, we only need to define the \textit{actor categories} once, whereas if the \textit{actor categories} would not be a class on its own but only a property of the scenario category, we would need to define the \textit{actor categories} twice.



\subsection{Scenario and its attributes}
\label{sec:domain scenario}

\cstartb The class \textit{Scenario} is a subclass of \textit{Time interval} and, therefore, it has \textit{events} that define the start and the end of the scenario. \cendb
The \textit{Scenario} also has \cstarte\textit{physical element}\cende, \textit{activities}, \textit{actors}, and \textit{events} as attributes. 
%The ego vehicle and the dynamic environment are quantitatively described by activities and actors. 
It is similar to the \textit{Scenario category}, as the \cstarte\textit{physical elements}\cende, \textit{activities}, and \textit{actors} are the quantitative counterparts of the \cstarte\textit{physical element categories}\cende, \textit{activity categories}, \cendb and \textit{actor categories}, just as a \textit{scenario} is the quantitative counterpart of a \textit{scenario category}. 
As with the \textit{Scenario category}, the \textit{Scenario} contains a list of acts that describe which actors perform which activities.

% Static environment
%The \textit{static environment} ``has'' a \textit{static environment category}. The most notable difference between the \textit{static environment category} and the \textit{static environment}, is that the \textit{static environment} \cstartb ``has'' \textit{static physical things} instead of \textit{static physical thing categories}. \cendb
\cstartb A \textit{physical element} has a \textit{physical element category} and it may have multiple properties that quantitatively define the object, such as its \cendb size, weight, color, radar cross section, etc.  \cendb Physical elements can be used to define, e.g., the road layout, static weather and lighting conditions, and infrastructural elements.

% Activity
According to \cref{def:activity}, an activity quantitatively describes the evolution of one or more state variables in a time interval. The state variable(s) are defined by the \textit{activity category} that the \textit{activity} has as an attribute. Together with the \textit{model} that is contained by the \textit{activity category}, the time evolution of the state variable is described by a set of parameters. The values of the parameters are part of the \textit{activity}. 
%\cstartb Note that because the class \textit{activity} is a subclass of \textit{time interval}, it has two events defining the start and the end of the \textit{activity}. \cendb
%Note that the model can be a time interpolation of (measured) data points, in which case the parameters are a sequence of time and state values. The time interval is defined by a duration of the activity.

% Set and triggered activity
%Two different types of activities can be defined. A set activity describes an activity that happens at a certain fixed time. This is often used to describe real-world scenarios that are extracted from real-world data. On the other hand, the starting time of a triggered activity is in general not defined beforehand, as the activity is triggered by an event. This is often used to describe test cases for scenario-based testing, e.g., see the example presented in \cref{sec:example test case}. 
%%Here, the starting time of the activity of the pedestrian, i.e., walking on the pedestrian crossing, is not defined beforehand and depends on, e.g., the speed of the ego vehicle \autocite{seiniger2015test}. 
%Both the \textit{set activity} and the \textit{triggered activity} are subclasses of the \textit{activity}, as shown in \cref{fig:framework classes}. Additionally, the \textit{set activity} ``has'' a starting time and the \textit{triggered activity} ``has'' an \textit{event} that triggers the activity.

% Event
Following \cref{def:event}, an \textit{event} contains conditions that describe the threshold or mode transition at the time of the \textit{event}.

% Actor
Similar to a \textit{physical element} and an \textit{activity}, an \textit{actor} has its qualitative counterpart --- an \textit{actor category} --- as an attribute. Additionally, the \textit{Actor} contains an initial state vector and a desired state vector, that can be used to specify the intent, as attributes.
\cendb Describing the intent is especially useful for defining a test case in terms of the objective of the ego vehicle rather than its activities. \cendb
%Additionally, the goals can be formulated as text if they cannot be formulated using a desired state vector.
%Note that the goals are typically on a strategic level, such as destinations and waypoints, as opposed to the tactical and operational levels\footnote{ See \autocite[p.~7, Figure 1]{sae2018j3016} for an overview of the difference between the strategic, tactical, and operational functions.}.

%According to \cref{def:event}, an events marks the time instant at which the system reaches a specified boundary or at which a mode transition occurs. To describe this time instant, one or more conditions are specified. For example, a condition could be that the distance of the ego vehicle and a pedestrian crossing should be less than \SI{30}{\meter}. For this example, the moment at which this condition is met for the first time corresponds to the event.

% Another advantage of the blue blocks -> only one activity category for multiple activities
An advantage of having the qualitative counterparts of the \cstarte \textit{Physical element}\cende, \textit{Activity}, and \textit{Actor} is that the qualitative description can be reused and exchanged. For example, there can be many different braking activities, but there needs to be only one \textit{activity category} for qualitatively defining the braking activity. Here, it is assumed that all braking activities are modeled with the same model and that similar tags apply. If this is not the case, multiple \textit{activity categories} need to be defined, but the number of \textit{activity categories} will still be substantially lower than the number of activities.


\section{Case Study} % Erwin & Hala
\label{sec:example}

In this section, we present a case study to illustrate the method of quantifying the risk for a type of scenario, i.e., a scenario class \cite{elrofai2018scenario}. We will first explain the scenario class and the use case. The actual system for which the risk is computed is presented in \cref{sec:system}. Next, we will describe the conditions and activities in \cref{sec:conditions,sec:activities}, respectively. Finally, we will present the results in \cref{sec:results}.

\subsection{The scenario class and its use case}
\label{sec:scenario class}

We want to quantify the risk for scenarios that are linguistically described as follows: while the ego vehicle drives at a moderate to high speed while staying in its lane, another vehicle cuts into the lane of the ego vehicle, such that this vehicle becomes the ego vehicle's lead vehicle. Furthermore, the ego vehicle needs to brake to prevent a collision.

For the quantification of the risk, 60 hours of data (see also \cite{deGelder2017assessment}) are collected by driving a specific route in and between Eindhoven and Helmond, The Netherlands, with twenty different drivers, each driving the route twice. Therefore, it is assumed that the use case of the automation system is simply driving this route. We will use the data for the estimation of the risk. Hence, we will make use of the following assumption:
\begin{assumption}
	The recorded naturalistic driving data is representative for what a vehicle with an automation system might encounter along the same route.
\end{assumption}

\subsection{System-under-test}
\label{sec:system}

To reduce efforts for the assessment, often simulations are employed. However, even simulations can consume considerable time, as these simulations might run real-time \cite{shah2018airsim} or slower when a higher level of detail is used \cite{zofka2016testing}. For our method, we will simplify the simulations, such that the total required time on a common computer is in the order of minutes. Since we are interested in approximate results, a high level of detail is not required. 

To simplify the system-under-test, it is assumed that the system's desired acceleration is similar to the adaptive cruise control defined in \cite{deGelder2017assessment}, i.e.,
\begin{equation}
	\label{eq:desired acceleration} 
	u(t) = k_{\mathrm{d}}(v(t))(d(t) - \tau_{\mathrm{h}} v(t) - s_0) + k_{\mathrm{v}}\left(\dot{d}(t) - ha(t) \right),
\end{equation}
with
\begin{equation}
	\label{eq:gain}
	k_{\mathrm{d}}(v(t)) = k_{\mathrm{d1}} + \left( k_{\mathrm{d2}} - k_{\mathrm{d1}} \right) \exp \left\{ -\frac{v(t)^2}{2\sigma_{\mathrm{d}}} \right\}.
\end{equation}
Here, $v$ is the speed of the ego vehicle, $d$ denotes the clearance between the ego vehicle and its predecessor, i.e., the vehicle that performs the cut-in. The relative speed is denoted by $\dot{d}$ and $a$ refers to the acceleration of the ego vehicle. The ego vehicle is modeled using a first order model with a time delay, i.e.,
\begin{equation}
	\label{eq:vehicle model}
	\tau \dot{a}(t) + a(t) = u(t - \theta).
\end{equation}
Furthermore, the deceleration is limited at \unit[-6]{ms^{-2}}. A description of the constants of \cref{eq:desired acceleration,eq:gain,eq:vehicle model} are listed in \cref{tab:constants}.

\begin{table}
	\centering
	\caption{The constants used for the simple automation system of \cref{eq:desired acceleration,eq:gain,eq:vehicle model}.}
	\label{tab:constants}
	\begin{tabular}{clc}
		\toprule
		Parameter & Description & Value \\ \otoprule
		$\tau_{\mathrm{h}}$ & Desired time headway & \unit[1.0]{s} and \unit[2.0]{s} \\
		$s_0$ & Safety distance & \unit[1.5]{m} \\
		$k_{\mathrm{d1}}$ & Distance gain at high speed & $\unit[0.7]{s^{-2}}$ \\
		$k_{\mathrm{d1}}$ & Distance gain at low speed & $\unit[2.0]{s^{-2}}$ \\
		$\sigma_{\mathrm{d}}$ & Shaping coefficient of distance gain & $\unit[5]{ms^{-1}}$ \\
		$k_{\mathrm{v}}$ & Speed difference gain & $\unit[0.35]{s^{-1}}$ \\
		$\tau$ & Time constant of vehicle model & \unit[0.1]{s} \\
		$\theta$ & Delay of the vehicle response & \unit[0.2]{s} \\
		\bottomrule
	\end{tabular}
\end{table}

\subsection{Conditions}
\label{sec:conditions}

All scenarios are subject to the following conditions:
\begin{itemize}
	\item $C_1$: The speed of the ego vehicle is within the range of \unit[60]{km/h} and \unit[130]{km/h}.
	\item $C_2$: There are no restrictions on the weather conditions.
	\item $C_3$: There are no restrictions on the lighting conditions.
\end{itemize}

Obviously, because there are no restrictions to the weather and lighting conditions, we have $P(C_2,C_3)=1$. For the first condition, we can use the data to estimate the likelihood. The data, however, has been recorded during sunny weather at daylight. Therefore, we need to following assumption.

\begin{assumption} \label{asm:conditions}
	Let $C_2'$ and $C_3'$ denote the conditions of having sunny weather and daylight, respectively. Then we have $P(C_1|C_2,C_3)=P(C_1|C_2',C_3')$.
\end{assumption}

From the data, it appeared that $P(C_1|C_2',C_3')=0.20$. Using \cref{asm:conditions}, we have
\begin{dmath}
	P(C) = P(C_1,C_2,C_3)=P(C_1|C_2',C_3')\cdot P(C_2,C_3)=0.20.
\end{dmath}

\subsection{Activities}
\label{sec:activities}

\subsection{Results}
\label{sec:results}


%\section{Discussion and future outlook} % Hala & Erwin & Arash
\label{sec:discussion}

In the calculation of the estimated risk, many assumptions are made to simplify the calculation or because there are unknowns due to lack of data. This reduces the accuracy of the estimated risk. 
However, we still believe that the quantified risk is valuable because of the following reasons:
\begin{itemize}
	\item All the assumptions that were made for estimating the risk are explicit. In contrast, when people assign ASIL levels to a hazardous event, often many assumptions are implicitly made. By making the assumptions explicit, it is much clearer why a certain risk is associated with --- in this case --- a certain/specific scenario.
	\item It provides a good estimate of the order of the risk for each scenario. For example, if the estimated risk of two different scenarios differ by a factor 10, it is still reasonable to argue that one scenario introduces a higher risk than the other, even though the factor 10 might not be exact.
	\item Because our/the proposed method explicates all the steps and assumptions that lead to a certain estimated risk, it is easily possible to update the risk when more information of the system is known or when more data is available.
\end{itemize}

To be discussed:
\begin{itemize}
	\item Method gives only order of risk.
	\item ``Controllability'' not considered.
	\item A lot of assumptions: with this method, these assumptions are made explicit, whereas often people make these assumptions implicit (and implicit assumptions are the mother of all fuck-ups; should be rephrased :)).
\end{itemize}

\section{Conclusions}
\label{sec:conclusions}

\cstarte
% Summarize two-step approach: tagging and mining based on tags.
For the scenario-based assessment of automated vehicles, scenarios captured from real-world data collected on the level of individual vehicles can be used to define the tests.
We have proposed a two-step approach for mining real-world scenarios from a data set.
The first step consists in labeling the data with tags that describe, e.g., the lateral and longitudinal activities of the different actors.
The second step mines the scenarios by searching for particular combinations of tags.
We have illustrated the approach with two examples, a cut in and an overtaking before a lane change. \cende
\cstartf These examples demonstrated that the proposed approach is suitable for mining scenarios from real-world data.
% Improve tagging, e.g., using machine learning techniques.
% Have more tags!
Future work includes labeling the data with more tags and exploring the possibilities of using techniques that are used in the field of natural language processing.
\cendf

\section*{Acknowledgment}

The research leading to this paper has been partially realized with the Centre of Excellence for Testing and Research of Autonomous Vehicles - NTU (CETRAN). Responsibility for the information and view set out in this publication lies entirely with the authors. 



%\addtolength{\textheight}{-12cm}  % This command serves to balance the column lengths
                                  % on the last page of the document manually. It shortens
                                  % the textheight of the last page by a suitable amount.
                                  % This command does not take effect until the next page
                                  % so it should come on the page before the last. Make
                                  % sure that you do not shorten the textheight too much.

\printbibliography
%\bibliographystyle{abbrvnat}
%\bibliography{../bib}

\appendices
\crefalias{section}{appendix}
\section{Nomenclature}
\label{sec:nomenclature}

For the definition of \emph{scenario}, several notions are adopted from literature. 
In this section, the concepts of \emph{ego vehicle}, \emph{actor}, \emph{state variable}, \emph{state vector}, \emph{model}, \emph{mode}, \emph{act}, \emph{static environment}, and \emph{dynamic environment}, \cstartb which are adopted from literature\cendb, are detailed. 

\subsection{Ego vehicle}
\label{sec:ego vehicle}

The ego vehicle is the main subject of a scenario. In particular, the ego vehicle refers to the vehicle that is perceiving the world through its sensors (e.g., see \autocite{Bonnin2014}). When performing tests, the ego vehicle also refers to the vehicle that must perform a specific task (e.g., see \autocite{althoff2017CommonRoad, catapult2018musicc}). In this case, the ego vehicle is often referred to as the system under test \autocite{stellet2015taxonomy}, the vehicle under test \autocite{gietelink2006development}, or the host vehicle \autocite{gietelink2006development}.
%The ontology presented by Geyer~et~al.\ ``is described from the ego-vehicle's point of view'' \autocite{geyer2014}. 
%Note that in case a sensor-equipped vehicle is used to extract scenarios from real-world driving, the ego vehicle in an extracted scenario does not necessarily have to correspond to the sensor-equipped vehicle that is used to acquire the real-world data.

\subsection{Static environment}
\label{sec:static environment}

The static environment refers to the part of the environment that does not change during a scenario. This includes geo-spatially stationary elements \autocite{ulbrich2015},  such as the road network.
%Although one might argue whether light and weather conditions are dynamic or not \autocite{geyer2014,bach2016modelbased}, in most cases it is reasonable to assume that these conditions are not subject to significant changes during the time frame of a scenario. 
%Hence, light and weather conditions are usually part of the static environment.


\subsection{Dynamic environment}
\label{sec:dynamic environment}

As opposed to the static environment, the dynamic environment refers to the part of the environment that changes during the time frame of a scenario. 
%The dynamic environment is described using the activities that describe the way the states evolve over time. 
In practice, the dynamic environment mainly consists of the moving actors (other than the ego vehicle) that are relevant to the ego vehicle.
For example, the primary use case of OpenSCENARIO, a file format for the description of the dynamic content of driving simulations, is to describe ``complex, synchronized maneuvers that involve multiple entities like vehicles, pedestrians, and other traffic participants'' \autocite{openscenario}, so for OpenSCENARIO, these maneuvers represent the dynamic environment.
Roadside units that communicate with vehicles within the communication range \autocite{alsultan2014comprehensive} are also part of the dynamic environment. Furthermore, changing (weather) conditions are part of the dynamic environment.

\begin{remark}
	Note that it might not always be obvious whether an element of the environment belongs to the static or dynamic environment. 
	%For example, the post of a traffic light can be considered as part of the static environment, while the signal of the traffic light can be considered as part of the dynamic environment.
	Most important, however, is that all parts of the environment that are relevant to the assessment of an AV are described in either the static or the dynamic environment.
\end{remark}



\cstarte
\subsection{Physical element}
\label{sec:physical element}


A physical element refers to an object that exists in the three-dimensional space.
\cende


%\cstartc
%\subsection{Static physical thing}
%\label{sec:static physical thing}
%
%A static physical thing is a physical thing that does not experience a (relevant) change during a scenario. All static physical things form the static environment.
%
%
%
%\subsection{Dynamic physical thing}
%\label{sec:dynamic physical thing}
%
%A dynamic physical thing is a physical thing that experiences a (relevant) change during a scenario. 
%\cendc



\subsection{Actor}
\label{sec:actor}

According to \textcite{catapult2018musicc}, ``actors are all dynamic components of a scenario, excluding the ego vehicle itself.'' 
%\cstartc In this paper, we distinguish between dynamic components that might have an intent (e.g., a driver of a car, a cyclist, a pedestrian, a driving automation system, or a combination of a driver and a driving automation system \autocite{geyer2014}) and dynamic components that do not necessarily have an intent (e.g., moving tumbleweed, loose tire that got off a car). The former is called an actor where the latter is a dynamic physical thing. \cendc 
Note that, in contrast to \autocite{catapult2018musicc}, in the current paper, the ego vehicle's driver, and/or automation system are considered as actors, similar to \autocite{geyer2014},  because they have the same properties as another driver or automation system.
\cstarte While the aforementioned definition of \textcite{catapult2018musicc} provides a good idea of what an actor could be, we use another definition in order to avoid a circular definition: an actor is a dynamic physical element, i.e., a physical element that experiences change. \cende

\cstartc
\begin{remark}
	An actor is also a physical element whereas a physical element is not necessarily an actor.
\end{remark}
\cendc



\subsection{Act}
\label{sec:act}

We define acts as the combination of \cstarte actors \cende and the activities that are performed by the \cstarte actors \cende or the combination of \cstarte actors \cende and the activities they are subjected to.  %Additionally, the act can contain conditions that mark the start or end of the act.
This is in accordance with the use of the term \emph{act} in \autocite{openscenario}. 

%: ``in this case, [the actor is] the ego-vehicle with driver/automation.''
%An actor is an element of a scenario acting on its own behalf \autocite{ulbrich2015}. 
% Traffic light?
 
\subsection{State variable} 
\label{sec:state variable}
\textcite[p.~163]{dorf2011modern} write that ``the state variables describe the present configuration of a system and can be used to determine the future response, given the excitation inputs and the equations describing the dynamics.'' In our case, ``the system'' could refer to an actor, a component, or a simulation. E.g., a state variable could be the acceleration of an actor.



\subsection{State vector}
\label{sec:state vector}
A state vector refers to ``the vector containing all $n$ state variables'' \autocite[p.~233]{dorf2011modern}.



\subsection{Model}
\label{sec:model}

Typically, a dynamical system is modeled using a differential equation of the form $\statedot(\time)=\function_{\parameter}(\state(\time), \inputsystem(\time), \time)$ \autocite{norman2011control}, where $\state(\time)$ represents the state vector at time $\time$, $\inputsystem(\time)$ represents an external input vector, and the function $\function_{\parameter}(\cdot)$ is parameterized by $\parameter$.  Note that, technically speaking, $\state(\cdot)$, $\inputsystem(\cdot)$, $\time$, and $\parameter$ are inputs of the function $\function$, but $\parameter$ is assumed to be constant for a certain time interval. For example, the following first-order model is parameterized by $\parameter=(\parametera,\parameterb)$:
\begin{equation}
	\statedot(\time) = \parametera \state(\time) + \parameterb \inputsystem(\time).
\end{equation}

% The input $u$ is a function of time, that needs to be quantified. For this purpose, a parametrized function can be used, i.e. $u=g_{\theta}(t)$ with parameter vector $\theta$, such that the differential equation can be rewritten to $\dot{x}=h_{\theta}(x,t)$. In the context of this paper, model refers to a parametrized function, such as $h_{\theta}(x,t)$. It might be more practical to directly model the state (i.e., the result of the differential equation) using a function $x=k_{\theta}(t)$, such that no explicit information is required about the system dynamics. For example, see \autocite{deGelder2017assessment}.



\subsection{Mode}
\label{sec:mode}

In some systems, the behavior of the system may suddenly change abruptly, e.g., due to a sudden change in an input, a model parameter, or the model function. Such a transition is called a mode switch.
In each mode, the behavior of the system is described by a model with a fixed function $f_{\theta}$ and smooth input $u(\cdot)$ \autocite{deschutter2000optimal}.

%\subsubsection{Activity}
%\label{sec:activity}
%An activity refers to the behavior of a particular mode. For example, an activity could be described by the label `braking' or `changing lane'.
%A scenario contains the quantitative description of the ongoing activity of the ego vehicle and its dynamic environment. Here, the description refers to the changing states that are relevant for the scenario, e.g., acceleration and velocity. The activity is described using the models that describe the way the state evolves over time.




\end{document}
