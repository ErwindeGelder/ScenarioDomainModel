\section{Ontology for scenarios}
\label{sec:ontology}

We have already explained the use of an ontology in \cref{sec:why ontology}. In this section, we present our ontology for scenarios for the assessment of automated vehicles. 
The ontology is formally represented through a domain model that is briefly presented in \cref{sec:domain model}. Next, in \cref{sec:domain scenario category}, we explain how a scenario category is formally represented using the domain model. Similarly, in \cref{sec:domain scenario}, we describe how a scenario is formally represented using the domain model. 
%For more details on the domain model that is representing the ontology, see \cref{sec:appendix domain model}.



\subsection{Domain model}
\label{sec:domain model}

The classes of the domain model that are used to define a scenario and a scenario category are shown in \cref{fig:ontology classes}, where each block represents a class\cbstartd\footnote{\cbstartd In the remainder of this paper, when referring to (an instance of) a class, italic font is used.\cbend}\cbend.
The blue blocks represent the classes that are used to qualitatively describe a scenario whereas the orange blocks represent the classes that are used to quantitatively describe a scenario. 

\begin{figure}
	\centering
	\definecolor{scenarioclass}{RGB}{30, 144, 255}
\definecolor{category}{RGB}{0, 191, 255}
\definecolor{scenario}{RGB}{255, 69, 0}
\definecolor{otherclass}{RGB}{255, 127, 80}
\newlength\blockwidth
\newlength\blockheight
\newlength\blockx
\newlength\blocky
\newlength\legendwidth
\setlength{\blockwidth}{5.3em}
\setlength{\blockheight}{4em}
\setlength{\blockx}{6.4em}
\setlength{\blocky}{-7em}
\setlength{\legendwidth}{3.5em}
\tikzstyle{class}=[draw, text width=\blockwidth-.5em, align=center, minimum height=\blockheight, line width=1pt, minimum width=\blockwidth]
\tikzstyle{aggregation}=[-{Diamond[width=8pt, length=12pt, fill=white]}, line width=1pt]
\tikzstyle{falls into}=[->, line width=1pt]
\tikzstyle{superclass}=[-{Triangle[width=8pt, length=12pt, fill=white]}, line width=1pt]
\begin{tikzpicture}
% Classes
\node[class, fill=scenarioclass](scenario class) at (.5\blockx,0) {Scenario class};
\node[class, fill=category](staticcategory) at (\blockx, \blocky) {Static environment category};
\node[class, fill=category](activitycategory) at (2\blockx, \blocky) {Activity category};
\node[class, fill=category](model) at (1.5\blockx+0.25\blockwidth, 2\blocky) {Model};
\node[class, fill=category](actorcategory) at (3\blockx, \blocky) {Actor category};
\node[class, fill=scenario](scenario) at (.5\blockx, 2\blocky) {Scenario};
\node[class, fill=otherclass](static) at (\blockx, 3\blocky) {Static environment};
\node[class, fill=otherclass](activity) at (2\blockx, 3\blocky) {Acitivity};
\node[class, fill=otherclass](actor) at (3\blockx, 3\blocky) {Actor};
\node[class, fill=otherclass](triggered) at (1.5\blockx, 4\blocky) {Triggered activity};
\node[class, fill=otherclass](detected) at (2.5\blockx, 4\blocky) {Detected activity};

% Aggregation arrows for the scenario class
\node[coordinate, below of=scenario class, node distance=-\blocky/2, xshift=\blockwidth/3](helper scenario class){};
\node[coordinate, below of=scenario class, node distance=\blockheight/2, xshift=\blockwidth/3](aggregation scenario class){};
\foreach \class in {static, activity, actor}
{
	\node[coordinate, above of=\class category, node distance=\blockheight/2](helper \class){};  % Needed for later
	\draw[aggregation] (\class category) |- (helper scenario class) -- (aggregation scenario class);
}
\node[anchor=south east] at (helper static) {1};
\node[anchor=south east] at (helper activity) {$\mathbb{N}$};
\node[anchor=south east] at (helper actor) {$\mathbb{N}$};

% Aggregation arrow for the model
\node[coordinate, above of=model, node distance=\blockheight/2, xshift=-\blockwidth/8+\blockx/4](aggregation model){};
\node[coordinate, below of=activitycategory, node distance=\blockheight/2, xshift=\blockwidth/8-\blockx/4](aggregation activity category){};
\draw[aggregation] (aggregation model) -- (aggregation activity category);

% Aggregation arrow for scenario
\node[coordinate, below of=scenario, node distance=-\blocky/2](helper scenario){};
\node[coordinate, below of=scenario, node distance=\blockheight/2+1pt](aggregation scenario){};
\foreach \class in {static, activity, actor}
{
	\node[coordinate, above of=\class, node distance=\blockheight/2, xshift=-\blockwidth/4](helper \class){};
	\draw[aggregation] (helper \class) |- (helper scenario) -- (aggregation scenario);
}
\node[anchor=south east] at (helper static) {1};
\node[anchor=south east] at (helper activity) {$\mathbb{N}$};
\node[anchor=south east] at (helper actor) {$\mathbb{N}$};

% Aggregations for static environment, activity, and actor
\foreach \class in {static, activity, actor}
{
	\node[coordinate, below of=\class category, node distance=\blockheight/2, xshift=\blockwidth/4](category helper){};
	\node[coordinate, above of=\class, node distance=\blockheight/2, xshift=\blockwidth/4](helper){};
	\draw[aggregation] (category helper) -- (helper);
	\node[anchor=north east] at (category helper) {1};
}

% falls into arrows
\node[coordinate, right of=scenario class, node distance=\blockwidth/2+1pt, yshift=-\blockheight/3](helper1){};
\node[coordinate, right of=scenario class, node distance=\blockwidth/2+1pt, yshift=\blockheight/3](helper2){};
\node[coordinate, right of=helper1, node distance=\blockwidth/2](helper3){};
\node[coordinate, right of=helper2, node distance=\blockwidth/2](helper4){};
\draw[falls into] (helper1) -- (helper3) -- node[fill=white]{falls into} (helper4) -- (helper2);
\node[coordinate, above of=scenario, node distance=\blockheight/2+1pt, xshift=-\blockwidth/3](helper1){};
\node[coordinate, below of=scenario class, node distance=\blockheight/2+1pt, xshift=-\blockwidth/3](helper2){};
\draw[falls into] (helper1) -- node[fill=white, align=center]{falls\\into} (helper2);

% Superclass arrows
\node[coordinate, below of=activity, node distance=-.6\blocky](helper activity){};
\draw[superclass] (triggered) |- (helper activity) -- (activity);
\draw[superclass] (detected) |- (helper activity) -- (activity);

% Legend
\node[coordinate](legend) at (1.8\blockx, -.5em) {};
\node[draw, left of=legend, node distance=0.3em, minimum height=4em, minimum width=\legendwidth+6em, anchor=west, fill=gray!10]{};
\node[coordinate, right of=legend, node distance=\legendwidth](legend right){};
\draw[aggregation] (legend) -- (legend right);
\node[right of=legend right, node distance=0em, anchor=west]{Aggregation};
\node[coordinate, below of=legend, node distance=1.2em](helper1){};
\node[coordinate, below of=legend right, node distance=1.2em](helper2){};
\draw[superclass] (helper1) -- (helper2);
\node[right of=helper2, node distance=0em, anchor=west]{Superclass};
\node[coordinate, above of=legend, node distance=1.2em](helper1){};
\node[coordinate, above of=legend right, node distance=1.2em](helper2){};
\draw[falls into] (helper1) -- (helper2);
\node[right of=helper2, node distance=0em, anchor=west]{Method};

\end{tikzpicture}
	\caption{\cbstart Schematic overview of most classes of the domain model for representing the ontology for scenarios for the assessment of automated vehicles. The aggregation arrow denotes the ``has'' relation, where ``\hasone'' indicates that one class ``has'' exactly one instance of the other class and ``\hasn'' indicates that one class ``has'' zero, one, or multiple instances of the other class.\cbend}
	\label{fig:ontology classes}
\end{figure}

The arrows in \cref{fig:ontology classes} represent the relations of the different classes. 
There are three different types of arrows. 
We use UML for implementing the ontology, so the same type of arrows are used.
%because UML focuses on describing implementation related issues, which is useful considering our Python implementation.
The arrows with the text ``falls into'' indicate that both a scenario and a scenario category may fall into a scenario category, as explained in \cref{sec:scenario category}. \cbstartc More specifically, the ``falls into'' method from the scenario to the scenario category is denoted by $\scenariofallsinto$ and the other ``falls into'' method is denoted by $\scenariocategoryfallsinto$, see \cref{eq:fall into scenario category}. \cbend 
The arrows with the diamond are best described by the verb ``to have''\footnote{In UML, this is called an aggregation. \cbstartb This is typically implemented using pointers. For example, if object A ``has'' B, it means that A contains a pointer to B.\cbend}. For example, a scenario ``has'' a static environment. To be more precise: a scenario ``has'' exactly one static environment, which is indicated by the ``\hasone'' at the start of the arrow. In a similar manner, a scenario ``has'' zero, one, or multiple activities, which is indicated by ``\hasn'' indicating any integer number starting from 0. The third arrow denotes a subclass relation. For example, the class \textit{triggered activity} is a subclass of the class \textit{activity} such that the class \textit{triggered activity} inherits all properties from the \textit{class} activity.



\subsection{Scenario category and its attributes}
\label{sec:domain scenario category}

The blue blocks in \cref{fig:ontology classes} represent the classes that are used to model a scenario category according to the definition of a scenario category, see \cref{def:scenario category}. 
%According to \cref{def:scenario category}, a scenario category is a qualitative description of the ego vehicle, its static environment, and its dynamic environment. 
The ego vehicle and the dynamic environment are qualitatively described by \textit{activity categories} and \textit{actor categories}. Similarly, the static environment is qualitatively described by a \textit{static environment category}. 

\cbstartc
Because the \textit{static environment category} qualitatively describes the static environment, it contains a human-interpretable name and description of the static environment. Furthermore, it may contain predefined tags that are also interpretable by a software agent.
\cbend

\cbstart
The \textit{activity category} also contains a human interpretable name of the activity and it may contain predefined tags, such as the examples shown in \cref{fig:tree vehicle activities}. Furthermore, in line with the definition of an activity (\cref{def:activity}), the \textit{activity category} includes the state.
The \textit{model} that is used to describe the time evolution of the state is specified. For example, the \textit{model} to describe the speed of the ego vehicle during a braking activity could be a sinusoidal function:
\cbstartc
\begin{equation} \label{eq:sinusoidal}
	\egoacceleration(\time) = \frac{\pi \amplitude}{2\duration} \sin \left( \frac{\pi \time}{\duration} \right),\ \egospeed(\inittime) = \egospeedinit,\ \time \in [\inittime, \inittime+\duration].
\end{equation}
Here, $\egospeed$ denotes the speed of the ego vehicle and $\egoacceleration$ denotes the time derivative of $\egospeed$, i.e., the acceleration. Thus, in case of a braking activity, the state corresponds to the speed. 
The parameters of the \textit{model} are the total speed difference ($\amplitude$) between the start of the activity and the end of the activity, the duration of the braking activity ($\duration$), and the initial speed ($\egospeedinit$) at time $\inittime$. 
\cbstart
The \textit{model} of \cref{eq:sinusoidal} describes the evolution of the state from time $\time=\inittime$ until $\time=\inittime+\duration$. Since the \textit{activity category} is a qualitative description, the values of these parameters of the \textit{model} are not included.
\cbend

Similar to the \textit{static environment category} and the \textit{activity category}, the \textit{actor category} has a name and tags. Furthermore, the type of the road user is specified from a predefined list. To indicate that an actor is an ego vehicle, the tag ``Ego vehicle'' is added to the list of tags of the \textit{actor category}.

% Scenario category
The \textit{scenario category} ``has'' a \textit{static environment category}, \textit{activity categories}, and \textit{actor categories}. As with the other classes, a \textit{scenario category} contains a name and may contain predefined tags that describe parts of the scenario that are not described by the other classes.
Another attribute of the scenario category, is the list of acts. %\cbstart\footnote{\cbstartc In line with the definition of \emph{act} in \cref{sec:act}, for a \textit{scenario category}, an \emph{act} is a combination of \textit{activity categories} and \textit{actor categories}.\cbend}\cbend. 
These acts describe which actors perform which activities. Naturally, it is possible that one actor performs multiple activities and that one activity is performed by multiple actors.

% Explain why we have these different classes
\cbstartc
The reader might wonder why we introduce the different classes for describing a scenario class, i.e., the blue blocks, instead of only one class for modeling a scenario class. 
%To explain this, consider the following example. We have two scenario classes that ``have'' the same static environment category. In the current domain model, a static environment category exists independently of a scenario class. As a result, when the first scenario class is, for some reason, deleted, the static environment category is not affected. Therefore, the other scenario class, which ``has'' the same static environment category is also not affected. Another reason for having the static environment category, the activity category, and the actor category is because of the quantitative counterparts, i.e., the static environment, activity, and actor (orange blocks in \cref{fig:ontology classes}), which we explain in the next section.
The main advantage of the different classes is the re-usability of the instances of the classes, because these instances can be exchanged among different \textit{scenario categories}. For example, if two \textit{scenarios classes} ``have'' the same \textit{static environment category}, we only need to define the \textit{static environment category} once, whereas if the \textit{static environment category} would not be a class on its own but only a property of the scenario category, we would need to define the \textit{static environment category} twice.
\cbend


\subsection{Scenario and its attributes}
\label{sec:domain scenario}

The orange blocks in \cref{fig:ontology classes} represent the classes that are used to quantitatively model a scenario according to the definition of a scenario, see \cref{def:scenario}. A \textit{scenario} ``has'' a \textit{static environment}, \textit{activities}, \textit{actors}, and \textit{events}. 
\cbstartb
The ego vehicle and the dynamic environment are quantitatively described by activities and actors. 
\cbend
It is very similar to a \textit{scenario category}, as the \textit{static environment}, \textit{activities}, and \textit{actors} are the quantitative counterparts of the \textit{static environment category}, \textit{activity categories}, and \textit{actor categories}, just as a \textit{scenario} is the quantitative counterpart of a \textit{scenario category}. 

% Static environment
\cbstartc
As opposed to the \textit{static environment category}, the \textit{static environment} is a quantitative description.
\cbend
The \textit{static environment} ``has'' a \textit{static environment category}, which defines the static environment qualitatively. The most notable difference between the \textit{static environment category} and the \textit{static environment}, is that the \textit{static environment} can have multiple properties that quantitatively define the static environment. These properties define the road layout, static weather and lighting conditions, and infrastructural elements, etc.

% Activity
According to \cref{def:activity}, an activity quantitatively describes the evolution of a state in a time interval. The state that is described by an activity is defined by the \textit{activity category} that the \textit{activity} ``has''. Together with the \textit{model} that is contained by the \textit{activity category}, the time evolution of the state is described by a set of parameters. The values of the parameters are part of the \textit{activity}. 
%Note that the model can be a time interpolation of (measured) data points, in which case the parameters are a sequence of time and state values. The time interval is defined by a duration of the activity.

% Set and triggered activity
\cbstart
Two different types of activities can be defined. A set activity describes an activity that happens at a certain fixed time. This is often used to describe real-world scenarios that are extracted from real-world data. On the other hand, the starting time of a triggered activity is in general not defined beforehand, as the activity is triggered by an event. This is often used to describe test cases for scenario-based testing, e.g., see the example with the ego vehicle approaching a pedestrian crossing with a pedestrian in \cref{sec:event}. Here, the starting time of the activity of the pedestrian, i.e., walking on the pedestrian crossing, is not defined beforehand and depends on, e.g., the speed of the ego vehicle \cite{seiniger2015test}. Both the \textit{set activity} and the \textit{triggered activity} are subclasses of the \textit{activity}, as shown in \cref{fig:ontology classes}, which means that the \textit{set activity} and the \textit{triggered activity} inherit all properties from the \textit{activity}. Additionally, the \textit{set activity} ``has'' a starting time and the \textit{triggered activity} ``has'' an \textit{event} that triggers the activity.
\cbend

% Actor
\cbstart
Similar to the \textit{static environment} and the \textit{activity}, the \textit{actor} ``has'' its qualitative counterpart, the \textit{actor category}. Additionally, because the \textit{actor} involves a quantitative description, it may have multiple properties defined, such as its size, weight, color, radar cross section, etc. The \textit{actor} may have an initial state and a desired state. The desired state can be used to formulate the goals of the actor. This is especially useful for defining a test case that describes the objective of the ego vehicle rather than its activities. Additionally, the goals can be formulated as text if they cannot be formulated using a desired state.
%Note that the goals are typically on a strategic level, such as destinations and waypoints, as opposed to the tactical and operational levels\footnote{\cbstart See \cite[p.~7, Figure 1]{sae2018j3016} for an overview of the difference between the strategic, tactical, and operational functions.\cbend}.
\cbend

% Event
\cbstartd
Following the definition of an event (\cref{def:event}), an \textit{event} contains conditions that describe the threshold or mode transition at the time of an event.
\cbend
%According to \cref{def:event}, an events marks the time instant at which the system reaches a specified boundary or at which a mode transition occurs. To describe this time instant, one or more conditions are specified. For example, a condition could be that the distance of the ego vehicle and a pedestrian crossing should be less than \SI{30}{\meter}. For this example, the moment at which this condition is met for the first time corresponds to the event.

% Another advantage of the blue blocks -> only one activity category for multiple activities
An advantage of having the qualitative counterparts of the \textit{static environment}, the \textit{activity}, and the \textit{actor} is that the qualitative description can be reused and exchanged. For example, there can be many different braking activities, but there needs to be only one \textit{activity category} for qualitatively defining the braking activity. Here, it is assumed that all braking activities are modeled with the same model and that similar tags apply. If this is not the case, multiple \textit{activity categories} need to be defined, but the number of \textit{activity categories} will still be significantly lower than the number of activities.

% Scenario
As with the \textit{static environment}, \textit{activity}, and \textit{actor}, the \textit{scenario} is the quantitative counterpart of the \textit{scenario category}. As a result, a \textit{scenario} ``has'' a \textit{static environment}, \textit{activities}, and \textit{text}. Additionally, in compliance with \cref{def:scenario}, a \textit{scenario} ``has'' \textit{events}. As with a \textit{scenario category}, a \textit{scenario} contains a list of acts.
%\cbstartc\footnote{\cbstartc In line with the definition of act in \cref{sec:act}, for a scenario, an act is a combination of an activity, an actor, and, possibly, the times or events that marks the start and end of the activity.\cbend}. 
The acts are used to describe which actors perform which activities at what time.\cbend
