\section{Domain model}
\label{sec:ontology}

\todo{This section should be rewritten.}

The classes of the domain model that are used to define a scenario and a scenario class are shown in \cref{fig:ontology classes}. The blue blocks represent the classes that are used to qualitatively describe a scenario whereas the orange blocks represent the classes that are used to quantitatively describe a scenario. 

\cbstart
There are three different types of arrows used in \cref{fig:ontology classes}. When translating the domain model to object-oriented code, the arrow ``Falls into'' is defines a method for the class it originates from with as input the class it points to. The two methods shown in this figure are the ``falls into'' methods that are described in \cref{sec:scenario class}. Most arrows represent an aggregation, which is best described by the verb ``to have'', i.e., an aggregation arrow from class A to class B means that an object of class A ``has'' zero, one, or multiple objects of class B. The number of objects of class B that A can ``have'' is shown at the starting point of the aggregation arrow and this is either one (1) or any integer number ($\mathbb{N}$), i.e., 0, 1, 2, etc. The third arrow denotes a subclass relation, i.e., the class from which this arrow originates inherits all properties from the class the arrow is pointing to. More specifically, the classes of triggered and detected activities are subclasses of the class ``Activity''.
\cbend

\begin{figure}
	\centering
	\definecolor{scenarioclass}{RGB}{30, 144, 255}
\definecolor{category}{RGB}{0, 191, 255}
\definecolor{scenario}{RGB}{255, 69, 0}
\definecolor{otherclass}{RGB}{255, 127, 80}
\newlength\blockwidth
\newlength\blockheight
\newlength\blockx
\newlength\blocky
\newlength\legendwidth
\setlength{\blockwidth}{5.3em}
\setlength{\blockheight}{4em}
\setlength{\blockx}{6.4em}
\setlength{\blocky}{-7em}
\setlength{\legendwidth}{3.5em}
\tikzstyle{class}=[draw, text width=\blockwidth-.5em, align=center, minimum height=\blockheight, line width=1pt, minimum width=\blockwidth]
\tikzstyle{aggregation}=[-{Diamond[width=8pt, length=12pt, fill=white]}, line width=1pt]
\tikzstyle{falls into}=[->, line width=1pt]
\tikzstyle{superclass}=[-{Triangle[width=8pt, length=12pt, fill=white]}, line width=1pt]
\begin{tikzpicture}
% Classes
\node[class, fill=scenarioclass](scenario class) at (.5\blockx,0) {Scenario class};
\node[class, fill=category](staticcategory) at (\blockx, \blocky) {Static environment category};
\node[class, fill=category](activitycategory) at (2\blockx, \blocky) {Activity category};
\node[class, fill=category](model) at (1.5\blockx+0.25\blockwidth, 2\blocky) {Model};
\node[class, fill=category](actorcategory) at (3\blockx, \blocky) {Actor category};
\node[class, fill=scenario](scenario) at (.5\blockx, 2\blocky) {Scenario};
\node[class, fill=otherclass](static) at (\blockx, 3\blocky) {Static environment};
\node[class, fill=otherclass](activity) at (2\blockx, 3\blocky) {Acitivity};
\node[class, fill=otherclass](actor) at (3\blockx, 3\blocky) {Actor};
\node[class, fill=otherclass](triggered) at (1.5\blockx, 4\blocky) {Triggered activity};
\node[class, fill=otherclass](detected) at (2.5\blockx, 4\blocky) {Detected activity};

% Aggregation arrows for the scenario class
\node[coordinate, below of=scenario class, node distance=-\blocky/2, xshift=\blockwidth/3](helper scenario class){};
\node[coordinate, below of=scenario class, node distance=\blockheight/2, xshift=\blockwidth/3](aggregation scenario class){};
\foreach \class in {static, activity, actor}
{
	\node[coordinate, above of=\class category, node distance=\blockheight/2](helper \class){};  % Needed for later
	\draw[aggregation] (\class category) |- (helper scenario class) -- (aggregation scenario class);
}
\node[anchor=south east] at (helper static) {1};
\node[anchor=south east] at (helper activity) {$\mathbb{N}$};
\node[anchor=south east] at (helper actor) {$\mathbb{N}$};

% Aggregation arrow for the model
\node[coordinate, above of=model, node distance=\blockheight/2, xshift=-\blockwidth/8+\blockx/4](aggregation model){};
\node[coordinate, below of=activitycategory, node distance=\blockheight/2, xshift=\blockwidth/8-\blockx/4](aggregation activity category){};
\draw[aggregation] (aggregation model) -- (aggregation activity category);

% Aggregation arrow for scenario
\node[coordinate, below of=scenario, node distance=-\blocky/2](helper scenario){};
\node[coordinate, below of=scenario, node distance=\blockheight/2+1pt](aggregation scenario){};
\foreach \class in {static, activity, actor}
{
	\node[coordinate, above of=\class, node distance=\blockheight/2, xshift=-\blockwidth/4](helper \class){};
	\draw[aggregation] (helper \class) |- (helper scenario) -- (aggregation scenario);
}
\node[anchor=south east] at (helper static) {1};
\node[anchor=south east] at (helper activity) {$\mathbb{N}$};
\node[anchor=south east] at (helper actor) {$\mathbb{N}$};

% Aggregations for static environment, activity, and actor
\foreach \class in {static, activity, actor}
{
	\node[coordinate, below of=\class category, node distance=\blockheight/2, xshift=\blockwidth/4](category helper){};
	\node[coordinate, above of=\class, node distance=\blockheight/2, xshift=\blockwidth/4](helper){};
	\draw[aggregation] (category helper) -- (helper);
	\node[anchor=north east] at (category helper) {1};
}

% falls into arrows
\node[coordinate, right of=scenario class, node distance=\blockwidth/2+1pt, yshift=-\blockheight/3](helper1){};
\node[coordinate, right of=scenario class, node distance=\blockwidth/2+1pt, yshift=\blockheight/3](helper2){};
\node[coordinate, right of=helper1, node distance=\blockwidth/2](helper3){};
\node[coordinate, right of=helper2, node distance=\blockwidth/2](helper4){};
\draw[falls into] (helper1) -- (helper3) -- node[fill=white]{falls into} (helper4) -- (helper2);
\node[coordinate, above of=scenario, node distance=\blockheight/2+1pt, xshift=-\blockwidth/3](helper1){};
\node[coordinate, below of=scenario class, node distance=\blockheight/2+1pt, xshift=-\blockwidth/3](helper2){};
\draw[falls into] (helper1) -- node[fill=white, align=center]{falls\\into} (helper2);

% Superclass arrows
\node[coordinate, below of=activity, node distance=-.6\blocky](helper activity){};
\draw[superclass] (triggered) |- (helper activity) -- (activity);
\draw[superclass] (detected) |- (helper activity) -- (activity);

% Legend
\node[coordinate](legend) at (1.8\blockx, -.5em) {};
\node[draw, left of=legend, node distance=0.3em, minimum height=4em, minimum width=\legendwidth+6em, anchor=west, fill=gray!10]{};
\node[coordinate, right of=legend, node distance=\legendwidth](legend right){};
\draw[aggregation] (legend) -- (legend right);
\node[right of=legend right, node distance=0em, anchor=west]{Aggregation};
\node[coordinate, below of=legend, node distance=1.2em](helper1){};
\node[coordinate, below of=legend right, node distance=1.2em](helper2){};
\draw[superclass] (helper1) -- (helper2);
\node[right of=helper2, node distance=0em, anchor=west]{Superclass};
\node[coordinate, above of=legend, node distance=1.2em](helper1){};
\node[coordinate, above of=legend right, node distance=1.2em](helper2){};
\draw[falls into] (helper1) -- (helper2);
\node[right of=helper2, node distance=0em, anchor=west]{Method};

\end{tikzpicture}
	\caption{Schematic overview of most classes of the ontology.}
	\label{fig:ontology classes}
\end{figure}


\color{red}
TODO: 
\begin{itemize}
	\item Explain difference between qualitative descriptions (blue) and quantitative descriptions (orange).
	\item Briefly explain the different classes.
\end{itemize}
\color{black}
