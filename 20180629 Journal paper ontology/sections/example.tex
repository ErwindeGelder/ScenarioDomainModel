\section{Example: pedestrian crossing}
\label{sec:example}

To illustrate the use of the ontology, we describe a scenario using the domain model presented in \cref{sec:ontology}. The scenario is schematically shown in \cref{fig:scenario overview}. The ego vehicle is driving on a two-lane road and a pedestrian is walking on a footway that intersects the road the ego vehicle is driving on. Both the ego vehicle and the pedestrian are initially approaching the pedestrian crossing. The ego vehicle brakes and comes to a full stop in front of the pedestrian crossing. While the ego vehicle is stationary, the pedestrian crosses the road using the pedestrian crossing. When the pedestrian has passed the ego vehicle, the ego vehicle accelerates.

\setlength{\figurewidth}{\linewidth}
\begin{figure}
	\centering
	% This file was created by matplotlib2tikz v0.6.17.
\begin{tikzpicture}

\definecolor{color0}{rgb}{0.8,1,0.8}
\definecolor{color1}{rgb}{1,0.9,0.8}

\begin{axis}[
xmin=-10, xmax=10,
ymin=-6, ymax=6,
width=\figurewidth,
height=0.60\figurewidth,
tick align=outside,
tick pos=left,
x grid style={white!69.01960784313725!black},
y grid style={white!69.01960784313725!black},
axis background/.style={fill=color0},
ticks=none,
scale only axis
]
\path [draw=white!80.0!black, fill=white!80.0!black] (axis cs:3.49,2.5)
--(axis cs:3.49,2.5)
--(axis cs:3.49173648177667,2.50015192246988)
--(axis cs:3.49342020143326,2.50060307379214)
--(axis cs:3.495,2.50133974596216)
--(axis cs:3.49642787609687,2.50233955556881)
--(axis cs:3.49766044443119,2.50357212390313)
--(axis cs:3.49866025403784,2.505)
--(axis cs:3.49939692620786,2.50657979856674)
--(axis cs:3.49984807753012,2.50826351822333)
--(axis cs:3.5,2.51)
--(axis cs:3.5,2.51)
--(axis cs:6.5,2.51)
--(axis cs:6.5,2.51)
--(axis cs:6.50015192246988,2.50826351822333)
--(axis cs:6.50060307379214,2.50657979856674)
--(axis cs:6.50133974596216,2.505)
--(axis cs:6.50233955556881,2.50357212390313)
--(axis cs:6.50357212390313,2.50233955556881)
--(axis cs:6.505,2.50133974596216)
--(axis cs:6.50657979856674,2.50060307379214)
--(axis cs:6.50826351822333,2.50015192246988)
--(axis cs:6.51,2.5)
--(axis cs:6.51,2.5)
--(axis cs:6.51,-2.5)
--(axis cs:6.51,-2.5)
--(axis cs:6.50826351822333,-2.50015192246988)
--(axis cs:6.50657979856674,-2.50060307379214)
--(axis cs:6.505,-2.50133974596216)
--(axis cs:6.50357212390313,-2.50233955556881)
--(axis cs:6.50233955556881,-2.50357212390313)
--(axis cs:6.50133974596216,-2.505)
--(axis cs:6.50060307379214,-2.50657979856674)
--(axis cs:6.50015192246988,-2.50826351822333)
--(axis cs:6.5,-2.51)
--(axis cs:6.5,-2.51)
--(axis cs:3.5,-2.51)
--(axis cs:3.5,-2.51)
--(axis cs:3.49984807753012,-2.50826351822333)
--(axis cs:3.49939692620786,-2.50657979856674)
--(axis cs:3.49866025403784,-2.505)
--(axis cs:3.49766044443119,-2.50357212390313)
--(axis cs:3.49642787609687,-2.50233955556881)
--(axis cs:3.495,-2.50133974596216)
--(axis cs:3.49342020143326,-2.50060307379214)
--(axis cs:3.49173648177667,-2.50015192246988)
--(axis cs:3.49,-2.5)
--(axis cs:3.49,-2.5)
--cycle;

\path [draw=white!80.0!black, fill=white!80.0!black] (axis cs:-10,2.5)
--(axis cs:3.49,2.5)
--(axis cs:3.49,-2.5)
--(axis cs:-10,-2.5)
--cycle;

\path [draw=color1, fill=color1] (axis cs:3.5,-6)
--(axis cs:3.5,-2.51)
--(axis cs:6.5,-2.51)
--(axis cs:6.5,-6)
--cycle;

\path [draw=white!80.0!black, fill=white!80.0!black] (axis cs:6.51,2.5)
--(axis cs:25,2.5)
--(axis cs:25,-2.5)
--(axis cs:6.51,-2.5)
--cycle;

\path [draw=color1, fill=color1] (axis cs:3.5,2.51)
--(axis cs:3.5,6)
--(axis cs:6.5,6)
--(axis cs:6.5,2.51)
--cycle;

\addplot [semithick, white, forget plot]
table {%
-10 0
-9.43791666666667 0
};
\addplot [semithick, white, forget plot]
table {%
-7.18958333333333 0
-6.06541666666667 0
};
\addplot [semithick, white, forget plot]
table {%
-3.81708333333333 0
-2.69291666666667 0
};
\addplot [semithick, white, forget plot]
table {%
-0.444583333333333 0
0.679583333333334 0
};
\addplot [semithick, white, forget plot]
table {%
2.92791666666667 0
3.49 0
};
\addplot [semithick, white, forget plot]
table {%
6.51 0
7.02361111111111 0
};
\addplot [semithick, white, forget plot]
table {%
9.07805555555556 0
10.1052777777778 0
};
\addplot [semithick, white, forget plot]
table {%
12.1597222222222 0
13.1869444444444 0
};
\addplot [semithick, white, forget plot]
table {%
15.2413888888889 0
16.2686111111111 0
};
\addplot [semithick, white, forget plot]
table {%
18.3230555555556 0
19.3502777777778 0
};
\addplot [semithick, white, forget plot]
table {%
21.4047222222222 0
22.4319444444444 0
};
\addplot [semithick, white, forget plot]
table {%
24.4863888888889 0
25 0
};
\path [draw=white, fill=white] (axis cs:6.51,2.30083333333333)
--(axis cs:6.51,1.8825)
--(axis cs:3.49,1.8825)
--(axis cs:3.49,2.30083333333333)
--cycle;

\path [draw=white, fill=white] (axis cs:6.51,1.46416666666667)
--(axis cs:6.51,1.04583333333333)
--(axis cs:3.49,1.04583333333333)
--(axis cs:3.49,1.46416666666667)
--cycle;

\path [draw=white, fill=white] (axis cs:6.51,0.6275)
--(axis cs:6.51,0.209166666666666)
--(axis cs:3.49,0.209166666666666)
--(axis cs:3.49,0.6275)
--cycle;

\path [draw=white, fill=white] (axis cs:6.51,-0.209166666666667)
--(axis cs:6.51,-0.6275)
--(axis cs:3.49,-0.6275)
--(axis cs:3.49,-0.209166666666667)
--cycle;

\path [draw=white, fill=white] (axis cs:6.51,-1.04583333333333)
--(axis cs:6.51,-1.46416666666667)
--(axis cs:3.49,-1.46416666666667)
--(axis cs:3.49,-1.04583333333333)
--cycle;

\path [draw=white, fill=white] (axis cs:6.51,-1.8825)
--(axis cs:6.51,-2.30083333333333)
--(axis cs:3.49,-2.30083333333333)
--(axis cs:3.49,-1.8825)
--cycle;

\path [draw=blue, fill=blue] (axis cs:5,5.5)
--(axis cs:5.16234973460234,5.47290862085032)
--(axis cs:5.30710635634483,5.3945702546982)
--(axis cs:5.41858323913126,5.27347407906121)
--(axis cs:5.48470013296967,5.1227427435704)
--(axis cs:5.49829224650334,4.95871032726383)
--(axis cs:5.45788666332753,4.79915228767352)
--(axis cs:5.36786195533657,4.66135921418713)
--(axis cs:5.23797369651854,4.56026312439676)
--(axis cs:5.08229729514037,4.50681934829864)
--(axis cs:4.91770270485963,4.50681934829864)
--(axis cs:4.76202630348146,4.56026312439676)
--(axis cs:4.63213804466343,4.66135921418713)
--(axis cs:4.54211333667247,4.79915228767351)
--(axis cs:4.50170775349666,4.95871032726383)
--(axis cs:4.51529986703033,5.1227427435704)
--(axis cs:4.58141676086874,5.27347407906121)
--(axis cs:4.69289364365517,5.3945702546982)
--(axis cs:4.83765026539766,5.47290862085032)
--(axis cs:5,5.5)
--cycle;

\addplot [semithick, black, forget plot]
table {%
3.49 2.5
3.49 2.5
3.49173648177667 2.50015192246988
3.49342020143326 2.50060307379214
3.495 2.50133974596216
3.49642787609687 2.50233955556881
3.49766044443119 2.50357212390313
3.49866025403784 2.505
3.49939692620786 2.50657979856674
3.49984807753012 2.50826351822333
3.5 2.51
3.5 2.51
};
\addplot [semithick, black, forget plot]
table {%
6.5 2.51
6.5 2.51
6.50015192246988 2.50826351822333
6.50060307379214 2.50657979856674
6.50133974596216 2.505
6.50233955556881 2.50357212390313
6.50357212390313 2.50233955556881
6.505 2.50133974596216
6.50657979856674 2.50060307379214
6.50826351822333 2.50015192246988
6.51 2.5
6.51 2.5
};
\addplot [semithick, black, forget plot]
table {%
6.51 -2.5
6.51 -2.5
6.50826351822333 -2.50015192246988
6.50657979856674 -2.50060307379214
6.505 -2.50133974596216
6.50357212390313 -2.50233955556881
6.50233955556881 -2.50357212390313
6.50133974596216 -2.505
6.50060307379214 -2.50657979856674
6.50015192246988 -2.50826351822333
6.5 -2.51
6.5 -2.51
};
\addplot [semithick, black, forget plot]
table {%
3.5 -2.51
3.5 -2.51
3.49984807753012 -2.50826351822333
3.49939692620786 -2.50657979856674
3.49866025403784 -2.505
3.49766044443119 -2.50357212390313
3.49642787609687 -2.50233955556881
3.495 -2.50133974596216
3.49342020143326 -2.50060307379214
3.49173648177667 -2.50015192246988
3.49 -2.5
3.49 -2.5
};
\addplot [semithick, black, forget plot]
table {%
-10 2.5
3.49 2.5
};
\addplot [semithick, black, forget plot]
table {%
3.49 -2.5
-10 -2.5
};
\addplot [semithick, color0, forget plot]
table {%
3.5 -6
3.5 -2.51
};
\addplot [semithick, color0, forget plot]
table {%
6.5 -2.51
6.5 -6
};
\addplot [semithick, black, forget plot]
table {%
6.51 2.5
25 2.5
};
\addplot [semithick, black, forget plot]
table {%
25 -2.5
6.51 -2.5
};
\addplot [semithick, color0, forget plot]
table {%
3.5 2.51
3.5 6
};
\addplot [semithick, color0, forget plot]
table {%
6.5 6
6.5 2.51
};
\addplot [red, forget plot]
table {%
-7 1.22857142857143
-7 1.57142857142857
-6.89542483660131 1.80714285714286
-6.3202614379085 1.93571428571429
-4.41176470588235 1.93571428571429
-4.54248366013072 2.04285714285714
-4.41176470588235 1.93571428571429
-3.49673202614379 1.93571428571429
-3.13071895424837 1.80714285714286
-3 1.50714285714286
-3 1.25
-3 1.48571428571429
-3.86274509803922 1.76428571428571
};
\addplot [red, forget plot]
table {%
-7 1.27142857142857
-7 0.928571428571428
-6.89542483660131 0.692857142857143
-6.3202614379085 0.564285714285714
-4.41176470588235 0.564285714285714
-4.54248366013072 0.457142857142857
-4.41176470588235 0.564285714285714
-3.49673202614379 0.564285714285714
-3.13071895424837 0.692857142857143
-3 0.992857142857143
-3 1.25
-3 1.01428571428571
-3.86274509803922 0.735714285714286
};
\addplot [red, forget plot]
table {%
-6.03267973856209 1.85
-5.69281045751634 1.76428571428571
-4.90849673202614 1.76428571428571
-4.54248366013072 1.85
};
\addplot [red, forget plot]
table {%
-6.03267973856209 0.65
-5.69281045751634 0.735714285714286
-4.90849673202614 0.735714285714286
-4.54248366013072 0.65
};
\addplot [red, forget plot]
table {%
-6.03267973856209 1.25
-6.03267973856209 1.4
-5.9281045751634 1.7
-6.24183006535948 1.7
-6.45098039215686 1.63571428571429
-6.55555555555556 1.55
-6.63398692810457 1.37857142857143
-6.63398692810457 1.25
};
\addplot [red, forget plot]
table {%
-6.03267973856209 1.25
-6.03267973856209 1.1
-5.9281045751634 0.8
-6.24183006535948 0.8
-6.45098039215686 0.864285714285714
-6.55555555555556 0.95
-6.63398692810457 1.12142857142857
-6.63398692810457 1.25
};
\addplot [red, forget plot]
table {%
-4.72549019607843 1.25
-4.72549019607843 1.52857142857143
-4.7516339869281 1.7
-4.12418300653595 1.85
-3.99346405228758 1.35714285714286
-3.99346405228758 1.25
};
\addplot [red, forget plot]
table {%
-4.72549019607843 1.25
-4.72549019607843 0.971428571428571
-4.7516339869281 0.8
-4.12418300653595 0.65
-3.99346405228758 1.14285714285714
-3.99346405228758 1.25
};
\addplot [semithick, red, forget plot]
table {%
-3 1.25
8 1.25
};
\addplot [semithick, red, forget plot]
table {%
7.25 2
8 1.25
7.25 0.5
};
\addplot [semithick, blue, forget plot]
table {%
5 5
5 -5
};
\addplot [semithick, blue, forget plot]
table {%
5.75 -4.25
5 -5
4.25 -4.25
};
\end{axis}

\end{tikzpicture}
	\caption{Schematic overview of a scenario where both the ego vehicle (red car) and a pedestrian (blue dot) are approaching a non-signalized pedestrian crossing. The pedestrian has priority.}
	\label{fig:scenario overview}
\end{figure}


To describe the scenario according to the presented domain model, objects are instantiated from the classes presented in \cref{fig:ontology classes}. \Cref{fig:example qualitative} shows the objects for describing the scenario qualitatively. There are two actor categories: one for the ego vehicle and one for the pedestrian. Five different activity categories are defined: braking, stationary, accelerating, driving straight, and walking straight. These objects, together with the static environment category, are used by the scenario class. The scenario class has five acts. The first four acts assign the first four activity categories to the ego vehicle. The last act assigns the activity category ``walking straight'' to the pedestrian.

\begin{figure}
	\centering
	%\newlength\objectwidth\setlength{\objectwidth}{24em}
%\tikzstyle{object}=[draw, text width=\objectwidth-.5em, align=left, line width=1pt, minimum width=\objectwidth, anchor=north west, node distance=3pt]
%\resizebox{\textwidth}{!}{%
\begin{tikzpicture}
\tikzstyle{every node}=[font=\footnotesize]

\node[object, fill=scenariocategory](scenario category){{\bfseries Crossing pedestrian::Scenario category}\\
	description: A pedestrian is crossing the road \\
	\leavevmode\phantom{description: }on a zebra crossing in front of the \\
	\leavevmode\phantom{description: }ego vehicle\\
	static physical things: Pedestrian crossing\\
	\leavevmode\phantom{static physical things: }qualitative\\
	actors: Ego qualitative, Pedestrian qualitative\\
	activities: Braking, Stationary, Accelerating, \\
	\leavevmode\phantom{activities: }Walking straight\\
	acts: (Ego qualitative, Braking), \\
	\leavevmode\phantom{acts: }(Ego qualitative, Stationary), \\
	\leavevmode\phantom{acts: }(Ego qualitative, Accelerating), \\
	\leavevmode\phantom{acts: }(Pedestrian qualitative, Walking straight)\\
	tags:};

\node[object, fill=category, right=of scenario category.north east, anchor=north west](ego qualitative){{\bfseries Ego qualitative::Actor category}\\
	type: Vehicle\\
	tags: Ego vehicle, Passenger car};

\node[object, fill=category, below=of ego qualitative.south](pedestrian qualitative){{\bfseries Pedestrian qualitative::Actor category}\\
	type: Pedestrian\\
	tags: Pedestrian};

\node[object, fill=category, below=of pedestrian qualitative.south](braking){{\bfseries Braking::Activity category}\\
	model: Sinusoid, see \cref{eq:sinusoidala,eq:sinusoidalb}\\
	state: Speed ($\egospeed$)\\
	tags: Braking};

\node[object, fill=category, below=of braking.south](stationary){{\bfseries Stationary::Activity category}\\
	model: Constant, see \cref{eq:constant model}\\
	state: Speed ($\egospeed$)\\
	tags: Stationary};

\node[object, fill=category, right=of ego qualitative.north east, anchor=north west](accelerating){{\bfseries Accelerating::Activity category}\\
	model: Linear, see \cref{eq:ego accelerating}\\
	state: Speed ($\egospeed$)\\
	tags: Accelerating};

%\node[object, fill=category, below=of accelerating.south](straight){{\bfseries Driving straight::Activity category}\\
%	model: Linear\\
%	state: Lateral position\\
%	tags: Driving straight};

\node[object, fill=category, below=of accelerating.south](walking){{\bfseries Walking straight::Activity category}\\
	model: Linear, see \cref{eq:pedestrian walking}\\
	state: Position ($\pednorth$)\\
	tags: Walking straight};

\node[object, fill=category, below=of walking.south](static physical){{\bfseries Pedestrian crossing qualitative::Static physical thing category}\\
	description: Straight road with two lanes and a\\
	\leavevmode\phantom{description: }pedestrian crossing\\
	tags: Non-signalized zebra crossing};

\end{tikzpicture}
%}
	\caption{The objects that are used to qualitatively describe the scenario that is schematically shown in \cref{fig:scenario overview}. The first line of each blocks shows the name (before the double colon) and the class from which the object is instantiated. The following lines show the attributes of the object.}
	\label{fig:example qualitative}
\end{figure}

\todo{Also describe the quantitative part of the scenario.}

\todo{Explain how a test case would look like. In that case, the activities of the ego vehicle will not be defined. Instead, an initial goal and a desired goal will be defined. Furthermore, the walking activity of the pedestrian will be triggered. Lastly, the ego vehicle will start at a larger distance from the pedestrian crossing.}