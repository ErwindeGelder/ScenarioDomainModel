\section{Example: pedestrian crossing}
\label{sec:example}

To illustrate the use of the ontology, we describe a scenario using the domain model presented in \cref{sec:ontology}. The scenario is schematically shown in \cref{fig:scenario overview}. The ego vehicle is driving on the right lane of a two-lane road and a pedestrian is walking on a footway that intersects the road the ego vehicle is driving on. Both the ego vehicle and the pedestrian are initially approaching the pedestrian crossing. The ego vehicle brakes and comes to a full stop in front of the pedestrian crossing. While the ego vehicle is stationary, the pedestrian crosses the road using the pedestrian crossing. When the pedestrian has passed the ego vehicle, the ego vehicle accelerates. \cstart The code of this example is publicly available\footnote{\cstart See \url{https://github.com/ErwindeGelder/ScenarioDomainModel}. The repository also contains other examples.\cend}\cend.

\setlength{\figurewidth}{0.4\linewidth}
\begin{figure}[t]
	\centering
	% This file was created by matplotlib2tikz v0.6.17.
\begin{tikzpicture}

\definecolor{color0}{rgb}{0.8,1,0.8}
\definecolor{color1}{rgb}{1,0.9,0.8}
\definecolor{color2}{rgb}{0,0.4375,0.75}
\definecolor{color3}{rgb}{0.9,0.8,0.7}
\definecolor{color4}{rgb}{0.75,0.75,1}
\definecolor{color5}{rgb}{0.5,0.25,0.125}

\begin{axis}[
xmin=-10, xmax=10,
ymin=-6, ymax=6,
width=\figurewidth,
height=0.60\figurewidth,
tick align=outside,
xticklabel style = {align=center,text width=1},
yticklabel style = {align=right,text width=1},
tick pos=left,
x grid style={white!69.01960784313725!black},
y grid style={white!69.01960784313725!black},
axis background/.style={fill=color0},
ticks=none,
scale only axis
]
\path [draw=white!80.0!black, fill=white!80.0!black] (axis cs:3.4,2.8)
--(axis cs:3.4,2.8)
--(axis cs:3.41736481776669,2.80151922469878)
--(axis cs:3.43420201433257,2.80603073792141)
--(axis cs:3.45,2.81339745962156)
--(axis cs:3.46427876096865,2.8233955556881)
--(axis cs:3.4766044443119,2.83572123903135)
--(axis cs:3.48660254037844,2.85)
--(axis cs:3.49396926207859,2.86579798566743)
--(axis cs:3.49848077530122,2.88263518223331)
--(axis cs:3.5,2.9)
--(axis cs:3.5,2.9)
--(axis cs:6.5,2.9)
--(axis cs:6.5,2.9)
--(axis cs:6.50151922469878,2.88263518223331)
--(axis cs:6.50603073792141,2.86579798566743)
--(axis cs:6.51339745962156,2.85)
--(axis cs:6.5233955556881,2.83572123903135)
--(axis cs:6.53572123903135,2.8233955556881)
--(axis cs:6.55,2.81339745962156)
--(axis cs:6.56579798566743,2.80603073792141)
--(axis cs:6.58263518223331,2.80151922469878)
--(axis cs:6.6,2.8)
--(axis cs:6.6,2.8)
--(axis cs:6.6,-2.8)
--(axis cs:6.6,-2.8)
--(axis cs:6.58263518223331,-2.80151922469878)
--(axis cs:6.56579798566743,-2.80603073792141)
--(axis cs:6.55,-2.81339745962156)
--(axis cs:6.53572123903135,-2.8233955556881)
--(axis cs:6.5233955556881,-2.83572123903135)
--(axis cs:6.51339745962156,-2.85)
--(axis cs:6.50603073792141,-2.86579798566743)
--(axis cs:6.50151922469878,-2.88263518223331)
--(axis cs:6.5,-2.9)
--(axis cs:6.5,-2.9)
--(axis cs:3.5,-2.9)
--(axis cs:3.5,-2.9)
--(axis cs:3.49848077530122,-2.88263518223331)
--(axis cs:3.49396926207859,-2.86579798566743)
--(axis cs:3.48660254037844,-2.85)
--(axis cs:3.4766044443119,-2.83572123903135)
--(axis cs:3.46427876096865,-2.8233955556881)
--(axis cs:3.45,-2.81339745962156)
--(axis cs:3.43420201433257,-2.80603073792141)
--(axis cs:3.41736481776669,-2.80151922469878)
--(axis cs:3.4,-2.8)
--(axis cs:3.4,-2.8)
--cycle;

\path [draw=white!80.0!black, fill=white!80.0!black] (axis cs:-10,2.8)
--(axis cs:3.4,2.8)
--(axis cs:3.4,-2.8)
--(axis cs:-10,-2.8)
--cycle;

\path [draw=color1, fill=color1] (axis cs:3.5,-6)
--(axis cs:3.5,-2.9)
--(axis cs:6.5,-2.9)
--(axis cs:6.5,-6)
--cycle;

\path [draw=white!80.0!black, fill=white!80.0!black] (axis cs:6.6,2.8)
--(axis cs:25,2.8)
--(axis cs:25,-2.8)
--(axis cs:6.6,-2.8)
--cycle;

\path [draw=color1, fill=color1] (axis cs:3.5,2.9)
--(axis cs:3.5,6)
--(axis cs:6.5,6)
--(axis cs:6.5,2.9)
--cycle;

\addplot [semithick, black, forget plot]
table {%
3.4 2.8
3.4 2.8
3.41736481776669 2.80151922469878
3.43420201433257 2.80603073792141
3.45 2.81339745962156
3.46427876096865 2.8233955556881
3.4766044443119 2.83572123903135
3.48660254037844 2.85
3.49396926207859 2.86579798566743
3.49848077530122 2.88263518223331
3.5 2.9
3.5 2.9
};
\addplot [semithick, black, forget plot]
table {%
6.5 2.9
6.5 2.9
6.50151922469878 2.88263518223331
6.50603073792141 2.86579798566743
6.51339745962156 2.85
6.5233955556881 2.83572123903135
6.53572123903135 2.8233955556881
6.55 2.81339745962156
6.56579798566743 2.80603073792141
6.58263518223331 2.80151922469878
6.6 2.8
6.6 2.8
};
\addplot [semithick, black, forget plot]
table {%
6.6 -2.8
6.6 -2.8
6.58263518223331 -2.80151922469878
6.56579798566743 -2.80603073792141
6.55 -2.81339745962156
6.53572123903135 -2.8233955556881
6.5233955556881 -2.83572123903135
6.51339745962156 -2.85
6.50603073792141 -2.86579798566743
6.50151922469878 -2.88263518223331
6.5 -2.9
6.5 -2.9
};
\addplot [semithick, black, forget plot]
table {%
3.5 -2.9
3.5 -2.9
3.49848077530122 -2.88263518223331
3.49396926207859 -2.86579798566743
3.48660254037844 -2.85
3.4766044443119 -2.83572123903135
3.46427876096865 -2.8233955556881
3.45 -2.81339745962156
3.43420201433257 -2.80603073792141
3.41736481776669 -2.80151922469878
3.4 -2.8
3.4 -2.8
};
\addplot [semithick, black, forget plot]
table {%
-10 2.8
3.4 2.8
};
\addplot [semithick, black, forget plot]
table {%
3.4 -2.8
-10 -2.8
};
\addplot [semithick, white, forget plot]
table {%
-10 0
-9.44166666666667 0
};
\addplot [semithick, white, forget plot]
table {%
-7.20833333333333 0
-6.09166666666667 0
};
\addplot [semithick, white, forget plot]
table {%
-3.85833333333333 0
-2.74166666666667 0
};
\addplot [semithick, white, forget plot]
table {%
-0.508333333333332 0
0.608333333333335 0
};
\addplot [semithick, white, forget plot]
table {%
2.84166666666667 0
3.4 0
};
\addplot [semithick, color0, forget plot]
table {%
3.5 -6
3.5 -2.9
};
\addplot [semithick, color0, forget plot]
table {%
6.5 -2.9
6.5 -6
};
\addplot [semithick, black, forget plot]
table {%
6.6 2.8
25 2.8
};
\addplot [semithick, black, forget plot]
table {%
25 -2.8
6.6 -2.8
};
\addplot [semithick, white, forget plot]
table {%
6.6 0
7.11111111111111 0
};
\addplot [semithick, white, forget plot]
table {%
9.15555555555556 0
10.1777777777778 0
};
\addplot [semithick, white, forget plot]
table {%
12.2222222222222 0
13.2444444444444 0
};
\addplot [semithick, white, forget plot]
table {%
15.2888888888889 0
16.3111111111111 0
};
\addplot [semithick, white, forget plot]
table {%
18.3555555555556 0
19.3777777777778 0
};
\addplot [semithick, white, forget plot]
table {%
21.4222222222222 0
22.4444444444444 0
};
\addplot [semithick, white, forget plot]
table {%
24.4888888888889 0
25 0
};
\addplot [semithick, color0, forget plot]
table {%
3.5 2.9
3.5 6
};
\addplot [semithick, color0, forget plot]
table {%
6.5 6
6.5 2.9
};
\path [draw=white, fill=white] (axis cs:6.6,2.71875)
--(axis cs:6.6,2.35625)
--(axis cs:3.4,2.35625)
--(axis cs:3.4,2.71875)
--cycle;

\path [draw=white, fill=white] (axis cs:6.6,1.99375)
--(axis cs:6.6,1.63125)
--(axis cs:3.4,1.63125)
--(axis cs:3.4,1.99375)
--cycle;

\path [draw=white, fill=white] (axis cs:6.6,1.26875)
--(axis cs:6.6,0.90625)
--(axis cs:3.4,0.90625)
--(axis cs:3.4,1.26875)
--cycle;

\path [draw=white, fill=white] (axis cs:6.6,0.54375)
--(axis cs:6.6,0.18125)
--(axis cs:3.4,0.18125)
--(axis cs:3.4,0.54375)
--cycle;

\path [draw=white, fill=white] (axis cs:6.6,-0.18125)
--(axis cs:6.6,-0.54375)
--(axis cs:3.4,-0.54375)
--(axis cs:3.4,-0.18125)
--cycle;

\path [draw=white, fill=white] (axis cs:6.6,-0.90625)
--(axis cs:6.6,-1.26875)
--(axis cs:3.4,-1.26875)
--(axis cs:3.4,-0.90625)
--cycle;

\path [draw=white, fill=white] (axis cs:6.6,-1.63125)
--(axis cs:6.6,-1.99375)
--(axis cs:3.4,-1.99375)
--(axis cs:3.4,-1.63125)
--cycle;

\path [draw=white, fill=white] (axis cs:6.6,-2.35625)
--(axis cs:6.6,-2.71875)
--(axis cs:3.4,-2.71875)
--(axis cs:3.4,-2.35625)
--cycle;

\path [draw=black, fill=color2] (axis cs:-7.25,-1.39598214285714)
--(axis cs:-7.24189189189189,-0.873660714285714)
--(axis cs:-7.18513513513513,-0.672767857142857)
--(axis cs:-7.10405405405405,-0.600446428571429)
--(axis cs:-6.97432432432432,-0.552232142857143)
--(axis cs:-6.51216216216216,-0.495982142857143)
--(axis cs:-3.09864864864865,-0.536160714285714)
--(axis cs:-3.00945945945946,-0.592410714285714)
--(axis cs:-2.87162162162162,-0.769196428571428)
--(axis cs:-2.81486486486487,-0.962053571428571)
--(axis cs:-2.75,-1.39598214285714)
--(axis cs:-2.75,-1.40401785714286)
--(axis cs:-2.81486486486487,-1.83794642857143)
--(axis cs:-2.87162162162162,-2.03080357142857)
--(axis cs:-3.00945945945946,-2.20758928571429)
--(axis cs:-3.09864864864865,-2.26383928571429)
--(axis cs:-6.51216216216216,-2.30401785714286)
--(axis cs:-6.97432432432432,-2.24776785714286)
--(axis cs:-7.10405405405405,-2.19955357142857)
--(axis cs:-7.18513513513513,-2.12723214285714)
--(axis cs:-7.24189189189189,-1.92633928571429)
--(axis cs:-7.25,-1.40401785714286)
--cycle;

\path [draw=black, fill=color2] (axis cs:-4.33108108108108,-0.568303571428571)
--(axis cs:-4.33918918918919,-0.367410714285714)
--(axis cs:-4.33108108108108,-0.319196428571428)
--(axis cs:-4.29864864864865,-0.343303571428571)
--(axis cs:-4.25,-0.552232142857143)
--cycle;

\path [draw=black, fill=color2] (axis cs:-4.33108108108108,-2.23169642857143)
--(axis cs:-4.33918918918919,-2.43258928571429)
--(axis cs:-4.33108108108108,-2.48080357142857)
--(axis cs:-4.29864864864865,-2.45669642857143)
--(axis cs:-4.25,-2.24776785714286)
--cycle;

\path [draw=black, fill=white] (axis cs:-6.71486486486486,-1.39598214285714)
--(axis cs:-6.69864864864865,-1.02633928571429)
--(axis cs:-6.65,-0.801339285714286)
--(axis cs:-6.58513513513514,-0.680803571428571)
--(axis cs:-6.52837837837838,-0.672767857142857)
--(axis cs:-6.01756756756757,-0.809375)
--(axis cs:-6.06621621621622,-0.945982142857143)
--(axis cs:-6.07432432432432,-1.15491071428571)
--(axis cs:-6.07432432432432,-1.64508928571429)
--(axis cs:-6.06621621621622,-1.85401785714286)
--(axis cs:-6.01756756756757,-1.990625)
--(axis cs:-6.52837837837838,-2.12723214285714)
--(axis cs:-6.58513513513514,-2.11919642857143)
--(axis cs:-6.65,-1.99866071428571)
--(axis cs:-6.69864864864865,-1.77366071428571)
--(axis cs:-6.71486486486486,-1.40401785714286)
--cycle;

\path [draw=black, fill=white] (axis cs:-3.91756756756757,-1.39598214285714)
--(axis cs:-3.94189189189189,-0.962053571428571)
--(axis cs:-4.03108108108108,-0.688839285714286)
--(axis cs:-4.08783783783784,-0.632589285714286)
--(axis cs:-4.61486486486486,-0.841517857142857)
--(axis cs:-4.56621621621622,-1.034375)
--(axis cs:-4.55,-1.203125)
--(axis cs:-4.55,-1.596875)
--(axis cs:-4.56621621621622,-1.765625)
--(axis cs:-4.61486486486486,-1.95848214285714)
--(axis cs:-4.08783783783784,-2.16741071428571)
--(axis cs:-4.03108108108108,-2.11116071428571)
--(axis cs:-3.94189189189189,-1.83794642857143)
--(axis cs:-3.91756756756757,-1.40401785714286)
--cycle;

\path [draw=black, fill=white] (axis cs:-6.02567567567568,-0.568303571428571)
--(axis cs:-5.81486486486487,-0.568303571428571)
--(axis cs:-5.81486486486487,-0.720982142857143)
--(axis cs:-5.92027027027027,-0.672767857142857)
--cycle;

\path [draw=black, fill=white] (axis cs:-6.02567567567568,-2.23169642857143)
--(axis cs:-5.81486486486487,-2.23169642857143)
--(axis cs:-5.81486486486487,-2.07901785714286)
--(axis cs:-5.92027027027027,-2.12723214285714)
--cycle;

\path [draw=black, fill=white] (axis cs:-5.76621621621622,-0.737053571428571)
--(axis cs:-5.76621621621622,-0.608482142857143)
--(axis cs:-5.73378378378378,-0.576339285714286)
--(axis cs:-5.20675675675676,-0.576339285714286)
--(axis cs:-5.18243243243243,-0.616517857142857)
--(axis cs:-5.23108108108108,-0.761160714285714)
--(axis cs:-5.30405405405405,-0.793303571428571)
--(axis cs:-5.57162162162162,-0.777232142857143)
--cycle;

\path [draw=black, fill=white] (axis cs:-5.76621621621622,-2.06294642857143)
--(axis cs:-5.76621621621622,-2.19151785714286)
--(axis cs:-5.73378378378378,-2.22366071428571)
--(axis cs:-5.20675675675676,-2.22366071428571)
--(axis cs:-5.18243243243243,-2.18348214285714)
--(axis cs:-5.23108108108108,-2.03883928571429)
--(axis cs:-5.30405405405405,-2.00669642857143)
--(axis cs:-5.57162162162162,-2.02276785714286)
--cycle;

\path [draw=black, fill=white] (axis cs:-5.15,-0.817410714285714)
--(axis cs:-5.02027027027027,-0.568303571428571)
--(axis cs:-4.28243243243243,-0.568303571428571)
--(axis cs:-4.29054054054054,-0.616517857142857)
--(axis cs:-4.70405405405405,-0.793303571428571)
--cycle;

\path [draw=black, fill=white] (axis cs:-5.15,-1.98258928571429)
--(axis cs:-5.02027027027027,-2.23169642857143)
--(axis cs:-4.28243243243243,-2.23169642857143)
--(axis cs:-4.29054054054054,-2.18348214285714)
--(axis cs:-4.70405405405405,-2.00669642857143)
--cycle;

\path [draw=black, fill=white] (axis cs:-3.2527027027027,-0.552232142857143)
--(axis cs:-3.09054054054054,-0.560267857142857)
--(axis cs:-2.96891891891892,-0.680803571428571)
--(axis cs:-2.91216216216216,-0.777232142857143)
--(axis cs:-2.89594594594595,-0.873660714285714)
--(axis cs:-2.89594594594595,-1.04241071428571)
--(axis cs:-2.98513513513514,-0.889732142857143)
--cycle;

\path [draw=black, fill=white] (axis cs:-3.2527027027027,-2.24776785714286)
--(axis cs:-3.09054054054054,-2.23973214285714)
--(axis cs:-2.96891891891892,-2.11919642857143)
--(axis cs:-2.91216216216216,-2.02276785714286)
--(axis cs:-2.89594594594595,-1.92633928571429)
--(axis cs:-2.89594594594595,-1.75758928571429)
--(axis cs:-2.98513513513514,-1.91026785714286)
--cycle;

\path [draw=blue, fill=blue] (axis cs:4.8375796178344,-4.96449704142012)
--(axis cs:4.8343949044586,-4.92603550295858)
--(axis cs:4.85668789808917,-4.87869822485207)
--(axis cs:4.84713375796178,-4.8491124260355)
--(axis cs:4.74522292993631,-4.88165680473373)
--(axis cs:4.68152866242038,-4.94378698224852)
--(axis cs:4.65286624203822,-4.93195266272189)
--(axis cs:4.60828025477707,-4.94970414201183)
--(axis cs:4.57643312101911,-4.96745562130178)
--(axis cs:4.50955414012739,-5.03550295857988)
--(axis cs:4.5,-5.12426035502959)
--(axis cs:4.51273885350319,-5.17751479289941)
--(axis cs:4.5796178343949,-5.25443786982249)
--(axis cs:4.73566878980892,-5.35207100591716)
--(axis cs:4.82484076433121,-5.38757396449704)
--(axis cs:5.10509554140127,-5.43786982248521)
--(axis cs:5.1656050955414,-5.43786982248521)
--(axis cs:5.24203821656051,-5.4112426035503)
--(axis cs:5.30891719745223,-5.34615384615385)
--(axis cs:5.37261146496815,-5.31065088757396)
--(axis cs:5.4171974522293,-5.30769230769231)
--(axis cs:5.47770700636943,-5.23668639053254)
--(axis cs:5.4968152866242,-5.16272189349112)
--(axis cs:5.43949044585987,-5.06213017751479)
--(axis cs:5.42993630573248,-4.9792899408284)
--(axis cs:5.39808917197452,-4.90828402366864)
--(axis cs:5.35350318471338,-4.85207100591716)
--(axis cs:5.28662420382166,-4.80473372781065)
--(axis cs:5.19426751592357,-4.78698224852071)
--(axis cs:5.20063694267516,-4.86094674556213)
--(axis cs:5.18152866242038,-4.89644970414201)
--(axis cs:5.1624203821656,-4.90236686390533)
--(axis cs:5.14968152866242,-4.92603550295858)
--(axis cs:5.14649681528662,-4.95266272189349)
--(axis cs:5.15605095541401,-5.02662721893491)
--(axis cs:5.17197452229299,-5.13609467455621)
--(axis cs:5.15605095541401,-5.20414201183432)
--(axis cs:5.11146496815287,-5.24852071005917)
--(axis cs:5.01273885350319,-5.28698224852071)
--(axis cs:4.95222929936306,-5.28698224852071)
--(axis cs:4.85668789808917,-5.25739644970414)
--(axis cs:4.82484076433121,-5.22781065088757)
--(axis cs:4.79936305732484,-5.14792899408284)
--(axis cs:4.80254777070064,-5.05029585798817)
--(axis cs:4.81210191082803,-5.00591715976331)
--cycle;

\path [draw=black, fill=black] (axis cs:4.82484076433121,-5.38757396449704)
--(axis cs:4.9203821656051,-5.47633136094675)
--(axis cs:5.06687898089172,-5.49704142011834)
--(axis cs:5.10191082802548,-5.47633136094675)
--(axis cs:5.10509554140127,-5.43786982248521)
--cycle;

\path [draw=black, fill=black] (axis cs:5.04140127388535,-4.86686390532544)
--(axis cs:5.10191082802548,-4.87278106508876)
--(axis cs:5.14331210191083,-4.88757396449704)
--(axis cs:5.18152866242038,-4.89644970414201)
--(axis cs:5.20063694267516,-4.86094674556213)
--(axis cs:5.19426751592357,-4.78698224852071)
--(axis cs:5.28662420382166,-4.80473372781065)
--(axis cs:5.31528662420382,-4.62130177514793)
--(axis cs:5.30573248407643,-4.60059171597633)
--(axis cs:5.22611464968153,-4.57396449704142)
--(axis cs:5.18789808917197,-4.57988165680473)
--(axis cs:5.1656050955414,-4.70118343195266)
--(axis cs:5.12738853503185,-4.77218934911243)
--cycle;

\path [draw=color3, fill=color3] (axis cs:4.65286624203822,-4.93195266272189)
--(axis cs:4.64012738853503,-4.9112426035503)
--(axis cs:4.60191082802548,-4.88461538461539)
--(axis cs:4.57643312101911,-4.9112426035503)
--(axis cs:4.57643312101911,-4.96745562130178)
--(axis cs:4.60828025477707,-4.94970414201183)
--cycle;

\path [draw=color3, fill=color3] (axis cs:4.87261146496815,-4.94082840236686)
--(axis cs:4.96178343949045,-4.90236686390533)
--(axis cs:5,-4.89940828402367)
--(axis cs:5.05095541401274,-4.91420118343195)
--(axis cs:5.07006369426752,-4.93491124260355)
--(axis cs:5.0828025477707,-4.96449704142012)
--(axis cs:5.09235668789809,-4.97337278106509)
--(axis cs:5.09872611464968,-4.93786982248521)
--(axis cs:5.10509554140127,-4.92603550295858)
--(axis cs:5.07643312101911,-4.89644970414201)
--(axis cs:5.04140127388535,-4.86686390532544)
--(axis cs:4.97770700636943,-4.85798816568047)
--(axis cs:4.92356687898089,-4.8698224852071)
--(axis cs:4.87261146496815,-4.90532544378698)
--cycle;

\path [draw=color4, fill=color4] (axis cs:4.8375796178344,-4.96449704142012)
--(axis cs:4.87261146496815,-4.94082840236686)
--(axis cs:4.87261146496815,-4.90532544378698)
--(axis cs:4.92356687898089,-4.8698224852071)
--(axis cs:4.89171974522293,-4.8698224852071)
--(axis cs:4.85668789808917,-4.87869822485207)
--(axis cs:4.8343949044586,-4.92603550295858)
--cycle;

\path [draw=color4, fill=color4] (axis cs:5.04140127388535,-4.86686390532544)
--(axis cs:5.07643312101911,-4.89644970414201)
--(axis cs:5.10509554140127,-4.92603550295858)
--(axis cs:5.14649681528662,-4.95266272189349)
--(axis cs:5.14968152866242,-4.92603550295858)
--(axis cs:5.1624203821656,-4.90236686390533)
--(axis cs:5.18152866242038,-4.89644970414201)
--(axis cs:5.14331210191083,-4.88757396449704)
--(axis cs:5.10191082802548,-4.87278106508876)
--cycle;

\path [draw=white!10.0!black, fill=white!10.0!black] (axis cs:4.8375796178344,-4.96449704142012)
--(axis cs:4.81210191082803,-5.00591715976331)
--(axis cs:4.80254777070064,-5.05029585798817)
--(axis cs:4.79936305732484,-5.14792899408284)
--(axis cs:4.82484076433121,-5.22781065088757)
--(axis cs:4.85668789808917,-5.25739644970414)
--(axis cs:4.95222929936306,-5.28698224852071)
--(axis cs:5.01273885350319,-5.28698224852071)
--(axis cs:5.11146496815287,-5.24852071005917)
--(axis cs:5.15605095541401,-5.20414201183432)
--(axis cs:5.17197452229299,-5.13609467455621)
--(axis cs:5.15605095541401,-5.02662721893491)
--(axis cs:5.14649681528662,-4.95266272189349)
--(axis cs:5.10509554140127,-4.92603550295858)
--(axis cs:5.09872611464968,-4.93786982248521)
--(axis cs:5.09235668789809,-4.97337278106509)
--(axis cs:5.0828025477707,-4.96449704142012)
--(axis cs:5.07006369426752,-4.93491124260355)
--(axis cs:5.05095541401274,-4.91420118343195)
--(axis cs:5,-4.89940828402367)
--(axis cs:4.96178343949045,-4.90236686390533)
--(axis cs:4.87261146496815,-4.94082840236686)
--cycle;

\path [draw=color5, fill=color5] (axis cs:5.18789808917197,-4.57988165680473)
--(axis cs:5.22611464968153,-4.57396449704142)
--(axis cs:5.30573248407643,-4.60059171597633)
--(axis cs:5.31528662420382,-4.56508875739645)
--(axis cs:5.26114649681529,-4.50295857988166)
--(axis cs:5.22929936305732,-4.49704142011834)
--(axis cs:5.20700636942675,-4.51479289940828)
--cycle;

\addplot [black, forget plot]
table {%
-6.63378378378378 -0.656696428571429
-6.86081081081081 -0.664732142857143
-7.07162162162162 -0.720982142857143
-7.13648648648649 -0.801339285714286
-7.13648648648649 -1.99866071428571
-7.07162162162162 -2.07901785714286
-6.86081081081081 -2.13526785714286
-6.63378378378378 -2.14330357142857
};
\addplot [black, forget plot]
table {%
-3.00945945945946 -0.970089285714286
-2.98513513513514 -0.945982142857143
-2.89594594594595 -1.08258928571429
-2.89594594594595 -1.71741071428571
-2.98513513513514 -1.85401785714286
-3.00945945945946 -1.82991071428571
};
\addplot [black, forget plot]
table {%
-3.26081081081081 -0.664732142857143
-4.0472972972973 -0.632589285714286
-3.00945945945946 -0.970089285714286
-3.00945945945946 -1.82991071428571
-4.0472972972973 -2.16741071428571
-3.26081081081081 -2.13526785714286
};
\addplot [semithick, blue, forget plot]
table {%
-2.75 -1.4
4.5 -1.4
};
\addplot [semithick, blue, forget plot]
table {%
3.75 -0.65
4.5 -1.4
3.75 -2.15
};
\addplot [semithick, blue, forget plot]
table {%
5 -4.85
5 5
};
\addplot [semithick, blue, forget plot]
table {%
4.25 4.25
5 5
5.75 4.25
};
\addplot [semithick, black, forget plot]
table {%
5 0
5 2
};
\addplot [semithick, black, forget plot]
table {%
4.625 1.625
5 2
5.375 1.625
};
\addplot [semithick, black, forget plot]
table {%
5 0
7 0
};
\addplot [semithick, black, forget plot]
table {%
6.625 0.375
7 0
6.625 -0.375
};
\addplot [semithick, black, forget plot]
table {%
5 1
5.1 1
5.1743214109251 0.996926043706003
5.24813513125266 0.98772517306245
5.32093693842672 0.972460239345397
5.39222952228421 0.951235517530571
5.46152588218767 0.924195993989552
5.52835265373337 0.89152637608584
5.59225334231018 0.853449830436276
5.6527914414207 0.810226458456754
5.70955341446317 0.762151519605818
5.76215151960582 0.709553414463167
5.81022645845675 0.652791441420701
5.85344983043628 0.592253342310184
5.89152637608584 0.528352653733366
5.92419599398955 0.461525882187673
5.95123551753057 0.392229522284215
5.9724602393454 0.320936938426719
5.98772517306245 0.248135131252661
5.996926043706 0.174321410925099
6 0.1
6 0
};
\addplot [semithick, black, forget plot]
table {%
6.375 0.375
6 0
5.625 0.375
};
\node at (axis cs:5,2)[
  anchor=south west,
  text=black,
  rotate=0.0
]{ $\north$};
\node at (axis cs:7,0)[
  anchor= west,
  text=black,
  rotate=0.0
]{ $\east$};
\node at (axis cs:5.7,0.7)[
  anchor=south west,
  text=black,
  rotate=0.0
]{ $\head$};
\node at (axis cs:5,0)[
  anchor= east,
  text=black,
  rotate=0.0
]{ $\origin$};
\end{axis}

\end{tikzpicture}

	\caption{Schematic overview of a scenario where both the ego vehicle and a pedestrian are approaching a non-signalized pedestrian crossing. The pedestrian has priority. 
	%The origin of the coordinate system is represented by $O$. The easting and northing coordinates are denoted by $\east$ and $\north$, respectively. The angle is denoted by $\head$.
	}
	\label{fig:scenario overview}
\end{figure}

%\begin{figure}
%	\centering
%	\scentags{>Ego; Braking; Stationary; Accelerating|
%		>Actor; Pedestrian; Initial state, Direction/Crossing, Long.\ position/In front of ego; Walking straight|
%	Road layout, non-signalized zebra crossing}
%\end{figure}


We first describe the scenario qualitatively using our proposed ontology. Next, the scenario is described quantitatively in \cref{sec:example quantitative}. Finally, in \cref{sec:example test case}, we show the differences if a test case is considered with a crossing pedestrian.



\subsection{Qualitative description of the pedestrian crossing}
\label{sec:example qualitative}

To describe the scenario according to the presented domain model, objects are instantiated from the classes presented in \cref{fig:ontology classes}. \Cref{fig:example qualitative} shows the objects for describing the scenario qualitatively. There are two \textit{actor categories}: one for the ego vehicle and one for the pedestrian. Four different activity categories are defined: \emph{braking}, \emph{stationary}, \emph{accelerating}, and \emph{walking straight}. The braking activity is described with the sinusoidal model of \cref{eq:sinusoidal}. The stationary activity is simply modeled using a constant, i.e.:
\begin{equation} \label{eq:constant model}
	\egoacceleration(t) = 0,\ \egospeed(\inittimeb) = \egospeedinitb,
\end{equation}
with $\egospeedinitb$ being the only parameter.
Both the activities \emph{accelerating} and \emph{walking straight} are described using a linear model:
\begin{align}
	\egoacceleration(t) &= \slopeego,\ \egospeed(\inittimec) = \egospeedinitc, \label{eq:ego accelerating} \\
	\pedspeed(t) &= \slopepedestrian,\ \pednorth(\inittime) = \pednorthinit. \label{eq:pedestrian walking}
\end{align}
Here, the parameters $\slopeego$ and $\slopepedestrian$ describe the rate of change of $\egospeed$ and $\pednorth$, respectively. The parameters $\egospeedinitc$ and $\pednorthinit$ describe the initial value of $\egospeed$ and $\pednorth$, respectively.


\begin{figure}
	\centering
	%\newlength\objectwidth\setlength{\objectwidth}{24em}
%\tikzstyle{object}=[draw, text width=\objectwidth-.5em, align=left, line width=1pt, minimum width=\objectwidth, anchor=north west, node distance=3pt]
\resizebox{\textwidth}{!}{%
\begin{tikzpicture}
\tikzstyle{every node}=[font=\footnotesize]
\node[object, fill=category](ego qualitative){{\bfseries Ego qualitative::Actor category}\\
	type: Vehicle\\
	tags: Ego vehicle, Passenger car};

\node[object, fill=category, below=of ego qualitative.south](pedestrian qualitative){{\bfseries Pedestrian qualitative::Actor category}\\
	type: Pedestrian\\
	tags: Pedestrian};

\node[object, fill=category, below=of pedestrian qualitative.south](braking){{\bfseries Braking::Activity category}\\
	model: Sinusoid, see \cref{eq:sinusoidal}\\
	state: Speed ($\egospeed$)\\
	tags: Braking};

\node[object, fill=category, below=of braking.south](stationary){{\bfseries Stationary::Activity category}\\
	model: Constant, see \cref{eq:constant model}\\
	state: Speed ($\egospeed$)\\
	tags: Stationary};

\node[object, fill=category, right=of ego qualitative.north east, anchor=north west](accelerating){{\bfseries Accelerating::Activity category}\\
	model: Linear, see \cref{eq:ego accelerating}\\
	state: Speed ($\egospeed$)\\
	tags: Accelerating};

%\node[object, fill=category, below=of accelerating.south](straight){{\bfseries Driving straight::Activity category}\\
%	model: Linear\\
%	state: Lateral position\\
%	tags: Driving straight};

\node[object, fill=category, below=of accelerating.south](walking){{\bfseries Walking straight::Activity category}\\
	model: Linear, see \cref{eq:pedestrian walking}\\
	state: Position ($\pednorth$)\\
	tags: Walking straight};

\node[object, fill=category, below=of walking.south](static category){{\bfseries Pedestrian crossing qualitative::Static environment category}\\
	description: Straight road with two lanes and a\\
	\leavevmode\phantom{description: }pedestrian crossing\\
	tags: Non-signalized zebra crossing};

\node[object, fill=scenariocategory, right=of accelerating.north east, anchor=north west](scenario category){{\bfseries Crossing pedestrian::Scenario category}\\
	description: A pedestrian is crossing the road \\
	\leavevmode\phantom{description: }on a zebra crossing in front of the \\
	\leavevmode\phantom{description: }ego vehicle\\
	static environment category: Pedestrian \\
	\leavevmode\phantom{static environment category: }crossing \\
	\leavevmode\phantom{static environment category: }qualitative\\
	actors: Ego qualitative, Pedestrian qualitative\\
	activities: Braking, Stationary, Accelerating, Walking straight\\
	acts: (Ego qualitative, Braking), , \\
	\leavevmode\phantom{acts: }(Ego qualitative, Stationary), \\
	\leavevmode\phantom{acts: }(Ego qualitative, Accelerating), \\
	\leavevmode\phantom{acts: }(Pedestrian qualitative, Walking straight)\\
	tags:};
\end{tikzpicture}
}

	\caption{The objects that are used to qualitatively describe the scenario that is schematically shown in \cref{fig:scenario overview}. The first line of each block shows the name (before the double colon) and the class from which the object is instantiated. The following lines show the attributes of the object with the name and value of the attribute before and after the colon, respectively.}
	\label{fig:example qualitative}
\end{figure}


The two \textit{actor categories}, the four \textit{activity categories}, and the \textit{static environment category}, are used by the \textit{scenario category}. The \textit{scenario category} has four acts. The first three acts assign the first three activity categories to the ego vehicle. The last act assigns the activity category \emph{Walking straight} to the pedestrian. 



\subsection{Quantitative description of the pedestrian crossing}
\label{sec:example quantitative}

The objects to describe the scenario quantitatively are shown in \cref{fig:example quantitative}. The two \textit{actors} refer to the quantitative counterparts of the \textit{actor categories} in \cref{fig:example qualitative}. Initial \cstart state vectors \cend are listed for each \textit{actor} using the coordinate frame that is shown in \cref{fig:scenario overview}. 
%It is not necessary to describe the initial northing position of the pedestrian, because it will be described by an activity. For a similar reason, the initial speed of both actors is not included. 
Since we are describing a real-world scenario, there is no need to define desired \cstart state vectors \cend or goals for the actors.


\begin{figure}
	\centering
	%\newlength\objectwidth\setlength{\objectwidth}{24em}
%\tikzstyle{object}=[draw, text width=\objectwidth-.5em, align=left, line width=1pt, minimum width=\objectwidth, anchor=north west, node distance=3pt]
\begin{tikzpicture}
\tikzstyle{every node}=[font=\footnotesize]
\node[object, fill=otherclass](ego){{\bfseries Ego::Actor}\\
	actor category: Ego qualitative\\
	initial states: $\egoeast=\SI{-20}{\meter}$, $\egonorth=\SI{-1.5}{\meter}$, $\egoheading=\ang{90}$\\
	desired states: \\
	goals: \\
	tags: };

\node[object, fill=otherclass, below=of ego.south](pedestrian){{\bfseries Pedestrian::Actor}\\
	actor category: Pedestrian qualitative\\
	initial states: $\pedeast=\SI{0}{\meter}$, $\pedheading=\ang{0}$\\
	desired states: \\
	goals: \\
	tags: };

\node[object, fill=otherclass, below=of pedestrian.south](ego braking){{\bfseries Ego braking::Set activity}\\
	activity category: Braking \\
	parameters: $\amplitude=\SI{-8}{\meter\per\second}$, $\duration=\SI{4}{\second}$, $\egospeedinit=\SI{8}{\meter\per\second}$ \\
	\attrtstart: $\inittime = \SI{0}{\second}$ \\
	duration: \SI{4}{\second} \\
	tags: };

\node[object, fill=otherclass, below=of ego braking.south](ego stationary){{\bfseries Ego stationary::Set activity}\\
	activity category: Stationary \\
	parameters: $\egospeedinitb=\SI{0}{\meter\per\second}$ \\
	\attrtstart: $\inittimeb = \SI{4}{\second}$ \\
	duration: \SI{3}{\second} \\
	tags:};

\node[object, fill=otherclass, below=of ego stationary.south](ego accelerating){{\bfseries Ego accelerating::Set activity}\\
	activity category: Accelerating \\
	parameters: $\slopeego=\SI{1.5}{\meter\per\second\squared}$, $\egospeedinitc=\SI{0}{\meter\per\second}$ \\
	\attrtstart: $\inittimec = \SI{7}{\second}$ \\
	duration: \SI{5}{\second} \\
	tags:};

%\node[object, fill=category, below=of accelerating.south](straight){{\bfseries Driving straight::Activity category}\\
%	model: Linear\\
%	state: Lateral position\\
%	tags: Driving straight};

\node[object, fill=otherclass, below=of ego accelerating.south](pedestrian walking){{\bfseries Pedestrian walking::Set activity}\\
	activity category: Walking \\
	parameters: $\slopepedestrian=\SI{1}{\meter\per\second}$, $\pednorthinit=\SI{-6}{\meter}$\\
	\attrtstart: $\inittime = \SI{0}{\second}$ \\
	duration: \SI{12}{\second} \\
	tags: };

\node[object, fill=otherclass, below=of pedestrian walking.south](static environment){{\bfseries Pedestrian crossing::Static environment}\\
	static environment category: Pedestrian crossing qualitative\\
	properties: \{road: \{lanes: 2, lanewidth: \SI{3}{\meter}, \\
	\leavevmode\phantom{properties: \{road: \{}coordinates: [(-60, 0), (60, 0)]\}, \\
	\leavevmode\phantom{properties: \{}footway: \{width: \SI{3}{\meter}, coordinates: [(0, 6), (0, -6)]\}\}\\
	tags: };

\node[object, fill=scenario, below=of static environment.south](scenario){{\bfseries Ego braking for crossing pedestrian::Scenario}\\
	static environment: Pedestrian crossing\\
	actors: Ego, Pedestrian\\
	activities: Ego braking, Ego stationary, Ego accelerating, \\
	\leavevmode\phantom{activities: }Pedestrian walking\\
	acts: (Ego, Ego braking), (Ego, Ego stationary), \\
	\leavevmode\phantom{acts: }(Ego, Ego accelerating), (Pedestrian, Pedestrian walking)\\
	events: Start braking, End braking, Start accelerating, \\
	\leavevmode\phantom{events: }End accelerating, Start walking, Stop walking\\
	\attrtstart: \SI{0}{\second} \\
	\attrtend: \SI{12}{\second} \\
	tags:};
\end{tikzpicture}
	\caption{The objects that are used to quantitatively describe the scenario that is schematically shown in \cref{fig:scenario overview}.}
	\label{fig:example quantitative}
\end{figure}

There are four \textit{activities} defined and each of these \textit{activities} refers to its qualitative counterpart. The \textit{activities} contain the values of the parameters as well as the start time and the duration. As described by the first \textit{activity} (\emph{ego braking}), the ego vehicle starts with a speed of \SI{8}{\meter\per\second} and brakes in \SI{4}{\second} to come to a full stop. By integrating the sinusoidal function of \cref{eq:sinusoidal} twice, it can be shown that the ego vehicle stops at \SI{4}{\meter} from the center of the pedestrian crossing. After waiting for \SI{3}{\second}, see the second \textit{activity} (\emph{ego stationary}), the ego vehicle accelerates with \SI{1.5}{\meter\per\second\square} towards a speed of \SI{7.5}{\meter\per\second}, see the third \textit{activity} (\emph{ego accelerating}). The fourth activity describes the position of the pedestrian.

The \textit{static environment} describes the main road the ego vehicle is driving on and the footway the pedestrian is walking on. The example in \cref{fig:example quantitative} shows some properties to illustrate how the static environment can be described. \cstart Note that, in practice, the quantitative description of the static environment may contain many more facets than the ones mentioned in \cref{fig:example quantitative}. \cend As mentioned in \cref{sec:scenario}, it is possible to refer to another source that contains a description of (part of) the static environment, see, e.g., \autocite{dupuis2010opendrive}. 

The \textit{scenario}, the last object in \cref{fig:example quantitative}, ``has'' the previously defined \textit{static environment}, \textit{actors}, and \textit{activities}. The acts are used to assign the first three \textit{activities} to the ego vehicle and the last \textit{activity} to the pedestrian. The events happen at the starts and ends of the activities. For the sake of brevity, the events itself are not defined in \cref{fig:example quantitative}. The scenario also ``has'' a start time and an end time.

\cstart A different scenario can be defined by, e.g., changing the parameter values. This illustrates that the scenario category in \cref{fig:example qualitative} comprises multiple scenarios, namely all scenarios that only differ from the scenario in \cref{fig:example quantitative} because of different parameter values.\cend

\subsection{Test case with pedestrian crossing}
\label{sec:example test case}
% Differences:
% - No activities of ego vehicle
% - Initial distance further away
% - Initial speed defined (one might argue that it should not be 0)
% - Note: no additional goals like "stay in lane" and "do not collide with pedestrian". This should always be the goal, not only in this case.
% - Pedestrian walking is now a triggered activity (see also event)
% - Scenario does not describe activities of ego vehicle anymore


\begin{figure}
	\centering
	%\newlength\objectwidth\setlength{\objectwidth}{24em}
%\tikzstyle{object}=[draw, text width=\objectwidth-.5em, align=left, line width=1pt, minimum width=\objectwidth, anchor=north west, node distance=3pt]
\begin{tikzpicture}
\tikzstyle{every node}=[font=\footnotesize]

\node[object, fill=scenariocategory](scenario category){{\bfseries Crossing pedestrian::Scenario category}\\
	description: A pedestrian is crossing the road \\
	\leavevmode\phantom{description: }on a zebra crossing in front of the \\
	\leavevmode\phantom{description: }ego vehicle\\
	static environment category: Pedestrian \\
	\leavevmode\phantom{static environment category: }crossing \\
	\leavevmode\phantom{static environment category: }qualitative\\
	actors: Ego qualitative, Pedestrian qualitative\\
	activities: Walking straight\\
	acts: (Pedestrian qualitative, Walking straight)\\
	tags:};

\node[object, fill=otherclass, right=of scenario category.north east, anchor=north west](event){{\bfseries Start walking::Event}\\
conditions: $|\egoeast/\egospeed| \leq \SI{2.5}{\second}$};

\node[object, fill=otherclass, below=of event.south](ego){{\bfseries Ego::Actor}\\
	actor category: Ego qualitative\\
	initial state vector: $\egoeast=\SI{-40}{\meter}$, \\
	\leavevmode\phantom{initial state vector: }$\egonorth=\SI{-1.5}{\meter}$, \\
	\leavevmode\phantom{initial state vector: }$\egoheading=\ang{90}$, \\
	\leavevmode\phantom{initial state vector: }$\egospeed=\SI{8}{\meter\per\second}$\\
	desired state vector: $\egoeast=\SI{20}{\meter}$, \\ 
	\leavevmode\phantom{desired state vector: }$\egonorth=\SI{-1.5}{\meter}$, \\ 
	\leavevmode\phantom{desired state vector: }$\egoheading=\ang{90}$, \\ 
	\leavevmode\phantom{desired state vector: }$\egospeed=\SI{8}{\meter\per\second}$\\
	goals: \\
	tags: };

\node[object, fill=otherclass, right=of event.north east, anchor=north west](walking){{\bfseries Pedestrian walking::Triggered activity}\\
	activity category: Walking\\
	parameters: $\slopepedestrian=\SI{1}{\meter\per\second}$, $\pednorthinit=\SI{-6}{\meter}$\\
	trigger: Start walking \\
	duration: \SI{12}{\second} \\
	tags: };

\node[object, fill=scenario, below=of walking.south](scenario){{\bfseries Ego brakes for pedestrian::Scenario}\\
	static environment: Pedestrian crossing\\
	actors: Ego, Pedestrian\\
	activities: Pedestrian walking\\
	acts: (Pedestrian, Pedestrian walking)\\
	events: Start walking, Stop walking\\
	\attrtstart: \SI{0}{\second} \\
	\attrtend: \SI{100}{\second}\\
	tags:};
\end{tikzpicture}
	\caption{The objects that, next to the objects \emph{Ego qualitative}, \emph{Pedestrian qualitative}, \emph{Walking straight}, and \emph{Pedestrian crossing qualitative} from \cref{fig:example qualitative} and \emph{Pedestrian} and \emph{Pedestrian crossing} from \cref{fig:example quantitative}, describe a test case that is schematically shown in \cref{fig:scenario overview}.}
	\label{fig:example test case}
\end{figure}


In this example, we consider a test case based on the previously illustrated real-world scenario, see \cref{fig:scenario overview}. This text case might be used, e.g., for the assessment of a pedestrian automatic emergency braking system \autocite{seiniger2015test}. \Cref{fig:example test case} shows the objects that are used to describe this test case. Additionally, the two \textit{actor categories} (\emph{ego qualitative} and \emph{pedestrian qualitative}) are shown in \cref{fig:example qualitative} and the \textit{actor} describing the pedestrian (\emph{pedestrian crossing}) is shown in \cref{fig:example quantitative}.


%\Cref{fig:example test case} shows the objects that change when describing a test case instead of the scenario that is described in \cref{fig:example qualitative,fig:example quantitative}. There are five objects that are different:

The \textit{scenario category} only differs from the \textit{scenario category} shown in \cref{fig:example qualitative} in that it does not contain the \textit{activity categories} that describe the activity of the ego vehicle.

Two attributes of the quantitative description of the ego vehicle are different. First, the initial state \cstart vector \cend also includes the speed, denoted by $\egospeed$, at the start of the scenario and the initial position is further away from the pedestrian crossing, such that the ego vehicle's driver or automation system has more time to perceive the pedestrian. 
%Note that we choose for a start at rest as this is most practical when performing a test in real life. Therefore, the ego vehicle needs to accelerate to approach the pedestrian crossing, which is why the initial distance is larger (\SI{60}{\meter} instead of \SI{20}{\meter}). 
Secondly, because there are no activities defined for the ego vehicle, the desired \cstart state vector is \cend defined. The goal is to reach the point \SI{80}{\meter} in front of the ego vehicle while driving with a speed of $\egospeed=\SI{8}{\meter\per\second}$.

The \textit{event} that marks the start of the walking activity of the pedestrian is triggered when the ego vehicle is \SI{2.5}{\second} away from the center of the footway, assuming that the speed of the ego vehicle is constant. In case the ego vehicle drives with a speed of $\egospeed=\SI{8}{\meter\per\second}$, this is at a distance of \SI{20}{\meter}, similar to the scenario described in \cref{fig:example quantitative}.

For the test case, the pedestrian starts walking when the corresponding event is triggered. Hence, this \textit{activity} is a \textit{triggered activity} instead of a \textit{set activity}.

As with the \textit{scenario category}, the \textit{scenario} does not contain \textit{activities} of the ego vehicle. Furthermore, the end time is set to a much larger number. The test can also end earlier if the ego vehicle reaches its destination, so the end time is an upper bound in case the ego vehicle does not manage to reach its destination.

