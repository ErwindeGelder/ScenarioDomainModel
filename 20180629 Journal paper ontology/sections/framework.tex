\cstartb
\section{Object-oriented framework for scenarios}
\label{sec:oo framework}
\cendb

We have already explained the use of an \cstartb object-oriented framework \cendb in \cref{sec:why oo framework}. In this section, we present our \cstartb object-oriented framework \cendb for scenarios for the assessment of AVs. 
The \cstartb overview of the framework \cendb is formally represented through \cstartb class diagrams \cendb that are briefly presented in \cref{sec:class diagram}. Next, in \cref{sec:domain scenario category}, we explain how a scenario category is formally represented \cstartb in our framework\cendb. Similarly, in \cref{sec:domain scenario}, we describe how a scenario is formally represented. 



\cstartb
\subsection{Class diagrams}\cendb
\label{sec:class diagram}

The classes that are used to define a scenario and a scenario category are shown in \cref{fig:class overview}, where each block represents a class\footnote{In the remainder of this paper, when referring to (an instance of) a class, italic font is used.}.
The blue blocks represent the classes that are used to qualitatively describe a scenario whereas the orange blocks represent the classes that are used to quantitatively describe a scenario. \cstartb The yellow blocks represents so-called abstract classes. Abstract classes cannot be instantiated. \cendb

\begin{figure}[t]
	\centering
	\begin{tikzpicture}
\tikzstyle{every node}=[font=\footnotesize]

% Top classes
\node[class, fill=abstractclass](thing) at (0, 0) {Thing};
\node[class, fill=abstractclass](qualitative) at (-\blockx, 0) {Qualitative thing};
\node[class, fill=abstractclass](quantitative) at (\blockx, 0) {Quantitative thing};
\node[class, fill=abstractclass](physicalcat) at (-1.5\blockx, 2\blocky) {Physical thing category};
\node[class, fill=abstractclass](timeinterval) at (.5\blockx, \blocky) {Time interval};
\node[class, fill=abstractclass](physical) at (1.5\blockx, 2\blocky) {Physical thing};

% Qualitative classes
\node[class, fill=scenariocategory](scencat) at (-1.5\blockx, \blocky) {Scenario category};
\node[class, fill=category](activitycat) at (-.5\blockx, \blocky) {Activity category};
\node[class, fill=category](dynamiccat) at (-1.5\blockx, 3\blocky) {Dynamic physical thing category};
\node[class, fill=category](staticcat) at (-.5\blockx, 3\blocky) {Static physical thing category};
\node[class, fill=category](actorcat) at (-1.5\blockx, 4\blocky) {Actor category};

% Quantitative classes
\node[class, fill=scenario](scenario) at (-.5\blockx, 2\blocky) {Scenario};
\node[class, fill=otherclass](activity) at (.5\blockx, 2\blocky) {Activity};
\node[class, fill=otherclass](event) at (1.5\blockx, \blocky) {Event};
\node[class, fill=otherclass](dynamic) at (1.5\blockx, 3\blocky) {Dynamic physical thing};
\node[class, fill=otherclass](actor) at (1.5\blockx, 4\blocky) {Actor};
\node[class, fill=otherclass](static) at (.5\blockx, 3\blocky) {Static physical thing};

% Superclass arrows
\foreach \fromclass/\toclass in {scencat/qualitative,
								 activitycat/qualitative,
								 physicalcat/qualitative,
								 staticcat/physicalcat,
								 timeinterval/quantitative,
								 scenario/timeinterval,
								 event/quantitative,
								 physical/quantitative,
								 static/physical} {
	\node[coordinate, above of=\fromclass, node distance=-.4\blocky](helper){};
	\draw[superclass] (\fromclass) -- (helper) -| (\toclass);
}
\foreach \fromclass/\toclass in {qualitative/thing, 
								 quantitative/thing,
								 dynamiccat/physicalcat,
								 actorcat/dynamiccat,
								 activity/timeinterval,
								 dynamic/physical,
								 actor/dynamic} {
	\draw[superclass] (\fromclass) -- (\toclass);
}

\end{tikzpicture}
	\caption{\color{red}Schematic overview of most classes of our object-oriented framework.\color{black}}
	\label{fig:class overview}
\end{figure}

The arrows in \cref{fig:class overview} denote subclass relations as is best described by the verb ``is''. Because all other classes in \cref{fig:class overview} are subclasses of a \textit{thing}, these classes inherit the attributes and procedures of a \textit{thing}. In our framework, a \textit{thing} has a human-interpretable name, a unique ID, and possibly predefined tags that are also interpretable by a software agent. So, all other classes in \cref{fig:class overview} also have there attributes. 

\cstartb \Cref{fig:class relations} shows the relation between the classes that are used to represent a scenario. \cendb
The arrow with the diamond in \cref{fig:class relations} is best described by the verb ``to have''\footnote{In UML, this is called an aggregation.  This is typically implemented using pointers. For example, if object A ``has'' B, it means that A contains a pointer to B.}. 
\cstartb Here, the ``\hasone'' at the start of the arrow from \textit{static environment} to \textit{scenario} indicates that a \textit{scenario} ``has'' one \textit{static environment}. Similarly, ``\hastwo'' at the aggregation arrow from \textit{event} to \textit{time interval} indicates that a \textit{time interval} ``has'' two \textit{events}, i.e., the events that define the start and the end of the time interval. \cendb 
A ``\hasn'' at the start of an aggregation arrow indicates that an object ``has'' zero, one, or multiple objects of the corresponding class.

\begin{figure}[t]
	\centering
	\setlength{\blockwidth}{7.1em}
\begin{tikzpicture}
\tikzstyle{every node}=[font=\footnotesize]

% Qualitative classes
\node[class, fill=scenariocategory](scenario category) at (.5\blockx,0) {Scenario category};
\node[class, fill=category](actorcategory) at (\blockx, \blocky) {Actor category};
\node[class, fill=category](activitycategory) at (2\blockx, \blocky) {Activity category};
\node[class, fill=abstractclass](model) at (1.5\blockx+0.25\blockwidth, 2\blocky) {\textit{Model}};
\node[class, fill=category](elementcategory) at (3\blockx, \blocky) {Physical element category};

% Quantitative classes
\node[class, fill=scenario](scenario) at (.5\blockx, 2\blocky) {Scenario};
\node[class, fill=otherclass](actor) at (\blockx, 3\blocky) {Actor};
\node[class, fill=otherclass](activity) at (2\blockx, 3\blocky) {Activity};
\node[class, fill=otherclass](element) at (3\blockx, 3\blocky) {Physical element};
\node[class, fill=otherclass](event) at (4\blockx, 3\blocky) {Event};
\node[class, fill=abstractclass](timeinterval) at (4\blockx, 2\blocky) {\textit{Time interval}};

% Aggregation arrows for the scenario category
\node[coordinate, below of=scenario category, node distance=-\blocky/2, xshift=\blockwidth/3](helper scenario category){};
\node[coordinate, below of=scenario category, node distance=\blockheight/2, xshift=\blockwidth/3](aggregation scenario category){};
\foreach \class in {actor, activity, element}
{
	\node[coordinate, above of=\class category, node distance=\blockheight/2](helper \class){};  % Needed for later
	\draw[aggregation] (\class category) |- (helper scenario category) -- (aggregation scenario category);
	\node[anchor=south east] at (helper \class) {\hasn};
}

% Aggregation arrow for the model
\foreach \fromclass/\toclass in {model/activitycategory} {
	\node[coordinate, above of=\fromclass, node distance=\blockheight/2, xshift=-\blockwidth/8+\blockx/4](aggregation \fromclass){};
	\node[coordinate, below of=\toclass, node distance=\blockheight/2, xshift=\blockwidth/8-\blockx/4](aggregation \toclass){};
	\draw[aggregation] (aggregation \fromclass) -- (aggregation \toclass);
}
\node[anchor=south east] at (aggregation model) {\hasone};

% Aggregation arrow for scenario
\node[coordinate, below of=scenario, node distance=-\blocky/2](helper scenario){};
\node[coordinate, below of=scenario, node distance=\blockheight/2](aggregation scenario){};
\foreach \class in {actor, activity, element}
{
	\node[coordinate, above of=\class, node distance=\blockheight/2](helper \class){};
	\draw[aggregation] (helper \class) |- (helper scenario) -- (aggregation scenario);
	\node[anchor=south east] at (helper \class) {\hasn};
}
\node[coordinate, above of=event, node distance=\blockheight/2](helper event){};
\draw[aggregation] (helper event) |- (helper scenario) -- (aggregation scenario);
\node[anchor=south east] at (helper event) {$2,3,\ldots$};

% Aggregations for static thing, activity, and actor
\foreach \class in {element, activity, actor}
{
	\node[coordinate, below of=\class category, node distance=\blockheight/2, xshift=\blockwidth/4](category helper){};
	\node[coordinate, above of=\class, node distance=\blockheight/2, xshift=\blockwidth/4](helper){};
	\draw[aggregation] (category helper) -- (helper);
	\node[anchor=north east] at (category helper) {\hasone};
}

% Aggregation for event -> time interval
\node[coordinate, above of=event, node distance=\blockheight/2, xshift=\blockwidth/4](helper event){};
\node[coordinate, below of=timeinterval, node distance=\blockheight/2, xshift=\blockwidth/4](helper timeinterval){};
\draw[aggregation] (helper event) -- (helper timeinterval);
\node[anchor=south east] at (helper event) {\hastwo};

% falls into arrows
\node[coordinate, right of=scenario category, node distance=\blockwidth/2+1pt, yshift=-\blockheight/3](helper1){};
\node[coordinate, right of=scenario category, node distance=\blockwidth/2+1pt, yshift=\blockheight/3](helper2){};
\node[coordinate, right of=helper1, node distance=\blockwidth/2](helper3){};
\node[coordinate, right of=helper2, node distance=\blockwidth/2](helper4){};
\draw[falls into] (helper1) -- (helper3) -- node[fill=white]{includes} (helper4) -- (helper2);
\node[coordinate, above of=scenario, node distance=\blockheight/2, xshift=-.28\blockwidth](helper1){};
\node[coordinate, below of=scenario category, node distance=\blockheight/2, xshift=-.28\blockwidth](helper2){};
\draw[falls into] (helper2) -- node[fill=white, align=center, text width=3.55em]{comprises} (helper1);

% Legend
\node[class, fill=TNOlightgray, minimum height=6.5em, text width=2\blockwidth+1em](legend) at (4.5\blockx, .4\blocky) {};
\node[yshift=2.65em] at (legend) {Legend};
\node[class, fill=abstractclass, minimum height=1.5em, text width=2\blockwidth, yshift=1.3em](abstract) at (legend) {\textit{Abstract class}};
\node[class, fill=category, minimum height=1.5em, text width=2\blockwidth, yshift=-.4em](category) at (legend) {Class for qualitative description};
\node[class, fill=otherclass, minimum height=1.5em, text width=2\blockwidth, yshift=-2.1em](category) at (legend) {Class for quantitative description};

\end{tikzpicture}
	\caption{Schematic overview of the classes for representing the scenarios for the assessment of automated vehicles.}
	\label{fig:class relations}
\end{figure}

The arrow with the text ``comprises'' and ``includes'' represent methods that are explained in \cref{sec:scenario category}. Here, ``comprises'' can be denoted by $\comprises$ and ``includes'' can be denoted by $\includes$, see \cref{eq:scenario category include}. 



\subsection{Scenario category and its attributes}
\label{sec:domain scenario category}

The blue blocks in \cref{fig:class overview,fig:class relations} represent the classes that are used to model a scenario category according to the definition of a scenario category, see \cref{def:scenario category}.
The ego vehicle and the dynamic environment are qualitatively described by \textit{activity categories}, \cstartb\textit{dynamic physical thing categories}\cendb, and actor categories. 

The static environment is qualitatively described by a \textit{static environment category}. 
\cstartb The \textit{static environment category} ``has'' all relevant \textit{static physical thing categories}\cendb.
Because the \textit{static environment category} qualitatively describes the static environment, it contains a human-interpretable description of the static environment \cstartb and, similarly, a \textit{static physical thing category} contains a human-interpretable description of the static physical thing. \cendb

In line with the definition of an activity (\cref{def:activity}), the \textit{activity category} includes the state variable(s).
The \textit{model} that is used to describe the time evolution of the state  variable(s) is specified. For example, the \textit{model} to describe the speed of the ego vehicle during a braking activity could be a sinusoidal function:
\begin{equation} \label{eq:sinusoidal}
	\egoacceleration(\time) = \frac{\pi \amplitude}{2\duration} \sin \left( \frac{\pi \left( \time - \inittime\right)}{\duration} \right),\ \egospeed(\inittime) = \egospeedinit,\ \time \in [\inittime, \inittime+\duration].
\end{equation}
Here, $\egospeed$ and $\egoacceleration$ denote the speed and acceleration of the ego vehicle, respectively. Thus, in this case, the state variable corresponds to the speed. 
The parameters of the \textit{model} are the total speed difference ($\amplitude$) between the start of the activity and the end of the activity, the duration of the braking activity ($\duration$), and the initial speed ($\egospeedinit$) at time $\inittime$. 
The \textit{model} of \cref{eq:sinusoidal} describes the evolution of the state variable $\egospeed$ from time $\time=\inittime$ until $\time=\inittime+\duration$. Since the \textit{activity category} is a qualitative description, the values of these parameters of the \textit{model} are not included.
\cstartb Note that a \textit{model} is an abstract class that serves as a template for different models, such as a \textit{sinusoidal model} as described in \cref{eq:sinusoidal}. \cendb

Regarding the \cstartb \textit{dynamic physical thing category} and \cendb \textit{actor category}, the type of object is specified from a predefined list. 
\cstartb Here, the difference between an \textit{actor category} and a \textit{dynamic physical thing category} is that an \textit{actor category} has a (possibly unknown) intent. \cendb
To indicate that an actor is an ego vehicle, the tag ``Ego vehicle'' is added to the list of tags of the \textit{actor category}.

% Scenario category
The \textit{scenario category} ``has'' a \textit{static environment category}, \textit{activity categories}, \cstartb \textit{dynamic physical thing categories}, \cendb and \textit{actor categories}. 
%As with the other classes, a \textit{scenario category} contains a name and may contain predefined tags that describe parts of the scenario that are not described by the other classes.
Another attribute of the scenario category is the list of acts. %\footnote{ In line with the definition of \emph{act} in \cref{sec:act}, for a \textit{scenario category}, an \emph{act} is a combination of \textit{activity categories} and \textit{actor categories}.}. 
These acts describe which \cstartb dynamic things and \cendb actors perform which activities. Naturally, it is possible that one \cstartb dynamic thing or \cendb actor performs multiple activities and that one activity is performed by multiple \cstartb dynamic things and \cendb actors.

% Explain why we have these different classes
The reader might wonder why we introduce the different classes for describing a scenario category, i.e., the blue blocks, instead of only one class for modeling a scenario category. 
The main advantage of the different classes is the re-usability of the instances of the classes, because these instances can be exchanged among different \textit{scenario categories}. For example, if two \textit{scenarios categories} ``have'' the same \textit{static environment category}, we only need to define the \textit{static environment category} once, whereas if the \textit{static environment category} would not be a class on its own but only a property of the scenario category, we would need to define the \textit{static environment category} twice.



\subsection{Scenario and its attributes}
\label{sec:domain scenario}

The orange blocks in \cref{fig:class overview,fig:class relations} represent the classes that are used to quantitatively model a scenario according to the definition of a scenario, see \cref{def:scenario}. 
\cstartb A scenario ``is'' a \textit{time interval} and, therefore, it ``has'' events that define the start and the end of the scenario. \cendb
A \textit{scenario} ``has'' a \textit{static environment}, \textit{activities}, \cstartb \textit{dynamic physical things}, \cendb \textit{actors}, and \textit{events}. 
%The ego vehicle and the dynamic environment are quantitatively described by activities and actors. 
It is very similar to a \textit{scenario category}, as the \textit{static environment}, \textit{activities}, \cstartb \textit{dynamic physical things}, \cendb and \textit{actors} are the quantitative counterparts of the \textit{static environment category}, \textit{activity categories}, \cstartb \textit{dynamic physical thing categories}, \cendb and \textit{actor categories}, just as a \textit{scenario} is the quantitative counterpart of a \textit{scenario category}. 

% Static environment
The \textit{static environment} ``has'' a \textit{static environment category}. The most notable difference between the \textit{static environment category} and the \textit{static environment}, is that the \textit{static environment} \cstartb ``has'' \textit{static physical things} instead of \textit{static physical thing categories}. A \textit{static physical thing} ``has'' a \textit{static physical thing category} and it may have multiple properties that quantitatively define the static object. \cendb These properties define, e.g., the road layout, static weather and lighting conditions, and infrastructural elements, etc.

% Activity
According to \cref{def:activity}, an activity quantitatively describes the evolution of one or more state variables in a time interval. The state variable(s) are defined by the \textit{activity category} that the \textit{activity} ``has''. Together with the \textit{model} that is contained by the \textit{activity category}, the time evolution of the state variable is described by a set of parameters. The values of the parameters are part of the \textit{activity}. 
\cstartb Note that because an \textit{activity} ``is'' a \textit{time interval}, it ``has'' two events defining the start and the end of the \textit{activity}. \cendb
%Note that the model can be a time interpolation of (measured) data points, in which case the parameters are a sequence of time and state values. The time interval is defined by a duration of the activity.

% Set and triggered activity
%Two different types of activities can be defined. A set activity describes an activity that happens at a certain fixed time. This is often used to describe real-world scenarios that are extracted from real-world data. On the other hand, the starting time of a triggered activity is in general not defined beforehand, as the activity is triggered by an event. This is often used to describe test cases for scenario-based testing, e.g., see the example presented in \cref{sec:example test case}. 
%%Here, the starting time of the activity of the pedestrian, i.e., walking on the pedestrian crossing, is not defined beforehand and depends on, e.g., the speed of the ego vehicle \autocite{seiniger2015test}. 
%Both the \textit{set activity} and the \textit{triggered activity} are subclasses of the \textit{activity}, as shown in \cref{fig:framework classes}. Additionally, the \textit{set activity} ``has'' a starting time and the \textit{triggered activity} ``has'' an \textit{event} that triggers the activity.

% Event
Following the definition of an event (\cref{def:event}), an \textit{event} contains conditions that describe the threshold or mode transition at the time of the event.

% Actor
Similar to the \textit{static environment} and the \textit{activity}, the \cstartb\textit{dynamic physical thing} and \textit{actor} ``have'' their qualitative counterparts, the \textit{dynamic physical thing category} and \textit{actor category}, respectively. Additionally, because the \textit{dynamic physical thing} and \textit{actor} involve a quantitative description, they may have an initial state vector and multiple properties defined, such as their \cendb size, weight, color, radar cross section, etc. \cstartb The intent of the actor can be used to specify the desired state vector. \cendb This is especially useful for defining a test case that describes the objective of the ego vehicle rather than its activities. 
%Additionally, the goals can be formulated as text if they cannot be formulated using a desired state vector.
%Note that the goals are typically on a strategic level, such as destinations and waypoints, as opposed to the tactical and operational levels\footnote{ See \autocite[p.~7, Figure 1]{sae2018j3016} for an overview of the difference between the strategic, tactical, and operational functions.}.

%According to \cref{def:event}, an events marks the time instant at which the system reaches a specified boundary or at which a mode transition occurs. To describe this time instant, one or more conditions are specified. For example, a condition could be that the distance of the ego vehicle and a pedestrian crossing should be less than \SI{30}{\meter}. For this example, the moment at which this condition is met for the first time corresponds to the event.

% Another advantage of the blue blocks -> only one activity category for multiple activities
An advantage of having the qualitative counterparts of the \textit{static environment}, \cstartb \textit{static physical thing}\cendb, \textit{activity}, \cstartb \textit{dynamic physical thing},\cendb and \textit{actor} is that the qualitative description can be reused and exchanged. For example, there can be many different braking activities, but there needs to be only one \textit{activity category} for qualitatively defining the braking activity. Here, it is assumed that all braking activities are modeled with the same model and that similar tags apply. If this is not the case, multiple \textit{activity categories} need to be defined, but the number of \textit{activity categories} will still be significantly lower than the number of activities.

% Scenario
As with the \textit{static environment}, \textit{activity}, and \textit{actor}, the \textit{scenario} is the quantitative counterpart of the \textit{scenario category}. As a result, a \textit{scenario} ``has'' a \textit{static environment}, \textit{activities}, \cstartb \textit{dynamic physical things}, \cendb and \textit{actors}. 
Additionally, in compliance with \cref{def:scenario}, a \textit{scenario} ``has'' \textit{events}. 
As with a \textit{scenario category}, a \textit{scenario} contains a list of acts.
The acts are used to describe which \cstartb \textit{dynamic physical things} and \cendb actors perform which activities.
