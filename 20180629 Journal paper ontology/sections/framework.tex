\cstartb
\section{Object-oriented framework for scenarios}
\label{sec:oo framework}
\cendb

We have already explained the use of \cstartb \iac{oof} \cendb in \cref{sec:why oo framework}. In this section, we present our \cstartb \ac{oof} \cendb for scenarios for the assessment of \acp{av}. 
The \cstartb overview of the framework \cendb is formally represented through \cstartb class diagrams \cendb that are briefly presented in \cref{sec:class diagram}. Next, in \cref{sec:domain scenario category}, we explain how a scenario category is formally represented \cstartb in our framework\cendb. Similarly, in \cref{sec:domain scenario}, we describe how a scenario is formally represented. 



\cstartb
\subsection{Class diagrams}\cendb
\label{sec:class diagram}

In \cref{fig:class overview,fig:class relations}, the blue blocks represent the classes\footnote{In the remainder of this paper, when referring to (an instance of) a class, italic font is used. \cstarte Additionally, class names start with capital letters and instance names with lowercase letters.\cende} that are used to describe a scenario category according to \cref{def:scenario category} and the orange blocks represent the classes that are used to describe a scenario according to \cref{def:scenario}. \cstartb The yellow blocks represent so-called abstract classes. Abstract classes cannot be instantiated. \cendb\cstartf Each class serves as a template for creating objects whereas an object of a particular class is referred to as the instance of that particular class. \cendf

\cstartd \Cref{fig:class overview} shows the class-level relationships while \cref{fig:class relations} shows the instance-level relationships.
In \cref{fig:class overview}, the arrow from, e.g., \textit{Scenario} to \textit{Time interval}, denotes that \textit{Scenario} is a subclass of \textit{Time interval}. Therefore, all properties of the \textit{Time interval} are inherited by \textit{Scenario}. 
The arrow with the diamond in \cref{fig:class relations} denotes an aggregation%\footnote{This is typically implemented using pointers. For example, an aggregation arrow from \textit{B} to \textit{A} means that an object of class \textit{A} contains a pointer to an object of class \textit{B}.}
.
This means that, e.g., an \textit{actor}, which is an instance of the \textit{Actor} class, has an \textit{actor category} as an attribute. \cendd
Here, the ``\hasone'' at the start of the arrow from \textit{Actor category} to \textit{Actor} indicates that an \textit{actor} has \cstartd exactly \cendd one \textit{actor category}.
\cstartb Similarly, ``\hastwo'' at the aggregation arrow from \textit{Event} to \textit{Time interval} indicates that a \textit{time interval} contains two \textit{events}, i.e., the events that define the start and the end of the time interval. \cendb 
A ``\hasn'' at the start of an aggregation arrow indicates that an object has zero, one, or multiple objects of the corresponding class.
The arrow with the text ``comprises'' and ``includes'' represent methods that are explained in \cref{sec:scenario category}. Here, ``comprises'' can be denoted by $\comprises$ and ``includes'' can be denoted by $\includes$, see \cref{eq:scenario category include}. 

\begin{figure*}
	\centering
	\begin{tikzpicture}
\tikzstyle{every node}=[font=\footnotesize]

% Top classes
\node[class, fill=abstractclass](thing) at (0, 0) {Thing};
\node[class, fill=abstractclass](qualitative) at (-\blockx, 0) {Qualitative thing};
\node[class, fill=abstractclass](quantitative) at (\blockx, 0) {Quantitative thing};
\node[class, fill=abstractclass](physicalcat) at (-1.5\blockx, 2\blocky) {Physical thing category};
\node[class, fill=abstractclass](timeinterval) at (.5\blockx, \blocky) {Time interval};
\node[class, fill=abstractclass](physical) at (1.5\blockx, 2\blocky) {Physical thing};

% Qualitative classes
\node[class, fill=scenariocategory](scencat) at (-1.5\blockx, \blocky) {Scenario category};
\node[class, fill=category](activitycat) at (-.5\blockx, \blocky) {Activity category};
\node[class, fill=category](dynamiccat) at (-1.5\blockx, 3\blocky) {Dynamic physical thing category};
\node[class, fill=category](staticcat) at (-.5\blockx, 3\blocky) {Static physical thing category};
\node[class, fill=category](actorcat) at (-1.5\blockx, 4\blocky) {Actor category};

% Quantitative classes
\node[class, fill=scenario](scenario) at (-.5\blockx, 2\blocky) {Scenario};
\node[class, fill=otherclass](activity) at (.5\blockx, 2\blocky) {Activity};
\node[class, fill=otherclass](event) at (1.5\blockx, \blocky) {Event};
\node[class, fill=otherclass](dynamic) at (1.5\blockx, 3\blocky) {Dynamic physical thing};
\node[class, fill=otherclass](actor) at (1.5\blockx, 4\blocky) {Actor};
\node[class, fill=otherclass](static) at (.5\blockx, 3\blocky) {Static physical thing};

% Superclass arrows
\foreach \fromclass/\toclass in {scencat/qualitative,
								 activitycat/qualitative,
								 physicalcat/qualitative,
								 staticcat/physicalcat,
								 timeinterval/quantitative,
								 scenario/timeinterval,
								 event/quantitative,
								 physical/quantitative,
								 static/physical} {
	\node[coordinate, above of=\fromclass, node distance=-.4\blocky](helper){};
	\draw[superclass] (\fromclass) -- (helper) -| (\toclass);
}
\foreach \fromclass/\toclass in {qualitative/thing, 
								 quantitative/thing,
								 dynamiccat/physicalcat,
								 actorcat/dynamiccat,
								 activity/timeinterval,
								 dynamic/physical,
								 actor/dynamic} {
	\draw[superclass] (\fromclass) -- (\toclass);
}

\end{tikzpicture}
	\caption{Class-level relationships of most classes of our \acf{oof}.}
	\label{fig:class overview}
\end{figure*}

\begin{figure*}[t]
	\centering
	\setlength{\blockwidth}{7.1em}
\begin{tikzpicture}
\tikzstyle{every node}=[font=\footnotesize]

% Qualitative classes
\node[class, fill=scenariocategory](scenario category) at (.5\blockx,0) {Scenario category};
\node[class, fill=category](actorcategory) at (\blockx, \blocky) {Actor category};
\node[class, fill=category](activitycategory) at (2\blockx, \blocky) {Activity category};
\node[class, fill=abstractclass](model) at (1.5\blockx+0.25\blockwidth, 2\blocky) {\textit{Model}};
\node[class, fill=category](elementcategory) at (3\blockx, \blocky) {Physical element category};

% Quantitative classes
\node[class, fill=scenario](scenario) at (.5\blockx, 2\blocky) {Scenario};
\node[class, fill=otherclass](actor) at (\blockx, 3\blocky) {Actor};
\node[class, fill=otherclass](activity) at (2\blockx, 3\blocky) {Activity};
\node[class, fill=otherclass](element) at (3\blockx, 3\blocky) {Physical element};
\node[class, fill=otherclass](event) at (4\blockx, 3\blocky) {Event};
\node[class, fill=abstractclass](timeinterval) at (4\blockx, 2\blocky) {\textit{Time interval}};

% Aggregation arrows for the scenario category
\node[coordinate, below of=scenario category, node distance=-\blocky/2, xshift=\blockwidth/3](helper scenario category){};
\node[coordinate, below of=scenario category, node distance=\blockheight/2, xshift=\blockwidth/3](aggregation scenario category){};
\foreach \class in {actor, activity, element}
{
	\node[coordinate, above of=\class category, node distance=\blockheight/2](helper \class){};  % Needed for later
	\draw[aggregation] (\class category) |- (helper scenario category) -- (aggregation scenario category);
	\node[anchor=south east] at (helper \class) {\hasn};
}

% Aggregation arrow for the model
\foreach \fromclass/\toclass in {model/activitycategory} {
	\node[coordinate, above of=\fromclass, node distance=\blockheight/2, xshift=-\blockwidth/8+\blockx/4](aggregation \fromclass){};
	\node[coordinate, below of=\toclass, node distance=\blockheight/2, xshift=\blockwidth/8-\blockx/4](aggregation \toclass){};
	\draw[aggregation] (aggregation \fromclass) -- (aggregation \toclass);
}
\node[anchor=south east] at (aggregation model) {\hasone};

% Aggregation arrow for scenario
\node[coordinate, below of=scenario, node distance=-\blocky/2](helper scenario){};
\node[coordinate, below of=scenario, node distance=\blockheight/2](aggregation scenario){};
\foreach \class in {actor, activity, element}
{
	\node[coordinate, above of=\class, node distance=\blockheight/2](helper \class){};
	\draw[aggregation] (helper \class) |- (helper scenario) -- (aggregation scenario);
	\node[anchor=south east] at (helper \class) {\hasn};
}
\node[coordinate, above of=event, node distance=\blockheight/2](helper event){};
\draw[aggregation] (helper event) |- (helper scenario) -- (aggregation scenario);
\node[anchor=south east] at (helper event) {$2,3,\ldots$};

% Aggregations for static thing, activity, and actor
\foreach \class in {element, activity, actor}
{
	\node[coordinate, below of=\class category, node distance=\blockheight/2, xshift=\blockwidth/4](category helper){};
	\node[coordinate, above of=\class, node distance=\blockheight/2, xshift=\blockwidth/4](helper){};
	\draw[aggregation] (category helper) -- (helper);
	\node[anchor=north east] at (category helper) {\hasone};
}

% Aggregation for event -> time interval
\node[coordinate, above of=event, node distance=\blockheight/2, xshift=\blockwidth/4](helper event){};
\node[coordinate, below of=timeinterval, node distance=\blockheight/2, xshift=\blockwidth/4](helper timeinterval){};
\draw[aggregation] (helper event) -- (helper timeinterval);
\node[anchor=south east] at (helper event) {\hastwo};

% falls into arrows
\node[coordinate, right of=scenario category, node distance=\blockwidth/2+1pt, yshift=-\blockheight/3](helper1){};
\node[coordinate, right of=scenario category, node distance=\blockwidth/2+1pt, yshift=\blockheight/3](helper2){};
\node[coordinate, right of=helper1, node distance=\blockwidth/2](helper3){};
\node[coordinate, right of=helper2, node distance=\blockwidth/2](helper4){};
\draw[falls into] (helper1) -- (helper3) -- node[fill=white]{includes} (helper4) -- (helper2);
\node[coordinate, above of=scenario, node distance=\blockheight/2, xshift=-.28\blockwidth](helper1){};
\node[coordinate, below of=scenario category, node distance=\blockheight/2, xshift=-.28\blockwidth](helper2){};
\draw[falls into] (helper2) -- node[fill=white, align=center, text width=3.55em]{comprises} (helper1);

% Legend
\node[class, fill=TNOlightgray, minimum height=6.5em, text width=2\blockwidth+1em](legend) at (4.5\blockx, .4\blocky) {};
\node[yshift=2.65em] at (legend) {Legend};
\node[class, fill=abstractclass, minimum height=1.5em, text width=2\blockwidth, yshift=1.3em](abstract) at (legend) {\textit{Abstract class}};
\node[class, fill=category, minimum height=1.5em, text width=2\blockwidth, yshift=-.4em](category) at (legend) {Class for qualitative description};
\node[class, fill=otherclass, minimum height=1.5em, text width=2\blockwidth, yshift=-2.1em](category) at (legend) {Class for quantitative description};

\end{tikzpicture}
	\caption{Instance-level relationships of most classes of our \acf{oof}.}
	\label{fig:class relations}
\end{figure*}



\subsection{Scenario category and its attributes}
\label{sec:domain scenario category}

\cstartb Because all other classes in \cref{fig:class overview} are subclasses of \textit{Scenario element}, these classes inherit the attributes and procedures of \textit{Scenario element}. In our framework, a \textit{scenario element} has a human-interpretable name, a unique ID, and possibly predefined tags that are also interpretable by a software agent. So, all other classes in \cref{fig:class overview} also have these attributes. \cendb

The static environment is qualitatively described by \cstarte one or more \textit{physical element categories}. \cende
Because the \cstarte\textit{physical element categories} \cende qualitatively describe the static environment, they contain a human-interpretable description of the \cstartc physical things they describe\cendc.

The ego vehicle\cstartd(s) \cendd and the dynamic environment are qualitatively described by \textit{activity categories} and \textit{actor categories}. 
In line with \cref{def:activity}, \textit{Activity category} includes the state variable(s).
The \textit{Model} that is used to describe the time evolution of the state variable(s) is specified. 
\cstarte Note that \textit{Model} is an abstract class that serves as a template for different models, such as the three examples shown in \cref{fig:class overview}: \textit{Sinusoidal}, \textit{Linear}, and \textit{Constant}. 
Let $\state(\time)$ denote the state variable at time $\time$, then the \textit{Sinusoidal} model is defined as follows:
\begin{align}
	\statedot(\time) &= \frac{\pi \amplitude}{2\duration} \sin \left( \frac{\pi \left( \time - \timeinit\right)}{\duration} \right),\ \time \in [\timeinit, \timeinit+\duration], \label{eq:sinusoidala} \\
	\state(\timeinit) &= \stateinit. \label{eq:sinusoidalb}
\end{align}
Here, the amplitude ($\amplitude$), duration ($\duration$), initial time ($\timeinit$), and initial state ($\stateinit$) are parameters. 
The \textit{Linear} and \textit{Constant} models are described by the following equations, respectively:
\begin{align}
	\statedot(\time) &= \slope,\ \state(\timeinit) = \stateinit, \label{eq:linear} \\
	\state(\time) &= \stateinit. \label{eq:constant}
\end{align}
The \textit{Linear} model contains three parameters, i.e., the slope ($\slope$), initial time ($\timeinit$), and initial state ($\stateinit$). The \textit{Constant} model only has the parameter $\stateinit$.
Since an \textit{activity category} is a qualitative description, the values of the parameters of its \textit{model} are not part of the \textit{activity category}. \cende

The \textit{Actor category} has an attribute that specifies the type of object.
To indicate that an actor is an ego vehicle, the tag ``Ego vehicle'' is added to the list of tags of the \textit{actor category}.

% Scenario category
The \textit{Scenario category} has \cstarte \textit{physical element categories}\cende, \textit{activity categories}, and \textit{actor categories} as attributes. 
%As with the other classes, a \textit{scenario category} contains a name and may contain predefined tags that describe parts of the scenario that are not described by the other classes.
Another attribute of the \textit{Scenario category} is the list of acts. %\footnote{ In line with the definition of \emph{act} in \cref{sec:act}, for a \textit{scenario category}, an \emph{act} is a combination of \textit{activity categories} and \textit{actor categories}.}. 
These acts describe which actors perform which activities. Note that it is possible that one actor performs multiple activities and that one activity is performed by multiple actors.

% Explain why we have these different classes
The reader might wonder why we introduce the different classes for describing a scenario category, i.e., the blue blocks, instead of only one class for modeling a scenario category. 
The main advantage of the different classes is the reusability of the instances of the classes, because these instances can be exchanged among different \textit{scenario categories}. For example, if two \textit{scenario categories} have the same \textit{actor categories}, we only need to define the \textit{actor categories} once, whereas if the \textit{actor categories} would not be instances of a class but only properties of the scenario category, we would need to define the \textit{actor categories} twice.



\subsection{Scenario and its attributes}
\label{sec:domain scenario}

\cstartb The class \textit{Scenario} is a subclass of \textit{Time interval} and, therefore, it has \textit{events} that define the start and the end of the scenario. \cendb
The \textit{Scenario} also has \cstarte\textit{physical element}\cende, \textit{activities}, \textit{actors}, and \textit{events} as attributes. 
%The ego vehicle and the dynamic environment are quantitatively described by activities and actors. 
The \cstarte\textit{physical elements}\cende, \textit{activities}, and \textit{actors} are the quantitative counterparts of the \cstarte\textit{physical element categories}\cende, \textit{activity categories}, \cendb and \textit{actor categories}, just as a \textit{scenario} is the quantitative counterpart of a \textit{scenario category}. 
As with the \textit{Scenario category}, the \textit{Scenario} contains a list of acts that describe which actors perform which activities.

% Static environment
%The \textit{static environment} ``has'' a \textit{static environment category}. The most notable difference between the \textit{static environment category} and the \textit{static environment}, is that the \textit{static environment} \cstartb ``has'' \textit{static physical things} instead of \textit{static physical thing categories}. \cendb
\cstartb A \textit{physical element} has a \textit{physical element category} and it may have multiple properties that quantitatively define the object, such as its \cendb size, weight, color, radar cross section, etc.  \cendb Physical elements can be used to define, e.g., the road layout, static weather and lighting conditions, and infrastructural elements.

% Activity
According to \cref{def:activity}, an activity quantitatively describes the evolution of one or more state variables in a time interval. The state variable(s) are defined by the \textit{activity category} that the \textit{activity} has as an attribute. Together with the \textit{Model} that is contained by the \textit{activity category}, the time evolution of the state variable is described by a set of parameters. The values of the parameters are part of the \textit{activity}. 
%\cstartb Note that because the class \textit{activity} is a subclass of \textit{time interval}, it has two events defining the start and the end of the \textit{activity}. \cendb
%Note that the model can be a time interpolation of (measured) data points, in which case the parameters are a sequence of time and state values. The time interval is defined by a duration of the activity.

% Set and triggered activity
%Two different types of activities can be defined. A set activity describes an activity that happens at a certain fixed time. This is often used to describe real-world scenarios that are extracted from real-world data. On the other hand, the starting time of a triggered activity is in general not defined beforehand, as the activity is triggered by an event. This is often used to describe test cases for scenario-based testing, e.g., see the example presented in \cref{sec:example test case}. 
%%Here, the starting time of the activity of the pedestrian, i.e., walking on the pedestrian crossing, is not defined beforehand and depends on, e.g., the speed of the ego vehicle \autocite{seiniger2015test}. 
%Both the \textit{set activity} and the \textit{triggered activity} are subclasses of the \textit{activity}, as shown in \cref{fig:framework classes}. Additionally, the \textit{set activity} ``has'' a starting time and the \textit{triggered activity} ``has'' an \textit{event} that triggers the activity.

% Event
Following \cref{def:event}, an \textit{event} contains conditions that describe the threshold or mode transition at the time of the \textit{event}.

% Actor
Similar to a \textit{physical element} and an \textit{activity}, an \textit{actor} has its qualitative counterpart --- an \textit{actor category} --- as an attribute. Additionally, the \textit{Actor} contains an initial state vector and a desired state vector, that can be used to specify the intent, as attributes.
\cendb Describing the intent is especially useful for defining a test case in terms of the objective of the ego vehicle rather than its activities. \cendb
%Additionally, the goals can be formulated as text if they cannot be formulated using a desired state vector.
%Note that the goals are typically on a strategic level, such as destinations and waypoints, as opposed to the tactical and operational levels\footnote{ See \autocite[p.~7, Figure 1]{sae2018j3016} for an overview of the difference between the strategic, tactical, and operational functions.}.

%According to \cref{def:event}, an events marks the time instant at which the system reaches a specified boundary or at which a mode transition occurs. To describe this time instant, one or more conditions are specified. For example, a condition could be that the distance of the ego vehicle and a pedestrian crossing should be less than \SI{30}{\meter}. For this example, the moment at which this condition is met for the first time corresponds to the event.

% Another advantage of the blue blocks -> only one activity category for multiple activities
An advantage of having the qualitative counterparts of the \cstarte \textit{Physical element}\cende, \textit{Activity}, and \textit{Actor} is that the qualitative description can be reused and exchanged. For example, there can be many different braking activities, but there needs to be only one \textit{activity category} for qualitatively defining the braking activity. Here, it is assumed that all braking activities are modeled with the same model and that similar tags apply. If this is not the case, multiple \textit{activity categories} need to be defined, but the number of \textit{activity categories} will still be substantially lower than the number of activities.

