\begin{abstract}
	% Explain issue: assessment is important
	The development of assessment methods for the performance of Automated Vehicles (AVs) is essential to enable and speed up the deployment of automated driving technologies, due to the complex operational domain of AVs.
	% Introduce scenarios (for assessment)
	As traditional methods for assessing vehicles are not applicable for AVs, other approaches have been proposed. 
	Among these, real-world scenario-based assessment is widely supported by many players in the automotive field. 
	In this approach, test cases are derived from real-world scenarios that are obtained from driving data.
	% Problem -> definition needed for scenario
	\cstartc A proper specification of scenarios is crucial to minimize any ambiguity regarding these test cases and scenarios. 
	An intensional definition that provides a set of characteristics that are deemed to be both necessary and sufficient to qualify as a scenario could help to assure that the scenarios constructed are both complete and intercomparable.
	
	% Solution: concrete definitions
	In this paper, we develop a comprehensive and operable definition of scenario by bringing together existing definitions in the literature.
	This is achieved by proposing an object-oriented framework in which a scenario and its building blocks are defined as classes of objects having attributes, methods, and relationships with other objects.
	We also introduce the concept of a scenario category that enables the categorization of scenarios in terms of the categories of their typical components.
	% Benefits of our approach
	The object-oriented approach promotes clarity, modularity, reusability, and encapsulation of the objects. 
	We provide definitions and justifications of each of the terms
	Furthermore, the object-oriented framework is used to translate the terms in a coding language that is publicly available.\cendc
	% Example
	An example illustrates that the presented \cstartb framework \cendb is applicable for scenario-based assessment of AVs.
\end{abstract}
