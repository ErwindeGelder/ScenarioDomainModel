\section{Background}
\label{sec:background}

We first explain in \cref{sec:why oo framework} why we want to present an \cstartc object-oriented framework \cendc for describing scenarios and  scenario categories. Next, in \cref{sec:context}, we provide some information on the context for which we want to define scenarios. 


\cstartb
\subsection{Why an  object-oriented framework?}
\label{sec:why oo framework}

According to \textcite{johnson1988designing}, an object-oriented framework is a ``set of classes that embodies an abstract design for solutions to a family of related problems.''
The object orientation is used for ``a representation, modeling, and abstraction formalism'' \autocite{wegner1990concepts}, which is why it is considered ``not only useful but also fundamental'' \autocite{wegner1990concepts}. \cendb
\cstartd In addition, \textcite{patridge2005business} notes that object-oriented modeling can provide a bridge from traditional entity-relation based data modeling to data modeling that is fully grounded in a formalized ontology. \cendd
\cstartb An object-orientated framework offers the following benefits:
\begin{itemize}
	\item \cendb\cstartc\emph{Clarity:} \cendc\cstartb It provides ``a common vocabulary for designers to communicate, document, and explore design alternatives'' \autocite{gamma1993design}.
	\item \cendb\cstartc\emph{Modularity:} By decomposing a scenario into components, the complexity of a scenario itself is reduced. Thus, ``modularity makes it easier to understand the effect of changes'' \cite{johnson1988designing}.
	\item \emph{Reusability:} \cendc\cstartb An object-oriented framework promotes reusability \autocite{snyder1986encapsulation, meyer1987reusability, johnson1988designing}. E.g., if two classes share certain procedures and/or properties, these procedures and/or properties could be provided by a so-called superclass from which these two classes inherit the procedures and properties, such that these procedures and properties need to be defined only once.
	\item \cendb\cstartc\emph{Encapsulation:} \cendc\cstartb It offers the so-called encapsulation of data and procedures. This assures ``that compatible changes can be made safely, which facilitates program evolution and maintenance'' \autocite{snyder1986encapsulation}.
	\item \cendb\cstartc\emph{Translatable to object-oriented programming languages:} \cendc\cstartb As the framework consists of a set of classes, it can be directly used in an object-oriented coding language. The framework then specifies the relationships between the different classes and provided information on the properties of a class and the possible values.
\end{itemize}
\cendb
%According to Gruber \autocite{gruber1993ontology}, ``an ontology is an explicit specification of a conceptualization'', where a conceptualization refers to ``an abstract, simplified view of the world that we wish to represent for some purpose''. Ontologies are widely applied in all kinds of research areas, e.g., computing \autocite{chen2004soupa}, vehicle platooning \autocite{maiti2017conceptualization}, system performance \autocite{benvenuti2017ontologybased}, transportation \autocite{katsumi2018ontologies}, and vehicle path planning \autocite{provine2004ontology, schlenoff2003using}. 
%%Furthermore, ontologies are defined for various applications, e.g., for the profiling of users \autocite{golemati2007creating}, for the formalization of human activities \autocite{lee2017location}, and for product life cycle management \autocite{matsokis2010plm}. 
%An ontology offers the following benefits:
%\begin{itemize}
%	\item An ontology is useful for sharing knowledge between people and software agents \autocite{vanDamPhDThesis2009, noy2001ontology}. For example, the ontology can be used for communication between different software agents, such as different simulation tools \autocite{shao2019evaluating}.
%	\item An ontology can be directly translated into a class structure for an object-oriented software implementation \autocite{vanDamPhDThesis2009}. The ontology then specifies the relationships between the different classes and provides information on the properties of a class and the possible values.
%	\item Domain assumptions about terms and definitions are made explicit \autocite{noy2001ontology}. 
%	For example, we define the term \emph{scenario} in \cref{sec:scenario} and  this does not only help in preventing any ambiguity when communicating about scenarios, it also helps in understanding the underlying implementations, e.g., the programming language code, when assumptions are explicit.
%	\item An ontology can be used as a conceptual schema in database systems  \autocite{gruber1993ontology}. A conceptual schema allows ``databases to interoperate without having to share data structures'' \autocite{gruber1993ontology}. In addition, a conceptual schema can be directly used to design the structure of a database.
%\end{itemize}
%
%There are several languages dedicated to the implementation of an ontology. Two major technologies are the Web Ontology Language (OWL)\footnote{OWL is the standard abbreviation for the Web Ontology Language.} and the Unified Modeling Language (UML). 
%%We use UML for implementing the ontology, because UML focuses on describing implementation related issues, which is useful considering our Python implementation.
%For a detailed comparison between OWL and UML, we refer the interested reader to \autocite{kiko2005detailed}.



\subsection{Context of a scenario}
\label{sec:context}

Because the notion of scenario is used in many different contexts \cstartd outside of the domain of road traffic\cendd, a wide diversity in definitions of this notion exists (for an overview, see \autocite{vannotten2003updated, bishop2007scentechniques}). Therefore, it is reasonable to assume that ``there is no [generally] `correct' scenario definition'' \autocite{vannotten2003updated}. As a result, to define the notion of scenario, it is important to consider the context in which it will be used. 

In this paper, the context of a scenario is the assessment of AVs, where AVs refer to vehicles equipped with a driving automation system\footnote{According to \autocite{sae2018j3016}, a driving automation system is ``the hardware and software that are collectively capable of performing part or all of the dynamic driving task on a sustained basis. This term is used generically to describe any system capable of level 1-5 driving automation.'' Here, level 1 driving automation refers to ``driver assistance'' and level 5 refers to ``full driving automation''. For more details, see \autocite{sae2018j3016}.}. 
It is assumed that the assessment methodology uses scenarios. %(i.e., test cases) 
%for which some resulting metrics are compared with a reference \autocite{stellet2015taxonomy}. 
% - Scenarios that an (automated) vehicle can encounter
The ultimate goal is to build a database with all relevant scenarios that an AV has to cope with when driving in the real world \autocite{putz2017pegasus}. Hence, a scenario should be a description of a potential use case of an AV. 
% Whether these scenarios are obtained with a knowledge-based approach \autocite{gietelink2004systemvalidation, stellet2015taxonomy} or with a data-driven approach \autocite{deGelder2017assessment, stellet2015taxonomy}, a clear and unambiguous definition of such a test scenario is required. 

%In this paper, \emph{scenario} can refer to either an observed scenario in (real-world driving) data, i.e., a real-world scenario, or a scenario that is used for testing AVs, i.e., a test case. Note that, typically, the difference between the two is that with a real-world scenario, the activity of all actors is described, while for a test case, some goals are specified for the system under test (e.g., the goal could be to drive from A to B) instead of its activity. 



