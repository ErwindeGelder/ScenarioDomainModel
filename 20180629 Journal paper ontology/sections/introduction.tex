\section{Introduction}
\label{sec:introduction}

% Our contribution
An essential aspect in the development of automated vehicles (AVs) is the assessment of quality and performance aspects of the AVs, such as safety, comfort, and efficiency \cite{bengler2014threedecades, stellet2015taxonomy, Helmer2017safety, putz2017pegasus, roesener2017comprehensive, gietelink2006development, roesener2016scenariobased, wachenfeld2016release}.
\cbstartd
\textcite{bengler2014threedecades} comment that: ``Research on suitable assessment methods constitutes a part of the critical path to be taken in order to avoid driver assistance systems development being held up for decades.''
\cbend
For legal and public acceptance of AVs, a clear definition of system performance is important, as well as quantitative measures for system quality. 
Traditional methods, such as \cite{response2006code, ISO26262}, used for the evaluation of driver assistance systems, are no longer sufficient for the assessment of quality and performance aspects of an AV, because they would require too much resources \cite{wachenfeld2016release}. 
Therefore, among other methods, a scenario-based approach has been proposed \cite{roesener2016scenariobased, putz2017pegasus}. 
For scenario-based assessment, proper specification of scenarios is crucial since they are directly reflected in the test cases used for scenario-based assessment \cite{stellet2015taxonomy}. 

% Explain importance/relevance
The notion of scenario is frequently used in the context of automated driving \cite{putz2017pegasus, roesener2017comprehensive, gietelink2006development, hulshof2013autonomous, karaduman2013interactivebehavior, englund2016grand, xu2002effects, ebner2011identifying, ploeg2017GCDC, zofka2015datadrivetrafficscenarios, xiong2015orchestration, shao2019evaluating}, despite the fact that an explicit definition is often not provided. 
However, as mentioned by various authors \cite{stellet2015taxonomy, Helmer2017safety, alvarez2017prospective, zofka2015datadrivetrafficscenarios, aparicio2013pre, lesemann2011test, putz2017pegasus, geyer2014, ulbrich2015}, using a scenario in the context of the development or assessment of AVs requires a clear definition of a scenario. 
To this end, some definitions of a scenario in the context of (automated) driving have been proposed in literature \cite{geyer2014, ulbrich2015, elrofai2016scenario}.
\cbstart
These definitions, however, leave room for interpretation because, firstly, the definitions themselves are ambiguous and, secondly, the definitions use other undefined terms.
For the context of the assessment of AVs, scenarios are used to describe the test conditions \cite{stellet2015taxonomy} and this requires a more concrete definition of a scenario to minimize any ambiguity.
\cbend

% What is the added value of this paper w.r.t. other work
% - Easier to apply event/scenario mining, because more concrete
% - Important step towards quantification of completeness
% - More complete (also event defined etc.)
% - Backed up with literature (should we mention this???)
% - Real example given
We aim for a definition of a scenario that is, on the one hand, broadly consistent with existing definitions \cite{geyer2014, ulbrich2015, elrofai2016scenario} while, on the other hand, more concrete, such that it is applicable for scenario mining \cite{elrofai2016scenario} and scenario-based assessment \cite{stellet2015taxonomy, deGelder2017assessment}. We propose a definition that is concrete enough to be used in quantitative analysis required for the assessment of AVs. This is achieved by defining quantitative building blocks of a scenario in the form of activities, actors, and the static environment. In addition, we introduce the concept of a \cbstartd\emph{scenario category} \cbend 
that is used to qualitatively describe scenarios, i.e., an abstraction of a scenario. 

\cbstartb
Besides defining the notion of scenario, its building blocks, and the notion of scenario category, an ontology is presented that formalizes the description of a scenario and a scenario category. 
This ontology describes the relation between the different terms such as scenarios, activities, and actors.
Furthermore, the ontology enables the translation of the terms to object-oriented code.
This, in turn, is used to describe the scenarios in a coding language that can be understood by various software agents, such as simulation tools.
The ontology is also used as a schema for a database system for storing the scenarios and scenario categories.
An example is provided that illustrates the use of the ontology with a real-world case.
\cbend

% Outline
The outline of the paper is as follows. In \cref{sec:background}, we explain why an ontology is useful, what the context is, and we provide some definitions of terms that are adopted from literature. 
Using these definitions and considering the context, we define the notions of \emph{scenario}, \emph{event}, \emph{activity}, and \cbstartd\emph{scenario category} \cbend in \cref{sec:definitions}. 
The ontology that formalizes these definitions is presented in \cref{sec:ontology}. 
In \cref{sec:example}, an application example is provided to illustrate the use of the ontology with a real-world scenario. 
The paper is concluded in \cref{sec:conclusion}.
