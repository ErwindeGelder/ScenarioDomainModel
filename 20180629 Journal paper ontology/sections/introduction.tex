\section{Introduction}
\label{sec:introduction}

% For validating all traffic scenarios
An important aspect in the development of automated vehicles (AVs) is the assessment of quality and performance aspects of the AVs, such as safety, comfort, and efficiency \cite{bengler2014threedecades, stellet2015taxonomy, Helmer2017safety, putz2017pegasus, roesener2017comprehensive, gietelink2006development, roesener2016scenariobased, wachenfeld2016release}. 
For legal and public acceptance of AVs, it is important that there is a clear definition of system performance and that there are quantitative measures for the system quality. 
The more traditional methods, such as \cite{response2006code, ISO26262}, used for evaluation of driver assistance systems, are no longer valid for the assessment of quality and performance aspects of an AV \cite{wachenfeld2016release}. 
Therefore, a scenario-based approach has been proposed \cite{roesener2016scenariobased, putz2017pegasus}. 
For scenario-based assessment, proper specification of scenarios is crucial since they are directly reflected in the test cases used for scenario-based assessment \cite{stellet2015taxonomy}. 
The current paper proposes an ontology for these \emph{scenarios}.

% Explain importance/relevance
The notion of scenario is frequently used in the context of automated driving \cite{putz2017pegasus, roesener2017comprehensive, gietelink2006development, hulshof2013autonomous, karaduman2013interactivebehavior, englund2016grand, xu2002effects, ebner2011identifying, ploeg2017GCDC, zofka2015datadrivetrafficscenarios}, despite the fact that an explicit definition is often not provided. However, as mentioned by various authors \cite{stellet2015taxonomy, Helmer2017safety, alvarez2017prospective, zofka2015datadrivetrafficscenarios, aparicio2013pre, lesemann2011test, putz2017pegasus, geyer2014, ulbrich2015}, using a scenario in the context of the development or assessment of AVs requires a clear definition of a scenario. To this end, some definitions of a scenario in the context of (automated) driving have been proposed \cite{geyer2014, ulbrich2015, elrofai2016scenario}. 
These definitions, however, leave some room for interpretation because the definitions itself are somewhat ambiguous and because the definitions use other undefined terms.
For the context of the assessment of AVs, a more concrete definition of a scenario is required to minimize any ambiguity regarding the scenarios.

% What is the added value of this paper w.r.t. other work
% - Easier to apply event/scenario mining, because more concrete
% - Important step towards quantification of completeness
% - More complete (also event defined etc.)
% - Backed up with literature (should we mention this???)
% - Real example given
We aim for a definition of a scenario that is, on the one hand, broadly consistent with existing definitions \cite{geyer2014, ulbrich2015, elrofai2016scenario} while, on the other hand, more concrete, such that it is applicable for scenario mining \cite{elrofai2016scenario} and scenario-based assessment \cite{stellet2015taxonomy, deGelder2017assessment}. We propose a definition that is concrete enough to be used in quantitative analysis required for assessment of AVs. This is achieved by defining quantitative building blocks of scenarios in the form of activities, actors, and the static environment. In addition, we introduce the concept of a \emph{scenario class} that is used to qualitatively describe scenarios, i.e., an abstraction of a scenario. Finally, an ontology is presented that formalizes the description of a scenario and a scenario class. 
An example is provided that illustrates the use of the ontology with a real-world case.

% Clearly state the contribution of this paper
% \todo{Describe contribution of paper.}

% Outline
The outline of the paper is as follows. In \cref{sec:background}, we explain why an ontology is useful, what the context is, and some definitions of terms that adopted from literature. 
Using these definitions and considering the context, we define the notions of \emph{scenario}, \emph{event}, \emph{activity}, and \emph{scenario class} in \cref{sec:definitions}. 
The ontology that formalizes these definitions is presented in \cref{sec:ontology}. 
In \cref{sec:example}, an application example is provided to illustrate the use of the ontology with a real-world scenario. 
The paper is concluded in \cref{sec:conclusion}.
