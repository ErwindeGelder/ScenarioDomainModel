\section{Conclusions}
\label{sec:conclusion}

The performance assessment of AVs is essential for the legal and public acceptance of AVs as well as for technology development of AVs. 
Because scenarios are crucial for the assessment, a clear definition of a scenario is required.
In this work, we have proposed a new definition of a scenario in the context of the assessment of the performance of an AV. 
 
While our definition is consistent with other definitions from the literature, it is more concrete such that it can directly be implemented using code.
We have further defined the notions of event, activity, and scenario category. 
To formalize the definitions of a scenario, an event, an activity, and a scenario category, an \cstartb object-oriented framework \cendb has been proposed. Using the proposed \cstartb framework\cendb, it is possible to describe a scenario in both a qualitative and quantitative manner. The \cstartb framework\cendb, represented using \cstartb class diagrams\cendb, can be directly translated into a class structure for an object-oriented software implementation. This allows us to translate scenarios into code, such that both domain experts and software programs, such as simulation tools, are able to understand the content of the scenarios. 

The \cstartb framework \cendb has been illustrated with an example of an urban scenario with a pedestrian crossing. 
We also demonstrated how this particular scenario can be used as a test case. In addition, we showed how this test case can be represented using the proposed \cstartb framework\cendb.

The presented \cstartb framework \cendb is applicable for scenario mining \autocite{paardekooper2019dataset6000km, degelder2020scenariomining} and scenario-based assessment \autocite{elrofai2018scenario, putz2017pegasus} and, therefore, this paper provides a step towards the scenario-based performance assessment of AVs. The next step is to define scenarios and scenario categories\footnote{In \autocite{degelder2019scenariocategories}, already 67 scenario categories are defined.} that are relevant for an AV in a specific deployment area. 
\cstartc Future work is to create an ontology for scenarios for the assessment of \acp{av}. The presented object-oriented framework could be a good starting point \autocite{siricharoen2009ontology}. An ontology allows, among others, to add properties to relationships that enables automated reasoning. In this way, an ontology enables automated classification of scenarios, thereby helping to overcome problems of data ambiguity \autocite{OpenSCENARIO2}. \cendc

