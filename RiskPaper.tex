%%%%%%%%%%%%%%%%%%%%%%%%%%%%%%%%%%%%%%%%%%%%%%%%%%%%%%%%%%%%%%%%%%%%%%%%%%%%%%%%
%2345678901234567890123456789012345678901234567890123456789012345678901234567890
%        1         2         3         4         5         6         7         8

\documentclass[letterpaper, 10 pt, conference]{ieeeconf}  % Comment this line out if you need a4paper

%\documentclass[a4paper, 10pt, conference]{ieeeconf}      % Use this line for a4 paper

\IEEEoverridecommandlockouts                              % This command is only needed if
                                                          % you want to use the \thanks command

\overrideIEEEmargins                                      % Needed to meet printer requirements.

% See the \addtolength command later in the file to balance the column lengths
% on the last page of the document
\usepackage{amsfonts}
% The following packages can be found on http:\\www.ctan.org
\usepackage{graphics} % for pdf, bitmapped graphics files
\usepackage{multirow} % for tables
%\usepackage{cite} % for bibliography with biblatex
\usepackage{braket}
\usepackage{amsfonts}
\usepackage{amsmath}
\usepackage{amssymb}
\usepackage{breqn}
\usepackage[utf8]{inputenc}
\usepackage[style=ieee,doi=false,isbn=false,url=false,date=year,minbibnames=15,maxbibnames=15,backend=biber]{biblatex}
%\renewcommand*{\bibfont}{\footnotesize}		%% Use this for papers
\setlength{\biblabelsep}{\labelsep}
\bibliography{bibRiskPaper}
%\usepackage{subfig}
%\usepackage{amsthm}
%\usepackage{mathrsfs}
\usepackage{epsfig} % for postscript graphics files
%\usepackage{mathptmx} % assumes new font selection scheme installed
\usepackage{times} % assumes new font selection scheme installed
%\usepackage{amsmath} % assumes amsmath package installed
%\usepackage{amssymb}  % assumes amsmath package installed
\graphicspath{ {figures/} }
\usepackage[capitalize]{cleveref}
\usepackage{color}
\usepackage{units}
%\usepackage[English]{babel}             %% language support
\newtheorem{definition}{Definition}
\newtheorem{assumption}{Assumption}

% For footnote in tabular
\usepackage{footnote}
\makesavenoteenv{tabular}
\makesavenoteenv{table}

% Tikz stuff
\usepackage{standalone}
\usepackage{tikz}
\usepackage{pgfplots}                           %% support for TikZ figures
%\usepackage{xcolor}
\usepackage{calc}
%\pgfplotsset{compat=1.16}
\usepgfplotslibrary{groupplots}
\newlength{\figurewidth}
\newlength{\figureheight}

\newlength\blockwidth
\newlength\blockheight
\newlength\blockx
\newlength\blocky
\setlength{\blockwidth}{6em}
\setlength{\blockheight}{4em}
\setlength{\blockx}{6.6em}
\setlength{\blocky}{7em}

% Self defined commands
\newcommand*{\ud}{\mathrm{\,d}}

% Table stuff
\usepackage{booktabs}
\setlength{\heavyrulewidth}{0.1em}
\newcommand{\otoprule}{\midrule[\heavyrulewidth]}

\title{\LARGE \bf
A Method for Scenario Risk Quantification for Automated Driving Systems
}


\author{Erwin de Gelder, Arash Khabbaz Saberi, Hala Elrofai}
%\thanks{*This research has received funding from the European Unions Horizon 2020 research and innovation programme under ROADART Grant Agreement No 636565.}% <-this % stops a space
%\thanks{$^{1}$E. van Nunen is with the Department of Integrated Vehicle Safety, TNO, P.O.
%Box 756, 5700 AT Helmond, The Netherlands, and also with the Department of Mechanical Engineering, Eindhoven University of Technology, 5600 MB Eindhoven, The Netherlands
\newcommand{\todo}[1]{\color{red}TODO: #1\color{black}}

\begin{document}



\maketitle
\thispagestyle{empty}
\pagestyle{empty}


%%%%%%%%%%%%%%%%%%%%%%%%%%%%%%%%%%%%%%%%%%%%%%%%%%%%%%%%%%%%%%%%%%%%%%%%%%%%%%%%
\begin{abstract}
Recent innovations such as automated driving and smart mobility have elevated the safety-criticality of automotive systems due to the impact of these technologies in traffic behavior and safety. New safety validation and assessment methodologies are required to provide the level of assurance that matches the societal impact of these systems.
The objective of this paper is to introduce a novel method for assessment and quantification of the risk of a driving scenario taking into account the operational design domain. 
For our proposed method, we assume that a scenario consists of activities (performed by different actors) and environmental conditions that lead to a potentially hazardous consequence.
The risk of a driving scenario is the product of the probability of the exposure of a scenario and the severity of the hazardous consequence of that scenario.
We introduce a systematic method for calculating the probability of exposure, 
where we assume causal relations between the activities that constitute a scenario. 
By making educated assumptions on the dependencies among the different activities and environmental conditions, 
we simplify the calculation of the probability of the exposure. 
For estimating the severity, we employ Monte Carlo simulations.
We illustrate the use of our proposed method by applying it to an example of a collision avoidance system in a cut-in scenario. 
We use naturalistic driving data acquired from field studies on the Dutch highways to determine the risk. 
The presented example illustrates the potential of our proposed risk estimation method. 
Using our proposed method, we can compare the safety criticality of various scenarios in a quantitative manner, which can be used as a safety metric for evaluating automated driving systems. 
This can lead to a stronger justification for design decisions and test coverage for developing automated vehicle functionalities. 
\end{abstract}


%%%%%%%%%%%%%%%%%%%%%%%%%%%%%%%%%%%%%%%%%%%%%%%%%%%%%%%%%%%%%%%%%%%%%%%%%%%%%%%%

\section{Introduction}
\label{sec:introduction}

\cstartg
% Introduce scenario-based testing.
The development of Automated Vehicles (AVs) has made significant progress in the last years and it is expected that AVs will soon be introduced on our roads \autocite{madni2018autonomous,bimbraw2015autonomous} and become an integral part of intelligent transportation systems \autocite{eskandarian2012introduction,chanedmiston2020itsjpo}. \cendg
\cstarta An essential aspect in the development of AVs is the assessment of quality and performance aspects of the AVs, such as safety, comfort, and efficiency \autocite{bengler2014threedecades, stellet2015taxonomy}. 
Among other methods, a scenario-based approach has been proposed \autocite{elrofai2018scenario, putz2017pegasus}. 
% Explain that these scenarios may be based on real-world scenarios.
For scenario-based assessment, proper specification of scenarios is crucial since they are directly reflected in the test cases used for the assessment \autocite{stellet2015taxonomy}. 
One approach for specifying these test cases is to base them on captured scenarios from real-world data collected on the level of individual vehicles \autocite{elrofai2018scenario, putz2017pegasus, roesener2016scenariobased, deGelder2017assessment}. 

% Mention other literature that tries to extract scenarios.
Different techniques for capturing scenarios and driving maneuvers have been proposed in literature. 
\textcite{kasper2012oobayesnetworks} use object-oriented Bayesian networks for the recognition of 27 type of driving maneuvers. 
\textcite{krajewski2018highD} detect lane changes using lane crossings and \textcite{schlechtriemen2015lanechange} detect lane changes using a naive Bayes classifier and a hidden Markov model. 
%\textcite{paardekooper2019dataset6000km} present an approach for identification of scenarios and include results for scenarios labeled ``braking in front'' and ``cut in''. 
In \autocite{xie2017driving}, random forest classifiers are used for detecting accelerating, braking, and turning with features extracted using principal component analysis, stacked sparse auto-encoders, and statistical features.
In \autocite{cara2015carcyclist}, safety-critical car-cyclist scenarios are extracted from data collected by a vehicle using several machine-learning techniques, among which support vector machines and multiple instance learning.

% Contribution of this paper.
In this paper, we propose a new method for mining scenarios from real-world driving data using automated tagging and searching for combination of tags. 
Our method consists of two steps. 
First, the data is automatically tagged with relevant information. For example, a tag ``lane change'' is added to a vehicle at the time this vehicle is performing a lane change. 
Second, the scenarios are mined based on the aforementioned tags. \cenda
\cstartd To do this, we represent a scenario using a combination of tags and we search for this combination of tags in the tagged data from the previous step. \cendd

% Advantages of our method:
% 1. Tags are pretty basic --> easy.
% 2. Tagging can be very different, depending on the type of data --> scenario mining still the same!
% 3. Accuracy: by not only relying on past data, accuracy is improved.
% 4. Scalable: many more type of scenarios could be extracted.
\cstarta The proposed method brings several benefits. 
First, by tagging the data, characteristics that are shared among different type of scenarios need to be identified only once, whereas these characteristics would be identified multiple times if each type of scenarios would be identified completely independently. \cenda
\cstartf For example, a characteristic could be the presence of a lead vehicle, so if we independently identify two different types of scenarios that consider a lead vehicle, we would identify the lead vehicle two times. \cendf
\cstarta Second, by splitting the process in two parts, i.e., the tagging and the scenario mining, the scenario mining can be applied to different types of data (e.g., data from a vehicle \autocite{paardekooper2019dataset6000km} or a measurement unit above the road \autocite{kovvali2007video,krajewski2018highD}). 
It is also possible to have manually tagged data, e.g., see \autocite{fontana2018action}. 
%Thirdly, because the scenario mining is performed offline, we do not only rely on past data, which, in turn, increases the accuracy of the scenario mining. 
Third, our approach is easily scalable because additional types of scenarios can be mined by  describing them as a combination of (sequential) tags. \cenda
\cstartf Fourth, the approach reveals promising future possibilities, such as the generation of scenarios based on the mined scenarios. \cendf
\cstartg The generated scenarios can be used to define the test cases for the assessment of intelligent vehicles \autocite{elrofai2018scenario, putz2017pegasus, roesener2016scenariobased, deGelder2017assessment, stellet2015taxonomy, zhao2018evaluation}. \cendg

% Structure.
\cstarta In \cref{sec:problem}, we formulate the problem of scenario mining. \Cref{sec:tagging,sec:mining} describe the two steps of our proposed method, i.e., the tagging of the data and the scenario mining based on these tags. 
We illustrate the proposed scenario mining approach with few examples in \cref{sec:case study}. \cenda
\cstartf In \cref{sec:discussion}, we discuss the approach, results, and some possible future improvements. \cendf
We end this paper with conclusions and discuss next steps in \cref{sec:conclusions}. \cenda

%\section{Problem definitions}
\label{sec:problem} % Arash

The need for quantification on risk calculation for automated driving. 
TODO: formalizing this sentence in the context of scenario. 

Differentiating probability of a single (hazardous) event (state of art) to what we are doing: i.e. probability of whole scenario. 

%\section{Nomenclature} % Erwin (use the other paper) & Arash
\label{sec:definitions}

TODO: give more examples
First, we introduce the following notions:
\begin{enumerate}
\item Risk = Severity $\times$ Probability
\item Severity: a quantitative measure, depending on the impact velocity.
\item Condition:
\item Actor: the relevant actors here are the lead vehicle (denoted with L), the following vehicle (F) and the communication unit (V2V)
\item Activity: the relevant actions here are braking, and failure.
\item Scenario:
\end{enumerate}

\section{Proposed Risk Estimation Method} % Hala
\label{sec:method}

In this section we discuss the following:
\begin{enumerate}
	\item High level description of the method 
	\item calculating the probability of occurrence of a scenario based on its Events, Activities, and Conditions. This is in the context of the scenario that gives a specified temporal relation between its events and activities.
	
	We are not sure about the causal relation, but we have observed the temporal relation from data. 
	 
	\item Scenario Events $\rightarrow$ independent variables 
	\item Scenario Activities $\rightarrow$ some are dependent on each other (how to distinguish dependency?)
	\item Scenario Conditions $\rightarrow$ some activities/events may depend on these. (how to define the dependencies?)
	\item We make assumptions about the dependencies, and later validate whether the assumption was justified based on measured data. 	
	\item explaining how assumptions are being made based on data
	\item give both formulas for the two cases of dependent and independent variables. (there may be a third mixed option?)
	\item use the enumerators $i, j, k$ to make the formulas independent on the number of variables (activities, events, conditions)
\end{enumerate}

Note: We use the formula of independent variable in the case study. 





\begin{enumerate}
\item{Compare the risk of the platooning driving behavior ($R_p$)to the risk of human drivers ($R_h$) with respect to rear-end-collisions only. If $$R_p < \frac{1}{10} R_h, $$ the risk is assumed to be acceptable.}
\item{For calculating the risk of the autonomous driving behavior with respect to rear-end-collisions only, we list the relevant events as follows:
\begin{enumerate}
\item{E1: V2V Failure, for $t\in[t_{E1}, t_{E1}+\Delta t_{E1}]$}
\item{E2: Lead performs an emergency brake with $u_{L}(t)\leq u_1$ , for $t\in[t_{E2}, t_{E2}+\Delta t_{E2}]$}
\item{E3: The CACC controller of the follower vehicle reacts to the emergency brake (by braking) at $t\in[t_{E3}, t_{E3}+\Delta t_{E3}]$.}
\item{E4: Collision occurs with an impact speed higher than 10~kph.}
\end{enumerate}
}
\item{We calculate the total probability of these events to lead to a collision as:
\begin{equation}
P_{tot} = P(E1) \times P(E2|E1) \times P(E3|E2) \times P(E4|E3)
\end{equation}
Note that potentially the V2V failure will occur later than the brake action of the lead, (in which case $t_{E2}<t_{E1}$). However, since these events are independent, this does not have any consequences for the proposed approach. Also, due to its independency $ P(E2|E1)=P(E2)$.
Now, we assume that the controller will respond to it (although it may respond late, it will respond eventually), so $P(E3|E2)=1$. Further, let us assume that we can identify a convex set $\mathcal{S}$ for which all states lead to collision with an impact speed higher than 10~kph, than $P(E4|E3)=1$.  Now only the probability of states in this set needs to be calculated.
This collision set $\mathcal{S}$ includes states related to timings as well as initial conditions. We define state $x$ as
\begin{eqnarray*}
x&=&\{(t_{E1},\Delta t_{E1},u_{1},t_{E2},\Delta t_{E2},\\
&&d_i(t_{E2}),v_i(t_{E2}),a_i(t_{E2}), v_{i-1}(t_{E2}),a_{i-1}(t_{E2})) \}
\end{eqnarray*}
and $\mathcal{S} = \{x \in \mathbb{R}^9|x_L < x < x_U\}$ with $x_L \in \mathbb{R}^9$ the lower bound and $x_U \in \mathbb{R}^9$ the upper bound. }
\item{Now the risk can be calculated as $P_{tot} \times $...TODO? (ARASH)}
\end{enumerate}

\section{Case Study} % Erwin & Hala
\label{sec:example}

In this section, we present a case study to illustrate the method of quantifying the risk for a type of scenario, i.e., a scenario class \cite{elrofai2018scenario}. We will first explain the scenario class and the use case. The actual system for which the risk is computed is presented in \cref{sec:system}. Next, we will describe the conditions and activities in \cref{sec:conditions,sec:activities}, respectively. Finally, we will present the results in \cref{sec:results}.

\subsection{The scenario class and its use case}
\label{sec:scenario class}

We want to quantify the risk for scenarios that are linguistically described as follows: while the ego vehicle drives at a moderate to high speed while staying in its lane, another vehicle cuts into the lane of the ego vehicle, such that this vehicle becomes the ego vehicle's lead vehicle. Furthermore, the ego vehicle needs to brake to prevent a collision.

For the quantification of the risk, 60 hours of data (see also \cite{deGelder2017assessment}) are collected by driving a specific route in and between Eindhoven and Helmond, The Netherlands, with twenty different drivers, each driving the route twice. Therefore, it is assumed that the use case of the automation system is simply driving this route. We will use the data for the estimation of the risk. Hence, we will make use of the following assumption:
\begin{assumption}
	The recorded naturalistic driving data is representative for what a vehicle with an automation system might encounter along the same route.
\end{assumption}

\subsection{System-under-test}
\label{sec:system}

To reduce efforts for the assessment, often simulations are employed. However, even simulations can consume considerable time, as these simulations might run real-time \cite{shah2018airsim} or slower when a higher level of detail is used \cite{zofka2016testing}. For our method, we will simplify the simulations, such that the total required time on a common computer is in the order of minutes. Since we are interested in approximate results, a high level of detail is not required. 

To simplify the system-under-test, it is assumed that the system's desired acceleration is similar to the adaptive cruise control defined in \cite{deGelder2017assessment}, i.e.,
\begin{equation}
	\label{eq:desired acceleration} 
	u(t) = k_{\mathrm{d}}(v(t))(d(t) - \tau_{\mathrm{h}} v(t) - s_0) + k_{\mathrm{v}}\left(\dot{d}(t) - ha(t) \right),
\end{equation}
with
\begin{equation}
	\label{eq:gain}
	k_{\mathrm{d}}(v(t)) = k_{\mathrm{d1}} + \left( k_{\mathrm{d2}} - k_{\mathrm{d1}} \right) \exp \left\{ -\frac{v(t)^2}{2\sigma_{\mathrm{d}}} \right\}.
\end{equation}
Here, $v$ is the speed of the ego vehicle, $d$ denotes the clearance between the ego vehicle and its predecessor, i.e., the vehicle that performs the cut-in. The relative speed is denoted by $\dot{d}$ and $a$ refers to the acceleration of the ego vehicle. The ego vehicle is modeled using a first order model with a time delay, i.e.,
\begin{equation}
	\label{eq:vehicle model}
	\tau \dot{a}(t) + a(t) = u(t - \theta).
\end{equation}
Furthermore, the deceleration is limited at \unit[-6]{ms^{-2}}. A description of the constants of \cref{eq:desired acceleration,eq:gain,eq:vehicle model} are listed in \cref{tab:constants}.

\begin{table}
	\centering
	\caption{The constants used for the simple automation system of \cref{eq:desired acceleration,eq:gain,eq:vehicle model}.}
	\label{tab:constants}
	\begin{tabular}{clc}
		\toprule
		Parameter & Description & Value \\ \otoprule
		$\tau_{\mathrm{h}}$ & Desired time headway & \unit[1.0]{s} and \unit[2.0]{s} \\
		$s_0$ & Safety distance & \unit[1.5]{m} \\
		$k_{\mathrm{d1}}$ & Distance gain at high speed & $\unit[0.7]{s^{-2}}$ \\
		$k_{\mathrm{d1}}$ & Distance gain at low speed & $\unit[2.0]{s^{-2}}$ \\
		$\sigma_{\mathrm{d}}$ & Shaping coefficient of distance gain & $\unit[5]{ms^{-1}}$ \\
		$k_{\mathrm{v}}$ & Speed difference gain & $\unit[0.35]{s^{-1}}$ \\
		$\tau$ & Time constant of vehicle model & \unit[0.1]{s} \\
		$\theta$ & Delay of the vehicle response & \unit[0.2]{s} \\
		\bottomrule
	\end{tabular}
\end{table}

\subsection{Conditions}
\label{sec:conditions}

All scenarios are subject to the following conditions:
\begin{itemize}
	\item $C_1$: The speed of the ego vehicle is within the range of \unit[60]{km/h} and \unit[130]{km/h}.
	\item $C_2$: There are no restrictions on the weather conditions.
	\item $C_3$: There are no restrictions on the lighting conditions.
\end{itemize}

Obviously, because there are no restrictions to the weather and lighting conditions, we have $P(C_2,C_3)=1$. For the first condition, we can use the data to estimate the likelihood. The data, however, has been recorded during sunny weather at daylight. Therefore, we need to following assumption.

\begin{assumption} \label{asm:conditions}
	Let $C_2'$ and $C_3'$ denote the conditions of having sunny weather and daylight, respectively. Then we have $P(C_1|C_2,C_3)=P(C_1|C_2',C_3')$.
\end{assumption}

From the data, it appeared that $P(C_1|C_2',C_3')=0.20$. Using \cref{asm:conditions}, we have
\begin{dmath}
	P(C) = P(C_1,C_2,C_3)=P(C_1|C_2',C_3')\cdot P(C_2,C_3)=0.20.
\end{dmath}

\subsection{Activities}
\label{sec:activities}

\subsection{Results}
\label{sec:results}


\section{Discussion and future outlook} % Hala & Erwin & Arash
\label{sec:discussion}

In the calculation of the estimated risk, many assumptions are made to simplify the calculation or because there are unknowns due to lack of data. This reduces the accuracy of the estimated risk. 
However, we still believe that the quantified risk is valuable because of the following reasons:
\begin{itemize}
	\item All the assumptions that were made for estimating the risk are explicit. In contrast, when people assign ASIL levels to a hazardous event, often many assumptions are implicitly made. By making the assumptions explicit, it is much clearer why a certain risk is associated with --- in this case --- a certain/specific scenario.
	\item It provides a good estimate of the order of the risk for each scenario. For example, if the estimated risk of two different scenarios differ by a factor 10, it is still reasonable to argue that one scenario introduces a higher risk than the other, even though the factor 10 might not be exact.
	\item Because our/the proposed method explicates all the steps and assumptions that lead to a certain estimated risk, it is easily possible to update the risk when more information of the system is known or when more data is available.
\end{itemize}

To be discussed:
\begin{itemize}
	\item Method gives only order of risk.
	\item ``Controllability'' not considered.
	\item A lot of assumptions: with this method, these assumptions are made explicit, whereas often people make these assumptions implicit (and implicit assumptions are the mother of all fuck-ups; should be rephrased :)).
\end{itemize}
\section{Conclusions}
\label{sec:conclusions}

	Systematic quantification of the risk provides additional trust in the safety analysis that depends on the availability of data rather than experts judgment. 





\section*{Acknowledgment}

The authors would like to express their gratitudes to Ellen van Nunen for her contribution to this work. 

%%%%%%%%%%%%%%%%%%%%%%%%%%%%%%%%%%%%%%%%%%%%%%%%%%%%%%%%%%%%%%%%%%%%%%%%%%%%%%%%

% When using biber:
\printbibliography

%% When using BibLaTeX:
%\bibliography{bibRiskPaper}
%\bibliographystyle{ieeetr}


\end{document}
