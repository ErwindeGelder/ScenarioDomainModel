%%%%%%%%%%%%%%%%%%%%%%%%%%%%%%%%%%%%%%%%%%%%%%%%%%%%%%%%%%%%%%%%%%%%%%%%%%%%%%%%
%2345678901234567890123456789012345678901234567890123456789012345678901234567890
%        1         2         3         4         5         6         7         8

\documentclass[letterpaper, conference]{ieeeconf}  % Comment this line out if you need a4paper


% make changes take effect
\pagestyle{headings}
% adjust as needed
\addtolength{\footskip}{0\baselineskip}
\addtolength{\textheight}{-1\baselineskip}




%\documentclass[a4paper, 10pt, conference]{ieeeconf}      % Use this line for a4 paper

\IEEEoverridecommandlockouts                              % This command is only needed if 
                                                          % you want to use the \thanks command

% See the \addtolength command later in the file to balance the column lengths
% on the last page of the document

% The following packages can be found on http:\\www.ctan.org
\usepackage{graphicx} % for pdf, bitmapped graphics files
\usepackage{epstopdf}
%\usepackage{epsfig} % for postscript graphics files
%\usepackage{mathptmx} % assumes new font selection scheme installed
%\usepackage{times} % assumes new font selection scheme installed
\let\proof\relax
\let\endproof\relax
\usepackage{amsmath} % assumes amsmath package installed
\usepackage{amsthm}  % For special theorem style
\usepackage{amsfonts}
%\usepackage{amssymb}  % assumes amsmath package installed
\usepackage[dvipsnames]{xcolor}
\usepackage{tikz}
\usepackage{pgfplots}
\usetikzlibrary{shapes,arrows}
\usetikzlibrary{backgrounds}
\usepackage{multirow}
\usepackage[keeplastbox]{flushend}
%\usepackage{cite}  % Make sure that citation appear as [1]-[3] instead of [1], [2], [3]
\usepackage[utf8]{inputenc}    % utf8 support
\usepackage[T1]{fontenc}       % code for pdf file
\usepackage[USenglish]{babel}             %% language support
\usepackage{silence}  							%% For filtering warnings
\usepackage{csquotes}
% Add doi=false if no DOI
\usepackage[style=ieee,isbn=false,date=year,backend=biber,maxbibnames=15,maxcitenames=2,mincitenames=2,uniquelist=false,uniquename=false,giveninits=true]{biblatex}
% Filter warnings issued by package biblatex starting with "Patching footnotes failed"
\WarningFilter{biblatex}{Patching footnotes failed}
\renewcommand*{\bibfont}{\footnotesize}		%% Use this for papers
\setlength{\biblabelsep}{\labelsep}
\bibliography{references}



\usepackage{pgfplots}
\pgfplotsset{compat=1.9}  % Prevent warning, pgf running in backwards compatibility mode anyway
%\usetikzlibrary{external}                       %% Create pdf figures from TikZ. Use PDFTeXify ...
%\tikzexternalize[prefix=tikz/]                  %% ... with --tex-option=--shell-escape switch.
\usepackage[capitalize]{cleveref}
\Crefname{figure}{Fig.}{Figs.}
\crefname{equation}{}{}
\Crefname{equation}{Equation}{Equations}
\usepackage{subcaption}
\usepackage{xstring}
\usepackage{xparse}
\usepackage{siunitx}

\theoremstyle{plain}
\newtheorem{definition}{Definition}
\theoremstyle{remark}\newtheorem{remarkenv}{Remark}        %% remarks
\newenvironment{remark}{\begin{remarkenv}}%
	{\hfill$\lozenge$\end{remarkenv}}            %% end remark with a lozenge

% Table stuff
\usepackage{booktabs}
\usepackage{tabularx}
\newcolumntype{Y}{>{\raggedright\arraybackslash}X}
\setlength{\heavyrulewidth}{0.1em}
\newcommand{\otoprule}{\midrule[\heavyrulewidth]}

%\pgfplotsset{compat=newest} 
%\pgfplotsset{plot coordinates/math parser=false}

%Images path 
\graphicspath{ {figures/} }


\title{\LARGE \bf
	Real-world scenario mining for the assessment of automated vehicles
}

\author{Erwin de Gelder$^{1,2*}$, Jeroen Manders$^{1}$, Corrado Grappiolo$^{1}$, Jan-Pieter Paardekooper$^{1,3}$, \\ Olaf Op den Camp$^{1}$, Bart De Schutter$^{2}$%
\thanks{$^{1}$TNO, P.O. Box 756, 5700 AT Helmond, The Netherlands}%
\thanks{$^{2}$Delft University of Technology, Delft Center for Systems and Control, Delft, The Netherlands}%
\thanks{$^{3}$Radboud University, Donders Institute for Brain, Cognition and Behaviour, Nijmegen, The Netherlands}%
\thanks{$^{*}$Corresponding author. \newline E-mail address: {\tt\small erwin.degelder@tno.nl}}%
%\thanks{*This work was not supported by any organization}% <-this % stops a space
%\thanks{$^{1}$Albert Author is with Faculty of Electrical Engineering, Mathematics and Computer Science,
%        University of Twente, 7500 AE Enschede, The Netherlands
%        {\tt\small albert.author@papercept.net}}%
%\thanks{$^{2}$Bernard D. Researcheris with the Department of Electrical Engineering, Wright State University,
%        Dayton, OH 45435, USA
%        {\tt\small b.d.researcher@ieee.org}}%
}

%\definecolor{TNOlightgray}{RGB}{222,222,231}
%\tikzstyle{tag}=[text height=.8em, text depth=.1em, font=\small\sffamily, rounded corners=0.2em, fill=TNOlightgray, node distance=9em, text width=7em, align=center]
%\tikzstyle{tag wide}=[tag, text width=12em]
%\tikzstyle{tag next to wide}=[tag, node distance=11.5em]
%\tikzstyle{tagemphasize}=[]
%\tikzstyle{tagarrow}=[->, line width=0.75mm, color=TNOlightgray]
%\ExplSyntaxOn
%\NewDocumentCommand{\tags}{O{,}m}{
%	\seq_set_split:Nnn \arg { #1 } { #2 }
%	\seq_pop_left:NN \arg \argl
%	\begin{tikzpicture}
%	\showtag{tag~wide}{tag\arabic{tmpcounter}}{\argl};
%	
%	\ifnum \seq_count:N \arg > 0
%	\seq_pop_left:NN \arg \argl
%	\setcounter{tmpcounter2}{\value{tmpcounter}}
%	\stepcounter{tmpcounter}
%	\showtag{tag~next~to~wide, right~of=tag\arabic{tmpcounter2}}{tag\arabic{tmpcounter}}{\argl};
%	\draw[tagarrow] (tag\arabic{tmpcounter2}) -- (tag\arabic{tmpcounter});
%	
%	\seq_map_inline:Nn \arg {
%		\setcounter{tmpcounter2}{\value{tmpcounter}}
%		\stepcounter{tmpcounter}
%		\showtag{tag, right~of=tag\arabic{tmpcounter2}}{tag\arabic{tmpcounter}}{##1};
%		\draw[tagarrow] (tag\arabic{tmpcounter2}) -- (tag\arabic{tmpcounter});
%	}
%	\fi
%	\end{tikzpicture}
%}
%\ExplSyntaxOff



\newlength\figurewidth
\newlength\figureheight


\usetikzlibrary{arrows,positioning}
\usetikzlibrary{arrows.meta}

% Notations
\newcommand{\accelerationsymbol}{a}
\newcommand{\accelerationstart}{\accelerationsymbol_\mathrm{start}}
\newcommand{\accelerationcruise}{\accelerationsymbol_\mathrm{cruise}}
\newcommand{\defsym}{\equiv}
\newcommand{\indextarget}{i}
\newcommand{\lineleft}[1]{l\left(#1\right)}
\newcommand{\lineright}[1]{r\left(#1\right)}
\newcommand{\nan}{\mathrm{NaN}}
\newcommand{\sample}{k}
\newcommand{\sampledummy}{\tau}
\newcommand{\sampleenddec}[1]{\sample_{\mathrm{end}}^{-}\left(#1\right)}
\newcommand{\sampleendinc}[1]{\sample_{\mathrm{end}}^{+}\left(#1\right)}
\newcommand{\samplehorizon}{\sample_{\mathrm{h}}}
\newcommand{\sampleinit}{\sample_{0}}
\newcommand{\sampleaccevent}{\sample^{+}}
\newcommand{\sampledecevent}{\sample^{-}}
\newcommand{\speedsymbol}{v}
\newcommand{\speed}[1]{\speedsymbol\left(#1\right)}
\newcommand{\speedmax}[1]{\speedsymbol_{\mathrm{max}}\left(#1\right)}
\newcommand{\speedmin}[1]{\speedsymbol_{\mathrm{min}}\left(#1\right)}
\newcommand{\speedinc}[1]{\speedsymbol^{+}\left(#1\right)}
\newcommand{\speeddec}[1]{\speedsymbol^{-}\left(#1\right)}
\newcommand{\speeddiff}{\Delta_{\speedsymbol}}
\newcommand{\speedtargetiabs}[2]{\speedsymbol_{#2}^{\mathrm{abs}}\left(#1\right)}
\newcommand{\speedtargetirel}[2]{\speedsymbol_{#2}^{\mathrm{rel}}\left(#1\right)}
\renewcommand{\time}{t}
\newcommand{\sampletime}{\time_\mathrm{s}}
\newcommand{\timecruising}{\time_\mathrm{cruise}}


\newcommand{\todo}[1]{\color{red}TO DO: #1 \color{black}}

\usepackage{setspace}
\doublespacing


\begin{document}


\maketitle
\thispagestyle{empty}
\pagestyle{empty}


%%%%%%%%%%%%%%%%%%%%%%%%%%%%%%%%%%%%%%%%%%%%%%%%%%%%%%%%%%%%%%%%%%%%%%%%%%%%%%%%
\begin{abstract}
	\cstarte Scenario-based methods for the assessment of Automated Vehicles (AVs) are widely supported by many players in the automotive field.
	Scenarios captured from real-world data can be used to define the scenarios for the assessment and to estimate the relevance of the scenarios for the assessment.
	Therefore, different techniques are proposed for capturing scenarios from real-world data.
	In this paper, we propose a new method to capture scenarios from real-world data using a two-step approach.
	The first step is to automatically provide the data with tags.
	Next, by representing a scenario using a combination of tags, we are able to mine the scenarios based on the provided tags.
	One of the benefits of our approach is that the tags can be used to identify characteristics of a scenario that are shared among different type of scenarios. 
	In this way, these characteristics need to be identified only once, whereas these characteristics would be identified multiple times if each type of scenario would be identified completely independently.
	Furthermore, the method is not specific for one type of scenario and, therefore, it can be applied to a large variety of scenarios.
	We provide two examples to illustrate the method.
	This paper is concluded with some promising future possibilities for our approach, such as the generation of scenarios for the assessment of automated vehicles.
	\cende
\end{abstract}

%%%%%%%%%%%%%%%%%%%%%%%%%%%%%%%%%%%%%%%%%%%%%%%%%%%%%%%%%%%%%%%%%%%%%%%%%%%%%%%%
\section{Introduction}
\label{sec:introduction}

\cstartg
% Introduce scenario-based testing.
The development of Automated Vehicles (AVs) has made significant progress in the last years and it is expected that AVs will soon be introduced on our roads \autocite{madni2018autonomous,bimbraw2015autonomous} and become an integral part of intelligent transportation systems \autocite{eskandarian2012introduction,chanedmiston2020itsjpo}. \cendg
\cstarta An essential aspect in the development of AVs is the assessment of quality and performance aspects of the AVs, such as safety, comfort, and efficiency \autocite{bengler2014threedecades, stellet2015taxonomy}. 
Among other methods, a scenario-based approach has been proposed \autocite{elrofai2018scenario, putz2017pegasus}. 
% Explain that these scenarios may be based on real-world scenarios.
For scenario-based assessment, proper specification of scenarios is crucial since they are directly reflected in the test cases used for the assessment \autocite{stellet2015taxonomy}. 
One approach for specifying these test cases is to base them on captured scenarios from real-world data collected on the level of individual vehicles \autocite{elrofai2018scenario, putz2017pegasus, roesener2016scenariobased, deGelder2017assessment}. 

% Mention other literature that tries to extract scenarios.
Different techniques for capturing scenarios and driving maneuvers have been proposed in literature. 
\textcite{kasper2012oobayesnetworks} use object-oriented Bayesian networks for the recognition of 27 type of driving maneuvers. 
\textcite{krajewski2018highD} detect lane changes using lane crossings and \textcite{schlechtriemen2015lanechange} detect lane changes using a naive Bayes classifier and a hidden Markov model. 
%\textcite{paardekooper2019dataset6000km} present an approach for identification of scenarios and include results for scenarios labeled ``braking in front'' and ``cut in''. 
In \autocite{xie2017driving}, random forest classifiers are used for detecting accelerating, braking, and turning with features extracted using principal component analysis, stacked sparse auto-encoders, and statistical features.
In \autocite{cara2015carcyclist}, safety-critical car-cyclist scenarios are extracted from data collected by a vehicle using several machine-learning techniques, among which support vector machines and multiple instance learning.

% Contribution of this paper.
In this paper, we propose a new method for mining scenarios from real-world driving data using automated tagging and searching for combination of tags. 
Our method consists of two steps. 
First, the data is automatically tagged with relevant information. For example, a tag ``lane change'' is added to a vehicle at the time this vehicle is performing a lane change. 
Second, the scenarios are mined based on the aforementioned tags. \cenda
\cstartd To do this, we represent a scenario using a combination of tags and we search for this combination of tags in the tagged data from the previous step. \cendd

% Advantages of our method:
% 1. Tags are pretty basic --> easy.
% 2. Tagging can be very different, depending on the type of data --> scenario mining still the same!
% 3. Accuracy: by not only relying on past data, accuracy is improved.
% 4. Scalable: many more type of scenarios could be extracted.
\cstarta The proposed method brings several benefits. 
First, by tagging the data, characteristics that are shared among different type of scenarios need to be identified only once, whereas these characteristics would be identified multiple times if each type of scenarios would be identified completely independently. \cenda
\cstartf For example, a characteristic could be the presence of a lead vehicle, so if we independently identify two different types of scenarios that consider a lead vehicle, we would identify the lead vehicle two times. \cendf
\cstarta Second, by splitting the process in two parts, i.e., the tagging and the scenario mining, the scenario mining can be applied to different types of data (e.g., data from a vehicle \autocite{paardekooper2019dataset6000km} or a measurement unit above the road \autocite{kovvali2007video,krajewski2018highD}). 
It is also possible to have manually tagged data, e.g., see \autocite{fontana2018action}. 
%Thirdly, because the scenario mining is performed offline, we do not only rely on past data, which, in turn, increases the accuracy of the scenario mining. 
Third, our approach is easily scalable because additional types of scenarios can be mined by  describing them as a combination of (sequential) tags. \cenda
\cstartf Fourth, the approach reveals promising future possibilities, such as the generation of scenarios based on the mined scenarios. \cendf
\cstartg The generated scenarios can be used to define the test cases for the assessment of intelligent vehicles \autocite{elrofai2018scenario, putz2017pegasus, roesener2016scenariobased, deGelder2017assessment, stellet2015taxonomy, zhao2018evaluation}. \cendg

% Structure.
\cstarta In \cref{sec:problem}, we formulate the problem of scenario mining. \Cref{sec:tagging,sec:mining} describe the two steps of our proposed method, i.e., the tagging of the data and the scenario mining based on these tags. 
We illustrate the proposed scenario mining approach with few examples in \cref{sec:case study}. \cenda
\cstartf In \cref{sec:discussion}, we discuss the approach, results, and some possible future improvements. \cendf
We end this paper with conclusions and discuss next steps in \cref{sec:conclusions}. \cenda

\section{Problem formulation}
\label{sec:problem}

To formulate the risk quantification problem, we distinguish quantitative scenarios from qualitative scenarios, using the definitions of \emph{scenario} and \emph{scenario category} of \autocite{degelder2018ontology}:

\begin{definition}[Scenario]
	\label{def:scenario}
	A scenario is a quantitative description of the relevant characteristics of the ego vehicle, its activities and/or goals, its static environment, and its dynamic environment. In addition, a scenario contains all events that are relevant to the ego vehicle.
\end{definition}

\begin{definition}[Scenario category]
	\label{def:scenario category}
	A scenario category is a qualitative description of the ego vehicle, its activities and/or goals, its static environment, and its dynamic environment.
\end{definition}

A scenario category is an abstraction of a scenario and, therefore, a scenario category comprises multiple scenarios \autocite{degelder2018ontology}.
For example, the scenario category ``cut-in'' comprises all possible cut-in scenarios. 
Given such a scenario category, our goal is to find all corresponding scenarios in a given data set. 

The main objective of this paper is to quantify the risk of a system-under-test when operating in all possible scenarios comprised by a given scenario category. The focus is on the system-under-test itself, its occupants, and its direct surroundings. Hence, we define the risk quantification problem as follows:

\begin{problem}[Risk quantification]
	\label{problem:risk quantification}
	Given a scenario category, how to quantify the risk on vehicle level?
\end{problem}

Risk is typically a combination of the likelihood and the severity of any harm. In our case, the likelihood of any harm can be decomposed in two components: the likelihood of a scenario of the given scenario category and the likelihood of any harm if such a scenario occurs. Based on this observation, we can divide \cref{problem:risk quantification} into three sub-problems:

\begin{problem}[Exposure]
	\label{problem:exposure}
	Given a scenario category, how to find all scenarios that correspond to this scenario category in a given data set?
\end{problem}

\begin{problem}[Harm likelihood]
	\label{problem:controllability}
	Given a scenario category and a system-under-test, how to estimate the likelihood of any harm?
\end{problem}

\begin{problem}[Severity]
	\label{problem:severity}
	Given a scenario category, a system-under-test, and a possible harmful outcome, how to estimate the severity of that harm?
\end{problem}

\section{Proposed Risk Estimation Method} % Hala
\label{sec:method}

In this section we discuss the following:
\begin{enumerate}
	\item High level description of the method 
	\item calculating the probability of occurrence of a scenario based on its Events, Activities, and Conditions. This is in the context of the scenario that gives a specified temporal relation between its events and activities.
	
	We are not sure about the causal relation, but we have observed the temporal relation from data. 
	 
	\item Scenario Events $\rightarrow$ independent variables 
	\item Scenario Activities $\rightarrow$ some are dependent on each other (how to distinguish dependency?)
	\item Scenario Conditions $\rightarrow$ some activities/events may depend on these. (how to define the dependencies?)
	\item We make assumptions about the dependencies, and later validate whether the assumption was justified based on measured data. 	
	\item explaining how assumptions are being made based on data
	\item give both formulas for the two cases of dependent and independent variables. (there may be a third mixed option?)
	\item use the enumerators $i, j, k$ to make the formulas independent on the number of variables (activities, events, conditions)
\end{enumerate}

Note: We use the formula of independent variable in the case study. 





\begin{enumerate}
\item{Compare the risk of the platooning driving behavior ($R_p$)to the risk of human drivers ($R_h$) with respect to rear-end-collisions only. If $$R_p < \frac{1}{10} R_h, $$ the risk is assumed to be acceptable.}
\item{For calculating the risk of the autonomous driving behavior with respect to rear-end-collisions only, we list the relevant events as follows:
\begin{enumerate}
\item{E1: V2V Failure, for $t\in[t_{E1}, t_{E1}+\Delta t_{E1}]$}
\item{E2: Lead performs an emergency brake with $u_{L}(t)\leq u_1$ , for $t\in[t_{E2}, t_{E2}+\Delta t_{E2}]$}
\item{E3: The CACC controller of the follower vehicle reacts to the emergency brake (by braking) at $t\in[t_{E3}, t_{E3}+\Delta t_{E3}]$.}
\item{E4: Collision occurs with an impact speed higher than 10~kph.}
\end{enumerate}
}
\item{We calculate the total probability of these events to lead to a collision as:
\begin{equation}
P_{tot} = P(E1) \times P(E2|E1) \times P(E3|E2) \times P(E4|E3)
\end{equation}
Note that potentially the V2V failure will occur later than the brake action of the lead, (in which case $t_{E2}<t_{E1}$). However, since these events are independent, this does not have any consequences for the proposed approach. Also, due to its independency $ P(E2|E1)=P(E2)$.
Now, we assume that the controller will respond to it (although it may respond late, it will respond eventually), so $P(E3|E2)=1$. Further, let us assume that we can identify a convex set $\mathcal{S}$ for which all states lead to collision with an impact speed higher than 10~kph, than $P(E4|E3)=1$.  Now only the probability of states in this set needs to be calculated.
This collision set $\mathcal{S}$ includes states related to timings as well as initial conditions. We define state $x$ as
\begin{eqnarray*}
x&=&\{(t_{E1},\Delta t_{E1},u_{1},t_{E2},\Delta t_{E2},\\
&&d_i(t_{E2}),v_i(t_{E2}),a_i(t_{E2}), v_{i-1}(t_{E2}),a_{i-1}(t_{E2})) \}
\end{eqnarray*}
and $\mathcal{S} = \{x \in \mathbb{R}^9|x_L < x < x_U\}$ with $x_L \in \mathbb{R}^9$ the lower bound and $x_U \in \mathbb{R}^9$ the upper bound. }
\item{Now the risk can be calculated as $P_{tot} \times $...TODO? (ARASH)}
\end{enumerate}

\section{Case study}
\label{sec:case study}

Here we illustrate the method by applying the method to the dataset described in \autocite{paardekooper2019dataset6000km}.

\begin{table}
	\centering
	\caption{\cstartc Values of parameters used in the case study. \cendc}
	\label{tab:parameters}
	\cstartc
	\begin{tabularx}{\linewidth}{lXl}
		\toprule
		Parameter & Description & Value \\ \otoprule
		$\sampletime$ & Sample time & \SI{0.01}{\second} \\
		$\samplehorizon$ & Sample window & 100 \\
		$\accelerationstart$ & Threshold determining the start of an acceleration or deceleration activity & \SI{0.1}{\meter\per\second\squared} \\
		$\accelerationcruise$ & Threshold determining the end of an acceleration or deceleration activity & \SI{0.1}{\meter\per\second\squared} \\
		$\speeddiff$ & Minimum speed increase/decrease for an acceleration/deceleration activity & \SI{1}{\meter\per\second} \\
		$\samplescruising$ & Minimum cruising time & \SI{4}{\second} \\
		$\lanechangethreshold$ & A lane change is detected when the difference between consecutive lane line distances is larger than this threshold & \SI{1}{\meter} \\
		$\lanechangespeed$ & Threshold determining the start and end of a lane change & \SI{0.25}{\meter\per\second} \\
		$\factorgoalmax$ & Maximum factor of the lane width for a lane change of any other vehicle & 0.5 \\
		$\factorgoalmin$ & Minimum factor of the lane width for a lane change of any other vehicle & 0.1 \\
		\bottomrule
	\end{tabularx}
	\cendc
\end{table}

\todo{Write case study setup}

\todo{Obtain results (this will require quite some time!)}

\todo{Write about the results.}

\section{Conclusions}
\label{sec:conclusions}

\cstarte
% Summarize two-step approach: tagging and mining based on tags.
For the scenario-based assessment of automated vehicles, scenarios captured from real-world data collected on the level of individual vehicles can be used to define the tests.
We have proposed a two-step approach for mining real-world scenarios from a data set.
The first step consists in labeling the data with tags that describe, e.g., the lateral and longitudinal activities of the different actors.
The second step mines the scenarios by searching for particular combinations of tags.
We have illustrated the approach with two examples, a cut in and an overtaking before a lane change. \cende
\cstartf These examples demonstrated that the proposed approach is suitable for mining scenarios from real-world data.
% Improve tagging, e.g., using machine learning techniques.
% Have more tags!
Future work includes labeling the data with more tags and exploring the possibilities of using techniques that are used in the field of natural language processing.
\cendf



%\addtolength{\textheight}{-12cm}  % This command serves to balance the column lengths
                                  % on the last page of the document manually. It shortens
                                  % the textheight of the last page by a suitable amount.
                                  % This command does not take effect until the next page
                                  % so it should come on the page before the last. Make
                                  % sure that you do not shorten the textheight too much.

%\bibliographystyle{ieeetran}
%\bibliography{../bib}
\printbibliography

%\appendix
%\appendices
%\section{Details of the domain model}
\label{sec:appendix domain model}

In this section, we provide more details of the domain model, i.e., we describe all different classes and its attributes.



\end{document}
