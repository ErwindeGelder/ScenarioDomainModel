%%%%%%%%%%%%%%%%%%%%%%%%%%%%%%%%%%%%%%%%%%%%%%%%%%%%%%%%%%%%%%%%%%%%%%%%%%%%%%%%
%2345678901234567890123456789012345678901234567890123456789012345678901234567890
%        1         2         3         4         5         6         7         8

\documentclass[letterpaper, conference]{ieeeconf}  % Comment this line out if you need a4paper


% make changes take effect
\pagestyle{headings}
% adjust as needed
\addtolength{\footskip}{0\baselineskip}
\addtolength{\textheight}{-1\baselineskip}




%\documentclass[a4paper, 10pt, conference]{ieeeconf}      % Use this line for a4 paper

\IEEEoverridecommandlockouts                              % This command is only needed if 
                                                          % you want to use the \thanks command

% See the \addtolength command later in the file to balance the column lengths
% on the last page of the document

% The following packages can be found on http:\\www.ctan.org
\usepackage{graphicx} % for pdf, bitmapped graphics files
\usepackage{epstopdf}
%\usepackage{epsfig} % for postscript graphics files
%\usepackage{mathptmx} % assumes new font selection scheme installed
%\usepackage{times} % assumes new font selection scheme installed
\let\proof\relax
\let\endproof\relax
\usepackage{amsmath} % assumes amsmath package installed
\usepackage{amsthm}  % For special theorem style
\usepackage{amsfonts}
%\usepackage{amssymb}  % assumes amsmath package installed
\usepackage[dvipsnames]{xcolor}
\usepackage{tikz}
\usepackage{pgfplots}
\usetikzlibrary{shapes,arrows}
\usetikzlibrary{backgrounds}
\usepackage{multirow}
\usepackage[keeplastbox]{flushend}
%\usepackage{cite}  % Make sure that citation appear as [1]-[3] instead of [1], [2], [3]
\usepackage[utf8]{inputenc}    % utf8 support
\usepackage[T1]{fontenc}       % code for pdf file
\usepackage[USenglish]{babel}             %% language support
\usepackage{silence}  							%% For filtering warnings
\usepackage{csquotes}
% Add doi=false if no DOI
\usepackage[style=ieee,isbn=false,date=year,backend=biber,maxbibnames=15,maxcitenames=2,mincitenames=2,uniquelist=false,uniquename=false,giveninits=true]{biblatex}
% Filter warnings issued by package biblatex starting with "Patching footnotes failed"
\WarningFilter{biblatex}{Patching footnotes failed}
\renewcommand*{\bibfont}{\footnotesize}		%% Use this for papers
\setlength{\biblabelsep}{\labelsep}
\bibliography{references}



\usepackage{pgfplots}
\pgfplotsset{compat=1.9}  % Prevent warning, pgf running in backwards compatibility mode anyway
%\usetikzlibrary{external}                       %% Create pdf figures from TikZ. Use PDFTeXify ...
%\tikzexternalize[prefix=tikz/]                  %% ... with --tex-option=--shell-escape switch.
\usepackage[capitalize]{cleveref}
\Crefname{figure}{Fig.}{Figs.}
\crefname{equation}{}{}
\Crefname{equation}{Equation}{Equations}
\usepackage{subcaption}
\usepackage{xstring}
\usepackage{xparse}
\usepackage{siunitx}

\theoremstyle{plain}
\newtheorem{definition}{Definition}
\theoremstyle{remark}\newtheorem{remarkenv}{Remark}        %% remarks
\newenvironment{remark}{\begin{remarkenv}}%
	{\hfill$\lozenge$\end{remarkenv}}            %% end remark with a lozenge

% Table stuff
\usepackage{booktabs}
\usepackage{tabularx}
\newcolumntype{Y}{>{\raggedright\arraybackslash}X}
\setlength{\heavyrulewidth}{0.1em}
\newcommand{\otoprule}{\midrule[\heavyrulewidth]}

%\pgfplotsset{compat=newest} 
%\pgfplotsset{plot coordinates/math parser=false}

%Images path 
\graphicspath{ {figures/} }


\title{\LARGE \bf
	Real-world scenario mining for the assessment of automated vehicles
}

\author{Erwin de Gelder$^{1,2*}$, Jeroen Manders$^{1}$, Corrado Grappiolo$^{1}$, Jan-Pieter Paardekooper$^{1,3}$, \\ Olaf Op den Camp$^{1}$, Bart De Schutter$^{2}$%
\thanks{$^{1}$TNO, P.O. Box 756, 5700 AT Helmond, The Netherlands}%
\thanks{$^{2}$Delft University of Technology, Delft Center for Systems and Control, Delft, The Netherlands}%
\thanks{$^{3}$Radboud University, Donders Institute for Brain, Cognition and Behaviour, Nijmegen, The Netherlands}%
\thanks{$^{*}$Corresponding author. \newline E-mail address: {\tt\small erwin.degelder@tno.nl}}%
%\thanks{*This work was not supported by any organization}% <-this % stops a space
%\thanks{$^{1}$Albert Author is with Faculty of Electrical Engineering, Mathematics and Computer Science,
%        University of Twente, 7500 AE Enschede, The Netherlands
%        {\tt\small albert.author@papercept.net}}%
%\thanks{$^{2}$Bernard D. Researcheris with the Department of Electrical Engineering, Wright State University,
%        Dayton, OH 45435, USA
%        {\tt\small b.d.researcher@ieee.org}}%
}


\newlength\figurewidth
\newlength\figureheight


\usetikzlibrary{arrows,positioning}
\usetikzlibrary{arrows.meta}

% Notations
\newcommand{\accelerationsymbol}{a}
\newcommand{\accelerationstart}{\accelerationsymbol_\mathrm{start}}
\newcommand{\accelerationcruise}{\accelerationsymbol_\mathrm{cruise}}
\newcommand{\defsym}{\equiv}
\newcommand{\indextarget}{i}
\newcommand{\lineleft}[1]{l\left(#1\right)}
\newcommand{\lineright}[1]{r\left(#1\right)}
\newcommand{\nan}{\mathrm{NaN}}
\newcommand{\sample}{k}
\newcommand{\sampledummy}{\tau}
\newcommand{\sampleenddec}[1]{\sample_{\mathrm{end}}^{-}\left(#1\right)}
\newcommand{\sampleendinc}[1]{\sample_{\mathrm{end}}^{+}\left(#1\right)}
\newcommand{\samplehorizon}{\sample_{\mathrm{h}}}
\newcommand{\sampleinit}{\sample_{0}}
\newcommand{\sampleaccevent}{\sample^{+}}
\newcommand{\sampledecevent}{\sample^{-}}
\newcommand{\speedsymbol}{v}
\newcommand{\speed}[1]{\speedsymbol\left(#1\right)}
\newcommand{\speedmax}[1]{\speedsymbol_{\mathrm{max}}\left(#1\right)}
\newcommand{\speedmin}[1]{\speedsymbol_{\mathrm{min}}\left(#1\right)}
\newcommand{\speedinc}[1]{\speedsymbol^{+}\left(#1\right)}
\newcommand{\speeddec}[1]{\speedsymbol^{-}\left(#1\right)}
\newcommand{\speeddiff}{\Delta_{\speedsymbol}}
\newcommand{\speedtargetiabs}[2]{\speedsymbol_{#2}^{\mathrm{abs}}\left(#1\right)}
\newcommand{\speedtargetirel}[2]{\speedsymbol_{#2}^{\mathrm{rel}}\left(#1\right)}
\renewcommand{\time}{t}
\newcommand{\sampletime}{\time_\mathrm{s}}
\newcommand{\timecruising}{\time_\mathrm{cruise}}


\newcommand{\todo}[1]{\color{red}TO DO: #1 \color{black}}

\usepackage{setspace}
\doublespacing


\begin{document}


\maketitle
\thispagestyle{empty}
\pagestyle{empty}


%%%%%%%%%%%%%%%%%%%%%%%%%%%%%%%%%%%%%%%%%%%%%%%%%%%%%%%%%%%%%%%%%%%%%%%%%%%%%%%%
\begin{abstract}
	\todo{Write abstract}
\end{abstract}

%%%%%%%%%%%%%%%%%%%%%%%%%%%%%%%%%%%%%%%%%%%%%%%%%%%%%%%%%%%%%%%%%%%%%%%%%%%%%%%%
\section{Introduction}
\label{sec:introduction}

TODO

\color{red}
For Arash and Hala: please use \\{\tt \textbackslash cref\{sec:introduction\}}\\ to refer to sections, figures etc. It automatically adds things as `Section': e.g., \cref{sec:introduction}.
\color{black}

The introduction contains: 
\begin{itemize}
	\item Setting up the context of automated driving
	\item Giving some background information about the hazard analysis and risk assessment in the ISO~26262 standard
	\item The gap (the problem) We currently have this as a separate chapter, we could move it to introduction depending on the length. 
	\item our contribution. 
	\item paper structure
\end{itemize}
\section{Problem formulation}
\label{sec:problem}

To formulate the scenario mining problem, we distinguish quantitative scenarios from qualitative scenarios, using the definitions of \emph{scenario} and \emph{scenario category} of \autocite{degelder2018ontology}:

\begin{definition}[Scenario]
	\label{def:scenario}
	A scenario is a quantitative description of the relevant characteristics of the ego vehicle, its activities and/or goals, its static environment, and its dynamic environment. In addition, a scenario contains all events that are relevant to the ego vehicle.
\end{definition}

\begin{definition}[Scenario category \autocite{degelder2018ontology}]
	\label{def:scenario category}
	A scenario category is a qualitative description of the ego vehicle, its activities and/or goals, its static environment, and its dynamic environment.
\end{definition}

\cstarta
A scenario category is an abstraction of a scenario and, therefore, a scenario category comprises multiple scenarios \autocite{degelder2018ontology}.
For example, the scenario category ``cut in'' comprises all possible cut-in scenarios. 
Given such a scenario category, our goal is to find all corresponding scenarios in a given data set. 
Hence, we define the scenario mining problem as follows:
\begin{problem}[Scenario mining]
	Given a scenario category, how to find all scenarios that correspond to this scenario category in a given data set?
\end{problem}
\cenda


\section{Proposed Risk Estimation Method}
\label{sec:method}
 
In the Hazard Analysis and Risk Assessment (HARA) required by the ISO~26262 standard, the estimation of Automotive Safety Integrity Level (ASIL) is calculated based on a so-called single specific hazardous event \cite{ISO26262}.
Although the operational situation in which this single event occurs as well as the operating mode are considered in the analysis, still the proceeding and successive events are not taken into account.
In this paper, we propose a new method to estimate the risk of a certain scenario considering the whole chain of activities and conditions that constitute the scenario.
The estimated risk is based on real-world driving data. To estimate the risk, we will quantify the exposure and the severity. In \cref{Tab:Terms}, we present the definitions of the terms that are used in our proposed methodology. 

\begin{table}
	\centering
	\caption{The terms and definitions}
	\label{Tab:Terms}
	\begin{tabular}{p{0.15\linewidth} p{0.75\linewidth}} \hline
		\textbf{Term} & \textbf{Definition} \\ \hline
		Severity & An estimate of the extent of harm to one or more individuals that can occur in a potentially hazardous event~\cite{ISO26262} \\
		Exposure & The state of being in a driving scenario \\
		Risk & The combination of the probability of occurrence of harm and the severity of that harm~\cite{ISO26262} \\ 
		Condition & The constant parameters describing the environmental aspects of the operational design domain\footnote{``Operating conditions under which a given driving automation system or feature thereof is specifically designed to function'' \cite{sea2018j3016}.} \\
		Actor & An element of a scenario acting on its own behalf~\cite{ulbrich2015} \\ 
		Scenario & A quantitative description of the activities of the ego vehicle and other actors and the conditions from the static environment \\ \hline
	\end{tabular}
\end{table}

As explained in \cref{Tab:Terms}, a scenario consist of a set of conditions and activities, denoted by $A$ and $C$, respectively. We formulate the exposure as the average number of occurrences of the activities $A$ under the conditions $C$, denoted by $\lambda_{A,C}$. The severity is the likelihood of the potential hazardous consequence $R$ given the activities $A$ and the conditions $C$, denoted by the conditional probability $P(R|A,C)$. The risk is computed as the multiplication of the exposure and the severity. 

The proposed method is summarized in \cref{fig:method}. To compute the exposure, we calculate the likelihood of the conditions, denoted by $P(C)$, and the conditional likelihood of the activities, denoted by $P(A|C)$, based on real-world driving data. This is explained in detail in \cref{sec:exposure}. For the estimation of the severity, we consider all possible scenarios that are subject to a set of conditions $C$ and consist of the activities $A$. Therefore, we parametrize the scenarios using the parameter vector $\theta$. Based on the real-world driving data, the probability density function of the parameters $P(\theta|A,C)$ is estimated. Next, using simulations, we estimate $P(R|\theta,A,C)$, the likelihood of a potential hazardous consequence $R$ given a parametrized scenario. The details of the estimation of the severity are presented in \cref{sec:severity}. Finally, in \cref{sec:risk}, we describe how the risk is estimated based on the estimated exposure and severity.

\begin{figure}
	\centering
	\includestandalone[width=\linewidth]{figures/method}
	\caption{Proposed method for quantifying the risk. The risk is a multiplication of the exposure and the severity, explained in \cref{sec:exposure,sec:severity}, respectively.}
	\label{fig:method}
\end{figure}

%The following steps are followed when computing the risk:
%\begin{enumerate}
%	\item Estimate the likelihood of the conditions $P(C)$.
%	\item Estimate the exposure $\lambda_{A,C}$.
%	\item Parametrize the scenario with a parameter vector $\theta$.
%	\item Estimate the distribution of $\theta$ for this scenario class, i.e., $P(\theta|A,C)$.
%	\item Use simulations to estimate the severity, i.e., $P(R|A,C)$.
%	\item Compute the risk of a scenario class denoted by $\lambda$.
%	\item Compute the number of hours that can be driven without any harm associated with the given scenario class.
%\end{enumerate}


\subsection{Calculate exposure}
\label{sec:exposure}

The scenarios are subject to $n_C$ conditions, denoted by $C_1, \ldots, C_m$. For the sake of brevity, all conditions together are denoted by $C$, i.e., $P(C_1, \ldots, C_{n_C})=P(C)$. Many of these conditions might be based on the operational design domain of the AD system and might include conditions with respect to the infrastructure, weather conditions, lighting conditions, and geographical locations. 

The first step is to compute the joint probability of the conditions, i.e., $P(C)$. In case these conditions are independent, the probability can be computed by simply multiplying the individual likelihoods for each condition, i.e., $P(C)=P(C_1)\cdot\ldots\cdot P(C_{n_C})$. This, however, might not necessarily be the case, which requires either to compute the joint probability or to compute conditional probabilities. In some cases, it might also be reasonable to simply assume that the likelihood of certain conditions are independent.

Note that the the defined conditions might not be the same as the conditions under which the data is collected that is used to compute $P(C)$. This might require additional assumptions, see our example in \cref{sec:example exposure}.

To calculate the exposure, the average number of occurrences of the activities that constitute the scenarios that fall into the specified scenario class within a certain time interval need to be estimated. Let $n_A$ denote the number of activities, such that $A_1, \ldots, A_{n_A}$ denote the activities. For the sake of brevity, all activities together are denoted by $A$. 

Without loss of generality, we assume that the time interval is an hour. To estimate the number occurrences of the activities, the data for which the conditions $C$ are satisfied are analyzed. The average number of occurrences of the activities $A$ for each hour of driving for which the conditions $C$ are satisfied is denoted by $\lambda_{A|C}$. Next, we can calculate the average number of occurrences of the activities $A$ under the conditions $C$ for each hour of driving:
\begin{equation}
	\lambda_{A,C} = \lambda_{A|C} \cdot P(C).
\end{equation}

Regarding the scenarios that fall into the specified scenario class, we assume the following:
\begin{itemize}
	\item The occurrence of one scenario consisting of activities $A$ and conditions $C$ does not affect the probability that a second scenario consisting of activities $A$ and conditions $C$ occurs.
	\item The rate at which a scenario consisting of activities $A$ and conditions $C$ occurs is constant. I.e., $\lambda_{A,C}$ is constant.
	\item Two scenarios consisting of activities $A$ and conditions $C$ cannot occur at exactly the same time instant.
\end{itemize}
Based on these assumptions, the number of occurrences of scenarios consisting of activities $A$ and conditions $C$ is distributed according to the Poisson distribution:
\begin{equation}
	P(k\text{ times }A,C\text{ in an hour}) = \exp \left\{-\lambda_{A,C} \right\} \frac{\lambda_{A,C}^k}{k!}.
\end{equation}



\subsection{Severity}
\label{sec:severity}

The first step towards estimating the severity is to parametrize the scenarios with a parameter vector $\theta \in \mathbb{R}^d$. The parametrization enables the generation of infinitely many unique individual test cases that resemble the scenarios found in naturalistic driving \cite{deGelder2017assessment,elrofai2018scenario}.

In case the parameters are dependent, which is often the case, it is important that the number of parameters is limited to avoid the curse of dimensionality \cite{scott2015multivariate}. This often requires some assumptions. An example is presented in \cref{sec:example severity}.

To estimate the probability density function (pdf) of the parameter vector $\theta$, i.e., $P(\theta|A,C)$, either parametric models, non-parametric models, or a combination of the two can be used. In case of parametric models, a certain functional form of the pdf is assumed. For example, it might be assumed that the pdf can be modeled using a Gaussian distribution. In this paper, we present a non-parametric approach using Kernel Density Estimation (KDE) \cite{rosenblatt1956remarks, parzen1962estimation}. Using KDE, there is no assumption on the functional form of the pdf because the shape of the pdf is automatically computed.

Using KDE, the estimated pdf is given by
\begin{equation}
	\label{eq:kde}
	P(\theta|A,C) = \frac{1}{nh^d} \sum_{i=1}^n K\left(\frac{\theta - \theta_i}{h}\right).
\end{equation}
Here, $K(\cdot)$ is an appropriate kernel function and $h$ denotes the bandwidth. From the data, $n$ scenarios are extracted and each scenario is parametrized with $\theta_i$. The choice of the kernel $K(\cdot)$ is not as important as the choice of the bandwidth $h$ \cite{turlach1993bandwidthselection}. Often, a Gaussian kernel is used, which is given by
\begin{equation}
	\label{eq:gaussian kernel}
	K(u) = \frac{1}{\left( 2\pi \right)^{d/2}} \exp \left\{ -\frac{1}{2} \|u\|^2 \right\},
\end{equation}
where $\|u\|^2$ denotes the squared 2-norm of $u$, i.e., $u^T u$.

The bandwidth $h$ controls the amount of smoothing. For the kernel of \cref{eq:gaussian kernel}, the same amount of smoothing is applied in every direction, although this can easily be extended to a multi-dimensional bandwidth, see, e.g., \cite{scott2005multidimensional, chen2017tutorial}. There are many different ways of estimating the bandwidth, ranging from simple reference rules like, e.g., Scott's rule of thumb \cite{scott2015multivariate} or Silverman's rule of thumb \cite{silverman1986density} to more elaborate methods; see \cite{turlach1993bandwidthselection, bashtannyk2001bandwidth, jones1996brief, chiu1996comparative} for reviews of different bandwidth selection methods. 

Let $R$ denote a potential hazardous consequence of a scenario. We define the severity of a scenario with activities $A$ and conditions $C$ as the probability of $R$, given the activities $A$ and $C$, i.e., $P(R|A,C)$. We cannot evaluate $P(R|A,C)$ directly, because the outcome of a scenario highly depends on the parametrization $\theta$. Therefore, we estimate $P(R|\theta,A,C)$ through a simulation of the scenario with parameters $\theta$. Using $P(\theta|A,C)$ from \cref{eq:kde}, we can compute 
\begin{equation} \label{eq:probability R theta}
	P(R,\theta|A,C) = P(R|\theta,A,C) \cdot P(\theta|A,C).
\end{equation}
To obtain $P(R|A,C)$, we need to integrate \cref{eq:probability R theta} over $\theta$, i.e., 
\begin{equation} \label{eq:probability R}
	P(R|A,C) = \int_{\mathbb{R}^d} P(R|\theta,A,C) \cdot P(\theta|A,C) \ud \theta.
\end{equation}

One approach to evaluate the integral of \cref{eq:probability R} is to perform Monte Carlo simulations. For sufficiently large $N$, we have
\begin{equation} \label{eq:monte carlo}
	P(R|A,C) \approx \frac{1}{N} \sum_{k=1}^N P(R|\theta_k,A,C), \, \theta_k \sim P(\theta|A,C).
\end{equation}

To improve the accuracy of \cref{eq:monte carlo}, importance sampling can be used where the parameters $\theta$ are drawn from another distribution with a focus on the critical scenarios, see, e.g., \cite{deGelder2017assessment}.



\subsection{Calculating the risk}
\label{sec:risk}

Analogous to the exposure, we define the risk as the number of occurrences of the harmful outcome $R$ in a scenario consisting of activities $A$ and conditions $C$ in a certain time interval. Let $\lambda$ denote the average number of these occurrences in an hour of driving. The chain rule of probability tells us that this equals the sum of $\lambda_{A,C}$ (i.e., the exposure) and $P(R|A,C)$ (i.e., the severity):
\begin{equation} \label{eq:risk}
	\lambda = \lambda_{A,C} \cdot P(R|A,C)
\end{equation}

Analogous to the number of occurrences of a scenario consisting of activities $A$ and conditions $C$, we assume that the number of occurrences of a harmful outcome $R$ in a scenario consisting of activities $A$ and conditions $C$ can be modeled using a Poisson distribution:
\begin{equation} \label{eq:poisson risk}
	P(k\text{ times }R,A,C\text{ in an hour}) = \exp \left\{ -\lambda \right\} \frac{\lambda^k}{k!}.
\end{equation}

Using \cref{eq:poisson risk}, to calculate the probability of not having the harmful outcome $R$ in a scenario consisting of activities $A$ and conditions $C$ we simply need to use $k=0$:
\begin{equation} \label{eq:no harm}
	P(\text{no }R,A,C\text{ in one hour}) = \exp \left\{ -\lambda \right\}.
\end{equation}


%Let $E$ be the final event that might result in a risky situation/harm. The probability $P$ of the exposure of a harm/risk in a certain scenario is then given by 
%\begin{equation}
%P(E,A_1,A_2,...A_n,C_1,C_2,..,C_m)=P(E,A,C) \label{eq:secM1}
%\end{equation}
%where $n$ is the number of performed activities within the scenario and $m$ is the number of conditions in the same scenario.
%Since the harm event $E$ depends on the performed actives and scenario conditions, to compute \ref{eq3} requires computing $P(A,C)$ first.
%\begin{equation}
%P(A,C)=P(A|C) \cdot P(C) \label{eq:secM2}
%\end{equation}

\section{Case study}
\label{sec:case study}

Here we illustrate the method by applying the method to the dataset described in \autocite{paardekooper2019dataset6000km}.

\begin{table}
	\centering
	\caption{\cstartc Values of parameters used in the case study. \cendc}
	\label{tab:parameters}
	\cstartc
	\begin{tabularx}{\linewidth}{lXl}
		\toprule
		Parameter & Description & Value \\ \otoprule
		$\sampletime$ & Sample time & \SI{0.01}{\second} \\
		$\samplehorizon$ & Sample window & 100 \\
		$\accelerationstart$ & Threshold determining the start of an acceleration or deceleration activity & \SI{0.1}{\meter\per\second\squared} \\
		$\accelerationcruise$ & Threshold determining the end of an acceleration or deceleration activity & \SI{0.1}{\meter\per\second\squared} \\
		$\speeddiff$ & Minimum speed increase/decrease for an acceleration/deceleration activity & \SI{1}{\meter\per\second} \\
		$\samplescruising$ & Minimum number of samples for cruising activity & 400 \\
		$\lanechangethreshold$ & A lane change is detected when the difference between consecutive lane line distances is larger than this threshold & \SI{1}{\meter} \\
		$\lanechangespeed$ & Threshold determining the start and end of a lane change & \SI{0.25}{\meter\per\second} \\
		$\factorgoalmax$ & Maximum factor of the lane width for a lane change of any other vehicle & 0.5 \\
		$\factorgoalmin$ & Minimum factor of the lane width for a lane change of any other vehicle & 0.1 \\
		\bottomrule
	\end{tabularx}
	\cendc
\end{table}

\begin{table*}
	\centering
	\caption{\cstartc N-grams that describe the scenario category ``overtaking before lane change''.\cendc}
	\label{tab:overtaking lane change}
	\cstartc
	\begin{tabularx}{\linewidth}{lXXX}
		\toprule
		Subject & Item 1 & Item 2 & Item 3 \\ \otoprule
		Ego vehicle & Lateral activity: Following lane & Lateral activity: Following lane & Lateral activity: Changing lane left \\
		Other vehicle & Lateral state: Left AND \newline Longitudinal state: Rear & Lateral state: Left AND \newline Longitudinal state: Front & Lateral state: Left AND \newline  Longitudinal state: Front \\
		Static environment & On highway: Yes & On highway: Yes & On highway: Yes \\
		\bottomrule
	\end{tabularx}
	\cendc
\end{table*}

\begin{table}
	\centering
	\caption{\cstartc N-grams that describe the scenario category ``lead vehicle braking''. The ego vehicle is not included in this table because there are no activities defined for the ego vehicle for this scenario category. \cendc}
	\label{tab:lead vehicle braking}
	\cstartc
	\begin{tabularx}{\linewidth}{lX}
		\toprule
		Subject & Item 1 \\ \otoprule
		Other vehicle & Longitudinal activity: Braking AND \newline Lead vehicle: Yes \\
		Static environment & On highway: Yes \\
		\bottomrule
	\end{tabularx}
	\cendc
\end{table}

\todo{Write case study setup. I will consider three different scenarios: Cut-in (as shown in \cref{tab:ngrams cutin}), overtaking before lane change (as shown in \cref{tab:overtaking lane change}), and lead vehicle braking (as shown in \cref{tab:lead vehicle braking}).}

\todo{I will consider changing the tables (\cref{tab:ngrams cutin,tab:overtaking lane change,tab:lead vehicle braking}) to figures that look more like \cref{fig:tags cut in}. It will be visually more attractive but it will also consume more space.}

\todo{Obtain results (this will require quite some time!)}

\todo{Write about the results.}

\section{Conclusions}
\label{sec:conclusions}

\todo{Write the conclusions. Perhaps also a short discussion?}



%\addtolength{\textheight}{-12cm}  % This command serves to balance the column lengths
                                  % on the last page of the document manually. It shortens
                                  % the textheight of the last page by a suitable amount.
                                  % This command does not take effect until the next page
                                  % so it should come on the page before the last. Make
                                  % sure that you do not shorten the textheight too much.

%\bibliographystyle{ieeetran}
%\bibliography{../bib}
\printbibliography

%\appendix
%\appendices
%\section{Details of the domain model}
\label{sec:appendix domain model}

In this section, we provide more details of the domain model, i.e., we describe all different classes and their attributes.



\end{document}
