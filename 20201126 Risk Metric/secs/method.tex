\section{Methods}
\label{sec:methods}

The goal of the proposed method is to quantify the risk of a particular situation in which a vehicle - hereafter, the \textit{ego vehicle} - is in.
Our method for calculating this safety metric for quantifying the risk consists of four steps. 
First, we parameterize the current situation and the possible future situation.
Second, based on the current situation, we estimate the probability (density) for the possible future situations. 
The third step is to determine the probability of a collision based on the current and the future situations.
The final step is to use local regression to speed up the calculations and to make it possible to use the safety metric in real time. 
These four steps are described in the following sections.

In the remainder of this article, the following notation is used. 
To denote a probability, $\probability{\cdot}$ is used. 
A probability density is denoted by $\density{\cdot}$. 
The conditional probability $\probabilitycond{\dummyvara}{\dummyvarb}$ is verbalized as \textit{the probability of $\dummyvara$ given $\dummyvarb$}. 
Similarly, the conditional probability density is denoted by $\densitycond{\cdot}{\cdot}$. 
To denote the estimation of any of the aforementioned quantities, a circumflex is used. 
E.g, $\probabilityest{\dummyvara}$ denotes the estimated probability of $\dummyvara$.



\subsection{Parameterize current and future situations}
\label{sec:parametrization}

The first step is to parameterize the current situation in which the ego vehicle is in. 
In other words, the current situation needs to be described using $\situationcurrentdim$ numbers that are stacked into one vector $\situationcurrent \in \situationcurrentspace \subseteq \realnumbers^{\situationcurrentdim}$. 
As an example, $\situationcurrent$ could contain the speed of the ego vehicle and distance towards its preceding vehicle. 
In \cref{sec:case study}, we will see more examples.

Next to describing the current situation, the future situation is described using $\situationfuturedim$ numbers stacked into one vector $\situationfuture \in \situationfuturespace \subseteq \realnumbers^{\situationfuturedim}$. 
Together with $\situationcurrent$, $\situationfuture$ contains enough information to describe how the relevant future around the ego vehicle develops over time. 
As an example, $\situationfuture$ could contain the acceleration of the lead vehicle (if any) in front of the ego vehicle.

Let $\collision$ denote a collision, such that the probability of a collision is $\probability{\collision}$.
The goal of our safety metric is to estimate the probability of a collision given a particular situation $\situationcurrent$, i.e., $\probabilitycond{\collision}{\situationcurrent}$.
We do this by considering all future situations, $\situationfuturespace$, and calculating the probability of a collision given each possible value of $\situationfuture$. 
Using integrating, we obtain $\probabilitycond{\collision}{\situationcurrent}$:
\begin{equation}
	\label{eq:probability collision expectation}
	\probabilitycond{\collision}{\situationcurrent} 
	= \int_{\situationfuturespace} 
	\probabilitycond{\collision}{\situationcurrent, \situationfuture} 
	\densitycond{\situationfuture}{\situationcurrent} 
	\ud \situationfuture.
\end{equation}
In \cref{sec:estimate future}, we propose a method to estimate $\densitycond{\situationfuture}{\situationcurrent}$ and in \cref{sec:estimate collision}, we propose a method to estimate $\probabilitycond{\collision}{\situationcurrent, \situationfuture}$.



\subsection{Estimate $\densitycond{\situationfuture}{\situationcurrent}$}
\label{sec:estimate future}

In this section, we propose a method to estimate $\densitycond{\situationfuture}{\situationcurrent}$, i.e., the probability density of the $\situationfuture$ given $\situationcurrent$.
Using the product rule for probability, we can write:
\begin{equation}
	\densitycond{\situationfuture}{\situationcurrent} 
	= \frac{\density{\situationcurrent, \situationfuture}}{\density{\situationcurrent}}
	= \frac{\density{\situationcurrent, \situationfuture}}{
		\int_{\situationfuturespace} \density{\situationcurrent, \situationfuture} \ud\situationfuture
	}
\end{equation}
Thus, it suffices to estimate $\density{\situationcurrent, \situationfuture}$. 



\subsection{Estimate $\probabilitycond{\collision}{\situationcurrent, \situationfuture}$}
\label{sec:estimate collision}



