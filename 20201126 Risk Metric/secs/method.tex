\section{Methods}
\label{sec:methods}

The goal of the proposed method is to quantify the risk of a particular situation in which a vehicle - hereafter, the \textit{ego vehicle} - is in.
Our method for calculating this safety metric for quantifying the risk consists of four steps. 
First, we parameterize the current situation and the possible future situation.
Second, based on the current situation, we estimate the probability (density) for the possible future situations. 
The third step is to determine the probability of a collision based on the current and the future situations.
The final step is to use local regression to speed up the calculations and to make it possible to use the safety metric in real time. 
These four steps are described in the following sections.

In the remainder of this article, the following notation is used. 
To denote a probability, $\probability{\cdot}$ is used. 
\Iac{pdf} is denoted by $\density{\cdot}$. 
The conditional probability $\probabilitycond{\dummyvara}{\dummyvarb}$ is verbalized as \textit{the probability of $\dummyvara$ given $\dummyvarb$}. 
Similarly, the conditional \ac{pdf} is denoted by $\densitycond{\cdot}{\cdot}$. 
To denote the estimation of any of the aforementioned quantities, a circumflex is used. 
E.g, $\probabilityest{\dummyvara}$ denotes the estimated probability of $\dummyvara$.



\subsection{Parameterize current and future situations}
\label{sec:parametrization}

The first step is to parameterize the current situation in which the ego vehicle is in. 
In other words, the current situation needs to be described using $\situationcurrentdim$ numbers that are stacked into one vector $\situationcurrent \in \situationcurrentspace \subseteq \realnumbers^{\situationcurrentdim}$. 
As an example, $\situationcurrent$ could contain the speed of the ego vehicle and distance towards its preceding vehicle. 
In \cref{sec:case study}, we will see more examples.

Next to describing the current situation, the future situation is described using $\situationfuturedim$ numbers stacked into one vector $\situationfuture \in \situationfuturespace \subseteq \realnumbers^{\situationfuturedim}$. 
Together with $\situationcurrent$, $\situationfuture$ contains enough information to describe how the relevant future around the ego vehicle develops over time. 
As an example, $\situationfuture$ could contain the acceleration of the lead vehicle (if any) that is in front of the ego vehicle.

Let $\collision$ denote a collision, such that the probability of a collision is $\probability{\collision}$.
The goal of our safety metric is to estimate the probability of a collision given a particular situation $\situationcurrent$, i.e., $\probabilitycond{\collision}{\situationcurrent}$.
We do this by considering all future situations, $\situationfuturespace$, and calculating the probability of a collision given each possible value of $\situationfuture$. 
Using integrating, we obtain $\probabilitycond{\collision}{\situationcurrent}$:
\begin{equation}
	\label{eq:probability collision expectation}
	\probabilitycond{\collision}{\situationcurrent} 
	= \int_{\situationfuturespace} 
	\probabilitycond{\collision}{\situationcurrent, \situationfuture} 
	\densitycond{\situationfuture}{\situationcurrent} 
	\ud \situationfuture.
\end{equation}
In \cref{sec:estimate future}, we propose a method to estimate $\densitycond{\situationfuture}{\situationcurrent}$ and in \cref{sec:estimate collision}, we propose a method to estimate $\probabilitycond{\collision}{\situationcurrent, \situationfuture}$.



\subsection{Estimate $\densitycond{\situationfuture}{\situationcurrent}$}
\label{sec:estimate future}

In this section, we propose a method to estimate $\densitycond{\situationfuture}{\situationcurrent}$, i.e., the \ac{pdf} of the $\situationfuture$ given $\situationcurrent$.
Using the product rule for probability, we can write:
\begin{equation}
	\densitycond{\situationfuture}{\situationcurrent} 
	= \frac{\density{\situationcurrent, \situationfuture}}{\density{\situationcurrent}}
	= \frac{\density{\situationcurrent, \situationfuture}}{
		\int_{\situationfuturespace} \density{\situationcurrent, \situationfuture} \ud\situationfuture
	}
\end{equation}
Thus, it suffices to estimate $\density{\situationcurrent, \situationfuture}$. 

Our proposal is to estimate $\density{\situationcurrent, \situationfuture}$ in a data-driven manner. 
A data-driven approach brings several benefits.
First, the estimate automatically adapts to local driving styles and behaviors, which can change from region to region, provided that the data are obtained from the same local traffic.
Second, strong assumptions such as assuming a constant speed of other vehicles, are not needed.
For our data-driven approach, let us assume that we have obtained $\situationnumberof$ situations, denoted by $\situationcurrentinstance_{\situationindex}\in\situationcurrentspace, \situationindex\in\{1,\ldots,\situationnumberof\}$, and their corresponding future situations described by $\situationfutureinstance_{\situationindex}\in\situationfuturespace$.



\subsubsection{Special case: all parameters from one distribution}
\label{sec:one kde}

We first explain how to estimate $\density{\situationcurrent, \situationfuture}$ if we assume that all parameters depend on each other and that no further simplifications are possible. 
The shape of the \ac{pdf} $\density{\situationcurrent, \situationfuture}$ is unknown beforehand. 
Furthermore, the shape of the estimated \ac{pdf} might change as more data are acquired. 
Assuming a functional form of the \ac{pdf} and fitting the parameters of the \ac{pdf} to the data may therefore lead to inaccurate fits unless a lot of hand-tuning is applied.
We employ a non-parametric approach using \ac{kde} \autocite{rosenblatt1956remarks, parzen1962estimation} because the shape of the \ac{pdf} is automatically computed and \ac{kde} is highly flexible regarding the shape of the \ac{pdf}. 
Using the \ac{kde}, the estimated \ac{pdf} becomes:
\begin{equation}
	\label{eq:kde estimate}
	\densityest{\situationcurrent,\situationfuture}
	= \frac{1}{\situationnumberof} \sum_{\situationindex=1}^{\situationnumberof}
	\kernelfuncnormalized{\bandwidthmatrix}{
		\begin{bmatrix}
			\situationcurrent \\
			\situationfuture
		\end{bmatrix} -
		\begin{bmatrix}
			\situationcurrentinstance_{\situationindex} \\
			\situationfutureinstance_{\situationindex}
		\end{bmatrix}
	},
\end{equation}
where $\kernelfuncnormalized{\bandwidthmatrix}{\cdot}$ is an appropriate kernel function with bandwidth matrix $\bandwidthmatrix$. 
The choice of the kernel $\kernelfuncnormalized{\bandwidthmatrix}{\cdot}$ is not as important as the choice of the bandwidth matrix $\bandwidthmatrix$ \cite{turlach1993bandwidthselection}.
We use a Gaussian kernel, but our method also applies with other kernels.
The Gaussian kernel is given by
\begin{equation}
	\kernelfuncnormalized{\bandwidthmatrix}{\dummyvarkernel}
	= \frac{1}{\left( 2 \pi \right)^{\left( \situationcurrentdim + \situationfuturedim \right) / 2} 
	\left|\bandwidthmatrix\right|^{1/2} }
	\e{ -\frac{1}{2} \dummyvarkernel^T \bandwidthmatrix^{-1} \dummyvarkernel }.
\end{equation}

Drawing samples from the \ac{kde} in \cref{eq:kde estimate} is straightforward: two random numbers are drawn, one to choose a random generator Gaussian out of the $\situationnumberof$ Gaussians that are used to construct the \ac{kde}, and one random number from that Gaussian.
Sampling from $\densityestcond{\situationfuture}{\situationcurrent}$ works similarly, but instead of using an equal probability for each random generator Gaussian to be selected, different probabilities are used based on $\situationcurrent$.
For more information on sampling from a conditional \ac{pdf} obtained using \ac{kde}, see \autocite{holmes2012fast}.



\subsubsection{Not all parameters from one distribution}
\label{sec:no special case}

Estimating $\density{\situationcurrent, \situationfuture}$ with one \ac{kde} according to \cref{eq:kde estimate} becomes problematic if $\situationcurrentdim + \situationfuturedim$ becomes large, due to the curse of dimensionality \cite{scott2015multivariate}.
There are a few ways to avoid problems.
Without going into much detail, we will list a few options.

One option is to assume that one or more parameter are independent of the other parameters. 
E.g., suppose that $\situationfuture=\begin{bmatrix}\situationfutureparta & \situationfuturepartb\end{bmatrix}^T$, such that $\situationfuturepartb$ is independent of $\situationcurrent$ and $\situationfutureparta$.
Then we can write
\begin{equation}
	\density{\situationcurrent, \situationfuture}
	= \density{\situationcurrent, \situationfutureparta, \situationfuturepartb}
	= \density{\situationcurrent, \situationfutureparta} \cdot \density{\situationfuturepartb}.
\end{equation}
In this case, we would need to estimation $\density{\situationcurrent, \situationfutureparta}$ and $\density{\situationfuturepartb}$, which can be done in a similar manner as presented in \cref{sec:one kde}.
Because these two \acp{pdf} have less variables than $\density{\situationcurrent, \situationfuture}$, it will suffer less from the curse of dimensionality \cite{scott2015multivariate}.

Another option is to model $\densitycond{\situationfuture}{\situationcurrent}$ as a cascade of conditional probabilities. E.g., using the partitioning $\situationfuture=\begin{bmatrix}\situationfutureparta & \situationfuturepartb\end{bmatrix}^T$, $\densitycond{\situationcurrent}{\situationfuture}$ can be approximated using two conditional densities:
\begin{equation}
	\densitycond{\situationcurrent}{\situationfutureparta}
	= \densitycond{\situationfutureparta, \situationfuturepartb}{\situationcurrent}
	= \densitycond{\situationfutureparta}{\situationfuturepartb, \situationcurrent} \cdot \densitycond{\situationfuturepartb}{\situationcurrent}
	\approx \densitycond{\situationfutureparta}{\situationfuturepartb} \cdot \densitycond{\situationfuturepartb}{\situationcurrent}.
\end{equation}
The same partitioning can be applied to $\densitycond{\situationfutureparta}{\situationfuturepartb}$ and $\densitycond{\situationfuturepartb}{\situationcurrent}$ until only two-dimensional \acp{pdf} need to be estimated.
For more information on this, we refer the reader to \autocite{aas2009paircopula, nagler2016evading}.



\subsection{Estimate $\probabilitycond{\collision}{\situationcurrent, \situationfuture}$}
\label{sec:estimate collision}

To estimate $\probabilitycond{\collision}{\situationcurrent, \situationfuture}$, i.e., the probability of a collision given the current situation $\situationcurrent$ and the future situation $\situationfuture$, we employ simulations. 
The details of the simulation depends on the actual application. 
For example, if the goal of our surrogate safety metric is to evaluate the risk a current human driver is in, the simulation should involve human driver models. 
On the other hand, if the goal is to evaluate the risk of \iac{ads}, the simulation should include the same \iac{ads}.

A straightforward way to compute $\probabilitycond{\collision}{\situationcurrent, \situationfuture}$ is to repeat a certain number of simulations with the same $\situationcurrent$ and $\situationfuture$ and count the number of simulations that result in a collision.
If $\numberofsimulations$ denotes the number of simulations and $\numberofcollisions$ is the number of collisions, then $\probabilitycond{\collision}{\situationcurrent, \situationfuture}$ could be estimated using
\begin{equation}
	\probabilitycondest{\collision}{\situationcurrent, \situationfuture}
	= \frac{\numberofcollisions}{\numberofsimulations}.
\end{equation}

