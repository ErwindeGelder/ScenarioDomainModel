\section{Case study}
\label{sec:case study}

Here we illustrate the method by applying the method to the dataset described in \autocite{paardekooper2019dataset6000km}.

\begin{table}
	\centering
	\caption{\cstartc Values of parameters used in the case study. \cendc}
	\label{tab:parameters}
	\cstartc
	\begin{tabularx}{\linewidth}{lXl}
		\toprule
		Parameter & Description & Value \\ \otoprule
		$\sampletime$ & Sample time & \SI{0.01}{\second} \\
		$\samplehorizon$ & Sample window & 100 \\
		$\accelerationstart$ & Threshold determining the start of an acceleration or deceleration activity & \SI{0.1}{\meter\per\second\squared} \\
		$\accelerationcruise$ & Threshold determining the end of an acceleration or deceleration activity & \SI{0.1}{\meter\per\second\squared} \\
		$\speeddiff$ & Minimum speed increase/decrease for an acceleration/deceleration activity & \SI{1}{\meter\per\second} \\
		$\samplescruising$ & Minimum cruising time & \SI{4}{\second} \\
		$\lanechangethreshold$ & A lane change is detected when the difference between consecutive lane line distances is larger than this threshold & \SI{1}{\meter} \\
		$\lanechangespeed$ & Threshold determining the start and end of a lane change & \SI{0.25}{\meter\per\second} \\
		$\factorgoalmax$ & Maximum factor of the lane width for a lane change of any other vehicle & 0.5 \\
		$\factorgoalmin$ & Minimum factor of the lane width for a lane change of any other vehicle & 0.1 \\
		\bottomrule
	\end{tabularx}
	\cendc
\end{table}

\todo{Write case study setup}

\todo{Obtain results (this will require quite some time!)}

\todo{Write about the results.}
