\section{Discussion and future outlook} % Hala & Erwin & Arash
\label{sec:discussion}



We illustrated the applicability of our risk estimation method through an example in the previous section. 
However, our method has some limitations as well. 
As an example, many assumptions are made to simplify the calculation of estimated risk or because there are unknowns due to lack of data. These assumptions reduce the accuracy of the estimated risk. 
Another limitation is that we applied the proposed method for only one type of driving scenario, while the full potential can be better demonstrated by applying the method to a wider range of scenarios.

Despite the mentioned limitations, we believe that our proposed method  takes an important step towards objective hazard and risk analysis as we summarize in the following points: 
\begin{itemize}
	\item All the assumptions that were made for estimating the risk are explicit and based on measured data. By making the assumptions explicit, it is much clearer why a certain risk is associated with --- in this case --- a certain/specific scenario.
	\item Because our proposed method explicates all the steps and assumptions that lead to a certain estimated risk, it is easily possible to update the risk when more information of the system is known or when more data is available.
	\item The systematic quantification of the risk provides additional objectified trust in the safety analysis that depends on the availability of data rather than experts judgment. 
	\item The method can be scaled up to be applied to multiple scenarios and operational situations with small modifications. 
\end{itemize}



%To be discussed:
%\begin{itemize}
%	\item Method gives only order of risk.
%	\item ``Controllability'' not considered.
%	\item A lot of assumptions: with this method, these assumptions are made explicit, whereas often people make these assumptions implicit (and implicit assumptions are the mother of all fuck-ups; should be rephrased :)).
%\end{itemize}


