\section{Discussion and future outlook} % Hala & Erwin & Arash
\label{sec:discussion}

In the calculation of the estimated risk, many assumptions are made to simplify the calculation or because there are unknowns due to lack of data. This reduces the accuracy of the estimated risk. 
However, we still believe that the quantified risk is valuable because of the following reasons:
\begin{itemize}
	\item All the assumptions that were made for estimating the risk are explicit. In contrast, when people assign ASIL levels to a hazardous event, often many assumptions are implicitly made. By making the assumptions explicit, it is much clearer why a certain risk is associated with --- in this case --- a certain/specific scenario.
	\item It provides a good estimate of the order of the risk for each scenario. For example, if the estimated risk of two different scenarios differ by a factor 10, it is still reasonable to argue that one scenario introduces a higher risk than the other, even though the factor 10 might not be exact.
	\item Because our/the proposed method explicates all the steps and assumptions that lead to a certain estimated risk, it is easily possible to update the risk when more information of the system is known or when more data is available.
\end{itemize}

To be discussed:
\begin{itemize}
	\item Method gives only order of risk.
	\item ``Controllability'' not considered.
	\item A lot of assumptions: with this method, these assumptions are made explicit, whereas often people make these assumptions implicit (and implicit assumptions are the mother of all fuck-ups; should be rephrased :)).
\end{itemize}