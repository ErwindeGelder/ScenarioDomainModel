\section{Method}
\label{sec:method}

Method steps:
\begin{itemize}
    \item Identify scenarios within scope 
    \item Calculate probability of exposure to scenario based on observation in data
    \item Analyze relevant triggering conditions 
    \item Simulate scenarios and triggering conditions including controllablilty with driver model 
    \item Calculate probability or collision and relevant parameter for severity calculation 
    \item Calculate probability of injury for severity  
    \item Visualize risk using heat maps. 
\end{itemize}
Q: what about data collection? is that part of our method? 
We add an assumption that distribution of relevant scenario parameters are known, either though collected data or other reliable sources of information. 

\subsection{Identification of scenarios}

What do we need from a scenario?
\begin{itemize}
    \item Scene 
    \item ODD
    \item Actors 
    \item Activities and parameters 
\end{itemize}

Scenario (or operational situation is ISO~26262 terminology) is an important part risk assessment for both SOTIF and functional safety. 
Ensuring that relevant scenarios at the right level of detail are selected is important to ensure correct risk evaluation. 
To systematically evaluate scenarios, one method can be breaking it down into smaller parts for analysis.  
Several norms define what a scenario should include. 
According to SAE J2980 on considerations for ISO 26262 ASIL hazard classification, \textit{operational scenario}, includes specification of: Location (e.g.; highway or city), road conditions (e.g., road friction, slope), driving maneuvers (e.g., driving forward), vehicle state (e.g., accelerating, braking), and other relevant considerations and characteristics (e.g., weather condition). 
SOTIF also describes elements of a scenario (taking on the model from Ulbrich) as: Scene including dynamic and static element, actions, events, and goals. 
Another important term in this domain is ``operational design domain'' by SAE J3016, which specifies the conditions under which the AV is intended to be used.

Furthermore, other protocols such as OpenScneario consider: ....
\todo{finish this one.} 


\subsection{Triggering event analysis}



\subsection{Calculate severity}
\label{sec:severity}

\todo{Hala: summarize literature review for injury probability.}
