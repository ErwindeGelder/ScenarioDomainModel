\section{Problem definitions}
\label{sec:problem} % Arash

This Section presents: 
\begin{itemize}
	\item Why risk assessment matters?
	\item What is currently the risk estimation method? 
	\item What is lacking in this approach? 
	\item What need to be done more? 
\end{itemize}

In this section we position our research in the automotive safety engineering domain. 
We present the current risk assessment methods and discuss their limitation and the impact of these limits.
The argue that advancement in the risk assessment methods are required for achieving higher levels of automation in the automotive domain. 

\subsection{Why risk assessment?}

Safety means avoiding risk. 
The risk associated with driving may come from multiple sources. 
It could be a traffic situation in which a sequence of uncorrelated actions performed by different actors lead to an accident. 
The risk could also be due to technical failure originating from a system fault. 
%This type of faults and failures should be avoided by the manufacturers of the automotive systems (OEMs and Tiers), 
%	while the later is a responsibility of the traffic participants. 
For the former, the manufacturers are deemed responsible; 
	therefore, they put a lot of effort on the quality  assurance of their products 
		and for understanding and mitigating technical safety issues.  
The later comes into the public domain and road authorities are responsible for minimizing the risk by good design of roads and traffic rules. 
The  risk, however, cannot be avoided fully. 
There is always a certain amount of \textit{residual risk} remaining after taking risk avoidance/mitigation measures. 

Understanding the risk and measuring it is crucial in both directing the effort on avoiding or mitigating the impacts. 
Moreover, formulating and opinion about when using a system is ``safe enough'' depends on the ability to measure the risk. 
	

\subsection{Risk assessment in ISO~26262}

Risk is defined in ISO~26262 as:
\begin{definition}
	``combination of the probability of occurrence of harm and the severity of that harm'' 
\end{definition}



Risk assessment is an integral part of the safety life-cycle of ISO~26262 and one of the earlier activities. 
During risk assessment, the identified hazardous events are analyzed and the associated severity, probability of exposure and controllability are estimated and assigned to some predefined levels. 
The combination of the estimated severity, exposure and controllability contribute to construct the Automotive Safety Integrity Level (ASIL). 


This is the domain of functional safety. 
The goal of functional safety is to avoid \emph{unreasonable} risk\footnote{Unreasonable risk is judged according to the the society's acceptance of level of risk.} that is due to some sort of malfunctioning.

