\section{Problem definitions}
\label{sec:problem} % Arash

This Section presents: 
\begin{itemize}
	\item Why risk assessment matters?
	\item What is currently the risk estimation method? 
	\item What is lacking in this approach? 
	\item What need to be done more? 
\end{itemize}

In this section we position our research in the automotive safety engineering domain. 
We present the current risk assessment methods and discuss their limitation and the impact of these limits.
The argue that advancement in the risk assessment methods are required for achieving higher levels of automation in the automotive domain. 

\subsection{Why risk assessment?}

Safety means avoiding risk. 

The goal of functional safety is to avoid \emph{unreasonable} risk\footnote{Unreasonable risk is judged according to the the society's acceptance of level of risk.} that is due to some sort of malfunctioning .


\subsection{Risk assessment in ISO~26262}
Risk is defined in ISO~26262 as:
\begin{definition}
	``combination of the probability of occurrence of harm and the severity of that harm'' 
\end{definition}



Risk assessment is an integral part of the safety life-cycle of ISO~26262 and one of the earlier activities. 
During risk assessment, the identified hazardous events are analyzed and the associated severity, probability of exposure and controllability are estimated and assigned to some predefined levels. 
The combination of the estimated severity, exposure and controllability contribute to construct the Automotive Safety Integrity Level (ASIL). 




