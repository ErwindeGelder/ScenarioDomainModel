\section{Mining scenarios using tags}
\label{sec:mining}

\todo{Explain the approach of scenario mining. For each actor, an n-gram is constructed using the detected tags. A scenario category is then defined as a sequence of nodes, where the n-grams of multiple actors can be combined if needed.}

\cstartb
% Introduce n-grams and quickly explain what an n-gram is.
For the scenario mining, we will employ n-gram models. An n-gram is a sequence of $\nofngram$ items. 
For example, in the field of natural language processing --- where N-grams are a popular modeling technique \autocite{hull1982experiments,brown1992class} --- each item might correspond to a single word. 
In our case, an item corresponds to a combination of tags. 

% Explain that we use multiple n-gram models.
% What are the advantages of using multiple n-grams?
Instead of constructing one n-gram model with all tags within one dataset, we construct different n-gram models for the different subjects. 
I.e., we construct an n-gram for the ego vehicle, the static environment, and any other vehicle separately.
Constructing different n-gram models instead of one n-gram model bring several benefits:
\begin{itemize}
	\item The number of possible items is much lower when considering only one subject, such that it is more likely that most of the items are seen multiple times. As a result, analysis to predict tags or correct for wrong tags is possible \autocite{lesher1998optimal}.
	\item If different tags are needed for mining (other) scenarios, only the n-gram models of the corresponding subjects needs to be updated.
\end{itemize}

% Show example of one n-gram.

% Show how a scenario category is represented through a "template".

\cendb
