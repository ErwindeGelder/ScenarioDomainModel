\section{Mining scenarios using tags}
\label{sec:mining}

\begin{figure*}
	\centering
	%\resizebox{\textwidth}{!}{%
\begin{tikzpicture}
\node[block, text width=\subjectwidth-1em, minimum width=\subjectwidth, fill=egocolor, minimum height=2\tagheight] at (-\descriptionwidth, 0) {Ego vehicle};

% Longitudinal activity.
\node[block, text width=\descriptionwidth-1em, minimum width=\descriptionwidth, fill=egocolor] at (0, 0) {Longitudinal activity};
\node[tag, text width=\egovehiclecruising-1em, minimum width=\egovehiclecruising, fill=egocolor] at (0, 0) {Cruising};
\node[tag, text width=\egovehicleaccelerating-1em, minimum width=\egovehicleaccelerating, fill=egocolor] at (\egovehiclecruising, 0) {Accelerating};
\node[tag, text width=\tagtotalwidth-\egovehicleaccelerating-\egovehiclecruising-1em, minimum width=\tagtotalwidth-\egovehicleaccelerating-\egovehiclecruising, fill=egocolor] at (\egovehicleaccelerating+\egovehiclecruising, 0) {Cruising};

% Lateral activity
\node[block, text width=\descriptionwidth-1em, minimum width=\descriptionwidth, fill=egocolor] at (0, -\tagheight) {Lateral activity};
\node[tag, text width=\egovehiclefollowing-1em, minimum width=\egovehiclefollowing, fill=egocolor] at (0, -\tagheight) {Following lane};
\node[tag, text width=\egovehiclechanging-1em, minimum width=\egovehicleaccelerating, fill=egocolor] at (\egovehiclefollowing, -\tagheight) {Changing lane left};
\node[tag, text width=\tagtotalwidth-\egovehiclechanging-\egovehiclefollowing-1em, minimum width=\tagtotalwidth-\egovehiclechanging-\egovehiclefollowing, fill=egocolor] at (\egovehiclechanging+\egovehiclefollowing, -\tagheight) {Following lane};

% Items.
\foreach \x/\xend [count=\i] in {0/\egovehiclecruising, \egovehiclecruising/\egovehiclefollowing, \egovehiclefollowing/\egovehiclecruising+\egovehicleaccelerating, \egovehiclecruising+\egovehicleaccelerating/\egovehiclefollowing+\egovehiclechanging, \egovehiclefollowing+\egovehiclechanging/\tagtotalwidth} {%
	\node[minimum width=\xend-(\x), align=center, minimum height=\itempos, anchor=north west] at (\x, \itempos) {Item \i};
	\draw[showitem] (\x, 0) -- (\x, \itempos);
}
\draw[showitem] (\tagtotalwidth, 0) -- (\tagtotalwidth, \itempos);.

% Timeline.
\draw[timeline, ->] (0, -2\tagheight-\timepos) -> (\tagtotalwidth, -2\tagheight-\timepos);
\node[minimum width=\tagtotalwidth, align=center, anchor=north west, minimum height=\timepos] at (0, -2\tagheight) {Time};
\end{tikzpicture}
%}

	\caption{\cstartb Example of tags of the ego vehicle and the associated items.\cendb}
	\label{fig:ego tags}
\end{figure*}

\cstartb
% Introduce n-grams and quickly explain what an n-gram is.
For the scenario mining, we will employ n-grams and n-gram models. An n-gram is a sequence of $\nofngram$ items. 
For example, in the field of natural language processing --- where n-grams are a popular modeling technique \autocite{hull1982experiments, brown1992class} --- each item might correspond to a single word. 
In our case, an item corresponds to a combination of tags. 

% Explain that we use multiple n-gram models.
% What are the advantages of using multiple n-grams?
Instead of constructing one n-gram model with all tags within one dataset, we construct different n-gram models for the different subjects, i.e., we construct an n-gram for the ego vehicle, any other vehicle, and the static environment separately.
Constructing different n-gram models instead of one n-gram model brings several benefits:
\begin{itemize}
	\item The number of possible items is much lower when considering only one subject, such that it is more likely that most of the items are seen multiple times. 
	As a result, analysis to predict tags or correct for wrong tags is possible \autocite{lesher1998optimal}.
	\item If different tags are needed for mining (other) scenarios, only the n-gram models of the corresponding subjects need to be updated.
	\item Searching for n-grams within the n-gram models is much faster since the n-gram models itself are much smaller.
\end{itemize}

% Explain items
As mentioned before, an item corresponds to a combination of tags. 
To further explain this, consider the following example: The ego vehicle cruises while following its lane. 
At some point, the ego vehicle starts to accelerate before changing lane to the left. 
Next, the ego vehicle starts cruising before the lane change is finished. 
The corresponding tags are shown in \cref{fig:ego tags}.
This example contains five items and each item starts and ends with an event, i.e., the start or end of an activity.
The first item contains the tags ``cruising'' and ``following lane'', the second item contains the tags ``accelerating'' and ``following lane'', etc.

% Show how a scenario category is represented through a "template".
To mine scenarios of a specific scenario category, we write a scenario category using n-grams. 
To do this, the following rules apply:
\begin{itemize}
	\item For each subject that is considered in the scenario category, an n-gram is defined.
	\item Each n-gram uses the same number of items ($\nofngram$). However, $\nofngram$ may vary from scenario category to scenario category.
	\item For each item, the tags that apply are described. Logical AND or OR rules may apply.
\end{itemize}

\begin{table}
	\centering
	\caption{\cstartb N-grams that describe the scenario category ``cut in''. Because $\nofngram=2$, these are also called bigrams.\cendb}
	\label{tab:ngrams cutin}
	\cstartb
	\begin{tabularx}{\linewidth}{p{5.5em}XX}
		\toprule
		Subject & Item 1 & Item 2 \\ \otoprule
		Ego vehicle & Lateral activity: \newline Following lane & Lateral activity: \newline Following lane \\
		Other vehicle & Lateral activity: \newline \{Changing lane left OR \newline \phantom{\{}Changing lane right\}\newline AND Lead vehicle: No & Lateral activity: \newline \{Changing lane left OR \newline \phantom{\{}Changing lane right\}\newline AND Lead vehicle: Yes \\
		Static environment & On highway: Yes & On highway: Yes \\
		\bottomrule
	\end{tabularx}
	\cendb
\end{table}

In \cref{tab:ngrams cutin}, the n-grams are shown for the scenario category ``cut-in''. 
This scenario category considers a vehicle other than the ego vehicle that changes lane (other vehicle, item 1) and becomes the lead vehicle (other vehicle, item 2). 
In the meantime the ego vehicle follows its lane (ego vehicle, items 1 and 2) and the scenario category only considers highway driving (static environment, items 1 and 2).

% How scenarios are mined using the template.
The scenarios are mined by searching for matches of the defined n-grams within the n-gram models. 
This searching is subject to two rules:
\begin{enumerate}
	\item For each item, there need to be a match for all subjects \emph{at the same time}.
	\item The different items need to happen \emph{right after each other}. 
\end{enumerate}
An example is shown in \cref{fig:tags cut in}. 
The two vertical dashed lines indicate the start and the end of the cut in that is defined in \cref{tab:ngrams cutin}.
\cendb

\begin{figure*}
	\centering
	\setlength{\tagtotalwidth}{24em}
\setlength{\egovehiclefollowing}{15.5em}
\begin{tikzpicture}
% Ego vehicle.
\node[block, text width=\subjectwidth-1em, minimum width=\subjectwidth, fill=egocolor] at (-\descriptionwidth, 0) {Ego vehicle};
\node[block, text width=\descriptionwidth-1em, minimum width=\descriptionwidth, fill=egocolor] at (0, 0) {Lateral activity};
\node[tag, minimum width=\egovehiclefollowing, fill=egocolor] at (0, 0) {Following lane};
\node[tag, minimum width=\tagtotalwidth-\egovehiclefollowing, text width=\tagtotalwidth-\egovehiclefollowing-1em, fill=egocolor] at (\egovehiclefollowing, 0) {Changing lane left};

% Other vehicle, lateral activity.
\node[block, text width=\subjectwidth-1em, minimum width=\subjectwidth, fill=othervehicle, minimum height=2\tagheight] at (-\descriptionwidth, -\tagheight-\tagsep) {Other vehicle};
\node[block, text width=\descriptionwidth-1em, minimum width=\descriptionwidth, fill=othervehicle] at (0, -\tagheight-\tagsep) {Lateral activity};
\node[tag, text width=\otherfollowing-1em, minimum width=\otherfollowing, fill=othervehicle] at (0, -\tagheight-\tagsep) {Following lane};
\node[tag, text width=\otherchanging-1em, minimum width=\otherchanging, fill=othervehicle] at (\otherfollowing, -\tagheight-\tagsep) {Changing lane right};
\node[tag, text width=\tagtotalwidth-\otherfollowing-\otherchanging-1em, minimum width=\tagtotalwidth-\otherfollowing-\otherchanging, fill=othervehicle] at (\otherfollowing+\otherchanging, -\tagheight-\tagsep) {Following lane};

% Other vehicle, lead vehicle.
\node[block, text width=\descriptionwidth-1em, minimum width=\descriptionwidth, fill=othervehicle] at (0, -2\tagheight-\tagsep) {Lead vehicle};
\node[tag, minimum width=\othernolead, fill=othervehicle] at (0, -2\tagheight-\tagsep) {No leader};
\node[tag, minimum width=\tagtotalwidth-\othernolead, fill=othervehicle] at (\othernolead, -2\tagheight-\tagsep) {Leader};

% Static environment.
\node[block, text width=\subjectwidth-1em, minimum width=\subjectwidth, fill=staticenvironment] at (-\descriptionwidth, -3\tagheight-2\tagsep) {Static environment};
\node[block, text width=\descriptionwidth-1em, minimum width=\descriptionwidth, fill=staticenvironment] at (0, -3\tagheight-2\tagsep) {On highway};
\node[tag, minimum width=\tagtotalwidth, fill=staticenvironment] at (0, -3\tagheight-2\tagsep) {Highway};

% Show where the cut in is.
\draw[cutinline, dashed] (\otherfollowing, \cutinheight) -- (\otherfollowing, -4\tagheight-2\tagsep);
\draw[cutinline, dashed] (\egovehiclefollowing, \cutinheight) -- (\egovehiclefollowing, -4\tagheight-2\tagsep);
\draw[cutinline, <->] (\otherfollowing, \arrowheight) -- (\egovehiclefollowing, \arrowheight);
\node[anchor=north west, text width=\egovehiclefollowing-\otherfollowing-1em, minimum width=\egovehiclefollowing-\otherfollowing, minimum height=\cutinheight-\arrowheight, align=center] at (\otherfollowing, \cutinheight) {Cut in};

% Timeline.
\draw[timeline, ->] (0, -4\tagheight-2\tagsep-\timepos) -> (\tagtotalwidth, -4\tagheight-2\tagsep-\timepos);
\node[minimum width=\tagtotalwidth, align=center, anchor=north west, minimum height=\timepos] at (0, -4\tagheight-2\tagsep) {Time};
\end{tikzpicture}
%}

	\caption{\cstartb Example of tags describing a cut in. Note that only the tags that are relevant for the cut in, as defined in \cref{tab:ngrams cutin}, are shown.\cendb}
	\label{fig:tags cut in}
\end{figure*}



