\section{Introduction}
\label{sec:introduction}

Objective/Research question of the paper: how to quantify safety risk associated with driving automated vehicles 

Suggested structure: 
\begin{itemize}
    \item Context: describe the context of the problem
    \item Background info: short intro to relevant topics: SOTIF, ISO 26262, Scenario based analysis, previous risk quantification efforts, 
    \item Gap: what is missing?  
    \item Contributions: what are we adding in this paper: Adding severity quantification, adding controllability quantification, adding triggering conditions, applying it for multiple relevant scenarios
    \item structure of the paper
\end{itemize}

SOTIF: Absence of unreasonable risk due to hazards resulting from functional inefficiencies of the intended functionality or from reasonably foreseeable misuse by persons.
Hazard identification and evaluation is part of the process to ensure safety according to ISO~21448.
This includes:
\begin{itemize}
    \item Determining the probability of exposure of the ``operational situation''.
    \item Determining the \emph{controllability}, which is defined as ``ability to avoid a specified harm or damage through the timely reactions of the persons involved, possibly with the support from external measures'' \autocite{ISO26262}.
\end{itemize}
