\section{Proposed Risk Estimation Method} % Arash & Hala
\label{sec:method}


\begin{enumerate}
	\item Back ground on risk (ISO 26262/SOTIF) and acceptable level of risk
	\item High level description of the method 
	\item explaining how assumptions are being made
\end{enumerate}


\begin{enumerate}
\item{Compare the risk of the platooning driving behavior ($R_p$)to the risk of human drivers ($R_h$) with respect to rear-end-collisions only. If $$R_p < \frac{1}{10} R_h, $$ the risk is assumed to be acceptable.}
\item{For calculating the risk of the autonomous driving behavior with respect to rear-end-collisions only, we list the relevant events as follows:
\begin{enumerate}
\item{E1: V2V Failure, for $t\in[t_{E1}, t_{E1}+\Delta t_{E1}]$}
\item{E2: Lead performs an emergency brake with $u_{L}(t)\leq u_1$ , for $t\in[t_{E2}, t_{E2}+\Delta t_{E2}]$}
\item{E3: The CACC controller of the follower vehicle reacts to the emergency brake (by braking) at $t\in[t_{E3}, t_{E3}+\Delta t_{E3}]$.}
\item{E4: Collision occurs with an impact speed higher than 10~kph.}
\end{enumerate}
}
\item{We calculate the total probability of these events to lead to a collision as:
\begin{equation}
P_{tot} = P(E1) \times P(E2|E1) \times P(E3|E2) \times P(E4|E3)
\end{equation}
Note that potentially the V2V failure will occur later than the brake action of the lead, (in which case $t_{E2}<t_{E1}$). However, since these events are independent, this does not have any consequences for the proposed approach. Also, due to its independency $ P(E2|E1)=P(E2)$.
Now, we assume that the controller will respond to it (although it may respond late, it will respond eventually), so $P(E3|E2)=1$. Further, let us assume that we can identify a convex set $\mathcal{S}$ for which all states lead to collision with an impact speed higher than 10~kph, than $P(E4|E3)=1$.  Now only the probability of states in this set needs to be calculated.
This collision set $\mathcal{S}$ includes states related to timings as well as initial conditions. We define state $x$ as
\begin{eqnarray*}
x&=&\{(t_{E1},\Delta t_{E1},u_{1},t_{E2},\Delta t_{E2},\\
&&d_i(t_{E2}),v_i(t_{E2}),a_i(t_{E2}), v_{i-1}(t_{E2}),a_{i-1}(t_{E2})) \}
\end{eqnarray*}
and $\mathcal{S} = \{x \in \mathbb{R}^9|x_L < x < x_U\}$ with $x_L \in \mathbb{R}^9$ the lower bound and $x_U \in \mathbb{R}^9$ the upper bound. }
\item{Now the risk can be calculated as $P_{tot} \times $...TODO? (ARASH)}
\end{enumerate}
