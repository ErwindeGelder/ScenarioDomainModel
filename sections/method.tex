\section{Proposed Risk Estimation Method} % Hala
\label{sec:method}

%In this section we discuss the following:
%\begin{enumerate}
%	\item High level description of the method 
%	\item calculating the probability of occurrence of a scenario based on its Events, Activities, and Conditions. This is in the context of the scenario that gives a specified temporal relation between its events and activities.
%	
%	We are not sure about the causal relation, but we have observed the temporal relation from data. 
%	 
%	\item Scenario Events $\rightarrow$ independent variables 
%	\item Scenario Activities $\rightarrow$ some are dependent on each other (how to distinguish dependency?)
%	\item Scenario Conditions $\rightarrow$ some activities/events may depend on these. (how to define the dependencies?)
%	\item We make assumptions about the dependencies, and later validate whether the assumption was justified based on measured data. 	
%	\item explaining how assumptions are being made based on data
%	\item give both formulas for the two cases of dependent and independent variables. (there may be a third mixed option?)
%	\item use the enumerators $i, j, k$ to make the formulas independent on the number of variables (activities, events, conditions)
%\end{enumerate}
%
%Note: We use the formula of independent variable in the case study. 
 
In the Hazard Analysis and Risk Assessment required by the ISO~26262 standard, the estimation of Automotive Safety Integrity Level (ASIL) is calculated based on a single specific hazardous event.
%/action (I need to add a concrete description of the event in this context). 
Although the operational situation in which this single event occurs as well as the operating mode are considered in the analysis, still the proceeding and successive events are not taken into account.
In this paper, we propose a new method to estimate the risk of a certain traffic scenario taking into account the whole chain of events and conditions occur during this scenario [ref-definition section].
 %class, i.e., the risk of a set of scenarios that are linguistically described similarly. cederation
 This method considers the conditions under which this scenario happened and activities that are performed by different road users, within this scenario, and their dependency on each other. This is in compliance with the analysis required by SOTIF (needs more explanation (ERWIN: Why not just a reference to SOTIF?)).  The estimated risk is based on real-world driving data. To estimate the risk, we will quantify the exposure and the severity.

\color{red}
\boldsymbol{This should move to conclusion section}
In the calculation of estimated risk many assumption made to simplify the calculation or because there are unknown due to lack of data. This affect/reduce the accuracy of the calculated/estimated risk. 
However, we still believe that the quantified risk is very valuable and makes the steps forward towards meeting the requirements of SOTIF standard , because of the following reasons:
\begin{itemize}
	\item all the assumptions that were made for estimating the risk are explicit. In contrast, when people assign ASIL levels to a hazardous event, often many assumptions are implicitly made. By making the assumptions explicit, it is much clearer why a certain risk is associated with --- in this case --- a certain/specific scenario.
	\item It provides a good estimate of the order of the risk for each scenario. For example, if the estimated risk of two different scenarios differ by a factor 10, it is still reasonable to argue that one scenario introduces a higher risk than the other, even though the factor 10 might not be exact.
	\item Because our/the proposed method explicates all the steps and assumptions that lead to a certain estimated risk, it is easily possible to update the risk when more information of the system is known or when more data is available.
\end{itemize}
\color{black}
As explained in the previous section [ref-section], a scenario consist of:
\begin{itemize}
	\item $A$: activities performed by the ego vehicle and other road users;
	\item $C$: conditions of the scenario.
\end{itemize}
Let $E$ denote the associated risk of a scenario. For this study we assume that $E$ can only results in two possibilities/categories, risky or safe scenarios. The level of severity of the risk is not taken into account. We formulate the exposure as the average number of occurrences of the activities $A$ under the conditions $C$, denoted by $\lambda_{A,C}$. The severity is denoted by the conditional probability $P(R|A,C)$, i.e., the likelihood of the severe outcome $R$ given the activities $A$ and the conditions $C$. The risk can be computed by combining these two numbers. The following steps are followed when computing the risk:
\begin{enumerate}
	\item Estimate the likelihood of the conditions $P(C)$.
	\item Estimate the exposure $\lambda_{A,C}$.
	\item Parametrize the scenario with a parameter vector $\theta$.
	\item Estimate the distribution of $\theta$ for this scenario class, i.e., $P(\theta|A,C)$.
	\item Use simulations to estimate the severity, i.e., $P(R|A,C)$.
	\item Compute the risk of a scenario class denoted by $\lambda$.
	\item Compute the number of hours that can be driven without any harm associated with the given scenario class.
\end{enumerate}

In the following sections, we will explain these steps one by one. 



\subsection{Likelihood of the conditions}
\label{sec:conditions}

Different instances for the same traffic scenario are subject to $n_C$ conditions, denoted by $C_1, \ldots, C_m$. For the sake of brevity, all conditions together are denoted by $C$, i.e., $P(C_1, \ldots, C_{n_C})=P(C)$. Many of these conditions might be based on the operational design domain\footnote{``Operating conditions under which a given driving automation system or feature thereof is specifically designed to function'' \cite{sea2018j3016}.} of the automation system and might include conditions with respect to the infrastructure, weather conditions, lighting conditions, and geographical locations. 

The first step is to compute the joint probability of the conditions, i.e., $P(C)$. In case these conditions are independent, the probability can be computed by simply multiplying the individual likelihoods for each condition, i.e., $P(C)=P(C_1)\cdot\ldots\cdot P(C_{n_C})$. This, however, might not necessarily be the case, which requires either to compute the joint probability or to compute conditional probabilities. In some cases, it might also be reasonable to simply assume that the likelihood of certain conditions are independent.

Note that the the defined conditions might not be the same as the conditions under which the data is collected that is used to compute $P(C)$. This might require additional assumptions, see our example in \cref{sec:example conditions}.



\subsection{Calculate exposure}
\label{sec:exposure}

To calculate the exposure, the average number of occurrences of the activities that constitute the scenarios that fall into the specified scenario class within a certain time interval need to be estimated. Let $n_A$ denote the number of activities, such that $A_1, \ldots, A_{n_A}$ denote the activities. For the sake of brevity, all activities together are denoted by $A$. 

Without loss of generality, we assume that the time interval is an hour. To estimate the number occurrences of the activities, the data for which the conditions $C$ are satisfied are analyzed. The average number of occurrences of the activities $A$ for each hour of driving for which the conditions $C$ are satisfied is denoted by $\lambda_{A|C}$. Next, we can calculate the average number of occurrences of the activities $A$ under the conditions $C$ for each hour of driving:
\begin{equation}
	\lambda_{A,C} = \lambda_{A|C} \cdot P(C).
\end{equation}

Regarding the scenarios that fall into the specified scenario class, we assume the following:
\begin{itemize}
	\item The occurrence of one scenario consisting of activities $A$ and conditions $C$ does not affect the probability that a second scenario consisting of activities $A$ and conditions $C$ occurs.
	\item The rate at which a scenario consisting of activities $A$ and conditions $C$ occurs is constant. I.e., $\lambda_{A,C}$ is constant.
	\item Two scenarios consisting of activities $A$ and conditions $C$ cannot occur at exactly the same time instant.
\end{itemize}
Based on these assumptions, the number of occurrences of scenarios consisting of activities $A$ and conditions $C$ is distributed according to the Poisson distribution:
\begin{equation}
	P(k\text{ times }A,C\text{ in an hour}) = \exp \left\{-\lambda_{A,C} \right\} \frac{\lambda_{A,C}^k}{k!}.
\end{equation}



\subsection{Parametrization}
\label{sec:parametrization}

The third step is to parametrize the scenarios with a parameter vector $\theta \in \mathbb{R}^d$. The parametrization enables the generation of infinitely many unique individual test cases that resemble the scenarios found in naturalistic driving \cite{deGelder2017assessment,elrofai2018scenario}.

In case the parameters are dependent, which is often the case, it is important that the number of parameters is limited to avoid the curse of dimensionality \cite{scott2015multivariate}. This often requires some assumptions. An example is presented in \cref{sec:example parametrization}.



\subsection{Estimation of the pdf}
\label{sec:pdf}

To estimate the probability density function (pdf) of the parameter vector $\theta$, i.e., $P(\theta|A,C)$, either parametric models, non-parametric models, or a combination of the two can be used. In case of parametric models, a certain functional form of the pdf is assumed. For example, it might be assumed that the pdf can be modeled using a Gaussian distribution. In this paper, we present a non-parametric approach using Kernel Density Estimation (KDE) \cite{rosenblatt1956remarks, parzen1962estimation}. Using KDE, there is no assumption on the functional form of the pdf because the shape of the pdf is automatically computed.

Using KDE, the estimated pdf is given by
\begin{equation}
	\label{eq:kde}
	P(\theta|A,C) = \frac{1}{nh^d} \sum_{i=1}^n K\left(\frac{\theta - \theta_i}{h}\right).
\end{equation}
Here, $K(\cdot)$ is an appropriate kernel function and $h$ denotes the bandwidth. From the data, $n$ scenarios are extracted and each scenario is parametrized with $\theta_i$. The choice of the kernel $K(\cdot)$ is not as important as the choice of the bandwidth $h$ \cite{turlach1993bandwidthselection}. Often, a Gaussian kernel is used, which is given by
\begin{equation}
	\label{eq:gaussian kernel}
	K(u) = \frac{1}{\left( 2\pi \right)^{d/2}} \exp \left\{ -\frac{1}{2} \|u\|^2 \right\},
\end{equation}
where $\|u\|^2$ denotes the squared 2-norm of $u$, i.e., $u^T u$.

The bandwidth $h$ controls the amount of smoothing. For the kernel of \cref{eq:gaussian kernel}, the same amount of smoothing is applied in every direction, although this can easily be extended to a multi-dimensional bandwidth, see, e.g., \cite{scott2005multidimensional, chen2017tutorial}. There are many different ways of estimating the bandwidth, ranging from simple reference rules like, e.g., Scott's rule of thumb \cite{scott2015multivariate} or Silverman's rule of thumb \cite{silverman1986density} to more elaborate methods; see \cite{turlach1993bandwidthselection, bashtannyk2001bandwidth, jones1996brief, chiu1996comparative} for reviews of different bandwidth selection methods. 



\subsection{Calculating severity}
\label{sec:simulations}

Let $R$ denote a harmful outcome of a scenario. We define the severity of a scenario with activities $A$ and conditions $C$ as the probability of $R$, given the activities $A$ and $C$, i.e., $P(R|A,C)$. We cannot evaluate $P(R|A,C)$ directly, because the outcome of a scenario highly depends on the parametrization $\theta$. Hence, $P(R|\theta,A,C)$ can be evaluated, for example, through a simulation of the scenario with parameters $\theta$. Using $P(\theta|A,C)$ from \cref{eq:kde}, we can compute 
\begin{equation} \label{eq:probability R theta}
	P(R,\theta|A,C) = P(R|\theta,A,C) \cdot P(\theta|A,C).
\end{equation}
To obtain $P(R|A,C)$, we need to integrate \cref{eq:probability R theta} over $\theta$, i.e., 
\begin{equation} \label{eq:probability R}
	P(R|A,C) = \int_{\mathbb{R}^d} P(R|\theta,A,C) \cdot P(\theta|A,C) \ud \theta.
\end{equation}

One approach to evaluate the integral of \cref{eq:probability R} is to perform Monte Carlo simulations. For sufficiently large $N$, we have
\begin{equation} \label{eq:monte carlo}
	P(R|A,C) \approx \frac{1}{N} \sum_{k=1}^N P(R|\theta_k,A,C), \, \theta_k \sim P(\theta|A,C).
\end{equation}

To improve the accuracy of \cref{eq:monte carlo}, importance sampling can be used where the parameters $\theta$ are drawn from another distribution with a focus on the critical scenarios, see, e.g., \cite{deGelder2017assessment}.



\subsection{Calculating the risk}
\label{sec:risk}

Analogous to the exposure, we define the risk as the number of occurrences of the harmful outcome $R$ in a scenario consisting of activities $A$ and conditions $C$ in a certain time interval. Let $\lambda$ denote the average number of these occurrences in an hour of driving. The chain rule of probability tells us that this equals the sum of $\lambda_{A,C}$ (i.e., the exposure) and $P(R|A,C)$ (i.e., the severity):
\begin{equation} \label{eq:risk}
	\lambda = \lambda_{A,C} \cdot P(R|A,C)
\end{equation}

Analogous to the number of occurrences of a scenario consisting of activities $A$ and conditions $C$, we assume that the number of occurrences of a harmful outcome $R$ in a scenario consisting of activities $A$ and conditions $C$ can be modeled using a Poisson distribution:
\begin{equation} \label{eq:poisson risk}
	P(k\text{ times }R,A,C\text{ in an hour}) = \exp \left\{ -\lambda \right\} \frac{\lambda^k}{k!}.
\end{equation}


\subsection{Likelihood of no harm}
\label{sec:no harm}

Using \cref{eq:poisson risk}, to calculate the probability of not having the harmful outcome $R$ in a scenario consisting of activities $A$ and conditions $C$ we simply need to use $k=0$:
\begin{equation} \label{eq:no harm}
	P(\text{no }R,A,C\text{ in one hour}) = \exp \left\{ -\lambda \right\}.
\end{equation}


%Let $E$ be the final event that might result in a risky situation/harm. The probability $P$ of the exposure of a harm/risk in a certain scenario is then given by 
%\begin{equation}
%P(E,A_1,A_2,...A_n,C_1,C_2,..,C_m)=P(E,A,C) \label{eq:secM1}
%\end{equation}
%where $n$ is the number of performed activities within the scenario and $m$ is the number of conditions in the same scenario.
%Since the harm event $E$ depends on the performed actives and scenario conditions, to compute \ref{eq3} requires computing $P(A,C)$ first.
%\begin{equation}
%P(A,C)=P(A|C) \cdot P(C) \label{eq:secM2}
%\end{equation}
