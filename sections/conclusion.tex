\section{Conclusions}
\label{sec:conclusions}

\cstarte
% Summarize two-step approach: tagging and mining based on tags.
For the scenario-based assessment of automated vehicles, scenarios captured from real-world data collected on the level of individual vehicles can be used to define the tests.
We have proposed a two-step approach for mining real-world scenarios from a data set.
The first step is to provide the data with tags that describe, e.g., the activities of the different actors.
The second step is to mine the scenarios by searching for particular combinations of tags.
We illustrated the approach with two examples, i.e., a cut in and an overtaking before a lane change.

% Improve tagging, e.g., using machine learning techniques.
% Have more tags!
Future work is to include more tags, e.g., ``turning left'' or ``turning right'', and to consider more actors, e.g., pedestrians and cyclists. This will enable the mining of many more scenarios. 
% Use n-grams for mining scenarios. New possibilities: correcting mistakes, graphical representation, test case generation.
Furthermore, the analogy between the proposed scenario mining --- searching for certain combinations of tags --- and natural language processing (NLP) --- searching for certain combinations of words --- will be explored in future research. 
In NLP, n-gram models are successfully used to correct \autocite{hull1982experiments} and predict \autocite{brown1992class} words and to generate text \autocite{oh2002stochastic}, so n-gram models might be used to correct and predict tags and to generate new scenarios for the assessment of automated vehicles.
\cende
