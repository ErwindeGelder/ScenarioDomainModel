\section{Case Study} % Erwin & Hala
%\label{sec:example}

In this section, we present a case study to illustrate the method of quantifying the risk for a cut-in scenario. We will first describe the cut-in scenario and the use case. The actual system for which the risk is computed is presented in next. After these two steps, we will go through all the steps that are presented in our method.



\subsection{The cut-in scenario and the use case}
%\label{sec:scenario class}

We want to quantify the risk for cut-in scenarios that are linguistically described as follows: while the ego vehicle drives at a moderate to high speed while staying in its lane, another vehicle cuts into the lane of the ego vehicle, such that this vehicle becomes the ego vehicle's lead vehicle. Furthermore, the ego vehicle needs to brake to prevent a collision.

For the quantification of the risk, 60 hours of data (see also \cite{deGelder2017assessment}) are collected by driving a specific route in and between Eindhoven and Helmond, The Netherlands, with twenty different drivers, each driving the route twice. Therefore, it is assumed that the use case of the AD system is driving this route. We will use the data for the estimation of the risk. Hence, we will make use of the following assumption:
\begin{assumption}
	The recorded naturalistic driving data is representative for what a vehicle with the AD system might encounter along the same route.
\end{assumption}



\subsection{System-under-test}
%\label{sec:system}

To reduce efforts for the assessment, often simulations are employed. However, even simulations can consume considerable time, as these simulations might run real-time \cite{shah2018airsim} or slower when a higher level of detail is used \cite{zofka2016testing}. For our method, we will simplify the simulations, such that the total required time on a common computer is in the order of minutes. Since we are interested in approximate results, a high level of detail is not required. 

To simplify the system-under-test, it is assumed that the system's desired acceleration is similar to the adaptive cruise control defined in \cite{deGelder2017assessment}, i.e.,
\begin{equation}
	\label{eq:desired acceleration} 
	u(t) = k_{\mathrm{d}}(v(t))(d(t) - \tau_{\mathrm{h}} v(t) - s_0) + k_{\mathrm{v}}\left(\dot{d}(t) - ha(t) \right),
\end{equation}
with
\begin{equation}
	\label{eq:gain}
	k_{\mathrm{d}}(v(t)) = k_{\mathrm{d1}} + \left( k_{\mathrm{d2}} - k_{\mathrm{d1}} \right) \exp \left\{ -\frac{v(t)^2}{2\sigma_{\mathrm{d}}} \right\}.
\end{equation}
Here, $v$ is the speed of the ego vehicle, $d$ denotes the clearance between the ego vehicle and its predecessor, i.e., the vehicle that performs the cut-in. The relative speed is denoted by $\dot{d}$ and $a$ refers to the acceleration of the ego vehicle. The ego vehicle is modeled using a first order model with a time delay, i.e.,
\begin{equation}
	\label{eq:vehicle model}
	\tau \dot{a}(t) + a(t) = u(t - \theta).
\end{equation}
Furthermore, the deceleration is limited at $\unit[-6]{ms^{-2}}$. A description of the constants of \cref{eq:desired acceleration,eq:gain,eq:vehicle model} are listed in \cref{tab:constants}. The controller runs at \unit[100]{Hz}.

\begin{table}
	\centering
	\caption{The constants used for the simple automation system of \cref{eq:desired acceleration,eq:gain,eq:vehicle model}.}
	\label{tab:constants}
	\begin{tabular}{clc}
		\toprule
		Parameter & Description & Value \\ \otoprule
		$\tau_{\mathrm{h}}$ & Desired headway time & \unit[2.0]{s} \\
		$s_0$ & Safety distance & \unit[1.5]{m} \\
		$k_{\mathrm{d1}}$ & Distance gain at high speed & $\unit[0.7]{s^{-2}}$ \\
		$k_{\mathrm{d1}}$ & Distance gain at low speed & $\unit[2.0]{s^{-2}}$ \\
		$\sigma_{\mathrm{d}}$ & Shaping coefficient of distance gain & $\unit[5]{ms^{-1}}$ \\
		$k_{\mathrm{v}}$ & Speed difference gain & $\unit[0.35]{s^{-1}}$ \\
		$\tau$ & Time constant of vehicle model & \unit[0.1]{s} \\
		$\theta$ & Delay of the vehicle response & \unit[0.2]{s} \\
		\bottomrule
	\end{tabular}
\end{table}

Note that there is no intervention of a human:
\begin{assumption}
	The ego vehicle is fully controlled by the automation system as defined by \cref{eq:desired acceleration,eq:gain}. Hence, there is no intervention of a human.
\end{assumption}



\subsection{Calculate exposure}
%\label{sec:example exposure}

The cut-in scenarios are subject to the following conditions:
\begin{itemize}
	\item $C_1$: The speed of the ego vehicle is within the range of \unit[60]{km/h} and \unit[130]{km/h}.
	\item $C_2$: There are no restrictions on the weather conditions.
	\item $C_3$: There are no restrictions on the lighting conditions.
\end{itemize}

Obviously, because there are no restrictions to the weather and lighting conditions, we have $P(C_2,C_3)=1$. For the first condition, we can use the data to estimate the likelihood. The data, however, has been recorded during sunny weather at daylight. Therefore, we need to following assumption.

\begin{assumption} \label{asm:conditions}
	Let $C_2'$ and $C_3'$ denote the conditions of having sunny weather and daylight, respectively. Then we have $P(C_1|C_2,C_3)=P(C_1|C_2',C_3')$.
\end{assumption}

From the data, it appeared that $P(C_1|C_2',C_3')=0.20$. Using \cref{asm:conditions}, we have
\begin{equation}
	P(C) = P(C_1,C_2,C_3) = P(C_1|C_2',C_3')\cdot P(C_2,C_3) = 0.20.
\end{equation}

The cut-in scenarios consist of the following activities:
\begin{itemize}
	\item $A_1$: The ego vehicle is lane following.
	\item $A_2$: The target vehicle is driving in an adjacent lane in the same direction as the ego vehicle.
	\item $A_3$: After activity $A_2$, the target vehicle performs a lane change towards the lane of the ego vehicle, such that the ego vehicle needs to brake.
	\item $A_4$: The automation system detects the cut-in.
	\item $A_5$: After activity $A_4$, the automation system activates the brakes of the ego vehicle.
\end{itemize}

The likelihood of the activities $A_1$, $A_2$, and $A_3$ can be estimated using the data. It is assumed that the ego vehicle needs to brake if the target vehicle is driving slower and the headway time is less than three seconds. In case of a slower target vehicle with a larger headway time, the scenario is referred to as a gap closing scenario \cite{semsarkazerooni2016cacc, gelder2016pacc}.

For simplicity, we assume the following:
\begin{assumption}
	The automation system always detects the cut-in and activates the brakes after detecting the cut-in, such that $P(A_4,A_5|A_1,A_2,A_3,C) = 1$.
\end{assumption}

Using this assumption, we can compute $\lambda_{A|C}$ by detecting the number of occurrences of the activities $A_1$, $A_2$, and $A_3$ under the conditions $C$. Based on the dataset, we have $\lambda_{A|C}=\unit[9.9]{h^{-1}}$, i.e., in each hour that the ego vehicle is driving in a speed range of \unit[60]{km/h} and \unit[130]{km/h}, there are on average $9.9$ cut-ins with the target vehicle driving slower than the ego vehicle, such that the headway time after the cut-in is less than three seconds. From this, it simply follows that
\begin{equation}
	\lambda_{A,C} = \lambda_{A|C} \cdot P(C) = 2.0.
\end{equation}



\subsection{Calculating severity}
%\label{sec:example severity}

To limit the number of parameters, we assume the following:
\begin{assumption}
	The ego vehicle is driving at a constant speed at the moment of the cut-in of the target vehicle, i.e., the moment that the target vehicle enters the lane of the ego vehicle.
\end{assumption}
\begin{assumption}
	The target vehicle is driving at a constant speed.
\end{assumption}
Both assumptions can be justified using the data. In case of the ego vehicle, the average acceleration at the moment of the cut-in is $\unit[-0.29]{ms^{-2}}$ and the standard deviation equals $\unit[0.50]{ms^{-2}}$. In case of the target vehicle, the average deceleration at the moment of the cut-in is $\unit[0.05]{ms^{-2}}$ and the standard deviation equals $\unit[0.37]{ms^{-2}}$. As a result, the scenario is parametrized using $d=3$ parameters:
\begin{enumerate}
	\item The clearance between the target vehicle and the ego vehicle at the moment of the cut-in, i.e., the moment than the target vehicle enters the lane of the ego vehicle.
	\item The speed of the ego vehicle at the moment of the cut-in.
	\item The speed of the target vehicle throughout the whole scenario.
\end{enumerate}


A histogram of the data of the parameters is shown in \cref{fig:histogram}. The probability density function is estimated using the KDE of \cref{eq:kde} with the Gaussian kernel of \cref{eq:gaussian kernel}. Before applying KDE, the data is scaled, such that the standard deviation equals one for each parameter. We use leave-one-out cross validation to compute the bandwidth $h$ (see also \cite{duin1976parzen}) because this minimizes the Kullback-Leibler divergence between the real underlying pdf and the estimated pdf \cite{turlach1993bandwidthselection,zambom2013review}. The resulting bandwidth equals $h=0.198$. The marginal probability distributions coming from the resulting joint distribution, i.e. the KDE, are shown in \cref{fig:histogram} by the black lines.

\setlength\figurewidth{0.5\linewidth}
\setlength\figureheight{0.25\linewidth}
\begin{figure}
	\centering
	% This file was created by matplotlib2tikz v0.6.17.
\begin{tikzpicture}

\begin{groupplot}[group style={group size=1 by 3}]
\nextgroupplot[
xlabel={Clearance [m]},
ylabel={Density},
xmin=8.12085897227673, xmax=100.239482786353,
ymin=0, ymax=0.0358233749416452,
width=\figurewidth,
height=\figureheight,
tick align=outside,
tick pos=left,
x grid style={white!69.01960784313725!black},
y grid style={white!69.01960784313725!black},
axis x line*=bottom,
axis y line*=left,
scaled y ticks = false,
y tick label style={/pgf/number format/fixed}
]
\draw[draw=black,fill=gray] (axis cs:12.3080691456438,0) rectangle (axis cs:16.4952793190109,0.0120414705686202);
\draw[draw=black,fill=gray] (axis cs:16.4952793190109,0) rectangle (axis cs:20.682489492378,0.0100345588071835);
\draw[draw=black,fill=gray] (axis cs:20.682489492378,0) rectangle (axis cs:24.8696996657451,0.0260898528986771);
\draw[draw=black,fill=gray] (axis cs:24.8696996657451,0) rectangle (axis cs:29.0569098391121,0.034117499944424);
\draw[draw=black,fill=gray] (axis cs:29.0569098391122,0) rectangle (axis cs:33.2441200124792,0.0160552940914936);
\draw[draw=black,fill=gray] (axis cs:33.2441200124792,0) rectangle (axis cs:37.4313301858463,0.0180622058529303);
\draw[draw=black,fill=gray] (axis cs:37.4313301858463,0) rectangle (axis cs:41.6185403592134,0.0220760293758037);
\draw[draw=black,fill=gray] (axis cs:41.6185403592134,0) rectangle (axis cs:45.8057505325805,0.0100345588071835);
\draw[draw=black,fill=gray] (axis cs:45.8057505325805,0) rectangle (axis cs:49.9929607059476,0.0160552940914936);
\draw[draw=black,fill=gray] (axis cs:49.9929607059476,0) rectangle (axis cs:54.1801708793146,0.0220760293758038);
\draw[draw=black,fill=gray] (axis cs:54.1801708793146,0) rectangle (axis cs:58.3673810526817,0.0120414705686202);
\draw[draw=black,fill=gray] (axis cs:58.3673810526817,0) rectangle (axis cs:62.5545912260488,0.0080276470457468);
\draw[draw=black,fill=gray] (axis cs:62.5545912260488,0) rectangle (axis cs:66.7418013994159,0.00401382352287341);
\draw[draw=black,fill=gray] (axis cs:66.7418013994159,0) rectangle (axis cs:70.929011572783,0.00602073528431012);
\draw[draw=black,fill=gray] (axis cs:70.929011572783,0) rectangle (axis cs:75.1162217461501,0.0060207352843101);
\draw[draw=black,fill=gray] (axis cs:75.1162217461501,0) rectangle (axis cs:79.3034319195171,0.00401382352287341);
\draw[draw=black,fill=gray] (axis cs:79.3034319195171,0) rectangle (axis cs:83.4906420928842,0.00401382352287341);
\draw[draw=black,fill=gray] (axis cs:83.4906420928842,0) rectangle (axis cs:87.6778522662513,0.0020069117614367);
\draw[draw=black,fill=gray] (axis cs:87.6778522662513,0) rectangle (axis cs:91.8650624396184,0);
\draw[draw=black,fill=gray] (axis cs:91.8650624396184,0) rectangle (axis cs:96.0522726129855,0.0060207352843101);
\addplot [very thick, black, forget plot]
table {%
8.12085897227673 0.00183504532546135
10.0008308868497 0.00339657358685259
11.8808028014227 0.00528599275795651
13.7607747159957 0.00725825955475721
15.6407466305686 0.00939473355010381
17.5207185451416 0.0121118079549573
19.4006904597146 0.0156352071526778
21.2806623742876 0.019524214372825
23.1606342888605 0.0228741245174167
25.0406062034335 0.0249480173856481
26.9205781180065 0.0255164505880221
28.8005500325795 0.0247790331703545
30.6805219471524 0.0232196174226075
32.5604938617254 0.0214346789684533
34.4404657762984 0.0198361008585762
36.3204376908714 0.0184565224694155
38.2004096054443 0.0171178383723391
40.0803815200173 0.0158408817383565
41.9603534345903 0.0150085292584412
43.8403253491633 0.0149883504483441
45.7202972637363 0.0156889924355427
47.6002691783092 0.016633089010273
49.4802410928822 0.0172824782581887
51.3602130074552 0.0171709350329699
53.2401849220281 0.0160353433185709
55.1201568366011 0.0140440303580629
57.0001287511741 0.0117399946150153
58.8801006657471 0.00966924155768559
60.7600725803201 0.00812493023353695
62.640044494893 0.00714814073279193
64.520016409466 0.00659055446858629
66.399988324039 0.00619392002395415
68.279960238612 0.00576814790188572
70.1599321531849 0.00533478495791458
72.0399040677579 0.00502884268136011
73.9198759823309 0.00488620187167895
75.7998478969039 0.00480058510610992
77.6798198114768 0.00463086872766452
79.5597917260498 0.00426990390723946
81.4397636406228 0.00367352208580758
83.3197355551958 0.002911397952567
85.1997074697687 0.00217651417425062
87.0796793843417 0.00170433569140452
88.9596512989147 0.00163829842167866
90.8396232134877 0.00190499378008258
92.7195951280606 0.00221235703924666
94.5995670426336 0.00224116277006498
96.4795389572066 0.00188363475625321
98.3595108717796 0.00129365752109914
100.239482786353 0.000721425131598922
};
\nextgroupplot[
xlabel={Ego vehicle speed [km/h]},
ylabel={Density},
xmin=56.8026, xmax=132.1614,
ymin=0, ymax=0.0463664183487372,
width=\figurewidth,
height=\figureheight,
tick align=outside,
tick pos=left,
x grid style={white!69.01960784313725!black},
y grid style={white!69.01960784313725!black},
axis x line*=bottom,
axis y line*=left,
scaled y ticks = false,
y tick label style={/pgf/number format/fixed}
]
\draw[draw=black,fill=gray] (axis cs:60.228,0) rectangle (axis cs:63.6534,0.0294389957769761);
\draw[draw=black,fill=gray] (axis cs:63.6534,0) rectangle (axis cs:67.0788,0.044158493665464);
\draw[draw=black,fill=gray] (axis cs:67.0788,0) rectangle (axis cs:70.5042,0.0367987447212201);
\draw[draw=black,fill=gray] (axis cs:70.5042,0) rectangle (axis cs:73.9296,0.0220792468327321);
\draw[draw=black,fill=gray] (axis cs:73.9296,0) rectangle (axis cs:77.355,0.00981299859232536);
\draw[draw=black,fill=gray] (axis cs:77.355,0) rectangle (axis cs:80.7804,0.0122662482404067);
\draw[draw=black,fill=gray] (axis cs:80.7804,0) rectangle (axis cs:84.2058,0.00981299859232532);
\draw[draw=black,fill=gray] (axis cs:84.2058,0) rectangle (axis cs:87.6312,0.00735974894424402);
\draw[draw=black,fill=gray] (axis cs:87.6312,0) rectangle (axis cs:91.0566,0.0171727475365694);
\draw[draw=black,fill=gray] (axis cs:91.0566,0) rectangle (axis cs:94.482,0.022079246832732);
\draw[draw=black,fill=gray] (axis cs:94.482,0) rectangle (axis cs:97.9074,0.0196259971846507);
\draw[draw=black,fill=gray] (axis cs:97.9074,0) rectangle (axis cs:101.3328,0.00735974894424402);
\draw[draw=black,fill=gray] (axis cs:101.3328,0) rectangle (axis cs:104.7582,0.00735974894424402);
\draw[draw=black,fill=gray] (axis cs:104.7582,0) rectangle (axis cs:108.1836,0.014719497888488);
\draw[draw=black,fill=gray] (axis cs:108.1836,0) rectangle (axis cs:111.609,0.014719497888488);
\draw[draw=black,fill=gray] (axis cs:111.609,0) rectangle (axis cs:115.0344,0.00245324964808134);
\draw[draw=black,fill=gray] (axis cs:115.0344,0) rectangle (axis cs:118.4598,0.00735974894424402);
\draw[draw=black,fill=gray] (axis cs:118.4598,0) rectangle (axis cs:121.8852,0.00245324964808133);
\draw[draw=black,fill=gray] (axis cs:121.8852,0) rectangle (axis cs:125.3106,0.00245324964808133);
\draw[draw=black,fill=gray] (axis cs:125.3106,0) rectangle (axis cs:128.736,0.00245324964808134);
\addplot [very thick, black, forget plot]
table {%
56.8026 0.0051251569966451
58.3405346938776 0.00948273774153526
59.8784693877551 0.0152737911443766
61.4164040816326 0.0216226093112603
62.9543387755102 0.0272307114498158
64.4922734693878 0.0310165077928978
66.0302081632653 0.0326451670840618
67.5681428571429 0.0324529962834777
69.1060775510204 0.0308932664673115
70.644012244898 0.0281869811378229
72.1819469387755 0.0245386832434187
73.7198816326531 0.0204882159449357
75.2578163265306 0.0168287779986908
76.7957510204082 0.0141397740791014
78.3336857142857 0.0124556966394098
79.8716204081633 0.0114128586744221
81.4095551020408 0.0107144084500764
82.9474897959184 0.0104479311726113
84.4854244897959 0.0109388055897548
86.0233591836735 0.0123061387946482
87.561293877551 0.0142089192941123
89.0992285714286 0.0160347079489904
90.6371632653061 0.0172758605250283
92.1750979591837 0.0177022546970844
93.7130326530612 0.0172862137574363
95.2509673469388 0.0161263707827698
96.7889020408163 0.0144854204851373
98.3268367346939 0.0127945110615371
99.8647714285714 0.0115070745512157
101.402706122449 0.0109087496237425
102.940640816327 0.0110258700545376
104.478575510204 0.0116168743401693
106.016510204082 0.0122015258253593
107.554444897959 0.0122126107321703
109.092379591837 0.011316036665026
110.630314285714 0.00967556360656004
112.168248979592 0.00785465965638699
113.706183673469 0.00638972154271003
115.244118367347 0.00543292751134418
116.782053061224 0.00477235058212446
118.319987755102 0.0041313841930039
119.85792244898 0.00341979893967049
121.395857142857 0.00273966624795546
122.933791836735 0.00222136776855905
124.471726530612 0.00189080614806837
126.00966122449 0.00167050575312529
127.547595918367 0.00145897016799862
129.085530612245 0.00119601841250331
130.623465306122 0.000883660847297828
132.1614 0.000571013454877462
};
\nextgroupplot[
xlabel={Target vehicle speed [km/h]},
ylabel={Density},
xmin=48.2488141652528, xmax=129.081859577732,
ymin=0, ymax=0.0312190317572164,
width=\figurewidth,
height=\figureheight,
tick align=outside,
tick pos=left,
x grid style={white!69.01960784313725!black},
y grid style={white!69.01960784313725!black},
axis x line*=bottom,
axis y line*=left,
scaled y ticks = false,
y tick label style={/pgf/number format/fixed}
]
\draw[draw=black,fill=gray] (axis cs:51.9230435021837,0) rectangle (axis cs:55.5972728391146,0.0160097598754956);
\draw[draw=black,fill=gray] (axis cs:55.5972728391146,0) rectangle (axis cs:59.2715021760454,0.0274453026437067);
\draw[draw=black,fill=gray] (axis cs:59.2715021760454,0) rectangle (axis cs:62.9457315129763,0.029732411197349);
\draw[draw=black,fill=gray] (axis cs:62.9457315129763,0) rectangle (axis cs:66.6199608499072,0.0297324111973489);
\draw[draw=black,fill=gray] (axis cs:66.6199608499072,0) rectangle (axis cs:70.2941901868381,0.0251581940900645);
\draw[draw=black,fill=gray] (axis cs:70.2941901868381,0) rectangle (axis cs:73.968419523769,0.0114355427682111);
\draw[draw=black,fill=gray] (axis cs:73.9684195237689,0) rectangle (axis cs:77.6426488606998,0.0114355427682111);
\draw[draw=black,fill=gray] (axis cs:77.6426488606998,0) rectangle (axis cs:81.3168781976307,0.0091484342145689);
\draw[draw=black,fill=gray] (axis cs:81.3168781976307,0) rectangle (axis cs:84.9911075345616,0.0251581940900646);
\draw[draw=black,fill=gray] (axis cs:84.9911075345616,0) rectangle (axis cs:88.6653368714925,0.0137226513218533);
\draw[draw=black,fill=gray] (axis cs:88.6653368714925,0) rectangle (axis cs:92.3395662084233,0.0137226513218534);
\draw[draw=black,fill=gray] (axis cs:92.3395662084233,0) rectangle (axis cs:96.0137955453542,0.0137226513218533);
\draw[draw=black,fill=gray] (axis cs:96.0137955453542,0) rectangle (axis cs:99.6880248822851,0.0114355427682112);
\draw[draw=black,fill=gray] (axis cs:99.6880248822851,0) rectangle (axis cs:103.362254219216,0.0114355427682111);
\draw[draw=black,fill=gray] (axis cs:103.362254219216,0) rectangle (axis cs:107.036483556147,0.0091484342145689);
\draw[draw=black,fill=gray] (axis cs:107.036483556147,0) rectangle (axis cs:110.710712893078,0);
\draw[draw=black,fill=gray] (axis cs:110.710712893078,0) rectangle (axis cs:114.384942230009,0.0091484342145689);
\draw[draw=black,fill=gray] (axis cs:114.384942230009,0) rectangle (axis cs:118.059171566939,0.00228710855364223);
\draw[draw=black,fill=gray] (axis cs:118.059171566939,0) rectangle (axis cs:121.73340090387,0);
\draw[draw=black,fill=gray] (axis cs:121.73340090387,0) rectangle (axis cs:125.407630240801,0.00228710855364223);
\addplot [very thick, black, forget plot]
table {%
48.2488141652528 0.00168027351915601
49.8984681532626 0.00364136394523865
51.5481221412724 0.00688176107580989
53.1977761292821 0.0114489015470479
54.8474301172919 0.0168709546332398
56.4970841053017 0.0221330359214515
58.1467380933115 0.0260855665048589
59.7963920813213 0.0281384416069486
61.4460460693311 0.0285958499818234
63.0957000573408 0.0281924497114946
64.7453540453506 0.0273196705422846
66.3950080333604 0.0258020442276619
68.0446620213702 0.0233588397540983
69.6943160093799 0.0201327171673932
71.3439699973897 0.0167342119826331
72.9936239853995 0.013855278897072
74.6432779734093 0.0119505628514347
76.2929319614191 0.0112658440630234
77.9425859494289 0.0119182568524426
79.5922399374386 0.0137110725781274
81.2418939254484 0.0159345040339956
82.8915479134582 0.0176106969018885
84.541201901468 0.0181007199210612
86.1908558894778 0.0174565218739648
87.8405098774875 0.0161714322178921
89.4901638654973 0.0147087680617018
91.1398178535071 0.013322338242064
92.7894718415169 0.0121557097427847
94.4391258295267 0.0113256717531443
96.0887798175364 0.0108960916673038
97.7384338055462 0.0108124450380329
99.388087793556 0.0108638117514665
101.037741781566 0.0107358182498313
102.687395769576 0.0101646288594111
104.337049757585 0.00909804954681981
105.986703745595 0.00775174572590957
107.636357733605 0.00650285877305912
109.286011721615 0.00563802203977032
110.935665709624 0.00512673120774266
112.585319697634 0.00467017312959077
114.234973685644 0.00400580078752985
115.884627673654 0.00312601636465223
117.534281661664 0.00221697926245841
119.183935649673 0.00148313963112521
120.833589637683 0.00104082813978185
122.483243625693 0.000878756940049195
124.132897613703 0.000864231690922192
125.782551601712 0.000826393843663879
127.432205589722 0.000676661115011963
129.081859577732 0.000451217339567434
};
\end{groupplot}

\end{tikzpicture}
	\caption{Histogram of the data of the parameters (bars) and their estimated marginal probabilities (lines).}
	\label{fig:histogram}
\end{figure}


Let $R$ denote the result of having a collision. Given a certain parameter vector $\theta$, we have $P(R|\theta,A,C)=1$ if the outcome of the simulation is a collision and $P(R|\theta,A,C)=0$ otherwise. For the simulation, we used the forward Euler method with a step size of \unit[0.01]{s}, similar as the sample time of the controller. On a regular computer, approximately 2000 simulations are performed in a second. We performed a million simulations, i.e., $N=10^6$. In total, 28 simulations ended with a collision, thus, according to \cref{eq:monte carlo}, we have:
\begin{equation}
	P(R|A,C) = 2.8 \cdot 10^{-5}.
\end{equation}



\subsection{Calculating the risk}
%\label{sec:example risk}

Let $\lambda$ denote the average number of collisions with a cut-in scenario as described earlier along the specified route for a vehicle with the automation system as described above. Using \cref{eq:risk}, we have:
\begin{equation}
	\lambda = \lambda_{A,C} \cdot P(R|A,C) = \unit[5.5 \cdot 10^{-5}]{h^{-1}}.
\end{equation}

Using \cref{eq:no harm}, the probability of having no collision in a cut-in scenario as described above during an hour of driving is
\begin{equation}
	P(\text{no }R,A,C\text{ during an hour}) = 0.999945.
\end{equation}

By solving the Poisson distribution of \cref{eq:poisson risk} for $\lambda$ with $k=0$, we can also conclude that with \unit[95]{\%} certainty, there will be no collision in a cut-in scenario as described earlier when driving \unit[925]{h}.
