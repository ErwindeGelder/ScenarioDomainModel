\section{Problem formulation}
\label{sec:problem}

Before formulating the scenario mining problem, we distinguish quantitative scenarios from qualitative scenarios, using the definitions of \emph{scenario} and \emph{scenario category}:

\begin{definition}[Scenario \autocite{degelder2018ontology}]
	\label{def:scenario}
	A scenario is a quantitative description of the relevant characteristics of the ego vehicle, its activities and/or goals, its static environment, and its dynamic environment. In addition, a scenario contains all events that are relevant to the ego vehicle.
\end{definition}

\begin{definition}[Scenario category \autocite{degelder2018ontology}]
	\label{def:scenario category}
	A scenario category is a qualitative description of the ego vehicle, its activities and/or goals, its static environment, and its dynamic environment.
\end{definition}

As described in \cref{def:scenario}, events and activities are constituents of a scenario. In \autocite{degelder2018ontology}, the terms \emph{event} and \emph{activity} are defined as follows:
\begin{definition}[Event \autocite{degelder2018ontology}]
	\label{def:event}
	An event marks the time instant at which the system reaches a specified threshold or at which a mode transition occurs, such that before and after an event, the state vector of the system corresponds to two different modes.
\end{definition}
\begin{definition}[Activity \autocite{degelder2018ontology}]
	\label{def:activity}
	An activity quantitatively describes the time evolution of one or more state variables of an actor between two events.
\end{definition}


A scenario category is an abstraction of a scenario and, therefore, a scenario category comprises multiple scenarios \autocite{degelder2018ontology}. Our goal of the scenario mining is to find all scenarios in the data that a given scenario category comprises. Hence, the problem is as follows:

\emph{Given a scenario category, how to find all scenarios in a given dataset that the scenario category comprises?}


%\subsection{Scenario mining}
%\label{sec:scenario mining}
%
%Here, we explain why we need scenario mining and how others did this.
