\section{Problem formulation}
\label{sec:problem}

To formulate the scenario mining problem, we distinguish quantitative scenarios from qualitative scenarios, using the definitions of \emph{scenario} and \emph{scenario category}:

\begin{definition}[Scenario \autocite{degelder2018ontology}]
	\label{def:scenario}
	A scenario is a quantitative description of the relevant characteristics of the ego vehicle, its activities and/or goals, its static environment, and its dynamic environment. In addition, a scenario contains all events that are relevant to the ego vehicle.
\end{definition}

\begin{definition}[Scenario category \autocite{degelder2018ontology}]
	\label{def:scenario category}
	A scenario category is a qualitative description of the ego vehicle, its activities and/or goals, its static environment, and its dynamic environment.
\end{definition}

\cstarta
A scenario category is an abstraction of a scenario and, therefore, a scenario category comprises multiple scenarios \autocite{degelder2018ontology}.
For example, the scenario category ``cut in'' comprises all possible cut-in scenarios. 
Given such a scenario category, our goal is to find all corresponding scenarios. 
Hence, we define the scenario mining problem as follows:
\begin{problem}[Scenario mining]
	Given a scenario category, how to find all scenarios that correspond to this scenario category in a given dataset?
\end{problem}
\cenda

