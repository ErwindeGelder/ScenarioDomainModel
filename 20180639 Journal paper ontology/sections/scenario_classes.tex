\section{Scenario classes}
\label{sec:scenario classes}

% Introduce term scenario class (i.e. qualitative description of scenario)
In this section, the notion of scenario class is introduced. As proposed above, a scenario in the context of the performance assessment of an AV needs to be quantitative. However, a qualitative description of each scenario exists. 
A scenario class refers to the qualitative description of a scenario.
The qualitative description can be regarded as an abstraction of the quantitative scenario. Therefore, a scenario class refers to multiple scenarios with a common characteristic \cite{elrofai2018scenario}.

% What is the purpose of this?
% - Human interpretable
% - Group scenarios that are very similar --> Analysis is easier
% - Completeness
\cbstart
Introducing the concept of scenario classes brings the following benefits:
\begin{itemize}
	\item Whereas the quantitative description might be difficult to interpret as a human, a qualitative description is easily interpreted by a human.
	\item It enables to reference to a group of scenarios that have something in common. This makes communication much easier.
	\item It enables easier comparison of scenarios. For example, two scenarios that fall into the same scenario class may look more similar than two scenarios that do not fall into different scenario classes. This can be used to quantify the completeness of a set of scenarios, see \cite{degelder2019completeness} for more details on this.
\end{itemize}

\subsection{Scenario classes versus scenarios}
\label{sec:fall into method}

% Explain scenarios fall into scenario class
We describe the formal relation between a scenario and a scenario class with the verb ``to fall into''. If a specific scenario class is an abstraction of a specific scenario, then we say that the specific scenario falls into that specific scenario class. Multiple scenarios can fall into a scenario class. For example, consider the scenario classes with the names ``Day'' and ``Rain'', see \cref{fig:venn diagram scenario class}. All scenarios that occur during the day fall into the scenario class ``Day''. Similarly, all scenarios with rain fall into the scenario class ``Rain''.

\cbend
\setlength{\venncircle}{7em}
\begin{figure}
	\centering
	\begin{tikzpicture}
		\fill[red, fill opacity=0.5] (-\venncircle/2, 0) circle (\venncircle);
		\fill[green, fill opacity=0.5] (\venncircle/2, 0) circle (\venncircle);
		\draw (-\venncircle/2, 0) circle (\venncircle);
		\draw (\venncircle/2, 0) circle (\venncircle);
		
		\node[anchor=east](daylight) at (-4/3*\venncircle, 3/4*\venncircle) {Day};
		\draw (daylight) -- ({(-sqrt(3)/2-1/2)*\venncircle}, \venncircle/2);
		\node[anchor=west](rain) at (4/3*\venncircle, 3/4*\venncircle) {Rain};
		\draw (rain) -- ({(sqrt(3)/2+1/2)*\venncircle}, \venncircle/2);
		
		\node[text width=\venncircle, align=center] at (-\venncircle, 0) {Scenarios without rain during day};
		\node[text width=\venncircle, align=center] at (0, 0) {Scenarios with rain during day};
		\node[text width=\venncircle, align=center] at (\venncircle, 0) {Scenarios with rain during night};
	\end{tikzpicture}
	\caption{The red and green circles correspond to the scenario classes ``Day'' and ``Rain'', respectively. Scenarios that occur during the day with rain fall into both scenario classes ``Day'' and ``Rain''. The new scenario class ``Day and rain'' can be defined as the scenario class in which all scenarios that occur during the day with rain fall. The scenario class ``Day and rain'' is falls into the scenario classes ``Day'' and ``Rain''.}
	\label{fig:venn diagram scenario class}
\end{figure}
\cbstart

% Explain scenario can fall into multiple scenario classes
Whereas multiple scenarios can fall into one scenario class, it is also possible for one scenario to fall into multiple scenario classes. Continuing the example of \cref{fig:venn diagram scenario class}, a scenario with rain during the day falls into both the scenario classes ``Day'' and ``Rain''.

% Explain scenario class can fall into scenario classes
The verb ``to fall into'' is also used to describe the relation between two scenario classes. A scenario class A is said to fall into a scenario class B if all scenarios that fall into scenario class A also fall into scenario class B. 
For example, consider the scenario class ``Rain during day'', which is represented by the intersection of the red and green circles in \cref{fig:venn diagram scenario class}. All scenarios with rain that occur during the day fall into this scenario class. Obviously, these scenarios also fall into the scenario classes ``Day'' and ``Rain''. Hence, the scenario class ``Rain during day'' falls into the scenario classes ``Day'' and ``Rain''.

\subsection{Tags}
\label{sec:tags}

When providing extra information to a piece of data, often tags are used \cite{smith2007tagging}. A tag is a keyword or a term that helps describing an item. For example, items in a database can contain some tags that enables users to quickly obtain several items that share a certain characteristic described by a tag \cite{craft2004tagging, vasquez2019controlling}. Applications are very broad, e.g., from classification of audio data \cite{kong2017joint} and capturing musical characteristics from songs \cite{ellis2011semantic} to tagging of Wikipedia pages \cite{voss2006collaborative}.

It is proposed to provide scenarios and scenario classes with applicable tags that describe the scenario in a qualitative manner. The tags of a scenario determines which scenario classes the scenario falls into. For example, if a scenario occurred during the day, it will contain the tag ``Day''. As a result, the scenario is automatically recognized as an instance of the scenario class ``Day'' which is characterized by the single tag ``Day''. The use of these tags brings some benefits:
\begin{itemize}
	\item The scenarios do not need to be directly classified. This can be a time consuming effort if the number of scenario classes is high.
	\item If a scenario class that only contains known tags is added to the database of scenario classes, it can be easily seen which scenarios fall into this scenario class by only inspecting the tags of the scenarios.
	\item It is easy to select scenarios from a scenario database or a scenario library by using tags or a combination of tags.
\end{itemize}

There is a balance between having generic scenario classes - and thus a high variety among the scenarios belonging to the scenario class - and having specific scenario classes without much variety among the scenarios in the scenario class. For some systems, one is interested in very specific set of scenarios, while for another system one might be interested in a set of scenarios with a high variety. To accommodate this, tags are structured in hierarchical trees \cite{molloy2017dynamic, badger2012dynamic}. The different layers of the trees can be regarded as different abstraction levels \cite{Bonnin2014}. 

In \cite{degelder2019scenarioclasses}, several trees of tags are defined and \cref{fig:tree vehicle activities} shows two examples of trees of tags taken from \cite{degelder2019scenarioclasses}. These tags describe possible activities of a vehicle, i.e., the lateral motion control (via steering) and longitudinal motion control (via acceleration and deceleration) are reflected into tags. The tags may refer to the objective of the ego vehicle in case no activities are defined. For example, a test case in which the ego vehicle's objective is to make a left turn, the tags ``Turning'' and ``Left'' are applicable. 

\cbend
\begin{figure*}
	\centering
	\begin{subfigure}{\linewidth}
		\centering
		\tree{Vehicle lateral activity}{Going straight; Changing lane, Left, Right; Turning, Left, Right; Swerving, Left, Right}
		\caption{Lateral activities of a vehicle.}
		\label{fig:tree vehicle lat act}
	\end{subfigure}
	\begin{subfigure}{\linewidth}
		\centering
		\tree{Vehicle longitudinal activity}{Reversing; Standing still; Driving forward, Braking, Cruising, Accelerating}
		\caption{Longitudinal activities of a vehicle.}
		\label{fig:tree vehicle long act}
	\end{subfigure}
	\caption{Tags for lateral and longitudinal activities of a vehicle. The lateral activity is relative to the lane in which the corresponding vehicle is driving. For example, if the vehicle is driving on a curved road, its lateral activity is ``Going straight''. Figure is taken from \cite{degelder2019scenarioclasses}.}
	\label{fig:tree vehicle activities}
\end{figure*}
\cbstart

Four different types of activities are identified regarding the lateral movement, see \cref{fig:tree vehicle lat act}. Here, it is assumed that ``Lateral'' refers to the direction perpendicular to the lane the vehicle is driving in (e.g., according to the Road Coordinate System in \cite{zofka2015datadrivetrafficscenarios}). Therefore, if the vehicle is driving on a curved road while staying more or less in its lane (lane-following), the tag ``Going straight'' is applicable. When the vehicle changes lane to an adjacent lane, the tag ``Changing lane'' is applicable. The tag ``Turning'' is applicable when the vehicle turns at a junction. The tag ``Swerving'' is applicable when the vehicle significantly changes lateral position without performing a complete lane change. For example, when the vehicle overtakes a cyclist that is riding at one side of the lane, the vehicle might swerve to the other side of the lane.

\cbend
