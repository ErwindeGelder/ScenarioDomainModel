\section{Preliminary results}
\label{sec:results}

I mainly worked on defining an appropriate ontology regarding scenarios and quantifying completeness. 

\subsection{Ontology of scenario for the assessment of automated vehicles}
\label{sec:ontology}

The notion of scenario is frequently used in the context of automated driving \cite{putz2017pegasus, roesener2017comprehensive, gietelink2006development, hulshof2013autonomous, karaduman2013interactivebehavior, englund2016grand, xu2002effects, ebner2011identifying, ploeg2017GCDC, zofka2015datadrivetrafficscenarios}, despite the fact that an explicit definition is often not provided. However, as mentioned by various authors \cite{stellet2015taxonomy, alvarez2017prospective, zofka2015datadrivetrafficscenarios, aparicio2013pre, lesemann2011test, putz2017pegasus, geyer2014, ulbrich2015}, using a scenario in the context of the development or assessment of AVs requires a clear definition of a scenario. To this end, few definitions of a scenario in the context of (automated) driving have been proposed \cite{geyer2014, ulbrich2015, elrofai2016scenario}. For the context of the scenario database that we aim for \cite{elrofai2018scenario}, however, a more concrete definition of a scenario is required to minimize any ambiguity regarding the scenarios. Ultimately, an ontology is desired to directly use the definitions for the scenario database.

Part of the ontology are the definitions of the different concepts that are adopted. For defining the notion of a scenario, the following characteristics are considered:
\begin{enumerate}
	\item A scenario corresponds to a time interval \cite{go2004blind, geyer2014, ulbrich2015, elrofai2016scenario}.
	\item A scenario consists of one or several events \cite{vannotten2003updated, go2004blind, geyer2014, ulbrich2015, kahn1962, englund2016grand, schoemaker1993multiple, cuppens2002alert, bach2016modelbased}.
	\item Real-world traffic scenarios are quantitative scenarios.
	\item The time interval of a scenario contains all relevant events \cite{geyer2014}.
	\item A scenario can contain goal(s) of one or multiple actors \cite{geyer2014, ulbrich2015, elrofai2016scenario}.
	\item A scenario includes the description of the environment \cite{geyer2014, ulbrich2015, elrofai2016scenario, ebner2011identifying, schuldt2013effiziente, althoff2017CommonRoad}.
\end{enumerate}

Hence, a scenario is defined as follows.
\begin{definition}[Scenario]\label{def:scenario}
	A scenario is a quantitative description of the ego vehicle, its activities and/or goals, its dynamic environment (consisting of traffic environment and conditions) and its static environment. From the perspective of the ego vehicle, a scenario contains all relevant events.
\end{definition}

For the ontology, the other notions, such as ego vehicle, activity, dynamic environment, static environment, conditions, and events, need to be defined as well. However, this is out of scope for this report.

Although a scenario is a quantitative description, there also exists a qualitative description of a scenario. We refer to the qualitative description of a scenario as a \emph{scenario class}. The qualitative description can be regarded as an abstraction of the quantitative scenario.

Scenarios are instances of scenario classes. A scenario class can contain multiple scenarios. On the other hand, a scenario may belong to one or multiple scenario classes. As an example, consider the scenario class ``Day'', which contains all scenarios that occur during the day, and the scenario class ``Rain'', which contains all scenarios with rain, see \cref{fig:venn diagram scenario class}. A scenario that occurs during the night without rain does not belong to any of the previously defined scenario classes. Likewise, a scenario that occurs during the day with rain belongs to both scenario classes ``Day'' and ``Rain''. 

\setlength{\venncircle}{10em}
\begin{figure}
	\centering
	\begin{tikzpicture}
		\fill[red, fill opacity=0.5] (-\venncircle/2, 0) circle (\venncircle);
		\fill[green, fill opacity=0.5] (\venncircle/2, 0) circle (\venncircle);
		\draw (-\venncircle/2, 0) circle (\venncircle);
		\draw (\venncircle/2, 0) circle (\venncircle);
		
		\node[anchor=east](daylight) at (-4/3*\venncircle, 3/4*\venncircle) {Day};
		\draw (daylight) -- ({(-sqrt(3)/2-1/2)*\venncircle}, \venncircle/2);
		\node[anchor=west](rain) at (4/3*\venncircle, 3/4*\venncircle) {Rain};
		\draw (rain) -- ({(sqrt(3)/2+1/2)*\venncircle}, \venncircle/2);
		
		\node[text width=\venncircle, align=center] at (-\venncircle, 0) {Scenarios without rain during day};
		\node[text width=\venncircle, align=center] at (0, 0) {Scenarios with rain during day};
		\node[text width=\venncircle, align=center] at (\venncircle, 0) {Scenarios with rain during night};
	\end{tikzpicture}
	\caption{The two circles correspond to the two scenario classes ``Day'' and ``Rain'', respectively. Scenarios that occur during the day with rain belong to both scenario classes ``Day'' and ``Rain''. The new scenario class ``Day and rain'' can be defined as the class that contains all scenarios that occur during the day with rain. The scenario class ``Day and rain'' is a subclass of the scenario classes ``Day'' and ``Rain''.}
	\label{fig:venn diagram scenario class}
\end{figure}

A scenario class can be a subclass of another scenario class. For example, when we continue our previous example and consider the scenario class ``Day and rain'', this scenario class is a subclass of the scenario classes ``Day'' and ``Rain''. Also, a scenario that occurs during the day with rain now belongs to three scenario classes: ``Day'', ``Rain'', and ``Day and rain''.

\subsection{Quantification of completeness}
\label{sec:completeness}

To draw conclusions on how an AV would perform in real-world traffic, it is necessary to know how representative the scenario database, that is used for the scenario-based assessment of the AV, is. Therefore, it is important to quantify how complete the scenario database is \cite{geyer2014, alvarez2017prospective, stellet2015taxonomy}. To quantify how complete a scenario database is, it is assumed that a scenario is defined according to \cref{def:scenario} and that a scenario class refers to a qualitative description of a scenario, such as described in \cref{sec:ontology}. The problem of quantifying the completeness can now be divided into two subproblems.

\begin{itemize}
	\item The first subproblem deals with the quantification of the completeness regarding the number of scenario classes. 
	\item The second subproblem deals with the quantification of the completeness regarding the scenarios of a specific scenario class.
\end{itemize}

During my research, I only focused on the second subproblem. To address this problem, it is assumed that each scenario of a specific scenario class can be parametrized using similar variables. Therefore, a probability density function (pdf) can be estimated based on the recorded scenarios. This pdf can be used to generate new instances that serve as test cases for the assessment of AVs. To obtain test cases that reflect the real-world traffic, the estimated pdf needs to resemble the true underlying pdf. Therefore, the second subproblem is addressed by quantifying the uncertainty of the estimated pdf.

One way to estimate the uncertainty of an estimated pdf employs the so-called Mean Integrated Squared Error. Let $x \in \mathbb{R}^d$ be the vector of parameters and $f(x)$ the true underlying distribution. The pdf $f(x)$ is estimated using $n$ datapoints. Let $\hat{f}(x;n)$ denote the estimated pdf. This the MISE is defined as follows:
\begin{equation} \label{eq:mise}
	\mise{n} = \expectation{ \int_{\mathbb{R}^d} \left(\hat{f}(x;n) - f(x)\right)^2 \, \textup{d}x} = \int_{\mathbb{R}^d} \expectation{\left(\hat{f}(x;n) - f(x)\right)^2} \, \textup{d}x,
\end{equation}

For calculating the MISE, the true pdf $f(x)$ is required. However, the MISE can be estimated using $\hat{f}(x, n)$. When $f(x)$ is estimated using a parametric statistics, e.g., a Gaussian distribution, the posterior distributions of the parameters of the pdf can be used \cite{bishop2006pattern}, whereas for nonparametric statistics, e.g., Kernel Density Estimation \cite{rosenblatt1956remarks, parzen1962estimation}, the Asymptotic MISE (AMISE) can be employed \cite{chen2017tutorial}. As this report is meant to give an overview of the research, more details on quantifying the completeness are omitted in this report.
