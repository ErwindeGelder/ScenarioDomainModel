\section{Research}
\label{sec:research}


\subsection{Introduction}
\label{sec:introduction}

% Automated Vehicles are introduced (in Singapore)
The development of automated vehicles (AVs) has made significant progress. It is expected that before 2020, automated and autonomous vehicles will be introduced in controlled environments, whereas autonomous vehicles will be mainstream by 2040 \cite{madni2018autonomous} or earlier \cite{bimbraw2015autonomous}. Especially in densely populated cities such as Singapore, there is a need for automated vehicles to increase traffic safety and traffic efficiency by enabling flexible, automated, mobility-on-demand systems \cite{spieser2014toward}, scheduled services for public transport needs, and automated freight and service vehicles to support 24 hours operations and labor shortage needs.

% Assessment of AVs is important
An important aspect in the development of AVs is the safety assessment of the AVs \cite{bengler2014threedecades, stellet2015taxonomy, putz2017pegasus, wachenfeld2016release}. For legal and public acceptance of AVs, a clear definition of system performance is important, as are quantitative measures for the system quality. The more traditional methods \cite{ISO26262, response2006code}, used for evaluation of driver assistance systems, are no longer sufficient for assessment of the safety of higher level AVs, as it is not feasible to complete the quantity of testing required by these methodologies \cite{wachenfeld2016release}. Therefore, the development of assessment methods is important to not delay the deployment of AVs \cite{bengler2014threedecades}.

% Scenario-based approach
One proposed way to assess safety is to test drive AVs in real traffic, observe their performance, and make statistical comparisons to human driver performance. However, this requires AVs to drive hundreds of millions of kilometers and sometimes hundreds of billions of kilometers to demonstrate their reliability in terms of fatalities and injuries \cite{kalra2016driving}. It does not seem to be feasible to drive these millions of kilometers with the increased speed of development of automated driving (AD) functions and the high level of safety requirements that are expected from these functions. As an alternative, a scenario-based assessment is adopted \cite{putz2017pegasus, stellet2015taxonomy, deGelder2017assessment, ploeg2018cetran, elrofai2018scenario}. 
% Scenarios are obtained with data
These test-scenarios can be knowledge-driven or data-driven \cite{stellet2015taxonomy}. A drawback of knowledge-based test-scenarios is that it does not allow to generalize the results to the performance of the system-under-test when operating in traffic, i.e. the test cases may not be valid or representative for real-world traffic. A data-driven approach does allow to
generalize the results, but here the challenge is to extract the interesting, e.g., performance critical, scenarios from the data, such that the number of simulations is still limited \cite{deGelder2017assessment}.

\subsection{Focus of research}
\label{sec:focus}

% Use summary Go/No Go meeting


\subsection{Preliminary results}
\label{sec:results}




\subsection{Submitted paper}
\label{sec:paper}


\subsection{Outlook}
\label{sec:outlook}
