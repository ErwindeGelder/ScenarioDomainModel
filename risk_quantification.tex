\documentclass[journal]{IEEEtran}
\usepackage[utf8]{inputenc}
\usepackage{graphicx} % for pdf, bitmapped graphics files
\usepackage{amsmath} % assumes amsmath package installed
\usepackage{amsthm}  % For special theorem style
\usepackage{amsfonts}
\usepackage{dsfont}
\usepackage[USenglish]{babel}  % language support
\usepackage[capitalize]{cleveref}\crefname{equation}{}{}\Crefname{equation}{Equation}{Equations}
\usepackage{siunitx}
\usepackage[nolist,nohyperlinks]{acronym}
\usepackage{xcolor}
\usepackage{csquotes}
\usepackage{tikz}\usetikzlibrary{shapes,arrows,backgrounds}
\usepackage{pgfplots}\pgfplotsset{compat=1.16}
\usepackage{tabularx}
\usepackage{booktabs}\setlength{\heavyrulewidth}{0.1em}\newcommand{\otoprule}{\midrule[\heavyrulewidth]}	



\usepackage{silence}  							%% For filtering warnings
\usepackage[style=ieee,doi=false,isbn=false,url=false,date=year,backend=biber,maxbibnames=15,maxcitenames=2,mincitenames=1,uniquelist=false,uniquename=false,giveninits=true]{biblatex}
% Filter warnings issued by package biblatex starting with "Patching footnotes failed"
\WarningFilter{biblatex}{Patching footnotes failed}
\renewcommand*{\bibfont}{\footnotesize}		%% Use this for papers
\renewcommand*{\bibfont}{\small}
\setlength{\biblabelsep}{\labelsep}
\bibliography{references_rq}



\theoremstyle{remark}\newtheorem{remarkenv}{Remark}        %% remarks
\newenvironment{remark}{\begin{remarkenv}}%
	{\hfill$\lozenge$\end{remarkenv}}            %% end remark with a lozenge



% Notations
\newcommand{\e}[1]{\exp\left\{ #1 \right\}}
\newcommand{\expectation}[1]{\mathds{E}\left[ #1 \right]}
\newcommand{\injury}{\mathrm{injury}}
\newcommand{\numberofencounters}{k}
\newcommand{\parameters}{x}
  \newcommand{\pardim}{d}
\newcommand{\poissonparameter}{\lambda}
\newcommand{\probability}[1]{\mathds{P}\left( #1 \right)}
  \newcommand{\probabilitycond}[2]{\probability{ #1 | #2 }}
\newcommand{\realnumbers}{\mathds{R}}
\newcommand{\scenariocategory}{\mathcal{C}}
\newcommand{\simulationoutcome}[1]{R\left(#1\right)}


\newcommand{\todo}[1]{\color{red}TO DO: #1 \color{black}}

% Acronyms
\begin{acronym}[AAAAAAAA]
    \acro{acc}[ACC]{Adaptive Cruise Control}\acroindefinite{acc}{an}{an}
    \acro{ads}[ADS]{Automated Driving System}\acroindefinite{ads}{an}{an}
	\acro{av}[AV]{Automated Vehicle}\acroindefinite{av}{an}{an}
	\acro{fcw}[FCW]{Forward Collision Warning}
	\acro{hdm}[HDM]{Human Driver Model}
	\acro{idm}[IDM]{Intelligent Driver Model}
	\acro{idmplus}[IDM+]{Intelligent Driver Model plus}
	\acro{odd}[ODD]{Operational Design Domain}\acroindefinite{odd}{an}{an}
\end{acronym}


\begin{document}
\date{}

\title{Risk Quantification for Automated Driving Systems in Driving Scenarios}

\author{Erwin~de~Gelder$^{1,2,*}$,
	    Hala~Elrofai$^{1}$,
	    Arash~Khabbaz~Saberi$^{3}$,
	    Jan-Pieter~Paardekooper$^{1,4}$,
	    Olaf~Op~den~Camp$^{1}$,
	    Bart~De~Schutter$^{2}$%
\thanks{$^1$ TNO, Dept.\ of Integrated Vehicle Safety, Helmond, The Netherlands}%
\thanks{$^2$ Delft University of Technology, Delft Center for Systems and Control, Delft, The Netherlands}%
\thanks{$^3$ TomTom, Automated Driving product unit, Eindhoven, The Netherlands}%
\thanks{$^4$ Radboud University, Donders Institute for Brain, Cognition and Behaviour, Nijmegen, The Netherlands}%
\thanks{$^*$ Corresponding author: \textit{erwin.degelder@tno.nl}}}%

\maketitle
\input{sections/00-abs}
\acresetall

%%%%%%%%%%%%%%%%%%%%%%%%%%%%%%%%%%%%%%%%%%%%%%%%%%%%%%%%%%%%%%%%%%%%%%%%%%%%%%%%
\section{Introduction}
\label{sec:introduction}

Objective/Research question of the paper: how to quantify safety risk associated with driving automated vehicles 

Suggested structure: 
\begin{itemize}
    \item Context: describe the context of the problem
    \item Background info: short intro to relevant topics: SOTIF, ISO 26262, Scenario based analysis, previous risk quantification efforts, 
    \item Gap: what is missing?  
    \item Contributions: what are we adding in this paper: Adding severity quantification, adding controllability quantification, adding triggering conditions, applying it for multiple relevant scenarios
    \item structure of the paper
\end{itemize}

SOTIF: Absence of unreasonable risk due to hazards resulting from functional inefficiencies of the intended functionality or from reasonably foreseeable misuse by persons.
Hazard identification and evaluation is part of the process to ensure safety according to ISO~21448.
This includes:
\begin{itemize}
    \item Context: 
    Automated driving development requires simulation both at the design phase as well as the assessment phase for covering ``corner'' or ``edge'' cases. 
    Quantifying risk is pivotal for successfully covering the huge ``design space''
    \item Background: 
    Risk is defined by ISO~26262 with three main elements: probability of exposure to the ``\textit{operational situation}'', \textit{severity} of the outcome of scenario, and \textit{controllability} of the situation. 
    \item Determining the probability of exposure of the ``operational situation''.
    \item Determining the \textit{severity} based on models from the literature that depend on parameters such as impact speed. 
    \item Determining the \emph{controllability}, which is defined as ``ability to avoid a specified harm or damage through the timely reactions of the persons involved, possibly with the support from external measures'' \autocite{ISO26262}.
    \item Background continued: SOTIF, as an addition to ISO~26262, is concerned by hazards that are the result of \textit{functional insufficiency} as opposed to \textit{malfunctioning behavior}. Functional insufficiency is either performance limitation or specification issues that could be exposed by a \textit{triggering condition}. Triggering condition is defined in ISO/DIS~21448 as ``specific conditions of a scenario that serve as an initiator for a subsequent system reaction leading to hazardous behaviour.'' 
    
    
    \item Gap: There is not risk quantification method that is able to consider are these factors together. 
    
    \item contribution 1: method for quantifying risk considering exposure, severity, controllability, and triggering condition. 
    \item contribution 2: illustrative case studies on several scenarios and triggering conditions: 
    TODO: make a list of scenarios. 
    
    

\end{itemize}

\section{Method}
\label{sec:method}

Method steps:
\begin{itemize}
    \item Identify scenarios within scope 
    \item Calculate probability of exposure to scenario based on observation in data
    \item Analyze relevant triggering conditions 
    \item Simulate scenarios and triggering conditions including controllablilty with driver model 
    \item Calculate probability or collision and relevant parameter for severity calculation 
    \item Calculate probability of injury for severity  
    \item Visualize risk using heat maps. 
\end{itemize}
Q: what about data collection? is that part of our method? 
We add an assumption that distribution of relevant scenario parameters are known, either though collected data or other reliable sources of information. 

\subsection{Identification of scenarios}

What do we need from a scenario?
\begin{itemize}
    \item Scene 
    \item ODD
    \item Actors 
    \item Activities and parameters 
\end{itemize}

Scenario (or operational situation is ISO~26262 terminology) is an important part risk assessment for both SOTIF and functional safety. 
Ensuring that relevant scenarios at the right level of detail are selected is important to ensure correct risk evaluation. 
To systematically evaluate scenarios, one method can be breaking it down into smaller parts for analysis.  
Several norms define what a scenario should include. 
According to SAE J2980 on considerations for ISO 26262 ASIL hazard classification, \textit{operational scenario}, includes specification of: Location (e.g.; highway or city), road conditions (e.g., road friction, slope), driving maneuvers (e.g., driving forward), vehicle state (e.g., accelerating, braking), and other relevant considerations and characteristics (e.g., weather condition). 
SOTIF also describes elements of a scenario (taking on the model from Ulbrich) as: Scene including dynamic and static element, actions, events, and goals. 
Another important term in this domain is ``operational design domain'' by SAE J3016, which specifies the conditions under which the AV is intended to be used.

Furthermore, other protocols such as OpenScneario consider: ....
\todo{finish this one.} 


\subsection{Triggering event analysis}



\subsection{Calculate severity}
\label{sec:severity}

\todo{Hala: summarize literature review for injury probability.}

\section{Case study}
\label{sec:case study}



\subsection{Automated driving system under test}
\label{sec:ads under test}



\subsection{Scenario categories}
\label{sec:scenario categories}



\subsection{Triggering conditions}
\label{sec:triggering conditions}

\section{Results}
\label{sec:results}



\subsection{Exposure}
\label{sec:exposure results}



\subsection{Severity and controllability}
\label{sec:main results}



\subsection{Triggering conditions}
\label{sec:results triggering conditions}

\acresetall
\input{sections/05-conc}




\printbibliography

\end{document}
