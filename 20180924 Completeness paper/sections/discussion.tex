\section{Discussion}
\label{sec:discussion}

The measure for quantification of completeness of the set of activities that is presented in this work is based on the amount of data and the chosen parametrization. More data might be used to achieve a certain threshold. However, it might also be possible to adapt the parametrization to achieve a certain threshold if a parametrization exists that achieves a certain threshold. Hence, the presented method can be used to determine an appropriate parametrization of activities.

The method for quantifying the completeness of a set of activities presented in this work depends on a threshold that needs to be chosen. \cstart Only in case of an infinite set of data, the measure for completeness approaches zero, so this threshold needs to be larger than zero. \cend This threshold might be different for different applications. For example, when the data are used for determining test scenarios \cite{elrofai2018scenario, ploeg2018cetran}, the desired threshold might be lower than when the data are used for determining driver models \cite{wang2017much, sadigh2014data}. \cstart Furthermore, the threshold depends on the number of parameters for one activity, denoted by $d$ in \cref{sec:method}. Based on experience with the dataset used in \cref{sec:result real}, assuming that the dataset is normalized such that the standard deviation equals one, a threshold between 0.01 and 0.001 gives good results. When a threshold of 0.01 is reached, a reasonable reliable pdf can be constructed to analyze nominal driving behavior, whereas a threshold of around 0.001 is required to also accurately analyze the edge cases. 
\cend

\cstart
When using our measure for completeness, the following might be considered. As explained in \cref{sec:method}, the measure for completeness is based on the AMISE. It is also mentioned that the AMISE only differs from the MISE by higher-order terms under some mild conditions. This requires that the real pdf is smooth, i.e., without large spikes \cite{marron1992exact}. \textcite{marron1992exact} also state that the AMISE is strictly higher than the MISE under some mild conditions\footnote{The Laplacian of $f(\cdot)$ needs to be continuous and square-integrable and $K(u) \leq 0, \forall u$.}. As a result, it is likely that the measure for completeness, which is an approximation of the AMISE, is higher than the MISE. This could lead to an overestimation of the number of required samples.
\cend

The measure for completeness that is proposed in this paper can be regarded as a approximation of the MISE of \cref{eq:mise}. To minimize the MISE, the approximated pdf should be similar to the real pdf. It might be, however, that one is not interested in the exact likelihoods of certain values of the parameters, but in all possible values that the parameters can have. In this case, one might be interested in the support of the real pdf, because the support of the pdf defines all possible values for which the likelihood is larger than zero, see, e.g., \textcite{scholkopf2001estimating}.

As mentioned in \cref{sec:problem}, our problem of quantifying the completeness of a dataset can be divided into three subproblems. The first subproblem, i.e., how to quantify the completeness regarding the scenario classes, can be regarded as the so-called unseen species problem \cite{bunge1993estimating, gandolfi2004nonparametric} or species estimation problem \cite{yang2012estimating}. In case of the unseen species problem, the entire population is partitioned into $C$ classes and the objective is to estimate $C$ given only a part of the entire population. To continue the analogy, the second subproblem, i.e., how to quantify the completeness regarding all scenarios that fall into a specific scenario class, relates to quantifying whether we have a complete view on the variety among one species, given the number of individuals that we have seen. The third subproblem addresses a part of the scenarios, i.e., the activities. In line with the previous analogy, this can be seen as quantifying whether we have a complete view of the parts of the species, e.g., its limbs or organs.

Our proposed method answers the third subproblem, i.e., how to quantify the completeness regarding the activities. 
\cstart 
The advantage of using the activities for determining the completeness is that there is only a limited number of types of activities. As a result, for each type of activity, it is expected that there is no need for an extremely large dataset to obtain a fair number of similar activities. On the other hand, however, it is not known how much data is required to obtain the desired threshold, because, e.g., this depends on the parametrization that is chosen. 
\cend 
The next step is to quantify the completeness regarding all scenarios that fall into a specific scenario class. Here, the joint probability of the parameters of different activities in the same scenario class might be considered. Although the presented method can be applied, this might be impractical because the number of parameters will be higher than for the activities. The problems of quantifying the completeness regarding all scenarios that fall into a specific scenario class and quantifying the completeness regarding the scenario classes remain future work.
