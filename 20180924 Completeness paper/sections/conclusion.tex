\section{Conclusions}
\label{sec:conclusion}

% Objective
More and more field data from (naturalistic) driving data become available. The data are used for all kinds of driving-related research, developments, assessments, and evaluations. When deducing claims based on the collected data, we require knowledge about the degree of completeness of the data. 
% Method
We considered the data as a sequence of scenarios, whereas activities are the building blocks of these scenarios. To obtain knowledge about the degree of completeness of the data, we propose a measure to quantify the completeness of the activities. This measure allows to partly answer questions like ``have we collected enough field data?'' 
% Results
We illustrated the method using an artificial dataset, for which the underlying distributions are known. These results show that the proposed method correctly quantifies the completeness of the activities. We also applied the method on a dataset with naturalistic driving to show that the method can be used to estimate the required number of samples.
% Future work
In future work, we will extend the method to whole traffic scenarios and scenario classes \cstart and we will investigate the appropriate thresholds for the measure to quantify completeness in different applications. \cend
