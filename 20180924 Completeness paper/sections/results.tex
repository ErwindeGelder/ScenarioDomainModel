\section{Examples}
\label{sec:results}

In this section, the proposed method of \cref{sec:method} is illustrated by means of two examples. The first example applies the method with data generated from a known distribution. Because the distribution is known, the real MISE can be accurately approximated and compared with the results from \cref{eq:measure,eq:measure independent}. Secondly, in \cref{sec:result real}, the proposed method is applied on a dataset containing naturalistic driving data.

\subsection{Example with known underlying distribution}
\label{sec:result artificial}

In this example, the data samples $Y_i$ with $i \in \{1, \ldots, n\}$ are independently and identically distributed random variables that are distributed according to the pdf $g(\cdot)$. Each data sample $Y_i$ corresponds to a scalar, i.e., $d_y=1$. Similarly, the data samples $Z_i$ with $i \in \{1, \ldots, n\}$ are independently and identically distributed random variables that are distributed according to the pdf $h(\cdot)$. The data samples are combined, similar to \cref{eq:combine}, such that the likelihood of $X_i$ is $f(X_i)=g(Y_i)h(Z_i)$. 

\Cref{fig:true pdf} shows the distributions $g(\cdot)$ (black solid line) and $h(\cdot)$ (gray dashed line). Both distributions are Gaussian mixtures, i.e., both pdfs equal the sum of multiple weighted Gaussian distributions. The pdf $g(\cdot)$ corresponds to the average of two Gaussian distributions with means of $-1$ and $1$ and standard deviations $0.5$ and $0.3$, respectively. The pdf $h(\cdot)$ corresponds to the average of three Gaussian distributions with means $-0.5$, $0.5$, and $1.5$, and standard deviations $0.3$, $0.5$, and $0.3$, respectively. 

\setlength\figurewidth{\linewidth}
\setlength\figureheight{0.7\linewidth}
\begin{figure}
	\centering
	% This file was created by matplotlib2tikz v0.6.17.
\begin{tikzpicture}

\begin{axis}[
xlabel={$x$},
xmin=-3, xmax=3,
ymin=-0.0184547497061878, ymax=0.387729405483734,
width=\figurewidth,
height=\figureheight,
tick align=outside,
tick pos=left,
xmajorgrids,
x grid style={white!69.01960784313725!black},
ymajorgrids,
y grid style={white!69.01960784313725!black}
]
\addplot [line width=2.0pt, black, forget plot]
table {%
-3 0.0051667463394783
-2.94 0.00654461880712224
-2.88 0.00823047028690945
-2.82 0.0102763296308067
-2.76 0.0127386815677213
-2.7 0.0156777601691089
-2.64 0.0191565206269158
-2.58 0.023239261718724
-2.52 0.0279898842884903
-2.46 0.0334697879276774
-2.4 0.0397354284821519
-2.34 0.0468355823644219
-2.28 0.0548083888784006
-2.22 0.0636782674682525
-2.16 0.0734528312802731
-2.1 0.0841199397542916
-2.04 0.0956450491215403
-1.98 0.107969028724005
-1.92 0.121006611269839
-1.86 0.134645635208032
-1.8 0.148747216648567
-1.74 0.163146956719893
-1.68 0.177657248832099
-1.62 0.192070700762076
-1.56 0.206164631385174
-1.5 0.219706544535379
-1.44 0.232460426663028
-1.38 0.244193664665147
-1.32 0.254684339351727
-1.26 0.263728621854501
-1.2 0.271147987405017
-1.14 0.276795964656363
-1.08 0.28056415903362
-1.02 0.282387323863288
-0.96 0.282247300114256
-0.9 0.280175699999137
-0.84 0.2762552659681
-0.78 0.270619888859475
-0.72 0.263453311461989
-0.66 0.254986571801805
-0.6 0.245494251193712
-0.54 0.235289585090982
-0.48 0.224718472646555
-0.42 0.214152389477757
-0.36 0.203980176189738
-0.3 0.194598653753013
-0.24 0.186402017773098
-0.18 0.179769998141526
-0.12 0.175054846746348
-0.0600000000000001 0.17256733708436
0 0.172562122174252
0.0600000000000001 0.17522298954899
0.12 0.180648754601943
0.18 0.188840719481647
0.24 0.199692762934888
0.3 0.212985185217932
0.36 0.228383383748615
0.42 0.24544226096369
0.48 0.263616960871767
0.54 0.282280106795447
0.6 0.300745199333918
0.66 0.318295276493306
0.72 0.334215395119643
0.78 0.347827027559912
0.84 0.358522140450913
0.9 0.365794582910501
0.96 0.369266489338738
1.02 0.368707703149793
1.08 0.364046730896733
1.14 0.355372395028413
1.2 0.342926101534303
1.26 0.327085397833125
1.32 0.308340186572005
1.38 0.287263511255486
1.44 0.264479186177656
1.5 0.240628676360264
1.56 0.216339540352097
1.62 0.192197452977527
1.68 0.168723371051431
1.74 0.146356851706521
1.8 0.125445945226761
1.86 0.106243524463756
1.92 0.0889094334160046
1.98 0.0735174756140632
2.04 0.0600660380843174
2.1 0.0484910607031717
2.16 0.038680100242474
2.22 0.0304863785692694
2.28 0.0237419140051494
2.34 0.0182690807761867
2.4 0.0138901932543357
2.46 0.0104349442798379
2.52 0.00774572250174328
2.58 0.00568098236166017
2.64 0.00411693920594205
2.7 0.00294791398464616
2.76 0.0020856640947567
2.82 0.00145801842773233
2.88 0.00100709550737067
2.94 0.000687333044196065
3 0.000463503099710759
};
\addplot [line width=2.0pt, gray, dashed, forget plot]
table {%
-3 8.1664388086521e-06
-2.94 1.32735202490426e-05
-2.88 2.13387791689166e-05
-2.82 3.39258027886395e-05
-2.76 5.33362290729395e-05
-2.7 8.29100251642051e-05
-2.64 0.000127423906520495
-2.58 0.000193608915017404
-2.52 0.000290807792346734
-2.46 0.000431789798324551
-2.4 0.000633734019064902
-2.34 0.00091938096234404
-2.28 0.00131833546725512
-2.22 0.00186848114109809
-2.16 0.00261743777187522
-2.1 0.00362395943191086
-2.04 0.00495913440614299
-1.98 0.00670721206951642
-1.92 0.00896585114329068
-1.86 0.011845564243823
-1.8 0.0154681318232826
-1.74 0.0199637809273986
-1.68 0.0254669760107816
-1.62 0.0321107534491946
-1.56 0.0400196480213446
-1.5 0.0493014037079956
-1.44 0.0600378228547278
-1.38 0.0722752722788968
-1.32 0.0860155133804601
-1.26 0.101207634482779
-1.2 0.117741916385013
-1.14 0.135446438583287
-1.08 0.1540871223875
-1.02 0.173371706088462
-0.96 0.192957865503134
-0.9 0.212465351490892
-0.84 0.231491645817723
-0.78 0.249630277094908
-0.72 0.266490631597324
-0.66 0.281717879345709
-0.6 0.295011545561037
-0.54 0.306141309919284
-0.48 0.314958812686675
-0.42 0.321404571515163
-0.36 0.325509532392705
-0.3 0.327391246981737
-0.24 0.327245133223743
-0.18 0.325331683437325
-0.12 0.32196078811603
-0.0600000000000001 0.317474511566724
0 0.312229672541257
0.0600000000000001 0.306581453876023
0.12 0.300869013353034
0.18 0.295403732183918
0.24 0.290460365912753
0.3 0.286271006120811
0.36 0.283021466840972
0.42 0.280849513311206
0.48 0.279844273733986
0.54 0.280046220548248
0.6 0.281447262148016
0.66 0.283990719381609
0.72 0.287571232834903
0.78 0.292034910796478
0.84 0.297180238960591
0.9 0.302760393757091
0.96 0.308487606833803
1.02 0.314040110199294
1.08 0.319071959355165
1.14 0.323225711965173
1.2 0.326147572588347
1.26 0.327504248653923
1.32 0.327000450029882
1.38 0.324395749979032
1.44 0.319519443140001
1.5 0.312282103937755
1.56 0.302682764707317
1.62 0.290810975654381
1.68 0.276843440838287
1.74 0.261035396430749
1.8 0.243707355588743
1.86 0.225228236896635
1.92 0.205996178247795
1.98 0.186418487497521
2.04 0.166892184938355
2.1 0.147786458198762
2.16 0.129428100879865
2.22 0.112090677090447
2.28 0.0959877862306995
2.34 0.0812704376110851
2.4 0.0680282197135535
2.46 0.0562936922511944
2.52 0.0460492576282755
2.58 0.037235687154538
2.64 0.0297614809220083
2.7 0.0235123146900695
2.76 0.0183599531400235
2.82 0.014170164981365
2.88 0.0108093409047305
2.94 0.00814967260779642
3 0.0060728868851914
};
\end{axis}

\end{tikzpicture}
	\caption{The true probability density functions $g(\cdot)$ (black solid line) and $h(\cdot)$ (gray dashed line) that are used to illustrate the quantification of the completeness.}
	\label{fig:true pdf}
\end{figure}

The real MISE of \cref{eq:mise} is not calculated exactly. The expectation $\expectation{\cdot}$ is estimated by repeating the estimation of the pdf 200 times, i.e., 
\begin{equation}
	\label{eq:approx mise}
	\mise{f}{n} \approx \frac{1}{m} \sum_{j=1}^m \int \left( f(x) - \hat{f}_j(x;n)\right)^2 \ud x,
\end{equation}
where $\hat{f}_j(x;n)$ is the $j$-th estimate and $m=200$. 

All three pdfs are estimated using \cref{eq:kde}. We use one-leave-out cross validation to compute the bandwidth $h$ (see also \textcite{duin1976parzen}) because this minimizes the Kullback-Leibler divergence between the real pdf $f(x)$ and the estimated pdf $\hat{f}(x;n)$ \cite{turlach1993bandwidthselection,zambom2013review}. Note that although the estimation of the pdfs is repeated 200 times to accurately approximate the MISE using \cref{eq:approx mise}, the bandwidth is only determined once for a specific number of samples. All the other 199 times, the same bandwidths are adopted. The resulting bandwidths are shown in \cref{fig:bandwidth}. The bandwidth of $\hat{f}(x;n)$ (black dashed line) is significantly larger than the bandwidths of $\hat{g}(y;n)$ (gray solid line) and $\hat{h}(z;n)$ (gray dotted line). This result is not surprising: because $\hat{f}(x;n)$ represents a bivariate distribution, it requires more data to have a similar bandwidth compared with a univariate distribution \cite{scott2005multidimensional}.

\setlength\figurewidth{0.9\linewidth}
\setlength\figureheight{0.7\linewidth}
\begin{figure}
	\centering
	% This file was created by matplotlib2tikz v0.7.5.
\begin{tikzpicture}

\begin{axis}[
axis x line*=bottom,
axis y line*=left,
every x tick/.style={black},
every y tick/.style={black},
height=\figureheight,
log basis x={10},
log basis y={10},
tick align=outside,
tick pos=left,
width=\figurewidth,
x grid style={white!69.01960784313725!black},
xlabel={Number of samples},
xmin=100, xmax=5000,
xmode=log,
xtick style={color=black},
y grid style={white!69.01960784313725!black},
ylabel={Bandwidth},
ymin=0.12073463121632, ymax=0.514133563500785,
ymode=log,
ytick style={color=black}
]
\addplot [line width=0.7000000000000001pt, gray]
table {%
100 0.333902544590215
108 0.334746992843899
117 0.33249691495283
127 0.313299239565951
138 0.308128449857578
149 0.287087379333731
161 0.275852692097465
175 0.247590919809407
189 0.240371849561726
205 0.245019151799335
222 0.238213129893222
241 0.232594918722967
261 0.223326930772757
282 0.210746068090978
306 0.216497613329582
331 0.253957966161198
359 0.263488258335087
389 0.259249215169254
421 0.254883009145746
456 0.248740533199762
494 0.235012928468767
535 0.220218276842796
579 0.202869246927223
627 0.20290435756291
679 0.190874804682713
736 0.214007512145567
797 0.217916819764872
863 0.207377564314436
935 0.198103970787746
1013 0.195903264029502
1097 0.178583258596411
1188 0.182778529580626
1287 0.182014773896209
1394 0.185411253471078
1510 0.185224570599584
1635 0.173469219282776
1771 0.173703922947084
1918 0.172726937037639
2078 0.168995061714128
2250 0.161121478215138
2437 0.155267211210851
2640 0.16282926628745
2859 0.163813484923627
3097 0.158158303309362
3354 0.152430615691622
3633 0.144719178998213
3935 0.144723048908827
4262 0.155973985312413
4616 0.152543706717652
5000 0.148094988830411
};
\addplot [line width=0.7000000000000001pt, gray, dotted]
table {%
100 0.345479296425376
108 0.331495634249555
117 0.330685959154714
127 0.330651878039332
138 0.303374311500354
149 0.288045984767778
161 0.283579062143539
175 0.285759824891524
189 0.277264201851355
205 0.276537918621916
222 0.261552908075927
241 0.254902806261265
261 0.252412415046285
282 0.24735242620911
306 0.250473019101765
331 0.23635267972024
359 0.211317117665954
389 0.215603831320825
421 0.202279662613266
456 0.188193458503748
494 0.189605745892871
535 0.201713723998427
579 0.209045736912432
627 0.211462713508606
679 0.216255722295364
736 0.221059383003018
797 0.222508953506758
863 0.225867820591947
935 0.212422814712975
1013 0.211964288180541
1097 0.194652580532807
1188 0.181325953000876
1287 0.172744928875239
1394 0.169711397542423
1510 0.165168112863284
1635 0.165322206212235
1771 0.179486727055505
1918 0.177499902113277
2078 0.181573937061172
2250 0.182260098277007
2437 0.171506667247533
2640 0.168201046072913
2859 0.152023238575331
3097 0.155422356811239
3354 0.155193483787524
3633 0.154252249184797
3935 0.140909746785724
4262 0.141282622652018
4616 0.128953716819263
5000 0.130361483791334
};
\addplot [line width=0.7000000000000001pt, black, dashed]
table {%
100 0.481364381859579
108 0.464755356694629
117 0.457494316870976
127 0.442958436477485
138 0.426965188024608
149 0.414614453029318
161 0.407976328043523
175 0.39725704373531
189 0.382483201693791
205 0.385859304906856
222 0.369497595628203
241 0.355044028195492
261 0.3380847678872
282 0.325353057253603
306 0.321464161136803
331 0.32700287636065
359 0.316724687825919
389 0.331722993032317
421 0.323197185465412
456 0.322278622486992
494 0.312073396763021
535 0.317061922891473
579 0.308485945169615
627 0.306149332868807
679 0.295514190576037
736 0.294761943585706
797 0.289053331019986
863 0.288527190861965
935 0.277416827257005
1013 0.270282189571542
1097 0.270107424486116
1188 0.266672971052669
1287 0.251826057832938
1394 0.249748440634064
1510 0.241650768377245
1635 0.239740744554738
1771 0.24373690213204
1918 0.252909698876046
2078 0.247765724960661
2250 0.240223079383641
2437 0.232685578181925
2640 0.23580710707978
2859 0.238777227365712
3097 0.229057424148005
3354 0.22657521666947
3633 0.226081512867401
3935 0.226503357699914
4262 0.2221922417631
4616 0.218478444785539
5000 0.214155431920472
};
\end{axis}

\end{tikzpicture}
	\caption{The bandwidths of $\hat{f}(x;n)$ (black dashed line), $\hat{g}(y;n)$ (gray solid line), and $\hat{h}(z;n)$ (gray dotted line) for the example of \cref{sec:result artificial}. The bandwidths are computed using one-leave-out cross validation for different number of samples $n$.}
	\label{fig:bandwidth}
\end{figure}

\Cref{fig:mise example} shows the result this example. The black lines show the real MISEs, approximated using \cref{eq:approx mise}, where the black solid line represents the MISE when $f(x)$ is directly estimated and the black dashed line represented the MISE when use is made of \cref{eq:independency}. The MISE is significantly lower when it is assumed that the two parameters are independent. One way to look at this is that the degree of freedom of $f(x)$ is reduced when assuming that the two parameters are independent and this lower degree in freedom leads to a more certain estimate. Hence, the MISE is lower.

\setlength\figurewidth{\linewidth}
\setlength\figureheight{0.7\linewidth}
\begin{figure}
	\centering
	% This file was created by matplotlib2tikz v0.6.17.
\begin{tikzpicture}

\definecolor{color0}{rgb}{0.12156862745098,0.466666666666667,0.705882352941177}
\definecolor{color1}{rgb}{1,0.498039215686275,0.0549019607843137}

\begin{axis}[
xlabel={Number of samples},
ylabel={MISE},
xmin=100, xmax=5000,
ymin=0.000403611520094206, ymax=0.0146240416594672,
xmode=log,
ymode=log,
width=\figurewidth,
height=\figureheight,
tick align=outside,
tick pos=left,
xmajorgrids,
x grid style={white!69.01960784313725!black},
ymajorgrids,
y grid style={white!69.01960784313725!black},
legend style={draw=white!80.0!black},
legend entries={{Real MISE},{Estimated MISE}},
legend cell align={left}
]
\addlegendimage{no markers, color0}
\addlegendimage{no markers, color1}
\addplot [line width=2.0pt, color0]
table {%
100 0.00859558582257646
108 0.00812354841097345
117 0.00767960999089539
127 0.00724156968333626
138 0.00672419333167824
149 0.00660572197053461
161 0.00769490075185661
175 0.0070536539874133
189 0.00734188669662077
205 0.00560004520248748
222 0.00497320029942687
241 0.00458341225138433
261 0.0044193967132435
282 0.00411277466976824
306 0.00391024863593416
331 0.00371502137121955
359 0.00345931539882796
389 0.00325192901866895
421 0.00303951306817579
456 0.00280851639425498
494 0.00271323547625052
535 0.00264215520746293
579 0.00258984470938745
627 0.00237308394789097
679 0.00219208629809438
736 0.00204851187007183
797 0.00192558480862347
863 0.00184092307760793
935 0.00174854822001538
1013 0.00167027944732962
1097 0.00154219545609572
1188 0.00141474033936746
1287 0.00135239133962665
1394 0.00123566339441748
1510 0.00116519558723847
1635 0.00109955350545929
1771 0.00105316800665945
1918 0.000998804164198836
2078 0.000938865893448118
2250 0.000909986663373976
2437 0.00092481765671545
2640 0.000852028795585114
2859 0.000824250734934949
3097 0.000833149925120803
3354 0.0007652585851753
3633 0.000661896203261313
3935 0.000612342178439652
4262 0.000532870354595377
4616 0.000499248115923246
5000 0.000475151288126954
};
\addplot [line width=2.0pt, color1]
table {%
100 0.0124222154744985
108 0.0114047924111373
117 0.0110856998560715
127 0.00908932470364904
138 0.00878628215427483
149 0.00986933776249793
161 0.0123917508129278
175 0.0105715440500389
189 0.0110696465088209
205 0.00858025418107768
222 0.00747320359552377
241 0.00632682919019152
261 0.00661365722756003
282 0.00596364492904691
306 0.00537243882058895
331 0.00518129038406335
359 0.00495020249335441
389 0.00456345424271336
421 0.00455613142317589
456 0.00414160123883459
494 0.00392694207708443
535 0.00389371221570144
579 0.00377971960642565
627 0.00337406939576007
679 0.00290753753157886
736 0.00266130816611476
797 0.00246875767172887
863 0.00238672019361078
935 0.00233647608913928
1013 0.00224908007641507
1097 0.00210606135141524
1188 0.00194369485769856
1287 0.00186906375094596
1394 0.00161076503782372
1510 0.00154713660477321
1635 0.00144645969912583
1771 0.00138043601579382
1918 0.00134242289195381
2078 0.00114594022552776
2250 0.00113333040382035
2437 0.00117663255862454
2640 0.00107084030741098
2859 0.00105339633754792
3097 0.00109031230531131
3354 0.000987151733632913
3633 0.000840833236739904
3935 0.000793096424147307
4262 0.000683616696267234
4616 0.00064411866574568
5000 0.000614216157862005
};
\end{axis}

\end{tikzpicture}
	\caption{The real MISEs (black lines) of the example of \cref{sec:result artificial}, approximated using \cref{eq:approx mise}, and the measures that are used to quantify the completeness (gray lines). The solid lines show the result of estimating a bivariate pdf, so \cref{eq:measure} is used to quantify the completeness. The dashed lines show the result of estimating two univariate pdfs and combining them according to \cref{eq:independency} to create a bivariate pdf, so \cref{eq:measure independent} is used to quantify the completeness. The gray areas show the interval $[\mu-3\sigma,\mu+3\sigma]$, where $\mu$ and $\sigma$ denote the mean and standard deviation, respectively, of the measures of \cref{eq:measure,eq:measure independent} when repeating the experiment 200 times.} 
	\label{fig:mise example}
\end{figure}

The gray lines in \cref{fig:mise example} show the measures to quantify the completeness of the data. The gray solid line shows the result of applying \cref{eq:measure} and the gray dashed line shows the result of applying \cref{eq:measure independent}. Both lines follow the same trend as their black equivalents. This illustrates that the measures \cref{eq:measure,eq:measure independent} are applicable for estimated the real MISE of \cref{eq:mise}. To show that this is not a mere coincidence, the gray areas in \cref{fig:mise example} show the interval $[\mu-3\sigma,\mu+3\sigma]$, where $\mu$ and $\sigma$ denote the mean and standard deviation, respectively, of the measures of \cref{eq:measure,eq:measure independent} when repeating the experiment 200 times. Note that the measures of completeness are consequently higher than the real MISE. This can be explained from the fact that the measures of completeness are approximations of the AMISE and the AMISE itself is always higher than the real MISE under some mild conditions, see Theorem 4.2 of \textcite{marron1992exact}.

\subsection{Example with real data}
\label{sec:result real}

In this example, 60 hours of naturalistic driving data from 20 different drivers (see also \textcite{deGelder2017assessment}) is used to extract approximately 2800 braking activities. Three parameters are used to describe each braking activity: the average deceleration, the total speed difference, and the end speed. A histogram of each of these parameters is shown in \cref{fig:histogram}. Note that these braking activities do not include full stops, i.e., activities where the end speed is zero, because the distribution of the end speed will have a large peak at zero. The AMISE of \cref{eq:amise} deviates more from the real MISE of \cref{eq:mise}, especially for larger bandwidths, when such peaks are present in the underlying distribution \cite{marron1992exact}. Because the measure \cref{eq:measure} we use for quantification of completeness is based on the AMISE of \cref{eq:amise}, we want to avoid these peaks as much as possible. Therefore, the full stops are excluded.

\setlength\figurewidth{\linewidth}
\setlength\figureheight{0.5\linewidth}
\begin{figure}
	\centering
	% This file was created by matplotlib2tikz v0.6.17.
\begin{tikzpicture}

\begin{groupplot}[group style={group size=1 by 3, vertical sep=1.3cm}]
\nextgroupplot[
xlabel={Average deceleration [m/s$^2$]},
ylabel={Frequency},
xmin=0, xmax=3.29356704206494,
ymin=0, ymax=581.7,
width=\figurewidth,
height=\figureheight,
tick align=outside,
tick pos=left,
x grid style={white!69.01960784313725!black},
y grid style={white!69.01960784313725!black}
]
\draw[draw=black,fill=gray] (axis cs:0.235489409566475,0) rectangle (axis cs:0.381112153971163,36);
\draw[draw=black,fill=gray] (axis cs:0.381112153971163,0) rectangle (axis cs:0.526734898375852,71);
\draw[draw=black,fill=gray] (axis cs:0.526734898375852,0) rectangle (axis cs:0.672357642780541,175);
\draw[draw=black,fill=gray] (axis cs:0.672357642780541,0) rectangle (axis cs:0.817980387185229,343);
\draw[draw=black,fill=gray] (axis cs:0.817980387185229,0) rectangle (axis cs:0.963603131589918,479);
\draw[draw=black,fill=gray] (axis cs:0.963603131589918,0) rectangle (axis cs:1.10922587599461,554);
\draw[draw=black,fill=gray] (axis cs:1.10922587599461,0) rectangle (axis cs:1.2548486203993,438);
\draw[draw=black,fill=gray] (axis cs:1.2548486203993,0) rectangle (axis cs:1.40047136480398,302);
\draw[draw=black,fill=gray] (axis cs:1.40047136480398,0) rectangle (axis cs:1.54609410920867,179);
\draw[draw=black,fill=gray] (axis cs:1.54609410920867,0) rectangle (axis cs:1.69171685361336,107);
\draw[draw=black,fill=gray] (axis cs:1.69171685361336,0) rectangle (axis cs:1.83733959801805,50);
\draw[draw=black,fill=gray] (axis cs:1.83733959801805,0) rectangle (axis cs:1.98296234242274,23);
\draw[draw=black,fill=gray] (axis cs:1.98296234242274,0) rectangle (axis cs:2.12858508682743,12);
\draw[draw=black,fill=gray] (axis cs:2.12858508682743,0) rectangle (axis cs:2.27420783123212,8);
\draw[draw=black,fill=gray] (axis cs:2.27420783123212,0) rectangle (axis cs:2.4198305756368,2);
\draw[draw=black,fill=gray] (axis cs:2.4198305756368,0) rectangle (axis cs:2.56545332004149,1);
\draw[draw=black,fill=gray] (axis cs:2.56545332004149,0) rectangle (axis cs:2.71107606444618,1);
\draw[draw=black,fill=gray] (axis cs:2.71107606444618,0) rectangle (axis cs:2.85669880885087,1);
\draw[draw=black,fill=gray] (axis cs:2.85669880885087,0) rectangle (axis cs:3.00232155325556,2);
\draw[draw=black,fill=gray] (axis cs:3.00232155325556,0) rectangle (axis cs:3.14794429766025,2);
\nextgroupplot[
xlabel={Speed difference [m/s]},
ylabel={Frequency},
xmin=0, xmax=26.9465,
ymin=0, ymax=819,
width=\figurewidth,
height=\figureheight,
tick align=outside,
tick pos=left,
x grid style={white!69.01960784313725!black},
y grid style={white!69.01960784313725!black}
]
\draw[draw=black,fill=gray] (axis cs:1.4,0) rectangle (axis cs:2.6165,780);
\draw[draw=black,fill=gray] (axis cs:2.6165,0) rectangle (axis cs:3.833,479);
\draw[draw=black,fill=gray] (axis cs:3.833,0) rectangle (axis cs:5.0495,366);
\draw[draw=black,fill=gray] (axis cs:5.0495,0) rectangle (axis cs:6.266,261);
\draw[draw=black,fill=gray] (axis cs:6.266,0) rectangle (axis cs:7.4825,222);
\draw[draw=black,fill=gray] (axis cs:7.4825,0) rectangle (axis cs:8.699,175);
\draw[draw=black,fill=gray] (axis cs:8.699,0) rectangle (axis cs:9.9155,147);
\draw[draw=black,fill=gray] (axis cs:9.9155,0) rectangle (axis cs:11.132,104);
\draw[draw=black,fill=gray] (axis cs:11.132,0) rectangle (axis cs:12.3485,87);
\draw[draw=black,fill=gray] (axis cs:12.3485,0) rectangle (axis cs:13.565,74);
\draw[draw=black,fill=gray] (axis cs:13.565,0) rectangle (axis cs:14.7815,26);
\draw[draw=black,fill=gray] (axis cs:14.7815,0) rectangle (axis cs:15.998,21);
\draw[draw=black,fill=gray] (axis cs:15.998,0) rectangle (axis cs:17.2145,19);
\draw[draw=black,fill=gray] (axis cs:17.2145,0) rectangle (axis cs:18.431,13);
\draw[draw=black,fill=gray] (axis cs:18.431,0) rectangle (axis cs:19.6475,4);
\draw[draw=black,fill=gray] (axis cs:19.6475,0) rectangle (axis cs:20.864,3);
\draw[draw=black,fill=gray] (axis cs:20.864,0) rectangle (axis cs:22.0805,3);
\draw[draw=black,fill=gray] (axis cs:22.0805,0) rectangle (axis cs:23.297,0);
\draw[draw=black,fill=gray] (axis cs:23.297,0) rectangle (axis cs:24.5135,1);
\draw[draw=black,fill=gray] (axis cs:24.5135,0) rectangle (axis cs:25.73,1);
\nextgroupplot[
xlabel={End speed [m/s]},
ylabel={Frequency},
xmin=0, xmax=35.1355,
ymin=0, ymax=584.85,
width=\figurewidth,
height=\figureheight,
tick align=outside,
tick pos=left,
x grid style={white!69.01960784313725!black},
y grid style={white!69.01960784313725!black}
]
\draw[draw=black,fill=gray] (axis cs:0.37,0) rectangle (axis cs:2.0255,557);
\draw[draw=black,fill=gray] (axis cs:2.0255,0) rectangle (axis cs:3.681,290);
\draw[draw=black,fill=gray] (axis cs:3.681,0) rectangle (axis cs:5.3365,343);
\draw[draw=black,fill=gray] (axis cs:5.3365,0) rectangle (axis cs:6.992,419);
\draw[draw=black,fill=gray] (axis cs:6.992,0) rectangle (axis cs:8.6475,440);
\draw[draw=black,fill=gray] (axis cs:8.6475,0) rectangle (axis cs:10.303,251);
\draw[draw=black,fill=gray] (axis cs:10.303,0) rectangle (axis cs:11.9585,134);
\draw[draw=black,fill=gray] (axis cs:11.9585,0) rectangle (axis cs:13.614,112);
\draw[draw=black,fill=gray] (axis cs:13.614,0) rectangle (axis cs:15.2695,78);
\draw[draw=black,fill=gray] (axis cs:15.2695,0) rectangle (axis cs:16.925,35);
\draw[draw=black,fill=gray] (axis cs:16.925,0) rectangle (axis cs:18.5805,22);
\draw[draw=black,fill=gray] (axis cs:18.5805,0) rectangle (axis cs:20.236,10);
\draw[draw=black,fill=gray] (axis cs:20.236,0) rectangle (axis cs:21.8915,17);
\draw[draw=black,fill=gray] (axis cs:21.8915,0) rectangle (axis cs:23.547,16);
\draw[draw=black,fill=gray] (axis cs:23.547,0) rectangle (axis cs:25.2025,17);
\draw[draw=black,fill=gray] (axis cs:25.2025,0) rectangle (axis cs:26.858,15);
\draw[draw=black,fill=gray] (axis cs:26.858,0) rectangle (axis cs:28.5135,16);
\draw[draw=black,fill=gray] (axis cs:28.5135,0) rectangle (axis cs:30.169,9);
\draw[draw=black,fill=gray] (axis cs:30.169,0) rectangle (axis cs:31.8245,2);
\draw[draw=black,fill=gray] (axis cs:31.8245,0) rectangle (axis cs:33.48,3);
\end{groupplot}

\end{tikzpicture}
	\caption{Histogram of the data that used for the example with the real data.}
	\label{fig:histogram}
\end{figure}

The three parameters are correlated so this advocates the use of a multivariate KDE. However, as we have seen in the first example, the higher the dimension, the higher the measure for completeness will generally be. So there is a trade-off: Assuming that certain parameters are independent results in an error of the estimated pdf but the resulting MISE, and hence the measure of completeness, will be lower. To illustrate this, we estimate the pdf while assuming all parameters to be dependent and we estimate the pdf while assuming that the average deceleration is independent from the other two parameters. Note that the correlation between the average deceleration and the other parameters is fairly low, so this justifies this choice. The speed difference and end speed are highly correlated, so we will not assume that these two parameters are independent. In this example, $\hat{f}(x;n)$ denotes the estimated 3-dimensional pdf using all three features, $\hat{g}(y;n)$ denotes the estimated univariate pdf of the average deceleration, and $\hat{h}(z;n)$ denotes the estimated bivariate pdf of the speed difference and the end speed. 

\Cref{fig:bandwidth real} shows the bandwidths of the three estimated pdfs for different number of samples, starting from $n=600$ samples to approximately 2800 samples. As opposed to the bandwidths of our previous example, see \cref{fig:bandwidth}, the bandwidth of $\hat{f}(x;n)$ (black dashed line) is not larger than then bandwidth of $\hat{g}(y;n)$ (gray solid line)for low values of $n$. This is caused by some outliers of the average deceleration, because these outliers have a large influence on the bandwidth of $\hat{g}(y;n)$ \cite{hall1992global}. These outliers also influence the bandwidth of $\hat{f}(x;n)$, but this influence is less because the bandwidth of $\hat{f}(x;n)$ is also influences by the other parameters.

\setlength\figurewidth{0.9\linewidth}
\setlength\figureheight{0.7\linewidth}
\begin{figure}
	\centering
	% This file was created by matplotlib2tikz v0.6.17.
\begin{tikzpicture}

\begin{axis}[
xlabel={Number of samples},
ylabel={Bandwidth},
xmin=600, xmax=2786,
ymin=0.109247362356382, ymax=0.366828009723677,
xmode=log,
ymode=log,
width=\figurewidth,
height=\figureheight,
tick align=outside,
tick pos=left,
x grid style={white!69.01960784313725!black},
y grid style={white!69.01960784313725!black},
xtick={600, 800, 1000, 1500, 2000, 2500},
xticklabels={600, 800, 1000, 1500, 2000, 2500},
axis x line*=bottom,
axis y line*=left,
every x tick/.style={black},
every y tick/.style={black}
]
\addplot [line width=0.7000000000000001pt, gray, forget plot]
table {%
600 0.347177088177924
633 0.34415549792894
667 0.336476689247834
703 0.326308673312604
742 0.32006302558553
782 0.323394485869815
824 0.318557989516295
869 0.309534849842418
916 0.302031398158497
966 0.296551720933659
1019 0.290768353886935
1074 0.286621088067857
1133 0.283477189936542
1194 0.283352324315637
1259 0.270809648871598
1328 0.265972255532396
1400 0.267813344019836
1476 0.262658668569752
1556 0.25710726304771
1641 0.250966186455868
1730 0.26134714201301
1824 0.217803682398688
1923 0.219572861610422
2028 0.213334470138317
2138 0.200164649286904
2254 0.156445812526581
2377 0.154803867030243
2506 0.150447516213836
2642 0.146332828305586
2786 0.147635865443625
};
\addplot [line width=0.7000000000000001pt, gray, dotted, forget plot]
table {%
600 0.192723686765745
633 0.1911630941321
667 0.191526673143347
703 0.190974331185176
742 0.190148383476589
782 0.189145925222053
824 0.180658749851802
869 0.176694109786229
916 0.166441140267674
966 0.162961348319603
1019 0.163142615354096
1074 0.15871180245092
1133 0.156642065432789
1194 0.154791534655983
1259 0.151574274155267
1328 0.154409869581638
1400 0.149288696885913
1476 0.149292466516925
1556 0.149292466516925
1641 0.149292466516925
1730 0.149292466516925
1824 0.149292466516925
1923 0.149292466516925
2028 0.149292466516925
2138 0.149292466516925
2254 0.149292466516925
2377 0.149292466516925
2506 0.149292466516925
2642 0.116747556114718
2786 0.115430982819393
};
\addplot [line width=0.7000000000000001pt, black, dashed, forget plot]
table {%
600 0.30374188664615
633 0.301111016006188
667 0.296792524335698
703 0.293975983961105
742 0.290175149188604
782 0.288448132724042
824 0.284603497625661
869 0.283078445805476
916 0.278020533416426
966 0.275106135032696
1019 0.271424618124295
1074 0.266940454643877
1133 0.262533743528913
1194 0.259298456296754
1259 0.25523573778013
1328 0.245897227634794
1400 0.241874251135003
1476 0.241671052396134
1556 0.23545079173757
1641 0.230578581284294
1730 0.228883413086335
1824 0.229167165105227
1923 0.22221522147977
2028 0.217077810032304
2138 0.211225861749565
2254 0.213282549043498
2377 0.20923447598693
2506 0.205127943299399
2642 0.204064263896855
2786 0.201109030617368
};
\end{axis}

\end{tikzpicture}
	\caption{The bandwidths of $\hat{f}(x;n)$ (black dashed line), $\hat{g}(y;n)$ (gray solid line), and $\hat{h}(z;n)$ (gray dotted line) for the example of \cref{sec:result real}. The bandwidths are computed using one-leave-out cross validation for different number of samples $n$.}
	\label{fig:bandwidth real}
\end{figure}

The measures of completeness of the data of the braking activities are shown in \cref{fig:mise real}. The solid gray line results from the estimated 3-dimensional pdf, i.e., $\hat{f}(x;n)$, where \cref{eq:measure} is used to quantify the completeness. The dashed gray line results from the estimated univariate and bivariate pdfs $\hat{g}(y;n)$ and $\hat{h}(z;n)$, where \cref{eq:measure independent} is used to quantify the completeness. The measure for the completeness is much lower for the latter case, indicating that the the uncertainty of the pdf is much lower when it is assumed that the average deceleration is independent from the other two parameters.

\setlength\figurewidth{\linewidth}
\setlength\figureheight{\linewidth}
\begin{figure}
	\centering
	% This file was created by matplotlib2tikz v0.6.17.
\begin{tikzpicture}

\begin{axis}[
xlabel={Number of samples},
xmin=600, xmax=2786,
ymin=0.00189507779777645, ymax=0.00666984596004181,
xmode=log,
ymode=log,
width=\figurewidth,
height=\figureheight,
tick align=outside,
tick pos=left,
x grid style={white!69.01960784313725!black},
y grid style={white!69.01960784313725!black},
xtick={600, 800, 1000, 1500, 2000, 2500},
xticklabels={600, 800, 1000, 1500, 2000, 2500},
axis x line*=bottom,
axis y line*=left
]
\addplot [thick, black, forget plot]
table {%
600 0.00617119091933664
633 0.00611328139296377
667 0.00605721765330875
703 0.00600140679206849
742 0.00594461775444455
782 0.00588990801091082
824 0.00583589672647936
869 0.00578150836748496
916 0.00572813055760985
966 0.00567477164657542
1019 0.00562164672614927
1074 0.00556984782165848
1133 0.00551764123554845
1194 0.00546692408079083
1259 0.00541613068137679
1328 0.00536548081626672
1400 0.00531582702230746
1476 0.00526657217598526
1556 0.00521784794163268
1641 0.00516920610079642
1730 0.00512135299112982
1824 0.00507385809481668
1923 0.00502685315057844
2028 0.00498001264458989
2138 0.00493390650599325
2254 0.00488821442385912
2377 0.00484269206608144
2506 0.00479783366736208
2642 0.00475339106358312
2786 0.00470917577132667
};
\addplot [line width=2.0pt, gray, forget plot]
table {%
600 0.00629905507851159
633 0.0060943185478227
667 0.00602349564042051
703 0.00604214785330053
742 0.00597056277331412
782 0.00586829038317906
824 0.0057448102425999
869 0.0056304149173922
916 0.00564340505408798
966 0.00551378411243598
1019 0.00545276745159859
1074 0.00552506142048785
1133 0.00552984092126124
1194 0.00546736019785219
1259 0.00560897096185979
1328 0.00558412191619361
1400 0.00547498544601871
1476 0.00523100254067616
1556 0.00521884250619987
1641 0.00521936144837386
1730 0.00517092503627769
1824 0.00507718403292554
1923 0.00519661895263249
2028 0.00512715100449337
2138 0.00509416164648494
2254 0.00486192570740022
2377 0.00471732168334417
2506 0.00466865762072244
2642 0.00464634327650706
2786 0.00458157046703965
};
\addplot [thick, black, dashed, forget plot]
table {%
600 0.00313149900516056
633 0.00308770644229241
667 0.0030455043775126
703 0.00300368378875267
742 0.00296132688409245
782 0.00292070934403824
824 0.00288079266291455
869 0.0028407812085917
916 0.00280169345183148
966 0.00276279886926654
1019 0.002724253818273
1074 0.00268684362636012
1133 0.00264931241586993
1194 0.00261301945902607
1259 0.00257683817798545
1328 0.00254092557970226
1400 0.00250588133086331
1476 0.00247127795906354
1556 0.00243720421250117
1641 0.00240334442697221
1730 0.00237018683195343
1824 0.00233742835355726
1923 0.00230515653113186
2028 0.00227314547911659
2138 0.0022417811406495
2254 0.0022108409091033
2377 0.00218015728045536
2506 0.00215006016963421
2642 0.00212037872759722
2786 0.00209098475482998
};
\addplot [line width=2.0pt, gray, dashed, forget plot]
table {%
600 0.00321865226910244
633 0.00316665631453516
667 0.00305307791812238
703 0.0029585581802199
742 0.00282766829703585
782 0.00279108220783996
824 0.00278939620444727
869 0.00272175342539425
916 0.00283388493283317
966 0.00282214552714268
1019 0.00275462511476464
1074 0.00278879039805045
1133 0.00273362889035608
1194 0.00270449897466737
1259 0.00271944828497743
1328 0.0026093498592944
1400 0.0025650693688651
1476 0.00244075253679859
1556 0.00237821815749155
1641 0.00234871590390851
1730 0.00231699031879955
1824 0.00229054118751129
1923 0.00230232978989502
2028 0.00227963536037206
2138 0.00221483490110693
2254 0.00217361171925752
2377 0.00207103025699059
2506 0.00200663065109296
2642 0.00226285691691723
2786 0.00222415898365518
};
\end{axis}

\end{tikzpicture}
	\caption{The measures for completeness of the example of \cref{sec:result real} with the assumption that all three parameters depend on each other (gray solid line) and with the assumption that the first parameter, i.e., the average deceleration, does not depend on the other two parameters (gray dashed line). The corresponding black lines represent the least squares logarithmic fits given by \cref{eq:log mise dependent,eq:log mise independent}.}
	\label{fig:mise real}
\end{figure}

Whether it is better to assume that all parameters are dependent or not depends on the threshold that defines the desired measure and the amount of data. If the threshold is not met, the result can be used to guess how much more data is required by extrapolating the result. To illustrate this, the straight black lines in \cref{fig:mise example} represent the least squares logarithmic fits of the corresponding gray lines that can be used for extrapolation. These straight solid and dashed black lines are described by the formulas
\begin{align}
	0.019 \cdot n^{-0.18}, \label{eq:log mise dependent} \\
	0.017 \cdot n^{-0.26}, \label{eq:log mise independent}
\end{align}
respectively. As an example, let us assume that the threshold equals $0.003$. In that case, $n \approx 800$ would suffice if we assume that the average deceleration is independent from the speed difference and end speed, see the dashed lines in \cref{fig:mise real} and \cref{eq:log mise independent}. This threshold, however, is not yet reached when assuming that all parameters are dependent, see the solid lines in \cref{fig:mise real}. Extrapolating the result using \cref{eq:log mise dependent} provides a rough estimate of the required number of samples: $n \approx 28000$, i.e., ten times as many samples as we used in this example.

