\section{Examples}
\label{sec:results}

In this section, the proposed method of \cref{sec:method} is illustrated by means of two examples. The first example applies the method with data generated from a known distribution. Because the distribution is known, the real MISE can be computed and compared with the results from \cref{eq:measure,eq:measure independent}. Secondly, in \cref{sec:result real}, the proposed method is applied on a dataset containing naturalistic driving data.

\subsection{Example with known underlying distribution}
\label{sec:result artificial}

\setlength\figurewidth{\linewidth}
\setlength\figureheight{0.7\linewidth}
\begin{figure}
	\centering
	% This file was created by matplotlib2tikz v0.6.17.
\begin{tikzpicture}

\begin{axis}[
xlabel={$x$},
xmin=-3, xmax=3,
ymin=-0.0184547497061878, ymax=0.387729405483734,
width=\figurewidth,
height=\figureheight,
tick align=outside,
tick pos=left,
xmajorgrids,
x grid style={white!69.01960784313725!black},
ymajorgrids,
y grid style={white!69.01960784313725!black}
]
\addplot [line width=2.0pt, black, forget plot]
table {%
-3 0.0051667463394783
-2.94 0.00654461880712224
-2.88 0.00823047028690945
-2.82 0.0102763296308067
-2.76 0.0127386815677213
-2.7 0.0156777601691089
-2.64 0.0191565206269158
-2.58 0.023239261718724
-2.52 0.0279898842884903
-2.46 0.0334697879276774
-2.4 0.0397354284821519
-2.34 0.0468355823644219
-2.28 0.0548083888784006
-2.22 0.0636782674682525
-2.16 0.0734528312802731
-2.1 0.0841199397542916
-2.04 0.0956450491215403
-1.98 0.107969028724005
-1.92 0.121006611269839
-1.86 0.134645635208032
-1.8 0.148747216648567
-1.74 0.163146956719893
-1.68 0.177657248832099
-1.62 0.192070700762076
-1.56 0.206164631385174
-1.5 0.219706544535379
-1.44 0.232460426663028
-1.38 0.244193664665147
-1.32 0.254684339351727
-1.26 0.263728621854501
-1.2 0.271147987405017
-1.14 0.276795964656363
-1.08 0.28056415903362
-1.02 0.282387323863288
-0.96 0.282247300114256
-0.9 0.280175699999137
-0.84 0.2762552659681
-0.78 0.270619888859475
-0.72 0.263453311461989
-0.66 0.254986571801805
-0.6 0.245494251193712
-0.54 0.235289585090982
-0.48 0.224718472646555
-0.42 0.214152389477757
-0.36 0.203980176189738
-0.3 0.194598653753013
-0.24 0.186402017773098
-0.18 0.179769998141526
-0.12 0.175054846746348
-0.0600000000000001 0.17256733708436
0 0.172562122174252
0.0600000000000001 0.17522298954899
0.12 0.180648754601943
0.18 0.188840719481647
0.24 0.199692762934888
0.3 0.212985185217932
0.36 0.228383383748615
0.42 0.24544226096369
0.48 0.263616960871767
0.54 0.282280106795447
0.6 0.300745199333918
0.66 0.318295276493306
0.72 0.334215395119643
0.78 0.347827027559912
0.84 0.358522140450913
0.9 0.365794582910501
0.96 0.369266489338738
1.02 0.368707703149793
1.08 0.364046730896733
1.14 0.355372395028413
1.2 0.342926101534303
1.26 0.327085397833125
1.32 0.308340186572005
1.38 0.287263511255486
1.44 0.264479186177656
1.5 0.240628676360264
1.56 0.216339540352097
1.62 0.192197452977527
1.68 0.168723371051431
1.74 0.146356851706521
1.8 0.125445945226761
1.86 0.106243524463756
1.92 0.0889094334160046
1.98 0.0735174756140632
2.04 0.0600660380843174
2.1 0.0484910607031717
2.16 0.038680100242474
2.22 0.0304863785692694
2.28 0.0237419140051494
2.34 0.0182690807761867
2.4 0.0138901932543357
2.46 0.0104349442798379
2.52 0.00774572250174328
2.58 0.00568098236166017
2.64 0.00411693920594205
2.7 0.00294791398464616
2.76 0.0020856640947567
2.82 0.00145801842773233
2.88 0.00100709550737067
2.94 0.000687333044196065
3 0.000463503099710759
};
\addplot [line width=2.0pt, gray, dashed, forget plot]
table {%
-3 8.1664388086521e-06
-2.94 1.32735202490426e-05
-2.88 2.13387791689166e-05
-2.82 3.39258027886395e-05
-2.76 5.33362290729395e-05
-2.7 8.29100251642051e-05
-2.64 0.000127423906520495
-2.58 0.000193608915017404
-2.52 0.000290807792346734
-2.46 0.000431789798324551
-2.4 0.000633734019064902
-2.34 0.00091938096234404
-2.28 0.00131833546725512
-2.22 0.00186848114109809
-2.16 0.00261743777187522
-2.1 0.00362395943191086
-2.04 0.00495913440614299
-1.98 0.00670721206951642
-1.92 0.00896585114329068
-1.86 0.011845564243823
-1.8 0.0154681318232826
-1.74 0.0199637809273986
-1.68 0.0254669760107816
-1.62 0.0321107534491946
-1.56 0.0400196480213446
-1.5 0.0493014037079956
-1.44 0.0600378228547278
-1.38 0.0722752722788968
-1.32 0.0860155133804601
-1.26 0.101207634482779
-1.2 0.117741916385013
-1.14 0.135446438583287
-1.08 0.1540871223875
-1.02 0.173371706088462
-0.96 0.192957865503134
-0.9 0.212465351490892
-0.84 0.231491645817723
-0.78 0.249630277094908
-0.72 0.266490631597324
-0.66 0.281717879345709
-0.6 0.295011545561037
-0.54 0.306141309919284
-0.48 0.314958812686675
-0.42 0.321404571515163
-0.36 0.325509532392705
-0.3 0.327391246981737
-0.24 0.327245133223743
-0.18 0.325331683437325
-0.12 0.32196078811603
-0.0600000000000001 0.317474511566724
0 0.312229672541257
0.0600000000000001 0.306581453876023
0.12 0.300869013353034
0.18 0.295403732183918
0.24 0.290460365912753
0.3 0.286271006120811
0.36 0.283021466840972
0.42 0.280849513311206
0.48 0.279844273733986
0.54 0.280046220548248
0.6 0.281447262148016
0.66 0.283990719381609
0.72 0.287571232834903
0.78 0.292034910796478
0.84 0.297180238960591
0.9 0.302760393757091
0.96 0.308487606833803
1.02 0.314040110199294
1.08 0.319071959355165
1.14 0.323225711965173
1.2 0.326147572588347
1.26 0.327504248653923
1.32 0.327000450029882
1.38 0.324395749979032
1.44 0.319519443140001
1.5 0.312282103937755
1.56 0.302682764707317
1.62 0.290810975654381
1.68 0.276843440838287
1.74 0.261035396430749
1.8 0.243707355588743
1.86 0.225228236896635
1.92 0.205996178247795
1.98 0.186418487497521
2.04 0.166892184938355
2.1 0.147786458198762
2.16 0.129428100879865
2.22 0.112090677090447
2.28 0.0959877862306995
2.34 0.0812704376110851
2.4 0.0680282197135535
2.46 0.0562936922511944
2.52 0.0460492576282755
2.58 0.037235687154538
2.64 0.0297614809220083
2.7 0.0235123146900695
2.76 0.0183599531400235
2.82 0.014170164981365
2.88 0.0108093409047305
2.94 0.00814967260779642
3 0.0060728868851914
};
\end{axis}

\end{tikzpicture}
	\caption{The true probability density functions that are used to illustrate the quantification of the completeness.}
	\label{fig:true pdf}
\end{figure}

% We use one-leave-out cross validation to compute the bandwidth $h$ (see also \textcite{duin1976parzen}) because this minimizes the Kullback-Leibler divergence between the real pdf $f(x)$ and the estimated pdf $\hat{f}(x;n)$ \cite{turlach1993bandwidthselection,zambom2013review}.

\subsection{Example with real data}
\label{sec:result real}

