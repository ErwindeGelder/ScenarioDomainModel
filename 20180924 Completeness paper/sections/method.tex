\section{Method}
\label{sec:method}

% Introduce f(x) and fhat(x;n)
In this section, we present how to quantify the completeness regarding the activities. As explained in \cref{sec:problem}, all scenarios that fall into a specific scenario class are parametrized similarly. Therefore, all similar types of activities are also parametrized similarly. For example, all activities labeled ``braking'' are parametrized similarly. In the remainder of this section, we assume that all activities that we are dealing with are a similar type of activities, such that they are parametrized similarly. 

Let $n$ denote the number of activities such that we have $n$ parameter vectors that describe these activities, denoted by $X_i \in \mathbb{R}^d$ with $i\in \{1,\ldots,n\}$ and $d$ denoting the number of parameters for one activity. We will estimate the underlying distribution of $X_i$. Let $f(\cdot)$ denote the true probability density function (pdf) and let $f(x)$ denote the probability density evaluated at $x$. Similarly, let $\hat{f}(\cdot;n)$ denote the estimated pdf using $n$ parameter vectors.

% Introduce MISE
To quantify the completeness of the collection of the $n$ activities, we use the estimated pdf $\hat{f}(\cdot;n)$. For example, suppose that $\hat{f}(x;n)$ equals $f(x)$ for all $x \in \mathbb{R}^d$. In this case, it would be reasonable to say that the $n$ activities give a complete view of the variety and the distribution of the different activities that are labeled similarly. On the other hand, when $\hat{f}(x;n)$ is very different from $f(x)$, it would be reasonable to say that the opposite is the case, i.e., the $n$ scenarios do not give a complete view. One common measure for comparing the estimated pdf with the true pdf is the Mean Integrated Squared Error (MISE):
\begin{equation}
	\label{eq:mise}
	\mise{f}{n} = \expectation{\int_{\mathbb{R}^d} \left( f(x) - \hat{f}(x;n) \right)^2 \ud x}.
\end{equation}
The index $f$ indicates that the MISE is calculated with respect to the pdf $f(\cdot)$.

A low MISE indicates a high degree of completeness whereas a high MISE indicates a low degree of completeness, because the expected integrated squared error is high. Therefore MISE can be used to quantify the completeness of set of activities that are of a similar type. The problem is, however, that \cref{eq:mise} depends on the true pdf $f(\cdot)$ which is unknown. So the MISE of \cref{eq:mise} cannot be evaluated.

In the remainder of this section, we will explain how the MISE of \cref{eq:mise} can be estimated when Kernel Density Estimation (KDE) is employed. First, KDE will be explained. Next, in \cref{sec:mise dependent}, a method is presented for estimating the MISE when assuming that the $d$ parameters are correlated. \Cref{sec:mise independent} shows how the MISE can be approximated when some of the $d$ parameters are independent from each other.

% Explain everything about KDE
\subsection{Estimating the distribution using Kernel Density Estimation}
\label{sec:kde}

The shape of the probability densities is unknown beforehand. Furthermore, the shape of the estimated pdf might change as more data are acquired. Assuming a functional form of the pdf and fitting the parameters of the pdf to the data may therefore lead to inaccurate fits unless a lot of hand-tuning is applied. We employ a non-parametric approach using Kernel Density Estimation (KDE) \cite{rosenblatt1956remarks, parzen1962estimation} because the shape of the pdf is automatically computed and KDE is highly flexible regarding the shape of the pdf.

In KDE, the estimated pdf is given by
\begin{equation}
	\label{eq:kde}
	\hat{f}(x;n) = \frac{1}{nh^d} \sum_{i=1}^n K\left(\frac{x - X_i}{h}\right).
\end{equation}
Here, $K(\cdot)$ is an appropriate kernel function and $h$ denotes the bandwidth. The choice of the kernel $K(\cdot)$ is not as important as the choice of the bandwidth $h$ \cite{turlach1993bandwidthselection}. We use a Gaussian kernel because it will simplify some of our calculations. The Gaussian kernel is given by
\begin{equation}
	\label{eq:gaussian kernel}
	K(u) = \frac{1}{\left( 2\pi \right)^{d/2}} \exp \left\{ -\frac{1}{2} \|u\|^2 \right\},
\end{equation}
where $\|u\|^2$ denotes the squared 2-norm of $u$, i.e., $u^T u$.

The bandwidth $h$ controls the amount of smoothing. For the kernel of \cref{eq:gaussian kernel}, the same amount of smoothing is applied in every direction, although our method can easily be extended to a multi-dimensional bandwidth, see, e.g., \textcite{scott2005multidimensional, chen2017tutorial}. There are many different ways of estimating the bandwidth, ranging from simple reference rules like, e.g., Scott's rule of thumb \cite{scott2015multivariate} or Silverman's rule of thumb \cite{silverman1986density} to more elaborate methods; see \textcite{turlach1993bandwidthselection, bashtannyk2001bandwidth, jones1996brief, chiu1996comparative} for reviews of different bandwidth selection methods. 

% Explain how MISE is computed
\subsection{Estimating the Mean Integrated Squared Error for dependent parameters}
\label{sec:mise dependent}

As an approximation of the MISE of \cref{eq:mise}, the asymptotic mean integrated squared error (AMISE) is often used. With the KDE of \cref{eq:kde} employed, the AMISE is defined as follows \cite{marron1992exact}:
\begin{equation}
	\label{eq:amise}
	\amise{f}{n} = \frac{h^4}{4} \sigma_K^4 \int_{\mathbb{R}^d} \left( \nabla^2 f(x) \right)^2 \ud x + \frac{\mu_K}{nh^d}.
\end{equation}
Here, $\sigma_K$ and $\mu_K$ are constants that depend on the choice of the kernel $K(\cdot)$:
\begin{align}
	\sigma_K &= \int_{\mathbb{R}^d} u^2 K(u) \ud u, \\
	\mu_K &= \int_{\mathbb{R}^d} K(u)^2 \ud u.
\end{align}
Because we use the Gaussian kernel of \cref{eq:gaussian kernel}, we have $\sigma_K=1$ and $\mu_K=(2\sqrt{\pi})^{-d}$. In \cref{eq:amise}, $\nabla^2 f(x)$ denotes the Laplacian of $f(x)$, i.e., 
\begin{equation}
	\nabla^2 f(x) = \sum_{l=1}^d \frac{\partial^2 f(x)}{\partial x_l^2}.
\end{equation}
Note that the Laplacian equals the trace of the Hessian. Assuming that $h \rightarrow 0$ and $nh^d \rightarrow \infty$ as $n \rightarrow \infty$, the AMISE only differs from the MISE by higher-order terms \cite{silverman1986density}.

The influence of the bandwidth $h$ is demonstrated in an illustrative way by the AMISE of \cref{eq:amise}. The first term of the AMISE of \cref{eq:amise} corresponds to the asymptotic bias introduced by smoothing the pdf. Therefore, this term approaches zero when $h \rightarrow 0$. However, when $h \rightarrow 0$, the variance goes to infinity, as can be seen by the second term of the AMISE, which corresponds to the asymptotic variance. 

As with the MISE, we cannot evaluate the AMISE because it depends on the true pdf $f(\cdot)$. As suggested by \textcite{chen2017tutorial, calonico2018effect}, we can estimate the quantity $\nabla^2 f(x)$ by $\nabla^2 \hat{f}(x;n)$, with $\hat{f}(x;n)$ defined in \cref{eq:kde}. Given that the Gaussian kernel of \cref{eq:gaussian kernel} is used, we have
\begin{equation}
	\label{eq:laplacian gaussian}
	\nabla^2 \hat{f}(x;n) = \frac{1}{nh^{d+2}} \sum_{i=1}^n K\left( \frac{x - X_i}{h} \right) \left( \frac{\left(x-X_i\right)^2}{h^2} - d \right).
\end{equation}
Substituting $f(x)$ in \cref{eq:amise} with $\hat{f}(x;n)$ gives the measure that we will use to quantify the completeness:
\begin{equation}
	\label{eq:measure}
	\measure{f}{n} = \frac{h^4}{4} \sigma_K^4 \int_{\mathbb{R}^d} \left( \nabla^2 \hat{f}(x;n) \right)^2 \ud x + \frac{\mu_K}{nh^d}.
\end{equation}

% Explain how MISE can be estimated with independent data
\subsection{Estimating the Mean Integrated Squared Error for independent parameters}
\label{sec:mise independent}

As explained in \cref{sec:kde}, KDE is employed because the KDE is highly flexible regarding the shape of the pdf. However, when a large number of parameters are used, i.e., for large values of $d$, the KDE becomes unreliable due to the curse of dimensionality \cite{scott2015multivariate}. One way to overcome this, is to assume that certain parameters are independent. 

Without loss of generality, consider the parameter vector $x$ that can be decomposed into two parts:
\begin{equation}
	\label{eq:combine}
	x = \begin{bmatrix}
		y \\ z
	\end{bmatrix},
\end{equation}
such that $y \in \mathbb{R}^{d_y}$ and $z \in \mathbb{R}^{d_z}$ with $d_y+d_z=d$. If the parameter vectors $y$ and $z$ are independent, the probability density of $x$ equals
\begin{equation}
	\label{eq:independency}
	f(x) = g(y) h(z),
\end{equation}
where $g(\cdot)$ and $h(\cdot)$ are pdfs.
Because $y$ and $z$ have a lower dimensionality than $x$, the estimated pdfs of $g(\cdot)$ and $h(\cdot)$ will be more reliable. However, we cannot use the measure of \cref{eq:measure} to quantify the completeness anymore. Therefore, we will show in this section how $\measure{f}{n}$ can be computed in case the real distribution is assumed to take the form \cref{eq:independency}.

The first step is to estimate $g(\cdot)$ and $h(\cdot)$ using $\hat{g}(\cdot;n)$ and $\hat{h}(\cdot;n)$, respectively, where $\hat{g}(\cdot;n)$ and $\hat{h}(\cdot;n)$ are similarly defined as $\hat{f}(\cdot;n)$ in \cref{eq:kde}. Note that the bandwidths of $\hat{g}(\cdot;n)$ and $\hat{h}(\cdot;n)$ are generally different. Now let the MISE of $g(\cdot)$ and $h(\cdot)$ be defined similar as the MISE of $f(\cdot)$ in \cref{eq:mise}. It can be shown\footnote{For the sake of brevity, the proof is omitted from this paper. The main idea is based on the variance of the product of two independent variables, see \textcite{goodman1960exact}, and the assumptions $\expectation{\hat{g}(y;n)} \approx g(y)$ for all $y$ and $\expectation{\hat{h}(z;n)} \approx h(z)$ for all $z$.} that if \cref{eq:independency} holds, then the MISE of $f(x)$ approximately equals
\begin{dmath}
	\label{eq:mise independent}
	\mise{f}{n} \approx \mise{g}{n} \int_{\mathbb{R}^{d_z}} h(z)^2 \ud z + \mise{h}{n} \int_{\mathbb{R}^{d_y}} g(y)^2 \ud y + \mise{g}{n} \cdot \mise{h}{n}.
\end{dmath}

We can estimate the MISE of $g(\cdot)$ and $h(\cdot)$ in a similar manner as we did for the MISE of $f(\cdot)$ in \cref{sec:mise dependent}, such that we obtain $\measure{g}{n}$ and $\measure{h}{n}$. Since we cannot evaluate the integrals of \cref{eq:mise independent}, we estimate them by substituting the estimated pdfs. As a result, we have
\begin{dmath}
	\label{eq:measure independent}
	\measure{f}{n} = \measure{g}{n} \int_{\mathbb{R}^{d_z}} \hat{h}(z;n)^2 \ud z + \measure{h}{n} \int_{\mathbb{R}^{d_y}} \hat{g}(y;n)^2 \ud y + \measure{g}{n} \cdot \measure{h}{n}.
\end{dmath}

In this section, we assumed that the parameters $x$ can be split into two partitions that are independent. It is straightforward to extend the result of \cref{eq:measure independent} in case that the parameters $x$ can be split into three of more partitions. 
