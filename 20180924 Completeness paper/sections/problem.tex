\section{Problem definition}
\label{sec:problem}

% Explain that we want to quantify the amount of data
The required amount of data depends on the use of the data \cite{wang2017much}. For example, when investigating (near)-accident scenarios, more data might be required compared to studying nominal driving behavior, because of the low probability of having a (near)-accident scenario. Therefore, in this paper, the goal is to define a quantitative measure for the completeness of the data. To determine whether the data are enough for a specific use, this quantitative measure can be used, but determining the exact threshold is outside the scope of this paper.

% Explain assumptions

Before splitting the problem of quantifying the completeness of the data, few assumptions are made:
\begin{enumerate}
	\item The data are interpreted as an endless sequence of scenarios, where scenarios might overlap in time \cite{elrofai2018scenario}. The term scenario in the context of traffic data is defined by, among others, \textcite{geyer2014, ulbrich2015, elrofai2016scenario, elrofai2018scenario}. Because we want to differentiate between quantitative and qualitative descriptions, the definition of the term scenario is adopted from \textcite{elrofai2018scenario} as it explicitly defines a scenario as a quantitative description: ``A scenario is a quantitative description of the ego vehicle, its activities and/or goals, its dynamic environment (consisting of traffic environment and conditions) and its static environment. From the perspective of the ego vehicle, a scenario contains all relevant events.''
	
	\item Whereas a scenario refers to a quantitative description, these scenarios can be abstracted by means of a qualitative description, referred to as scenario class, see also \textcite{ploeg2018cetran, elrofai2018scenario}. An example of a scenario class could have the name ``ego vehicle braking'', i.e., this scenario qualitatively describes a scenario in which the ego vehicle brakes. An actual (real-world) scenario in which the ego vehicle is braking would fall into this scenario class. It is assumed that all scenarios can be categorized into these scenario classes. This assumption does not necessarily limit the applicability of this paper. However, it might require a large number of scenario classes to describe all scenarios that are in the data.
	
	\item It is assumed that all scenarios that fall into a specific scenario class can be parametrized similarly. As with the previous assumption, this does not necessarily limit the applicability of this paper. However, it might constrain the variety of scenarios that fall into a scenario class. 
\end{enumerate}

% Define subproblems
Describing the data in terms of scenarios, we can now, instead of quantifying the completeness of the data, quantify the completeness of the scenarios. Furthermore, having the distinction between qualitative scenarios, i.e., scenario classes, and quantitative scenarios, we can describe the problem of quantifying the completeness of scenarios into two subproblems:
\begin{enumerate}
	\item How to quantify the completeness regarding the scenario classes?
	\item How to quantify the completeness regarding all scenarios that fall into a specific scenario class?
\end{enumerate}

% Why splitting problem?
% Comparing different scenario types directly is like comparing apples and oranges
The first subproblem can be regarded as the so-called unseen species problem \cite{bunge1993estimating} or species estimation problem \cite{yang2012estimating}. In case of the unseen species problem, the entire population is partioned into $C$ classes and the objective is to estimate $C$ given only a part of the entire population. To continue the analogy, the second subproblem relates to quantifying whether we have a complete view on the variety among one species, given the number of individuals that we have seen. Both these problems require a different approach, which is why the main problem is divided into the two subproblems.

% Mention that we only address sub-problem 2 (and why)
In this paper, only the second subproblem is addressed. Because of the different approach required for answering the first subproblem, this will be addressed in a forthcoming paper.
