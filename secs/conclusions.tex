\section{Conclusions}
\label{sec:conclusions}

% Safety is important
Road safety is an important research topic because of the societal and economical losses caused by accidents.
% Quantify safety with surrogate metrics
To quantify the safety at a vehicle level, use is made of \acp{ssm} that characterize the risk of a collision. 
% What we did
We have proposed a method for deriving \iac{ssm} that calculates the probability that a certain event, e.g., a collision, will happen in the near future.
% Advantages of our method
With our data-driven approach, it is possible to adapt the \ac{ssm} to the local traffic context.
Besides, the presented method could be applied for various types of scenarios.
We have illustrated that our method is a generalization of already existing \acp{ssm}.
In an example, we have derived \iac{ssm} based on the \ac{ngsim} data set.
Through few explanatory scenarios, it has been shown that the derived \ac{ssm} provides a quantification of the collision risk.
We have also presented how the evaluation of the partial derivatives of the \ac{ssm} can be used to benchmark \iac{ssm} with few expected causal tendencies.

%Concluding remarks about our method
Our proposed method has the potential for deriving multiple \acp{ssm} for quantifying the safety of a --- possibly automated --- driver.
These metrics can be used to warn drivers for unsafe situations and ensuring that proper attention is being paid to the road situation.
Furthermore, the metrics can measure the impact of newly introduced systems on the traffic safety.
% Future work
A limitation of the current study is that the presented approach is only applied to longitudinal traffic interactions. 
Future work involves the consideration of lateral traffic interactions and interactions with vulnerable road users. 
