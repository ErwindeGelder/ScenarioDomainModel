\section{Conclusions}
\label{sec:conclusions}

% Safety is important
Road safety is an important research topic because of the societal and economical losses caused by accidents.
% Quantify safety with surrogate metrics
To quantify the safety at a vehicle level, use is made of \acp{ssm} that characterize the risk of a collision. 
% What we did
\cstarta We have proposed a novel approach called the \ac{ourmethod} method for deriving \acp{ssm} that calculate the probability that a certain event, e.g., a collision, will happen in the near future. 
% Advantages of our method
Whereas traditional \acp{ssm} are generally only applicable in certain types of scenarios, the \ac{ourmethod} method could be applied to various types of scenarios.
Furthermore, because the \ac{ourmethod} method is data-driven, the derived \acp{ssm} can be adapted to the local traffic behavior. 
Also, no assumptions on the driver behavior are made.
Therefore, the \ac{ourmethod} method has the potential for deriving multiple \acp{ssm} for quantifying the safety of a --- possibly automated --- vehicle.

We have illustrated that the \ac{ourmethod} method can be used to reproduce known probabilistic \acp{ssm}.
In an example, we have derived a new \ac{ssm} based on the \ac{ngsim} data set that calculates the risk of a collision in a longitudinal interaction between two vehicles.
Through several explanatory scenarios, it has been shown that the derived \ac{ssm} correctly provides a quantification of the collision risk.
We have also presented how the evaluation of the partial derivatives of the \ac{ssm} can be used to benchmark \iac{ssm} using expected risk tendencies. \cenda

%Concluding remarks about our method
The \acp{ssm} derived using the presented \ac{ourmethod} method can be used to warn drivers for unsafe situations and ensuring that proper attention is being paid to the road situation.
Furthermore, the derived metrics can measure the impact of newly introduced systems on the traffic safety. 
% Future work
A limitation of the current study is that the presented approach is only applied to longitudinal traffic interactions. 
\cstarta Future work involves applying the \ac{ourmethod} method for the derivation of \acp{ssm} that measure the risk of lateral traffic interactions and interactions with vulnerable road users. \cenda
