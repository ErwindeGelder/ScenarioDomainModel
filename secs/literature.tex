\section{Literature review}
\label{sec:literature review}

Risk in the context of traffic safety is \cstartc often \cendc defined as the probability of an accident occurring \autocite{hakkert2002uses}.
\cstartb As stated in the introduction, most \acp{ssm} are derived under specific assumptions of the expected behavior of the  driving participants under a specific  driving scenario. \cendb
Several \acp{ssm} have been developed \cstartb under such assumptions \cendb with the goal of quantifying the risk involved in driving in traffic on the road \autocite{minderhoud2001extended, ozbay2008derivation, cunto2009simulated, laureshyn2010evaluation}.
In general, the risk is quantified in terms of the proximity between two traffic agents in time and/or space, the ability to perform evasive actions like braking or swerving, or the magnitude of such actions \autocite{shi2018key,zheng2020modeling}. 
In a potential collision situation, the proximity index is close to zero while the magnitude of evasive action is close to the limits of the driver and the vehicle \autocite{zheng2020modeling}. 
\cstartb The above clustering of \acp{ssm} in terms of time, space, and evasive action is common in literature, so our literature review follows this pattern of clustering \acp{ssm}.
We focus on the most commonly used measures in each cluster and their underlying assumptions. \cendb

The most common \acp{ssm} are time-based. 
A popular time-based proximity indicator is the \ac{ttc}, which is defined as the time remaining until two vehicles collide if they would continue on the same course and speed \autocite{hayward1972near}.
\cstartb The assumption for \ac{ttc} is that the relative speed and course will remain the same. 
In addition, \ac{ttc} is only relevant when two objects are approaching each other. 
These assumptions make it difficult to use it  for various driving scenarios. \cendb
Several other time-based \acp{ssm} have been derived from or based on the \ac{ttc}. 
Notable among those are 
\begin{itemize}
    \item the time exposed \ac{ttc}, which measures the amount of time the \ac{ttc} is below a certain threshold \autocite{minderhoud2001extended}, 
    \item the \ac{tit}, which calculates the total area in \iac{ttc} versus time diagram where the \ac{ttc} is below a certain threshold  \autocite{minderhoud2001extended} and,
    \item the \ac{mttc}, which is able to calculate the \ac{ttc} for cases where vehicles do not keep a constant speed and the follower is slower than the leader \autocite{ozbay2008derivation}.
\end{itemize}
\cstartb For the \ac{mttc}, the relative speed is not assumed to be constant, but new assumptions on the  acceleration and speed of the objects are introduced. \cendb 

Other time-based proximity indicators include \ac{pet}, which measures the ``time between the moment that a vehicle leaves the area of potential collision, i.e., the area in which the paths of the two vehicles intersect, and the other vehicle arrives in the same area'' \autocite{mahmud2017application} and \ac{thw}.
\cstartb \ac{pet} can only be calculated when the collision area of the two participants is known. 
This assumption makes it mostly useful for scenarios with obvious crossing conflicts like intersections. \cendb

For distance-based proximity indicators, the \ac{picud} measures the remaining distance between vehicles during an emergency stop \autocite{iida2001traffic, uno2003objective} and the \ac{psd} measures the remaining distance to the potential point of collision divided by the minimum acceptable stopping distance \autocite{allen1978analysis, guido2011comparing, mahmud2017application}. 
\cstartb These two measures assume that the vehicles will apply the maximum deceleration during emergency situations. 
This makes them suitable for emergency situations for which the assumption will most likely hold. 
For non-critical situations, however, the deceleration that the drivers will apply may vary. \cendb
%Other similar distance-based \acp{ssm} are the ``margin to collision'' and ``difference of space distance and stopping distance'', which are computed similar to the \ac{psd} and the \ac{picud}, respectively \autocite{kitajima2009estimation, okamura2011impact}. 
More recently, a distance-based measure that assumes ``correct'' driving behavior has been proposed \autocite{shalev2017formal}. 
This measure calculates the minimum safety distance between a follower and its leader, such that no collision occurs if the leader vehicle brakes with a specified deceleration and the follower brakes after a specified reaction time with another specified deceleration. 
\cstartb Based on the definition of this measure, it is not suitable for driving situations where the driver does not follow the description of ``correct'' driving given above. \cendb

In terms of indicators relating to performing evasive actions, the \ac{drac} is the most widely used. 
The \ac{drac} is calculated as the ratio of the difference in speed between a following vehicle and a leading vehicle and their closing time \autocite{almqvist1991use, mahmud2017application}. 
Another indicator is the \ac{cpi}, which calculates the probability that a vehicle's \ac{drac} will exceed its \ac{madr} in a given time interval \autocite{cunto2009simulated}. 
\cstartc The \ac{drac} does not really have a risk measurement on its own, if it is not compared with the braking capacity.
It will be a minimum braking requirements and does not give an indication of risk.
This is a limitation and this is why the measure \ac{cpi} has been developed. \cendc
\cstartb Both \ac{drac} and \ac{cpi} are mostly suitable for a car-following situation and not suitable for lateral movements \autocite{mahmud2017application}. \cendb

\textcite{wang2014evaluation} derive a ``crash propensity metric'' using a combination of the \ac{ttc}, the vehicle braking capability, and the driver reaction time. 
Although this metric is suitable for various car-following situations, lane-change and crossing conflicts, it is limited because it uses \ac{ttc} in its calculations, so this metric is undefined when the \ac{ttc} is undefined. 
In addition, it assumes that the driver will keep a constant speed before reacting and, after a reaction time, the driver will apply the maximum deceleration. \cendb
% \cstartb This assumption may not hold for all situations except for emergency situations. \cendb

%Criticality Index(CI) which is the ratio of the square of the speed and the TTC. 
%Criticality index measures the risk and the severity of collision. It can also be interpreted as the product of the speed and the required deceleration rate \autocite{chan2006defining}.

\textcite{shi2018key} use indicators like \ac{tit}, \ac{cpi}, and \ac{psd} to measure the effectiveness of risk indicators for predicting accidents. 
The idea is that some indicators are able to distinguish accidents from non-accidents. 
These indicators are called key risk indicators. 
\cstartb Even though these are new indicators, they inherit the assumptions of other indicators because the key risk indicators are calculated based on \ac{psd}, \ac{ttc}, and \ac{cpi}. \cendb
\textcite{mullakkal2020probabilistic} propose a probabilistic driving risk field.
The method derives the risk a vehicle is exposed to using a kinematic approach with the inclusion of uncertainty in the vehicle's future state. 
\textcite{mullakkal2020probabilistic} define this for an encounter between the ego vehicle and a road obstacle, such as other vehicles or objects. 
\cstartb This research shares similar ideas with our proposed method of risk estimation, but it does not use a data-driven approach to derive the \ac{ssm}. 
Furthermore, the future state of the vehicle is estimated with a fixed distribution (i.e., a normal distribution). 
This limits the application in scenarios where the data may have an entirely different distribution. \cendb

%Apart from the traditional proximity indicators, efforts have been made to characterize what is considered as ``safe driving'' using a combination of risk indicators.
%Some studies have used the fundamental relationship between velocity and acceleration to classify whether a vehicle's driving style  is ``safe'' or ``unsafe''. 
%\textcite{eboli2016combining} use a vehicle's acceleration and velocity to create a safety boundary.
%If and only if the vehicle's velocity and acceleration are within that boundary, then it is considered safe \autocite{eboli2016combining}.
%\cstartb Although these indicators are suitable for various situations, they are derived using a fixed model and, thus, may not be applicable for all data obtained from different locations.
%Another limitation of current \acp{ssm} is that the input parameters used for the calculation and the equations for such metrics are fixed. 
%In contrast, our method can use any set of input parameters to calculate the probability of collision.\cendb
