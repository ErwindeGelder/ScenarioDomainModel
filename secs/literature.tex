\section{Literature review}
\label{sec:literature review}

Several risk metrics have been developed with the goal of quantifying the risk involved in driving \autocite{minderhoud2001extended, ozbay2008derivation, cunto2009simulated, laureshyn2010evaluation}.
In general, the risk is measured in terms of proximity in time and/or space, ability to perform evasive action like braking or swerving and the magnitude of such actions \autocite{shi2018key,zheng2020modeling}. 
For a collision to occur, the proximity index should be close to zero while the magnitude of evasive actions should be close to the limits of the driver and the vehicle \autocite{zheng2020modeling}. 

The most common indicators are time-based. By far, the most  popular time-based proximity indicator is the \ac{ttc} which  is defined as the time remaining until two vehicles collide if they continue on the same collision course and speed\autocite{van1990time,hyden1996traffic}. Several other time-based indicators have been derived from  or based on the \ac{ttc}. Notable among those are the \ac{mttc} which is able to  calculate \ac{ttc} for case when vehicles do not keep constant speed and follower is slower than leader\autocite{ozbay2008derivation}. Time Exposed Time to Collision(TET), which measures the amount of time \ac{ttc} is below a certain threshold, \ac{tit} which calculates the total area where \ac{ttc} is below a certain threshold\autocite{minderhoud2001extended}. Other time-based proximity indicators include time headway, and \ac{pet} which measures the ``time between the moment that a vehicle leaves the area of potential collision and the other vehicle arrives collision area'' \autocite{mahmud2017application}. 

For distance based proximity indicators, the Potential Index for Collision with Urgent Deceleration (PICUD) measures the remaining distance between vehicles during an emergency stop\autocite{iida2001traffic,uno2003objective} and the Proportion of Stopping Distance (PSD) which is  the remaining distance to the potential point of collision divided by the minimum acceptable stopping distance\autocite{allen1978analysis,guido2011comparing,mahmud2017application} are commonly used. Other similar distance-base metric are Margin To Collision(MTC)  and  Difference of Space distance and Stopping distance(DSS) which are calculated similar to PSD and PICUD respectively\autocite{kitajima2009estimation,okamura2011impact}. Recently a distance based-metric which assumes `correct' driving behavior was proposed\autocite{shalev2017formal}. The proposed distance based metric calculates the minimum safety distance between the front of follower and rear of leader during a situation where the leader breaks with a maximum deceleration . The follower is assumed to accelerate with maximum acceleration during the reaction time and finally breaks with at least the minimum deceleration after that to avoid the stop. 

In terms of  indicators relating to  performing evasive actions the Deceleration Rate to Avoid Collision(DRAC) is the most widely used. The DRAC is calculated as  the ratio of the Differential speed between a following vehicle and lead vehicle   and their closing time\autocite{almqvist1991use,mahmud2017application}. Another indicator is the Crash Potential Index(CPI) which calculates the probability that a vehicle's DRAC will exceed its maximum available deceleration rate (MADR) in a given time interval\autocite{cunto2009simulated} Criticality Index(CI) which is the ratio of the square of the speed and the TTC.  Criticality index measures the risk and the severity of collision. It can also be interpreted as the product of the speed and the required deceleration rate \autocite{chan2006defining}.

Refer to \cref{fig:tux}.

\begin{figure}
    \centering
    \includegraphics{figs/tux.png}
    \caption{This is an example of a figure.}
    \label{fig:tux}
\end{figure}
