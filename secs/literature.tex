\section{Literature review}
\label{sec:literature review}

Risk in the context of traffic safety is defined as the probability of an accident occurring \autocite{hakkert2002uses}.
Several risk metrics have been developed with the goal of quantifying the risk involved in driving in traffic on the road \autocite{minderhoud2001extended, ozbay2008derivation, cunto2009simulated, laureshyn2010evaluation}.
In general, the risk is quantified in terms of the proximity between two traffic agents in time and/or space, the ability to perform evasive actions like braking or swerving, or the magnitude of such actions \autocite{shi2018key,zheng2020modeling}. 
In a potential collision situation , the proximity index is close to zero while the magnitude of evasive action is close to the limits of the driver and the vehicle \autocite{zheng2020modeling}. 

The most common risk indicators are time based. 
A popular time-based proximity indicator is the \ac{ttc} which is defined as the time remaining until two vehicles collide if they continue on the same collision course and speed \autocite{vanderhorst1990time}. 
Several other time-based indicators have been derived from or based on the \ac{ttc}. 
Notable among those are 
\begin{itemize}
    \item the \ac{mttc} which is able to calculate the \ac{ttc} for the cases where vehicles do not keep constant speed and the follower is slower than the leader \autocite{ozbay2008derivation},
    \item the \ac{tet}, which measures the amount of time the \ac{ttc} is below a certain threshold \autocite{minderhoud2001extended}, and
    \item the \ac{tit} which calculates the total area where the \ac{ttc} is below a certain threshold  \autocite{minderhoud2001extended}. 
\end{itemize}
Other time-based proximity indicators include time headway and the \ac{pet} which measures the ``time between the moment that a vehicle leaves the area of potential collision, i.e., the area in which the paths of the two vehicles intersect, and the other vehicle arrives in the same area'' \autocite{mahmud2017application}. 

For distance-based proximity indicators, the \ac{picud} measures the remaining distance between vehicles during an emergency stop \autocite{iida2001traffic, uno2003objective} and the \ac{psd} measures the remaining distance to the potential point of collision divided by the minimum acceptable stopping distance \autocite{allen1978analysis, guido2011comparing, mahmud2017application}. 
Other similar distance-based metrics are the \ac{mtc} and \ac{dss} which are calculated similar to the \ac{psd} and the \ac{picud}, respectively \autocite{kitajima2009estimation, okamura2011impact}. 
Recently, a distance-based metric that assumes `correct' driving behavior was proposed \autocite{shalev2017formal}. 
This metric calculates the minimum safety distance between a follower and its leader, such that no collision occurs if the leader vehicle brakes with a specified deceleration and the follower brakes after a specified reaction time with another specified deceleration. 

In terms of indicators relating to performing evasive actions, the \ac{drac} is the most widely used. 
The \ac{drac} is calculated as the ratio of the difference in speed between a following vehicle and a lead vehicle and their closing time \autocite{almqvist1991use, mahmud2017application}. 
Another indicator is the \ac{cpi}, which calculates the probability that a vehicle's \ac{drac} will exceed its \ac{madr} in a given time interval \autocite{cunto2009simulated} 

%Criticality Index(CI) which is the ratio of the square of the speed and the TTC. 
%Criticality index measures the risk and the severity of collision. It can also be interpreted as the product of the speed and the required deceleration rate \autocite{chan2006defining}.

Apart from the traditional proximity indicators, efforts have been made to characterize what is considered as ``safe-driving'' using a combination of risk indicators.
E.g., \textcite{wang2014evaluation} derive \iac{cpm} using a combination of the \ac{ttc}, the vehicle braking capability, and the driver reaction time.
Some studies have used the fundamental relationship between velocity and acceleration to classify whether a vehicle’s driving style  is `safe' or `unsafe'. 
This method uses a vehicle's acceleration and velocity to create a safety boundary.
If and only if the vehicle's velocity and acceleration are within that boundary, then it is considered safe  \autocite{eboli2016combining}.
Similarly, \textcite{shi2018key} used indicators like \ac{tit}, \ac{cpi}, and \ac{psd} to measure the effectiveness of risk indicators for predicting accident. 
The idea is that some indicators are able to distinguish accidents from non-accidents. 
These indicators are called key risk indicators. 
In \autocite{mullakkal2020modelling}, a probabilistic driving risk field was proposed.
The method derives the risk a vehicle is exposed to using a kinematic approach with the inclusion of uncertainty in the future state vehicle. 
\textcite{mullakkal2020modelling} define this for an encounter between the subject vehicle and a road obstacle, such as other vehicles or objects. 
