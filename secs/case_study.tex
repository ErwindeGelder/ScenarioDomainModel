\section{Case study}
\label{sec:case study}



\subsection{Comparison with Wang \& Stamatiadis' metric}
\label{sec:wang stamatiadis}

In this section, we will show that our method can be used to reproduce the metric of \textcite{wang2014evaluation}.
\textcite{wang2014evaluation} provide a metric that calculated the probability of a collision under certain assumptions. 
Because our method can be used to reproduce this metric, we argue that our method is a generalization of the metric of \textcite{wang2014evaluation}.
We will first explain the metric of \textcite{wang2014evaluation}. 
Next, in \cref{sec:wang stamatiadis replicate}, we will show how we estimate this metric using our method.
In \cref{sec:wang stamatiadis comparison}, we will illustrate the results of both.



\subsubsection{Metric of Wang \& Stamatiadis}
\label{sec:wang stamatiadis explanation}

The metric of \textcite{wang2014evaluation}, which we denote by $\wangstamatiadis$, calculates the probability of collision of the ``ego vehicle'' and the ``lead vehicle'', where the ego vehicle is following the lead vehicle.
The metric is based on the following assumptions:
\begin{itemize}
	\item The lead vehicle keeps a constant speed;
	\item The (driver of the) ego vehicle starts to react after its reaction time, denoted by $\timereact$;
	\item The reaction time $\timereact$ is distributed according to a log-normal distributions, such that the mean is \SI{0.92}{\second} and the standard deviation is \SI{0.28}{\second};
	\item When the ego vehicle reacts, it brakes with its \ac{madr}, denoted by $\accelerationmax$; and
	\item The \ac{madr} $\accelerationmax$ is distributed according to a truncated normal distribution with mean $\SI{9.7}{\meter\per\second\squared}$, standard deviation $\SI{1.3}{\meter\per\second\squared}$, lower bound $\lowerbound=\SI{4.2}{\meter\per\second\squared}$, and upper bound $\upperbound=\SI{12.7}{\meter\per\second\squared}$.
\end{itemize}
To calculate $\wangstamatiadis(\time)$ at a given time $\time$, the speed difference $\speeddifference{\time}$ and \ac{ttc} $\ttc{\time}$ are used.
If $\speeddifference{\time} \leq 0$, then the ego vehicle drives slower and there is no risk of collision according to the metric of \textcite{wang2014evaluation}, so $\wangstamatiadis=0$.
Note that the \ac{ttc} $\ttc{\time}$ is the ratio of the gap $\gap{\time}$ between the ego vehicle and the lead vehicle and the speed difference $\speeddifference{\time}$.
Given $\accelerationmax$, the driver of the ego vehicle needs to react before
\begin{equation}
	\ttc{\time} - \frac{\speeddifference{\time}}{2 \accelerationmax}
\end{equation}
in order to avoid a collision. 
Using the distributions of $\accelerationmax$ and $\timereact$, we can calculate the probability that this is the case, resulting in:
\begin{equation}
	\label{eq:ws}
	\wangstamatiadis(\time) = \begin{cases}
		0 & \text{if}\quad \speeddifference{\time} \leq 0 \\
		\int_{\lowerboundadapted}^{\upperbound}
		\probability{\timereact \leq \ttc{\time} - \frac{\speeddifference{\time}}{2 \accelerationmax} }
		\density{\accelerationmax} \ud \accelerationmax
		& \text{if}\quad \speeddifference{\time} > 0 \wedge \frac{\speeddifference{\time}}{2\ttc{\time}} < \upperbound \\
		1 & \text{otherwise}
	\end{cases},
\end{equation}
with $\lowerboundadapted=\max \left( \lowerbound, \frac{\speeddifference{\time}}{2\ttc{\time}}\right)$, $\probability{\timereact \leq \dummyvar}$ is the log-normal probability that the reaction time is lower than $\dummyvar$, and $\density{\accelerationmax}$ is the truncated normal probability density of $\accelerationmax$.



\subsubsection{Replicate Wang \& Stamatiadis' metric}
\label{sec:wang stamatiadis replicate}

Because the metric of \textcite{wang2014evaluation} uses the speed difference $\speeddifference{\time}$ and \ac{ttc} $\ttc{\time}$, these two variables are also used by our method to describe the current situation:
\begin{equation}
	\label{eq:situation current ws}
	\situationcurrent^T(\time) = \begin{bmatrix}
		\speeddifference{\time} & \ttc{\time}
	\end{bmatrix}.
\end{equation}
The lead vehicle is assumed to have a constant speed, so we can already describe the future situation of the lead vehicle using $\situationcurrent(\time)$ of \cref{eq:situation current ws}.
Therefore, $\situationfuturedim=0$ and there is no need to estimate $\densityestcond{\situationfuture}{\situationcurrent}$.
To estimate $\probabilitycond{\collision}{\situationcurrent}$, we use simulations. 
At the initial simulation time, the driver of the ego vehicle is not braking. 
After the reaction time $\timereact$, the driver starts braking with $\accelerationmax$.
The random parameters $\timereact$ and $\accelerationmax$ are similarly distributed as described in \cref{sec:wang stamatiadis explanation}.



\subsubsection{Comparison}
\label{sec:wang stamatiadis comparison}

\Cref{fig:ws comparison} shows the results of the comparison between the metric of \textcite{wang2014evaluation} and the metric derived using our proposed approach in \cref{sec:method}.
The black lines in \cref{fig:ws comparison} denote $\wangstamatiadis$ of \cref{eq:ws}.
These lines shows that for lower values of $\ttcsymbol$, $\wangstamatiadis$ increases.
Also, for increasing values of $\speeddifferencesymbol$ (solid, dashed, and dotted lines), the risk metric $\wangstamatiadis$ increases.

The gray lines in \cref{fig:ws comparison} denote $\probabilityestcond{\collision}{\situationcurrent,\situationfuture}$ of \cref{eq:estimate probability of collision}.
Because we have $\situationfuturedim=0$, we have $\probabilityestcond{\collision}{\situationcurrent} = \probabilityestcond{\collision}{\situationcurrent,\situationfuture}$.
\Cref{fig:ws comparison} illustrates that $\probabilityestcond{\collision}{\situationcurrent,\situationfuture}$ follows the same trend as $\wangstamatiadis$.
\Cref{fig:ws comparison} also illustrates the effect of the choice of the threshold $\simulationthreshold$.
In general, for a lower value of $\simulationthreshold$, the number of simulations $\numberofsimulations$ used in \cref{eq:kde simulation result} is higher. 
As a result, it can be expected that the estimation $\probabilityestcond{\collision}{\situationcurrent,\situationfuture}$ is closer to $\probabilitycond{\collision}{\situationcurrent,\situationfuture}$.
A comparison of the left plot in \cref{fig:ws comparison} ($\simulationthreshold=0.2$) with the right plot ($\simulationthreshold=0.02$) demonstrates this effect.

\setlength{\figurewidth}{.47\linewidth}
\setlength{\figureheight}{.7\figurewidth}
\begin{figure}
	\centering
	% This file was created by tikzplotlib v0.9.8.
\begin{tikzpicture}

\begin{axis}[
height=\figureheight,
scaled y ticks=false,
tick align=outside,
tick pos=left,
width=\figurewidth,
x grid style={white!69.0196078431373!black},
xlabel={$\ttcsymbol$ [\si{\second}]},
xmajorgrids,
xmin=0.5, xmax=4,
xtick style={color=black},
xticklabel style={align=center},
y grid style={white!69.0196078431373!black},
ylabel={\acl{ssm}},
ymajorgrids,
ymin=0, ymax=1,
ytick style={color=black},
yticklabel style={/pgf/number format/fixed,/pgf/number format/precision=3}
]
\addplot [very thick, black]
table {%
0.5 1
0.6 0.999999998869978
0.7 0.99999857290027
0.8 0.999897180595153
0.9 0.998283318978398
1 0.98817993034985
1.1 0.954499368567494
1.2 0.88238322312223
1.3 0.770109264089317
1.4 0.631738967436461
1.5 0.488426235243335
1.6 0.358220116360871
1.7 0.251152278540543
1.8 0.16961171159941
1.9 0.11109898777971
2 0.0710131882324527
2.1 0.0445256450697417
2.2 0.027506842417679
2.3 0.0168046929172315
2.4 0.0101836167889724
2.5 0.00613677052673833
2.6 0.00368492901741024
2.7 0.00220845719489648
2.8 0.00132282408579654
2.9 0.000792755323229866
3 0.000475750748254011
3.1 0.000286106476016124
3.2 0.00017251527884421
3.3 0.000104345369502878
3.4 6.33312409136222e-05
3.5 3.85817786129339e-05
3.6 2.35971162777515e-05
3.7 1.44916803196393e-05
3.8 8.93745841601401e-06
3.9 5.53582162532429e-06
4 3.44387841788585e-06
};
\addplot [very thick, gray]
table {%
0.5 1
0.6 1
0.7 1
0.8 1
0.9 1
1 1
1.1 0.809136438408709
1.2 0.88037881641836
1.3 0.946732286920236
1.4 0.664103660865121
1.5 0.550000730625837
1.6 0.399607344682456
1.7 0.186885497433421
1.8 0.141162195100395
1.9 0.0995490090001088
2 0.0745790502215188
2.1 4.19109191795997e-16
2.2 0.0615271114976857
2.3 5.6621374255883e-16
2.4 0.0322796213194603
2.5 5.01050480133935e-08
2.6 3.97408611085004e-06
2.7 2.17152573700474e-11
2.8 0
2.9 0
3 0
3.1 0
3.2 5.07927033766009e-16
3.3 0
3.4 0
3.5 0
3.6 0
3.7 0
3.8 0
3.9 0
4 0
};
\addplot [very thick, black, dashed]
table {%
0.5 1
0.6 1
0.7 1
0.8 1
0.9 1
1 0.999999999348448
1.1 0.99999950235446
1.2 0.999973209728235
1.3 0.999598995569285
1.4 0.997158301080912
1.5 0.987738504067729
1.6 0.962905518012083
1.7 0.913879059405361
1.8 0.836569727613501
1.9 0.734465168099655
2 0.617381972771856
2.1 0.497529753222982
2.2 0.385599610401107
2.3 0.288575568937833
2.4 0.209429765332663
2.5 0.14799359805209
2.6 0.102200293054103
2.7 0.0691792751327623
2.8 0.0460033719080388
2.9 0.0300971283204075
3 0.0193911312382948
3.1 0.0123154370919198
3.2 0.00772044024779683
3.3 0.004785490293998
3.4 0.0029387118341998
3.5 0.00179149195992223
3.6 0.00108628571142166
3.7 0.000656315154883846
3.8 0.000395725169891836
3.9 0.000238431669817296
4 0.000143716534243055
};
\addplot [very thick, gray, dashed]
table {%
0.5 1
0.6 1
0.7 1
0.8 1
0.9 1
1 1
1.1 1
1.2 0.999999679904892
1.3 0.999991577212567
1.4 0.999999999999979
1.5 0.802035142448822
1.6 0.993036708392475
1.7 0.87995411750343
1.8 0.799943056631864
1.9 0.810677208364332
2 0.794484974606894
2.1 0.617311404621259
2.2 0.548954796080919
2.3 0.396428643638844
2.4 0.308010064092293
2.5 0.197884249798686
2.6 0.197690068263262
2.7 1.4061749341876e-07
2.8 7.7390005819078e-06
2.9 0
3 1.06098463348303e-13
3.1 0.0448067720448657
3.2 0
3.3 0
3.4 2.60902410786912e-16
3.5 0
3.6 0
3.7 0
3.8 0
3.9 7.90739695943898e-13
4 0
};
\addplot [very thick, black, dotted]
table {%
0.5 1
0.6 1
0.7 1
0.8 1
0.9 1
1 1
1.1 1
1.2 1
1.3 0.999999999999999
1.4 0.999999999377245
1.5 0.999999657092481
1.6 0.999985156187676
1.7 0.999809817862939
1.8 0.998773565728781
1.9 0.994896330962281
2 0.984383486717914
2.1 0.962047971359528
2.2 0.922923942475201
2.3 0.864322979544098
2.4 0.787151363442989
2.5 0.69581547694205
2.6 0.596926356773646
2.7 0.497553920246217
2.8 0.403757956805445
2.9 0.319768708556622
3 0.247822552515213
3.1 0.18844874034795
3.2 0.140960526053158
3.3 0.10396026119255
3.4 0.0757525703567505
3.5 0.0546306025901766
3.6 0.0390439069936283
3.7 0.0276755214076069
3.8 0.0194589062185967
3.9 0.0135605539107035
4 0.00934782588289695
};
\addplot [very thick, gray, dotted]
table {%
0.5 1
0.6 1
0.7 1
0.8 1
0.9 1
1 1
1.1 1
1.2 1
1.3 1
1.4 1
1.5 1
1.6 0.999999995304703
1.7 1
1.8 0.999999998694335
1.9 0.993516168571737
2 0.902892993982995
2.1 0.915708612928547
2.2 0.931607706819998
2.3 0.851192631209404
2.4 0.651548789690542
2.5 0.745594977938927
2.6 0.641826661972804
2.7 0.63786654983661
2.8 0.590441431807042
2.9 0.249998652005759
3 0.446240446327888
3.1 0.328211938406514
3.2 0.0506219251868484
3.3 0.294080939792723
3.4 0.0999828665857648
3.5 1.90691162638146e-08
3.6 0.0497726118024459
3.7 2.20810666948235e-07
3.8 0
3.9 5.87102183535748e-10
4 8.85408701273294e-08
};
\end{axis}

\end{tikzpicture}

	% This file was created by tikzplotlib v0.9.8.
\begin{tikzpicture}

\begin{axis}[
height=\figureheight,
scaled y ticks=false,
tick align=outside,
tick pos=left,
width=\figurewidth,
x grid style={white!69.0196078431373!black},
xlabel={$\ttcsymbol$ [\si{\second}]},
xmajorgrids,
xmin=0.5, xmax=4,
xtick style={color=black},
xticklabel style={align=center},
y grid style={white!69.0196078431373!black},
ylabel={Surrogate risk metric},
ymajorgrids,
ymin=0, ymax=1,
ytick style={color=black},
yticklabel style={/pgf/number format/fixed,/pgf/number format/precision=3}
]
\addplot [very thick, black]
table {%
0.5 1
0.6 0.999999998869978
0.7 0.99999857290027
0.8 0.999897180595153
0.9 0.998283318978398
1 0.98817993034985
1.1 0.954499368567494
1.2 0.88238322312223
1.3 0.770109264089317
1.4 0.631738967436461
1.5 0.488426235243335
1.6 0.358220116360871
1.7 0.251152278540543
1.8 0.16961171159941
1.9 0.11109898777971
2 0.0710131882324527
2.1 0.0445256450697417
2.2 0.027506842417679
2.3 0.0168046929172315
2.4 0.0101836167889724
2.5 0.00613677052673833
2.6 0.00368492901741024
2.7 0.00220845719489648
2.8 0.00132282408579654
2.9 0.000792755323229866
3 0.000475750748254011
3.1 0.000286106476016124
3.2 0.00017251527884421
3.3 0.000104345369502878
3.4 6.33312409136222e-05
3.5 3.85817786129339e-05
3.6 2.35971162777515e-05
3.7 1.44916803196393e-05
3.8 8.93745841601401e-06
3.9 5.53582162532429e-06
4 3.44387841788585e-06
};
\addplot [very thick, gray]
table {%
0.5 1
0.6 1
0.7 1
0.8 1
0.9 1
1 1
1.1 0.929999989802346
1.2 0.919999999864079
1.3 0.816963525369272
1.4 0.599889108288165
1.5 0.529816237609428
1.6 0.270025344456169
1.7 0.243838773770293
1.8 0.129676626227738
1.9 0.157051606620671
2 0.0798999589837033
2.1 0.0583155287914021
2.2 0.000228370736932204
2.3 2.27040608535845e-15
2.4 2.40320502947755e-07
2.5 0.000199615909248951
2.6 0
2.7 8.32667268468867e-18
2.8 2.08718042848943e-12
2.9 0
3 0.000346230459247746
3.1 0
3.2 0
3.3 0
3.4 0
3.5 0
3.6 0
3.7 0
3.8 0
3.9 0
4 0
};
\addplot [very thick, black, dashed]
table {%
0.5 1
0.6 1
0.7 1
0.8 1
0.9 1
1 0.999999999348448
1.1 0.99999950235446
1.2 0.999973209728235
1.3 0.999598995569285
1.4 0.997158301080912
1.5 0.987738504067729
1.6 0.962905518012083
1.7 0.913879059405361
1.8 0.836569727613501
1.9 0.734465168099655
2 0.617381972771856
2.1 0.497529753222982
2.2 0.385599610401107
2.3 0.288575568937833
2.4 0.209429765332663
2.5 0.14799359805209
2.6 0.102200293054103
2.7 0.0691792751327623
2.8 0.0460033719080388
2.9 0.0300971283204075
3 0.0193911312382948
3.1 0.0123154370919198
3.2 0.00772044024779683
3.3 0.004785490293998
3.4 0.0029387118341998
3.5 0.00179149195992223
3.6 0.00108628571142166
3.7 0.000656315154883846
3.8 0.000395725169891836
3.9 0.000238431669817296
4 0.000143716534243055
};
\addplot [very thick, gray, dashed]
table {%
0.5 1
0.6 1
0.7 1
0.8 1
0.9 1
1 1
1.1 1
1.2 1
1.3 0.99992336981054
1.4 0.992465821902752
1.5 0.997383759575367
1.6 0.965199201904733
1.7 0.944996463627467
1.8 0.760611438765378
1.9 0.785307596187305
2 0.587838942746191
2.1 0.519783117719536
2.2 0.427527950269574
2.3 0.312824479524701
2.4 0.192568475492249
2.5 0.195547333272698
2.6 0.0680569153866018
2.7 0
2.8 0.0683491467328832
2.9 1.31664360411565e-07
3 1.55012064267934e-08
3.1 9.45910016980633e-15
3.2 3.38492289753134e-12
3.3 0.0197779518060416
3.4 0
3.5 6.42887899027222e-08
3.6 0
3.7 0
3.8 1.66533453693773e-17
3.9 0
4 0
};
\addplot [very thick, black, dotted]
table {%
0.5 1
0.6 1
0.7 1
0.8 1
0.9 1
1 1
1.1 1
1.2 1
1.3 0.999999999999999
1.4 0.999999999377245
1.5 0.999999657092481
1.6 0.999985156187676
1.7 0.999809817862939
1.8 0.998773565728781
1.9 0.994896330962281
2 0.984383486717914
2.1 0.962047971359528
2.2 0.922923942475201
2.3 0.864322979544098
2.4 0.787151363442989
2.5 0.69581547694205
2.6 0.596926356773646
2.7 0.497553920246217
2.8 0.403757956805445
2.9 0.319768708556622
3 0.247822552515213
3.1 0.18844874034795
3.2 0.140960526053158
3.3 0.10396026119255
3.4 0.0757525703567505
3.5 0.0546306025901766
3.6 0.0390439069936283
3.7 0.0276755214076069
3.8 0.0194589062185967
3.9 0.0135605539107035
4 0.00934782588289695
};
\addplot [very thick, gray, dotted]
table {%
0.5 1
0.6 1
0.7 1
0.8 1
0.9 1
1 1
1.1 1
1.2 1
1.3 1
1.4 1
1.5 1
1.6 0.999999999999998
1.7 1
1.8 0.999999989486865
1.9 0.981763698086574
2 0.999999999999999
2.1 0.971180803401473
2.2 0.882467406681377
2.3 0.761196759070653
2.4 0.823648674823992
2.5 0.766514377800562
2.6 0.605470387140889
2.7 0.499909832764079
2.8 0.342767273286868
2.9 0.307283542535188
3 0.221670474371101
3.1 0.249096389267906
3.2 0.129812401804595
3.3 0.114423966773786
3.4 0.0401403359457413
3.5 0.0818432197500936
3.6 0.0214285534318538
3.7 0.00178683712126506
3.8 0.0196881499206903
3.9 0
4 1.52121291210339e-06
};
\end{axis}

\end{tikzpicture}

	\caption{Comparison of $\wangstamatiadis$ of \cref{eq:ws} (black lines) and $\probabilityestcond{\collision}{\situationcurrent}$ (gray lines) for a speed difference of $\speeddifferencesymbol=\SI{10}{\meter\per\second}$ (solid lines), $\speeddifferencesymbol=\SI{20}{\meter\per\second}$ (dashed lines), and $\speeddifferencesymbol=\SI{30}{\meter\per\second}$ (dotted lines).
		Here, $\probabilityestcond{\collision}{\situationcurrent}$ is based on the same underlying assumptions as $\wangstamatiadis$, see \cref{sec:wang stamatiadis explanation}.
		The influence of the parameter $\simulationthreshold$, which determines the number of simulations to estimate $\probabilitycond{\collision}{\situationcurrent}$, is illustrated by using $\simulationthreshold=0.2$ (left plot) and $\simulationthreshold=0.02$ (right plot).}
	\label{fig:ws comparison}
\end{figure}


