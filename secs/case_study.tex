\section{Case study}
\label{sec:case study}



\subsection{Comparison with Wang \& Stamatiadis' metric}
\label{sec:wang stamatiadis}

In this section, we will show that our method can be used to reproduce the metric of \textcite{wang2014evaluation}.
\textcite{wang2014evaluation} provide a metric that calculated the probability of a collision under certain assumptions. 
Because our method can be used to reproduce this metric, we argue that our method is a generalization of the metric of \textcite{wang2014evaluation}.
We will first explain the metric of \textcite{wang2014evaluation}. 
Next, in \cref{sec:wang stamatiadis replicate}, we will show how we estimate this metric using our method.
In \cref{sec:wang stamatiadis comparison}, we will illustrate the results of both.



\subsubsection{Metric of Wang \& Stamatiadis}
\label{sec:wang stamatiadis explanation}

The metric of \textcite{wang2014evaluation}, which we denote by $\wangstamatiadis$, calculates the probability of collision of the ``ego vehicle'' and the ``lead vehicle'', where the ego vehicle is following the lead vehicle.
The metric is based on the following assumptions:
\begin{itemize}
	\item The lead vehicle keeps a constant speed;
	\item The (driver of the) ego vehicle starts to react after its reaction time, denoted by $\timereact$;
	\item The reaction time $\timereact$ is distributed according to a log-normal distributions, such that the mean is \SI{0.92}{\second} and the standard deviation is \SI{0.28}{\second};
	\item When the ego vehicle reacts, it brakes with its \ac{madr}, denoted by $\accelerationmax$; and
	\item The \ac{madr} $\accelerationmax$ is distributed according to a truncated normal distribution with mean $\SI{9.7}{\meter\per\second\squared}$, standard deviation $\SI{1.3}{\meter\per\second\squared}$, lower bound $\lowerbound=\SI{4.2}{\meter\per\second\squared}$, and upper bound $\upperbound=\SI{12.7}{\meter\per\second\squared}$.
\end{itemize}
To calculate $\wangstamatiadis(\time)$ at a given time $\time$, the speed difference $\speeddifference{\time}$ and \ac{ttc} $\ttc{\time}$ are used.
If $\speeddifference{\time} \leq 0$, then the ego vehicle drives slower and there is no risk of collision according to the metric of \textcite{wang2014evaluation}, so $\wangstamatiadis=0$.
Note that the \ac{ttc} $\ttc{\time}$ is the ratio of the gap $\gap{\time}$ between the ego vehicle and the lead vehicle and the speed difference $\speeddifference{\time}$.
Given $\accelerationmax$, the driver of the ego vehicle needs to react before
\begin{equation}
	\ttc{\time} - \frac{\speeddifference{\time}}{2 \accelerationmax}
\end{equation}
in order to avoid a collision. 
Using the distributions of $\accelerationmax$ and $\timereact$, we can calculate the probability that this is the case, resulting in:
\begin{equation}
	\label{eq:ws}
	\wangstamatiadis(\time) = \begin{cases}
		0 & \text{if}\quad \speeddifference{\time} \leq 0 \\
		\int_{\lowerboundadapted}^{\upperbound}
		\probability{\timereact \leq \ttc{\time} - \frac{\speeddifference{\time}}{2 \accelerationmax} }
		\density{\accelerationmax} \ud \accelerationmax
		& \text{if}\quad \speeddifference{\time} > 0 \wedge \frac{\speeddifference{\time}}{2\ttc{\time}} < \upperbound \\
		1 & \text{otherwise}
	\end{cases},
\end{equation}
with $\lowerboundadapted=\max \left( \lowerbound, \frac{\speeddifference{\time}}{2\ttc{\time}}\right)$, $\probability{\timereact \leq \dummyvar}$ is the log-normal probability that the reaction time is lower than $\dummyvar$, and $\density{\accelerationmax}$ is the truncated normal probability density of $\accelerationmax$.



\subsubsection{Replicate Wang \& Stamatiadis' metric}
\label{sec:wang stamatiadis replicate}

Because the metric of \textcite{wang2014evaluation} uses the speed difference $\speeddifference{\time}$ and \ac{ttc} $\ttc{\time}$, these two variables are also used by our method to describe the current situation:
\begin{equation}
	\label{eq:situation current ws}
	\situationcurrent^T(\time) = \begin{bmatrix}
		\speeddifference{\time} & \ttc{\time}
	\end{bmatrix}.
\end{equation}
The lead vehicle is assumed to have a constant speed, so we can already describe the future situation of the lead vehicle using $\situationcurrent(\time)$ of \cref{eq:situation current ws}.
Therefore, $\situationfuturedim=0$ and there is no need to estimate $\densityestcond{\situationfuture}{\situationcurrent}$.
To estimate $\probabilitycond{\collision}{\situationcurrent}$, we use simulations. 
At the initial simulation time, the driver of the ego vehicle is not braking. 
After the reaction time $\timereact$, the driver starts braking with $\accelerationmax$.
The random parameters $\timereact$ and $\accelerationmax$ are similarly distributed as described in \cref{sec:wang stamatiadis explanation}.



\subsubsection{Comparison}
\label{sec:wang stamatiadis comparison}

\Cref{fig:ws comparison} shows the results of the comparison between the metric of \textcite{wang2014evaluation} and the metric derived using our proposed approach in \cref{sec:method}.
The black lines in \cref{fig:ws comparison} denote $\wangstamatiadis$ of \cref{eq:ws}.
These lines shows that for lower values of $\ttcsymbol$, $\wangstamatiadis$ increases.
Also, for increasing values of $\speeddifferencesymbol$ (solid, dashed, and dotted lines), the risk metric $\wangstamatiadis$ increases.

The gray lines in \cref{fig:ws comparison} denote $\probabilityestcond{\collision}{\situationcurrent,\situationfuture}$ of \cref{eq:estimate probability of collision}.
Because we have $\situationfuturedim=0$, we have $\probabilityestcond{\collision}{\situationcurrent} = \probabilityestcond{\collision}{\situationcurrent,\situationfuture}$.
\Cref{fig:ws comparison} illustrates that $\probabilityestcond{\collision}{\situationcurrent,\situationfuture}$ follows the same trend as $\wangstamatiadis$.
\Cref{fig:ws comparison} also illustrates the effect of the choice of the threshold $\simulationthreshold$.
In general, for a lower value of $\simulationthreshold$, the number of simulations $\numberofsimulations$ used in \cref{eq:kde simulation result} is higher. 
As a result, it can be expected that the estimation $\probabilityestcond{\collision}{\situationcurrent,\situationfuture}$ is closer to $\probabilitycond{\collision}{\situationcurrent,\situationfuture}$.
A comparison of the left plot in \cref{fig:ws comparison} ($\simulationthreshold=0.2$) with the right plot ($\simulationthreshold=0.02$) demonstrates this effect.

\setlength{\figurewidth}{.47\linewidth}
\setlength{\figureheight}{.7\figurewidth}
\begin{figure}
	\centering
	% This file was created by tikzplotlib v0.9.8.
\begin{tikzpicture}

\begin{axis}[
height=\figureheight,
scaled y ticks=false,
tick align=outside,
tick pos=left,
width=\figurewidth,
x grid style={white!69.0196078431373!black},
xlabel={$\ttcsymbol$ [\si{\second}]},
xmajorgrids,
xmin=0.5, xmax=4,
xtick style={color=black},
xticklabel style={align=center},
y grid style={white!69.0196078431373!black},
ylabel={Surrogate risk metric},
ymajorgrids,
ymin=0, ymax=1,
ytick style={color=black},
yticklabel style={/pgf/number format/fixed,/pgf/number format/precision=3}
]
\addplot [very thick, black]
table {%
0.5 1
0.6 0.999999998869978
0.7 0.99999857290027
0.8 0.999897180595153
0.9 0.998283318978398
1 0.98817993034985
1.1 0.954499368567494
1.2 0.88238322312223
1.3 0.770109264089317
1.4 0.631738967436461
1.5 0.488426235243335
1.6 0.358220116360871
1.7 0.251152278540543
1.8 0.16961171159941
1.9 0.11109898777971
2 0.0710131882324527
2.1 0.0445256450697417
2.2 0.027506842417679
2.3 0.0168046929172315
2.4 0.0101836167889724
2.5 0.00613677052673833
2.6 0.00368492901741024
2.7 0.00220845719489648
2.8 0.00132282408579654
2.9 0.000792755323229866
3 0.000475750748254011
3.1 0.000286106476016124
3.2 0.00017251527884421
3.3 0.000104345369502878
3.4 6.33312409136222e-05
3.5 3.85817786129339e-05
3.6 2.35971162777515e-05
3.7 1.44916803196393e-05
3.8 8.93745841601401e-06
3.9 5.53582162532429e-06
4 3.44387841788585e-06
};
\addplot [very thick, gray]
table {%
0.5 1
0.6 1
0.7 1
0.8 1
0.9 1
1 1
1.1 0.809136438408709
1.2 0.88037881641836
1.3 0.946732286920236
1.4 0.664103660865121
1.5 0.550000730625837
1.6 0.399607344682456
1.7 0.186885497433421
1.8 0.141162195100395
1.9 0.0995490090001088
2 0.0745790502215188
2.1 4.19109191795997e-16
2.2 0.0615271114976857
2.3 5.6621374255883e-16
2.4 0.0322796213194603
2.5 5.01050480133935e-08
2.6 3.97408611085004e-06
2.7 2.17152573700474e-11
2.8 0
2.9 0
3 0
3.1 0
3.2 5.07927033766009e-16
3.3 0
3.4 0
3.5 0
3.6 0
3.7 0
3.8 0
3.9 0
4 0
};
\addplot [very thick, black, dashed]
table {%
0.5 1
0.6 1
0.7 1
0.8 1
0.9 1
1 0.999999999348448
1.1 0.99999950235446
1.2 0.999973209728235
1.3 0.999598995569285
1.4 0.997158301080912
1.5 0.987738504067729
1.6 0.962905518012083
1.7 0.913879059405361
1.8 0.836569727613501
1.9 0.734465168099655
2 0.617381972771856
2.1 0.497529753222982
2.2 0.385599610401107
2.3 0.288575568937833
2.4 0.209429765332663
2.5 0.14799359805209
2.6 0.102200293054103
2.7 0.0691792751327623
2.8 0.0460033719080388
2.9 0.0300971283204075
3 0.0193911312382948
3.1 0.0123154370919198
3.2 0.00772044024779683
3.3 0.004785490293998
3.4 0.0029387118341998
3.5 0.00179149195992223
3.6 0.00108628571142166
3.7 0.000656315154883846
3.8 0.000395725169891836
3.9 0.000238431669817296
4 0.000143716534243055
};
\addplot [very thick, gray, dashed]
table {%
0.5 1
0.6 1
0.7 1
0.8 1
0.9 1
1 1
1.1 1
1.2 0.999999679904892
1.3 0.999991577212567
1.4 0.999999999999979
1.5 0.802035142448822
1.6 0.993036708392475
1.7 0.87995411750343
1.8 0.799943056631864
1.9 0.810677208364332
2 0.794484974606894
2.1 0.617311404621259
2.2 0.548954796080919
2.3 0.396428643638844
2.4 0.308010064092293
2.5 0.197884249798686
2.6 0.197690068263262
2.7 1.4061749341876e-07
2.8 7.7390005819078e-06
2.9 0
3 1.06098463348303e-13
3.1 0.0448067720448657
3.2 0
3.3 0
3.4 2.60902410786912e-16
3.5 0
3.6 0
3.7 0
3.8 0
3.9 7.90739695943898e-13
4 0
};
\addplot [very thick, black, dotted]
table {%
0.5 1
0.6 1
0.7 1
0.8 1
0.9 1
1 1
1.1 1
1.2 1
1.3 0.999999999999999
1.4 0.999999999377245
1.5 0.999999657092481
1.6 0.999985156187676
1.7 0.999809817862939
1.8 0.998773565728781
1.9 0.994896330962281
2 0.984383486717914
2.1 0.962047971359528
2.2 0.922923942475201
2.3 0.864322979544098
2.4 0.787151363442989
2.5 0.69581547694205
2.6 0.596926356773646
2.7 0.497553920246217
2.8 0.403757956805445
2.9 0.319768708556622
3 0.247822552515213
3.1 0.18844874034795
3.2 0.140960526053158
3.3 0.10396026119255
3.4 0.0757525703567505
3.5 0.0546306025901766
3.6 0.0390439069936283
3.7 0.0276755214076069
3.8 0.0194589062185967
3.9 0.0135605539107035
4 0.00934782588289695
};
\addplot [very thick, gray, dotted]
table {%
0.5 1
0.6 1
0.7 1
0.8 1
0.9 1
1 1
1.1 1
1.2 1
1.3 1
1.4 1
1.5 1
1.6 0.999999995304703
1.7 1
1.8 0.999999998694335
1.9 0.993516168571737
2 0.902892993982995
2.1 0.915708612928547
2.2 0.931607706819998
2.3 0.851192631209404
2.4 0.651548789690542
2.5 0.745594977938927
2.6 0.641826661972804
2.7 0.63786654983661
2.8 0.590441431807042
2.9 0.249998652005759
3 0.446240446327888
3.1 0.328211938406514
3.2 0.0506219251868484
3.3 0.294080939792723
3.4 0.0999828665857648
3.5 1.90691162638146e-08
3.6 0.0497726118024459
3.7 2.20810666948235e-07
3.8 0
3.9 5.87102183535748e-10
4 8.85408701273294e-08
};
\end{axis}

\end{tikzpicture}

	% This file was created by tikzplotlib v0.9.8.
\begin{tikzpicture}

\begin{axis}[
height=\figureheight,
scaled y ticks=false,
tick align=outside,
tick pos=left,
width=\figurewidth,
x grid style={white!69.0196078431373!black},
xlabel={$\ttcsymbol$ [\si{\second}]},
xmajorgrids,
xmin=0.5, xmax=4,
xtick style={color=black},
xticklabel style={align=center},
y grid style={white!69.0196078431373!black},
ylabel={Surrogate risk metric},
ymajorgrids,
ymin=0, ymax=1,
ytick style={color=black},
yticklabel style={/pgf/number format/fixed,/pgf/number format/precision=3}
]
\addplot [very thick, black]
table {%
0.5 1
0.6 0.999999998869978
0.7 0.99999857290027
0.8 0.999897180595153
0.9 0.998283318978398
1 0.98817993034985
1.1 0.954499368567494
1.2 0.88238322312223
1.3 0.770109264089317
1.4 0.631738967436461
1.5 0.488426235243335
1.6 0.358220116360871
1.7 0.251152278540543
1.8 0.16961171159941
1.9 0.11109898777971
2 0.0710131882324527
2.1 0.0445256450697417
2.2 0.027506842417679
2.3 0.0168046929172315
2.4 0.0101836167889724
2.5 0.00613677052673833
2.6 0.00368492901741024
2.7 0.00220845719489648
2.8 0.00132282408579654
2.9 0.000792755323229866
3 0.000475750748254011
3.1 0.000286106476016124
3.2 0.00017251527884421
3.3 0.000104345369502878
3.4 6.33312409136222e-05
3.5 3.85817786129339e-05
3.6 2.35971162777515e-05
3.7 1.44916803196393e-05
3.8 8.93745841601401e-06
3.9 5.53582162532429e-06
4 3.44387841788585e-06
};
\addplot [very thick, gray]
table {%
0.5 1
0.6 1
0.7 1
0.8 1
0.9 1
1 1
1.1 0.929999989802346
1.2 0.919999999864079
1.3 0.816963525369272
1.4 0.599889108288165
1.5 0.529816237609428
1.6 0.270025344456169
1.7 0.243838773770293
1.8 0.129676626227738
1.9 0.157051606620671
2 0.0798999589837033
2.1 0.0583155287914021
2.2 0.000228370736932204
2.3 2.27040608535845e-15
2.4 2.40320502947755e-07
2.5 0.000199615909248951
2.6 0
2.7 8.32667268468867e-18
2.8 2.08718042848943e-12
2.9 0
3 0.000346230459247746
3.1 0
3.2 0
3.3 0
3.4 0
3.5 0
3.6 0
3.7 0
3.8 0
3.9 0
4 0
};
\addplot [very thick, black, dashed]
table {%
0.5 1
0.6 1
0.7 1
0.8 1
0.9 1
1 0.999999999348448
1.1 0.99999950235446
1.2 0.999973209728235
1.3 0.999598995569285
1.4 0.997158301080912
1.5 0.987738504067729
1.6 0.962905518012083
1.7 0.913879059405361
1.8 0.836569727613501
1.9 0.734465168099655
2 0.617381972771856
2.1 0.497529753222982
2.2 0.385599610401107
2.3 0.288575568937833
2.4 0.209429765332663
2.5 0.14799359805209
2.6 0.102200293054103
2.7 0.0691792751327623
2.8 0.0460033719080388
2.9 0.0300971283204075
3 0.0193911312382948
3.1 0.0123154370919198
3.2 0.00772044024779683
3.3 0.004785490293998
3.4 0.0029387118341998
3.5 0.00179149195992223
3.6 0.00108628571142166
3.7 0.000656315154883846
3.8 0.000395725169891836
3.9 0.000238431669817296
4 0.000143716534243055
};
\addplot [very thick, gray, dashed]
table {%
0.5 1
0.6 1
0.7 1
0.8 1
0.9 1
1 1
1.1 1
1.2 1
1.3 0.99992336981054
1.4 0.992465821902752
1.5 0.997383759575367
1.6 0.965199201904733
1.7 0.944996463627467
1.8 0.760611438765378
1.9 0.785307596187305
2 0.587838942746191
2.1 0.519783117719536
2.2 0.427527950269574
2.3 0.312824479524701
2.4 0.192568475492249
2.5 0.195547333272698
2.6 0.0680569153866018
2.7 0
2.8 0.0683491467328832
2.9 1.31664360411565e-07
3 1.55012064267934e-08
3.1 9.45910016980633e-15
3.2 3.38492289753134e-12
3.3 0.0197779518060416
3.4 0
3.5 6.42887899027222e-08
3.6 0
3.7 0
3.8 1.66533453693773e-17
3.9 0
4 0
};
\addplot [very thick, black, dotted]
table {%
0.5 1
0.6 1
0.7 1
0.8 1
0.9 1
1 1
1.1 1
1.2 1
1.3 0.999999999999999
1.4 0.999999999377245
1.5 0.999999657092481
1.6 0.999985156187676
1.7 0.999809817862939
1.8 0.998773565728781
1.9 0.994896330962281
2 0.984383486717914
2.1 0.962047971359528
2.2 0.922923942475201
2.3 0.864322979544098
2.4 0.787151363442989
2.5 0.69581547694205
2.6 0.596926356773646
2.7 0.497553920246217
2.8 0.403757956805445
2.9 0.319768708556622
3 0.247822552515213
3.1 0.18844874034795
3.2 0.140960526053158
3.3 0.10396026119255
3.4 0.0757525703567505
3.5 0.0546306025901766
3.6 0.0390439069936283
3.7 0.0276755214076069
3.8 0.0194589062185967
3.9 0.0135605539107035
4 0.00934782588289695
};
\addplot [very thick, gray, dotted]
table {%
0.5 1
0.6 1
0.7 1
0.8 1
0.9 1
1 1
1.1 1
1.2 1
1.3 1
1.4 1
1.5 1
1.6 0.999999999999998
1.7 1
1.8 0.999999989486865
1.9 0.981763698086574
2 0.999999999999999
2.1 0.971180803401473
2.2 0.882467406681377
2.3 0.761196759070653
2.4 0.823648674823992
2.5 0.766514377800562
2.6 0.605470387140889
2.7 0.499909832764079
2.8 0.342767273286868
2.9 0.307283542535188
3 0.221670474371101
3.1 0.249096389267906
3.2 0.129812401804595
3.3 0.114423966773786
3.4 0.0401403359457413
3.5 0.0818432197500936
3.6 0.0214285534318538
3.7 0.00178683712126506
3.8 0.0196881499206903
3.9 0
4 1.52121291210339e-06
};
\end{axis}

\end{tikzpicture}

	\caption{Comparison of $\wangstamatiadis$ of \cref{eq:ws} (black lines) and $\probabilityestcond{\collision}{\situationcurrent}$ (gray lines) for a speed difference of $\speeddifferencesymbol=\SI{10}{\meter\per\second}$ (solid lines), $\speeddifferencesymbol=\SI{20}{\meter\per\second}$ (dashed lines), and $\speeddifferencesymbol=\SI{30}{\meter\per\second}$ (dotted lines).
		Here, $\probabilityestcond{\collision}{\situationcurrent}$ is based on the same underlying assumptions as $\wangstamatiadis$, see \cref{sec:wang stamatiadis explanation}.
		The influence of the parameter $\simulationthreshold$, which determines the number of simulations to estimate $\probabilitycond{\collision}{\situationcurrent}$, is illustrated by using $\simulationthreshold=0.2$ (left plot) and $\simulationthreshold=0.02$ (right plot).}
	\label{fig:ws comparison}
\end{figure}



\subsection{Developing a surrogate safety metric based on the \acs{ngsim} data set}
\label{sec:ngsim metric}

To illustrate the proposed method, we applied it to derive a surrogate safety metric that is based on the \ac{ngsim} data set.
The \ac{ngsim} data set contains vehicles' trajectories obtained from video footage of cameras that were located at several motorways in the US \autocite{kovvali2007video}. 
The surrogate safety metric that we derived estimates the risk of collision of the ego vehicle with its lead vehicle.
To describe the current situation at time $\time$, $\situationcurrentdim=4$ parameters were used:
\begin{itemize}
	\item The speed of the leading vehicle ($\speedlead{\time}$);
	\item The acceleration of the leading vehicle ($\accelerationlead{\time}$);
	\item The speed of the ego vehicle ($\speedego{\time}$); and
	\item The log of the gap between the leading vehicle and the ego vehicle $\ln{\gap{\time}}$.
\end{itemize}
Thus, we had:
\begin{equation}
	\situationcurrent^T(\time) = \begin{bmatrix}
		\speedlead{\time} & \accelerationlead{\time} & \speedego{\time} & \ln{\gap{\time}}
	\end{bmatrix}.
\end{equation}

To describe the future situation, we considered the speed of the lead vehicle at $\situationfuturehorizon=50$ time instances, each $\situationfuturetimestep=\SI{0.1}{\second}$ apart:
\begin{equation}
	\label{eq:example future situation}
	\situationfuture^T(\time) = \begin{bmatrix}
		\speedlead{\time+\situationfuturetimestep} & \ldots & \speedlead{\time+\situationfuturehorizon\situationfuturetimestep}
	\end{bmatrix}.
\end{equation}
It is assumed that $\situationfuture(\time)$ depends on $\speedlead{\time}$ and $\accelerationlead{\time}$. 
To model this with one \ac{kde} would give us \iac{pdf} with $\situationfuturehorizon+2$ dimensions.
To reduce the dimensionality, we used \iac{svd} as described in \cref{sec:parameter reduction} with $\dimension=4$.
In total, 18182 longitudinal interactions between two vehicles were analyzed.
For each second of an interaction, we extracted a ``current situation'' $\situationcurrentinstance{\situationindex}$ and a corresponding ``future situation'' $\situationfutureinstance{\situationindex}$. 
This led to $\situationnumberof=469453$ data points.
Based on Silverman's rule of thumb \autocite{silverman1986density}, we used a bandwidth matrix $\bandwidthmatrix=\bandwidth^2\identitymatrix{4}$ for the \ac{kde} with $\bandwidth\approx 0.186$ and $\identitymatrix{4}$ denoting the 4-by-4 identity matrix.

To demonstrate the conditional sampling from the estimated density $\densityest{\situationcurrent,\situationfuture}$ of \cref{eq:kde estimate}, the plots in \cref{fig:speed profiles} show 50 different future situations in the form of \cref{eq:example future situation}.
The left plot assumes a current situation with $\speedleadsymbol=\SI{15}{\meter\per\second}$ and $\accelerationleadsymbol=\SI{1}{\meter\per\second\squared}$ and the right plot assumes a current situation with $\speedleadsymbol=\SI{15}{\meter\per\second}$ and $\accelerationleadsymbol=\SI{-1}{\meter\per\second\squared}$.
In case a simulation takes longer than \SI{5}{\second}, the speed of the lead vehicle is assumed to remain constant after these \SI{5}{\second}.

\setlength{\figurewidth}{.49\linewidth}
\setlength{\figureheight}{.7\figurewidth}
\begin{figure}
	\centering
	% This file was created by tikzplotlib v0.9.8.
\begin{tikzpicture}

\begin{axis}[
height=\figureheight,
scaled y ticks=false,
tick align=outside,
tick pos=left,
width=\figurewidth,
x grid style={white!69.0196078431373!black},
xlabel={Future horizon [\si{\second}]},
xmin=0, xmax=5,
xtick style={color=black},
xticklabel style={align=center},
y grid style={white!69.0196078431373!black},
ylabel={$\speedleadsymbol [\si{\meter\per\second}]$},
ymin=14.4524539279329, ymax=21.237314747831,
ytick style={color=black},
yticklabel style={/pgf/number format/fixed,/pgf/number format/precision=3}
]
\addplot [semithick, white!40!black]
table {%
0 15
0.1 15.0974590281545
0.2 15.1929000534814
0.3 15.2860604785272
0.4 15.3766791729592
0.5 15.4644637830695
0.6 15.5491626006458
0.7 15.6305522980996
0.8 15.7083506822841
0.9 15.7824191723993
1 15.8526451646941
1.1 15.9189445462627
1.2 15.9813028803499
1.3 16.0397428228406
1.4 16.0943334974839
1.5 16.1452255297797
1.6 16.1925966296897
1.7 16.2366631994714
1.8 16.2777304681818
1.9 16.3161213077445
2 16.3522369544881
2.1 16.3864726595574
2.2 16.4192508061432
2.3 16.4509578273814
2.4 16.4820102711249
2.5 16.5128302887906
2.6 16.5438293389752
2.7 16.5754032614966
2.8 16.6079321546608
2.9 16.6417580098132
3 16.6771735080323
3.1 16.7144537397498
3.2 16.7538243045852
3.3 16.7954222132351
3.4 16.8393526583102
3.5 16.8856673914625
3.6 16.9343816211309
3.7 16.9854727308674
3.8 17.0388740747397
3.9 17.094444902769
4 17.1520394494247
4.1 17.2114720754442
4.2 17.2725181119285
4.3 17.3349438976711
4.4 17.3985164541403
4.5 17.4629864820337
4.6 17.528116053779
4.7 17.593674271701
4.8 17.6594613667507
4.9 17.7252557045039
5 17.7909037057795
};
\addplot [semithick, white!40!black]
table {%
0 15
0.1 15.0987715543838
0.2 15.1964314886504
0.3 15.292745270439
0.4 15.3874678716576
0.5 15.4803291559558
0.6 15.5710769805541
0.7 15.6594779118473
0.8 15.745241126794
0.9 15.8281645387893
1 15.9080482838878
1.1 15.9846959722054
1.2 16.0579438972166
1.3 16.1276512726702
1.4 16.1936922408602
1.5 16.2559916501059
1.6 16.3144780385524
1.7 16.3691145283591
1.8 16.4199085443112
1.9 16.4668844474739
2 16.5101169125056
2.1 16.5496788909739
2.2 16.5856672395005
2.3 16.6181826564136
2.4 16.6473523495557
2.5 16.6733250213351
2.6 16.6962548242087
2.7 16.716303109369
2.8 16.7336449459125
2.9 16.7484595175786
3 16.760909580483
3.1 16.771164213673
3.2 16.7793908657028
3.3 16.7857343217867
3.4 16.790332262764
3.5 16.7933215814465
3.6 16.7948308645904
3.7 16.7949795334064
3.8 16.7938770811823
3.9 16.7916135632426
4 16.7882669781081
4.1 16.783903979501
4.2 16.7785857999244
4.3 16.7723630137274
4.4 16.7652831520085
4.5 16.7573895160021
4.6 16.7487371082643
4.7 16.7393734148111
4.8 16.7293551588145
4.9 16.718724704455
5 16.7075383021079
};
\addplot [semithick, white!40!black]
table {%
0 15
0.1 15.0980614310873
0.2 15.1946736740273
0.3 15.2896115541992
0.4 15.3826552099114
0.5 15.4735449393555
0.6 15.5620667613637
0.7 15.6480327919948
0.8 15.7312113368954
0.9 15.8115027165736
1 15.8888338895573
1.1 15.9631612040718
1.2 16.03450477006
1.3 16.1029186452761
1.4 16.168504062098
1.5 16.2314338754482
1.6 16.2919092232179
1.7 16.3501651066786
1.8 16.4065245282596
1.9 16.4613181094907
2 16.5149598596709
2.1 16.567849052951
2.2 16.6204041272362
2.3 16.6729987649686
2.4 16.7260327130044
2.5 16.7799030712513
2.6 16.8349906401176
2.7 16.8916581812885
2.8 16.9502465953345
2.9 17.0110461073405
3 17.0743011009959
3.1 17.1402305110683
3.2 17.208995508225
3.3 17.280662661107
3.4 17.355266728326
3.5 17.4327819252377
3.6 17.5131430204072
3.7 17.5962477230334
3.8 17.6819477648428
3.9 17.7700165095614
4 17.8602292466914
4.1 17.9523171427004
4.2 18.0459723724897
4.3 18.1408808677094
4.4 18.2367344174115
4.5 18.3332114057707
4.6 18.4300021511594
4.7 18.5268109823643
4.8 18.6233787673939
4.9 18.7194292097219
5 18.8147608287555
};
\addplot [semithick, white!40!black]
table {%
0 15
0.1 15.0987642456013
0.2 15.1964429109779
0.3 15.2928062999728
0.4 15.3876161865704
0.5 15.4806068786296
0.6 15.5715339510457
0.7 15.6601721802203
0.8 15.7462418696959
0.9 15.8295551293108
1 15.9099287761389
1.1 15.9871857591777
1.2 16.0611841674763
1.3 16.1318057727464
1.4 16.1989505393301
1.5 16.2625701746478
1.6 16.3226226342795
1.7 16.3790998819389
1.8 16.432042420964
1.9 16.4815057217786
2 16.5275993654651
2.1 16.5704290083742
2.2 16.6101229102083
2.3 16.6468074443632
2.4 16.6806350264874
2.5 16.7117763162611
2.6 16.7404046857869
2.7 16.7666979561933
2.8 16.7908435070847
2.9 16.8130260987834
3 16.8334115164359
3.1 16.8521679036723
3.2 16.8694553556086
3.3 16.8854035585129
3.4 16.9001326508897
3.5 16.9137553292537
3.6 16.9263725997837
3.7 16.9380736234556
3.8 16.9489338632451
3.9 16.9590030684274
4 16.9683210401434
4.1 16.976912547832
4.2 16.9847936224418
4.3 16.9919703608066
4.4 16.9984472378072
4.5 17.0042238954289
4.6 17.0093117587908
4.7 17.0137175956324
4.8 17.017460677319
4.9 17.0205461298637
5 17.0229996999397
};
\addplot [semithick, white!40!black]
table {%
0 15
0.1 15.0998608545449
0.2 15.1996130446603
0.3 15.2990855955372
0.4 15.398102426321
0.5 15.496448680865
0.6 15.5939340811294
0.7 15.6903825680776
0.8 15.7855855075054
0.9 15.8793997532932
1 15.9716836394602
1.1 16.062298524762
1.2 16.1511259024569
1.3 16.2380639364016
1.4 16.3230244528569
1.5 16.4059514069335
1.6 16.4867926106423
1.7 16.5655222213135
1.8 16.6421539199021
1.9 16.7167018234661
2 16.789236447004
2.1 16.8598125542962
2.2 16.9284954925376
2.3 16.9953430641471
2.4 17.0604326585115
2.5 17.1238508452855
2.6 17.1856816869419
2.7 17.2460142292346
2.8 17.3049434612292
2.9 17.3625511129456
3 17.4189107071276
3.1 17.4740910357127
3.2 17.528149367261
3.3 17.5811155117688
3.4 17.633014088823
3.5 17.683861326928
3.6 17.7336625564173
3.7 17.7824174022646
3.8 17.8301149882606
3.9 17.876722208836
4 17.922204748596
4.1 17.9665121697629
4.2 18.0095912606471
4.3 18.051382533615
4.4 18.0918316613126
4.5 18.130885800501
4.6 18.1685047574439
4.7 18.2046511559125
4.8 18.2393040502248
4.9 18.2724364256231
5 18.3040420795239
};
\addplot [semithick, white!40!black]
table {%
0 15
0.1 15.1007488615807
0.2 15.2022621926412
0.3 15.3044297466799
0.4 15.407143956584
0.5 15.5102431885681
0.6 15.6136013752606
0.7 15.717103790902
0.8 15.8206285865104
0.9 15.9241053854172
1 16.0274688865479
1.1 16.1306609449374
1.2 16.233637339511
1.3 16.3363666455485
1.4 16.4388355839713
1.5 16.5410494060134
1.6 16.6430216920691
1.7 16.7447845823277
1.8 16.8464129684401
1.9 16.9479652034347
2 17.0495675001919
2.1 17.1513149951903
2.2 17.2533001917195
2.3 17.3555889324827
2.4 17.4582601335379
2.5 17.5613862056712
2.6 17.6650258373277
2.7 17.7692362106164
2.8 17.8740669780402
2.9 17.979528764216
3 18.0856264653312
3.1 18.1923443937082
3.2 18.2996362880881
3.3 18.4074113431036
3.4 18.5155710055894
3.5 18.623990774093
3.6 18.7325273729871
3.7 18.841030141693
3.8 18.9493310575893
3.9 19.0572269294551
4 19.1645258944613
4.1 19.2710097722533
4.2 19.376454176861
4.3 19.4806332952203
4.4 19.5833356632537
4.5 19.6843550024182
4.6 19.7834986060188
4.7 19.8805899705415
4.8 19.9754805027387
4.9 20.0680227544628
5 20.158107223675
};
\addplot [semithick, white!40!black]
table {%
0 15
0.1 15.1002445055604
0.2 15.2007560932679
0.3 15.3013897876527
0.4 15.4019987804455
0.5 15.5023909921945
0.6 15.6024035191908
0.7 15.7018864127351
0.8 15.8006680059939
0.9 15.8986359334125
1 15.9956807510498
1.1 16.0916977048031
1.2 16.1865993848059
1.3 16.2803133454666
1.4 16.3727825936097
1.5 16.4639763486794
1.6 16.5538695037026
1.7 16.6424599618795
1.8 16.7297863548298
1.9 16.8158805104388
2 16.9008354333066
2.1 16.9847218593803
2.2 17.0676154607451
2.3 17.1495763679153
2.4 17.2306815074389
2.5 17.3110103598221
2.6 17.3906351789809
2.7 17.4696305231152
2.8 17.5480712629985
2.9 17.6260081864197
3 17.7034850573118
3.1 17.7805342419509
3.2 17.8571686454698
3.3 17.9333666838338
3.4 18.009100587393
3.5 18.0843269126317
3.6 18.158988068646
3.7 18.2330201533999
3.8 18.3063459698335
3.9 18.3788607853471
4 18.4504639760388
4.1 18.5210345587729
4.2 18.5904474427721
4.3 18.6585733221941
4.4 18.7252919547086
4.5 18.7904862067288
4.6 18.8540519886779
4.7 18.9158936794315
4.8 18.9759368984872
4.9 19.0341043009
5 19.0903464503873
};
\addplot [semithick, white!40!black]
table {%
0 15
0.1 15.099290534214
0.2 15.1978915279864
0.3 15.2955897746368
0.4 15.3921607327742
0.5 15.4873524096704
0.6 15.5809282977188
0.7 15.672667672695
0.8 15.762298996925
0.9 15.8496234344049
1 15.9344397985692
1.1 16.0165456451586
1.2 16.0957611973567
1.3 16.1719253950838
1.4 16.2448859636223
1.5 16.3145309308599
1.6 16.3807477387548
1.7 16.4434555621104
1.8 16.5026084460543
1.9 16.5581730500629
2 16.6101627151644
2.1 16.6585862824571
2.2 16.7034725183812
2.3 16.7448584833851
2.4 16.7828053234592
2.5 16.8173954841522
2.6 16.8487178193619
2.7 16.8768720306052
2.8 16.9019749999751
2.9 16.9241511040918
3 16.9435164496504
3.1 16.9601950723005
3.2 16.9743156982154
3.3 16.9859953406923
3.4 16.9953489565269
3.5 17.0024985018087
3.6 17.0075621801337
3.7 17.0106551984964
3.8 17.0118889848315
3.9 17.011364910318
4 17.0091737450896
4.1 17.0053989667292
4.2 17.0001255075545
4.3 16.9934284474503
4.4 16.9853812192451
4.5 16.976056709162
4.6 16.9655398064468
4.7 16.9539078239535
4.8 16.9412450982744
4.9 16.9276250984158
5 16.9131269404149
};
\addplot [semithick, white!40!black]
table {%
0 15
0.1 15.1001578355888
0.2 15.2003994775717
0.3 15.3005582756581
0.4 15.4004589961003
0.5 15.4998899563841
0.6 15.5986576497101
0.7 15.6965804220442
0.8 15.7934431914584
0.9 15.8890828259192
1 15.9833313442674
1.1 16.0760171077205
1.2 16.1669793285481
1.3 16.2560704055681
1.4 16.3431480105746
1.5 16.4280946249621
1.6 16.5107903084217
1.7 16.5911407793401
1.8 16.6690795868415
1.9 16.7445413568648
2 16.8175092890581
2.1 16.8879526798982
2.2 16.9558512145662
2.3 17.0211882712318
2.4 17.0839661970301
2.5 17.1442013979556
2.6 17.2019124997226
2.7 17.2571293383006
2.8 17.3098959797673
2.9 17.3602554213654
3 17.4082508526747
3.1 17.4539276978436
3.2 17.4973331827917
3.3 17.5385049853975
3.4 17.5774821205779
3.5 17.6143097066277
3.6 17.6490296621599
3.7 17.6816857075902
3.8 17.712320230845
3.9 17.7409681824495
4 17.7676608702272
4.1 17.7924213662553
4.2 17.8152788432005
4.3 17.8362555308479
4.4 17.8553774079417
4.5 17.8726748904716
4.6 17.8881910845648
4.7 17.9019674952555
4.8 17.914055828918
4.9 17.9245033176527
5 17.9333631389882
};
\addplot [semithick, white!40!black]
table {%
0 15
0.1 15.0993693192227
0.2 15.1980576053464
0.3 15.2958459651258
0.4 15.3925006483582
0.5 15.4877642515777
0.6 15.5813888498543
0.7 15.6731410737148
0.8 15.7627325105061
0.9 15.8499400739483
1 15.9345334282616
1.1 16.0162757597897
1.2 16.0949473787489
1.3 16.1703454522768
1.4 16.2422694797344
1.5 16.3105561365874
1.6 16.3750365231717
1.7 16.4355741101057
1.8 16.4920586760644
1.9 16.5443954848977
2 16.5925294226652
2.1 16.6364043469161
2.2 16.6759857952786
2.3 16.7112581179172
2.4 16.7422302536668
2.5 16.7689380634077
2.6 16.7914286946184
2.7 16.8097653842579
2.8 16.8240362171878
2.9 16.8343489023855
3 16.8408083437007
3.1 16.8435344660884
3.2 16.8426638403045
3.3 16.8383361753343
3.4 16.8306940114831
3.5 16.8198994402609
3.6 16.8061172602208
3.7 16.7895147143753
3.8 16.7702625996252
3.9 16.7485337538819
4 16.724486935889
4.1 16.6982805332999
4.2 16.670081027356
4.3 16.6400438915549
4.4 16.6083206608367
4.5 16.5750638402427
4.6 16.5404378341168
4.7 16.504594495425
4.8 16.4676867440078
4.9 16.429856705551
5 16.3912394034596
};
\addplot [semithick, white!40!black]
table {%
0 15
0.1 15.0991034683571
0.2 15.1973929617537
0.3 15.2946520358194
0.4 15.3906549064379
0.5 15.4851471118098
0.6 15.5778933842201
0.7 15.6686756846906
0.8 15.757225396966
0.9 15.8433548496162
1 15.9268777653544
1.1 16.0076105592107
1.2 16.0853979296346
1.3 16.160105416059
1.4 16.2316123099156
1.5 16.2998427265105
1.6 16.3647239002193
1.7 16.4262152923719
1.8 16.48431818941
1.9 16.5390462915744
2 16.5904645344409
2.1 16.6386324136866
2.2 16.683629608767
2.3 16.7255375811689
2.4 16.7644623173407
2.5 16.8005283556713
2.6 16.8338639379068
2.7 16.8646045009575
2.8 16.8928978536398
2.9 16.918892289495
3 16.9427227845889
3.1 16.9645282507387
3.2 16.9844445298592
3.3 17.0025852285178
3.4 17.0190580224535
3.5 17.0339691112642
3.6 17.047416424895
3.7 17.0594904272981
3.8 17.070272396474
3.9 17.0798248507692
4 17.0882010269923
4.1 17.0954422647
4.2 17.1015860902903
4.3 17.1066605242804
4.4 17.1106926849969
4.5 17.1137073391323
4.6 17.1157412221979
4.7 17.1168259906889
4.8 17.1170039249964
4.9 17.1163054070895
5 17.114775168487
};
\addplot [semithick, white!40!black]
table {%
0 15
0.1 15.098497945785
0.2 15.1957337640564
0.3 15.2914732448307
0.4 15.385476524446
0.5 15.4774744830136
0.6 15.5672246076489
0.7 15.6545056528893
0.8 15.7390422924056
0.9 15.8206627935304
1 15.8992055065187
1.1 15.9745205846706
1.2 16.0465013589575
1.3 16.115067880046
1.4 16.180165520456
1.5 16.2417978486012
1.6 16.2999800016622
1.7 16.3547618169278
1.8 16.4062516479675
1.9 16.4545724923068
2 16.4999080101663
2.1 16.5424365573909
2.2 16.582359435906
2.3 16.6198666861011
2.4 16.6551750068643
2.5 16.6885153693989
2.6 16.7201175561062
2.7 16.7502105488938
2.8 16.7790259664994
2.9 16.806782705536
3 16.8336734568412
3.1 16.8598875120882
3.2 16.8855949496577
3.3 16.9109203189608
3.4 16.9359730623212
3.5 16.9608429934447
3.6 16.9856017436928
3.7 17.0103026887032
3.8 17.0349777407803
3.9 17.0596205955199
4 17.084216918255
4.1 17.1087307202475
4.2 17.1331097153945
4.3 17.1572921904489
4.4 17.1812158972873
4.5 17.2048111751021
4.6 17.2280200991248
4.7 17.2507836953648
4.8 17.2730606784832
4.9 17.2947941552019
5 17.3159603641747
};
\addplot [semithick, white!40!black]
table {%
0 15
0.1 15.0984082292634
0.2 15.1955454169032
0.3 15.2911839527617
0.4 15.3850946312313
0.5 15.4770146075847
0.6 15.5667145624308
0.7 15.6539878530135
0.8 15.7385786451076
0.9 15.8203431655699
1 15.899153361391
1.1 15.9748989897728
1.2 16.0475193543469
1.3 16.1169826133161
1.4 16.1832896721156
1.5 16.2465032139209
1.6 16.3067032325577
1.7 16.3640036811761
1.8 16.4185868787718
1.9 16.4706464700366
2 16.5204448917984
2.1 16.5682352692263
2.2 16.6142916425203
2.3 16.6588646753235
2.4 16.7022311047124
2.5 16.7446754587734
2.6 16.7864754557876
2.7 16.8279019790982
2.8 16.8692197182728
2.9 16.9106666664398
3 16.9524483290667
3.1 16.9947586409149
3.2 17.0377585562807
3.3 17.0815464051569
3.4 17.1261997870546
3.5 17.1717622210689
3.6 17.2182516078376
3.7 17.2656613335449
3.8 17.313954881542
3.9 17.3630435935224
4 17.4128347931336
4.1 17.4632061803318
4.2 17.5140115198095
4.3 17.5650964825144
4.4 17.616308848822
4.5 17.6674872459406
4.6 17.7184821573101
4.7 17.7691487515581
4.8 17.8193667476049
4.9 17.8690001791686
5 17.9179608902813
};
\addplot [semithick, white!40!black]
table {%
0 15
0.1 15.1014127117341
0.2 15.2038709959807
0.3 15.3072503227621
0.4 15.4114119571197
0.5 15.516179445893
0.6 15.6213824923626
0.7 15.7268547486223
0.8 15.8324068064358
0.9 15.9378572066208
1 16.0430041330302
1.1 16.1476259062134
1.2 16.2514832218799
1.3 16.3543385053255
1.4 16.4559387582474
1.5 16.5560289993877
1.6 16.6543368938911
1.7 16.7506100042564
1.8 16.8445933704523
1.9 16.9360263916192
2 17.0246813708831
2.1 17.1103141139259
2.2 17.1926836887975
2.3 17.2715740387921
2.4 17.3467831369994
2.5 17.4181285256135
2.6 17.4854372925119
2.7 17.5485615200795
2.8 17.607383451663
2.9 17.6618047282874
3 17.7117513812772
3.1 17.7571628076311
3.2 17.7980071271867
3.3 17.8342833476183
3.4 17.8660083811479
3.5 17.8932361057731
3.6 17.9160343148972
3.7 17.9344922873421
3.8 17.948719609964
3.9 17.9588524118694
4 17.9650227845603
4.1 17.9673710561271
4.2 17.9660662024584
4.3 17.9612708526227
4.4 17.9531521111236
4.5 17.941891439121
4.6 17.9276835591746
4.7 17.9107161924793
4.8 17.8911759608963
4.9 17.869252970893
5 17.8451111831066
};
\addplot [semithick, white!40!black]
table {%
0 15
0.1 15.1008807242756
0.2 15.2022251586663
0.3 15.3038629513624
0.4 15.4056012900087
0.5 15.507223199107
0.6 15.6085053251776
0.7 15.7092291581895
0.8 15.8091323492823
0.9 15.907963668007
1 16.005444924822
1.1 16.1012708293112
1.2 16.1951179308371
1.3 16.2866655369029
1.4 16.3755690053689
1.5 16.4614882209816
1.6 16.5440584442246
1.7 16.6229406293294
1.8 16.6977837310059
1.9 16.7682449635125
2 16.8340006936733
2.1 16.8947240902698
2.2 16.9501019977352
2.3 16.9998680002242
2.4 17.043774535411
2.5 17.0816089863456
2.6 17.1131800073643
2.7 17.1383300945011
2.8 17.1569475313148
2.9 17.1689677257959
3 17.1743534623335
3.1 17.1730976331425
3.2 17.1652448982831
3.3 17.150894993701
3.4 17.1301714521439
3.5 17.1032573184112
3.6 17.0703603733492
3.7 17.0317156548399
3.8 16.9875893890721
3.9 16.9382933878137
4 16.8841240005143
4.1 16.8253985923863
4.2 16.7624711513177
4.3 16.6956850619206
4.4 16.6253800310529
4.5 16.5519089177893
4.6 16.4756371318674
4.7 16.3969098091169
4.8 16.316058162435
4.9 16.2334120844261
5 16.1492528928951
};
\addplot [semithick, white!40!black]
table {%
0 15
0.1 15.1024791888939
0.2 15.2068102176928
0.3 15.312902569572
0.4 15.4206460667002
0.5 15.5298925643407
0.6 15.640488735086
0.7 15.7522783263977
0.8 15.8650897916138
0.9 15.9787211507886
1 16.0929360470955
1.1 16.2074635880248
1.2 16.3219902514924
1.3 16.4361942639538
1.4 16.5497197356358
1.5 16.6621857145048
1.6 16.7731803374258
1.7 16.8823069561444
1.8 16.9891391786175
1.9 17.0932399677452
2 17.1941905725966
2.1 17.2915543180358
2.2 17.3848922062932
2.3 17.4738100840217
2.4 17.5579238041288
2.5 17.6368746011915
2.6 17.7103204274238
2.7 17.7779568812232
2.8 17.8395246883372
2.9 17.8948036875977
3 17.9436195178356
3.1 17.9858220121771
3.2 18.0213148034346
3.3 18.0500694131759
3.4 18.0720901713885
3.5 18.0874468750392
3.6 18.0962389348515
3.7 18.0986048756677
3.8 18.0947232008869
3.9 18.0848299674915
4 18.0691561199611
4.1 18.0479560989879
4.2 18.0215332733676
4.3 17.9901849163105
4.4 17.9542128935443
4.5 17.9139420489998
4.6 17.8697109436218
4.7 17.8218455550264
4.8 17.7706600340824
4.9 17.7164786889993
5 17.6595701108424
};
\addplot [semithick, white!40!black]
table {%
0 15
0.1 15.100264023417
0.2 15.2008125556213
0.3 15.3015016725711
0.4 15.4021856807541
0.5 15.5026734112659
0.6 15.602802933838
0.7 15.7024251850827
0.8 15.8013697769029
0.9 15.899525157538
1 15.9967826423646
1.1 16.0930381844646
1.2 16.1882048169147
1.3 16.2822104131232
1.4 16.3749982221713
1.5 16.4665373577478
1.6 16.5568025676195
1.7 16.6457914725875
1.8 16.733542266423
1.9 16.8200860780514
2 16.9055152584052
2.1 16.9898996765305
2.2 17.0733139220155
2.3 17.1558169352351
2.4 17.2374843370911
2.5 17.3183941371433
2.6 17.3986170221283
2.7 17.4782259892748
2.8 17.5572942790122
2.9 17.6358708512352
3 17.7139978306578
3.1 17.7917058133371
3.2 17.8690058645977
3.3 17.9458746038202
3.4 18.0222825394217
3.5 18.0981844810554
3.6 18.173521101024
3.7 18.2482268666054
3.8 18.3222230024459
3.9 18.3954032521514
4 18.4676656261587
4.1 18.5388877518173
4.2 18.6089432510558
4.3 18.6777015967464
4.4 18.7450414477492
4.5 18.810844683426
4.6 18.8750062426735
4.7 18.9374296693787
4.8 18.9980398219011
4.9 19.0567587382236
5 19.1135363764968
};
\addplot [semithick, white!40!black]
table {%
0 15
0.1 15.0991239023791
0.2 15.1973852962137
0.3 15.2945581135464
0.4 15.3904029331744
0.5 15.4846564342108
0.6 15.5770677949122
0.7 15.6674023881142
0.8 15.7553691385018
0.9 15.8407514908537
1 15.9233291740746
1.1 16.0028791446881
1.2 16.0792015312574
1.3 16.1521157054063
1.4 16.2214480760825
1.5 16.2870676524331
1.6 16.3488413055909
1.7 16.4066692762588
1.8 16.4604848985804
1.9 16.5102378856144
2 16.5559214122205
2.1 16.5975276684864
2.2 16.6350716373532
2.3 16.6685817994833
2.4 16.6981120864229
2.5 16.7237416009974
2.6 16.7455587362471
2.7 16.7636647205987
2.8 16.7781816642522
2.9 16.7892459154321
3 16.7969857814909
3.1 16.8015415983798
3.2 16.803063773371
3.3 16.8016963732107
3.4 16.7975825650039
3.5 16.790877677227
3.6 16.7817357112308
3.7 16.7703087268271
3.8 16.7567473433296
3.9 16.7411963044878
4 16.7237867926541
4.1 16.7046456465504
4.2 16.6839027225866
4.3 16.6616769050131
4.4 16.6380832955895
4.5 16.6132359121928
4.6 16.5872605780876
4.7 16.5602722126987
4.8 16.5323896818092
4.9 16.503719554901
5 16.4743689446781
};
\addplot [semithick, white!40!black]
table {%
0 15
0.1 15.0992300796427
0.2 15.1977219345784
0.3 15.2952599514698
0.4 15.3916173094882
0.5 15.486539973326
0.6 15.5797897338665
0.7 15.6711445130005
0.8 15.7603306985613
0.9 15.8471492908187
1 15.9313995087251
1.1 16.0128799038326
1.2 16.0914128968086
1.3 16.1668401253274
1.4 16.2390127759809
1.5 16.3078235632717
1.6 16.3731651515431
1.7 16.4349622608119
1.8 16.4931756420983
1.9 16.5477791335859
2 16.5987937346262
2.1 16.6462362342923
2.2 16.6901437900959
2.3 16.7305612494257
2.4 16.7675578200078
2.5 16.801223982748
2.6 16.8316564741616
2.7 16.8589624126145
2.8 16.883265631658
2.9 16.9046969647544
3 16.9233779884615
3.1 16.9394379617246
3.2 16.9530100021844
3.3 16.9642141244565
3.4 16.9731676628412
3.5 16.9799939450497
3.6 16.9848119657637
3.7 16.9877369653319
3.8 16.9888796331922
3.9 16.9883394165223
4 16.9862050240081
4.1 16.9825573449722
4.2 16.977477873911
4.3 16.9710381701298
4.4 16.9633080049834
4.5 16.9543561658514
4.6 16.9442634092899
4.7 16.93310296557
4.8 16.9209553953597
4.9 16.9078899890019
5 16.893982743507
};
\addplot [semithick, white!40!black]
table {%
0 15
0.1 15.0997430260036
0.2 15.1992249575224
0.3 15.298260887955
0.4 15.3966575370728
0.5 15.4941876123979
0.6 15.590643238594
0.7 15.6858306252426
0.8 15.7795165884463
0.9 15.8715320398528
1 15.9617061266095
1.1 16.0498675092453
1.2 16.1358631588443
1.3 16.2195564270325
1.4 16.3008200992861
1.5 16.3795598759342
1.6 16.4556818640898
1.7 16.5291202536957
1.8 16.5998436367477
1.9 16.6678256151537
2 16.7330903842925
2.1 16.7956509841875
2.2 16.8555343446106
2.3 16.912768865519
2.4 16.9674039714902
2.5 17.0195040689191
2.6 17.0691356602958
2.7 17.1163741781295
2.8 17.1613074866525
2.9 17.2040212563658
3 17.2445954034314
3.1 17.2831117243666
3.2 17.3196502575246
3.3 17.3542745306992
3.4 17.3870459334298
3.5 17.4180272774764
3.6 17.4472753182118
3.7 17.474844293459
3.8 17.5007830624255
3.9 17.5251269973266
4 17.5479061620797
4.1 17.5691400018612
4.2 17.5888493259825
4.3 17.6070471834793
4.4 17.6237489166337
4.5 17.6389714909808
4.6 17.6527443007578
4.7 17.6650945129825
4.8 17.6760605022972
4.9 17.6856733164972
5 17.6939749736294
};
\addplot [semithick, white!40!black]
table {%
0 15
0.1 15.0987598672786
0.2 15.1962115519782
0.3 15.2920900006647
0.4 15.3861082548312
0.5 15.4779682833638
0.6 15.567371167066
0.7 15.6540341233962
0.8 15.7375992331669
0.9 15.8177810414415
1 15.894282264638
1.1 15.9667940451071
1.2 16.0350268911727
1.3 16.098710213516
1.4 16.1575699057894
1.5 16.2113775936478
1.6 16.2598940787018
1.7 16.3029184136332
1.8 16.3402701631019
1.9 16.3717977981098
2 16.3973782771884
2.1 16.4169000032409
2.2 16.4302833145916
2.3 16.4374854243816
2.4 16.4384930422644
2.5 16.4333334035458
2.6 16.4220551039819
2.7 16.404729636938
2.8 16.3814661767772
2.9 16.3524170411355
3 16.3177321636877
3.1 16.2775908185326
3.2 16.2322075218749
3.3 16.1818182773943
3.4 16.1266657450278
3.5 16.0670298953917
3.6 16.0032017083002
3.7 15.9354780228835
3.8 15.8641671546016
3.9 15.7895937004502
4 15.7120577154562
4.1 15.6318690475593
4.2 15.5493508032561
4.3 15.4648111118607
4.4 15.3785466005902
4.5 15.2908528072986
4.6 15.2020364610369
4.7 15.112380058897
4.8 15.0221564634419
4.9 14.9316225625942
5 14.8410106083328
};
\addplot [semithick, white!40!black]
table {%
0 15
0.1 15.0957935555418
0.2 15.1879764673318
0.3 15.2761794576405
0.4 15.3600228595499
0.5 15.4391211029039
0.6 15.5131132037975
0.7 15.5816725893711
0.8 15.6443703134853
0.9 15.7009513249879
1 15.7511832067256
1.1 15.794857924545
1.2 15.8318520219775
1.3 15.862087389373
1.4 15.8855284409758
1.5 15.9022478482651
1.6 15.9123404891839
1.7 15.9159536703426
1.8 15.91332323755
1.9 15.9047316636825
2 15.8905235519383
2.1 15.8710631147794
2.2 15.8467645770739
2.3 15.818034106029
2.4 15.7853194955089
2.5 15.7490989906543
2.6 15.7098573506671
2.7 15.6680720962741
2.8 15.6242244999873
2.9 15.5787966631202
3 15.5322133097544
3.1 15.4849057678234
3.2 15.4372827113243
3.3 15.3896856039623
3.4 15.3424255547539
3.5 15.2957839660064
3.6 15.2500158036516
3.7 15.2053377160389
3.8 15.1619300144781
3.9 15.1199145423446
4 15.0793876032567
4.1 15.040419648427
4.2 15.0030440110039
4.3 14.9672769709302
4.4 14.933120281761
4.5 14.9005516598358
4.6 14.8695585842638
4.7 14.8401144748634
4.8 14.8122068951908
4.9 14.7857883823546
5 14.7608566924737
};
\addplot [semithick, white!40!black]
table {%
0 15
0.1 15.0999572718977
0.2 15.1999173905227
0.3 15.2997186581391
0.4 15.3991961371004
0.5 15.4981432051671
0.6 15.5963807029138
0.7 15.6937436131714
0.8 15.790038679126
0.9 15.885138096002
1 15.9789171938627
1.1 16.0712561297547
1.2 16.162055730603
1.3 16.2512334406275
1.4 16.3387225132156
1.5 16.4244873193384
1.6 16.5084978774875
1.7 16.590749384323
1.8 16.6712790546438
1.9 16.750121635765
2 16.8273714743378
2.1 16.9031043312473
2.2 16.9774044064728
2.3 17.0503433200766
2.4 17.122011309925
2.5 17.1925043915223
2.6 17.2619134323749
2.7 17.3303321690224
2.8 17.3978566710243
2.9 17.4645634579999
3 17.5305198405965
3.1 17.5957846181328
3.2 17.660399691816
3.3 17.7243736834424
3.4 17.7877085074187
3.5 17.8503923619702
3.6 17.9123999575255
3.7 17.9736987623012
3.8 18.0342430741192
3.9 18.0939603470633
4 18.1527792990504
4.1 18.2106095224024
4.2 18.2673557805444
4.3 18.3229174716036
4.4 18.3772009044391
4.5 18.4301139859851
4.6 18.4815774177826
4.7 18.5315176654656
4.8 18.5798805617938
4.9 18.6266067910394
5 18.6716631675217
};
\addplot [semithick, white!40!black]
table {%
0 15
0.1 15.1009112116844
0.2 15.2024778086378
0.3 15.3045574922387
0.4 15.4069956495381
0.5 15.509600892208
0.6 15.6121922019188
0.7 15.7145955857234
0.8 15.8166092989071
0.9 15.9180566123002
1 16.0187461964952
1.1 16.1184728632362
1.2 16.217024757762
1.3 16.3141962054819
1.4 16.4097738006026
1.5 16.503552699566
1.6 16.5953162175744
1.7 16.6848699818652
1.8 16.7720284799708
1.9 16.856603613024
2 16.9384457660769
2.1 17.0173902716751
2.2 17.0932788200276
2.3 17.1659705432437
2.4 17.235340668455
2.5 17.3012823823731
2.6 17.3636959858467
2.7 17.4225017569872
2.8 17.4776445170153
2.9 17.529081523508
3 17.5767851969604
3.1 17.6207375931127
3.2 17.6609398846286
3.3 17.6974093549509
3.4 17.7301750031348
3.5 17.759291652195
3.6 17.7848218054367
3.7 17.806842057486
3.8 17.8254413335203
3.9 17.8407226403843
4 17.8527847173018
4.1 17.8617286182505
4.2 17.867675620829
4.3 17.8707402836522
4.4 17.8710410879646
4.5 17.8687070078587
4.6 17.8638800348506
4.7 17.85669679661
4.8 17.8472967483655
4.9 17.8358196013852
5 17.8223905405219
};
\addplot [semithick, white!40!black]
table {%
0 15
0.1 15.0992655955879
0.2 15.1977983713245
0.3 15.2953804024954
0.4 15.3917810649862
0.5 15.4867441150155
0.6 15.5800265861776
0.7 15.6714011116725
0.8 15.7605870315899
0.9 15.8473751132892
1 15.9315522531752
1.1 16.0129024541937
1.2 16.0912312030398
1.3 16.1663623995319
1.4 16.2381267393138
1.5 16.3063950802353
1.6 16.3710361034845
1.7 16.4319508030423
1.8 16.4890725454922
1.9 16.5423489791017
2 16.5917719148349
2.1 16.6373304083026
2.2 16.6790346079434
2.3 16.7169068157261
2.4 16.7509938961563
2.5 16.7813663618406
2.6 16.8081030463885
2.7 16.8312953958346
2.8 16.8510548260444
2.9 16.8675048925956
3 16.8807622382725
3.1 16.8909541918929
3.2 16.8982170197288
3.3 16.9026802072601
3.4 16.9044726466279
3.5 16.9037345605366
3.6 16.9006045876387
3.7 16.8952199378071
3.8 16.8877163948792
3.9 16.8782236525233
4 16.8668592024939
4.1 16.8537356591935
4.2 16.8389690769055
4.3 16.8226650900059
4.4 16.8049265826441
4.5 16.7858561116676
4.6 16.7655681605683
4.7 16.7441675841368
4.8 16.7217640410212
4.9 16.6984559655458
5 16.6743430762009
};
\addplot [semithick, white!40!black]
table {%
0 15
0.1 15.0992621718737
0.2 15.1979338485972
0.3 15.295820268836
0.4 15.3927228428025
0.5 15.4884065293259
0.6 15.5826642723537
0.7 15.6753066814752
0.8 15.7661046822425
0.9 15.8549136364642
1 15.9415960453031
1.1 16.0260233038704
1.2 16.1080988629859
1.3 16.1877478046408
1.4 16.2649164635826
1.5 16.3395954457167
1.6 16.4117845422708
1.7 16.4815130811762
1.8 16.5488614454761
1.9 16.6139151181787
2 16.6768207638882
2.1 16.7377121440724
2.2 16.7967379792766
2.3 16.8540334111105
2.4 16.9097558849042
2.5 16.9640717354856
2.6 17.0171432531869
2.7 17.0691330746063
2.8 17.1202051335563
2.9 17.1705051459029
3 17.2201608240727
3.1 17.2692926721391
3.2 17.318001375722
3.3 17.3663463262778
3.4 17.4143755256799
3.5 17.4621185640201
3.6 17.5095883540133
3.7 17.5567845756495
3.8 17.6036887306311
3.9 17.6502482954501
4 17.6964081949702
4.1 17.742091981678
4.2 17.7872119916946
4.3 17.8316734723129
4.4 17.8753854489171
4.5 17.9182540947153
4.6 17.9601978980261
4.7 18.0011386868326
4.8 18.041017785885
4.9 18.0797664649182
5 18.1173473595277
};
\addplot [semithick, white!40!black]
table {%
0 15
0.1 15.0978852866774
0.2 15.1938914909517
0.3 15.2877401669741
0.4 15.3791409544786
0.5 15.4677858750656
0.6 15.5533845024224
0.7 15.6356694575195
0.8 15.7143003509091
0.9 15.7890487622188
1 15.8596927718894
1.1 15.9260182337015
1.2 15.9878574100346
1.3 16.0450716358569
1.4 16.0975430286559
1.5 16.1452208784772
1.6 16.1880618216174
1.7 16.2260628462969
1.8 16.2592753620885
1.9 16.2877780577821
2 16.3117006720844
2.1 16.3311792324177
2.2 16.3463824487248
2.3 16.357483536403
2.4 16.3646870914645
2.5 16.3682242847881
2.6 16.3683340759347
2.7 16.3652603294349
2.8 16.3592607128831
2.9 16.3506009935396
3 16.3395201298674
3.1 16.3262664126774
3.2 16.3110849432856
3.3 16.2941905368825
3.4 16.2757857457029
3.5 16.2560685850894
3.6 16.2352258330266
3.7 16.2134283506413
3.8 16.1908319457036
3.9 16.1675659325866
4 16.1437419758924
4.1 16.1194587721214
4.2 16.094802553772
4.3 16.0698464461497
4.4 16.0446560443207
4.5 16.0192872278152
4.6 15.9938069366832
4.7 15.9682705257007
4.8 15.9427416456328
4.9 15.9172632424439
5 15.8918965605862
};
\addplot [semithick, white!40!black]
table {%
0 15
0.1 15.0987087867176
0.2 15.1963663874953
0.3 15.292753594455
0.4 15.3876476164442
0.5 15.4807925039285
0.6 15.5719618910604
0.7 15.6609500474591
0.8 15.7475035466861
0.9 15.8314695511362
1 15.9127064346281
1.1 15.9910856491935
1.2 16.0665206128958
1.3 16.1389506234774
1.4 16.2083417258296
1.5 16.2747150331417
1.6 16.3381045729032
1.7 16.3985771542277
1.8 16.4562593349541
1.9 16.5112880560146
2 16.5638640678293
2.1 16.6141789233082
2.2 16.662443829222
2.3 16.7088535487751
2.4 16.7536279001963
2.5 16.7969968803559
2.6 16.8391863251465
2.7 16.8804194227151
2.8 16.9209183323502
2.9 16.9608856136035
3 17.000497932563
3.1 17.0399243445148
3.2 17.0793094496338
3.3 17.1187474004371
3.4 17.1583163657598
3.5 17.1980699740372
3.6 17.238041424324
3.7 17.2782449667359
3.8 17.3186713407873
3.9 17.3592692994524
4 17.3999827841922
4.1 17.4407312495569
4.2 17.4814166722355
4.3 17.5219328365508
4.4 17.5621753372693
4.5 17.6020331727189
4.6 17.641407306034
4.7 17.6802011538885
4.8 17.7183389146453
4.9 17.7557309214448
5 17.7923254590624
};
\addplot [semithick, white!40!black]
table {%
0 15
0.1 15.09915648505
0.2 15.1974878895702
0.3 15.2947712396286
0.4 15.3907708230762
0.5 15.4852260640182
0.6 15.5778898346894
0.7 15.6685311723665
0.8 15.7568641072694
0.9 15.8426771548236
1 15.9257556544368
1.1 16.005882761487
1.2 16.0828649658257
1.3 16.1565279767369
1.4 16.2267052409232
1.5 16.2932724564349
1.6 16.3561037684744
1.7 16.4151063023128
1.8 16.4702210868744
1.9 16.5214045649874
2 16.5686576932324
2.1 16.6119795034189
2.2 16.6513911064125
2.3 16.6869254544878
2.4 16.7186406272937
2.5 16.7466187518688
2.6 16.7709503747958
2.7 16.7917381851116
2.8 16.8091045664065
2.9 16.8231840677742
3 16.8341028783519
3.1 16.8419979697816
3.2 16.8470146192528
3.3 16.8492898569783
3.4 16.8489593198312
3.5 16.8461690587757
3.6 16.8410629490364
3.7 16.8337824256457
3.8 16.8244666105304
3.9 16.8132472397501
4 16.8002433069004
4.1 16.7855684760407
4.2 16.7693387601816
4.3 16.751659502041
4.4 16.7326328418782
4.5 16.712359878981
4.6 16.6909535663736
4.7 16.6685169261699
4.8 16.6451578933804
4.9 16.6209724157139
5 16.5960587283478
};
\addplot [semithick, white!40!black]
table {%
0 15
0.1 15.1013888515416
0.2 15.2039058768077
0.3 15.3074419472696
0.4 15.4118800000477
0.5 15.51705773662
0.6 15.622829457602
0.7 15.7290549857143
0.8 15.8355803781893
0.9 15.9422694489928
1 16.0489735840145
1.1 16.1555327875544
1.2 16.2617772848058
1.3 16.3675414664828
1.4 16.472654716336
1.5 16.5769477536571
1.6 16.6802421079469
1.7 16.7823773724327
1.8 16.8832041408378
1.9 16.9825610903061
2 17.0803319157667
2.1 17.1763767959752
2.2 17.2705550247666
2.3 17.3627324981957
2.4 17.4527876544493
2.5 17.5406081341252
2.6 17.6260823913997
2.7 17.7091151014619
2.8 17.789627831161
2.9 17.8675400649912
3 17.9427875442092
3.1 18.0153067302613
3.2 18.0850423317912
3.3 18.1519452309823
3.4 18.2159764165351
3.5 18.2771126132423
3.6 18.3353336630159
3.7 18.3906323404371
3.8 18.4430096977248
3.9 18.4924733042444
4 18.5390334153744
4.1 18.5826967601758
4.2 18.6234881284689
4.3 18.661428271574
4.4 18.6965469476978
4.5 18.7288863445525
4.6 18.7585021718551
4.7 18.7854522567168
4.8 18.809803746987
4.9 18.8316279527753
5 18.850991525274
};
\addplot [semithick, white!40!black]
table {%
0 15
0.1 15.0979473438237
0.2 15.1941553391118
0.3 15.2883624254837
0.4 15.3803004813592
0.5 15.4696768142352
0.6 15.5562250210031
0.7 15.6397026550293
0.8 15.7198034438645
0.9 15.7963391041522
1 15.8691340609131
1.1 15.9380271785025
1.2 16.0029089948903
1.3 16.063700580513
1.4 16.1203519219725
1.5 16.1728814198316
1.6 16.2213212815776
1.7 16.2657420064164
1.8 16.3062788412634
1.9 16.3430881298017
2 16.3763872857283
2.1 16.4063934257592
2.2 16.4333520839427
2.3 16.4574980209652
2.4 16.4790956889381
2.5 16.498427012008
2.6 16.5157742148721
2.7 16.5314173328205
2.8 16.5456391458904
2.9 16.5587122772043
3 16.5708767272652
3.1 16.5823710197897
3.2 16.5934135813546
3.3 16.6041726754305
3.4 16.6147982853319
3.5 16.6254185243164
3.6 16.6361415466304
3.7 16.6470530911788
3.8 16.658214298604
3.9 16.6696437918026
4 16.6813486602329
4.1 16.6933133793658
4.2 16.7055017950105
4.3 16.7178667945031
4.4 16.7303579234369
4.5 16.7429139140725
4.6 16.7554848354543
4.7 16.7680171414151
4.8 16.780474346435
4.9 16.7928004111454
5 16.8049750807174
};
\addplot [semithick, white!40!black]
table {%
0 15
0.1 15.0980179125951
0.2 15.1943481136414
0.3 15.2887310181918
0.4 15.3809000143755
0.5 15.4705640310452
0.6 15.5574573626393
0.7 15.6413377810506
0.8 15.7218995850139
0.9 15.7989523702134
1 15.8723173833402
1.1 15.9418292007103
1.2 16.0073722936197
1.3 16.0688609677078
1.4 16.1262370376341
1.5 16.1795091571464
1.6 16.2286987559673
1.7 16.2738652787065
1.8 16.315130896551
1.9 16.352638655097
2 16.386591501987
2.1 16.4171921135334
2.2 16.4446712865697
2.3 16.4692506694764
2.4 16.491181362686
2.5 16.5107324975948
2.6 16.5281741266772
2.7 16.5437750947382
2.8 16.5578082019053
2.9 16.570537754214
3 16.5821969821025
3.1 16.5930185669557
3.2 16.6032170914626
3.3 16.6129598302984
3.4 16.6223968912794
3.5 16.6316587487335
3.6 16.6408571346309
3.7 16.6500826710131
3.8 16.6594028806185
3.9 16.6688451496352
4 16.6784251456058
4.1 16.6881371263018
4.2 16.6979562343315
4.3 16.7078466310526
4.4 16.7177690666203
4.5 16.7276740778761
4.6 16.7375235639818
4.7 16.7472752863801
4.8 16.7569031839808
4.9 16.766362069298
5 16.7756402280358
};
\addplot [semithick, white!40!black]
table {%
0 15
0.1 15.1026046459562
0.2 15.2074866292132
0.3 15.3146125355735
0.4 15.423948849027
0.5 15.5353993406684
0.6 15.6488947408726
0.7 15.7643667100108
0.8 15.8817636368651
0.9 16.0010283634305
1 16.1220927252752
1.1 16.2448790843233
1.2 16.3692881705053
1.3 16.4952185577166
1.4 16.6225653543812
1.5 16.7512050980958
1.6 16.8810076460158
1.7 17.0118511507966
1.8 17.1436232478846
1.9 17.2761793522467
2 17.4094301706611
2.1 17.5432451530409
2.2 17.6774767332705
2.3 17.8119659708385
2.4 17.9465582747181
2.5 18.0810914869458
2.6 18.2153928488469
2.7 18.3493008410038
2.8 18.4826583359498
2.9 18.6152805153911
3 18.747005730713
3.1 18.8776570508353
3.2 19.0070488651691
3.3 19.1349893258401
3.4 19.2612966218349
3.5 19.3857901070699
3.6 19.5082862722164
3.7 19.6286159917181
3.8 19.7466143527427
3.9 19.8621141314158
4 19.9749648762609
4.1 20.0850039019059
4.2 20.1920865439397
4.3 20.2960696743419
4.4 20.3968296680137
4.5 20.4942611943762
4.6 20.5882735831681
4.7 20.6787924734082
4.8 20.7657638619118
4.9 20.8491474047032
5 20.9289119832902
};
\addplot [semithick, white!40!black]
table {%
0 15
0.1 15.0989568481029
0.2 15.1971133773263
0.3 15.2942684618357
0.4 15.3902199925995
0.5 15.4847279103106
0.6 15.5775855135806
0.7 15.6686060308749
0.8 15.7575627593701
0.9 15.8443262887497
1 15.9287799511802
1.1 16.0108218750316
1.2 16.0903909237291
1.3 16.1674509111505
1.4 16.241994342207
1.5 16.3140651852101
1.6 16.3837221157251
1.7 16.4510542820227
1.8 16.5162123759385
1.9 16.5793522887914
2 16.6406977384225
2.1 16.7004584764953
2.2 16.7588599210507
2.3 16.816104473738
2.4 16.872417662563
2.5 16.9280301044917
2.6 16.9831645297702
2.7 17.0380386208348
2.8 17.0928643730846
2.9 17.1478255286635
3 17.203080099306
3.1 17.2587731996247
3.2 17.3150187901893
3.3 17.3718739590981
3.4 17.4293785801063
3.5 17.4875416578494
3.6 17.5463488449125
3.7 17.6057658115082
3.8 17.6657319990867
3.9 17.7261399110431
4 17.7868812051173
4.1 17.8478194142325
4.2 17.9087989339792
4.3 17.9696574821259
4.4 18.0302374672995
4.5 18.0903756157213
4.6 18.1499208948015
4.7 18.2087290867591
4.8 18.2666806663904
4.9 18.323644315714
5 18.3795328949997
};
\addplot [semithick, white!40!black]
table {%
0 15
0.1 15.0999161903637
0.2 15.1998351323667
0.3 15.2995987906125
0.4 15.3990479916391
0.5 15.4979795724343
0.6 15.5962214043551
0.7 15.6936162025303
0.8 15.7899810564039
0.9 15.885202739224
1 15.9791740420444
1.1 16.0717956565956
1.2 16.1629921494029
1.3 16.2527057695041
1.4 16.3408983712894
1.5 16.4275646778766
1.6 16.5127080005177
1.7 16.5963564126384
1.8 16.6785850275284
1.9 16.7594647185087
2 16.8391301476557
2.1 16.9176952770271
2.2 16.9952814085097
2.3 17.0719910103954
2.4 17.1479448383781
2.5 17.223266052464
2.6 17.2980697512275
2.7 17.3724708027439
2.8 17.4465818521576
2.9 17.520488762351
3 17.5942650075101
3.1 17.6679713173748
3.2 17.7416444306417
3.3 17.8152790323455
3.4 17.888860229458
3.5 17.9623519795094
3.6 18.0357009333613
3.7 18.1088433105848
3.8 18.1816978367427
3.9 18.2541492515872
4 18.3260856613089
4.1 18.3973719447349
4.2 18.467864252955
4.3 18.5374140739934
4.4 18.6058811970546
4.5 18.6731261437436
4.6 18.7390222960367
4.7 18.8034517820913
4.8 18.8663196424332
4.9 18.9275257664793
5 18.987003719949
};
\addplot [semithick, white!40!black]
table {%
0 15
0.1 15.0973884067674
0.2 15.1925813603709
0.3 15.2852940874135
0.4 15.3752360635043
0.5 15.4620948310462
0.6 15.545586768692
0.7 15.625455341068
0.8 15.7013729609175
0.9 15.7731471634958
1 15.8406030021178
1.1 15.9035849416818
1.2 15.9619997971151
1.3 16.0157894093881
1.4 16.064930999965
1.5 16.1094813661182
1.6 16.1495155976337
1.7 16.1851501761376
1.8 16.2165762933735
1.9 16.2440109895373
2 16.2677361075317
2.1 16.2880362909781
2.2 16.3052289518231
2.3 16.3196162042574
2.4 16.3315324884524
2.5 16.3413300820832
2.6 16.3493606614572
2.7 16.3559699125875
2.8 16.3615027333787
2.9 16.3662905441968
3 16.3706234732498
3.1 16.3747886306332
3.2 16.3790465240295
3.3 16.3835960671786
3.4 16.3886125704985
3.5 16.3942414001657
3.6 16.4006032178385
3.7 16.4077897498189
3.8 16.4158616778731
3.9 16.4248273560905
4 16.4346819355463
4.1 16.4453937625303
4.2 16.4569033022021
4.3 16.4691391583548
4.4 16.4820250204976
4.5 16.4954698243027
4.6 16.5093935131843
4.7 16.5237123432297
4.8 16.5383618604181
4.9 16.5532542841079
5 16.5683461754693
};
\addplot [semithick, white!40!black]
table {%
0 15
0.1 15.0992655369447
0.2 15.1980184174085
0.3 15.296076079457
0.4 15.3932567155961
0.5 15.4893364377527
0.6 15.5841269302015
0.7 15.6774585831838
0.8 15.7691292114325
0.9 15.8590276400792
1 15.9470554867773
1.1 16.0331293169178
1.2 16.1172031240245
1.3 16.1992541567001
1.4 16.2792883377456
1.5 16.3573578847505
1.6 16.4335300387492
1.7 16.5079001124807
1.8 16.5806240496486
1.9 16.6518580991018
2 16.7218284635415
2.1 16.7907431630163
2.2 16.8588219627093
2.3 16.9262577749854
2.4 16.9932646228756
2.5 17.0600577785977
2.6 17.1268420795394
2.7 17.1938163970665
2.8 17.2611712699376
2.9 17.3290633406767
3 17.3976255891027
3.1 17.4669745412577
3.2 17.5371920845485
3.3 17.6083010580594
3.4 17.6803073736122
3.5 17.7531832150444
3.6 17.8268763222711
3.7 17.9013151543596
3.8 17.9764013827497
3.9 18.0519882886229
4 18.1279316104904
4.1 18.2040572175229
4.2 18.2801722451793
4.3 18.3560784720807
4.4 18.4315848475201
4.5 18.506496198312
4.6 18.5806298704203
4.7 18.6538132607171
4.8 18.7259008453137
4.9 18.7967376850462
5 18.8662156939734
};
\addplot [semithick, white!40!black]
table {%
0 15
0.1 15.0979948280714
0.2 15.1941394669201
0.3 15.2881503047277
0.4 15.3797279605481
0.5 15.4685594404648
0.6 15.5543426788595
0.7 15.636797190805
0.8 15.7155652055232
0.9 15.7903921842248
1 15.8610245687843
1.1 15.9272106988764
1.2 15.9887388505679
1.3 16.0454241618354
1.4 16.0970952761922
1.5 16.1436441378126
1.6 16.1849644330606
1.7 16.2209907578294
1.8 16.251702414256
1.9 16.277108892129
2 16.2972629145495
2.1 16.3122271244149
2.2 16.3220985624895
2.3 16.3269903771217
2.4 16.3270475293007
2.5 16.3224476365469
2.6 16.3133814342725
2.7 16.3000504114002
2.8 16.2826783272448
2.9 16.2615103381581
3 16.2367710919388
3.1 16.2087023264262
3.2 16.1775558356221
3.3 16.1435697258913
3.4 16.1069753708782
3.5 16.0680135676754
3.6 16.0269210858226
3.7 15.9839249598173
3.8 15.9392454011314
3.9 15.8930896705197
4 15.8456436790351
4.1 15.7970880564584
4.2 15.7475984808947
4.3 15.6973363092875
4.4 15.646452951083
4.5 15.5950919149296
4.6 15.5434076717027
4.7 15.4915377148854
4.8 15.4396212777797
4.9 15.3877771312029
5 15.3361281492132
};
\addplot [semithick, white!40!black]
table {%
0 15
0.1 15.0985856110466
0.2 15.1958705762847
0.3 15.2916066520412
0.4 15.3855330901028
0.5 15.4773676991353
0.6 15.5668433531217
0.7 15.6537121503244
0.8 15.7376628870753
0.9 15.8184754799466
1 15.8959308464101
1.1 15.9698120108822
1.2 16.0399355329623
1.3 16.1061415675291
1.4 16.1682836436102
1.5 16.2262686711734
1.6 16.2800058295541
1.7 16.3294406316963
1.8 16.3745614221254
1.9 16.4153774293633
2 16.4519450697225
2.1 16.4843226795568
2.2 16.512595530806
2.3 16.5368578374697
2.4 16.5572317841154
2.5 16.5738649646364
2.6 16.5869132991438
2.7 16.5965417881535
2.8 16.6029327676479
2.9 16.6062793795244
3 16.6067582903143
3.1 16.6045565727202
3.2 16.5998648841521
3.3 16.5928562670543
3.4 16.5836976546152
3.5 16.5725601362831
3.6 16.5596087878964
3.7 16.5450003737789
3.8 16.5288838806044
3.9 16.5113927567699
4 16.4926453559278
4.1 16.4727515249903
4.2 16.4518170321681
4.3 16.4299358297208
4.4 16.4071966234209
4.5 16.383683220906
4.6 16.3594909213274
4.7 16.3347041506367
4.8 16.3094135413392
4.9 16.2836938054263
5 16.2576286710628
};
\addplot [semithick, white!40!black]
table {%
0 15
0.1 15.0993424910023
0.2 15.1980925330619
0.3 15.2960478609207
0.4 15.3929981407927
0.5 15.4887012539125
0.6 15.5829358672036
0.7 15.6754969115071
0.8 15.7661343217223
0.9 15.8546740051537
1 15.9409432100187
1.1 16.0247718957094
1.2 16.1060156564325
1.3 16.1845495878989
1.4 16.2602624040223
1.5 16.333083606609
1.6 16.4029459664281
1.7 16.4698126454819
1.8 16.5336877682693
1.9 16.5945841604752
2 16.6525673879854
2.1 16.7076943970604
2.2 16.7600393330597
2.3 16.8096753680718
2.4 16.856698657422
2.5 16.9012210588153
2.6 16.943356263801
2.7 16.9832245506524
2.8 17.0209566772643
2.9 17.0566797709155
3 17.0905093349162
3.1 17.1225622041591
3.2 17.1529496859867
3.3 17.1817594560981
3.4 17.2090735718115
3.5 17.2349706493492
3.6 17.2595203160058
3.7 17.2827853748208
3.8 17.3048191501311
3.9 17.3256553156377
4 17.3453207401397
4.1 17.3638291823206
4.2 17.3811910226248
4.3 17.3974081305268
4.4 17.4124833387071
4.5 17.4264183570349
4.6 17.4392270761564
4.7 17.4509207048372
4.8 17.461522800815
4.9 17.4710468543547
5 17.4795225250176
};
\addplot [semithick, white!40!black]
table {%
0 15
0.1 15.1009244908068
0.2 15.2025442259782
0.3 15.3047221193912
0.4 15.4073105245035
0.5 15.5101227872017
0.6 15.6129855012248
0.7 15.7157326207087
0.8 15.8181732503019
0.9 15.9201436838948
1 16.021467696992
1.1 16.1219574394567
1.2 16.2214202422634
1.3 16.3196701478894
1.4 16.4165161974328
1.5 16.5117765371438
1.6 16.6052596323152
1.7 16.6967956276197
1.8 16.7862270155529
1.9 16.8733917471975
2 16.9581695680053
2.1 17.0404230640213
2.2 17.1200198435315
2.3 17.196839923378
2.4 17.2707788978534
2.5 17.3417473600462
2.6 17.4096605666708
2.7 17.4744513934854
2.8 17.5360736131794
2.9 17.5944874463634
3 17.6496662750769
3.1 17.7015895763152
3.2 17.7502503554664
3.3 17.7956511117548
3.4 17.8378040338053
3.5 17.8767413875845
3.6 17.9125002193094
3.7 17.9451294686709
3.8 17.974687216583
3.9 18.0012402838258
4 18.0248531952828
4.1 18.0455895963591
4.2 18.0635306060603
4.3 18.0787513131661
4.4 18.0913320740062
4.5 18.1013633312402
4.6 18.1089486308628
4.7 18.1141887504387
4.8 18.1171901779137
4.9 18.1180599853841
5 18.1168965209482
};
\addplot [semithick, white!40!black]
table {%
0 15
0.1 15.09707401128
0.2 15.1917559107197
0.3 15.2837574867853
0.4 15.3727888919382
0.5 15.4585353518744
0.6 15.5407184250054
0.7 15.6190893608257
0.8 15.6933299275119
0.9 15.7632719789835
1 15.828772124736
1.1 15.8897140351262
1.2 15.9460540722719
1.3 15.9977874687425
1.4 16.0449544241774
1.5 16.0876826601014
1.6 16.1261253917359
1.7 16.1604778163116
1.8 16.1910231305606
1.9 16.2180692508495
2 16.2419980106225
2.1 16.263191590335
2.2 16.2820648435664
2.3 16.2990041720966
2.4 16.3144288466505
2.5 16.3287700887845
2.6 16.3424529100357
2.7 16.3558891406882
2.8 16.3694801379086
2.9 16.3835993923882
3 16.3985696706218
3.1 16.414702472358
3.2 16.4322671838098
3.3 16.4514509517295
3.4 16.4724098828509
3.5 16.4952533428849
3.6 16.5200571062596
3.7 16.5468594892555
3.8 16.575657239803
3.9 16.6063777553534
4 16.6389382784116
4.1 16.6732200920235
4.2 16.7090664884986
4.3 16.746309768032
4.4 16.7847791304082
4.5 16.8242857729582
4.6 16.8646518768362
4.7 16.9057012479232
4.8 16.9472842944861
4.9 16.9892264681784
5 17.0314147698169
};
\addplot [semithick, white!40!black]
table {%
0 15
0.1 15.0985591397317
0.2 15.1959074874201
0.3 15.2918135993249
0.4 15.38604037485
0.5 15.4783210637434
0.6 15.5684153856496
0.7 15.656104004754
0.8 15.7411144272132
0.9 15.8232760015117
1 15.9024277384076
1.1 15.9784200207424
1.2 16.0511453414988
1.3 16.1205224521635
1.4 16.1864948574219
1.5 16.2490630748017
1.6 16.3082388088424
1.7 16.3640680978188
1.8 16.4166545852087
1.9 16.4661159493656
2 16.5126302866264
2.1 16.5563699517331
2.2 16.597529709723
2.3 16.6362933114775
2.4 16.672870855993
2.5 16.7074865321766
2.6 16.7403633196467
2.7 16.7717236955575
2.8 16.8017929807621
2.9 16.830783844888
3 16.8588835944779
3.1 16.8862761331297
3.2 16.9131265863525
3.3 16.9395554699042
3.4 16.9656686714175
3.5 16.9915530474563
3.6 17.0172776464725
3.7 17.042893861541
3.8 17.0684321669377
3.9 17.0938856253498
4 17.1192395451824
4.1 17.1444578807169
4.2 17.1694889418255
4.3 17.1942717422748
4.4 17.2187449961899
4.5 17.2428404123052
4.6 17.2665014751855
4.7 17.2896707531734
4.8 17.3123084008041
4.9 17.3343593892294
5 17.3558011743399
};
\addplot [semithick, white!40!black]
table {%
0 15
0.1 15.0988748752342
0.2 15.196753082015
0.3 15.2934092995898
0.4 15.3886094336281
0.5 15.4820915035843
0.6 15.5736141537203
0.7 15.6629545911427
0.8 15.749836851978
0.9 15.8340732661325
1 15.9154798155347
1.1 15.9938775193486
1.2 16.0691203248415
1.3 16.1410849425388
1.4 16.2096648605713
1.5 16.2748030629101
1.6 16.3364477872944
1.7 16.3945806922731
1.8 16.4492298290976
1.9 16.5004373622096
2 16.5482986681084
2.1 16.5929046810244
2.2 16.6343681352731
2.3 16.6728010160428
2.4 16.708340850709
2.5 16.7411434771217
2.6 16.771367740275
2.7 16.7991778015993
2.8 16.8247482556075
2.9 16.8482520200039
3 16.8698448570399
3.1 16.8896853606746
3.2 16.9079256339168
3.3 16.924689957248
3.4 16.9400942181911
3.5 16.9542487395596
3.6 16.967253234734
3.7 16.9791969729933
3.8 16.9901569586488
3.9 17.0001866860537
4 17.0093299410565
4.1 17.0176164621255
4.2 17.0250688218896
4.3 17.031699806201
4.4 17.0375208287414
4.5 17.0425392737465
4.6 17.0467743700417
4.7 17.0502405814155
4.8 17.0529642960391
4.9 17.0549584922596
5 17.0562547930249
};
\addplot [semithick, white!40!black]
table {%
0 15
0.1 15.0994670291761
0.2 15.1985358409671
0.3 15.2970242144496
0.4 15.3947473373751
0.5 15.491481102352
0.6 15.5870309958302
0.7 15.68121937707
0.8 15.7738339321697
0.9 15.8647427366565
1 15.9538211426168
1.1 16.0409536079964
1.2 16.1260545513596
1.3 16.2090589275463
1.4 16.2899230695234
1.5 16.3686441896023
1.6 16.445229002599
1.7 16.519712140169
1.8 16.5921788598048
1.9 16.6627161836663
2 16.7314738882505
2.1 16.798585952422
2.2 16.8641986534982
2.3 16.9284419004221
2.4 16.991466546625
2.5 17.0534296470242
2.6 17.1144824039626
2.7 17.17477564074
2.8 17.2344595455333
2.9 17.2936620563656
3 17.3524943754908
3.1 17.4110579670984
3.2 17.469431892456
3.3 17.5276521484877
3.4 17.5857434311115
3.5 17.6437098460009
3.6 17.7015379541757
3.7 17.7592014380738
3.8 17.8166552726794
3.9 17.8738191800328
4 17.9306126140212
4.1 17.986932342447
4.2 18.0426640462189
4.3 18.0976872256718
4.4 18.151886866847
4.5 18.2051461151512
4.6 18.2573606192641
4.7 18.3084316379502
4.8 18.3582816496418
4.9 18.406824666847
5 18.4540081275458
};
\addplot [semithick, white!40!black]
table {%
0 15
0.1 15.0992888505529
0.2 15.1979758485336
0.3 15.2958620203726
0.4 15.3927425014398
0.5 15.4883783261841
0.6 15.5825550386719
0.7 15.6750752319763
0.8 15.7656990432448
0.9 15.8542672823383
1 15.9406251621822
1.1 16.0246238684311
1.2 16.1061437344034
1.3 16.18508578177
1.4 16.2613686852053
1.5 16.3349539353868
1.6 16.4058094067599
1.7 16.4739330005645
1.8 16.5393689467054
1.9 16.6021684509048
2 16.6624398424444
2.1 16.7202807204902
2.2 16.7758048419102
2.3 16.829118468497
2.4 16.8803505593417
2.5 16.9296423152086
2.6 16.9771337569623
2.7 17.0229682279698
2.8 17.0672947951023
2.9 17.1102513912137
3 17.151960879186
3.1 17.1925430699095
3.2 17.2321048216547
3.3 17.270720106897
3.4 17.3084542197425
3.5 17.3453612062267
3.6 17.3814821003993
3.7 17.4168477002915
3.8 17.451474748541
3.9 17.4853527941347
4 17.5184667088106
4.1 17.5507839516016
4.2 17.5822644468974
4.3 17.6128603116713
4.4 17.6425260235575
4.5 17.6712139669693
4.6 17.6988887679962
4.7 17.7255154353029
4.8 17.7510750169225
4.9 17.7755384137447
5 17.7989006289595
};
\addplot [semithick, white!40!black]
table {%
0 15
0.1 15.1005728646495
0.2 15.2016778313115
0.3 15.303183091222
0.4 15.4049543479483
0.5 15.506810703182
0.6 15.6085986416975
0.7 15.7101757234886
0.8 15.8113817600211
0.9 15.9121055382821
1 16.012235719754
1.1 16.1116624974767
1.2 16.2102869246458
1.3 16.308022314509
1.4 16.4047933442153
1.5 16.5005442647455
1.6 16.5952221368039
1.7 16.6887952664371
1.8 16.7812664597457
1.9 16.8726291134765
2 16.9629352652988
2.1 17.0522130695652
2.2 17.1404931849908
2.3 17.2277939053375
2.4 17.3141488185808
2.5 17.3995941497961
2.6 17.4841596767178
2.7 17.5678799613103
2.8 17.6507923298991
2.9 17.7329125887386
3 17.8142548360418
3.1 17.8948230387302
3.2 17.97460610367
3.3 18.0535658588369
3.4 18.131661290983
3.5 18.2088410554824
3.6 18.2850427521329
3.7 18.3602017372908
3.8 18.4342441965142
3.9 18.5070751053267
4 18.5786043338473
4.1 18.64872418725
4.2 18.7173274179281
4.3 18.7843030263074
4.4 18.8495498682252
4.5 18.9129722736523
4.6 18.9744878019925
4.7 19.0340222617841
4.8 19.0915210963738
4.9 19.1469289769984
5 19.2002128504815
};
\addplot [semithick, white!40!black]
table {%
0 15
0.1 15.1003878366208
0.2 15.2008344626238
0.3 15.3011486204241
0.4 15.4011171405543
0.5 15.5005052192867
0.6 15.5990736460514
0.7 15.6965907751363
0.8 15.7927744945055
0.9 15.8873686779881
1 15.9800942762934
1.1 16.0706493040715
1.2 16.1587228340565
1.3 16.2440106000298
1.4 16.3261898169548
1.5 16.4049519939748
1.6 16.4799677595342
1.7 16.5509362218772
1.8 16.6175529266426
1.9 16.6795260414475
2 16.7365858602369
2.1 16.7884624006858
2.2 16.8349033081622
2.3 16.8756994587078
2.4 16.9106629476581
2.5 16.9396414140453
2.6 16.9625032039909
2.7 16.9791473825276
2.8 16.9895160356219
2.9 16.9935960127777
3 16.991394101246
3.1 16.9829461713788
3.2 16.968334661839
3.3 16.9476876512357
3.4 16.9211525221391
3.5 16.8889296043912
3.6 16.8512400608628
3.7 16.8083267819095
3.8 16.7604584742168
3.9 16.7079413006387
4 16.6510643087686
4.1 16.59013417438
4.2 16.5254880746101
4.3 16.4574517247244
4.4 16.3863456532544
4.5 16.3125001128581
4.6 16.2362576093951
4.7 16.1579400686965
4.8 16.0778571849625
4.9 15.9963139828564
5 15.9135738967323
};
\addplot [semithick, white!40!black]
table {%
0 15
0.1 15.1003103882094
0.2 15.2007493990515
0.3 15.3011439231023
0.4 15.40130717893
0.5 15.50102117189
0.6 15.600077321954
0.7 15.6982769239863
0.8 15.7953823215203
0.9 15.891196047393
1 15.9855084441142
1.1 16.0780983849196
1.2 16.1687469120513
1.3 16.2572452762568
1.4 16.3433803218218
1.5 16.4269584379144
1.6 16.5077761450952
1.7 16.585656310989
1.8 16.6604366917016
1.9 16.7319598799959
2 16.8001066927557
2.1 16.8647487797706
2.2 16.9257703740596
2.3 16.9830747483802
2.4 17.0365846743453
2.5 17.086244986371
2.6 17.1320097703737
2.7 17.1738520812464
2.8 17.2117704075974
2.9 17.2457797306775
3 17.2759036498841
3.1 17.3021782386858
3.2 17.3246589129271
3.3 17.343413569646
3.4 17.3585188050193
3.5 17.3700758236916
3.6 17.3781921961
3.7 17.3829855324475
3.8 17.3845830451475
3.9 17.3831224900013
4 17.3787331321762
4.1 17.3715461865669
4.2 17.3617089697102
4.3 17.3493602698939
4.4 17.3346394642344
4.5 17.3176928154913
4.6 17.2986791519946
4.7 17.2777486007972
4.8 17.2550528229556
4.9 17.2307393803316
5 17.2049429585958
};
\addplot [semithick, white!40!black]
table {%
0 15
0.1 15.0989313495277
0.2 15.1968900480025
0.3 15.293649576223
0.4 15.3889732113793
0.5 15.4825977274709
0.6 15.5742780318752
0.7 15.6637869673656
0.8 15.7508428597114
0.9 15.8352486351251
1 15.9168087161206
1.1 15.995330272287
1.2 16.0706507249494
1.3 16.142629319512
1.4 16.2111392335263
1.5 16.2761013981912
1.6 16.3374398217213
1.7 16.3951120449304
1.8 16.4491181631486
1.9 16.4994733053968
2 16.5462428511768
2.1 16.5894889706994
2.2 16.6292961313965
2.3 16.6657524173964
2.4 16.6989715347537
2.5 16.7290877076158
2.6 16.7562401378703
2.7 16.7805755895391
2.8 16.8022544527635
2.9 16.8214403895041
3 16.8382824606046
3.1 16.8529354339485
3.2 16.8655526103402
3.3 16.8762658604963
3.4 16.8852008367479
3.5 16.8924829576452
3.6 16.8982298024682
3.7 16.9025509540132
3.8 16.9055469227883
3.9 16.9072999392127
4 16.9078812249084
4.1 16.9073508818924
4.2 16.9057648009075
4.3 16.9031686697797
4.4 16.8996059688859
4.5 16.8951169275649
4.6 16.8897535936142
4.7 16.883561285034
4.8 16.8765947868283
4.9 16.8688956740532
5 16.8605187354746
};
\end{axis}

\end{tikzpicture}

	% This file was created by tikzplotlib v0.9.8.
\begin{tikzpicture}

\begin{axis}[
height=\figureheight,
scaled y ticks=false,
tick align=outside,
tick pos=left,
width=\figurewidth,
x grid style={white!69.0196078431373!black},
xlabel={Future horizon [\si{\second}]},
xmin=0, xmax=5,
xtick style={color=black},
xticklabel style={align=center},
y grid style={white!69.0196078431373!black},
ylabel={$\speedleadsymbol [\si{\meter\per\second}]$},
ymin=1.93918041688508, ymax=17.5397703660699,
ytick style={color=black},
yticklabel style={/pgf/number format/fixed,/pgf/number format/precision=3}
]
\addplot [semithick, white!40!black]
table {%
0 15
0.1 14.8971314448948
0.2 14.7916219454697
0.3 14.6836515262728
0.4 14.5734107482539
0.5 14.4611131192767
0.6 14.3469572635354
0.7 14.2311719308236
0.8 14.1140041862657
0.9 13.9956134311924
1 13.8761881713353
1.1 13.7559104693384
1.2 13.6349548007875
1.3 13.5134708008183
1.4 13.3915922616509
1.5 13.2694369648049
1.6 13.1470735290577
1.7 13.0245899651028
1.8 12.9020230474957
1.9 12.7793923045077
2 12.6567005623982
2.1 12.5339457635444
2.2 12.4110979561821
2.3 12.2881316663365
2.4 12.165025530246
2.5 12.0417489727418
2.6 11.9182698909284
2.7 11.7945606865359
2.8 11.6705968329382
2.9 11.5463611941513
3 11.4218590099368
3.1 11.2971207686017
3.2 11.1721741707072
3.3 11.0470940956509
3.4 10.9219736082439
3.5 10.7969221749835
3.6 10.6720527301534
3.7 10.5475058325289
3.8 10.4234296493534
3.9 10.2999890794122
4 10.1773576667649
4.1 10.0557281958685
4.2 9.93528844704672
4.3 9.81623474113417
4.4 9.69875709954914
4.5 9.58303074743803
4.6 9.46921505941318
4.7 9.35747005874313
4.8 9.24791743052793
4.9 9.14067771920761
5 9.03584117135492
};
\addplot [semithick, white!40!black]
table {%
0 15
0.1 14.8978402420288
0.2 14.7936855717308
0.3 14.6877563129538
0.4 14.5802864308615
0.5 14.4715244365121
0.6 14.3617075754496
0.7 14.251100204274
0.8 14.140000522096
0.9 14.0286033515115
1 13.9171316383473
1.1 13.8058010597527
1.2 13.694811063891
1.3 13.5843320471695
1.4 13.4745170853392
1.5 13.3654909722068
1.6 13.2573289334916
1.7 13.1501203527558
1.8 13.0438994232257
1.9 12.9386727730793
2 12.8344335166802
2.1 12.7311612176403
2.2 12.6287991565622
2.3 12.5272888485621
2.4 12.4265715177122
2.5 12.3265718333369
2.6 12.2272083199081
2.7 12.1284031202047
2.8 12.0300772356827
2.9 11.9321491294495
3 11.8345654898503
3.1 11.7372918723576
3.2 11.6402859098657
3.3 11.5435508560233
3.4 11.44710988346
3.5 11.3509989659303
3.6 11.2552565415272
3.7 11.1599514578822
3.8 11.065160556062
3.9 10.9709768565646
4 10.8775086450492
4.1 10.7848810857863
4.2 10.6932167241863
4.3 10.6026493463066
4.4 10.5133114802275
4.5 10.4253246892055
4.6 10.3387952852259
4.7 10.2538363650155
4.8 10.1705267020483
4.9 10.0889492692263
5 10.0091598863219
};
\addplot [semithick, white!40!black]
table {%
0 15
0.1 14.8998146426236
0.2 14.7992454359827
0.3 14.698594687572
0.4 14.5981754787068
0.5 14.4983061258135
0.6 14.399284674111
0.7 14.301425223897
0.8 14.2051012350041
0.9 14.1105222602927
1 14.0179089792108
1.1 13.9274572311512
1.2 13.8393089064086
1.3 13.7535609683381
1.4 13.670270215955
1.5 13.5894257354832
1.6 13.5109511930304
1.7 13.4347734633511
1.8 13.3607290278817
1.9 13.2886100237853
2 13.2181818784108
2.1 13.1491855322939
2.2 13.0813102772543
2.3 13.0142596376608
2.4 12.9477275418926
2.5 12.8813900817659
2.6 12.8149203978787
2.7 12.7480086830761
2.8 12.6803564800003
2.9 12.6116745416365
3 12.5417324775197
3.1 12.4703242200379
3.2 12.3972587762053
3.3 12.3224311737105
3.4 12.2457751695723
3.5 12.1672658617984
3.6 12.0868975852211
3.7 12.0047181229556
3.8 11.9208060328825
3.9 11.8352907224354
4 11.7483227558498
4.1 11.6600843898675
4.2 11.5707808866829
4.3 11.4806319364637
4.4 11.3898615222143
4.5 11.2986965831866
4.6 11.2073499694314
4.7 11.1160415624248
4.8 11.0249490366388
4.9 10.9342675922491
5 10.8441350297603
};
\addplot [semithick, white!40!black]
table {%
0 15
0.1 14.8948209193214
0.2 14.7852100734111
0.3 14.6712667244714
0.4 14.5531098009572
0.5 14.4308850091594
0.6 14.3047432791536
0.7 14.17487993785
0.8 14.0414876584771
0.9 13.9047506897954
1 13.7649094266902
1.1 13.6222260407167
1.2 13.4770061006687
1.3 13.3295509905966
1.4 13.180182373903
1.5 13.0292546784326
1.6 12.8770991018665
1.7 12.7240772266419
1.8 12.5705527227484
1.9 12.4168856508619
2 12.2634458812376
2.1 12.1106045767318
2.2 11.9587189755349
2.3 11.8081153312124
2.4 11.6591334149571
2.5 11.5120956732286
2.6 11.3673112377001
2.7 11.2250700502431
2.8 11.0856383598401
2.9 10.9492562993878
3 10.8161433430256
3.1 10.6865261809134
3.2 10.5605831011438
3.3 10.4384698609058
3.4 10.3203313800666
3.5 10.2062762434263
3.6 10.0963870647452
3.7 9.99073945978219
3.8 9.8893791085072
3.9 9.79230969577139
4 9.69954294870255
4.1 9.61108177292065
4.2 9.52688456696644
4.3 9.44691671299847
4.4 9.37113505596684
4.5 9.29946372486758
4.6 9.2318098925704
4.7 9.16808953817531
4.8 9.10819908873093
4.9 9.05201901225566
5 8.99945443196248
};
\addplot [semithick, white!40!black]
table {%
0 15
0.1 14.8995142501769
0.2 14.7984989012781
0.3 14.6972597449127
0.4 14.5961198533505
0.5 14.495401522973
0.6 14.395418208873
0.7 14.2965025048146
0.8 14.1990518303304
0.9 14.1033177427418
1 14.0095730305127
1.1 13.9180763654487
1.2 13.8290454030101
1.3 13.742657471588
1.4 13.659063082352
1.5 13.5783537994317
1.6 13.5005659394017
1.7 13.425738779534
1.8 13.3538393098861
1.9 13.2847864224108
2 13.218485944177
2.1 13.1548139180791
2.2 13.0935928580407
2.3 13.034639476405
2.4 12.9777607475442
2.5 12.9227358974889
2.6 12.869332252244
2.7 12.8173237634335
2.8 12.7664810616473
2.9 12.7165614282642
3 12.6673687834118
3.1 12.6187182941338
3.2 12.5704170412376
3.3 12.5223283961338
3.4 12.4743442277894
3.5 12.4263730731808
3.6 12.3783298397343
3.7 12.3301713231048
3.8 12.2818701843911
3.9 12.2334255733323
4 12.1848635146803
4.1 12.13622819523
4.2 12.0875729038843
4.3 12.0389671571769
4.4 11.9904883965135
4.5 11.9422131892824
4.6 11.8942041042333
4.7 11.8465395837966
4.8 11.7992671137966
4.9 11.7524504947065
5 11.7061212902957
};
\addplot [semithick, white!40!black]
table {%
0 15
0.1 14.8992322254841
0.2 14.7975939953964
0.3 14.6953616076756
0.4 14.59282228542
0.5 14.4902720831544
0.6 14.387988289808
0.7 14.2862673503846
0.8 14.185456472752
0.9 14.0857558024864
1 13.9873807652968
1.1 13.8905262091227
1.2 13.7953433366021
1.3 13.7019428564294
1.4 13.6104009220884
1.5 13.5207372892276
1.6 13.4329100815772
1.7 13.3468840803102
1.8 13.2625425988725
1.9 13.179730277585
2 13.0982676081624
2.1 13.0179546178977
2.2 12.9385447005916
2.3 12.8598027770369
2.4 12.7814870950606
2.5 12.7033396003497
2.6 12.6250993135339
2.7 12.5465193037492
2.8 12.4673617662902
2.9 12.3873970176245
3 12.3064460596527
3.1 12.2243540088127
3.2 12.1409765218613
3.3 12.0562460570481
3.4 11.9701290963776
3.5 11.882627370059
3.6 11.7937580775825
3.7 11.7035859922007
3.8 11.6122012510113
3.9 11.519736797141
4 11.4263442358096
4.1 11.3322037850226
4.2 11.2375122541039
4.3 11.1424796787618
4.4 11.0473182379808
4.5 10.9522391563937
4.6 10.8574391960512
4.7 10.7631210617345
4.8 10.6694464268871
4.9 10.5765904662049
5 10.4846775245885
};
\addplot [semithick, white!40!black]
table {%
0 15
0.1 14.8958426416315
0.2 14.7882293959599
0.3 14.6773247836924
0.4 14.5633201941918
0.5 14.4464183484592
0.6 14.3268369253336
0.7 14.2048342845687
0.8 14.08069271352
0.9 13.9546673939823
1 13.8270716166588
1.1 13.6982428746626
1.2 13.5685528323765
1.3 13.4383638801308
1.4 13.3080609914739
1.5 13.1780454582187
1.6 13.0486982436111
1.7 12.9204222911824
1.8 12.7936226590742
1.9 12.6686830889802
2 12.5460069550036
2.1 12.4259833039054
2.2 12.3089732707173
2.3 12.1952901008172
2.4 12.0852535131962
2.5 11.979150777123
2.6 11.8772453902741
2.7 11.7797766176451
2.8 11.6869482223478
2.9 11.5989142087341
3 11.5158129598041
3.1 11.4377753753789
3.2 11.3648677198714
3.3 11.2971208878788
3.4 11.2345541252655
3.5 11.1771360207014
3.6 11.124803112388
3.7 11.0774853371499
3.8 11.0350781109828
3.9 10.9974254860408
4 10.9643920703031
4.1 10.9358251806519
4.2 10.9115265714057
4.3 10.8913098988778
4.4 10.8749895600743
4.5 10.8623519899375
4.6 10.8531676512048
4.7 10.8472286464554
4.8 10.8443178282317
4.9 10.8442101544416
5 10.8467190067081
};
\addplot [semithick, white!40!black]
table {%
0 15
0.1 14.8970646591548
0.2 14.7915331358314
0.3 14.683598621831
0.4 14.5734710064594
0.5 14.4613760212115
0.6 14.3475348675902
0.7 14.2322006466799
0.8 14.1156532502966
0.9 13.9980957799897
1 13.8797685272565
1.1 13.7609139763285
1.2 13.6417754865146
1.3 13.5225742959025
1.4 13.4035264334607
1.5 13.284836014894
1.6 13.1666662818883
1.7 13.0491983266487
1.8 12.9325759326358
1.9 12.8169199046268
2 12.7023464166089
2.1 12.5889601748049
2.2 12.4768342974949
2.3 12.3660282582039
2.4 12.2566043994064
2.5 12.1486057994873
2.6 12.0420654471686
2.7 11.937012007567
2.8 11.8334640381254
2.9 11.731426339564
3 11.6309175028207
3.1 11.5319689499854
3.2 11.4345888964461
3.3 11.3388078114877
3.4 11.2446663514775
3.5 11.152200299903
3.6 11.061438037126
3.7 10.972426742412
3.8 10.8852090017897
3.9 10.7998238943247
4 10.7163255526122
4.1 10.6347755839322
4.2 10.5552197305478
4.3 10.4777144495756
4.4 10.4023141827472
4.5 10.3290563941895
4.6 10.2579629252671
4.7 10.1890651279918
4.8 10.1223663231187
4.9 10.0578690555545
5 9.99556713471695
};
\addplot [semithick, white!40!black]
table {%
0 15
0.1 14.8999585362012
0.2 14.7994448654683
0.3 14.6987342240164
0.4 14.5980998413452
0.5 14.4978348852809
0.6 14.3981907633331
0.7 14.2994312299737
0.8 14.2018618990827
0.9 14.1056017688037
1 14.0107641548255
1.1 13.917419581376
1.2 13.8255669767391
1.3 13.7351546612673
1.4 13.6460686830231
1.5 13.5581187742148
1.6 13.4710320198377
1.7 13.3845418788528
1.8 13.298262446919
1.9 13.2117753130815
2 13.1246102952613
2.1 13.0362863435418
2.2 12.9462783746075
2.3 12.8541131517699
2.4 12.7593103899901
2.5 12.6613928044371
2.6 12.5598979176831
2.7 12.4543986434404
2.8 12.3445066017869
2.9 12.2298864962951
3 12.110279770209
3.1 11.98547794726
3.2 11.8553300198771
3.3 11.7198227845718
3.4 11.5789984004174
3.5 11.4329845505882
3.6 11.2819507378743
3.7 11.1261382732211
3.8 10.9658446005618
3.9 10.8014600655183
4 10.6333829017356
4.1 10.4620674677777
4.2 10.2880137140476
4.3 10.1117315200546
4.4 9.93372619217456
4.5 9.75451055352335
4.6 9.57458286902199
4.7 9.39443006382663
4.8 9.21447546702179
4.9 9.03515921381626
5 8.85681925526624
};
\addplot [semithick, white!40!black]
table {%
0 15
0.1 14.9007191978875
0.2 14.802045907135
0.3 14.7043606689818
0.4 14.6080689579635
0.5 14.5135582860171
0.6 14.4212174091802
0.7 14.3314496456202
0.8 14.2447531916864
0.9 14.161457125763
1 14.0819130300734
1.1 14.0064601152058
1.2 13.9353846978779
1.3 13.8689264629864
1.4 13.807299627031
1.5 13.7506396805549
1.6 13.6990292128505
1.7 13.6525441275145
1.8 13.6111863056761
1.9 13.5748895219057
2 13.5435844573389
2.1 13.5171545526858
2.2 13.4954138089406
2.3 13.4781529282686
2.4 13.4651447835392
2.5 13.4561180627183
2.6 13.450778394673
2.7 13.4488332485095
2.8 13.4499746155605
2.9 13.4538561344196
3 13.4601850706453
3.1 13.4686642987723
3.2 13.4789721877633
3.3 13.4908315245155
3.4 13.5039936580208
3.5 13.5182125325967
3.6 13.5332427229432
3.7 13.5488822761341
3.8 13.5649412834398
3.9 13.5812479212643
4 13.5976710818523
4.1 13.6140893939792
4.2 13.6303907304139
4.3 13.6464846753039
4.4 13.6622990589294
4.5 13.6777666812765
4.6 13.6928074670038
4.7 13.7073711187846
4.8 13.7213871391117
4.9 13.7348107425506
5 13.7475782948387
};
\addplot [semithick, white!40!black]
table {%
0 15
0.1 14.8996897606937
0.2 14.7991275117076
0.3 14.6986479129534
0.4 14.5986108598398
0.5 14.4993645992935
0.6 14.4012613410606
0.7 14.3046733461187
0.8 14.2100526122367
0.9 14.1177118431342
1 14.0279936308511
1.1 13.9412357755133
1.2 13.8577414367268
1.3 13.7777749902006
1.4 13.7015853139606
1.5 13.6293625753681
1.6 13.5612507806376
1.7 13.4973933351828
1.8 13.4378754710523
1.9 13.3827243015803
2 13.3319684401976
2.1 13.2855963297912
2.2 13.2435358139548
2.3 13.2056865546918
2.4 13.1719355423242
2.5 13.1421282366219
2.6 13.1160870361142
2.7 13.0936308224563
2.8 13.0745589961843
2.9 13.058630561206
3 13.0456436625298
3.1 13.0353916159288
3.2 13.0276351046914
3.3 13.0221628000758
3.4 13.0187835760769
3.5 13.0172980166317
3.6 13.017500617862
3.7 13.0192189188495
3.8 13.0222828735107
3.9 13.0265261946469
4 13.0318189392536
4.1 13.0380353929047
4.2 13.0450476908233
4.3 13.0527475688335
4.4 13.0610412350809
4.5 13.0698329455446
4.6 13.0790134224432
4.7 13.0885012924805
4.8 13.0981971176611
4.9 13.1080200634087
5 13.1178821762907
};
\addplot [semithick, white!40!black]
table {%
0 15
0.1 14.8982725675812
0.2 14.7947860549036
0.3 14.6897599579994
0.4 14.5834196422098
0.5 14.4760117211423
0.6 14.3677577321676
0.7 14.2589022534897
0.8 14.1497182914439
0.9 14.0403518238215
1 13.9309643817052
1.1 13.821696906338
1.2 13.7126573844882
1.3 13.6039187041581
1.4 13.495519813367
1.5 13.3874594883342
1.6 13.2796743329074
1.7 13.1721149635291
1.8 13.064654093054
1.9 12.9571406172121
2 12.8493933332092
2.1 12.741223285155
2.2 12.6324068335833
2.3 12.5227427892006
2.4 12.4120294673439
2.5 12.3000602420735
2.6 12.1866330014406
2.7 12.0715620585653
2.8 11.9546783491459
2.9 11.8358372886094
3 11.7149381187647
3.1 11.5919145546806
3.2 11.4667205635666
3.3 11.339392333696
3.4 11.2099987643758
3.5 11.0786517315254
3.6 10.9454813866312
3.7 10.8106628147317
3.8 10.6743975496111
3.9 10.5369333574125
4 10.3985267751579
4.1 10.2594677330595
4.2 10.1200609273801
4.3 9.98062030433536
4.4 9.84145449938315
4.5 9.70286624646137
4.6 9.56514296804351
4.7 9.42856847521943
4.8 9.29337870115484
4.9 9.15981573559858
5 9.02806369491855
};
\addplot [semithick, white!40!black]
table {%
0 15
0.1 14.8981430508348
0.2 14.7945718663808
0.3 14.689524738939
0.4 14.583255214665
0.5 14.4760274465461
0.6 14.3680963418476
0.7 14.2597426970771
0.8 14.1512882252412
0.9 14.0429449664872
1 13.9349530193342
1.1 13.8275452402736
1.2 13.7209349038572
1.3 13.6153045366945
1.4 13.51081917862
1.5 13.4076104754384
1.6 13.3057606974213
1.7 13.2053639464209
1.8 13.1064580455081
1.9 13.0090485471741
2 12.9131294005425
2.1 12.818676973323
2.2 12.7256275675922
2.3 12.6339122233202
2.4 12.5434597357454
2.5 12.4541787318944
2.6 12.365969322624
2.7 12.2787344634166
2.8 12.1923735667901
2.9 12.1067782862251
3 12.0218706432895
3.1 11.937588223359
3.2 11.8538575689911
3.3 11.7706490665001
3.4 11.687953414484
3.5 11.605771616371
3.6 11.5241062251705
3.7 11.4429910136638
3.8 11.36246737327
3.9 11.2825917395475
4 11.2034389530404
4.1 11.1250991925073
4.2 11.0476605654018
4.3 10.9712236810225
4.4 10.8958902554258
4.5 10.8217525960074
4.6 10.7488880275047
4.7 10.677383699281
4.8 10.6072946263251
4.9 10.5386823428948
5 10.4715835425327
};
\addplot [semithick, white!40!black]
table {%
0 15
0.1 14.8966766599841
0.2 14.7904348235589
0.3 14.6814506535434
0.4 14.5699172245493
0.5 14.4560457103667
0.6 14.3400439458449
0.7 14.2221537574564
0.8 14.1026382101439
0.9 13.9816949654507
1 13.8595617824423
1.1 13.7364816028672
1.2 13.6127052422782
1.3 13.4884643890467
1.4 13.3639894436724
1.5 13.239506482988
1.6 13.1152033785183
1.7 12.991288114081
1.8 12.8679375491881
1.9 12.7453092248366
2 12.6235579749958
2.1 12.5028297154234
2.2 12.3832420361739
2.3 12.264896751167
2.4 12.1479004614395
2.5 12.0323413372421
2.6 11.9182973184074
2.7 11.8058398634217
2.8 11.6950286030412
2.9 11.5859082974483
3 11.4785319122861
3.1 11.37296487668
3.2 11.2692460248433
3.3 11.1674298623725
3.4 11.0675778093958
3.5 10.9697419742126
3.6 10.873964359311
3.7 10.7803017019934
3.8 10.6888023277457
3.9 10.5995053398474
4 10.5124633567994
4.1 10.4277342093398
4.2 10.3453554031684
4.3 10.2653744010571
4.4 10.187835302055
4.5 10.1127625904086
4.6 10.0401648838575
4.7 9.97005975148911
4.8 9.90243770221469
4.9 9.83728580331722
5 9.77458714767836
};
\addplot [semithick, white!40!black]
table {%
0 15
0.1 14.9023867532818
0.2 14.8066482045485
0.3 14.7132191219496
0.4 14.6225510728274
0.5 14.5350767873077
0.6 14.451213167537
0.7 14.371381021272
0.8 14.2961087891421
0.9 14.2256963565808
1 14.1604446828635
1.1 14.1006199401861
1.2 14.0463967796569
1.3 13.9978877569536
1.4 13.9551513769303
1.5 13.9181315123584
1.6 13.8866984436292
1.7 13.8607083187142
1.8 13.8399015334872
1.9 13.8239421401561
2 13.8124690019599
2.1 13.8050710555683
2.2 13.8012587893164
2.3 13.8005494657361
2.4 13.8024361249434
2.5 13.8063760497842
2.6 13.8118141246994
2.7 13.8182163084601
2.8 13.8250556485247
2.9 13.831796308401
3 13.8379890607666
3.1 13.8431964181795
3.2 13.8469943122185
3.3 13.8490594051444
3.4 13.8491197431992
3.5 13.8469492031652
3.6 13.8423461637774
3.7 13.8351795138092
3.8 13.8253601949148
3.9 13.8128644753651
4 13.797708091126
4.1 13.7799396373485
4.2 13.7596480094011
4.3 13.7369443562698
4.4 13.711958548899
4.5 13.6848387967072
4.6 13.6557211636736
4.7 13.6247633444854
4.8 13.5920867174524
4.9 13.5578488599107
5 13.522143606981
};
\addplot [semithick, white!40!black]
table {%
0 15
0.1 14.8974598744813
0.2 14.7927056880357
0.3 14.6859566203691
0.4 14.577451618119
0.5 14.4674391485184
0.6 14.3561673837713
0.7 14.2439150237361
0.8 14.13099827748
0.9 14.0176497502572
1 13.9041403475565
1.1 13.790744481776
1.2 13.6777342075616
1.3 13.5653575917284
1.4 13.4538588440575
1.5 13.3434640286502
1.6 13.2343598203269
1.7 13.1267474041111
1.8 13.0207912620748
1.9 12.9166257537908
2 12.8143849509999
2.1 12.7141850906533
2.2 12.6161052144367
2.3 12.5202033584202
2.4 12.4265376301783
2.5 12.3351405507534
2.6 12.2460301115511
2.7 12.1592176383179
2.8 12.0746991849864
2.9 11.9924469335956
3 11.9124484520667
3.1 11.8346978818543
3.2 11.759158960397
3.3 11.6858116116243
3.4 11.614645288572
3.5 11.5456380821352
3.6 11.4787578468795
3.7 11.4139910135999
3.8 11.3513171119975
3.9 11.2907076497086
4 11.2321543371464
4.1 11.1756525567394
4.2 11.1211809504859
4.3 11.0687308844006
4.4 11.0182955069414
4.5 10.9698527478799
4.6 10.923365307302
4.7 10.878810774758
4.8 10.8361431627874
4.9 10.7953188656141
5 10.7562918278468
};
\addplot [semithick, white!40!black]
table {%
0 15
0.1 14.9001418532433
0.2 14.8004835144608
0.3 14.701391732803
0.4 14.6032629023715
0.5 14.5064734383281
0.6 14.4114100516824
0.7 14.3184781426948
0.8 14.2281764927811
0.9 14.1408581026174
1 14.0569082438413
1.1 13.9767100906836
1.2 13.9006095432881
1.3 13.8289119059209
1.4 13.7619099925845
1.5 13.6998311866536
1.6 13.6428595571698
1.7 13.591174468144
1.8 13.5448996899074
1.9 13.5040917689303
2 13.4688154417739
2.1 13.4390869610376
2.2 13.4148549415634
2.3 13.3960287920279
2.4 13.3825018188925
2.5 13.3741170701788
2.6 13.3706882014168
2.7 13.3720214317172
2.8 13.3778957020384
2.9 13.3880349120729
3 13.4022028096049
3.1 13.4201493429171
3.2 13.4415806852136
3.3 13.4662205673084
3.4 13.4938110795222
3.5 13.5240755905045
3.6 13.5567266138513
3.7 13.5915082594809
3.8 13.6281627188337
3.9 13.6664279530829
4 13.7060851366876
4.1 13.7469136540983
4.2 13.7886882575607
4.3 13.8312059362936
4.4 13.874283158648
4.5 13.9177361994868
4.6 13.961368289123
4.7 14.0050180292479
4.8 14.0485125394176
4.9 14.0917012753699
5 14.1344368070246
};
\addplot [semithick, white!40!black]
table {%
0 15
0.1 14.8984221634881
0.2 14.7955934503107
0.3 14.6918015805728
0.4 14.5873635173749
0.5 14.4825879021883
0.6 14.3777967113241
0.7 14.2733395706207
0.8 14.1696327967372
0.9 14.0669954270914
1 13.9657899828574
1.1 13.8663883473709
1.2 13.7691545477891
1.3 13.674424799144
1.4 13.5825380078662
1.5 13.4938006723343
1.6 13.4084860767769
1.7 13.3268732511689
1.8 13.2492102034116
1.9 13.1756953444398
2 13.1065412105369
2.1 13.0419247041894
2.2 12.9819704882478
2.3 12.9267584916782
2.4 12.8763613897755
2.5 12.8308075432225
2.6 12.7900971807725
2.7 12.7542153977887
2.8 12.7231152314009
2.9 12.6966943659348
3 12.6748672638481
3.1 12.6575356852761
3.2 12.6445469761419
3.3 12.635742116246
3.4 12.6309675813381
3.5 12.6300359782057
3.6 12.6327393764617
3.7 12.6388852227412
3.8 12.6482646685449
3.9 12.6606435538643
4 12.6758226410308
4.1 12.6935933856883
4.2 12.7137250713122
4.3 12.7360053510081
4.4 12.760234384126
4.5 12.7862008292069
4.6 12.813679160383
4.7 12.8424747669967
4.8 12.8723836222571
4.9 12.9032120384903
5 12.934785965988
};
\addplot [semithick, white!40!black]
table {%
0 15
0.1 14.8991247728102
0.2 14.7974249300695
0.3 14.6951937506165
0.4 14.5927437518079
0.5 14.4903868083745
0.6 14.3884300373266
0.7 14.2872021967023
0.8 14.1870939806794
0.9 14.0883641879714
1 13.99129792555
1.1 13.896171496088
1.2 13.8032292765389
1.3 13.7126789493835
1.4 13.6247081500279
1.5 13.5394539795926
1.6 13.4570031985023
1.7 13.3774472488524
1.8 13.3008151538788
1.9 13.2270897022889
2 13.1562458898443
2.1 13.0882294889564
2.2 13.0229348068075
2.3 12.9602431568118
2.4 12.9000275927093
2.5 12.8421313548934
2.6 12.7863832128859
2.7 12.7326139893306
2.8 12.6806457930742
2.9 12.6302803023654
3 12.5813580585594
3.1 12.5337269652803
3.2 12.4872177941891
3.3 12.4417042295574
3.4 12.3970830524267
3.5 12.3532574193676
3.6 12.3101311832945
3.7 12.2676436933701
3.8 12.2257430459076
3.9 12.1843926095688
4 12.14358298279
4.1 12.1033174176638
4.2 12.0636009191858
4.3 12.0244546151786
4.4 11.9859074900027
4.5 11.9479845151954
4.6 11.9106964957377
4.7 11.8740720980913
4.8 11.8381128923378
4.9 11.8028343216609
5 11.7682302745097
};
\addplot [semithick, white!40!black]
table {%
0 15
0.1 14.891214878801
0.2 14.7745706855679
0.3 14.6499390799909
0.4 14.5171870024769
0.5 14.3762614210056
0.6 14.2270818864323
0.7 14.0696257001369
0.8 13.9037752875237
0.9 13.7294720876804
1 13.5467080396126
1.1 13.3554896693961
1.2 13.1559006881002
1.3 12.9480389563796
1.4 12.7320161651847
1.5 12.5080352860586
1.6 12.2762671420178
1.7 12.0369422772406
1.8 11.7902954006267
1.9 11.5366190321932
2 11.2761828161048
2.1 11.0093116315488
2.2 10.736365079862
2.3 10.4577284428255
2.4 10.1738251153782
2.5 9.88511282725232
2.6 9.59207145727263
2.7 9.29517806423119
2.8 8.99492551628699
2.9 8.69186046974879
3 8.38648985547136
3.1 8.07937761257717
3.2 7.77109322120711
3.3 7.46222492524043
3.4 7.15335173915753
3.5 6.84506342609801
3.6 6.5379434771155
3.7 6.23256561285832
3.8 5.92948785054211
3.9 5.62925602567934
4 5.33238097820303
4.1 5.03939269732496
4.2 4.75077868553927
4.3 4.46701645946338
4.4 4.18854301613721
4.5 3.91574556661291
4.6 3.64899093875
4.7 3.38861098515658
4.8 3.13488335942804
4.9 2.88804143298422
5 2.64829814184802
};
\addplot [semithick, white!40!black]
table {%
0 15
0.1 14.8986414804591
0.2 14.7960268792347
0.3 14.6924234473152
0.4 14.5881161088053
0.5 14.4833941984058
0.6 14.3785407402709
0.7 14.2738625824095
0.8 14.1697187713529
0.9 14.0663479748094
1 13.9640165210301
1.1 13.8629832284625
1.2 13.7634815360841
1.3 13.6657112584009
1.4 13.5698540675477
1.5 13.4760497199269
1.6 13.3843886129101
1.7 13.2949692204636
1.8 13.2078314525671
1.9 13.1229754473702
2 13.0403924337739
2.1 12.9600496942035
2.2 12.8818683824665
2.3 12.8057593161122
2.4 12.7316279124493
2.5 12.6593538640041
2.6 12.5888047715221
2.7 12.5198501314016
2.8 12.4523524383062
2.9 12.3861586386527
3 12.3211498666934
3.1 12.2572178572329
3.2 12.1942389274554
3.3 12.1321312263193
3.4 12.0708341477635
3.5 12.010294068675
3.6 11.9504578091221
3.7 11.8913050544342
3.8 11.8328229346623
3.9 11.7750124941639
4 11.7178980944278
4.1 11.6615173361868
4.2 11.6059070374707
4.3 11.5511185056706
4.4 11.4972078740363
4.5 11.4442244949201
4.6 11.392203169809
4.7 11.3411931847636
4.8 11.2912149053059
4.9 11.2422990031255
5 11.1944543194921
};
\addplot [semithick, white!40!black]
table {%
0 15
0.1 14.8989321535721
0.2 14.797014663272
0.3 14.6945540317025
0.4 14.591884353059
0.5 14.4893301092288
0.6 14.3872252815614
0.7 14.2859284349044
0.8 14.1858700135165
0.9 14.0873662218043
1 13.9907712297102
1.1 13.8964428229411
1.2 13.8047201086263
1.3 13.7159099534562
1.4 13.6303145340009
1.5 13.5481930186893
1.6 13.4697661029065
1.7 13.3952576867067
1.8 13.324849642275
1.9 13.258670856883
2 13.1968591887534
2.1 13.1395150818565
2.2 13.0866834043506
2.3 13.0383710632463
2.4 12.9945755420913
2.5 12.9552511949789
2.6 12.9203263386254
2.7 12.8897188839641
2.8 12.863319826019
2.9 12.8409709579998
3 12.8225398971445
3.1 12.8078848047534
3.2 12.796818317872
3.3 12.7891606808371
3.4 12.7847435983262
3.5 12.7833753280277
3.6 12.7848494702293
3.7 12.7889821615905
3.8 12.7955808448388
3.9 12.8044394698786
4 12.8153874384315
4.1 12.8282503637325
4.2 12.8428398065851
4.3 12.8589861779724
4.4 12.8765331573171
4.5 12.8953167759417
4.6 12.9151591408134
4.7 12.9359120056551
4.8 12.9574141638021
4.9 12.9795180674137
5 13.0020849092491
};
\addplot [semithick, white!40!black]
table {%
0 15
0.1 14.8993233934847
0.2 14.7979720067565
0.3 14.6962454012489
0.4 14.5944613409964
0.5 14.4929369252992
0.6 14.3919823511514
0.7 14.2919281718032
0.8 14.1931682925512
0.9 14.0959575461026
1 14.0005744378469
1.1 13.9072859100522
1.2 13.8163222989372
1.3 13.7278753802576
1.4 13.6421133692027
1.5 13.5591496393695
1.6 13.4790446734132
1.7 13.4018627698342
1.8 13.3276006970273
1.9 13.2562080774286
2 13.1876239801743
2.1 13.121758008021
2.2 13.0584672720943
2.3 12.9975996661794
2.4 12.9389940801201
2.5 12.882460705889
2.6 12.8277966278769
2.7 12.7748033671678
2.8 12.7232765600142
2.9 12.672995151574
3 12.6237809620874
3.1 12.5754652307013
3.2 12.5278667989997
3.3 12.480854389098
3.4 12.434322619601
3.5 12.3881778589746
3.6 12.3423301293095
3.7 12.2967282583686
3.8 12.2513335218022
3.9 12.206128305371
4 12.1611220028602
4.1 12.1163395404203
4.2 12.0718114067187
4.3 12.0275842525933
4.4 11.9837125926336
4.5 11.9402485398641
4.6 11.8972301247346
4.7 11.8547121706472
4.8 11.8127203739892
4.9 11.7712955443189
5 11.7304513620794
};
\addplot [semithick, white!40!black]
table {%
0 15
0.1 14.9005676516578
0.2 14.8015943757208
0.3 14.7034504409939
0.4 14.6065297556908
0.5 14.511210832499
0.6 14.4178716105659
0.7 14.3269050902474
0.8 14.2387948600205
0.9 14.1538578329118
1 14.0724328552494
1.1 13.9948457427731
1.2 13.9213705184464
1.3 13.8522352470967
1.4 13.7876418165839
1.5 13.7277157253663
1.6 13.6725288518656
1.7 13.6221477065163
1.8 13.5765642994458
1.9 13.5357053949913
2 13.4994927746248
2.1 13.4678035549039
2.2 13.440447644044
2.3 13.4172148129184
2.4 13.3978781057896
2.5 13.3821690010115
2.6 13.3697977833383
2.7 13.3604776372666
2.8 13.3539084956567
2.9 13.3497562173497
3 13.3477398218337
3.1 13.3475765729059
3.2 13.3489623672746
3.3 13.3516403003202
3.4 13.3553824237523
3.5 13.3599662644649
3.6 13.3651712647732
3.7 13.3708205798741
3.8 13.3767505144923
3.9 13.3828175573769
4 13.3889168112421
4.1 13.394954790643
4.2 13.4008477824822
4.3 13.4065329706257
4.4 13.4119642425753
4.5 13.4170998140526
4.6 13.4218848695767
4.7 13.4262921381519
4.8 13.4302722470477
4.9 13.4338003091189
5 13.4368297820879
};
\addplot [semithick, white!40!black]
table {%
0 15
0.1 14.904636317458
0.2 14.813131421123
0.3 14.7260373368509
0.4 14.6439286950377
0.5 14.5673395799556
0.6 14.4967930214429
0.7 14.432805452686
0.8 14.3760439842868
0.9 14.3268914099017
1 14.2857234682306
1.1 14.2528731214393
1.2 14.2285495848678
1.3 14.212885155617
1.4 14.2059468057713
1.5 14.2076460838672
1.6 14.2178144832926
1.7 14.2362540297026
1.8 14.262629942767
1.9 14.2965025554721
2 14.3374093686399
2.1 14.384815092809
2.2 14.438082179147
2.3 14.4965717772136
2.4 14.5596078844791
2.5 14.626462176065
2.6 14.6963849195672
2.7 14.7686502350663
2.8 14.842534477925
2.9 14.9172874617345
3 14.9922692149137
3.1 15.0668394203413
3.2 15.1403679972827
3.3 15.212336380425
3.4 15.2822878059792
3.5 15.3498134736005
3.6 15.414532779787
3.7 15.4761498426364
3.8 15.5344196873854
3.9 15.5891736436712
4 15.6402991070659
4.1 15.6877163578422
4.2 15.7314002198127
4.3 15.771354816258
4.4 15.8076160070425
4.5 15.8402512927155
4.6 15.869317751986
4.7 15.8949076461855
4.8 15.9170829589278
4.9 15.9359582765437
5 15.9515807099395
};
\addplot [semithick, white!40!black]
table {%
0 15
0.1 14.8990470629966
0.2 14.7971484669947
0.3 14.6945850082092
0.4 14.5916533130906
0.5 14.4886540401785
0.6 14.3858776097298
0.7 14.2836357052609
0.8 14.1822955196605
0.9 14.0820896078872
1 13.9832731410571
1.1 13.8860885082767
1.2 13.7907434848443
1.3 13.6974085092891
1.4 13.606229146289
1.5 13.5173003734573
1.6 13.4306629501036
1.7 13.3463638582249
1.8 13.2643816509409
1.9 13.1846529653492
2 13.107100414968
2.1 13.0316218581434
2.2 12.9580667393828
2.3 12.8862810884591
2.4 12.816103945204
2.5 12.7473504488524
2.6 12.6798260508593
2.7 12.6133425893432
2.8 12.5477101042656
2.9 12.4827297911054
3 12.4182448864223
3.1 12.3541128773579
3.2 12.2901846185932
3.3 12.2263659665811
3.4 12.1625894089946
3.5 12.0988044783358
3.6 12.034966726108
3.7 11.9710709653885
3.8 11.9071264972708
3.9 11.8431676012219
4 11.7792517179028
4.1 11.7154548994143
4.2 11.6518597453383
4.3 11.5885635274934
4.4 11.5256685592027
4.5 11.4632735862243
4.6 11.4014629869415
4.7 11.3403338409204
4.8 11.279950611745
4.9 11.2203906380257
5 11.1616989679119
};
\addplot [semithick, white!40!black]
table {%
0 15
0.1 14.8995798810191
0.2 14.7988265452951
0.3 14.6980714475504
0.4 14.5976719539654
0.5 14.4979736685944
0.6 14.3993275152074
0.7 14.3021052033097
0.8 14.206757554327
0.9 14.1136002158522
1 14.0229803092527
1.1 13.9352418266231
1.2 13.8506968306744
1.3 13.7696196695335
1.4 13.6922713046487
1.5 13.6188564165501
1.6 13.5495350669728
1.7 13.4844671634046
1.8 13.4237574836556
1.9 13.3674530841586
2 13.3156042457235
2.1 13.268221095093
2.2 13.2252536560528
2.3 13.1866213833602
2.4 13.1522314454315
2.5 13.1219486917564
2.6 13.0956140129733
2.7 13.0730633221607
2.8 13.0541112698378
2.9 13.0385296891289
3 13.0261272067041
3.1 13.0167062747081
3.2 13.0100337604572
3.3 13.0059002648215
3.4 13.0041149176679
3.5 13.0044752411074
3.6 13.006770858628
3.7 13.0108224693545
3.8 13.0164509460199
3.9 13.0234773672631
4 13.0317593906888
4.1 13.0411571188352
4.2 13.0515262268096
4.3 13.0627420072715
4.4 13.0746942970753
4.5 13.0872700605207
4.6 13.1003426862185
4.7 13.1138142088739
4.8 13.1275698921994
4.9 13.1415129311406
5 13.1555428340572
};
\addplot [semithick, white!40!black]
table {%
0 15
0.1 14.8986634842825
0.2 14.7959396653697
0.3 14.6920727507191
0.4 14.5873154750671
0.5 14.4819360311663
0.6 14.3761810979168
0.7 14.2703189507627
0.8 14.1646563129571
0.9 14.0593655895054
1 13.9546353686589
1.1 13.8506344473604
1.2 13.7474950223795
1.3 13.6453122093598
1.4 13.5441479065436
1.5 13.4440175034424
1.6 13.3448752024489
1.7 13.2466860337559
1.8 13.1493369290983
1.9 13.0526843297472
2 12.9565581494103
2.1 12.8607746844665
2.2 12.7651102640821
2.3 12.6693574829236
2.4 12.5733057702804
2.5 12.476733989935
2.6 12.3794216550451
2.7 12.281162909329
2.8 12.1817642064444
2.9 12.0810477682978
3 11.9788816921602
3.1 11.8751631216721
3.2 11.7698035581205
3.3 11.6627921912091
3.4 11.5541507298484
3.5 11.443938707074
3.6 11.3322317714844
3.7 11.2191507866299
3.8 11.1048414947321
3.9 10.9894925930492
4 10.8733062310058
4.1 10.7565148922096
4.2 10.6393655835183
4.3 10.5221164067693
4.4 10.4050236307656
4.5 10.2883394692161
4.6 10.1723011965152
4.7 10.0571472429595
4.8 9.94307194075312
4.9 9.83027885102518
5 9.71891849849109
};
\addplot [semithick, white!40!black]
table {%
0 15
0.1 14.8996504055046
0.2 14.799037976319
0.3 14.6984991569609
0.4 14.5983969851796
0.5 14.4990814445248
0.6 14.400908816108
0.7 14.3042559529088
0.8 14.209580938923
0.9 14.117205663566
1 14.0274838541764
1.1 13.9407665210867
1.2 13.857372327213
1.3 13.7775819360374
1.4 13.7016630836179
1.5 13.6298261740373
1.6 13.5622374288983
1.7 13.499062277232
1.8 13.4404114080372
1.9 13.3863363748903
2 13.3368929876962
2.1 13.2920956124091
2.2 13.2518974182828
2.3 13.2162193033893
2.4 13.184969345872
2.5 13.1580119535609
2.6 13.1351865962651
2.7 13.1163271632009
2.8 13.1012450775157
2.9 13.0897066812731
3 13.0815152279418
3.1 13.0764664094112
3.2 13.0743186142446
3.3 13.0748523610959
3.4 13.0778664174637
3.5 13.0831463361731
3.6 13.0904690380916
3.7 13.0996423021242
3.8 13.1104734162711
3.9 13.1227686472867
4 13.1363719058133
4.1 13.1511286198506
4.2 13.1668794128328
4.3 13.1834849347198
4.4 13.2008211562224
4.5 13.2187614515583
4.6 13.2371656953034
4.7 13.2559235640344
4.8 13.2749089801171
4.9 13.2940143786272
5 13.3131300837646
};
\addplot [semithick, white!40!black]
table {%
0 15
0.1 14.895593884514
0.2 14.7872968316049
0.3 14.6752257630099
0.4 14.5595106521164
0.5 14.4403113247054
0.6 14.3177802036848
0.7 14.1921085125304
0.8 14.0634863532854
0.9 13.9320637128864
1 13.7980332682773
1.1 13.6615953111958
1.2 13.5229722551903
1.3 13.3823741376761
1.4 13.2400134015078
1.5 13.0961173141364
1.6 12.9508766857676
1.7 12.8045101562233
1.8 12.6572131361336
1.9 12.5091765634633
2 12.3605854914832
2.1 12.2116282824235
2.2 12.0624771743078
2.3 11.9132956063649
2.4 11.7642583018352
2.5 11.6155313009451
2.6 11.4672762357571
2.7 11.3196483916324
2.8 11.1727958045064
2.9 11.026863786092
3 10.8819958184665
3.1 10.7383558143645
3.2 10.5960860757556
3.3 10.4553435258071
3.4 10.3162882819868
3.5 10.1790736930045
3.6 10.0438430343426
3.7 9.91074889016843
3.8 9.77993318025853
3.9 9.65152666942437
4 9.52566448712924
4.1 9.4024889427469
4.2 9.28211692156262
4.3 9.16467147344824
4.4 9.05026523426895
4.5 8.93898512301737
4.6 8.83090131562076
4.7 8.72608485733772
4.8 8.6245750619342
4.9 8.52639976265928
5 8.43158100511777
};
\addplot [semithick, white!40!black]
table {%
0 15
0.1 14.8991887640809
0.2 14.7976493716229
0.3 14.6956848639879
0.4 14.5936201311769
0.5 14.491775811502
0.6 14.3904719730116
0.7 14.2900505904998
0.8 14.1909205645441
0.9 14.0933608726147
1 13.9976795974078
1.1 13.9041790322788
1.2 13.8131315255738
1.3 13.7247731893694
1.4 13.6393237438447
1.5 13.5569523288115
1.6 13.4777806852746
1.7 13.4019340288783
1.8 13.3294796927738
1.9 13.2604354235683
2 13.1948159289791
2.1 13.1326032392935
2.2 13.073725543378
2.3 13.0180907179795
2.4 12.9655973850057
2.5 12.916109811193
2.6 12.8694741276091
2.7 12.8255352212398
2.8 12.7841239838139
2.9 12.745042008012
3 12.7081273772747
3.1 12.6732202613783
3.2 12.640135688544
3.3 12.608722407907
3.4 12.5788495934108
3.5 12.5503846667616
3.6 12.5231916968281
3.7 12.497167408818
3.8 12.4722129180606
3.9 12.4482372374039
4 12.4251797457146
4.1 12.4029879344704
4.2 12.381607423699
4.3 12.3610010768222
4.4 12.3411417869924
4.5 12.3219981233036
4.6 12.3035246450065
4.7 12.2856977347532
4.8 12.2684708987055
4.9 12.2518123115826
5 12.2356767683798
};
\addplot [semithick, white!40!black]
table {%
0 15
0.1 14.8987498053158
0.2 14.7966858782858
0.3 14.6941368125648
0.4 14.5914703426368
0.5 14.489031643681
0.6 14.3871949218681
0.7 14.2863625931565
0.8 14.1870239738888
0.9 14.0895761548636
1 13.9944697383145
1.1 13.9021755290232
1.2 13.8131625093755
1.3 13.7278729483108
1.4 13.6467648844331
1.5 13.5702621326435
1.6 13.4987659251155
1.7 13.4326781524117
1.8 13.3723856294096
1.9 13.3182120124285
2 13.270512779607
2.1 13.2295940648463
2.2 13.1957000045466
2.3 13.1690025800664
2.4 13.149662313165
2.5 13.1377779272955
2.6 13.1334060821107
2.7 13.1365762083396
2.8 13.1472659286645
2.9 13.1653639662967
3 13.19076799117
3.1 13.2233429498928
3.2 13.2628695917872
3.3 13.3090887036299
3.4 13.3617370955913
3.5 13.4204877026419
3.6 13.4849780798916
3.7 13.5548512181106
3.8 13.6297180385418
3.9 13.7091372890491
4 13.7927149208468
4.1 13.8800307667171
4.2 13.9706296428693
4.3 14.0640791680778
4.4 14.1599680382466
4.5 14.2578728326047
4.6 14.3573565956965
4.7 14.458028502419
4.8 14.5595041787889
4.9 14.6614131965621
5 14.7634348820785
};
\addplot [semithick, white!40!black]
table {%
0 15
0.1 14.9058111691772
0.2 14.8163843454962
0.3 14.7323114367264
0.4 14.6542018624374
0.5 14.5826230992297
0.6 14.5181205986772
0.7 14.4612258551056
0.8 14.4126310089219
0.9 14.3727029365707
1 14.3417871515645
1.1 14.3201714444744
1.2 14.3079934008895
1.3 14.3053030230998
1.4 14.3120658885642
1.5 14.3280671448301
1.6 14.3529981223304
1.7 14.3865152361423
1.8 14.4281100354161
1.9 14.4771627164473
2 14.5330162993674
2.1 14.5949383899113
2.2 14.6620875442314
2.3 14.7336403545907
2.4 14.8087315932711
2.5 14.8864485855367
2.6 14.9658638705008
2.7 15.0460865098242
2.8 15.1262423571386
2.9 15.205449302943
3 15.2829578953983
3.1 15.3580284148632
3.2 15.4299560308048
3.3 15.4981846237862
3.4 15.5622351801362
3.5 15.6217049735753
3.6 15.676235208963
3.7 15.7255700151749
3.8 15.7695243677039
3.9 15.8080208237729
4 15.8410378223757
4.1 15.8686017927506
4.2 15.8908145363718
4.3 15.9078077763206
4.4 15.9197457779886
4.5 15.9268336862617
4.6 15.9292667665175
4.7 15.927270659965
4.8 15.9210304320285
4.9 15.9107912250889
5 15.8967012381212
};
\addplot [semithick, white!40!black]
table {%
0 15
0.1 14.8974172203252
0.2 14.7924593349559
0.3 14.6853234787157
0.4 14.5762189027036
0.5 14.4653740164299
0.6 14.353004348095
0.7 14.2393544118134
0.8 14.1246938024798
0.9 14.0091985935047
1 13.8930739733755
1.1 13.7765187896097
1.2 13.6597212071027
1.3 13.5428429704124
1.4 13.426029918791
1.5 13.3094070827176
1.6 13.1930505858534
1.7 13.0770537300706
1.8 12.9614576757906
1.9 12.8462818360127
2 12.7315308359703
2.1 12.6172005465261
2.2 12.5032553342251
2.3 12.3896605751941
2.4 12.2763838970897
2.5 12.1633802093266
2.6 12.0506005747504
2.7 11.9379997505233
2.8 11.8255331768051
2.9 11.7131585587755
3 11.6008579274138
3.1 11.4886353276565
3.2 11.376488883981
3.3 11.2644619971776
3.4 11.1526165541644
3.5 11.0410282932852
3.6 10.9297754587509
3.7 10.8189646069064
3.8 10.7087094394407
3.9 10.5991391356691
4 10.4903945466297
4.1 10.3826342021936
4.2 10.2760120491955
4.3 10.1706917870791
4.4 10.066833090388
4.5 9.96458228964935
4.6 9.86407012091606
4.7 9.76543092268486
4.8 9.66876285713345
4.9 9.57416514293261
5 9.48170907981187
};
\addplot [semithick, white!40!black]
table {%
0 15
0.1 14.8978514594194
0.2 14.7936291953311
0.3 14.6875398705822
0.4 14.5797983988849
0.5 14.4706407624383
0.6 14.3602827210494
0.7 14.248965842049
0.8 14.1369574042156
0.9 14.0244131440692
1 13.9115101687014
1.1 13.7984110901635
1.2 13.6852557086134
1.3 13.5721527483084
1.4 13.4591847439279
1.5 13.3464032749658
1.6 13.2338033922997
1.7 13.1213959366938
1.8 13.0091250710694
1.9 12.8969128811561
2 12.784657561668
2.1 12.6722498570646
2.2 12.5595478756549
2.3 12.4464236322132
2.4 12.3327501765793
2.5 12.2183929439187
2.6 12.1032187347468
2.7 11.9871054652896
2.8 11.8699412867976
2.9 11.7516302680348
3 11.6321115674917
3.1 11.5113540828293
3.2 11.3893363261664
3.3 11.2661035379112
3.4 11.1417275087528
3.5 11.0163109911407
3.6 10.8899684284266
3.7 10.7628519352042
3.8 10.6351318794564
3.9 10.5070119021107
4 10.3787050869619
4.1 10.2504514635594
4.2 10.122497445206
4.3 9.99509868145751
4.4 9.86850563241855
4.5 9.74295938829567
4.6 9.6186855579338
4.7 9.4959086752152
4.8 9.3748100080769
4.9 9.25557437680594
5 9.13834107432377
};
\addplot [semithick, white!40!black]
table {%
0 15
0.1 14.8977959656151
0.2 14.7935632870369
0.3 14.6875208344989
0.4 14.579901330275
0.5 14.4709520721789
0.6 14.3609095371043
0.7 14.2500375947372
0.8 14.1386336105322
0.9 14.0268931439479
1 13.915040440175
1.1 13.8032930639339
1.2 13.691853370356
1.3 13.580895071513
1.4 13.4705753091958
1.5 13.3610238942269
1.6 13.2523216120153
1.7 13.1445636051909
1.8 13.0377909230144
1.9 12.9320172738611
2 12.8272434289027
2.1 12.7234566859002
2.2 12.6206083024469
2.3 12.5186469873789
2.4 12.4175213295833
2.5 12.3171631481712
2.6 12.2174978416592
2.7 12.1184539234224
2.8 12.0199581773777
2.9 11.9219340846092
3 11.8243324822401
3.1 11.7271226568076
3.2 11.6302649823106
3.3 11.5337639786963
3.4 11.4376434853059
3.5 11.3419390122181
3.6 11.2466879089382
3.7 11.1519572235798
3.8 11.057821210805
3.9 10.9643690668704
4 10.871705280614
4.1 10.7799506152909
4.2 10.6892223993521
4.3 10.5996491855224
4.4 10.5113582520446
4.5 10.4244655588554
4.6 10.339071795009
4.7 10.2552846429473
4.8 10.173177881081
4.9 10.092829208096
5 10.0142903437649
};
\addplot [semithick, white!40!black]
table {%
0 15
0.1 14.8964852261077
0.2 14.7899992990108
0.3 14.6807270621349
0.4 14.5688768750037
0.5 14.4546683586273
0.6 14.3383291517549
0.7 14.2201234172553
0.8 14.1003438246597
0.9 13.9792327166225
1 13.8570820166106
1.1 13.7341989235732
1.2 13.6109096644347
1.3 13.4875251547466
1.4 13.3643675244831
1.5 13.241761286224
1.6 13.1200023719035
1.7 12.9994058929464
1.8 12.8802725369509
1.9 12.7628787275411
2 12.647511588208
2.1 12.5344431301291
2.2 12.4239141328823
2.3 12.3161297072739
2.4 12.2112990301334
2.5 12.1096024392604
2.6 12.0112009129893
2.7 11.9162389099356
2.8 11.8248345415584
2.9 11.7370682625691
3 11.6530178874063
3.1 11.5727604032998
3.2 11.4963234826008
3.3 11.4237219750644
3.4 11.3549681408234
3.5 11.2900409738002
3.6 11.2288969868532
3.7 11.1714967997154
3.8 11.1177784861596
3.9 11.0676476461725
4 11.0210297169321
4.1 10.9778421599009
4.2 10.9379692008235
4.3 10.9013070943015
4.4 10.8677528581403
4.5 10.8371807611955
4.6 10.8094493717696
4.7 10.7844354394243
4.8 10.7619998922477
4.9 10.741999775455
5 10.7243125197473
};
\addplot [semithick, white!40!black]
table {%
0 15
0.1 14.898177567318
0.2 14.7946018966878
0.3 14.6895019268007
0.4 14.5831176965034
0.5 14.4757047095787
0.6 14.3675022664764
0.7 14.2587744162276
0.8 14.1498202510835
0.9 14.0408221596825
1 13.9319852094774
1.1 13.8235014770736
1.2 13.7155379348118
1.3 13.6082290528454
1.4 13.5016847422189
1.5 13.3959789604575
1.6 13.2911307663159
1.7 13.1871721524016
1.8 13.0840695982748
1.9 12.981761282759
2 12.8801656841001
2.1 12.7791882135714
2.2 12.6786967994963
2.3 12.5785662850656
2.4 12.4786701590189
2.5 12.378868551446
2.6 12.2790188528404
2.7 12.1789871997192
2.8 12.0786450365015
2.9 11.9778702954127
3 11.8765769209253
3.1 11.7747027946403
3.2 11.6721884484411
3.3 11.569034906147
3.4 11.4652688145911
3.5 11.3609413675528
3.6 11.2561125740141
3.7 11.1508795103952
3.8 11.0453550054252
3.9 10.9396804398364
4 10.834011231032
4.1 10.7285260163526
4.2 10.6234085727982
4.3 10.5188536906044
4.4 10.4150543377514
4.5 10.3121954743782
4.6 10.2104469175492
4.7 10.1099823107291
4.8 10.0109362309758
4.9 9.91344945568247
5 9.81762349387851
};
\addplot [semithick, white!40!black]
table {%
0 15
0.1 14.8988415640391
0.2 14.7964290572375
0.3 14.6930121420803
0.4 14.588848011894
0.5 14.484209380975
0.6 14.3793454148603
0.7 14.2745257081088
0.8 14.170059468382
0.9 14.0661150517461
1 13.9628745507129
1.1 13.8604977032614
1.2 13.7591033707264
1.3 13.6587716359576
1.4 13.5595461042561
1.5 13.4614199795062
1.6 13.3643229054793
1.7 13.2681945987914
1.8 13.1728919528557
1.9 13.0782406237331
2 12.9840371353477
2.1 12.8900642539768
2.2 12.7960639054393
2.3 12.7017978630739
2.4 12.6070240849049
2.5 12.511491076199
2.6 12.4149493082129
2.7 12.317166113252
2.8 12.2179238122785
2.9 12.1170240787675
3 12.0143181428471
3.1 11.9096882614322
3.2 11.8030355129834
3.3 11.6943451745145
3.4 11.5836376345572
3.5 11.4709761508727
3.6 11.3564428631798
3.7 11.2401681830855
3.8 11.1223108464112
3.9 11.0030779907497
4 10.8826899333286
4.1 10.7614000322277
4.2 10.6394796959061
4.3 10.5172114376165
4.4 10.3948758941571
4.5 10.2727511109386
4.6 10.1511002663749
4.7 10.0301866377985
4.8 9.9102274718075
4.9 9.79145034664964
5 9.67402457712796
};
\addplot [semithick, white!40!black]
table {%
0 15
0.1 14.8986760645078
0.2 14.7962002673187
0.3 14.69285350086
0.4 14.5889389236648
0.5 14.4847582606171
0.6 14.3806144627519
0.7 14.2768351832679
0.8 14.1738078712566
0.9 14.0718052947146
1 13.9711333352604
1.1 13.8720962141746
1.2 13.7749776149611
1.3 13.6800289894926
1.4 13.5874907976596
1.5 13.4975630090387
1.6 13.4104018925207
1.7 13.3261701379873
1.8 13.2449810069966
1.9 13.1669028715526
2 13.0920038642856
2.1 13.0203226553442
2.2 12.9518482908363
2.3 12.8865462989625
2.4 12.8243754532759
2.5 12.7652610496718
2.6 12.7091098751448
2.7 12.6558244359185
2.8 12.6052906875483
2.9 12.5573633551329
3 12.5119261076643
3.1 12.4688639436987
3.2 12.428031808065
3.3 12.3893091466409
3.4 12.3525913427295
3.5 12.3177657119867
3.6 12.2847124177157
3.7 12.2533387310864
3.8 12.2235510143658
3.9 12.1952555529841
4 12.1683871120236
4.1 12.1428853301039
4.2 12.1186818596
4.3 12.0957246454996
4.4 12.0739699798775
4.5 12.0533663080914
4.6 12.0338477494882
4.7 12.0153697042234
4.8 11.9978662015143
4.9 11.9812824346554
5 11.965556963394
};
\addplot [semithick, white!40!black]
table {%
0 15
0.1 14.9043289003925
0.2 14.8123671202475
0.3 14.7246702014498
0.4 14.641822930572
0.5 14.564363419566
0.6 14.4928303919111
0.7 14.4277591257839
0.8 14.3698411418023
0.9 14.3195020829318
1 14.2771708546776
1.1 14.243244535171
1.2 14.218009668557
1.3 14.2016805813946
1.4 14.1944199038046
1.5 14.1962438045043
1.6 14.2070987834039
1.7 14.2269016248766
1.8 14.255450797222
1.9 14.2924360600299
2 14.3375382685968
2.1 14.3903600883168
2.2 14.4504000140898
2.3 14.5171347912459
2.4 14.5900038725626
2.5 14.6683842762944
2.6 14.7516224813809
2.7 14.839078174829
2.8 14.9300983046839
2.9 15.0239802546288
3 15.1201191431344
3.1 15.2178963994496
3.2 15.3166800478011
3.3 15.4159192757804
3.4 15.5151146243503
3.5 15.6137894111088
3.6 15.7114820152797
3.7 15.8078037323193
3.8 15.9024015423901
3.9 15.9949738587455
4 16.0852809919594
4.1 16.1731023254595
4.2 16.2582575888144
4.3 16.3405976457071
4.4 16.4200087994715
4.5 16.4964050796088
4.6 16.5696901856299
4.7 16.6398120211194
4.8 16.7066996942143
4.9 16.7703336738575
5 16.830652641107
};
\addplot [semithick, white!40!black]
table {%
0 15
0.1 14.8960913176906
0.2 14.788942372082
0.3 14.6787300509775
0.4 14.565658526212
0.5 14.4499411784408
0.6 14.3318064987159
0.7 14.2115224457793
0.8 14.0893853156898
0.9 13.9656578053392
1 13.8406595415673
1.1 13.7147331702514
1.2 13.5882516779127
1.3 13.4615770569843
1.4 13.3350922687087
1.5 13.2091919801714
1.6 13.0842495260696
1.7 12.9606585975235
1.8 12.8388121974256
1.9 12.719079092134
2 12.6018475069945
2.1 12.4874890721457
2.2 12.3763450311181
2.3 12.2687085013649
2.4 12.1648777027393
2.5 12.0651169510823
2.6 11.9696661103205
2.7 11.8787414314001
2.8 11.7925236030765
2.9 11.711142377968
3 11.634714778624
3.1 11.5633494650928
3.2 11.4970908433881
3.3 11.4359500213262
3.4 11.3799278844122
3.5 11.3289756548731
3.6 11.2830132975455
3.7 11.2419560472593
3.8 11.205686023522
3.9 11.1740359072485
4 11.1468605233588
4.1 11.1239978290016
4.2 11.1052421663554
4.3 11.0904004732785
4.4 11.0792817000809
4.5 11.0716683693406
4.6 11.0673272119939
4.7 11.0660477657761
4.8 11.0676106111242
4.9 11.0717902167267
5 11.078398295577
};
\addplot [semithick, white!40!black]
table {%
0 15
0.1 14.8954980238836
0.2 14.7872917292647
0.3 14.6755365897026
0.4 14.5604174499001
0.5 14.4421296618221
0.6 14.3208884247161
0.7 14.1969520106806
0.8 14.0706012485947
0.9 13.9421033407431
1 13.8117890444976
1.1 13.680019041382
1.2 13.5471971404081
1.3 13.4137213678743
1.4 13.2800195760118
1.5 13.1465437300789
1.6 13.0137307930188
1.7 12.8820409880223
1.8 12.7519470021536
1.9 12.6239010548718
2 12.4983811391099
2.1 12.3758505335629
2.2 12.2567458961216
2.3 12.141447398408
2.4 12.0303427960737
2.5 11.9237842857027
2.6 11.822096943913
2.7 11.7255765053345
2.8 11.6344768109863
2.9 11.5489930421486
3 11.469296919933
3.1 11.3955476914984
3.2 11.3278294703551
3.3 11.2661761711837
3.4 11.2106043462305
3.5 11.161068204303
3.6 11.1174835746309
3.7 11.0797530074408
3.8 11.0477367819778
3.9 11.0212314679064
4 11.0000553437673
4.1 10.9840030924748
4.2 10.9728160624072
4.3 10.9662477010362
4.4 10.9640527039751
4.5 10.9659548102684
4.6 10.9716616720971
4.7 10.9809054668533
4.8 10.9934138125993
4.9 11.0089043760096
5 11.0271453075657
};
\addplot [semithick, white!40!black]
table {%
0 15
0.1 14.8958881155261
0.2 14.7882032268523
0.3 14.677086975275
0.4 14.5626983825148
0.5 14.4452191158975
0.6 14.3248299793208
0.7 14.2017500276829
0.8 14.0762083213114
0.9 13.9483917072821
1 13.8185330803894
1.1 13.6868766386185
1.2 13.5536887167766
1.3 13.4192226193051
1.4 13.2837383859322
1.5 13.1475071724238
1.6 13.0107673953034
1.7 12.8737821696284
1.8 12.736796124581
1.9 12.6000419751042
2 12.463753702347
2.1 12.3281615128062
2.2 12.1934738942229
2.3 12.0598791304791
2.4 11.927574182434
2.5 11.7967392185602
2.6 11.6675438446672
2.7 11.5401467542206
2.8 11.4146912475345
2.9 11.2913033772778
3 11.1701060102381
3.1 11.0512336626315
3.2 10.9347872624858
3.3 10.8208699026399
3.4 10.7095848993268
3.5 10.6010171245696
3.6 10.4952357955312
3.7 10.3923165606395
3.8 10.2923188334236
3.9 10.1952810857633
4 10.1012522144963
4.1 10.0102816691077
4.2 9.92238945518942
4.3 9.83760400780628
4.4 9.75594768245404
4.5 9.6774178508117
4.6 9.60199552413922
4.7 9.52966957681801
4.8 9.4604038514629
4.9 9.3941533357575
5 9.33087883131375
};
\addplot [semithick, white!40!black]
table {%
0 15
0.1 14.9012491692325
0.2 14.8034601932936
0.3 14.7070230769274
0.4 14.6123472169561
0.5 14.5198273865304
0.6 14.4298492785202
0.7 14.3428090786764
0.8 14.2591973522547
0.9 14.1793123541433
1 14.1034643711025
1.1 14.0319402743341
1.2 13.9649582618322
1.3 13.9026839214461
1.4 13.8452434817317
1.5 13.7926716866342
1.6 13.7449400237856
1.7 13.7020118603875
1.8 13.6637570701752
1.9 13.629977873531
2 13.6004607083382
2.1 13.5749473081569
2.2 13.5531091577703
2.3 13.534612571206
2.4 13.519104759891
2.5 13.5061963503151
2.6 13.4954824190359
2.7 13.4865700777145
2.8 13.4790643507592
2.9 13.472551390466
3 13.4666851384116
3.1 13.4611262349545
3.2 13.4555324423161
3.3 13.449635321679
3.4 13.4432058382268
3.5 13.4360411933831
3.6 13.4279518283741
3.7 13.4188041720796
3.8 13.4084918070971
3.9 13.396950709183
4 13.3841539522543
4.1 13.3700971866498
4.2 13.3548000486523
4.3 13.3383029457631
4.4 13.3206625029603
4.5 13.3019453974949
4.6 13.2822055330645
4.7 13.2615196780057
4.8 13.239934384183
4.9 13.2175248508521
5 13.1943231483771
};
\addplot [semithick, white!40!black]
table {%
0 15
0.1 14.8965547377598
0.2 14.7902942343602
0.3 14.6814221495329
0.4 14.5701716182795
0.5 14.4567792944901
0.6 14.3414995751898
0.7 14.2246243913669
0.8 14.1064844114405
0.9 13.9873666661292
1 13.8676146742625
1.1 13.747594647702
1.2 13.627697679797
1.3 13.5083011744188
1.4 13.3898028002326
1.5 13.2726039821292
1.6 13.1570847536858
1.7 13.0436420319804
1.8 12.9326697991531
1.9 12.8245308867984
2 12.7196099790179
2.1 12.618269313341
2.2 12.520835143933
2.3 12.4275810495671
2.4 12.338782792456
2.5 12.2546771612419
2.6 12.17547324918
2.7 12.1013557344489
2.8 12.0324706394197
2.9 11.9689060070132
3 11.9107407738553
3.1 11.8580410166231
3.2 11.8108046786037
3.3 11.7689947599144
3.4 11.7325649760744
3.5 11.7014164853329
3.6 11.6754182585704
3.7 11.6544361436547
3.8 11.6383028209404
3.9 11.6268006643666
4 11.6197386956947
4.1 11.6169072252905
4.2 11.6180542423847
4.3 11.622942157938
4.4 11.6313388103955
4.5 11.6429880700461
4.6 11.6576184144426
4.7 11.674985375593
4.8 11.6948384201184
4.9 11.7169244159862
5 11.7410300775553
};
\addplot [semithick, white!40!black]
table {%
0 15
0.1 14.8916157791761
0.2 14.7763495186342
0.3 14.6541938978044
0.4 14.5251760005315
0.5 14.3893524659354
0.6 14.2468164855653
0.7 14.0977259233941
0.8 13.9422100945307
0.9 13.780502827308
1 13.6129341760964
1.1 13.4398978674983
1.2 13.2619042072763
1.3 13.0794887159637
1.4 12.8932606734324
1.5 12.7039307293246
1.6 12.5122249271386
1.7 12.3189142906665
1.8 12.1248502560392
1.9 11.9308974131985
2 11.737970912838
2.1 11.5469933887117
2.2 11.3588914876238
2.3 11.1745056675407
2.4 10.994702421936
2.5 10.8203162129857
2.6 10.6521489467895
2.7 10.4909476079454
2.8 10.3373939991018
2.9 10.1920902321914
3 10.0555554821802
3.1 9.92828694309386
3.2 9.81066341584773
3.3 9.70293641513391
3.4 9.60530384116333
3.5 9.517847159433
3.6 9.44057750434042
3.7 9.37344823325305
3.8 9.31632677840453
3.9 9.26895059324268
4 9.23106659613646
4.1 9.20237010535687
4.2 9.18245371698158
4.3 9.17091563126626
4.4 9.16734393591722
4.5 9.17126855986693
4.6 9.18220103356826
4.7 9.19967604258131
4.8 9.22323821309927
4.9 9.25239593522408
5 9.28676550700761
};
\addplot [semithick, white!40!black]
table {%
0 15
0.1 14.8991142990832
0.2 14.7971582456513
0.3 14.6943865915815
0.4 14.5910588513702
0.5 14.4874516767119
0.6 14.3838129956109
0.7 14.2804090902245
0.8 14.1775457276348
0.9 14.0753760094051
1 13.9740614877462
1.1 13.8737357874988
1.2 13.7744836353084
1.3 13.6763479327732
1.4 13.5793280910195
1.5 13.4833665918834
1.6 13.3883371347791
1.7 13.2941227273228
1.8 13.2005137089118
1.9 13.1072693281647
2 13.0141133313888
2.1 12.9207569167297
2.2 12.8268699675556
2.3 12.7321513020309
2.4 12.6362952463315
2.5 12.5389904461154
2.6 12.4399312611308
2.7 12.338834043496
2.8 12.235436849575
2.9 12.1295068279913
3 12.0208678684214
3.1 11.9093804261257
3.2 11.7949346377047
3.3 11.6775196343948
3.4 11.5571651371707
3.5 11.4339556230225
3.6 11.3080007982595
3.7 11.1794649710433
3.8 11.0485483922099
3.9 10.915511961566
4 10.7806279877017
4.1 10.6442082728072
4.2 10.5065901185179
4.3 10.3681214788263
4.4 10.2291474487892
4.5 10.0900131241504
4.6 9.95104878970623
4.7 9.81258139340643
4.8 9.67488683596463
4.9 9.53825287154237
5 9.4028967783152
};
\addplot [semithick, white!40!black]
table {%
0 15
0.1 14.8999618607912
0.2 14.7998495284937
0.3 14.7000018614518
0.4 14.6007798572953
0.5 14.5025348868303
0.6 14.4056165665962
0.7 14.3103924171157
0.8 14.2173090528043
0.9 14.1266615208551
1 14.0387690690795
1.1 13.9539401500268
1.2 13.8724401765314
1.3 13.7944926177117
1.4 13.7202979045685
1.5 13.6499910923473
1.6 13.5836554374459
1.7 13.521372938032
1.8 13.4631568943902
1.9 13.4089629817645
2 13.3587413790434
2.1 13.3124036847413
2.2 13.2698006572421
2.3 13.2307649090698
2.4 13.1951157917825
2.5 13.1626354466384
2.6 13.1330871592606
2.7 13.1062362795878
2.8 13.0818360799292
2.9 13.059610297393
3 13.0393293942105
3.1 13.0207652039184
3.2 13.003668845282
3.3 12.9878354859124
3.4 12.9730863676394
3.5 12.9592474213956
3.6 12.9461453911224
3.7 12.9336468226993
3.8 12.921628907988
3.9 12.9099858678945
4 12.8986461363854
4.1 12.8875494349729
4.2 12.8766413177158
4.3 12.8658863620307
4.4 12.8552623908431
4.5 12.8447479552822
4.6 12.8343081139064
4.7 12.8239319127957
4.8 12.8135847747905
4.9 12.8032521994268
5 12.7928992505382
};
\addplot [semithick, white!40!black]
table {%
0 15
0.1 14.8969887386131
0.2 14.7914508601212
0.3 14.6835975246912
0.4 14.5736647351179
0.5 14.4618948634451
0.6 14.3485394128742
0.7 14.2338840663047
0.8 14.1182526896085
0.9 14.0019063803111
1 13.88515403092
1.1 13.7683180571592
1.2 13.6517327219074
1.3 13.5357136686963
1.4 13.4205852761456
1.5 13.3066651876296
1.6 13.1942410540314
1.7 13.0836162647167
1.8 12.9750751298018
1.9 12.868871278546
2 12.7652696196148
2.1 12.6645148007313
2.2 12.5668148896169
2.3 12.4723403989598
2.4 12.3812630065022
2.5 12.2937217980675
2.6 12.2098344456992
2.7 12.1297026860726
2.8 12.053400765295
2.9 11.9809612275744
3 11.9124192266364
3.1 11.8478063292141
3.2 11.7871039869052
3.3 11.7302831371949
3.4 11.6773144267236
3.5 11.6281356224338
3.6 11.5826627911089
3.7 11.5408192513241
3.8 11.5025077063573
3.9 11.4676007574979
4 11.4359945879259
4.1 11.4075773604052
4.2 11.3822071452914
4.3 11.3597556290601
4.4 11.3400981944232
4.5 11.3230904448781
4.6 11.3085726703517
4.7 11.2964064150072
4.8 11.2864387980794
4.9 11.278516731164
5 11.2725067676687
};
\end{axis}

\end{tikzpicture}

	\caption{50 potential future situations samples from the \ac{kde} that is constructed using data from the \ac{ngsim} data set. 
		The left (right) plot assumes a current situation in which the speed of the lead vehicle is $\speedleadsymbol=\SI{15}{\meter\per\second}$ and the acceleration is $\accelerationleadsymbol=\SI{1}{\meter\per\second\squared}$ ($\accelerationleadsymbol=\SI{-1}{\meter\per\second\squared}$).}
	\label{fig:speed profiles}
\end{figure}

To estimate $\probabilitycond{\collision}{\situationcurrent, \situationfuture}$ (\cref{sec:estimate collision}), we used the driver behavior model \ac{idmplus} \autocite{schakel2010effects} for modeling the ego vehicle response.
In addition to \ac{idmplus}, we assumed that the driver has a reaction time that is similarly distributed as $\timereact$ in \cref{sec:wang stamatiadis explanation} and that the \ac{madr} is similarly distributed as $\accelerationmax$ in \cref{sec:wang stamatiadis explanation}.
The result of a single simulation was
\begin{equation}
	\simulationresult = \begin{cases}
		\speedlead{\timeend} - \speedego{\timeend} & \text{if collision} \\
		\gap{\timeend} & \text{otherwise}
	\end{cases},
\end{equation}
where $\timeend$ denotes the final time of the simulation.
The simulation ended if there was a collision or if the gap between the ego vehicle and lead vehicle was not decreasing.
Clearly, $\simulationresult<0$ indicates a collision, so $\spacecollision=(-\infty, 0)$.
The number of simulations was based on the threshold $\simulationthreshold=0.1$.

To calculate $\probabilitycond{\collision}{\situationcurrent}$, we created a grid of points $\situationcurrentinstance{\simulationindex}$, $\simulationindex\in\{1,\ldots,\numberofdesignpoints\}$ using the method explained in \cref{sec:final metric calculation} with $\weightmatrix$ being a diagonal matrix with on its diagonal: 0.25, 4, 0.25, and 16.
For the bandwidth matrix $\bandwidthnw$, we used a diagonal matrix with 0.0024, 0.068, 0.0025, and 0.019 on its diagonal.



\subsection{Analyzing the surrogate safety metric based on the \acs{ngsim} data set}
\label{sec:analyzing ngsim metric}

The heat maps in \cref{fig:heatmaps} show how the developed surrogate safety metric depends on the input variables $\speedleadsymbol$ and $\gapsymbol$. 
The other two parameters, $\speedegosymbol$ and $\accelerationleadsymbol$ are fixed for each heat map.
The heat maps shows that the estimated probability of collision is practically 0 if both $\speedleadsymbol$ and $\gapsymbol$ are large.
This seems reasonable, because in that case, the ego vehicle is at a safe distance of its lead vehicle while the approaching speed is small. 
If however, the difference in speed increases, i.e., for lower values of $\speedleadsymbol$, the risk increases. 
The same applies for a decreasing distance between the two vehicles.
For small values of $\speedegosymbol$ and $\accelerationleadsymbol$, the estimated probability of collision is practically 1.
The left and center heat maps of \cref{fig:heatmaps} show that for a higher speed of the ego vehicle, the probability of collision is estimated to be higher.
Similarly, the right and center heat maps of \cref{fig:heatmaps} show that for a lower initial acceleration of the lead vehicle, the probability of collision is estimated to be higher.

\setlength{\figurewidth}{.35\linewidth}
\setlength{\figureheight}{0.8\figurewidth}
\begin{figure}
	\centering
	% This file was created by tikzplotlib v0.9.8.
\begin{tikzpicture}

\begin{groupplot}[group style={group size=3 by 1}]
\nextgroupplot[
height=\figureheight,
scaled y ticks=false,
tick align=outside,
tick pos=left,
title={$\speedegosymbol=\SI{25}{\meter\per\second}$, $\accelerationleadsymbol=\SI{0}{\meter\per\second\squared}$},
width=\figurewidth,
x grid style={white!69.0196078431373!black},
xlabel={$\speedleadsymbol$ [\si{\meter\per\second}]},
xmin=10, xmax=20,
xtick style={color=black},
xticklabel style={align=center},
y grid style={white!69.0196078431373!black},
ylabel={$\gapsymbol$ [\si{\meter}]},
ymin=1, ymax=20,
ytick style={color=black},
yticklabel style={/pgf/number format/fixed,/pgf/number format/precision=3}
]
\addplot [draw=none, fill=white!95.2941176470588!black]
table{%
x  y
20 5.94961917011027
20 6.04081632653061
20 6.42857142857143
20 6.81632653061224
20 7.20408163265306
20 7.59183673469388
20 7.97959183673469
20 8.36734693877551
20 8.75510204081633
20 9.14285714285714
20 9.53061224489796
20 9.91836734693877
20 10.3061224489796
20 10.6938775510204
20 11.0816326530612
20 11.469387755102
20 11.8571428571429
20 12.2448979591837
20 12.6326530612245
20 13.0204081632653
20 13.4081632653061
20 13.7959183673469
20 14.1836734693878
20 14.5714285714286
20 14.9591836734694
20 15.3469387755102
20 15.734693877551
20 16.1224489795918
20 16.5102040816327
20 16.8979591836735
20 17.2857142857143
20 17.6734693877551
20 18.0612244897959
20 18.4489795918367
20 18.8367346938775
20 19.2244897959184
20 19.6122448979592
20 20
19.7959183673469 20
19.5918367346939 20
19.3877551020408 20
19.1836734693878 20
18.9795918367347 20
18.7755102040816 20
18.5714285714286 20
18.3673469387755 20
18.1632653061224 20
17.9591836734694 20
17.7551020408163 20
17.5510204081633 20
17.3469387755102 20
17.1428571428571 20
16.9387755102041 20
16.734693877551 20
16.530612244898 20
16.3265306122449 20
16.1224489795918 20
15.9183673469388 20
15.7142857142857 20
15.5102040816327 20
15.3061224489796 20
15.1020408163265 20
14.8979591836735 20
14.6938775510204 20
14.4897959183673 20
14.2857142857143 20
14.0816326530612 20
13.8775510204082 20
13.6734693877551 20
13.469387755102 20
13.265306122449 20
13.0612244897959 20
12.8571428571429 20
12.6530612244898 20
12.4907310278751 20
12.5391353811429 19.6122448979592
12.5879832615713 19.2244897959184
12.637314785981 18.8367346938775
12.6530612244898 18.7159034186207
12.6954425390225 18.4489795918367
12.7579589512152 18.0612244897959
12.8210777444238 17.6734693877551
12.8571428571429 17.4562751691435
12.8909788477584 17.2857142857143
12.9695595246306 16.8979591836735
13.0488754455018 16.5102040816327
13.0612244897959 16.4516496347277
13.1421009442336 16.1224489795918
13.2385092332426 15.734693877551
13.265306122449 15.6299199048811
13.3473856272882 15.3469387755102
13.4615124300259 14.9591836734694
13.469387755102 14.9333480046091
13.5910919182961 14.5714285714286
13.6734693877551 14.3313504942011
13.7280459505846 14.1836734693878
13.8750167165084 13.7959183673469
13.8775510204082 13.7894914720041
14.035928744317 13.4081632653061
14.0816326530612 13.301647896044
14.2066994954626 13.0204081632653
14.2857142857143 12.8492904155318
14.3881831473694 12.6326530612245
14.4897959183673 12.4269202898339
14.5811888605716 12.2448979591837
14.6938775510204 12.0313223212227
14.7869151555279 11.8571428571429
14.8979591836735 11.6607275705413
15.0073049708067 11.469387755102
15.1020408163265 11.3140166576465
15.2453160792652 11.0816326530612
15.3061224489796 10.9900074314211
15.5051040892636 10.6938775510204
15.5102040816327 10.6868919168228
15.7142857142857 10.4137937372529
15.7958822709599 10.3061224489796
15.9183673469388 10.1592241331448
16.1215859936204 9.91836734693877
16.1224489795918 9.9174459538873
16.3265306122449 9.70448842766538
16.493084692059 9.53061224489796
16.530612244898 9.49563253721587
16.734693877551 9.30618051894902
16.9066605482219 9.14285714285714
16.9387755102041 9.11574824918068
17.1428571428571 8.93980326854362
17.3449819581728 8.75510204081633
17.3469387755102 8.75351290285468
17.5510204081633 8.58032744536088
17.7551020408163 8.39094460685536
17.7789887523357 8.36734693877551
17.9591836734694 8.20794045389359
18.1632653061224 8.00705787139469
18.1892594995284 7.97959183673469
18.3673469387755 7.81026404387503
18.5714285714286 7.59186271852056
18.5714513877982 7.59183673469388
18.7755102040816 7.3818901948749
18.9297351871071 7.20408163265306
18.9795918367347 7.15224216671092
19.1836734693878 6.92273140794419
19.2703542802448 6.81632653061224
19.3877551020408 6.68636962068497
19.5918367346939 6.43622879188209
19.5977828191065 6.42857142857143
19.7959183673469 6.19918772907814
19.919916221049 6.04081632653061
20 5.94961917011027
};
\addplot [draw=none, fill=white!85.0980392156863!black]
table{%
x  y
19.7959183673469 4.71868053315991
20 4.52305634360283
20 4.87755102040816
20 5.26530612244898
20 5.6530612244898
20 5.94961917011027
19.919916221049 6.04081632653061
19.7959183673469 6.19918772907814
19.5977828191065 6.42857142857143
19.5918367346939 6.43622879188209
19.3877551020408 6.68636962068497
19.2703542802448 6.81632653061224
19.1836734693878 6.92273140794419
18.9795918367347 7.15224216671092
18.9297351871071 7.20408163265306
18.7755102040816 7.3818901948749
18.5714513877982 7.59183673469388
18.5714285714286 7.59186271852056
18.3673469387755 7.81026404387503
18.1892594995284 7.97959183673469
18.1632653061224 8.00705787139469
17.9591836734694 8.20794045389359
17.7789887523357 8.36734693877551
17.7551020408163 8.39094460685536
17.5510204081633 8.58032744536088
17.3469387755102 8.75351290285468
17.3449819581728 8.75510204081633
17.1428571428571 8.93980326854362
16.9387755102041 9.11574824918068
16.9066605482219 9.14285714285714
16.734693877551 9.30618051894902
16.530612244898 9.49563253721587
16.493084692059 9.53061224489796
16.3265306122449 9.70448842766539
16.1224489795918 9.9174459538873
16.1215859936204 9.91836734693877
15.9183673469388 10.1592241331448
15.7958822709599 10.3061224489796
15.7142857142857 10.4137937372529
15.5102040816327 10.6868919168228
15.5051040892636 10.6938775510204
15.3061224489796 10.9900074314211
15.2453160792652 11.0816326530612
15.1020408163265 11.3140166576465
15.0073049708067 11.469387755102
14.8979591836735 11.6607275705413
14.7869151555279 11.8571428571429
14.6938775510204 12.0313223212227
14.5811888605716 12.2448979591837
14.4897959183673 12.4269202898339
14.3881831473694 12.6326530612245
14.2857142857143 12.8492904155318
14.2066994954626 13.0204081632653
14.0816326530612 13.301647896044
14.035928744317 13.4081632653061
13.8775510204082 13.7894914720041
13.8750167165084 13.7959183673469
13.7280459505846 14.1836734693878
13.6734693877551 14.3313504942011
13.5910919182961 14.5714285714286
13.469387755102 14.9333480046091
13.4615124300259 14.9591836734694
13.3473856272882 15.3469387755102
13.265306122449 15.6299199048811
13.2385092332426 15.734693877551
13.1421009442336 16.1224489795918
13.0612244897959 16.4516496347277
13.0488754455018 16.5102040816327
12.9695595246306 16.8979591836735
12.8909788477584 17.2857142857143
12.8571428571429 17.4562751691435
12.8210777444238 17.6734693877551
12.7579589512152 18.0612244897959
12.6954425390225 18.4489795918367
12.6530612244898 18.7159034186207
12.637314785981 18.8367346938775
12.5879832615713 19.2244897959184
12.5391353811429 19.6122448979592
12.4907310278751 20
12.4489795918367 20
12.2448979591837 20
12.0408163265306 20
11.8367346938776 20
11.6326530612245 20
11.5960403887226 20
11.6207456166142 19.6122448979592
11.6326530612245 19.4292112395998
11.6485842923038 19.2244897959184
11.6793201284559 18.8367346938775
11.7104484072078 18.4489795918367
11.7420028765389 18.0612244897959
11.7740205553248 17.6734693877551
11.8065421396305 17.2857142857143
11.8367346938776 16.9321394275711
11.8402530858987 16.8979591836735
11.8813356913297 16.5102040816327
11.9230603318401 16.1224489795918
11.9654895690159 15.734693877551
12.0086930336311 15.3469387755102
12.0408163265306 15.0654410548818
12.0554775438974 14.9591836734694
12.1106328859201 14.5714285714286
12.1668445878962 14.1836734693878
12.224229920835 13.7959183673469
12.2448979591837 13.6609611193902
12.2915958913979 13.4081632653061
12.3652363297572 13.0204081632653
12.4405787705253 12.6326530612245
12.4489795918367 12.5912326286038
12.5330169360253 12.2448979591837
12.6296074925636 11.8571428571429
12.6530612244898 11.76697630749
12.744417845967 11.469387755102
12.8571428571429 11.1143012640845
12.8691645136561 11.0816326530612
13.0190316018995 10.6938775510204
13.0612244897959 10.5898410002976
13.1914912858871 10.3061224489796
13.265306122449 10.1538765140844
13.3915796600568 9.91836734693877
13.469387755102 9.78207250374072
13.6246566581154 9.53061224489796
13.6734693877551 9.45699075209691
13.8775510204082 9.16755112611718
13.8958525423425 9.14285714285714
14.0816326530612 8.91225946475336
14.2128074847646 8.75510204081633
14.2857142857143 8.67550907810181
14.4897959183673 8.46078396332103
14.581534801275 8.36734693877551
14.6938775510204 8.26450250994137
14.8979591836735 8.08425450029972
15.0207696358992 7.97959183673469
15.1020408163265 7.9181693571992
15.3061224489796 7.77028139403253
15.5102040816327 7.62799786795383
15.5640974535322 7.59183673469388
15.7142857142857 7.5036674491622
15.9183673469388 7.38781101713439
16.1224489795918 7.27441199798137
16.2499705199201 7.20408163265306
16.3265306122449 7.16733318415355
16.530612244898 7.06853708957154
16.734693877551 6.96641571505953
16.9387755102041 6.858464633543
17.0133829605142 6.81632653061224
17.1428571428571 6.75233260622999
17.3469387755102 6.64256909163921
17.5510204081633 6.52083031151153
17.6903443781482 6.42857142857143
17.7551020408163 6.39077875961495
17.9591836734694 6.25956241950153
18.1632653061224 6.11177320070235
18.2520573790794 6.04081632653061
18.3673469387755 5.95933506640301
18.5714285714286 5.80004190348084
18.7402173246648 5.6530612244898
18.7755102040816 5.62594357026979
18.9795918367347 5.45722498467804
19.1836734693878 5.26848600135721
19.1869213574459 5.26530612244898
19.3877551020408 5.09330701176316
19.5918367346939 4.89891337008285
19.6130718193122 4.87755102040816
19.7959183673469 4.71868053315991
};
\addplot [draw=none, fill=white!74.9019607843137!black]
table{%
x  y
19.5918367346939 4.05873873457414
19.7959183673469 3.91127305054635
20 3.75222831223723
20 4.10204081632653
20 4.48979591836735
20 4.52305634360283
19.7959183673469 4.71868053315991
19.6130718193122 4.87755102040816
19.5918367346939 4.89891337008285
19.3877551020408 5.09330701176316
19.1869213574459 5.26530612244898
19.1836734693878 5.26848600135721
18.9795918367347 5.45722498467804
18.7755102040816 5.62594357026979
18.7402173246648 5.6530612244898
18.5714285714286 5.80004190348084
18.3673469387755 5.95933506640301
18.2520573790794 6.04081632653061
18.1632653061224 6.11177320070235
17.9591836734694 6.25956241950153
17.7551020408163 6.39077875961495
17.6903443781482 6.42857142857143
17.5510204081633 6.52083031151153
17.3469387755102 6.64256909163921
17.1428571428571 6.75233260622999
17.0133829605142 6.81632653061224
16.9387755102041 6.858464633543
16.734693877551 6.96641571505953
16.530612244898 7.06853708957154
16.3265306122449 7.16733318415355
16.2499705199201 7.20408163265306
16.1224489795918 7.27441199798137
15.9183673469388 7.38781101713439
15.7142857142857 7.5036674491622
15.5640974535322 7.59183673469388
15.5102040816327 7.62799786795383
15.3061224489796 7.77028139403253
15.1020408163265 7.9181693571992
15.0207696358992 7.97959183673469
14.8979591836735 8.08425450029972
14.6938775510204 8.26450250994137
14.581534801275 8.36734693877551
14.4897959183673 8.46078396332103
14.2857142857143 8.67550907810181
14.2128074847646 8.75510204081633
14.0816326530612 8.91225946475336
13.8958525423425 9.14285714285714
13.8775510204082 9.16755112611718
13.6734693877551 9.45699075209691
13.6246566581154 9.53061224489796
13.469387755102 9.78207250374072
13.3915796600568 9.91836734693877
13.265306122449 10.1538765140844
13.1914912858871 10.3061224489796
13.0612244897959 10.5898410002976
13.0190316018995 10.6938775510204
12.8691645136561 11.0816326530612
12.8571428571429 11.1143012640845
12.744417845967 11.469387755102
12.6530612244898 11.76697630749
12.6296074925636 11.8571428571429
12.5330169360253 12.2448979591837
12.4489795918367 12.5912326286038
12.4405787705253 12.6326530612245
12.3652363297572 13.0204081632653
12.2915958913979 13.4081632653061
12.2448979591837 13.6609611193902
12.224229920835 13.7959183673469
12.1668445878962 14.1836734693878
12.1106328859201 14.5714285714286
12.0554775438974 14.9591836734694
12.0408163265306 15.0654410548818
12.0086930336311 15.3469387755102
11.9654895690159 15.734693877551
11.9230603318401 16.1224489795918
11.8813356913297 16.5102040816327
11.8402530858987 16.8979591836735
11.8367346938776 16.9321394275711
11.8065421396305 17.2857142857143
11.7740205553248 17.6734693877551
11.7420028765389 18.0612244897959
11.7104484072078 18.4489795918367
11.6793201284559 18.8367346938775
11.6485842923038 19.2244897959184
11.6326530612245 19.4292112395998
11.6207456166142 19.6122448979592
11.5960403887226 20
11.4285714285714 20
11.2244897959184 20
11.1217013905289 20
11.1441260445635 19.6122448979592
11.1669416502195 19.2244897959184
11.1901780063728 18.8367346938775
11.2138677714645 18.4489795918367
11.2244897959184 18.2793501259811
11.2398322659026 18.0612244897959
11.2676828401064 17.6734693877551
11.2960296240227 17.2857142857143
11.3249159201347 16.8979591836735
11.354389691589 16.5102040816327
11.3845042145465 16.1224489795918
11.4153188461708 15.734693877551
11.4285714285714 15.5728330068048
11.4496751083643 15.3469387755102
11.4868437398692 14.9591836734694
11.5248550244696 14.5714285714286
11.5637969524336 14.1836734693878
11.6037691365247 13.7959183673469
11.6326530612245 13.5243956112405
11.6469802349014 13.4081632653061
11.6964459404702 13.0204081632653
11.7472681621904 12.6326530612245
11.7996127360904 12.2448979591837
11.8367346938776 11.9796720690998
11.8568469142095 11.8571428571429
11.9230895823565 11.469387755102
11.9915247463556 11.0816326530612
12.0408163265306 10.8133754674281
12.0667699261899 10.6938775510204
12.1549420266596 10.3061224489796
12.2448979591837 9.92551663418676
12.2469008740476 9.91836734693877
12.3617607790627 9.53061224489796
12.4489795918367 9.24986614534843
12.4882133383113 9.14285714285714
12.6389033402864 8.75510204081633
12.6530612244898 8.72101294284461
12.8246505174135 8.36734693877551
12.8571428571429 8.30499853741685
13.0524476699499 7.97959183673469
13.0612244897959 7.96610982631419
13.265306122449 7.69041269448451
13.34637498199 7.59183673469388
13.469387755102 7.45582220841569
13.6734693877551 7.24985457247521
13.7225176704985 7.20408163265306
13.8775510204082 7.07425806970845
14.0816326530612 6.91492314625782
14.2160743192669 6.81632653061224
14.2857142857143 6.77105418955386
14.4897959183673 6.64644017451507
14.6938775510204 6.52932260358897
14.8804602951714 6.42857142857143
14.8979591836735 6.42029639478989
15.1020408163265 6.3293268081896
15.3061224489796 6.24375341556674
15.5102040816327 6.16273967450252
15.7142857142857 6.08521554396524
15.834380369818 6.04081632653061
15.9183673469388 6.01372515380036
16.1224489795918 5.94838937185498
16.3265306122449 5.88211449214058
16.530612244898 5.81316284900549
16.734693877551 5.7396782970104
16.9387755102041 5.65971906956781
16.9542945279182 5.6530612244898
17.1428571428571 5.58158326712143
17.3469387755102 5.49553185359237
17.5510204081633 5.3991379912167
17.7551020408163 5.29079064772744
17.7984932920065 5.26530612244898
17.9591836734694 5.18154901073038
18.1632653061224 5.06350897187598
18.3673469387755 4.93214444667942
18.4449313361795 4.87755102040816
18.5714285714286 4.79895630162889
18.7755102040816 4.66113494604127
18.9795918367347 4.5098800516809
19.0049783857605 4.48979591836735
19.1836734693878 4.36669172901212
19.3877551020408 4.21499744804459
19.5288339841745 4.10204081632653
19.5918367346939 4.05873873457414
};
\addplot [draw=none, fill=white!65.0980392156863!black]
table{%
x  y
20 3.21328859450824
20 3.3265306122449
20 3.71428571428571
20 3.75222831223723
19.7959183673469 3.91127305054635
19.5918367346939 4.05873873457414
19.5288339841745 4.10204081632653
19.3877551020408 4.21499744804459
19.1836734693878 4.36669172901212
19.0049783857605 4.48979591836735
18.9795918367347 4.5098800516809
18.7755102040816 4.66113494604127
18.5714285714286 4.79895630162889
18.4449313361795 4.87755102040816
18.3673469387755 4.93214444667942
18.1632653061224 5.06350897187598
17.9591836734694 5.18154901073038
17.7984932920065 5.26530612244898
17.7551020408163 5.29079064772744
17.5510204081633 5.3991379912167
17.3469387755102 5.49553185359237
17.1428571428571 5.58158326712143
16.9542945279182 5.6530612244898
16.9387755102041 5.65971906956781
16.734693877551 5.7396782970104
16.530612244898 5.81316284900549
16.3265306122449 5.88211449214058
16.1224489795918 5.94838937185498
15.9183673469388 6.01372515380036
15.834380369818 6.04081632653061
15.7142857142857 6.08521554396524
15.5102040816327 6.16273967450252
15.3061224489796 6.24375341556674
15.1020408163265 6.3293268081896
14.8979591836735 6.42029639478989
14.8804602951714 6.42857142857143
14.6938775510204 6.52932260358897
14.4897959183673 6.64644017451507
14.2857142857143 6.77105418955386
14.2160743192669 6.81632653061224
14.0816326530612 6.91492314625782
13.8775510204082 7.07425806970845
13.7225176704985 7.20408163265306
13.6734693877551 7.24985457247521
13.469387755102 7.45582220841569
13.34637498199 7.59183673469388
13.265306122449 7.69041269448451
13.0612244897959 7.96610982631419
13.0524476699499 7.97959183673469
12.8571428571429 8.30499853741685
12.8246505174135 8.36734693877551
12.6530612244898 8.72101294284461
12.6389033402864 8.75510204081633
12.4882133383113 9.14285714285714
12.4489795918367 9.24986614534843
12.3617607790627 9.53061224489796
12.2469008740476 9.91836734693877
12.2448979591837 9.92551663418676
12.1549420266596 10.3061224489796
12.0667699261899 10.6938775510204
12.0408163265306 10.8133754674281
11.9915247463556 11.0816326530612
11.9230895823565 11.469387755102
11.8568469142095 11.8571428571429
11.8367346938776 11.9796720690998
11.7996127360904 12.2448979591837
11.7472681621904 12.6326530612245
11.6964459404702 13.0204081632653
11.6469802349014 13.4081632653061
11.6326530612245 13.5243956112405
11.6037691365247 13.7959183673469
11.5637969524336 14.1836734693878
11.5248550244696 14.5714285714286
11.4868437398692 14.9591836734694
11.4496751083643 15.3469387755102
11.4285714285714 15.5728330068048
11.4153188461708 15.734693877551
11.3845042145465 16.1224489795918
11.354389691589 16.5102040816327
11.3249159201347 16.8979591836735
11.2960296240227 17.2857142857143
11.2676828401064 17.6734693877551
11.2398322659026 18.0612244897959
11.2244897959184 18.2793501259811
11.2138677714645 18.4489795918367
11.1901780063728 18.8367346938775
11.1669416502195 19.2244897959184
11.1441260445635 19.6122448979592
11.1217013905289 20
11.0204081632653 20
10.8163265306122 20
10.7487932867605 20
10.7701190255175 19.6122448979592
10.7919438718488 19.2244897959184
10.8143042184607 18.8367346938775
10.8163265306122 18.8026080215126
10.8381344291413 18.4489795918367
10.8625604199616 18.0612244897959
10.8875298489843 17.6734693877551
10.9130871598916 17.2857142857143
10.9392815701243 16.8979591836735
10.9661677389978 16.5102040816327
10.9938065529088 16.1224489795918
11.0204081632653 15.7600793580644
11.0223911135728 15.734693877551
11.0536123624906 15.3469387755102
11.0856603253514 14.9591836734694
11.118617426874 14.5714285714286
11.1525769263504 14.1836734693878
11.1876447969807 13.7959183673469
11.2239420163003 13.4081632653061
11.2244897959184 13.4025427780634
11.2648373941494 13.0204081632653
11.3071181344636 12.6326530612245
11.3508966766614 12.2448979591837
11.3963619188778 11.8571428571429
11.4285714285714 11.5938943424946
11.4454617108039 11.469387755102
11.5003458866896 11.0816326530612
11.5574860701613 10.6938775510204
11.6172190804593 10.3061224489796
11.6326530612245 10.2110760696795
11.6863439645851 9.91836734693877
11.7609485906266 9.53061224489796
11.8367346938776 9.15636297441804
11.8398847568095 9.14285714285714
11.9358355380362 8.75510204081633
12.0372752268884 8.36734693877551
12.0408163265306 8.35470623149242
12.1632460868839 7.97959183673469
12.2448979591837 7.74575368542463
12.3080768578244 7.59183673469388
12.4489795918367 7.27355256602685
12.4851961850194 7.20408163265306
12.6530612244898 6.90817374578042
12.7141672217163 6.81632653061224
12.8571428571429 6.6206183736128
13.0204164722098 6.42857142857143
13.0612244897959 6.38522880084809
13.265306122449 6.19709125714901
13.4577577971089 6.04081632653061
13.469387755102 6.03237867070056
13.6734693877551 5.90008835825321
13.8775510204082 5.78093496816569
14.0816326530612 5.6725249562792
14.1213994257528 5.6530612244898
14.2857142857143 5.58206582481642
14.4897959183673 5.50106776849604
14.6938775510204 5.42652570035568
14.8979591836735 5.35767484915405
15.1020408163265 5.29374725563746
15.1986036415983 5.26530612244898
15.3061224489796 5.23731627137465
15.5102040816327 5.18675253994918
15.7142857142857 5.13792760288392
15.9183673469388 5.0896749146152
16.1224489795918 5.04071033897515
16.3265306122449 4.98965591530243
16.530612244898 4.93506996433832
16.7276111761181 4.87755102040816
16.734693877551 4.87568663497567
16.9387755102041 4.81636881884756
17.1428571428571 4.75029149748933
17.3469387755102 4.67639104733323
17.5510204081633 4.59368110122166
17.7551020408163 4.50122847398961
17.7779413431618 4.48979591836735
17.9591836734694 4.40812205666094
18.1632653061224 4.30730444297498
18.3673469387755 4.19708598829017
18.5289764261738 4.10204081632653
18.5714285714286 4.07977199678191
18.7755102040816 3.96549976945009
18.9795918367347 3.84343254565848
19.1812573086671 3.71428571428571
19.1836734693878 3.71292837124108
19.3877551020408 3.59279225978004
19.5918367346939 3.46659795914052
19.7959183673469 3.33306037758089
19.8054317732866 3.3265306122449
20 3.21328859450824
};
\addplot [draw=none, fill=white!54.9019607843137!black]
table{%
x  y
19.7959183673469 2.87870370430333
20 2.77710858973377
20 2.93877551020408
20 3.21328859450824
19.8054317732866 3.3265306122449
19.7959183673469 3.33306037758089
19.5918367346939 3.46659795914052
19.3877551020408 3.59279225978004
19.1836734693878 3.71292837124108
19.1812573086671 3.71428571428571
18.9795918367347 3.84343254565848
18.7755102040816 3.96549976945009
18.5714285714286 4.07977199678191
18.5289764261738 4.10204081632653
18.3673469387755 4.19708598829017
18.1632653061224 4.30730444297498
17.9591836734694 4.40812205666094
17.7779413431618 4.48979591836735
17.7551020408163 4.50122847398961
17.5510204081633 4.59368110122166
17.3469387755102 4.67639104733323
17.1428571428571 4.75029149748933
16.9387755102041 4.81636881884756
16.734693877551 4.87568663497568
16.7276111761181 4.87755102040816
16.530612244898 4.93506996433832
16.3265306122449 4.98965591530243
16.1224489795918 5.04071033897515
15.9183673469388 5.0896749146152
15.7142857142857 5.13792760288392
15.5102040816327 5.18675253994918
15.3061224489796 5.23731627137465
15.1986036415983 5.26530612244898
15.1020408163265 5.29374725563746
14.8979591836735 5.35767484915405
14.6938775510204 5.42652570035568
14.4897959183673 5.50106776849604
14.2857142857143 5.58206582481642
14.1213994257528 5.6530612244898
14.0816326530612 5.6725249562792
13.8775510204082 5.78093496816569
13.6734693877551 5.90008835825321
13.469387755102 6.03237867070056
13.4577577971089 6.04081632653061
13.265306122449 6.19709125714901
13.0612244897959 6.38522880084809
13.0204164722098 6.42857142857143
12.8571428571429 6.6206183736128
12.7141672217163 6.81632653061225
12.6530612244898 6.90817374578042
12.4851961850194 7.20408163265306
12.4489795918367 7.27355256602685
12.3080768578244 7.59183673469388
12.2448979591837 7.74575368542463
12.1632460868839 7.97959183673469
12.0408163265306 8.35470623149242
12.0372752268884 8.36734693877551
11.9358355380362 8.75510204081633
11.8398847568095 9.14285714285714
11.8367346938776 9.15636297441804
11.7609485906266 9.53061224489796
11.6863439645851 9.91836734693878
11.6326530612245 10.2110760696795
11.6172190804593 10.3061224489796
11.5574860701613 10.6938775510204
11.5003458866896 11.0816326530612
11.4454617108039 11.469387755102
11.4285714285714 11.5938943424946
11.3963619188778 11.8571428571429
11.3508966766614 12.2448979591837
11.3071181344636 12.6326530612245
11.2648373941494 13.0204081632653
11.2244897959184 13.4025427780634
11.2239420163003 13.4081632653061
11.1876447969807 13.7959183673469
11.1525769263504 14.1836734693878
11.118617426874 14.5714285714286
11.0856603253514 14.9591836734694
11.0536123624906 15.3469387755102
11.0223911135728 15.734693877551
11.0204081632653 15.7600793580644
10.9938065529088 16.1224489795918
10.9661677389978 16.5102040816327
10.9392815701243 16.8979591836735
10.9130871598916 17.2857142857143
10.8875298489843 17.6734693877551
10.8625604199616 18.0612244897959
10.8381344291413 18.4489795918367
10.8163265306122 18.8026080215126
10.8143042184607 18.8367346938775
10.7919438718488 19.2244897959184
10.7701190255175 19.6122448979592
10.7487932867605 20
10.6122448979592 20
10.4081632653061 20
10.4017276687149 20
10.4081632653061 19.8888519321782
10.4235621561315 19.6122448979592
10.4456847843375 19.2244897959184
10.4683824765629 18.8367346938775
10.491697545169 18.4489795918367
10.5156765628046 18.0612244897959
10.5403709225014 17.6734693877551
10.5658374893934 17.2857142857143
10.5921393620211 16.8979591836735
10.6122448979592 16.6112093973468
10.6192056674157 16.5102040816327
10.6467216879858 16.1224489795918
10.6750927352312 15.734693877551
10.704397573542 15.3469387755102
10.7347248959538 14.9591836734694
10.7661749652802 14.5714285714286
10.7988615932722 14.1836734693878
10.8163265306122 13.9844869001351
10.8329324688249 13.7959183673469
10.8683268588051 13.4081632653061
10.9050957847949 13.0204081632653
10.9433996291925 12.6326530612245
10.9834245901423 12.2448979591837
11.0204081632653 11.9030212859654
11.0255202469727 11.8571428571429
11.0706104303375 11.469387755102
11.1177924654598 11.0816326530612
11.1673657871328 10.6938775510204
11.2196896694133 10.3061224489796
11.2244897959184 10.27242862823
11.2776298395552 9.91836734693877
11.3390455896285 9.53061224489796
11.4042422168644 9.14285714285714
11.4285714285714 9.00694832608354
11.4773820836521 8.75510204081633
11.5573656735563 8.36734693877551
11.6326530612245 8.0270120216908
11.6443265194741 7.97959183673469
11.7465608555055 7.59183673469388
11.8367346938775 7.27539295884048
11.8598849709167 7.20408163265306
11.9961240520357 6.81632653061225
12.0408163265306 6.69967615961227
12.1615624709064 6.42857142857143
12.2448979591837 6.25795737150603
12.3700984024819 6.04081632653061
12.4489795918367 5.91677200463388
12.6489541932144 5.6530612244898
12.6530612244898 5.64817540426037
12.8571428571429 5.44378565938127
13.0612244897959 5.27215135462167
13.0706970624518 5.26530612244898
13.265306122449 5.13936062397437
13.469387755102 5.02576056695536
13.6734693877551 4.92660839530261
13.787124262289 4.87755102040816
13.8775510204082 4.84260661172612
14.0816326530612 4.77185111015717
14.2857142857143 4.70824639438003
14.4897959183673 4.65067044855302
14.6938775510204 4.59821469282327
14.8979591836735 4.55006396184764
15.1020408163265 4.50541103834331
15.1775091138006 4.48979591836735
15.3061224489796 4.46555816005504
15.5102040816327 4.42851038605369
15.7142857142857 4.39205699143511
15.9183673469388 4.35520691030081
16.1224489795918 4.31691879662333
16.3265306122449 4.27612554696131
16.530612244898 4.2317634045082
16.734693877551 4.18280176321038
16.9387755102041 4.12826939139834
17.0265164944315 4.10204081632653
17.1428571428571 4.06970614588408
17.3469387755102 4.00666629625814
17.5510204081633 3.93699287631038
17.7551020408163 3.86036646952325
17.9591836734694 3.77652285472397
18.0981354892808 3.71428571428571
18.1632653061224 3.68751257687394
18.3673469387755 3.59742403630844
18.5714285714286 3.50184735870717
18.7755102040816 3.40067180604503
18.9165453244745 3.3265306122449
18.9795918367347 3.29672948424762
19.1836734693878 3.19587308714717
19.3877551020408 3.09165154891594
19.5918367346939 2.98353899764154
19.6725198314078 2.93877551020408
19.7959183673469 2.87870370430333
};
\addplot [draw=none, fill=white!45.0980392156863!black]
table{%
x  y
19.7959183673469 2.47657890240616
20 2.39171128238466
20 2.55102040816327
20 2.77710858973377
19.7959183673469 2.87870370430333
19.6725198314078 2.93877551020408
19.5918367346939 2.98353899764154
19.3877551020408 3.09165154891594
19.1836734693878 3.19587308714717
18.9795918367347 3.29672948424762
18.9165453244745 3.3265306122449
18.7755102040816 3.40067180604503
18.5714285714286 3.50184735870717
18.3673469387755 3.59742403630844
18.1632653061224 3.68751257687394
18.0981354892808 3.71428571428571
17.9591836734694 3.77652285472397
17.7551020408163 3.86036646952325
17.5510204081633 3.93699287631038
17.3469387755102 4.00666629625814
17.1428571428571 4.06970614588408
17.0265164944315 4.10204081632653
16.9387755102041 4.12826939139834
16.734693877551 4.18280176321038
16.530612244898 4.2317634045082
16.3265306122449 4.27612554696131
16.1224489795918 4.31691879662333
15.9183673469388 4.35520691030081
15.7142857142857 4.39205699143511
15.5102040816327 4.42851038605369
15.3061224489796 4.46555816005504
15.1775091138006 4.48979591836735
15.1020408163265 4.50541103834331
14.8979591836735 4.55006396184764
14.6938775510204 4.59821469282327
14.4897959183673 4.65067044855302
14.2857142857143 4.70824639438003
14.0816326530612 4.77185111015717
13.8775510204082 4.84260661172612
13.787124262289 4.87755102040816
13.6734693877551 4.92660839530261
13.469387755102 5.02576056695536
13.265306122449 5.13936062397437
13.0706970624518 5.26530612244898
13.0612244897959 5.27215135462167
12.8571428571429 5.44378565938127
12.6530612244898 5.64817540426037
12.6489541932144 5.6530612244898
12.4489795918367 5.91677200463388
12.370098402482 6.04081632653061
12.2448979591837 6.25795737150603
12.1615624709064 6.42857142857143
12.0408163265306 6.69967615961227
11.9961240520357 6.81632653061224
11.8598849709167 7.20408163265306
11.8367346938776 7.27539295884048
11.7465608555055 7.59183673469388
11.6443265194741 7.97959183673469
11.6326530612245 8.0270120216908
11.5573656735563 8.36734693877551
11.4773820836521 8.75510204081633
11.4285714285714 9.00694832608354
11.4042422168644 9.14285714285714
11.3390455896285 9.53061224489796
11.2776298395552 9.91836734693877
11.2244897959184 10.27242862823
11.2196896694133 10.3061224489796
11.1673657871328 10.6938775510204
11.1177924654598 11.0816326530612
11.0706104303375 11.469387755102
11.0255202469727 11.8571428571429
11.0204081632653 11.9030212859654
10.9834245901423 12.2448979591837
10.9433996291925 12.6326530612245
10.9050957847949 13.0204081632653
10.8683268588051 13.4081632653061
10.8329324688249 13.7959183673469
10.8163265306122 13.9844869001351
10.7988615932722 14.1836734693878
10.7661749652802 14.5714285714286
10.7347248959538 14.9591836734694
10.704397573542 15.3469387755102
10.6750927352312 15.734693877551
10.6467216879858 16.1224489795918
10.6192056674157 16.5102040816327
10.6122448979592 16.6112093973468
10.5921393620211 16.8979591836735
10.5658374893934 17.2857142857143
10.5403709225014 17.6734693877551
10.5156765628046 18.0612244897959
10.491697545169 18.4489795918367
10.4683824765629 18.8367346938775
10.4456847843375 19.2244897959184
10.4235621561315 19.6122448979592
10.4081632653061 19.8888519321782
10.4017276687149 20
10.2040816326531 20
10.0278644928357 20
10.0508573677681 19.6122448979592
10.0745834817743 19.2244897959184
10.0990945715675 18.8367346938775
10.1244475028639 18.4489795918367
10.1507049369924 18.0612244897959
10.1779361049924 17.6734693877551
10.2040816326531 17.3147938015483
10.2060113735116 17.2857142857143
10.2325079976407 16.8979591836735
10.259960316968 16.5102040816327
10.2884501637897 16.1224489795918
10.3180692088783 15.734693877551
10.3489205109565 15.3469387755102
10.3811203683814 14.9591836734694
10.4081632653061 14.6472847420177
10.4142779416073 14.5714285714286
10.4466700690829 14.1836734693878
10.4804717581648 13.7959183673469
10.5158351882091 13.4081632653061
10.5529356190117 13.0204081632653
10.5919759720887 12.6326530612245
10.6122448979592 12.4408719437573
10.6319306457536 12.2448979591837
10.6726931067526 11.8571428571429
10.7156554674372 11.469387755102
10.761120219806 11.0816326530612
10.8094484369397 10.6938775510204
10.8163265306122 10.6415633928859
10.8592617948536 10.3061224489796
10.9119181010173 9.91836734693877
10.9681736739381 9.53061224489796
11.0204081632653 9.19505787794204
11.02856487755 9.14285714285714
11.0928199317972 8.75510204081633
11.1620601602675 8.36734693877551
11.2244897959184 8.04473034283438
11.2375328709475 7.97959183673469
11.3207445656215 7.59183673469388
11.4118311254733 7.20408163265306
11.4285714285714 7.13839141026794
11.5161718871287 6.81632653061224
11.6326530612245 6.42877246834761
11.6327197971574 6.42857142857143
11.7731927166573 6.04081632653061
11.8367346938776 5.88288948693898
11.9422391802814 5.6530612244898
12.0408163265306 5.4597065672945
12.1571775211389 5.26530612244898
12.2448979591837 5.13338005782106
12.4489795918367 4.87838786023776
12.4497812637558 4.87755102040816
12.6530612244898 4.6861387195246
12.8571428571429 4.52937854398816
12.9187365914624 4.48979591836735
13.0612244897959 4.40670678717452
13.265306122449 4.30692499546387
13.469387755102 4.22237886449549
13.6734693877551 4.14958134380001
13.8255788408572 4.10204081632653
13.8775510204082 4.08705782689724
14.0816326530612 4.03455698025942
14.2857142857143 3.98766173670033
14.4897959183673 3.94535440394387
14.6938775510204 3.90679934484754
14.8979591836735 3.87124922897919
15.1020408163265 3.83798124107241
15.3061224489796 3.80625765928749
15.5102040816327 3.7753060258074
15.7142857142857 3.74431475193718
15.906504378392 3.71428571428571
15.9183673469388 3.71250104792304
16.1224489795918 3.68000983872794
16.3265306122449 3.64509591540798
16.530612244898 3.60704603788272
16.734693877551 3.56522135509478
16.9387755102041 3.51907821559613
17.1428571428571 3.46818080465088
17.3469387755102 3.41220284570035
17.5510204081633 3.35091671845606
17.6250481936731 3.3265306122449
17.7551020408163 3.28575353687818
17.9591836734694 3.21700459762126
18.1632653061224 3.14411263972955
18.3673469387755 3.06725487912549
18.5714285714286 2.98655037064066
18.6857425442613 2.93877551020408
18.7755102040816 2.90379508633504
18.9795918367347 2.82102971524864
19.1836734693878 2.73656460700578
19.3877551020408 2.65036458347491
19.5918367346939 2.56220698002571
19.616529892927 2.55102040816327
19.7959183673469 2.47657890240616
};
\addplot [draw=none, fill=white!34.9019607843137!black]
table{%
x  y
19.7959183673469 2.09285846019102
20 2.02228357659525
20 2.16326530612245
20 2.39171128238466
19.7959183673469 2.47657890240616
19.616529892927 2.55102040816327
19.5918367346939 2.56220698002571
19.3877551020408 2.65036458347491
19.1836734693878 2.73656460700578
18.9795918367347 2.82102971524864
18.7755102040816 2.90379508633504
18.6857425442613 2.93877551020408
18.5714285714286 2.98655037064066
18.3673469387755 3.06725487912549
18.1632653061224 3.14411263972955
17.9591836734694 3.21700459762126
17.7551020408163 3.28575353687818
17.6250481936731 3.3265306122449
17.5510204081633 3.35091671845606
17.3469387755102 3.41220284570035
17.1428571428571 3.46818080465088
16.9387755102041 3.51907821559613
16.734693877551 3.56522135509478
16.530612244898 3.60704603788272
16.3265306122449 3.64509591540798
16.1224489795918 3.68000983872794
15.9183673469388 3.71250104792304
15.906504378392 3.71428571428571
15.7142857142857 3.74431475193718
15.5102040816327 3.7753060258074
15.3061224489796 3.80625765928749
15.1020408163265 3.83798124107241
14.8979591836735 3.87124922897919
14.6938775510204 3.90679934484754
14.4897959183673 3.94535440394387
14.2857142857143 3.98766173670033
14.0816326530612 4.03455698025942
13.8775510204082 4.08705782689724
13.8255788408572 4.10204081632653
13.6734693877551 4.14958134380001
13.469387755102 4.22237886449549
13.265306122449 4.30692499546387
13.0612244897959 4.40670678717452
12.9187365914624 4.48979591836735
12.8571428571429 4.52937854398816
12.6530612244898 4.6861387195246
12.4497812637558 4.87755102040816
12.4489795918367 4.87838786023776
12.2448979591837 5.13338005782106
12.1571775211389 5.26530612244898
12.0408163265306 5.4597065672945
11.9422391802814 5.6530612244898
11.8367346938776 5.88288948693898
11.7731927166573 6.04081632653061
11.6327197971574 6.42857142857143
11.6326530612245 6.42877246834761
11.5161718871287 6.81632653061224
11.4285714285714 7.13839141026794
11.4118311254733 7.20408163265306
11.3207445656215 7.59183673469388
11.2375328709475 7.97959183673469
11.2244897959184 8.04473034283438
11.1620601602675 8.36734693877551
11.0928199317972 8.75510204081633
11.02856487755 9.14285714285714
11.0204081632653 9.19505787794204
10.9681736739381 9.53061224489796
10.9119181010173 9.91836734693878
10.8592617948536 10.3061224489796
10.8163265306122 10.6415633928859
10.8094484369397 10.6938775510204
10.761120219806 11.0816326530612
10.7156554674372 11.469387755102
10.6726931067526 11.8571428571429
10.6319306457536 12.2448979591837
10.6122448979592 12.4408719437573
10.5919759720887 12.6326530612245
10.5529356190117 13.0204081632653
10.5158351882091 13.4081632653061
10.4804717581648 13.7959183673469
10.4466700690829 14.1836734693878
10.4142779416073 14.5714285714286
10.4081632653061 14.6472847420177
10.3811203683814 14.9591836734694
10.3489205109565 15.3469387755102
10.3180692088783 15.734693877551
10.2884501637897 16.1224489795918
10.259960316968 16.5102040816327
10.2325079976407 16.8979591836735
10.2060113735116 17.2857142857143
10.2040816326531 17.3147938015483
10.1779361049924 17.6734693877551
10.1507049369924 18.0612244897959
10.1244475028639 18.4489795918367
10.0990945715675 18.8367346938775
10.0745834817743 19.2244897959184
10.0508573677681 19.6122448979592
10.0278644928357 20
10 20
10 19.6122448979592
10 19.2244897959184
10 18.8367346938775
10 18.4489795918367
10 18.0612244897959
10 17.6734693877551
10 17.2857142857143
10 16.8979591836735
10 16.5102040816327
10 16.1224489795918
10 15.734693877551
10 15.3469387755102
10 14.9591836734694
10 14.6096648485981
10.0033240859298 14.5714285714286
10.0383098189352 14.1836734693878
10.0751127254938 13.7959183673469
10.113928881166 13.4081632653061
10.1549845065828 13.0204081632653
10.1985420046173 12.6326530612245
10.2040816326531 12.585540274901
10.2394727194536 12.2448979591837
10.2821159091442 11.8571428571429
10.3275121038003 11.469387755102
10.3760421401179 11.0816326530612
10.4081632653061 10.8406118810602
10.4258487312435 10.6938775510204
10.4752728463896 10.3061224489796
10.5284037050745 9.91836734693877
10.5858639457999 9.53061224489796
10.6122448979592 9.36444196158276
10.644854085328 9.14285714285714
10.7061197816042 8.75510204081633
10.7730627118402 8.36734693877551
10.8163265306122 8.1369576213312
10.8445386837173 7.97959183673469
10.919701868723 7.59183673469388
11.003220653772 7.20408163265306
11.0204081632653 7.13066669809755
11.0929977457643 6.81632653061224
11.1930379904998 6.42857142857143
11.2244897959184 6.31804710922859
11.3053493530242 6.04081632653061
11.4285714285714 5.66757568830476
11.4336840992513 5.6530612244898
11.5848538392112 5.26530612244898
11.6326530612245 5.15666626236101
11.7698829770514 4.87755102040816
11.8367346938776 4.75651218858069
12.0085962546684 4.48979591836735
12.0408163265306 4.44490136434018
12.2448979591837 4.20548490162514
12.3532205245103 4.10204081632653
12.4489795918367 4.01844630903559
12.6530612244898 3.87322646484094
12.8571428571429 3.75646808494245
12.9461396251673 3.71428571428571
13.0612244897959 3.66300652903807
13.265306122449 3.58692200756269
13.469387755102 3.52280336428212
13.6734693877551 3.46775117946124
13.8775510204082 3.41969068080302
14.0816326530612 3.37712405627512
14.2857142857143 3.33894240130135
14.3583762650979 3.3265306122449
14.4897959183673 3.30436987337468
14.6938775510204 3.27259363717563
14.8979591836735 3.24292655760065
15.1020408163265 3.21477753766987
15.3061224489796 3.18756428767317
15.5102040816327 3.16069999931846
15.7142857142857 3.13359241642459
15.9183673469388 3.10565199380621
16.1224489795918 3.07630611262437
16.3265306122449 3.04501648592111
16.530612244898 3.01129700164043
16.734693877551 2.97472938822728
16.9190605238997 2.93877551020408
16.9387755102041 2.93490662593167
17.1428571428571 2.89116540131113
17.3469387755102 2.84424100593345
17.5510204081633 2.7941621307885
17.7551020408163 2.74103058046116
17.9591836734694 2.68499943129358
18.1632653061224 2.62624685551558
18.3673469387755 2.56494724548345
18.4108940743282 2.55102040816327
18.5714285714286 2.50074963214399
18.7755102040816 2.43508207923047
18.9795918367347 2.36845443120196
19.1836734693878 2.30106210235391
19.3877551020408 2.23301171318503
19.5918367346939 2.1643011026872
19.5947604772428 2.16326530612245
19.7959183673469 2.09285846019102
};
\addplot [draw=none, fill=white!24.7058823529412!black]
table{%
x  y
19.5918367346939 1.74616492072823
19.7959183673469 1.68663799956759
20 1.62777633658404
20 1.77551020408163
20 2.02228357659525
19.7959183673469 2.09285846019102
19.5947604772428 2.16326530612245
19.5918367346939 2.1643011026872
19.3877551020408 2.23301171318503
19.1836734693878 2.30106210235391
18.9795918367347 2.36845443120196
18.7755102040816 2.43508207923047
18.5714285714286 2.50074963214399
18.4108940743282 2.55102040816327
18.3673469387755 2.56494724548345
18.1632653061224 2.62624685551558
17.9591836734694 2.68499943129358
17.7551020408163 2.74103058046116
17.5510204081633 2.7941621307885
17.3469387755102 2.84424100593345
17.1428571428571 2.89116540131113
16.9387755102041 2.93490662593167
16.9190605238997 2.93877551020408
16.734693877551 2.97472938822728
16.530612244898 3.01129700164043
16.3265306122449 3.04501648592111
16.1224489795918 3.07630611262437
15.9183673469388 3.10565199380621
15.7142857142857 3.13359241642459
15.5102040816327 3.16069999931846
15.3061224489796 3.18756428767317
15.1020408163265 3.21477753766987
14.8979591836735 3.24292655760065
14.6938775510204 3.27259363717563
14.4897959183673 3.30436987337468
14.3583762650979 3.3265306122449
14.2857142857143 3.33894240130135
14.0816326530612 3.37712405627512
13.8775510204082 3.41969068080302
13.6734693877551 3.46775117946124
13.469387755102 3.52280336428212
13.265306122449 3.58692200756269
13.0612244897959 3.66300652903807
12.9461396251673 3.71428571428571
12.8571428571429 3.75646808494245
12.6530612244898 3.87322646484094
12.4489795918367 4.01844630903559
12.3532205245103 4.10204081632653
12.2448979591837 4.20548490162514
12.0408163265306 4.44490136434018
12.0085962546684 4.48979591836735
11.8367346938776 4.75651218858069
11.7698829770514 4.87755102040816
11.6326530612245 5.15666626236101
11.5848538392112 5.26530612244898
11.4336840992513 5.6530612244898
11.4285714285714 5.66757568830476
11.3053493530242 6.04081632653061
11.2244897959184 6.31804710922859
11.1930379904998 6.42857142857143
11.0929977457643 6.81632653061224
11.0204081632653 7.13066669809755
11.003220653772 7.20408163265306
10.919701868723 7.59183673469388
10.8445386837173 7.97959183673469
10.8163265306122 8.13695762133119
10.7730627118402 8.36734693877551
10.7061197816042 8.75510204081633
10.644854085328 9.14285714285714
10.6122448979592 9.36444196158276
10.5858639457999 9.53061224489796
10.5284037050745 9.91836734693877
10.4752728463896 10.3061224489796
10.4258487312435 10.6938775510204
10.4081632653061 10.8406118810602
10.3760421401179 11.0816326530612
10.3275121038003 11.469387755102
10.2821159091442 11.8571428571429
10.2394727194536 12.2448979591837
10.2040816326531 12.585540274901
10.1985420046173 12.6326530612245
10.1549845065828 13.0204081632653
10.113928881166 13.4081632653061
10.0751127254938 13.7959183673469
10.0383098189352 14.1836734693878
10.0033240859298 14.5714285714286
10 14.6096648485981
10 14.5714285714286
10 14.1836734693878
10 13.7959183673469
10 13.4081632653061
10 13.0204081632653
10 12.6326530612245
10 12.2448979591837
10 11.8571428571429
10 11.469387755102
10 11.0816326530612
10 10.6938775510204
10 10.3061224489796
10 10.2204305993431
10.0439819954732 9.91836734693877
10.1050987464921 9.53061224489796
10.1721621578082 9.14285714285714
10.2040816326531 8.97188223701794
10.2392793344626 8.75510204081633
10.3076407313579 8.36734693877551
10.3837855397877 7.97959183673469
10.4081632653061 7.86512576216855
10.4600263594703 7.59183673469388
10.541679732402 7.20408163265306
10.6122448979592 6.90588184712139
10.631693143006 6.81632653061224
10.7239112524574 6.42857142857143
10.8163265306122 6.09046710319611
10.8292228591611 6.04081632653061
10.9402649874603 5.65306122448979
11.0204081632653 5.41028131789489
11.067853665668 5.26530612244898
11.2133954027057 4.87755102040816
11.2244897959184 4.85113620745519
11.3833812360902 4.48979591836735
11.4285714285714 4.39991530418389
11.5940300708806 4.10204081632653
11.6326530612245 4.0402896137891
11.8367346938776 3.75741014921761
11.8736840175833 3.71428571428571
12.0408163265306 3.53548317898354
12.2448979591837 3.36305613005908
12.2976436437371 3.3265306122449
12.4489795918367 3.22641289112019
12.6530612244898 3.11862901112
12.8571428571429 3.03243204962374
13.0612244897959 2.96200446659043
13.1405499337606 2.93877551020408
13.265306122449 2.90157783996562
13.469387755102 2.84928313180615
13.6734693877551 2.80385793829926
13.8775510204082 2.76364073819161
14.0816326530612 2.72745562314228
14.2857142857143 2.69445175952658
14.4897959183673 2.6639876649743
14.6938775510204 2.6355492319559
14.8979591836735 2.60869357889749
15.1020408163265 2.58301288465775
15.3061224489796 2.55811366723124
15.3649559655814 2.55102040816327
15.5102040816327 2.53218170107107
15.7142857142857 2.5058078028302
15.9183673469388 2.47920860910155
16.1224489795918 2.45202372489192
16.3265306122449 2.42392672061239
16.530612244898 2.39463499555841
16.734693877551 2.36391802366066
16.9387755102041 2.33160296177532
17.1428571428571 2.29757697587506
17.3469387755102 2.26178597910148
17.5510204081633 2.2242297737303
17.7551020408163 2.18495384138941
17.8616334644502 2.16326530612245
17.9591836734694 2.14229183075816
18.1632653061224 2.09618502742263
18.3673469387755 2.04885581945264
18.5714285714286 2.00051376004868
18.7755102040816 1.95137705526846
18.9795918367347 1.90165857447226
19.1836734693878 1.85155174047802
19.3877551020408 1.80121664562254
19.4891229284161 1.77551020408163
19.5918367346939 1.74616492072823
};
\addplot [draw=none, fill=white!14.9019607843137!black]
table{%
x  y
18.7755102040816 1.38346168247547
18.9795918367347 1.33824010011072
19.1836734693878 1.29168681143927
19.3877551020408 1.24399954919528
19.5918367346939 1.19541102126551
19.7959183673469 1.14618174373462
20 1.09659193909172
20 1.38775510204082
20 1.62777633658404
19.7959183673469 1.68663799956759
19.5918367346939 1.74616492072823
19.4891229284161 1.77551020408163
19.3877551020408 1.80121664562254
19.1836734693878 1.85155174047802
18.9795918367347 1.90165857447226
18.7755102040816 1.95137705526846
18.5714285714286 2.00051376004868
18.3673469387755 2.04885581945264
18.1632653061224 2.09618502742263
17.9591836734694 2.14229183075816
17.8616334644502 2.16326530612245
17.7551020408163 2.18495384138941
17.5510204081633 2.2242297737303
17.3469387755102 2.26178597910148
17.1428571428571 2.29757697587506
16.9387755102041 2.33160296177532
16.734693877551 2.36391802366066
16.530612244898 2.39463499555841
16.3265306122449 2.42392672061239
16.1224489795918 2.45202372489192
15.9183673469388 2.47920860910155
15.7142857142857 2.5058078028302
15.5102040816327 2.53218170107107
15.3649559655814 2.55102040816327
15.3061224489796 2.55811366723124
15.1020408163265 2.58301288465775
14.8979591836735 2.60869357889749
14.6938775510204 2.6355492319559
14.4897959183673 2.6639876649743
14.2857142857143 2.69445175952658
14.0816326530612 2.72745562314228
13.8775510204082 2.76364073819161
13.6734693877551 2.80385793829926
13.469387755102 2.84928313180615
13.265306122449 2.90157783996562
13.1405499337606 2.93877551020408
13.0612244897959 2.96200446659043
12.8571428571429 3.03243204962374
12.6530612244898 3.11862901112
12.4489795918367 3.22641289112019
12.2976436437371 3.3265306122449
12.2448979591837 3.36305613005908
12.0408163265306 3.53548317898354
11.8736840175833 3.71428571428571
11.8367346938776 3.75741014921761
11.6326530612245 4.0402896137891
11.5940300708806 4.10204081632653
11.4285714285714 4.39991530418389
11.3833812360902 4.48979591836735
11.2244897959184 4.85113620745519
11.2133954027057 4.87755102040816
11.067853665668 5.26530612244898
11.0204081632653 5.41028131789489
10.9402649874603 5.6530612244898
10.8292228591611 6.04081632653061
10.8163265306122 6.09046710319611
10.7239112524574 6.42857142857143
10.631693143006 6.81632653061224
10.6122448979592 6.90588184712139
10.541679732402 7.20408163265306
10.4600263594703 7.59183673469388
10.4081632653061 7.86512576216855
10.3837855397877 7.97959183673469
10.3076407313579 8.36734693877551
10.2392793344626 8.75510204081633
10.2040816326531 8.97188223701794
10.1721621578082 9.14285714285714
10.1050987464921 9.53061224489796
10.0439819954732 9.91836734693877
10 10.2204305993431
10 9.91836734693877
10 9.53061224489796
10 9.14285714285714
10 8.75510204081633
10 8.36734693877551
10 7.97959183673469
10 7.59183673469388
10 7.20408163265306
10 6.81632653061224
10 6.57373815332841
10.0374683945803 6.42857142857143
10.1495032771125 6.04081632653061
10.2040816326531 5.87358158625336
10.2676865928431 5.6530612244898
10.3978467297327 5.26530612244898
10.4081632653061 5.23745006553435
10.5302967111368 4.87755102040816
10.6122448979592 4.67158177807773
10.6809016929619 4.48979591836735
10.8163265306122 4.18258255725744
10.8515572332859 4.10204081632653
11.0204081632653 3.76784519960826
11.0487850183609 3.71428571428571
11.2244897959184 3.42197222466274
11.2889026362676 3.3265306122449
11.4285714285714 3.1388663380204
11.6103624012783 2.93877551020408
11.6326530612245 2.91553948851726
11.8367346938776 2.73087139979451
12.0408163265306 2.59116358569388
12.112704752937 2.55102040816327
12.2448979591837 2.47423495661483
12.4489795918367 2.3793278967992
12.6530612244898 2.30425573132625
12.8571428571429 2.24318746547248
13.0612244897959 2.19204719908888
13.1930935943274 2.16326530612245
13.265306122449 2.1449354741759
13.469387755102 2.0986284825703
13.6734693877551 2.05741143573433
13.8775510204082 2.02010640650179
14.0816326530612 1.98591723681425
14.2857142857143 1.95430140458104
14.4897959183673 1.92487795968279
14.6938775510204 1.89736203734014
14.8979591836735 1.87151959684513
15.1020408163265 1.84713791725581
15.3061224489796 1.8240083884512
15.5102040816327 1.80191863722458
15.7142857142857 1.78065133090269
15.7645208780031 1.77551020408163
15.9183673469388 1.75696559178509
16.1224489795918 1.73289744460305
16.3265306122449 1.70917042297689
16.530612244898 1.68550563027812
16.734693877551 1.66165463703251
16.9387755102041 1.63740346561294
17.1428571428571 1.61257542473805
17.3469387755102 1.58703283583637
17.5510204081633 1.56067771616412
17.7551020408163 1.5334514648406
17.9591836734694 1.50533356608607
18.1632653061224 1.47633929922697
18.3673469387755 1.44651643945486
18.5714285714286 1.41594095115811
18.7548149528213 1.38775510204082
18.7755102040816 1.38346168247547
};
\addplot [draw=none, fill=white!4.70588235294118!black]
table{%
x  y
10.2040816326531 1
10.4081632653061 1
10.6122448979592 1
10.8163265306122 1
11.0204081632653 1
11.2244897959184 1
11.4285714285714 1
11.6326530612245 1
11.8367346938776 1
12.0408163265306 1
12.2448979591837 1
12.4489795918367 1
12.6530612244898 1
12.8571428571429 1
13.0612244897959 1
13.265306122449 1
13.469387755102 1
13.6734693877551 1
13.8775510204082 1
14.0816326530612 1
14.2857142857143 1
14.4897959183673 1
14.6938775510204 1
14.8979591836735 1
15.1020408163265 1
15.3061224489796 1
15.5102040816327 1
15.7142857142857 1
15.9183673469388 1
16.1224489795918 1
16.3265306122449 1
16.530612244898 1
16.734693877551 1
16.9387755102041 1
17.1428571428571 1
17.3469387755102 1
17.5510204081633 1
17.7551020408163 1
17.9591836734694 1
18.1632653061224 1
18.3673469387755 1
18.5714285714286 1
18.7755102040816 1
18.9795918367347 1
19.1836734693878 1
19.3877551020408 1
19.5918367346939 1
19.7959183673469 1
20 1
20 1.09659193909172
19.7959183673469 1.14618174373462
19.5918367346939 1.19541102126551
19.3877551020408 1.24399954919528
19.1836734693878 1.29168681143927
18.9795918367347 1.33824010011072
18.7755102040816 1.38346168247547
18.7548149528213 1.38775510204082
18.5714285714286 1.41594095115811
18.3673469387755 1.44651643945486
18.1632653061224 1.47633929922697
17.9591836734694 1.50533356608607
17.7551020408163 1.5334514648406
17.5510204081633 1.56067771616412
17.3469387755102 1.58703283583637
17.1428571428571 1.61257542473805
16.9387755102041 1.63740346561294
16.734693877551 1.66165463703251
16.530612244898 1.68550563027812
16.3265306122449 1.70917042297689
16.1224489795918 1.73289744460305
15.9183673469388 1.75696559178509
15.7645208780031 1.77551020408163
15.7142857142857 1.78065133090269
15.5102040816327 1.80191863722458
15.3061224489796 1.8240083884512
15.1020408163265 1.84713791725581
14.8979591836735 1.87151959684512
14.6938775510204 1.89736203734014
14.4897959183673 1.92487795968279
14.2857142857143 1.95430140458104
14.0816326530612 1.98591723681425
13.8775510204082 2.02010640650179
13.6734693877551 2.05741143573433
13.469387755102 2.0986284825703
13.265306122449 2.1449354741759
13.1930935943274 2.16326530612245
13.0612244897959 2.19204719908888
12.8571428571429 2.24318746547248
12.6530612244898 2.30425573132625
12.4489795918367 2.3793278967992
12.2448979591837 2.47423495661483
12.112704752937 2.55102040816327
12.0408163265306 2.59116358569388
11.8367346938776 2.73087139979451
11.6326530612245 2.91553948851726
11.6103624012783 2.93877551020408
11.4285714285714 3.13886633802041
11.2889026362676 3.3265306122449
11.2244897959184 3.42197222466274
11.0487850183609 3.71428571428571
11.0204081632653 3.76784519960826
10.8515572332859 4.10204081632653
10.8163265306122 4.18258255725744
10.6809016929619 4.48979591836735
10.6122448979592 4.67158177807773
10.5302967111368 4.87755102040816
10.4081632653061 5.23745006553435
10.3978467297327 5.26530612244898
10.2676865928431 5.6530612244898
10.2040816326531 5.87358158625336
10.1495032771125 6.04081632653061
10.0374683945803 6.42857142857143
10 6.57373815332841
10 6.42857142857143
10 6.04081632653061
10 5.6530612244898
10 5.26530612244898
10 4.87755102040816
10 4.48979591836735
10 4.10204081632653
10 3.71428571428571
10 3.3265306122449
10 2.93877551020408
10 2.55102040816327
10 2.16326530612245
10 1.77551020408163
10 1.38775510204082
10 1
10.2040816326531 1
};

\nextgroupplot[
height=\figureheight,
scaled y ticks=false,
scaled y ticks=manual:{}{\pgfmathparse{#1}},
tick align=outside,
tick pos=left,
title={$\speedegosymbol=\SI{20}{\meter\per\second}$, $\accelerationleadsymbol=\SI{0}{\meter\per\second\squared}$},
width=\figurewidth,
x grid style={white!69.0196078431373!black},
xlabel={$\speedleadsymbol$ [\si{\meter\per\second}]},
xmin=10, xmax=20,
xtick style={color=black},
xticklabel style={align=center},
y grid style={white!69.0196078431373!black},
ymin=1, ymax=20,
ytick style={color=black},
yticklabel style={/pgf/number format/fixed,/pgf/number format/precision=3},
yticklabels={}
]
\addplot [draw=none, fill=white!95.2941176470588!black]
table{%
x  y
20 4.06171831374716
20 4.10204081632653
20 4.48979591836735
20 4.87755102040816
20 5.26530612244898
20 5.6530612244898
20 6.04081632653061
20 6.42857142857143
20 6.81632653061224
20 7.20408163265306
20 7.59183673469388
20 7.97959183673469
20 8.36734693877551
20 8.75510204081633
20 9.14285714285714
20 9.53061224489796
20 9.91836734693877
20 10.3061224489796
20 10.6938775510204
20 11.0816326530612
20 11.469387755102
20 11.8571428571429
20 12.2448979591837
20 12.6326530612245
20 13.0204081632653
20 13.4081632653061
20 13.7959183673469
20 14.1836734693878
20 14.5714285714286
20 14.9591836734694
20 15.3469387755102
20 15.734693877551
20 16.1224489795918
20 16.5102040816327
20 16.8979591836735
20 17.2857142857143
20 17.6734693877551
20 18.0612244897959
20 18.4489795918367
20 18.8367346938775
20 19.2244897959184
20 19.6122448979592
20 20
19.7959183673469 20
19.5918367346939 20
19.3877551020408 20
19.1836734693878 20
18.9795918367347 20
18.7755102040816 20
18.5714285714286 20
18.3673469387755 20
18.1632653061224 20
17.9591836734694 20
17.7551020408163 20
17.5510204081633 20
17.3469387755102 20
17.1428571428571 20
16.9387755102041 20
16.734693877551 20
16.530612244898 20
16.3265306122449 20
16.1224489795918 20
15.9183673469388 20
15.7142857142857 20
15.5102040816327 20
15.3061224489796 20
15.1020408163265 20
14.8979591836735 20
14.6938775510204 20
14.4897959183673 20
14.2857142857143 20
14.0816326530612 20
13.8775510204082 20
13.6734693877551 20
13.469387755102 20
13.265306122449 20
13.0612244897959 20
12.8571428571429 20
12.742894928317 20
12.8339914932436 19.6122448979592
12.8571428571429 19.5159031445494
12.932680507375 19.2244897959184
13.0341517723737 18.8367346938775
13.0612244897959 18.7355279084449
13.1429306586662 18.4489795918367
13.2544250854164 18.0612244897959
13.265306122449 18.0243514555422
13.3742368236642 17.6734693877551
13.469387755102 17.3694966617196
13.4966570976849 17.2857142857143
13.6254645528057 16.8979591836735
13.6734693877551 16.7558485554091
13.758614006531 16.5102040816327
13.8775510204082 16.1708101860519
13.8947619417767 16.1224489795918
14.0359047080385 15.734693877551
14.0816326530612 15.6111876000703
14.1799461198029 15.3469387755102
14.2857142857143 15.0663780357426
14.3260383720215 14.9591836734694
14.4744580028573 14.5714285714286
14.4897959183673 14.5323588160736
14.6254823945843 14.1836734693878
14.6938775510204 14.0106394251649
14.777729721518 13.7959183673469
14.8979591836735 13.49294575384
14.9311418713817 13.4081632653061
15.086130085624 13.0204081632653
15.1020408163265 12.9816947991751
15.2432637116282 12.6326530612245
15.3061224489796 12.4803473765755
15.4018408074635 12.2448979591837
15.5102040816327 11.9838657396906
15.5620896019097 11.8571428571429
15.7142857142857 11.4936016193124
15.7243008675724 11.469387755102
15.8899679922149 11.0816326530612
15.9183673469388 11.0171163281518
16.0589426810234 10.6938775510204
16.1224489795918 10.5520146059046
16.2313802899775 10.3061224489796
16.3265306122449 10.0979219861175
16.4078481392385 9.91836734693877
16.530612244898 9.65624461636859
16.5889764703462 9.53061224489796
16.734693877551 9.22810869216528
16.775446396308 9.14285714285714
16.9387755102041 8.81429194626696
16.9679697256719 8.75510204081633
17.1428571428571 8.41519091522586
17.1672629346626 8.36734693877551
17.3469387755102 8.03082005585512
17.3740186780821 7.97959183673469
17.5510204081633 7.66083471764215
17.5888785422824 7.59183673469388
17.7551020408163 7.3045679707905
17.8124098175867 7.20408163265306
17.9591836734694 6.96107178722572
18.0450868049511 6.81632653061224
18.1632653061224 6.62915585884146
18.2872738867638 6.42857142857143
18.3673469387755 6.30742170761092
18.5392038472712 6.04081632653061
18.5714285714286 5.99429477412796
18.7755102040816 5.69367451998152
18.802572947705 5.6530612244898
18.9795918367347 5.40716140212229
19.0781625241395 5.26530612244898
19.1836734693878 5.1254854694542
19.3632333334991 4.87755102040816
19.3877551020408 4.84651928280473
19.5918367346939 4.58248101111436
19.6606870347127 4.48979591836735
19.7959183673469 4.32343694576656
19.9660553453049 4.10204081632653
20 4.06171831374716
};
\addplot [draw=none, fill=white!85.0980392156863!black]
table{%
x  y
20 2.4203215608755
20 2.55102040816327
20 2.93877551020408
20 3.3265306122449
20 3.71428571428571
20 4.06171831374716
19.9660553453049 4.10204081632653
19.7959183673469 4.32343694576656
19.6606870347127 4.48979591836735
19.5918367346939 4.58248101111436
19.3877551020408 4.84651928280473
19.3632333334991 4.87755102040816
19.1836734693878 5.1254854694542
19.0781625241395 5.26530612244898
18.9795918367347 5.40716140212229
18.802572947705 5.6530612244898
18.7755102040816 5.69367451998152
18.5714285714286 5.99429477412796
18.5392038472712 6.04081632653061
18.3673469387755 6.30742170761092
18.2872738867638 6.42857142857143
18.1632653061224 6.62915585884146
18.0450868049511 6.81632653061224
17.9591836734694 6.96107178722572
17.8124098175867 7.20408163265306
17.7551020408163 7.3045679707905
17.5888785422824 7.59183673469388
17.5510204081633 7.66083471764215
17.3740186780821 7.97959183673469
17.3469387755102 8.03082005585512
17.1672629346626 8.36734693877551
17.1428571428571 8.41519091522586
16.9679697256719 8.75510204081633
16.9387755102041 8.81429194626696
16.775446396308 9.14285714285714
16.734693877551 9.22810869216528
16.5889764703462 9.53061224489796
16.530612244898 9.65624461636859
16.4078481392385 9.91836734693877
16.3265306122449 10.0979219861175
16.2313802899775 10.3061224489796
16.1224489795918 10.5520146059046
16.0589426810234 10.6938775510204
15.9183673469388 11.0171163281518
15.8899679922149 11.0816326530612
15.7243008675724 11.469387755102
15.7142857142857 11.4936016193124
15.5620896019097 11.8571428571429
15.5102040816327 11.9838657396906
15.4018408074635 12.2448979591837
15.3061224489796 12.4803473765755
15.2432637116282 12.6326530612245
15.1020408163265 12.9816947991751
15.086130085624 13.0204081632653
14.9311418713817 13.4081632653061
14.8979591836735 13.49294575384
14.777729721518 13.7959183673469
14.6938775510204 14.0106394251649
14.6254823945843 14.1836734693878
14.4897959183673 14.5323588160736
14.4744580028573 14.5714285714286
14.3260383720215 14.9591836734694
14.2857142857143 15.0663780357426
14.1799461198029 15.3469387755102
14.0816326530612 15.6111876000703
14.0359047080385 15.734693877551
13.8947619417767 16.1224489795918
13.8775510204082 16.1708101860519
13.758614006531 16.5102040816327
13.6734693877551 16.7558485554091
13.6254645528057 16.8979591836735
13.4966570976849 17.2857142857143
13.469387755102 17.3694966617196
13.3742368236642 17.6734693877551
13.265306122449 18.0243514555422
13.2544250854164 18.0612244897959
13.1429306586662 18.4489795918367
13.0612244897959 18.7355279084449
13.0341517723737 18.8367346938775
12.932680507375 19.2244897959184
12.8571428571429 19.5159031445494
12.8339914932436 19.6122448979592
12.742894928317 20
12.6530612244898 20
12.4489795918367 20
12.2448979591837 20
12.0408163265306 20
11.8367346938776 20
11.6326530612245 20
11.4285714285714 20
11.2244897959184 20
11.1011507520104 20
11.1588626747693 19.6122448979592
11.2175656597208 19.2244897959184
11.2244897959184 19.1800441423495
11.2822472355837 18.8367346938775
11.3486782338406 18.4489795918367
11.4162960511744 18.0612244897959
11.4285714285714 17.9928816854849
11.490643190634 17.6734693877551
11.5675106157987 17.2857142857143
11.6326530612245 16.9639196535967
11.6471188168771 16.8979591836735
11.7347423103688 16.5102040816327
11.8239730306342 16.1224489795918
11.8367346938776 16.0687854704975
11.9226121636763 15.734693877551
12.0243033154738 15.3469387755102
12.0408163265306 15.2860412411554
12.1363055889456 14.9591836734694
12.2448979591837 14.5952370891744
12.2525214855649 14.5714285714286
12.3809761854446 14.1836734693878
12.4489795918367 13.983671823707
12.5169318781506 13.7959183673469
12.6530612244898 13.4296308705368
12.6614836598602 13.4081632653061
12.8189029734423 13.0204081632653
12.8571428571429 12.9291138486443
12.9866579513664 12.6326530612245
13.0612244897959 12.4672480368694
13.1646382267053 12.2448979591837
13.265306122449 12.0354391788046
13.3527259517109 11.8571428571429
13.469387755102 11.6272043542846
13.550190188482 11.469387755102
13.6734693877551 11.2370267415992
13.7558436404037 11.0816326530612
13.8775510204082 10.8603878232079
13.9682839497545 10.6938775510204
14.0816326530612 10.4937397162976
14.186143743112 10.3061224489796
14.2857142857143 10.1344340596491
14.4082880364662 9.91836734693877
14.4897959183673 9.78062095027334
14.6339306776602 9.53061224489796
14.6938775510204 9.4311282578652
14.862672504215 9.14285714285714
14.8979591836735 9.08532928265147
15.0944813393785 8.75510204081633
15.1020408163265 8.74300552391587
15.3061224489796 8.40769088410393
15.3299885318613 8.36734693877551
15.5102040816327 8.07790526984216
15.5696291434657 7.97959183673469
15.7142857142857 7.75285920600117
15.8140507736073 7.59183673469388
15.9183673469388 7.432803580181
16.0640990838798 7.20408163265306
16.1224489795918 7.11785326746155
16.3206567057536 6.81632653061225
16.3265306122449 6.80794054496399
16.530612244898 6.51137673615545
16.5861060443285 6.42857142857143
16.734693877551 6.22122370479515
16.8603324088598 6.04081632653061
16.9387755102041 5.93587827404134
17.1428571428571 5.65466269905999
17.1440058053561 5.6530612244898
17.3469387755102 5.39063390011035
17.4408806606058 5.26530612244898
17.5510204081633 5.12963775904952
17.7489061570104 4.87755102040816
17.7551020408163 4.87029893414131
17.9591836734694 4.62887393731478
18.0725995516062 4.48979591836735
18.1632653061224 4.388233363537
18.3673469387755 4.15233415549894
18.4099467426594 4.10204081632653
18.5714285714286 3.9288438671868
18.761889965946 3.71428571428571
18.7755102040816 3.70040950288054
18.9795918367347 3.48875779657772
19.1274998221094 3.3265306122449
19.1836734693878 3.27091395529284
19.3877551020408 3.06125779817732
19.5004053272972 2.93877551020408
19.5918367346939 2.84869540120245
19.7959183673469 2.63507567460693
19.8716845358267 2.55102040816327
20 2.4203215608755
};
\addplot [draw=none, fill=white!74.9019607843137!black]
table{%
x  y
19.7959183673469 1.66263789181917
20 1.45078746809421
20 1.77551020408163
20 2.16326530612245
20 2.4203215608755
19.8716845358267 2.55102040816327
19.7959183673469 2.63507567460693
19.5918367346939 2.84869540120245
19.5004053272972 2.93877551020408
19.3877551020408 3.06125779817732
19.1836734693878 3.27091395529284
19.1274998221094 3.3265306122449
18.9795918367347 3.48875779657772
18.7755102040816 3.70040950288054
18.761889965946 3.71428571428571
18.5714285714286 3.9288438671868
18.4099467426594 4.10204081632653
18.3673469387755 4.15233415549894
18.1632653061224 4.388233363537
18.0725995516062 4.48979591836735
17.9591836734694 4.62887393731478
17.7551020408163 4.87029893414131
17.7489061570104 4.87755102040816
17.5510204081633 5.12963775904952
17.4408806606058 5.26530612244898
17.3469387755102 5.39063390011035
17.1440058053561 5.6530612244898
17.1428571428571 5.65466269905999
16.9387755102041 5.93587827404134
16.8603324088598 6.04081632653061
16.734693877551 6.22122370479515
16.5861060443285 6.42857142857143
16.530612244898 6.51137673615545
16.3265306122449 6.80794054496399
16.3206567057536 6.81632653061224
16.1224489795918 7.11785326746155
16.0640990838798 7.20408163265306
15.9183673469388 7.432803580181
15.8140507736073 7.59183673469388
15.7142857142857 7.75285920600117
15.5696291434657 7.97959183673469
15.5102040816327 8.07790526984216
15.3299885318613 8.36734693877551
15.3061224489796 8.40769088410393
15.1020408163265 8.74300552391587
15.0944813393785 8.75510204081633
14.8979591836735 9.08532928265147
14.862672504215 9.14285714285714
14.6938775510204 9.4311282578652
14.6339306776602 9.53061224489796
14.4897959183673 9.78062095027334
14.4082880364662 9.91836734693877
14.2857142857143 10.1344340596491
14.186143743112 10.3061224489796
14.0816326530612 10.4937397162976
13.9682839497545 10.6938775510204
13.8775510204082 10.8603878232079
13.7558436404037 11.0816326530612
13.6734693877551 11.2370267415992
13.550190188482 11.469387755102
13.469387755102 11.6272043542846
13.3527259517109 11.8571428571429
13.265306122449 12.0354391788046
13.1646382267053 12.2448979591837
13.0612244897959 12.4672480368694
12.9866579513664 12.6326530612245
12.8571428571429 12.9291138486443
12.8189029734423 13.0204081632653
12.6614836598602 13.4081632653061
12.6530612244898 13.4296308705368
12.5169318781506 13.7959183673469
12.4489795918367 13.983671823707
12.3809761854446 14.1836734693878
12.2525214855649 14.5714285714286
12.2448979591837 14.5952370891744
12.1363055889456 14.9591836734694
12.0408163265306 15.2860412411554
12.0243033154738 15.3469387755102
11.9226121636763 15.734693877551
11.8367346938776 16.0687854704975
11.8239730306342 16.1224489795918
11.7347423103688 16.5102040816327
11.6471188168771 16.8979591836735
11.6326530612245 16.9639196535967
11.5675106157987 17.2857142857143
11.490643190634 17.6734693877551
11.4285714285714 17.9928816854849
11.4162960511744 18.0612244897959
11.3486782338406 18.4489795918367
11.2822472355837 18.8367346938775
11.2244897959184 19.1800441423495
11.2175656597208 19.2244897959184
11.1588626747693 19.6122448979592
11.1011507520104 20
11.0204081632653 20
10.8163265306122 20
10.6122448979592 20
10.4081632653061 20
10.2040816326531 20
10.1574386035826 20
10.2040816326531 19.633532922664
10.2068949498664 19.6122448979592
10.259538545041 19.2244897959184
10.3132986240375 18.8367346938775
10.3682612142035 18.4489795918367
10.4081632653061 18.1745047306994
10.4253519294415 18.0612244897959
10.4858398660382 17.6734693877551
10.5477542478306 17.2857142857143
10.611216942824 16.8979591836735
10.6122448979592 16.8918830641914
10.6799446617987 16.5102040816327
10.7504649775183 16.1224489795918
10.8163265306122 15.7700358611093
10.823292529115 15.734693877551
10.9023418257003 15.3469387755102
10.9835509772457 14.9591836734694
11.0204081632653 14.789001474831
11.0702865925185 14.5714285714286
11.1620629892708 14.1836734693878
11.2244897959184 13.9284125894527
11.2589095416944 13.7959183673469
11.3632743607887 13.4081632653061
11.4285714285714 13.173770404584
11.4740361920951 13.0204081632653
11.5932255576689 12.6326530612245
11.6326530612245 12.5092779515446
11.7225392492029 12.2448979591837
11.8367346938776 11.9208837886212
11.8606141618254 11.8571428571429
12.0118410492133 11.469387755102
12.0408163265306 11.3981309280451
12.1769396311685 11.0816326530612
12.2448979591837 10.9299587782048
12.3561787396765 10.6938775510204
12.4489795918367 10.5051633915294
12.551199355096 10.3061224489796
12.6530612244898 10.1162739486659
12.7629746602331 9.91836734693877
12.8571428571429 9.75631235497639
12.9915614742096 9.53061224489796
13.0612244897959 9.41898925025762
13.2360401493908 9.14285714285714
13.265306122449 9.09881547194739
13.469387755102 8.79385074014112
13.495308717785 8.75510204081633
13.6734693877551 8.5017933529685
13.766892542688 8.36734693877551
13.8775510204082 8.21619266606179
14.0474655131127 7.97959183673469
14.0816326530612 7.93453399669974
14.2857142857143 7.66023627729398
14.3353674175243 7.59183673469388
14.4897959183673 7.39089400552873
14.6288605430792 7.20408163265306
14.6938775510204 7.12179848649228
14.8979591836735 6.85555508810762
14.9272293226302 6.81632653061224
15.1020408163265 6.59629327719421
15.2306129917819 6.42857142857143
15.3061224489796 6.33633251060502
15.5102040816327 6.07887771123283
15.5396065279885 6.04081632653061
15.7142857142857 5.82980856040098
15.8554813007102 5.6530612244898
15.9183673469388 5.57983892109896
16.1224489795918 5.33554746026183
16.1796040916276 5.26530612244898
16.3265306122449 5.09803433769069
16.5134911869969 4.87755102040816
16.530612244898 4.8589192128707
16.734693877551 4.63291545971164
16.8597295175512 4.48979591836735
16.9387755102041 4.40675279691013
17.1428571428571 4.18698008472916
17.219791943466 4.10204081632653
17.3469387755102 3.97402006787325
17.5510204081633 3.76191020694416
17.5959488472798 3.71428571428571
17.7551020408163 3.56154010865716
17.9591836734694 3.35795026747416
17.9900909727061 3.3265306122449
18.1632653061224 3.16832606090574
18.3673469387755 2.97304264264142
18.4023133459413 2.93877551020408
18.5714285714286 2.79078399361372
18.7755102040816 2.60193161326603
18.8285879389764 2.55102040816327
18.9795918367347 2.42205799571324
19.1836734693878 2.23540274591755
19.258348447177 2.16326530612245
19.3877551020408 2.05143735288657
19.5918367346939 1.85923782723413
19.6742295456678 1.77551020408163
19.7959183673469 1.66263789181917
};
\addplot [draw=none, fill=white!65.0980392156863!black]
table{%
x  y
19.3877551020408 1.3488015859139
19.5918367346939 1.16838797504284
19.7585153792963 1
19.7959183673469 1
20 1
20 1.38775510204082
20 1.45078746809421
19.7959183673469 1.66263789181917
19.6742295456678 1.77551020408163
19.5918367346939 1.85923782723413
19.3877551020408 2.05143735288657
19.258348447177 2.16326530612245
19.1836734693878 2.23540274591755
18.9795918367347 2.42205799571324
18.8285879389764 2.55102040816327
18.7755102040816 2.60193161326603
18.5714285714286 2.79078399361372
18.4023133459413 2.93877551020408
18.3673469387755 2.97304264264142
18.1632653061224 3.16832606090574
17.9900909727061 3.3265306122449
17.9591836734694 3.35795026747416
17.7551020408163 3.56154010865716
17.5959488472798 3.71428571428571
17.5510204081633 3.76191020694416
17.3469387755102 3.97402006787325
17.219791943466 4.10204081632653
17.1428571428571 4.18698008472916
16.9387755102041 4.40675279691013
16.8597295175512 4.48979591836735
16.734693877551 4.63291545971164
16.530612244898 4.8589192128707
16.5134911869969 4.87755102040816
16.3265306122449 5.09803433769069
16.1796040916276 5.26530612244898
16.1224489795918 5.33554746026183
15.9183673469388 5.57983892109896
15.8554813007102 5.6530612244898
15.7142857142857 5.82980856040098
15.5396065279885 6.04081632653061
15.5102040816327 6.07887771123283
15.3061224489796 6.33633251060502
15.2306129917819 6.42857142857143
15.1020408163265 6.59629327719421
14.9272293226302 6.81632653061225
14.8979591836735 6.85555508810762
14.6938775510204 7.12179848649228
14.6288605430792 7.20408163265306
14.4897959183673 7.39089400552873
14.3353674175243 7.59183673469388
14.2857142857143 7.66023627729398
14.0816326530612 7.93453399669974
14.0474655131127 7.97959183673469
13.8775510204082 8.21619266606179
13.766892542688 8.36734693877551
13.6734693877551 8.5017933529685
13.495308717785 8.75510204081633
13.469387755102 8.79385074014112
13.265306122449 9.09881547194739
13.2360401493908 9.14285714285714
13.0612244897959 9.41898925025762
12.9915614742096 9.53061224489796
12.8571428571429 9.75631235497639
12.7629746602331 9.91836734693877
12.6530612244898 10.1162739486659
12.551199355096 10.3061224489796
12.4489795918367 10.5051633915294
12.3561787396765 10.6938775510204
12.2448979591837 10.9299587782048
12.1769396311685 11.0816326530612
12.0408163265306 11.3981309280451
12.0118410492133 11.469387755102
11.8606141618254 11.8571428571429
11.8367346938776 11.9208837886212
11.7225392492029 12.2448979591837
11.6326530612245 12.5092779515446
11.5932255576689 12.6326530612245
11.4740361920951 13.0204081632653
11.4285714285714 13.173770404584
11.3632743607887 13.4081632653061
11.2589095416944 13.7959183673469
11.2244897959184 13.9284125894527
11.1620629892708 14.1836734693878
11.0702865925185 14.5714285714286
11.0204081632653 14.789001474831
10.9835509772457 14.9591836734694
10.9023418257003 15.3469387755102
10.823292529115 15.734693877551
10.8163265306122 15.7700358611093
10.7504649775183 16.1224489795918
10.6799446617987 16.5102040816327
10.6122448979592 16.8918830641914
10.611216942824 16.8979591836735
10.5477542478306 17.2857142857143
10.4858398660382 17.6734693877551
10.4253519294415 18.0612244897959
10.4081632653061 18.1745047306994
10.3682612142035 18.4489795918367
10.3132986240375 18.8367346938775
10.259538545041 19.2244897959184
10.2068949498664 19.6122448979592
10.2040816326531 19.633532922664
10.1574386035826 20
10 20
10 19.6122448979592
10 19.2244897959184
10 18.8367346938775
10 18.4489795918367
10 18.0612244897959
10 17.6734693877551
10 17.2857142857143
10 16.8979591836735
10 16.5102040816327
10 16.1224489795918
10 15.7626065446854
10.0048850764819 15.734693877551
10.0750659871106 15.3469387755102
10.1475568763673 14.9591836734694
10.2040816326531 14.6671994147053
10.2230694005127 14.5714285714286
10.3027600457958 14.1836734693878
10.3853438332189 13.7959183673469
10.4081632653061 13.6928928368828
10.4731272138289 13.4081632653061
10.5650273499367 13.0204081632653
10.6122448979592 12.8291450115254
10.6625279281334 12.6326530612245
10.7658885543307 12.2448979591837
10.8163265306122 12.0636617739001
10.8762288144859 11.8571428571429
10.993652524378 11.469387755102
11.0204081632653 11.385020721888
11.1211192405447 11.0816326530612
11.2244897959184 10.7840899071403
11.2574271031261 10.6938775510204
11.4058177002951 10.3061224489796
11.4285714285714 10.2495327736799
11.5686448201578 9.91836734693877
11.6326530612245 9.7743334044634
11.7466164155032 9.53061224489796
11.8367346938776 9.34734392691076
11.9423406336184 9.14285714285714
12.0408163265306 8.96170322760391
12.1583069247849 8.75510204081633
12.2448979591837 8.61058366226084
12.3964757053154 8.36734693877551
12.4489795918367 8.28746334690944
12.6530612244898 7.98678423781026
12.6580804151833 7.97959183673469
12.8571428571429 7.70936725083831
12.9451077174701 7.59183673469388
13.0612244897959 7.44507774368586
13.2532068276116 7.20408163265306
13.265306122449 7.18973702117629
13.469387755102 6.94902761948144
13.5809791755097 6.81632653061224
13.6734693877551 6.71267710993673
13.8775510204082 6.48107959047818
13.9231275216379 6.42857142857143
14.0816326530612 6.25695072778382
14.2761623972742 6.04081632653061
14.2857142857143 6.03086504685276
14.4897959183673 5.81424267249944
14.636961609708 5.6530612244898
14.6938775510204 5.59480405213195
14.8979591836735 5.3805342111367
15.0043700527402 5.26530612244898
15.1020408163265 5.16682121485158
15.3061224489796 4.95489349284151
15.3785111499763 4.87755102040816
15.5102040816327 4.74705323895564
15.7142857142857 4.53834513522464
15.760580422021 4.48979591836735
15.9183673469388 4.33703878769769
16.1224489795918 4.1329650257063
16.1527051574237 4.10204081632653
16.3265306122449 3.93895649124333
16.530612244898 3.74121749368265
16.5578655068027 3.71428571428571
16.734693877551 3.55503078886406
16.9387755102041 3.36538334952825
16.9797644074178 3.3265306122449
17.1428571428571 3.1869056165425
17.3469387755102 3.00685235271057
17.4222766925013 2.93877551020408
17.5510204081633 2.83483114453563
17.7551020408163 2.66518190457074
17.887784698946 2.55102040816327
17.9591836734694 2.49676091825033
18.1632653061224 2.33710725116207
18.3673469387755 2.16859881391218
18.3736240684408 2.16326530612245
18.5714285714286 2.01589598273789
18.7755102040816 1.85278586723398
18.8666820527817 1.77551020408163
18.9795918367347 1.69125436662894
19.1836734693878 1.52671370506548
19.3398100788682 1.38775510204082
19.3877551020408 1.3488015859139
};
\addplot [draw=none, fill=white!54.9019607843137!black]
table{%
x  y
18.7755102040816 1.28936223026844
18.9795918367347 1.13880818463069
19.1489653557857 1
19.1836734693878 1
19.3877551020408 1
19.5918367346939 1
19.7585153792963 1
19.5918367346939 1.16838797504284
19.3877551020408 1.3488015859139
19.3398100788682 1.38775510204082
19.1836734693878 1.52671370506548
18.9795918367347 1.69125436662894
18.8666820527817 1.77551020408163
18.7755102040816 1.85278586723398
18.5714285714286 2.01589598273789
18.3736240684408 2.16326530612245
18.3673469387755 2.16859881391218
18.1632653061224 2.33710725116207
17.9591836734694 2.49676091825033
17.887784698946 2.55102040816327
17.7551020408163 2.66518190457074
17.5510204081633 2.83483114453563
17.4222766925013 2.93877551020408
17.3469387755102 3.00685235271057
17.1428571428571 3.1869056165425
16.9797644074178 3.3265306122449
16.9387755102041 3.36538334952825
16.734693877551 3.55503078886406
16.5578655068027 3.71428571428571
16.530612244898 3.74121749368265
16.3265306122449 3.93895649124333
16.1527051574237 4.10204081632653
16.1224489795918 4.1329650257063
15.9183673469388 4.33703878769769
15.760580422021 4.48979591836735
15.7142857142857 4.53834513522464
15.5102040816327 4.74705323895563
15.3785111499763 4.87755102040816
15.3061224489796 4.95489349284151
15.1020408163265 5.16682121485158
15.0043700527402 5.26530612244898
14.8979591836735 5.3805342111367
14.6938775510204 5.59480405213195
14.636961609708 5.6530612244898
14.4897959183673 5.81424267249944
14.2857142857143 6.03086504685276
14.2761623972742 6.04081632653061
14.0816326530612 6.25695072778382
13.9231275216379 6.42857142857143
13.8775510204082 6.48107959047817
13.6734693877551 6.71267710993673
13.5809791755097 6.81632653061224
13.469387755102 6.94902761948144
13.265306122449 7.18973702117629
13.2532068276116 7.20408163265306
13.0612244897959 7.44507774368586
12.9451077174701 7.59183673469388
12.8571428571429 7.70936725083831
12.6580804151833 7.97959183673469
12.6530612244898 7.98678423781026
12.4489795918367 8.28746334690944
12.3964757053154 8.36734693877551
12.2448979591837 8.61058366226084
12.1583069247849 8.75510204081633
12.0408163265306 8.96170322760391
11.9423406336184 9.14285714285714
11.8367346938776 9.34734392691076
11.7466164155032 9.53061224489796
11.6326530612245 9.7743334044634
11.5686448201578 9.91836734693877
11.4285714285714 10.2495327736799
11.4058177002951 10.3061224489796
11.2574271031261 10.6938775510204
11.2244897959184 10.7840899071403
11.1211192405447 11.0816326530612
11.0204081632653 11.385020721888
10.993652524378 11.469387755102
10.8762288144859 11.8571428571429
10.8163265306122 12.0636617739001
10.7658885543307 12.2448979591837
10.6625279281334 12.6326530612245
10.6122448979592 12.8291450115254
10.5650273499367 13.0204081632653
10.4731272138289 13.4081632653061
10.4081632653061 13.6928928368828
10.3853438332189 13.7959183673469
10.3027600457958 14.1836734693878
10.2230694005127 14.5714285714286
10.2040816326531 14.6671994147053
10.1475568763673 14.9591836734694
10.0750659871106 15.3469387755102
10.0048850764819 15.734693877551
10 15.7626065446854
10 15.734693877551
10 15.3469387755102
10 14.9591836734694
10 14.5714285714286
10 14.1836734693878
10 13.7959183673469
10 13.4081632653061
10 13.0204081632653
10 12.6326530612245
10 12.2448979591837
10 12.1333005585654
10.0707591579372 11.8571428571429
10.1750330223584 11.469387755102
10.2040816326531 11.3664501468311
10.2857292488646 11.0816326530612
10.4028226581268 10.6938775510204
10.4081632653061 10.6770525057766
10.528601483755 10.3061224489796
10.6122448979592 10.0622533826702
10.6630807926902 9.91836734693877
10.8077827518187 9.53061224489796
10.8163265306122 9.50893426246565
10.9657790000729 9.14285714285714
11.0204081632653 9.01651776680286
11.1380790909296 8.75510204081633
11.2244897959184 8.57389921488111
11.3273846816775 8.36734693877551
11.4285714285714 8.17565642217367
11.5368736088366 7.97959183673469
11.6326530612245 7.81598640075414
11.7699489480991 7.59183673469388
11.8367346938776 7.4889729456853
12.0298638685206 7.20408163265306
12.0408163265306 7.18884182983163
12.2448979591837 6.91565873843915
12.3217033500044 6.81632653061225
12.4489795918367 6.66102453412413
12.6448987102165 6.42857142857143
12.6530612244898 6.41943871747051
12.8571428571429 6.19603915430432
13.0008729987273 6.04081632653061
13.0612244897959 5.97941380613661
13.265306122449 5.7734607894686
13.3845841441164 5.65306122448979
13.469387755102 5.57255966851247
13.6734693877551 5.37796051781143
13.790249495908 5.26530612244898
13.8775510204082 5.18629488502822
14.0816326530612 4.99900490128787
14.2113640030034 4.87755102040816
14.2857142857143 4.81243619993447
14.4897959183673 4.63016490854149
14.6430569508131 4.48979591836735
14.6938775510204 4.4464018070495
14.8979591836735 4.26827605876085
15.0832020393682 4.10204081632653
15.1020408163265 4.08634491768776
15.3061224489796 3.91258144460889
15.5102040816327 3.73393113781019
15.5321265207813 3.71428571428571
15.7142857142857 3.5639248215652
15.9183673469388 3.39118867058384
15.9929757197046 3.3265306122449
16.1224489795918 3.22398199195611
16.3265306122449 3.05888380792486
16.47144452781 2.93877551020408
16.530612244898 2.89442720227377
16.734693877551 2.73878338893915
16.9387755102041 2.57921282819946
16.9740948728262 2.55102040816327
17.1428571428571 2.43120453602618
17.3469387755102 2.28279497853349
17.5064151827114 2.16326530612245
17.5510204081633 2.13394264536923
17.7551020408163 1.99658572352586
17.9591836734694 1.85356490266265
18.0660276742923 1.77551020408163
18.1632653061224 1.71368930447986
18.3673469387755 1.57787293859922
18.5714285714286 1.43219511202927
18.6301329255833 1.38775510204082
18.7755102040816 1.28936223026844
};
\addplot [draw=none, fill=white!45.0980392156863!black]
table{%
x  y
17.9591836734694 1.33969632047051
18.1632653061224 1.22018071200184
18.3673469387755 1.09267610621213
18.5050468692672 1
18.5714285714286 1
18.7755102040816 1
18.9795918367347 1
19.1489653557857 1
18.9795918367347 1.13880818463069
18.7755102040816 1.28936223026844
18.6301329255833 1.38775510204082
18.5714285714286 1.43219511202927
18.3673469387755 1.57787293859922
18.1632653061224 1.71368930447986
18.0660276742923 1.77551020408163
17.9591836734694 1.85356490266265
17.7551020408163 1.99658572352586
17.5510204081633 2.13394264536923
17.5064151827114 2.16326530612245
17.3469387755102 2.28279497853349
17.1428571428571 2.43120453602618
16.9740948728262 2.55102040816326
16.9387755102041 2.57921282819946
16.734693877551 2.73878338893915
16.530612244898 2.89442720227377
16.47144452781 2.93877551020408
16.3265306122449 3.05888380792486
16.1224489795918 3.22398199195612
15.9929757197046 3.3265306122449
15.9183673469388 3.39118867058384
15.7142857142857 3.5639248215652
15.5321265207813 3.71428571428571
15.5102040816327 3.73393113781019
15.3061224489796 3.91258144460889
15.1020408163265 4.08634491768776
15.0832020393682 4.10204081632653
14.8979591836735 4.26827605876085
14.6938775510204 4.4464018070495
14.6430569508131 4.48979591836735
14.4897959183673 4.63016490854149
14.2857142857143 4.81243619993447
14.2113640030034 4.87755102040816
14.0816326530612 4.99900490128786
13.8775510204082 5.18629488502821
13.790249495908 5.26530612244898
13.6734693877551 5.37796051781143
13.469387755102 5.57255966851247
13.3845841441164 5.6530612244898
13.265306122449 5.7734607894686
13.0612244897959 5.97941380613661
13.0008729987273 6.04081632653061
12.8571428571429 6.19603915430432
12.6530612244898 6.41943871747051
12.6448987102165 6.42857142857143
12.4489795918367 6.66102453412413
12.3217033500044 6.81632653061225
12.2448979591837 6.91565873843915
12.0408163265306 7.18884182983163
12.0298638685206 7.20408163265306
11.8367346938776 7.4889729456853
11.7699489480991 7.59183673469388
11.6326530612245 7.81598640075414
11.5368736088366 7.97959183673469
11.4285714285714 8.17565642217367
11.3273846816775 8.36734693877551
11.2244897959184 8.57389921488111
11.1380790909296 8.75510204081633
11.0204081632653 9.01651776680286
10.9657790000729 9.14285714285714
10.8163265306122 9.50893426246565
10.8077827518187 9.53061224489796
10.6630807926902 9.91836734693877
10.6122448979592 10.0622533826702
10.528601483755 10.3061224489796
10.4081632653061 10.6770525057766
10.4028226581268 10.6938775510204
10.2857292488646 11.0816326530612
10.2040816326531 11.3664501468311
10.1750330223584 11.469387755102
10.0707591579372 11.8571428571429
10 12.1333005585654
10 11.8571428571429
10 11.469387755102
10 11.0816326530612
10 10.6938775510204
10 10.3061224489796
10 9.91836734693877
10 9.53061224489796
10 9.4663565253735
10.1169366484109 9.14285714285714
10.2040816326531 8.91677326275861
10.267287100331 8.75510204081633
10.4081632653061 8.41719238317416
10.4294009916999 8.36734693877551
10.6054178918928 7.97959183673469
10.6122448979592 7.9654692698751
10.7982464203924 7.59183673469388
10.8163265306122 7.55780592374612
11.0110703065724 7.20408163265306
11.0204081632653 7.18817779987327
11.2244897959184 6.85369548625882
11.2482804520477 6.81632653061224
11.4285714285714 6.55018287998335
11.5147840090063 6.42857142857143
11.6326530612245 6.27215965036314
11.8151633154115 6.04081632653061
11.8367346938776 6.01506386145095
12.0408163265306 5.78035593575736
12.1559227469674 5.6530612244898
12.2448979591837 5.56024486128678
12.4489795918367 5.35418910632486
12.539326076296 5.26530612244898
12.6530612244898 5.15968665311184
12.8571428571429 4.97423457905735
12.9649990421551 4.87755102040816
13.0612244897959 4.79614259485584
13.265306122449 4.62497796594889
13.4269917676978 4.48979591836735
13.469387755102 4.45637165668786
13.6734693877551 4.29488651383614
13.8775510204082 4.13259289622238
13.9154198890477 4.10204081632653
14.0816326530612 3.97608265147764
14.2857142857143 3.81961495765492
14.4207352397841 3.71428571428571
14.4897959183673 3.66382895285244
14.6938775510204 3.51213355730905
14.8979591836735 3.35787879906658
14.9384906881861 3.3265306122449
15.1020408163265 3.20899875804527
15.3061224489796 3.06001924357031
15.469090184974 2.93877551020408
15.5102040816327 2.91056457624837
15.7142857142857 2.76808984282422
15.9183673469388 2.62370300556638
16.0191220558121 2.55102040816327
16.1224489795918 2.48334411939799
16.3265306122449 2.34796708341265
16.530612244898 2.21079515006812
16.599892046661 2.16326530612245
16.734693877551 2.08087063328701
16.9387755102041 1.95451030532722
17.1428571428571 1.82582779548385
17.2205793992331 1.77551020408163
17.3469387755102 1.70367616474829
17.5510204081633 1.58491733164344
17.7551020408163 1.46206297579729
17.8732812581159 1.38775510204082
17.9591836734694 1.33969632047051
};
\addplot [draw=none, fill=white!34.9019607843137!black]
table{%
x  y
17.1428571428571 1.31985696947419
17.3469387755102 1.21840895845898
17.5510204081633 1.11439688963458
17.7551020408163 1.00603169061019
17.7658428787725 1
17.9591836734694 1
18.1632653061224 1
18.3673469387755 1
18.5050468692672 1
18.3673469387755 1.09267610621213
18.1632653061224 1.22018071200184
17.9591836734694 1.33969632047051
17.8732812581159 1.38775510204082
17.7551020408163 1.46206297579729
17.5510204081633 1.58491733164344
17.3469387755102 1.70367616474829
17.2205793992331 1.77551020408163
17.1428571428571 1.82582779548385
16.9387755102041 1.95451030532722
16.734693877551 2.08087063328701
16.599892046661 2.16326530612245
16.530612244898 2.21079515006812
16.3265306122449 2.34796708341265
16.1224489795918 2.48334411939799
16.0191220558121 2.55102040816327
15.9183673469388 2.62370300556639
15.7142857142857 2.76808984282422
15.5102040816327 2.91056457624837
15.469090184974 2.93877551020408
15.3061224489796 3.06001924357031
15.1020408163265 3.20899875804527
14.9384906881861 3.3265306122449
14.8979591836735 3.35787879906658
14.6938775510204 3.51213355730905
14.4897959183673 3.66382895285244
14.4207352397841 3.71428571428571
14.2857142857143 3.81961495765492
14.0816326530612 3.97608265147764
13.9154198890477 4.10204081632653
13.8775510204082 4.13259289622238
13.6734693877551 4.29488651383614
13.469387755102 4.45637165668786
13.4269917676978 4.48979591836735
13.265306122449 4.62497796594889
13.0612244897959 4.79614259485584
12.9649990421551 4.87755102040816
12.8571428571429 4.97423457905735
12.6530612244898 5.15968665311184
12.539326076296 5.26530612244898
12.4489795918367 5.35418910632486
12.2448979591837 5.56024486128678
12.1559227469674 5.6530612244898
12.0408163265306 5.78035593575736
11.8367346938776 6.01506386145095
11.8151633154115 6.04081632653061
11.6326530612245 6.27215965036314
11.5147840090063 6.42857142857143
11.4285714285714 6.55018287998335
11.2482804520477 6.81632653061224
11.2244897959184 6.85369548625882
11.0204081632653 7.18817779987327
11.0110703065724 7.20408163265306
10.8163265306122 7.55780592374612
10.7982464203924 7.59183673469388
10.6122448979592 7.9654692698751
10.6054178918928 7.97959183673469
10.4294009916999 8.36734693877551
10.4081632653061 8.41719238317416
10.267287100331 8.75510204081633
10.2040816326531 8.91677326275861
10.1169366484109 9.14285714285714
10 9.4663565253735
10 9.14285714285714
10 8.75510204081633
10 8.36734693877551
10 7.97959183673469
10 7.59183673469388
10 7.33475269785668
10.0642544171057 7.20408163265306
10.2040816326531 6.93895876977226
10.270076912019 6.81632653061224
10.4081632653061 6.57683704634892
10.4961373091876 6.42857142857143
10.6122448979592 6.24564513859526
10.7471212096512 6.04081632653061
10.8163265306122 5.94235794662135
11.0204081632653 5.66417903755651
11.0288555689671 5.6530612244898
11.2244897959184 5.41092934049895
11.3484279510278 5.26530612244898
11.4285714285714 5.1764748098535
11.6326530612245 4.96024170981381
11.714096155412 4.87755102040816
11.8367346938776 4.75962013154368
12.0408163265306 4.57198829824741
12.133460843291 4.48979591836735
12.2448979591837 4.39583785817581
12.4489795918367 4.22953742350727
12.6104706714113 4.10204081632653
12.6530612244898 4.06999385115758
12.8571428571429 3.91891440790282
13.0612244897959 3.77150201661965
13.1409120627585 3.71428571428571
13.265306122449 3.62918047031337
13.469387755102 3.4906944899683
13.6734693877551 3.35326360946781
13.7127332285488 3.3265306122449
13.8775510204082 3.21990369052665
14.0816326530612 3.08794320869508
14.2857142857143 2.95570660606221
14.3113340941821 2.93877551020408
14.4897959183673 2.82740124848201
14.6938775510204 2.6997559336085
14.8979591836735 2.57153667516823
14.9299221315587 2.55102040816327
15.1020408163265 2.44780696040873
15.3061224489796 2.32531808365714
15.5102040816327 2.20259473574637
15.5744532420467 2.16326530612245
15.7142857142857 2.0845180113204
15.9183673469388 1.96952074832146
16.1224489795918 1.85471543538392
16.2617827424219 1.77551020408163
16.3265306122449 1.74214254493289
16.530612244898 1.63596334632913
16.734693877551 1.52990128006151
16.9387755102041 1.42287278427018
17.0038624003613 1.38775510204082
17.1428571428571 1.31985696947419
};
\addplot [draw=none, fill=white!24.7058823529412!black]
table{%
x  y
15.9183673469388 1.37860542981507
16.1224489795918 1.28994326245572
16.3265306122449 1.20326263301716
16.530612244898 1.11760265110482
16.734693877551 1.03192680995646
16.8083573224391 1
16.9387755102041 1
17.1428571428571 1
17.3469387755102 1
17.5510204081633 1
17.7551020408163 1
17.7658428787725 1
17.7551020408163 1.00603169061019
17.5510204081633 1.11439688963458
17.3469387755102 1.21840895845898
17.1428571428571 1.31985696947419
17.0038624003613 1.38775510204082
16.9387755102041 1.42287278427018
16.734693877551 1.52990128006151
16.530612244898 1.63596334632913
16.3265306122449 1.74214254493289
16.2617827424219 1.77551020408163
16.1224489795918 1.85471543538392
15.9183673469388 1.96952074832146
15.7142857142857 2.0845180113204
15.5744532420467 2.16326530612245
15.5102040816327 2.20259473574637
15.3061224489796 2.32531808365714
15.1020408163265 2.44780696040873
14.9299221315587 2.55102040816327
14.8979591836735 2.57153667516823
14.6938775510204 2.6997559336085
14.4897959183673 2.82740124848202
14.3113340941821 2.93877551020408
14.2857142857143 2.95570660606221
14.0816326530612 3.08794320869508
13.8775510204082 3.21990369052665
13.7127332285488 3.3265306122449
13.6734693877551 3.35326360946781
13.469387755102 3.4906944899683
13.265306122449 3.62918047031337
13.1409120627585 3.71428571428571
13.0612244897959 3.77150201661965
12.8571428571429 3.91891440790282
12.6530612244898 4.06999385115758
12.6104706714113 4.10204081632653
12.4489795918367 4.22953742350727
12.2448979591837 4.39583785817581
12.133460843291 4.48979591836735
12.0408163265306 4.57198829824741
11.8367346938776 4.75962013154368
11.714096155412 4.87755102040816
11.6326530612245 4.96024170981381
11.4285714285714 5.1764748098535
11.3484279510278 5.26530612244898
11.2244897959184 5.41092934049895
11.0288555689671 5.6530612244898
11.0204081632653 5.66417903755651
10.8163265306122 5.94235794662135
10.7471212096512 6.04081632653061
10.6122448979592 6.24564513859526
10.4961373091876 6.42857142857143
10.4081632653061 6.57683704634892
10.270076912019 6.81632653061224
10.2040816326531 6.93895876977226
10.0642544171057 7.20408163265306
10 7.33475269785668
10 7.20408163265306
10 6.81632653061224
10 6.42857142857143
10 6.04081632653061
10 5.6530612244898
10 5.48438295998398
10.1590737761134 5.26530612244898
10.2040816326531 5.20710233278182
10.4081632653061 4.95056886341851
10.4684042772914 4.87755102040816
10.6122448979592 4.71292125686904
10.8163265306122 4.49284293857723
10.8192274115804 4.48979591836735
11.0204081632653 4.28889131834717
11.2216803220949 4.10204081632653
11.2244897959184 4.09954179260764
11.4285714285714 3.92298918847357
11.6326530612245 3.7583515294729
11.6892296589188 3.71428571428571
11.8367346938776 3.60333420452163
12.0408163265306 3.45752708350576
12.2339981669544 3.3265306122449
12.2448979591837 3.31933212554257
12.4489795918367 3.18692744963155
12.6530612244898 3.06051340998881
12.8569167199019 2.93877551020408
12.8571428571429 2.93864328485116
13.0612244897959 2.81968255329398
13.265306122449 2.70441717157083
13.469387755102 2.59178532367449
13.5430688334996 2.55102040816327
13.6734693877551 2.48061457036684
13.8775510204082 2.37163828623928
14.0816326530612 2.26457372365122
14.2767673925353 2.16326530612245
14.2857142857143 2.15874645072622
14.4897959183673 2.05439545366198
14.6938775510204 1.9522314261994
14.8979591836735 1.85171203022783
15.0532038577623 1.77551020408163
15.1020408163265 1.7525259541173
15.3061224489796 1.6560994865218
15.5102040816327 1.56205231533469
15.7142857142857 1.46974942490182
15.8970442416623 1.38775510204082
15.9183673469388 1.37860542981507
};
\addplot [draw=none, fill=white!14.9019607843137!black]
table{%
x  y
14.2857142857143 1.34779888667638
14.4897959183673 1.26993389435111
14.6938775510204 1.19610045339461
14.8979591836735 1.12567990627058
15.1020408163265 1.05807581551182
15.2821333249786 1
15.3061224489796 1
15.5102040816327 1
15.7142857142857 1
15.9183673469388 1
16.1224489795918 1
16.3265306122449 1
16.530612244898 1
16.734693877551 1
16.8083573224391 1
16.734693877551 1.03192680995646
16.530612244898 1.11760265110482
16.3265306122449 1.20326263301716
16.1224489795918 1.28994326245572
15.9183673469388 1.37860542981507
15.8970442416623 1.38775510204082
15.7142857142857 1.46974942490182
15.5102040816327 1.56205231533469
15.3061224489796 1.6560994865218
15.1020408163265 1.7525259541173
15.0532038577623 1.77551020408163
14.8979591836735 1.85171203022783
14.6938775510204 1.9522314261994
14.4897959183673 2.05439545366198
14.2857142857143 2.15874645072622
14.2767673925353 2.16326530612245
14.0816326530612 2.26457372365122
13.8775510204082 2.37163828623928
13.6734693877551 2.48061457036684
13.5430688334996 2.55102040816327
13.469387755102 2.59178532367449
13.265306122449 2.70441717157083
13.0612244897959 2.81968255329398
12.8571428571429 2.93864328485116
12.8569167199019 2.93877551020408
12.6530612244898 3.06051340998881
12.4489795918367 3.18692744963155
12.2448979591837 3.31933212554257
12.2339981669544 3.3265306122449
12.0408163265306 3.45752708350576
11.8367346938776 3.60333420452163
11.6892296589188 3.71428571428571
11.6326530612245 3.7583515294729
11.4285714285714 3.92298918847357
11.2244897959184 4.09954179260764
11.2216803220949 4.10204081632653
11.0204081632653 4.28889131834717
10.8192274115804 4.48979591836735
10.8163265306122 4.49284293857723
10.6122448979592 4.71292125686904
10.4684042772914 4.87755102040816
10.4081632653061 4.95056886341851
10.2040816326531 5.20710233278182
10.1590737761134 5.26530612244898
10 5.48438295998398
10 5.26530612244898
10 4.87755102040816
10 4.48979591836735
10 4.10204081632653
10 3.71428571428571
10 3.67892923121259
10.2040816326531 3.50013772377949
10.4081632653061 3.33577414067933
10.4198349513018 3.3265306122449
10.6122448979592 3.17634115267945
10.8163265306122 3.02982288876021
10.9512101573388 2.93877551020408
11.0204081632653 2.89188768044116
11.2244897959184 2.75934816536026
11.4285714285714 2.63746996479601
11.5826449577641 2.55102040816327
11.6326530612245 2.52230170733238
11.8367346938776 2.4089051940278
12.0408163265306 2.30369618466667
12.2448979591837 2.20520087606953
12.334260663428 2.16326530612245
12.4489795918367 2.10729952229716
12.6530612244898 2.01156169204656
12.8571428571429 1.92117735245452
13.0612244897959 1.83512536782002
13.2064656789857 1.77551020408163
13.265306122449 1.75008899488536
13.469387755102 1.66307373460094
13.6734693877551 1.5803997785934
13.8775510204082 1.50137256043145
14.0816326530612 1.42538963586008
14.1834895830528 1.38775510204082
14.2857142857143 1.34779888667638
};
\addplot [draw=none, fill=white!4.70588235294118!black]
table{%
x  y
10.2040816326531 1
10.4081632653061 1
10.6122448979592 1
10.8163265306122 1
11.0204081632653 1
11.2244897959184 1
11.4285714285714 1
11.6326530612245 1
11.8367346938776 1
12.0408163265306 1
12.2448979591837 1
12.4489795918367 1
12.6530612244898 1
12.8571428571429 1
13.0612244897959 1
13.265306122449 1
13.469387755102 1
13.6734693877551 1
13.8775510204082 1
14.0816326530612 1
14.2857142857143 1
14.4897959183673 1
14.6938775510204 1
14.8979591836735 1
15.1020408163265 1
15.2821333249786 1
15.1020408163265 1.05807581551182
14.8979591836735 1.12567990627058
14.6938775510204 1.19610045339461
14.4897959183673 1.26993389435111
14.2857142857143 1.34779888667638
14.1834895830528 1.38775510204082
14.0816326530612 1.42538963586008
13.8775510204082 1.50137256043145
13.6734693877551 1.5803997785934
13.469387755102 1.66307373460094
13.265306122449 1.75008899488536
13.2064656789857 1.77551020408163
13.0612244897959 1.83512536782002
12.8571428571429 1.92117735245452
12.6530612244898 2.01156169204656
12.4489795918367 2.10729952229716
12.334260663428 2.16326530612245
12.2448979591837 2.20520087606953
12.0408163265306 2.30369618466667
11.8367346938776 2.4089051940278
11.6326530612245 2.52230170733238
11.5826449577641 2.55102040816327
11.4285714285714 2.63746996479601
11.2244897959184 2.75934816536026
11.0204081632653 2.89188768044116
10.9512101573388 2.93877551020408
10.8163265306122 3.02982288876021
10.6122448979592 3.17634115267945
10.4198349513018 3.3265306122449
10.4081632653061 3.33577414067933
10.2040816326531 3.50013772377949
10 3.67892923121259
10 3.3265306122449
10 2.93877551020408
10 2.55102040816327
10 2.16326530612245
10 1.77551020408163
10 1.38775510204082
10 1
10.2040816326531 1
};

\nextgroupplot[
height=\figureheight,
scaled y ticks=false,
scaled y ticks=manual:{}{\pgfmathparse{#1}},
tick align=outside,
tick pos=left,
title={$\speedegosymbol=\SI{20}{\meter\per\second}$, $\accelerationleadsymbol=\SI{-1}{\meter\per\second\squared}$},
width=\figurewidth,
x grid style={white!69.0196078431373!black},
xlabel={$\speedleadsymbol$ [\si{\meter\per\second}]},
xmin=10, xmax=20,
xtick style={color=black},
xticklabel style={align=center},
y grid style={white!69.0196078431373!black},
ymin=1, ymax=20,
ytick style={color=black},
yticklabel style={/pgf/number format/fixed,/pgf/number format/precision=3},
yticklabels={}
]
\addplot [draw=none, fill=white!95.2941176470588!black]
table{%
x  y
20 3.67092022366153
20 3.71428571428571
20 4.10204081632653
20 4.48979591836735
20 4.87755102040816
20 5.26530612244898
20 5.6530612244898
20 6.04081632653061
20 6.42857142857143
20 6.81632653061224
20 7.20408163265306
20 7.59183673469388
20 7.97959183673469
20 8.36734693877551
20 8.75510204081633
20 9.14285714285714
20 9.53061224489796
20 9.91836734693877
20 10.3061224489796
20 10.6938775510204
20 11.0816326530612
20 11.469387755102
20 11.8571428571429
20 12.2448979591837
20 12.6326530612245
20 13.0204081632653
20 13.4081632653061
20 13.7959183673469
20 14.1836734693878
20 14.5714285714286
20 14.9591836734694
20 15.3469387755102
20 15.734693877551
20 16.1224489795918
20 16.5102040816327
20 16.8979591836735
20 17.2857142857143
20 17.6734693877551
20 18.0612244897959
20 18.4489795918367
20 18.8367346938775
20 19.2244897959184
20 19.6122448979592
20 20
19.7959183673469 20
19.5918367346939 20
19.3877551020408 20
19.1836734693878 20
18.9795918367347 20
18.7755102040816 20
18.5714285714286 20
18.3673469387755 20
18.1632653061224 20
17.9591836734694 20
17.7551020408163 20
17.5510204081633 20
17.3469387755102 20
17.1428571428571 20
16.9387755102041 20
16.734693877551 20
16.530612244898 20
16.3265306122449 20
16.1224489795918 20
15.9183673469388 20
15.7142857142857 20
15.5102040816327 20
15.3061224489796 20
15.1020408163265 20
14.8979591836735 20
14.6938775510204 20
14.4897959183673 20
14.2857142857143 20
14.0816326530612 20
13.8775510204082 20
13.6734693877551 20
13.469387755102 20
13.2973387315355 20
13.3850217725199 19.6122448979592
13.469387755102 19.2403892097248
13.4731215614068 19.2244897959184
13.5663667302069 18.8367346938775
13.6596761995026 18.4489795918367
13.6734693877551 18.392938070713
13.757468591383 18.0612244897959
13.8559303676825 17.6734693877551
13.8775510204082 17.5900468049071
13.9583606936842 17.2857142857143
14.0616534954412 16.8979591836735
14.0816326530612 16.824511996789
14.1688915622327 16.5102040816327
14.2766522577197 16.1224489795918
14.2857142857143 16.0906212106947
14.388771225276 15.734693877551
14.4897959183673 15.3855742285997
14.5011457012795 15.3469387755102
14.617623609045 14.9591836734694
14.6938775510204 14.7063397632711
14.7350516747385 14.5714285714286
14.8550468356981 14.1836734693878
14.8979591836735 14.0470162250191
14.9775080131127 13.7959183673469
15.1006829925729 13.4081632653061
15.1020408163265 13.4040101413383
15.2281967683193 13.0204081632653
15.3061224489796 12.7845763955704
15.3565560311569 12.6326530612245
15.4869713906755 12.2448979591837
15.5102040816327 12.1774598234909
15.6207529623495 11.8571428571429
15.7142857142857 11.5885457955701
15.755806168917 11.469387755102
15.8935041275697 11.0816326530612
15.9183673469388 11.0135116001149
16.0347662451878 10.6938775510204
16.1224489795918 10.4568706315995
16.1780291034381 10.3061224489796
16.3237722378857 9.91836734693877
16.3265306122449 9.91132794975584
16.4747296178516 9.53061224489796
16.530612244898 9.39079277898568
16.6289222080676 9.14285714285714
16.734693877551 8.88413277440394
16.7869544943947 8.75510204081633
16.9387755102041 8.39331671936019
16.9495566067121 8.36734693877551
17.1188195126831 7.97959183673469
17.1428571428571 7.92736861012949
17.294901990087 7.59183673469388
17.3469387755102 7.48337364054716
17.4785570653694 7.20408163265306
17.5510204081633 7.06017730215868
17.6713172433475 6.81632653061224
17.7551020408163 6.65908993405936
17.8750622127256 6.42857142857143
17.9591836734694 6.28080682327817
18.0921025657642 6.04081632653061
18.1632653061224 5.92511113110105
18.3252791051611 5.6530612244898
18.3673469387755 5.59055895846001
18.5714285714286 5.27640316886907
18.5784879090369 5.26530612244898
18.7755102040816 4.99730627800677
18.8596477866714 4.87755102040816
18.9795918367347 4.73347572588602
19.1715062122418 4.48979591836735
19.1836734693878 4.47712650250361
19.3877551020408 4.26153855637074
19.5290461437786 4.10204081632653
19.5918367346939 4.04605413837571
19.7959183673469 3.85830517078114
19.9419184602073 3.71428571428571
20 3.67092022366153
};
\addplot [draw=none, fill=white!85.0980392156863!black]
table{%
x  y
19.7959183673469 2.91649459331432
20 2.81529637859774
20 2.93877551020408
20 3.3265306122449
20 3.67092022366153
19.9419184602073 3.71428571428571
19.7959183673469 3.85830517078114
19.5918367346939 4.04605413837571
19.5290461437786 4.10204081632653
19.3877551020408 4.26153855637074
19.1836734693878 4.47712650250361
19.1715062122418 4.48979591836735
18.9795918367347 4.73347572588602
18.8596477866714 4.87755102040816
18.7755102040816 4.99730627800677
18.5784879090369 5.26530612244898
18.5714285714286 5.27640316886907
18.3673469387755 5.59055895846001
18.3252791051611 5.6530612244898
18.1632653061224 5.92511113110105
18.0921025657642 6.04081632653061
17.9591836734694 6.28080682327817
17.8750622127256 6.42857142857143
17.7551020408163 6.65908993405936
17.6713172433475 6.81632653061224
17.5510204081633 7.06017730215868
17.4785570653694 7.20408163265306
17.3469387755102 7.48337364054716
17.294901990087 7.59183673469388
17.1428571428571 7.92736861012949
17.1188195126831 7.97959183673469
16.9495566067121 8.36734693877551
16.9387755102041 8.39331671936019
16.7869544943947 8.75510204081633
16.734693877551 8.88413277440394
16.6289222080676 9.14285714285714
16.530612244898 9.39079277898568
16.4747296178516 9.53061224489796
16.3265306122449 9.91132794975584
16.3237722378857 9.91836734693877
16.1780291034381 10.3061224489796
16.1224489795918 10.4568706315995
16.0347662451878 10.6938775510204
15.9183673469388 11.0135116001149
15.8935041275697 11.0816326530612
15.755806168917 11.469387755102
15.7142857142857 11.5885457955701
15.6207529623495 11.8571428571429
15.5102040816327 12.1774598234909
15.4869713906755 12.2448979591837
15.3565560311569 12.6326530612245
15.3061224489796 12.7845763955704
15.2281967683193 13.0204081632653
15.1020408163265 13.4040101413383
15.1006829925729 13.4081632653061
14.9775080131127 13.7959183673469
14.8979591836735 14.0470162250191
14.8550468356981 14.1836734693878
14.7350516747385 14.5714285714286
14.6938775510204 14.7063397632711
14.617623609045 14.9591836734694
14.5011457012795 15.3469387755102
14.4897959183673 15.3855742285997
14.388771225276 15.734693877551
14.2857142857143 16.0906212106947
14.2766522577197 16.1224489795918
14.1688915622327 16.5102040816327
14.0816326530612 16.824511996789
14.0616534954412 16.8979591836735
13.9583606936842 17.2857142857143
13.8775510204082 17.5900468049071
13.8559303676825 17.6734693877551
13.757468591383 18.0612244897959
13.6734693877551 18.392938070713
13.6596761995026 18.4489795918367
13.5663667302069 18.8367346938775
13.4731215614068 19.2244897959184
13.469387755102 19.2403892097248
13.3850217725199 19.6122448979592
13.2973387315355 20
13.265306122449 20
13.0612244897959 20
12.8571428571429 20
12.6530612244898 20
12.4489795918367 20
12.2448979591837 20
12.0408163265306 20
11.8367346938776 20
11.6326530612245 20
11.4285714285714 20
11.3697594958827 20
11.4285714285714 19.6438123249337
11.4342098309093 19.6122448979592
11.5052246928718 19.2244897959184
11.5771238557306 18.8367346938775
11.6326530612245 18.5422600395467
11.651571357835 18.4489795918367
11.7321549490427 18.0612244897959
11.8138238390842 17.6734693877551
11.8367346938776 17.5674654757352
11.9017240510377 17.2857142857143
11.9927780051409 16.8979591836735
12.0408163265306 16.6979200025441
12.0885343081422 16.5102040816327
12.1893182310124 16.1224489795918
12.2448979591837 15.9133023916733
12.2946863484691 15.734693877551
12.4052583843985 15.3469387755102
12.4489795918367 15.1974327429285
12.5213721413703 14.9591836734694
12.6414792941952 14.5714285714286
12.6530612244898 14.5352029092725
12.7687403275011 14.1836734693878
12.8571428571429 13.9201549174901
12.899738652587 13.7959183673469
13.0359744900088 13.4081632653061
13.0612244897959 13.3383659520541
13.1778381494809 13.0204081632653
13.265306122449 12.7870160943593
13.3236523396055 12.6326530612245
13.469387755102 12.2554486629072
13.4734817132044 12.2448979591837
13.6287825144799 11.8571428571429
13.6734693877551 11.7483351374143
13.7879256159127 11.469387755102
13.8775510204082 11.2563388411715
13.9507877250329 11.0816326530612
14.0816326530612 10.7774679054771
14.1173920886245 10.6938775510204
14.2857142857143 10.3109161872726
14.2878062117194 10.3061224489796
14.4630070543274 9.91836734693877
14.4897959183673 9.86094195517438
14.6423473251458 9.53061224489796
14.6938775510204 9.42273912592374
14.8260905464382 9.14285714285714
14.8979591836735 8.99619908819252
15.0146784960131 8.75510204081633
15.1020408163265 8.5817610236992
15.2086908993264 8.36734693877551
15.3061224489796 8.1799956487939
15.4088705882977 7.97959183673469
15.5102040816327 7.79152355189952
15.6161538713873 7.59183673469388
15.7142857142857 7.41691934226185
15.8317092304843 7.20408163265306
15.9183673469388 7.05659900307873
16.0569888602248 6.81632653061224
16.1224489795918 6.71069061344543
16.2938002754165 6.42857142857143
16.3265306122449 6.37889258605448
16.530612244898 6.06314154998087
16.5447991014273 6.04081632653061
16.734693877551 5.769125388698
16.8137020266789 5.6530612244898
16.9387755102041 5.48828601233282
17.1031495363166 5.26530612244898
17.1428571428571 5.21775022635828
17.3469387755102 4.96859828133624
17.4194395352864 4.87755102040816
17.5510204081633 4.73481483185295
17.7551020408163 4.5052537265258
17.7685856430911 4.48979591836735
17.9591836734694 4.30579280488272
18.1607482916111 4.10204081632653
18.1632653061224 4.09994791088875
18.3673469387755 3.92827690957187
18.5714285714286 3.74831817333299
18.6091143682181 3.71428571428571
18.7755102040816 3.59541354557189
18.9795918367347 3.44405955280408
19.1312872124678 3.3265306122449
19.1836734693878 3.29552463943718
19.3877551020408 3.17277005091408
19.5918367346939 3.04398448919545
19.7501333511838 2.93877551020408
19.7959183673469 2.91649459331432
};
\addplot [draw=none, fill=white!74.9019607843137!black]
table{%
x  y
19.5918367346939 2.51167527405787
19.7959183673469 2.4321915647003
20 2.34973817357248
20 2.55102040816327
20 2.81529637859774
19.7959183673469 2.91649459331432
19.7501333511838 2.93877551020408
19.5918367346939 3.04398448919545
19.3877551020408 3.17277005091408
19.1836734693878 3.29552463943718
19.1312872124678 3.3265306122449
18.9795918367347 3.44405955280408
18.7755102040816 3.59541354557189
18.6091143682181 3.71428571428571
18.5714285714286 3.74831817333299
18.3673469387755 3.92827690957187
18.1632653061224 4.09994791088875
18.1607482916111 4.10204081632653
17.9591836734694 4.30579280488272
17.7685856430911 4.48979591836735
17.7551020408163 4.5052537265258
17.5510204081633 4.73481483185295
17.4194395352864 4.87755102040816
17.3469387755102 4.96859828133624
17.1428571428571 5.21775022635828
17.1031495363166 5.26530612244898
16.9387755102041 5.48828601233282
16.8137020266789 5.65306122448979
16.734693877551 5.769125388698
16.5447991014273 6.04081632653061
16.530612244898 6.06314154998087
16.3265306122449 6.37889258605448
16.2938002754165 6.42857142857143
16.1224489795918 6.71069061344543
16.0569888602248 6.81632653061224
15.9183673469388 7.05659900307873
15.8317092304843 7.20408163265306
15.7142857142857 7.41691934226185
15.6161538713873 7.59183673469388
15.5102040816327 7.79152355189952
15.4088705882977 7.97959183673469
15.3061224489796 8.1799956487939
15.2086908993264 8.36734693877551
15.1020408163265 8.5817610236992
15.0146784960131 8.75510204081633
14.8979591836735 8.99619908819252
14.8260905464382 9.14285714285714
14.6938775510204 9.42273912592374
14.6423473251458 9.53061224489796
14.4897959183673 9.86094195517438
14.4630070543274 9.91836734693877
14.2878062117194 10.3061224489796
14.2857142857143 10.3109161872726
14.1173920886245 10.6938775510204
14.0816326530612 10.7774679054771
13.9507877250329 11.0816326530612
13.8775510204082 11.2563388411715
13.7879256159127 11.469387755102
13.6734693877551 11.7483351374143
13.6287825144799 11.8571428571429
13.4734817132044 12.2448979591837
13.469387755102 12.2554486629072
13.3236523396055 12.6326530612245
13.265306122449 12.7870160943593
13.1778381494809 13.0204081632653
13.0612244897959 13.3383659520541
13.0359744900088 13.4081632653061
12.899738652587 13.7959183673469
12.8571428571429 13.9201549174901
12.7687403275011 14.1836734693878
12.6530612244898 14.5352029092725
12.6414792941952 14.5714285714286
12.5213721413703 14.9591836734694
12.4489795918367 15.1974327429285
12.4052583843985 15.3469387755102
12.2946863484691 15.734693877551
12.2448979591837 15.9133023916733
12.1893182310124 16.1224489795918
12.0885343081422 16.5102040816327
12.0408163265306 16.6979200025441
11.9927780051409 16.8979591836735
11.9017240510377 17.2857142857143
11.8367346938776 17.5674654757352
11.8138238390842 17.6734693877551
11.7321549490427 18.0612244897959
11.651571357835 18.4489795918367
11.6326530612245 18.5422600395467
11.5771238557306 18.8367346938775
11.5052246928718 19.2244897959184
11.4342098309093 19.6122448979592
11.4285714285714 19.6438123249337
11.3697594958827 20
11.2244897959184 20
11.0204081632653 20
10.8163265306122 20
10.6122448979592 20
10.4081632653061 20
10.3253350370199 20
10.3766485340091 19.6122448979592
10.4081632653061 19.3785760346969
10.4303695690892 19.2244897959184
10.4874128092919 18.8367346938775
10.5453391648595 18.4489795918367
10.6042352867217 18.0612244897959
10.6122448979592 18.0098694879528
10.668464976885 17.6734693877551
10.7344784379528 17.2857142857143
10.8017220505803 16.8979591836735
10.8163265306122 16.8160798507085
10.8748906674389 16.5102040816327
10.9507545913029 16.1224489795918
11.0204081632653 15.774152288759
11.0288823569533 15.734693877551
11.1147832871513 15.3469387755102
11.2025680985279 14.9591836734694
11.2244897959184 14.8654270847063
11.2980769993546 14.5714285714286
11.3976585077385 14.1836734693878
11.4285714285714 14.0672272216387
11.5053042026017 13.7959183673469
11.6180250910498 13.4081632653061
11.6326530612245 13.3596712204878
11.7408908162025 13.0204081632653
11.8367346938776 12.7286526631102
11.8698329661316 12.6326530612245
12.0082100743906 12.2448979591837
12.0408163265306 12.1567965536402
12.1559686435135 11.8571428571429
12.2448979591837 11.6333271761605
12.3119030729569 11.469387755102
12.4489795918367 11.1452746757955
12.4764157964252 11.0816326530612
12.6495897566906 10.6938775510204
12.6530612244898 10.6864119964363
12.8314279995151 10.3061224489796
12.8571428571429 10.2533172524837
13.0204895855242 9.91836734693877
13.0612244897959 9.83798739571289
13.2163012051507 9.53061224489796
13.265306122449 9.4372440419067
13.4184183735198 9.14285714285714
13.469387755102 9.04878768082741
13.6265301094386 8.75510204081633
13.6734693877551 8.67104375540601
13.8405324882735 8.36734693877551
13.8775510204082 8.30301164493815
14.0605775994602 7.97959183673469
14.0816326530612 7.94412307134591
14.2857142857143 7.59431064390849
14.2871378107088 7.59183673469388
14.4897959183673 7.25738352462553
14.5215334039456 7.20408163265306
14.6938775510204 6.93051044023896
14.7645632381787 6.81632653061225
14.8979591836735 6.61381585464812
15.0178972219185 6.42857142857143
15.1020408163265 6.30726494753849
15.2838029371944 6.04081632653061
15.3061224489796 6.01052206138747
15.5102040816327 5.73124865267842
15.5665865262863 5.6530612244898
15.7142857142857 5.46581119193881
15.8700026648581 5.26530612244898
15.9183673469388 5.20903966669809
16.1224489795918 4.97025222323169
16.2008033364674 4.87755102040816
16.3265306122449 4.74557779094802
16.530612244898 4.52889496513914
16.5671935547689 4.48979591836735
16.734693877551 4.33430798014658
16.9387755102041 4.14238474945223
16.9813132532126 4.10204081632653
17.1428571428571 3.97214506177684
17.3469387755102 3.80620785495194
17.4583567056582 3.71428571428571
17.5510204081633 3.65113926844329
17.7551020408163 3.51103682014454
17.9591836734694 3.36837486273283
18.0180281673586 3.3265306122449
18.1632653061224 3.24409449007023
18.3673469387755 3.12690663979572
18.5714285714286 3.00724450139796
18.6854208843219 2.93877551020408
18.7755102040816 2.89698166413491
18.9795918367347 2.8012407016053
19.1836734693878 2.70350996516379
19.3877551020408 2.60264389832775
19.4887337906096 2.55102040816327
19.5918367346939 2.51167527405787
};
\addplot [draw=none, fill=white!65.0980392156863!black]
table{%
x  y
19.5918367346939 2.14061831790798
19.7959183673469 2.07768432294885
20 2.01285365104991
20 2.16326530612245
20 2.34973817357248
19.7959183673469 2.4321915647003
19.5918367346939 2.51167527405787
19.4887337906096 2.55102040816327
19.3877551020408 2.60264389832775
19.1836734693878 2.70350996516379
18.9795918367347 2.8012407016053
18.7755102040816 2.89698166413491
18.6854208843219 2.93877551020408
18.5714285714286 3.00724450139796
18.3673469387755 3.12690663979572
18.1632653061224 3.24409449007023
18.0180281673586 3.3265306122449
17.9591836734694 3.36837486273283
17.7551020408163 3.51103682014454
17.5510204081633 3.65113926844329
17.4583567056582 3.71428571428571
17.3469387755102 3.80620785495194
17.1428571428571 3.97214506177684
16.9813132532126 4.10204081632653
16.9387755102041 4.14238474945223
16.734693877551 4.33430798014658
16.5671935547689 4.48979591836735
16.530612244898 4.52889496513914
16.3265306122449 4.74557779094802
16.2008033364674 4.87755102040816
16.1224489795918 4.97025222323169
15.9183673469388 5.20903966669809
15.8700026648581 5.26530612244898
15.7142857142857 5.46581119193881
15.5665865262863 5.6530612244898
15.5102040816327 5.73124865267842
15.3061224489796 6.01052206138747
15.2838029371944 6.04081632653061
15.1020408163265 6.30726494753849
15.0178972219185 6.42857142857143
14.8979591836735 6.61381585464812
14.7645632381787 6.81632653061224
14.6938775510204 6.93051044023896
14.5215334039456 7.20408163265306
14.4897959183673 7.25738352462553
14.2871378107088 7.59183673469388
14.2857142857143 7.59431064390849
14.0816326530612 7.94412307134591
14.0605775994602 7.97959183673469
13.8775510204082 8.30301164493815
13.8405324882735 8.36734693877551
13.6734693877551 8.67104375540601
13.6265301094386 8.75510204081633
13.469387755102 9.04878768082741
13.4184183735198 9.14285714285714
13.265306122449 9.4372440419067
13.2163012051507 9.53061224489796
13.0612244897959 9.83798739571289
13.0204895855242 9.91836734693877
12.8571428571429 10.2533172524837
12.8314279995151 10.3061224489796
12.6530612244898 10.6864119964363
12.6495897566906 10.6938775510204
12.4764157964252 11.0816326530612
12.4489795918367 11.1452746757955
12.3119030729569 11.469387755102
12.2448979591837 11.6333271761605
12.1559686435135 11.8571428571429
12.0408163265306 12.1567965536402
12.0082100743906 12.2448979591837
11.8698329661316 12.6326530612245
11.8367346938776 12.7286526631102
11.7408908162025 13.0204081632653
11.6326530612245 13.3596712204878
11.6180250910498 13.4081632653061
11.5053042026017 13.7959183673469
11.4285714285714 14.0672272216387
11.3976585077385 14.1836734693878
11.2980769993546 14.5714285714286
11.2244897959184 14.8654270847063
11.2025680985279 14.9591836734694
11.1147832871513 15.3469387755102
11.0288823569533 15.734693877551
11.0204081632653 15.774152288759
10.9507545913029 16.1224489795918
10.8748906674389 16.5102040816327
10.8163265306122 16.8160798507085
10.8017220505803 16.8979591836735
10.7344784379528 17.2857142857143
10.668464976885 17.6734693877551
10.6122448979592 18.0098694879528
10.6042352867217 18.0612244897959
10.5453391648595 18.4489795918367
10.4874128092919 18.8367346938775
10.4303695690892 19.2244897959184
10.4081632653061 19.3785760346969
10.3766485340091 19.6122448979592
10.3253350370199 20
10.2040816326531 20
10 20
10 19.6122448979592
10 19.2244897959184
10 18.8367346938775
10 18.4489795918367
10 18.0612244897959
10 17.6734693877551
10 17.2857142857143
10 16.8979591836735
10 16.8488176883784
10.0542901643725 16.5102040816327
10.1179400320807 16.1224489795918
10.1831776525173 15.734693877551
10.2040816326531 15.6141144774716
10.2526483679995 15.3469387755102
10.3250983237216 14.9591836734694
10.3995839092237 14.5714285714286
10.4081632653061 14.5282718228362
10.4803962626213 14.1836734693878
10.5641393562695 13.7959183673469
10.6122448979592 13.5806316244378
10.6530772800491 13.4081632653061
10.7481276085536 13.0204081632653
10.8163265306122 12.7518963669031
10.8484949360166 12.6326530612245
10.9571603896388 12.2448979591837
11.0204081632653 12.0277699551875
11.0731786320263 11.8571428571429
11.1979438450593 11.469387755102
11.2244897959184 11.3903881925965
11.3342570993469 11.0816326530612
11.4285714285714 10.8271692384844
11.4805782173177 10.6938775510204
11.6326530612245 10.3204004018857
11.6387284311227 10.3061224489796
11.8113383998036 9.91836734693877
11.8367346938776 9.86387671715836
11.9972667724538 9.53061224489796
12.0408163265306 9.44428595512942
12.1964254078944 9.14285714285714
12.2448979591837 9.05326078345649
12.4083485110053 8.75510204081633
12.4489795918367 8.68443152196566
12.6320429405376 8.36734693877551
12.6530612244898 8.33266677852804
12.8571428571429 7.99499918635134
12.8663957818114 7.97959183673469
13.0612244897959 7.67076093883979
13.1103638773744 7.59183673469388
13.265306122449 7.35538654278635
13.362823570786 7.20408163265306
13.469387755102 7.04736592356046
13.6235870392759 6.81632653061224
13.6734693877551 6.74570264518356
13.8775510204082 6.45153560311151
13.8932529816727 6.42857142857143
14.0816326530612 6.16948886425387
14.1734337075224 6.04081632653061
14.2857142857143 5.89358180669881
14.4659618987229 5.6530612244898
14.4897959183673 5.62349119041698
14.6938775510204 5.36828226770389
14.7752202852069 5.26530612244898
14.8979591836735 5.12234521874696
15.1020408163265 4.88215184274634
15.1059462715998 4.87755102040816
15.3061224489796 4.66352359159503
15.4676863173711 4.48979591836735
15.5102040816327 4.4487916751489
15.7142857142857 4.25337912116852
15.8726067809307 4.10204081632653
15.9183673469388 4.06355862696897
16.1224489795918 3.89402408195135
16.3265306122449 3.72579653409848
16.3406006787057 3.71428571428571
16.530612244898 3.58138439921401
16.734693877551 3.44064291795245
16.9015547461052 3.3265306122449
16.9387755102041 3.30525606614362
17.1428571428571 3.18992759560641
17.3469387755102 3.07606264499289
17.5510204081633 2.96281277341361
17.5941157686446 2.93877551020408
17.7551020408163 2.86626156864034
17.9591836734694 2.77500031790199
18.1632653061224 2.68400390320512
18.3673469387755 2.59257339032068
18.4587141809523 2.55102040816327
18.5714285714286 2.51064591064078
18.7755102040816 2.43729591645554
18.9795918367347 2.36368803144664
19.1836734693878 2.28918850501237
19.3877551020408 2.21300777367309
19.5166074515595 2.16326530612245
19.5918367346939 2.14061831790798
};
\addplot [draw=none, fill=white!54.9019607843137!black]
table{%
x  y
20 1.73064678617334
20 1.77551020408163
20 2.01285365104991
19.7959183673469 2.07768432294885
19.5918367346939 2.14061831790798
19.5166074515595 2.16326530612245
19.3877551020408 2.21300777367309
19.1836734693878 2.28918850501237
18.9795918367347 2.36368803144664
18.7755102040816 2.43729591645554
18.5714285714286 2.51064591064078
18.4587141809523 2.55102040816327
18.3673469387755 2.59257339032068
18.1632653061224 2.68400390320512
17.9591836734694 2.77500031790199
17.7551020408163 2.86626156864034
17.5941157686446 2.93877551020408
17.5510204081633 2.96281277341361
17.3469387755102 3.07606264499289
17.1428571428571 3.18992759560641
16.9387755102041 3.30525606614362
16.9015547461052 3.3265306122449
16.734693877551 3.44064291795245
16.530612244898 3.58138439921401
16.3406006787057 3.71428571428571
16.3265306122449 3.72579653409848
16.1224489795918 3.89402408195135
15.9183673469388 4.06355862696897
15.8726067809307 4.10204081632653
15.7142857142857 4.25337912116852
15.5102040816327 4.4487916751489
15.4676863173711 4.48979591836735
15.3061224489796 4.66352359159503
15.1059462715998 4.87755102040816
15.1020408163265 4.88215184274634
14.8979591836735 5.12234521874696
14.7752202852069 5.26530612244898
14.6938775510204 5.36828226770389
14.4897959183673 5.62349119041698
14.4659618987229 5.6530612244898
14.2857142857143 5.89358180669881
14.1734337075224 6.04081632653061
14.0816326530612 6.16948886425387
13.8932529816727 6.42857142857143
13.8775510204082 6.45153560311151
13.6734693877551 6.74570264518356
13.6235870392759 6.81632653061224
13.469387755102 7.04736592356046
13.362823570786 7.20408163265306
13.265306122449 7.35538654278635
13.1103638773744 7.59183673469388
13.0612244897959 7.67076093883979
12.8663957818114 7.97959183673469
12.8571428571429 7.99499918635134
12.6530612244898 8.33266677852804
12.6320429405376 8.36734693877551
12.4489795918367 8.68443152196566
12.4083485110053 8.75510204081633
12.2448979591837 9.05326078345649
12.1964254078944 9.14285714285714
12.0408163265306 9.44428595512942
11.9972667724538 9.53061224489796
11.8367346938776 9.86387671715836
11.8113383998036 9.91836734693877
11.6387284311227 10.3061224489796
11.6326530612245 10.3204004018857
11.4805782173177 10.6938775510204
11.4285714285714 10.8271692384844
11.3342570993469 11.0816326530612
11.2244897959184 11.3903881925965
11.1979438450593 11.469387755102
11.0731786320263 11.8571428571429
11.0204081632653 12.0277699551875
10.9571603896388 12.2448979591837
10.8484949360166 12.6326530612245
10.8163265306122 12.7518963669031
10.7481276085536 13.0204081632653
10.6530772800491 13.4081632653061
10.6122448979592 13.5806316244378
10.5641393562695 13.7959183673469
10.4803962626213 14.1836734693878
10.4081632653061 14.5282718228362
10.3995839092237 14.5714285714286
10.3250983237216 14.9591836734694
10.2526483679995 15.3469387755102
10.2040816326531 15.6141144774716
10.1831776525173 15.734693877551
10.1179400320807 16.1224489795918
10.0542901643725 16.5102040816327
10 16.8488176883784
10 16.5102040816327
10 16.1224489795918
10 15.734693877551
10 15.3469387755102
10 14.9591836734694
10 14.5714285714286
10 14.1836734693878
10 13.7959183673469
10 13.4081632653061
10 13.0204081632653
10 12.9748168372585
10.0760233292418 12.6326530612245
10.1657358839705 12.2448979591837
10.2040816326531 12.0860335052152
10.2614397177553 11.8571428571429
10.3629062408371 11.469387755102
10.4081632653061 11.3041676837021
10.4719008030354 11.0816326530612
10.5883038088651 10.6938775510204
10.6122448979592 10.6179041206898
10.7154870188802 10.3061224489796
10.8163265306122 10.0166921418552
10.8524337005945 9.91836734693877
11.0022761719387 9.53061224489796
11.0204081632653 9.48610645815797
11.1676361675712 9.14285714285714
11.2244897959184 9.01726929801022
11.3490162374516 8.75510204081633
11.4285714285714 8.59643938239283
11.5483451471012 8.36734693877551
11.6326530612245 8.2146053351799
11.7667781783791 7.97959183673469
11.8367346938776 7.86350038200898
12.004380594596 7.59183673469388
12.0408163265306 7.53593046670038
12.2448979591837 7.227243018234
12.2603156574127 7.20408163265306
12.4489795918367 6.93567265236086
12.5327092151357 6.81632653061224
12.6530612244898 6.6540534925596
12.8189376845818 6.42857142857143
12.8571428571429 6.37952476925171
13.0612244897959 6.1148042846471
13.1174629462908 6.04081632653061
13.265306122449 5.85768396354239
13.4277579195806 5.6530612244898
13.469387755102 5.60385480797761
13.6734693877551 5.35942294982885
13.7508734920991 5.26530612244898
13.8775510204082 5.12166724948483
14.0816326530612 4.88715412204505
14.0899185549225 4.87755102040816
14.2857142857143 4.66777044575367
14.4504979928432 4.48979591836735
14.4897959183673 4.45085151074689
14.6938775510204 4.24887237582867
14.8423894698137 4.10204081632653
14.8979591836735 4.05231604767795
15.1020408163265 3.87159784459434
15.2818633188925 3.71428571428571
15.3061224489796 3.69539950018732
15.5102040816327 3.5395637124653
15.7142857142857 3.3876937555775
15.7982342778827 3.3265306122449
15.9183673469388 3.25065196431368
16.1224489795918 3.12591292137659
16.3265306122449 3.00554567425863
16.4428047821714 2.93877551020408
16.530612244898 2.89615685902571
16.734693877551 2.8002087666228
16.9387755102041 2.7078003433126
17.1428571428571 2.61822062432634
17.2990603322173 2.55102040816327
17.3469387755102 2.53399303453193
17.5510204081633 2.46243793307645
17.7551020408163 2.39272230153185
17.9591836734694 2.32432579076611
18.1632653061224 2.25674684308167
18.3673469387755 2.18948321966313
18.4458838031062 2.16326530612245
18.5714285714286 2.12899424243058
18.7755102040816 2.07325840516442
18.9795918367347 2.0176826786325
19.1836734693878 1.96183723125989
19.3877551020408 1.9051921373582
19.5918367346939 1.84707098543061
19.7959183673469 1.78659159074708
19.8311692567052 1.77551020408163
20 1.73064678617334
};
\addplot [draw=none, fill=white!45.0980392156863!black]
table{%
x  y
18.7755102040816 1.75972558079416
18.9795918367347 1.71609795143537
19.1836734693878 1.67223277864012
19.3877551020408 1.62772621200791
19.5918367346939 1.58207569745058
19.7959183673469 1.53463926326521
20 1.48458274265078
20 1.73064678617334
19.8311692567052 1.77551020408163
19.7959183673469 1.78659159074708
19.5918367346939 1.84707098543061
19.3877551020408 1.9051921373582
19.1836734693878 1.96183723125989
18.9795918367347 2.0176826786325
18.7755102040816 2.07325840516442
18.5714285714286 2.12899424243058
18.4458838031062 2.16326530612245
18.3673469387755 2.18948321966313
18.1632653061224 2.25674684308167
17.9591836734694 2.32432579076611
17.7551020408163 2.39272230153185
17.5510204081633 2.46243793307645
17.3469387755102 2.53399303453193
17.2990603322173 2.55102040816327
17.1428571428571 2.61822062432634
16.9387755102041 2.7078003433126
16.734693877551 2.8002087666228
16.530612244898 2.89615685902571
16.4428047821714 2.93877551020408
16.3265306122449 3.00554567425863
16.1224489795918 3.12591292137659
15.9183673469388 3.25065196431368
15.7982342778827 3.3265306122449
15.7142857142857 3.3876937555775
15.5102040816327 3.5395637124653
15.3061224489796 3.69539950018732
15.2818633188925 3.71428571428571
15.1020408163265 3.87159784459434
14.8979591836735 4.05231604767795
14.8423894698137 4.10204081632653
14.6938775510204 4.24887237582867
14.4897959183673 4.45085151074689
14.4504979928432 4.48979591836735
14.2857142857143 4.66777044575367
14.0899185549225 4.87755102040816
14.0816326530612 4.88715412204505
13.8775510204082 5.12166724948483
13.7508734920991 5.26530612244898
13.6734693877551 5.35942294982885
13.469387755102 5.60385480797761
13.4277579195806 5.6530612244898
13.265306122449 5.85768396354239
13.1174629462908 6.04081632653061
13.0612244897959 6.1148042846471
12.8571428571429 6.37952476925171
12.8189376845818 6.42857142857143
12.6530612244898 6.6540534925596
12.5327092151357 6.81632653061225
12.4489795918367 6.93567265236086
12.2603156574127 7.20408163265306
12.2448979591837 7.227243018234
12.0408163265306 7.53593046670038
12.004380594596 7.59183673469388
11.8367346938776 7.86350038200898
11.7667781783791 7.97959183673469
11.6326530612245 8.2146053351799
11.5483451471012 8.36734693877551
11.4285714285714 8.59643938239283
11.3490162374516 8.75510204081633
11.2244897959184 9.01726929801022
11.1676361675712 9.14285714285714
11.0204081632653 9.48610645815797
11.0022761719387 9.53061224489796
10.8524337005945 9.91836734693877
10.8163265306122 10.0166921418552
10.7154870188802 10.3061224489796
10.6122448979592 10.6179041206898
10.5883038088651 10.6938775510204
10.4719008030354 11.0816326530612
10.4081632653061 11.3041676837021
10.3629062408371 11.469387755102
10.2614397177553 11.8571428571429
10.2040816326531 12.0860335052152
10.1657358839705 12.2448979591837
10.0760233292418 12.6326530612245
10 12.9748168372585
10 12.6326530612245
10 12.2448979591837
10 11.8571428571429
10 11.469387755102
10 11.0816326530612
10 10.6938775510204
10 10.3061224489796
10 10.135209548611
10.0651840471238 9.91836734693877
10.1887652819779 9.53061224489796
10.2040816326531 9.48510393714177
10.3231392686704 9.14285714285714
10.4081632653061 8.91327619979619
10.4692111085832 8.75510204081633
10.6122448979592 8.40687327671485
10.6292607329525 8.36734693877551
10.8066448500376 7.97959183673469
10.8163265306122 7.95969163594328
11.0043963339349 7.59183673469388
11.0204081632653 7.56241439627383
11.2244897959184 7.20523835462882
11.2251799896907 7.20408163265306
11.4285714285714 6.88291644282168
11.4723831302681 6.81632653061224
11.6326530612245 6.5866191816315
11.7462747921873 6.42857142857143
11.8367346938775 6.30983143084834
12.0408163265306 6.04751414919564
12.046070181461 6.04081632653061
12.2448979591837 5.8014958096904
12.3688605100168 5.6530612244898
12.4489795918367 5.56252862289115
12.6530612244898 5.33152760570549
12.7111004567256 5.26530612244898
12.8571428571429 5.10844093914769
13.0612244897959 4.88772201987513
13.070504769117 4.87755102040816
13.265306122449 4.67746517507607
13.4467002091273 4.48979591836735
13.469387755102 4.46787455288342
13.6734693877551 4.26905336793251
13.844351651604 4.10204081632653
13.8775510204082 4.07198317749376
14.0816326530612 3.88692050201147
14.2732206927291 3.71428571428571
14.2857142857143 3.70396871627683
14.4897959183673 3.53656608415466
14.6938775510204 3.37256951927751
14.7519648091848 3.3265306122449
14.8979591836735 3.22258556188368
15.1020408163265 3.08196713245405
15.3061224489796 2.94669785432623
15.3183802990202 2.93877551020408
15.5102040816327 2.82993486143875
15.7142857142857 2.72036970270431
15.9183673469388 2.61677146674906
16.0541471129775 2.55102040816327
16.1224489795918 2.5224952774307
16.3265306122449 2.44124145591084
16.530612244898 2.36495962541521
16.734693877551 2.2929535078751
16.9387755102041 2.22454309199545
17.1294780319426 2.16326530612245
17.1428571428571 2.15958560908143
17.3469387755102 2.10453894414395
17.5510204081633 2.05173290544478
17.7551020408163 2.0006938786736
17.9591836734694 1.95099796305823
18.1632653061224 1.90225764394405
18.3673469387755 1.85410522464462
18.5714285714286 1.80617290063193
18.7003837548352 1.77551020408163
18.7755102040816 1.75972558079416
};
\addplot [draw=none, fill=white!34.9019607843137!black]
table{%
x  y
19.3877551020408 1.36436521536149
19.5918367346939 1.32310066350823
19.7959183673469 1.27969403676506
20 1.23343646931997
20 1.38775510204082
20 1.48458274265078
19.7959183673469 1.53463926326521
19.5918367346939 1.58207569745058
19.3877551020408 1.62772621200791
19.1836734693878 1.67223277864012
18.9795918367347 1.71609795143537
18.7755102040816 1.75972558079416
18.7003837548352 1.77551020408163
18.5714285714286 1.80617290063193
18.3673469387755 1.85410522464462
18.1632653061224 1.90225764394405
17.9591836734694 1.95099796305823
17.7551020408163 2.0006938786736
17.5510204081633 2.05173290544478
17.3469387755102 2.10453894414395
17.1428571428571 2.15958560908143
17.1294780319426 2.16326530612245
16.9387755102041 2.22454309199545
16.734693877551 2.2929535078751
16.530612244898 2.36495962541521
16.3265306122449 2.44124145591084
16.1224489795918 2.5224952774307
16.0541471129775 2.55102040816327
15.9183673469388 2.61677146674906
15.7142857142857 2.72036970270431
15.5102040816327 2.82993486143875
15.3183802990202 2.93877551020408
15.3061224489796 2.94669785432623
15.1020408163265 3.08196713245405
14.8979591836735 3.22258556188368
14.7519648091848 3.3265306122449
14.6938775510204 3.37256951927751
14.4897959183673 3.53656608415466
14.2857142857143 3.70396871627683
14.2732206927291 3.71428571428571
14.0816326530612 3.88692050201147
13.8775510204082 4.07198317749376
13.844351651604 4.10204081632653
13.6734693877551 4.26905336793251
13.469387755102 4.46787455288342
13.4467002091273 4.48979591836735
13.265306122449 4.67746517507607
13.070504769117 4.87755102040816
13.0612244897959 4.88772201987513
12.8571428571429 5.10844093914769
12.7111004567256 5.26530612244898
12.6530612244898 5.33152760570549
12.4489795918367 5.56252862289115
12.3688605100168 5.6530612244898
12.2448979591837 5.8014958096904
12.046070181461 6.04081632653061
12.0408163265306 6.04751414919564
11.8367346938775 6.30983143084834
11.7462747921873 6.42857142857143
11.6326530612245 6.58661918163149
11.4723831302681 6.81632653061224
11.4285714285714 6.88291644282168
11.2251799896907 7.20408163265306
11.2244897959184 7.20523835462882
11.0204081632653 7.56241439627383
11.0043963339348 7.59183673469388
10.8163265306122 7.95969163594328
10.8066448500376 7.97959183673469
10.6292607329525 8.36734693877551
10.6122448979592 8.40687327671485
10.4692111085832 8.75510204081633
10.4081632653061 8.91327619979619
10.3231392686704 9.14285714285714
10.2040816326531 9.48510393714177
10.1887652819779 9.53061224489796
10.0651840471238 9.91836734693877
10 10.135209548611
10 9.91836734693877
10 9.53061224489796
10 9.14285714285714
10 8.75510204081633
10 8.36734693877551
10 7.97959183673469
10 7.86805880374237
10.1160846442727 7.59183673469388
10.2040816326531 7.39703936401813
10.2945460716477 7.20408163265306
10.4081632653061 6.97825284009227
10.4934713024898 6.81632653061224
10.6122448979592 6.60579576274758
10.717587856407 6.42857142857143
10.8163265306122 6.27305165101764
10.9721326104189 6.04081632653061
11.0204081632653 5.97324949271918
11.2244897959184 5.7012781889198
11.262176489473 5.65306122448979
11.4285714285714 5.45226186200393
11.5903939172568 5.26530612244898
11.6326530612245 5.21913173668389
11.8367346938776 5.00118291562013
11.9551511542872 4.87755102040816
12.0408163265306 4.79280928033444
12.2448979591837 4.59332900802726
12.3514598034835 4.48979591836735
12.4489795918367 4.4000713809691
12.6530612244898 4.21285658718064
12.7738401236947 4.10204081632653
12.8571428571429 4.02986631983484
13.0612244897959 3.85274585864043
13.2213187569162 3.71428571428571
13.265306122449 3.67853721350087
13.469387755102 3.51216037118379
13.6734693877551 3.34823198065052
13.7003737718385 3.3265306122449
13.8775510204082 3.19394237224049
14.0816326530612 3.04492458696723
14.230172487934 2.93877551020408
14.2857142857143 2.90228905867575
14.4897959183673 2.77131888305807
14.6938775510204 2.64654158676492
14.8572370775229 2.55102040816327
14.8979591836735 2.52947766355092
15.1020408163265 2.42593949865811
15.3061224489796 2.32972887773994
15.5102040816327 2.24042205823906
15.6998038532214 2.16326530612245
15.7142857142857 2.15799698756304
15.9183673469388 2.08763522939428
16.1224489795918 2.02322538461785
16.3265306122449 1.96402100167118
16.530612244898 1.90930618014384
16.734693877551 1.85840455537593
16.9387755102041 1.81068636800996
17.0967797819186 1.77551020408163
17.1428571428571 1.76613028700091
17.3469387755102 1.72572719348979
17.5510204081633 1.68717496715048
17.7551020408163 1.65006094850791
17.9591836734694 1.61402448445137
18.1632653061224 1.57874356137792
18.3673469387755 1.54392000132461
18.5714285714286 1.50926309746579
18.7755102040816 1.47447137570761
18.9795918367347 1.43921190932492
19.1836734693878 1.40309625103469
19.2659768544206 1.38775510204082
19.3877551020408 1.36436521536149
};
\addplot [draw=none, fill=white!24.7058823529412!black]
table{%
x  y
17.3469387755102 1.36491989146147
17.5510204081633 1.33639581304746
17.7551020408163 1.30885543759046
17.9591836734694 1.28196730965835
18.1632653061224 1.25543082421252
18.3673469387755 1.22896240127817
18.5714285714286 1.2022812162355
18.7755102040816 1.17509424408969
18.9795918367347 1.1470802843001
19.1836734693878 1.11787252221606
19.3877551020408 1.08703904657865
19.5918367346939 1.05406056570607
19.7959183673469 1.01830431421839
19.889576888472 1
20 1
20 1.23343646931997
19.7959183673469 1.27969403676506
19.5918367346939 1.32310066350823
19.3877551020408 1.36436521536149
19.2659768544206 1.38775510204082
19.1836734693878 1.40309625103469
18.9795918367347 1.43921190932492
18.7755102040816 1.47447137570761
18.5714285714286 1.50926309746579
18.3673469387755 1.54392000132461
18.1632653061224 1.57874356137792
17.9591836734694 1.61402448445137
17.7551020408163 1.65006094850791
17.5510204081633 1.68717496715048
17.3469387755102 1.72572719348979
17.1428571428571 1.76613028700091
17.0967797819186 1.77551020408163
16.9387755102041 1.81068636800996
16.734693877551 1.85840455537593
16.530612244898 1.90930618014384
16.3265306122449 1.96402100167118
16.1224489795918 2.02322538461785
15.9183673469388 2.08763522939428
15.7142857142857 2.15799698756304
15.6998038532214 2.16326530612245
15.5102040816327 2.24042205823906
15.3061224489796 2.32972887773994
15.1020408163265 2.42593949865811
14.8979591836735 2.52947766355092
14.8572370775229 2.55102040816327
14.6938775510204 2.64654158676492
14.4897959183673 2.77131888305807
14.2857142857143 2.90228905867575
14.230172487934 2.93877551020408
14.0816326530612 3.04492458696723
13.8775510204082 3.19394237224049
13.7003737718385 3.3265306122449
13.6734693877551 3.34823198065052
13.469387755102 3.51216037118379
13.265306122449 3.67853721350087
13.2213187569162 3.71428571428571
13.0612244897959 3.85274585864043
12.8571428571429 4.02986631983484
12.7738401236947 4.10204081632653
12.6530612244898 4.21285658718064
12.4489795918367 4.4000713809691
12.3514598034835 4.48979591836735
12.2448979591837 4.59332900802726
12.0408163265306 4.79280928033444
11.9551511542872 4.87755102040816
11.8367346938776 5.00118291562013
11.6326530612245 5.21913173668389
11.5903939172568 5.26530612244898
11.4285714285714 5.45226186200393
11.262176489473 5.6530612244898
11.2244897959184 5.7012781889198
11.0204081632653 5.97324949271918
10.9721326104189 6.04081632653061
10.8163265306122 6.27305165101764
10.717587856407 6.42857142857143
10.6122448979592 6.60579576274758
10.4934713024898 6.81632653061224
10.4081632653061 6.97825284009227
10.2945460716477 7.20408163265306
10.2040816326531 7.39703936401813
10.1160846442727 7.59183673469388
10 7.86805880374237
10 7.59183673469388
10 7.20408163265306
10 6.81632653061224
10 6.42857142857143
10 6.04081632653061
10 5.89443800646295
10.1459979755731 5.6530612244898
10.2040816326531 5.56351433093989
10.4081632653061 5.26694552121559
10.409343970528 5.26530612244898
10.6122448979592 5.00064528724127
10.7137116535961 4.87755102040816
10.8163265306122 4.75987713550018
11.0204081632653 4.5408458380634
11.0701943119109 4.48979591836735
11.2244897959184 4.33904763759237
11.4285714285714 4.15159312631346
11.4840602252633 4.10204081632653
11.6326530612245 3.974805058456
11.8367346938775 3.80707352151227
11.9522661415266 3.71428571428571
12.0408163265306 3.64581135193749
12.2448979591837 3.49029770775382
12.4489795918367 3.33940666904784
12.4661968666239 3.3265306122449
12.6530612244898 3.19234481572625
12.8571428571429 3.0496536287611
13.0184381641586 2.93877551020408
13.0612244897959 2.91053710017445
13.265306122449 2.77608481653501
13.469387755102 2.64650994804966
13.6236507622625 2.55102040816327
13.6734693877551 2.52162472468699
13.8775510204082 2.40330920314578
14.0816326530612 2.29154684622695
14.2857142857143 2.18608464626698
14.3311091889177 2.16326530612245
14.4897959183673 2.08817504736366
14.6938775510204 1.99852123412111
14.8979591836735 1.91649388258342
15.1020408163265 1.84162194319401
15.2996083182509 1.77551020408163
15.3061224489796 1.77342654763041
15.5102040816327 1.7120420874465
15.7142857142857 1.65720555775759
15.9183673469388 1.608092356385
16.1224489795918 1.56392744889186
16.3265306122449 1.52399072307078
16.530612244898 1.48762125907562
16.734693877551 1.4542197322832
16.9387755102041 1.42324852138141
17.1428571428571 1.39422941659187
17.1896345316784 1.38775510204082
17.3469387755102 1.36491989146147
};
\addplot [draw=none, fill=white!14.9019607843137!black]
table{%
x  y
14.6938775510204 1.35236505967073
14.8979591836735 1.29681037067453
15.1020408163265 1.24833934033397
15.3061224489796 1.20606475794087
15.5102040816327 1.16916171721031
15.7142857142857 1.1368705987336
15.9183673469388 1.10849875057279
16.1224489795918 1.08342063672702
16.3265306122449 1.06107634566818
16.530612244898 1.04096849760041
16.734693877551 1.02265771568845
16.9387755102041 1.00575690417664
17.0111799972469 1
17.1428571428571 1
17.3469387755102 1
17.5510204081633 1
17.7551020408163 1
17.9591836734694 1
18.1632653061224 1
18.3673469387755 1
18.5714285714286 1
18.7755102040816 1
18.9795918367347 1
19.1836734693878 1
19.3877551020408 1
19.5918367346939 1
19.7959183673469 1
19.889576888472 1
19.7959183673469 1.01830431421839
19.5918367346939 1.05406056570607
19.3877551020408 1.08703904657865
19.1836734693878 1.11787252221606
18.9795918367347 1.1470802843001
18.7755102040816 1.17509424408969
18.5714285714286 1.2022812162355
18.3673469387755 1.22896240127817
18.1632653061224 1.25543082421252
17.9591836734694 1.28196730965835
17.7551020408163 1.30885543759046
17.5510204081633 1.33639581304746
17.3469387755102 1.36491989146147
17.1896345316784 1.38775510204082
17.1428571428571 1.39422941659187
16.9387755102041 1.42324852138141
16.734693877551 1.4542197322832
16.530612244898 1.48762125907562
16.3265306122449 1.52399072307078
16.1224489795918 1.56392744889186
15.9183673469388 1.608092356385
15.7142857142857 1.65720555775759
15.5102040816327 1.7120420874465
15.3061224489796 1.77342654763041
15.2996083182509 1.77551020408163
15.1020408163265 1.84162194319401
14.8979591836735 1.91649388258342
14.6938775510204 1.99852123412111
14.4897959183673 2.08817504736366
14.3311091889177 2.16326530612245
14.2857142857143 2.18608464626698
14.0816326530612 2.29154684622695
13.8775510204082 2.40330920314578
13.6734693877551 2.52162472468699
13.6236507622625 2.55102040816327
13.469387755102 2.64650994804966
13.265306122449 2.77608481653501
13.0612244897959 2.91053710017445
13.0184381641586 2.93877551020408
12.8571428571429 3.0496536287611
12.6530612244898 3.19234481572625
12.4661968666239 3.3265306122449
12.4489795918367 3.33940666904784
12.2448979591837 3.49029770775382
12.0408163265306 3.64581135193749
11.9522661415266 3.71428571428571
11.8367346938776 3.80707352151227
11.6326530612245 3.974805058456
11.4840602252633 4.10204081632653
11.4285714285714 4.15159312631346
11.2244897959184 4.33904763759237
11.0701943119109 4.48979591836735
11.0204081632653 4.5408458380634
10.8163265306122 4.75987713550018
10.7137116535961 4.87755102040816
10.6122448979592 5.00064528724127
10.409343970528 5.26530612244898
10.4081632653061 5.26694552121559
10.2040816326531 5.56351433093989
10.1459979755731 5.6530612244898
10 5.89443800646295
10 5.6530612244898
10 5.26530612244898
10 4.87755102040816
10 4.48979591836735
10 4.10204081632653
10 3.94513001969655
10.2040816326531 3.74028815562637
10.2313590505136 3.71428571428571
10.4081632653061 3.55135351956724
10.6122448979592 3.38288569056386
10.6849413329613 3.3265306122449
10.8163265306122 3.22632211960925
11.0204081632653 3.08226011733463
11.2244897959184 2.95077007147251
11.2434393303933 2.93877551020408
11.4285714285714 2.82143094987365
11.6326530612245 2.70058709821667
11.8367346938776 2.58702925320619
11.9020867698598 2.55102040816327
12.0408163265306 2.47339377956256
12.2448979591837 2.36329391182203
12.4489795918367 2.25883607085603
12.6441679356764 2.16326530612245
12.6530612244898 2.15877618949561
12.8571428571429 2.05517919000581
13.0612244897959 1.9580017013433
13.265306122449 1.86675858902853
13.469387755102 1.78106030570892
13.482646490556 1.77551020408163
13.6734693877551 1.69272405980949
13.8775510204082 1.61169654199943
14.0816326530612 1.53813731328102
14.2857142857143 1.47156389883197
14.4897959183673 1.41151618234446
14.5764546449354 1.38775510204082
14.6938775510204 1.35236505967073
};
\addplot [draw=none, fill=white!4.70588235294118!black]
table{%
x  y
10.2040816326531 1
10.4081632653061 1
10.6122448979592 1
10.8163265306122 1
11.0204081632653 1
11.2244897959184 1
11.4285714285714 1
11.6326530612245 1
11.8367346938776 1
12.0408163265306 1
12.2448979591837 1
12.4489795918367 1
12.6530612244898 1
12.8571428571429 1
13.0612244897959 1
13.265306122449 1
13.469387755102 1
13.6734693877551 1
13.8775510204082 1
14.0816326530612 1
14.2857142857143 1
14.4897959183673 1
14.6938775510204 1
14.8979591836735 1
15.1020408163265 1
15.3061224489796 1
15.5102040816327 1
15.7142857142857 1
15.9183673469388 1
16.1224489795918 1
16.3265306122449 1
16.530612244898 1
16.734693877551 1
16.9387755102041 1
17.0111799972469 1
16.9387755102041 1.00575690417664
16.734693877551 1.02265771568845
16.530612244898 1.04096849760041
16.3265306122449 1.06107634566818
16.1224489795918 1.08342063672702
15.9183673469388 1.10849875057279
15.7142857142857 1.1368705987336
15.5102040816327 1.16916171721031
15.3061224489796 1.20606475794087
15.1020408163265 1.24833934033397
14.8979591836735 1.29681037067453
14.6938775510204 1.35236505967073
14.5764546449354 1.38775510204082
14.4897959183673 1.41151618234446
14.2857142857143 1.47156389883197
14.0816326530612 1.53813731328102
13.8775510204082 1.61169654199943
13.6734693877551 1.69272405980949
13.482646490556 1.77551020408163
13.469387755102 1.78106030570892
13.265306122449 1.86675858902853
13.0612244897959 1.9580017013433
12.8571428571429 2.05517919000582
12.6530612244898 2.15877618949561
12.6441679356764 2.16326530612245
12.4489795918367 2.25883607085603
12.2448979591837 2.36329391182203
12.0408163265306 2.47339377956256
11.9020867698598 2.55102040816327
11.8367346938775 2.58702925320619
11.6326530612245 2.70058709821667
11.4285714285714 2.82143094987365
11.2434393303933 2.93877551020408
11.2244897959184 2.95077007147251
11.0204081632653 3.08226011733463
10.8163265306122 3.22632211960925
10.6849413329613 3.3265306122449
10.6122448979592 3.38288569056386
10.4081632653061 3.55135351956724
10.2313590505136 3.71428571428571
10.2040816326531 3.74028815562637
10 3.94513001969655
10 3.71428571428571
10 3.3265306122449
10 2.93877551020408
10 2.55102040816327
10 2.16326530612245
10 1.77551020408163
10 1.38775510204082
10 1
10.2040816326531 1
};
\end{groupplot}

\end{tikzpicture}

	\caption{Heat maps of the surrogate safety metric described in \cref{sec:ngsim metric} as a function of the speed of the lead vehicle ($\speedleadsymbol$) and the gap between the ego vehicle and the lead vehicle ($\gapsymbol$).
		For each heat map, the other two input parameters are fixed at $\speedegosymbol=\SI{25}{\meter\per\second}$ (left) or $\speedegosymbol=\SI{20}{\meter\per\second}$ (center and right) and $\accelerationleadsymbol=\SI{0}{\meter\per\second\squared}$ (left and center) or $\accelerationleadsymbol=\SI{-1}{\meter\per\second\squared}$ (right).
		The estimated probability of collision ranges from 0 (white) to 1 (black).}
	\label{fig:heatmaps}
\end{figure}

In \cref{fig:scenarios}, the evaluations of the metric described in \cref{sec:ngsim metric} are shown for three different scenarios. 
Each of the three scenarios consider an ego vehicle and a lead vehicle driving in front of the ego vehicle.
Both vehicles are driving in the same direction and in the same lane. 
The speeds of the ego vehicle (black solid line) and the lead vehicle (black dashed line) are shown in the left plots of \cref{fig:scenarios} alongside the distance between the two vehicles (gray dotted line, scale on the right of the plot).
The right plots of \cref{fig:scenarios} show the evaluations of the metric explained in \cref{sec:ngsim metric} for each of the three scenarios (black line).
Furthermore, the right plots include the evaluations of $\wangstamatiadis$ of \cref{eq:ws}.

\setlength{\figurewidth}{.45\linewidth}
\setlength{\figureheight}{0.6\figurewidth}
\begin{figure}
	\centering
	% This file was created by tikzplotlib v0.9.8.
\begin{tikzpicture}

\begin{axis}[
height=\figureheight,
scaled y ticks=false,
tick align=outside,
tick pos=left,
width=\figurewidth,
x grid style={white!69.0196078431373!black},
xlabel={Time [\si{\second}]},
xmajorgrids,
xmin=0, xmax=12,
xtick style={color=black},
xticklabel style={align=center},
y grid style={white!69.0196078431373!black},
ylabel={Speed [\si{\meter\per\second}]},
ymajorgrids,
ymin=7.2, ymax=24.8,
ytick style={color=black},
yticklabel style={/pgf/number format/fixed,/pgf/number format/precision=3}
]
\addplot [very thick, black]
table {%
0 24
0.01 24
0.02 24
0.03 24
0.04 24
0.05 24
0.06 24
0.07 24
0.08 24
0.09 24
0.1 24
0.11 24
0.12 24
0.13 24
0.14 24
0.15 24
0.16 24
0.17 24
0.18 24
0.19 24
0.2 24
0.21 24
0.22 24
0.23 24
0.24 24
0.25 24
0.26 24
0.27 24
0.28 24
0.29 24
0.3 24
0.31 24
0.32 24
0.33 24
0.34 24
0.35 24
0.36 24
0.37 24
0.38 24
0.39 24
0.4 24
0.41 24
0.42 24
0.43 24
0.44 24
0.45 24
0.46 24
0.47 24
0.48 24
0.49 24
0.5 24
0.51 24
0.52 24
0.53 24
0.54 24
0.55 24
0.56 24
0.57 24
0.58 24
0.59 24
0.6 24
0.61 24
0.62 24
0.63 24
0.64 24
0.65 24
0.66 24
0.67 24
0.68 24
0.69 24
0.7 24
0.71 24
0.72 24
0.73 24
0.74 24
0.75 24
0.76 24
0.77 24
0.78 24
0.79 24
0.8 24
0.81 24
0.82 24
0.83 24
0.84 24
0.85 24
0.86 24
0.87 24
0.88 24
0.89 24
0.9 24
0.91 24
0.92 24
0.93 24
0.94 24
0.95 24
0.96 24
0.97 24
0.98 24
0.99 24
1 24
1.01 24
1.02 24
1.03 24
1.04 24
1.05 24
1.06 24
1.07 24
1.08 24
1.09 24
1.1 24
1.11 24
1.12 24
1.13 24
1.14 24
1.15 24
1.16 24
1.17 24
1.18 24
1.19 24
1.2 24
1.21 24
1.22 24
1.23 24
1.24 24
1.25 24
1.26 24
1.27 24
1.28 24
1.29 24
1.3 24
1.31 24
1.32 24
1.33 24
1.34 24
1.35 24
1.36 24
1.37 24
1.38 24
1.39 24
1.4 24
1.41 24
1.42 24
1.43 24
1.44 24
1.45 24
1.46 24
1.47 24
1.48 24
1.49 24
1.5 24
1.51 24
1.52 24
1.53 24
1.54 24
1.55 24
1.56 24
1.57 24
1.58 24
1.59 24
1.6 24
1.61 24
1.62 24
1.63 24
1.64 24
1.65 24
1.66 24
1.67 24
1.68 24
1.69 24
1.7 24
1.71 24
1.72 24
1.73 24
1.74 24
1.75 24
1.76 24
1.77 24
1.78 24
1.79 24
1.8 24
1.81 24
1.82 24
1.83 24
1.84 24
1.85 24
1.86 24
1.87 24
1.88 24
1.89 24
1.9 24
1.91 24
1.92 24
1.93 24
1.94 24
1.95 24
1.96 24
1.97 24
1.98 24
1.99 24
2 24
2.01 23.9997532611583
2.02 23.9990130598533
2.03 23.997779441744
2.04 23.9960524829259
2.05 23.9938322899258
2.06 23.9911189996958
2.07 23.9879127796043
2.08 23.9842138274262
2.09 23.9800223713302
2.1 23.975338669865
2.11 23.9701630119435
2.12 23.9644957168246
2.13 23.9583371340938
2.14 23.9516876436414
2.15 23.9445476556394
2.16 23.9369176105158
2.17 23.9287979789278
2.18 23.9201892617325
2.19 23.911091989956
2.2 23.9015067247611
2.21 23.891434057412
2.22 23.8808746092382
2.23 23.8698290315963
2.24 23.8582980058295
2.25 23.8462822432258
2.26 23.8337824849741
2.27 23.8207995021183
2.28 23.80733409551
2.29 23.7933870957588
2.3 23.7789593631814
2.31 23.7640517877483
2.32 23.748665289029
2.33 23.7328008161353
2.34 23.7164593476624
2.35 23.6996418916292
2.36 23.6823494854155
2.37 23.6645831956986
2.38 23.6463441183866
2.39 23.627633378552
2.4 23.6084521303612
2.41 23.588801557004
2.42 23.5686828706204
2.43 23.5480973122255
2.44 23.5270461516338
2.45 23.5055306873799
2.46 23.4835522466389
2.47 23.4611121851448
2.48 23.438211887106
2.49 23.414852765121
2.5 23.3910362600903
2.51 23.3667638411282
2.52 23.3420370054718
2.53 23.3168572783889
2.54 23.2912262130836
2.55 23.2651453906007
2.56 23.2386164197282
2.57 23.211640936898
2.58 23.1842206060849
2.59 23.1563571187042
2.6 23.1280521935069
2.61 23.0993075764743
2.62 23.0701250407095
2.63 23.0405063863291
2.64 23.0104534403509
2.65 22.9799680565824
2.66 22.9490521155055
2.67 22.9177075241612
2.68 22.8859362160315
2.69 22.8537401509204
2.7 22.8211213148327
2.71 22.788081719852
2.72 22.7546234040161
2.73 22.7207484311915
2.74 22.6864588909462
2.75 22.6517568984204
2.76 22.6166445941965
2.77 22.5811241441669
2.78 22.5451977394002
2.79 22.5088675960063
2.8 22.4721359549996
2.81 22.4350050821607
2.82 22.3974772678967
2.83 22.3595548271001
2.84 22.3212400990055
2.85 22.282535447046
2.86 22.2434432587066
2.87 22.2039659453779
2.88 22.1641059422063
2.89 22.1238657079447
2.9 22.0832477248002
2.91 22.0422544982815
2.92 22.0008885570437
2.93 21.959152452733
2.94 21.9170487598289
2.95 21.8745800754855
2.96 21.8317490193713
2.97 21.7885582335076
2.98 21.7450103821055
2.99 21.7011081514016
3 21.6568542494924
3.01 21.6122514061668
3.02 21.5673023727385
3.03 21.5220099218755
3.04 21.4763768474295
3.05 21.4304059642635
3.06 21.3841001080782
3.07 21.3374621352368
3.08 21.2904949225892
3.09 21.2432013672944
3.1 21.1955843866415
3.11 21.1476469178702
3.12 21.0993919179895
3.13 21.050822363595
3.14 21.0019412506856
3.15 20.9527515944787
3.16 20.9032564292238
3.17 20.853458808016
3.18 20.8033618026071
3.19 20.7529685032163
3.2 20.7022820183398
3.21 20.6513054745586
3.22 20.6000420163462
3.23 20.5484948058741
3.24 20.496667022817
3.25 20.4445618641568
3.26 20.3921825439851
3.27 20.3395322933049
3.28 20.286614359832
3.29 20.2334320077935
3.3 20.1799885177276
3.31 20.1262871862804
3.32 20.072331326003
3.33 20.018124265147
3.34 19.9636693474593
3.35 19.9089699319756
3.36 19.8540293928137
3.37 19.7988511189648
3.38 19.7434385140846
3.39 19.6877949962837
3.4 19.6319239979164
3.41 19.575828965369
3.42 19.5195133588473
3.43 19.4629806521633
3.44 19.4062343325206
3.45 19.3492779002994
3.46 19.2921148688409
3.47 19.23474876423
3.48 19.1771831250782
3.49 19.1194215023055
3.5 19.0614674589207
3.51 19.0033245698023
3.52 18.9449964214774
3.53 18.8864866119011
3.54 18.8277987502341
3.55 18.7689364566199
3.56 18.7099033619623
3.57 18.6507031077006
3.58 18.5913393455852
3.59 18.5318157374527
3.6 18.4721359549996
3.61 18.412303679556
3.62 18.3523226018584
3.63 18.2921964218224
3.64 18.2319288483138
3.65 18.1715235989206
3.66 18.110984399723
3.67 18.050314985064
3.68 17.9895190973188
3.69 17.9286004866643
3.7 17.8675629108472
3.71 17.8064101349528
3.72 17.7451459311723
3.73 17.6837740785704
3.74 17.6222983628521
3.75 17.560722576129
3.76 17.4990505166858
3.77 17.4372859887455
3.78 17.3754328022353
3.79 17.3134947725509
3.8 17.2514757203218
3.81 17.1893794711754
3.82 17.1272098555007
3.83 17.0649707082124
3.84 17.0026658685144
3.85 16.9402991796627
3.86 16.8778744887284
3.87 16.8153956463604
3.88 16.7528665065481
3.89 16.6902909263834
3.9 16.6276727658228
3.91 16.5650158874493
3.92 16.5023241562345
3.93 16.4396014392996
3.94 16.3768516056771
3.95 16.3140785260725
3.96 16.251286072625
3.97 16.1884781186689
3.98 16.1256585384946
3.99 16.0628312071097
4 16
4.01 15.9371687928903
4.02 15.8743414615054
4.03 15.8115218813311
4.04 15.748713927375
4.05 15.6859214739275
4.06 15.6231483943229
4.07 15.5603985607004
4.08 15.4976758437655
4.09 15.4349841125507
4.1 15.3723272341772
4.11 15.3097090736166
4.12 15.2471334934519
4.13 15.1846043536396
4.14 15.1221255112716
4.15 15.0597008203373
4.16 14.9973341314856
4.17 14.9350292917876
4.18 14.8727901444993
4.19 14.8106205288246
4.2 14.7485242796782
4.21 14.6865052274491
4.22 14.6245671977647
4.23 14.5627140112545
4.24 14.5009494833142
4.25 14.439277423871
4.26 14.3777016371479
4.27 14.3162259214296
4.28 14.2548540688277
4.29 14.1935898650472
4.3 14.1324370891528
4.31 14.0713995133357
4.32 14.0104809026812
4.33 13.949685014936
4.34 13.889015600277
4.35 13.8284764010794
4.36 13.7680711516862
4.37 13.7078035781776
4.38 13.6476773981416
4.39 13.587696320444
4.4 13.5278640450004
4.41 13.4681842625473
4.42 13.4086606544148
4.43 13.3492968922994
4.44 13.2900966380377
4.45 13.2310635433801
4.46 13.1722012497659
4.47 13.1135133880989
4.48 13.0550035785226
4.49 12.9966754301977
4.5 12.9385325410793
4.51 12.8805784976945
4.52 12.8228168749218
4.53 12.76525123577
4.54 12.7078851311591
4.55 12.6507220997006
4.56 12.5937656674794
4.57 12.5370193478367
4.58 12.4804866411527
4.59 12.424171034631
4.6 12.3680760020836
4.61 12.3122050037163
4.62 12.2565614859154
4.63 12.2011488810352
4.64 12.1459706071863
4.65 12.0910300680244
4.66 12.0363306525407
4.67 11.981875734853
4.68 11.927668673997
4.69 11.8737128137196
4.7 11.8200114822724
4.71 11.7665679922065
4.72 11.713385640168
4.73 11.660467706695
4.74 11.6078174560149
4.75 11.5554381358432
4.76 11.503332977183
4.77 11.4515051941259
4.78 11.3999579836538
4.79 11.3486945254414
4.8 11.2977179816602
4.81 11.2470314967837
4.82 11.1966381973929
4.83 11.146541191984
4.84 11.0967435707762
4.85 11.0472484055213
4.86 10.9980587493144
4.87 10.949177636405
4.88 10.9006080820105
4.89 10.8523530821298
4.9 10.8044156133585
4.91 10.7567986327056
4.92 10.7095050774108
4.93 10.6625378647632
4.94 10.6158998919218
4.95 10.5695940357365
4.96 10.5236231525705
4.97 10.4779900781245
4.98 10.4326976272615
4.99 10.3877485938332
5 10.3431457505076
5.01 10.2988918485984
5.02 10.2549896178945
5.03 10.2114417664924
5.04 10.1682509806287
5.05 10.1254199245145
5.06 10.0829512401711
5.07 10.040847547267
5.08 9.99911144295632
5.09 9.95774550171853
5.1 9.91675227519975
5.11 9.87613429205529
5.12 9.83589405779369
5.13 9.79603405462213
5.14 9.75655674129336
5.15 9.71746455295404
5.16 9.67875990099448
5.17 9.64044517289992
5.18 9.60252273210328
5.19 9.56499491783932
5.2 9.52786404500042
5.21 9.4911324039937
5.22 9.45480226059981
5.23 9.41887585583312
5.24 9.3833554058035
5.25 9.34824310157964
5.26 9.31354110905384
5.27 9.27925156880846
5.28 9.24537659598388
5.29 9.21191828014797
5.3 9.17887868516726
5.31 9.1462598490796
5.32 9.11406378396845
5.33 9.08229247583876
5.34 9.05094788449447
5.35 9.02003194341762
5.36 8.98954655964909
5.37 8.95949361367094
5.38 8.92987495929045
5.39 8.90069242352574
5.4 8.87194780649306
5.41 8.8436428812958
5.42 8.81577939391508
5.43 8.78835906310204
5.44 8.76138358027184
5.45 8.73485460939935
5.46 8.70877378691644
5.47 8.68314272161109
5.48 8.65796299452815
5.49 8.63323615887179
5.5 8.60896373990971
5.51 8.58514723487903
5.52 8.56178811289399
5.53 8.53888781485525
5.54 8.51644775336106
5.55 8.49446931262013
5.56 8.4729538483662
5.57 8.45190268777445
5.58 8.43131712937964
5.59 8.41119844299596
5.6 8.39154786963877
5.61 8.37236662144799
5.62 8.35365588161336
5.63 8.33541680430145
5.64 8.31765051458446
5.65 8.30035810837082
5.66 8.28354065233762
5.67 8.26719918386474
5.68 8.25133471097095
5.69 8.23594821225166
5.7 8.22104063681859
5.71 8.2066129042412
5.72 8.19266590449002
5.73 8.17920049788167
5.74 8.16621751502587
5.75 8.15371775677416
5.76 8.14170199417049
5.77 8.13017096840371
5.78 8.11912539076181
5.79 8.10856594258804
5.8 8.0984932752389
5.81 8.08890801004396
5.82 8.07981073826754
5.83 8.0712020210722
5.84 8.06308238948418
5.85 8.05545234436059
5.86 8.04831235635856
5.87 8.04166286590619
5.88 8.03550428317536
5.89 8.02983698805647
5.9 8.02466133013498
5.91 8.01997762866984
5.92 8.01578617257383
5.93 8.0120872203957
5.94 8.00888100030424
5.95 8.00616771007422
5.96 8.00394751707415
5.97 8.002220558256
5.98 8.00098694014672
5.99 8.00024673884168
6 8
6.01 8.00003084235521
6.02 8.00012336751834
6.03 8.000277569782
6.04 8.00049343963427
6.05 8.00077096375928
6.06 8.00111012503803
6.07 8.00151090254946
6.08 8.00197327157173
6.09 8.00249720358373
6.1 8.00308266626687
6.11 8.00372962350706
6.12 8.00443803539692
6.13 8.00520785823827
6.14 8.00603904454482
6.15 8.00693154304507
6.16 8.00788529868552
6.17 8.00890025263402
6.18 8.00997634228344
6.19 8.01111350125549
6.2 8.01231165940486
6.21 8.0135707428235
6.22 8.01489067384523
6.23 8.01627137105046
6.24 8.01771274927131
6.25 8.01921471959677
6.26 8.02077718937823
6.27 8.02240006223521
6.28 8.02408323806125
6.29 8.02582661303015
6.3 8.02763007960232
6.31 8.02949352653146
6.32 8.03141683887137
6.33 8.03339989798309
6.34 8.0354425815422
6.35 8.03754476354635
6.36 8.03970631432306
6.37 8.04192710053768
6.38 8.04420698520167
6.39 8.046545827681
6.4 8.04894348370485
6.41 8.05139980537449
6.42 8.05391464117245
6.43 8.05648783597181
6.44 8.05911923104577
6.45 8.06180866407752
6.46 8.06455596917013
6.47 8.06736097685691
6.48 8.07022351411175
6.49 8.07314340435988
6.5 8.07612046748871
6.51 8.07915451985897
6.52 8.08224537431602
6.53 8.08539284020139
6.54 8.08859672336455
6.55 8.09185682617492
6.56 8.09517294753398
6.57 8.09854488288775
6.58 8.10197242423938
6.59 8.10545536016198
6.6 8.10899347581163
6.61 8.11258655294072
6.62 8.11623436991131
6.63 8.11993670170887
6.64 8.12369331995614
6.65 8.1275039929272
6.66 8.13136848556181
6.67 8.13528655947984
6.68 8.13925797299606
6.69 8.14328248113495
6.7 8.14735983564591
6.71 8.1514897850185
6.72 8.15567207449799
6.73 8.15990644610106
6.74 8.16419263863173
6.75 8.16853038769746
6.76 8.17291942572544
6.77 8.17735948197914
6.78 8.18185028257498
6.79 8.18639155049921
6.8 8.19098300562505
6.81 8.19562436472992
6.82 8.20031534151291
6.83 8.20505564661249
6.84 8.20984498762431
6.85 8.21468306911926
6.86 8.21956959266167
6.87 8.22450425682776
6.88 8.22948675722421
6.89 8.23451678650691
6.9 8.23959403439997
6.91 8.24471818771482
6.92 8.24988893036954
6.93 8.25510594340838
6.94 8.26036890502139
6.95 8.26567749056431
6.96 8.27103137257859
6.97 8.27643022081155
6.98 8.28187370223681
6.99 8.28736148107479
7 8.29289321881345
7.01 8.29846857422914
7.02 8.30408720340769
7.03 8.30974875976556
7.04 8.31545289407131
7.05 8.32119925446706
7.06 8.32698748649023
7.07 8.3328172330954
7.08 8.33868813467635
7.09 8.34459982908821
7.1 8.35055195166982
7.11 8.35654413526622
7.12 8.36257601025131
7.13 8.36864720455062
7.14 8.3747573436643
7.15 8.38090605069016
7.16 8.38709294634702
7.17 8.393317648998
7.18 8.39957977467412
7.19 8.40587893709796
7.2 8.41221474770753
7.21 8.41858681568017
7.22 8.42499474795672
7.23 8.43143814926574
7.24 8.43791662214787
7.25 8.4444297669804
7.26 8.45097718200187
7.27 8.45755846333688
7.28 8.46417320502101
7.29 8.47082099902581
7.3 8.47750143528405
7.31 8.48421410171495
7.32 8.49095858424963
7.33 8.49773446685663
7.34 8.50454133156759
7.35 8.51137875850305
7.36 8.51824632589829
7.37 8.5251436101294
7.38 8.53207018573943
7.39 8.53902562546454
7.4 8.54600950026045
7.41 8.55302137932888
7.42 8.56006083014409
7.43 8.56712741847959
7.44 8.57422070843493
7.45 8.58134026246257
7.46 8.58848564139489
7.47 8.59565640447126
7.48 8.60285210936522
7.49 8.61007231221181
7.5 8.61731656763491
7.51 8.62458442877472
7.52 8.63187544731532
7.53 8.63918917351236
7.54 8.64652515622074
7.55 8.65388294292251
7.56 8.66126207975471
7.57 8.66866211153743
7.58 8.67608258180185
7.59 8.68352303281841
7.6 8.69098300562505
7.61 8.6984620400555
7.62 8.7059596747677
7.63 8.7134754472722
7.64 8.72100889396077
7.65 8.72855955013492
7.66 8.73612695003463
7.67 8.743710626867
7.68 8.75131011283515
7.69 8.75892493916696
7.7 8.76655463614409
7.71 8.77419873313089
7.72 8.78185675860346
7.73 8.78952824017869
7.74 8.79721270464349
7.75 8.80490967798387
7.76 8.81261868541428
7.77 8.82033925140681
7.78 8.82807089972059
7.79 8.83581315343114
7.8 8.84356553495977
7.81 8.85132756610308
7.82 8.85909876806242
7.83 8.86687866147345
7.84 8.87466676643569
7.85 8.88246260254216
7.86 8.89026568890895
7.87 8.89807554420495
7.88 8.90589168668149
7.89 8.91371363420208
7.9 8.92154090427216
7.91 8.92937301406883
7.92 8.93720948047068
7.93 8.94504982008755
7.94 8.95289354929036
7.95 8.96074018424093
7.96 8.96858924092187
7.97 8.97644023516639
7.98 8.98429268268818
7.99 8.99214609911129
8 9
8.01 9.00785390088871
8.02 9.01570731731182
8.03 9.02355976483361
8.04 9.03141075907813
8.05 9.03925981575907
8.06 9.04710645070964
8.07 9.05495017991245
8.08 9.06279051952932
8.09 9.07062698593117
8.1 9.07845909572784
8.11 9.08628636579792
8.12 9.09410831331851
8.13 9.10192445579505
8.14 9.10973431109105
8.15 9.11753739745784
8.16 9.12533323356431
8.17 9.13312133852655
8.18 9.14090123193758
8.19 9.14867243389692
8.2 9.15643446504023
8.21 9.16418684656886
8.22 9.17192910027941
8.23 9.17966074859319
8.24 9.18738131458572
8.25 9.19509032201613
8.26 9.20278729535651
8.27 9.21047175982131
8.28 9.21814324139654
8.29 9.22580126686911
8.3 9.23344536385591
8.31 9.24107506083304
8.32 9.24868988716485
8.33 9.256289373133
8.34 9.26387304996537
8.35 9.27144044986508
8.36 9.27899110603923
8.37 9.2865245527278
8.38 9.29404032523231
8.39 9.3015379599445
8.4 9.30901699437495
8.41 9.31647696718159
8.42 9.32391741819815
8.43 9.33133788846257
8.44 9.33873792024529
8.45 9.34611705707749
8.46 9.35347484377926
8.47 9.36081082648764
8.48 9.36812455268468
8.49 9.37541557122528
8.5 9.38268343236509
8.51 9.38992768778819
8.52 9.39714789063478
8.53 9.40434359552874
8.54 9.41151435860511
8.55 9.41865973753743
8.56 9.42577929156507
8.57 9.43287258152041
8.58 9.43993916985591
8.59 9.44697862067112
8.6 9.45399049973955
8.61 9.46097437453546
8.62 9.46792981426058
8.63 9.4748563898706
8.64 9.48175367410171
8.65 9.48862124149695
8.66 9.49545866843241
8.67 9.50226553314337
8.68 9.50904141575037
8.69 9.51578589828505
8.7 9.52249856471595
8.71 9.52917900097419
8.72 9.535826794979
8.73 9.54244153666312
8.74 9.54902281799813
8.75 9.5555702330196
8.76 9.56208337785213
8.77 9.56856185073426
8.78 9.57500525204328
8.79 9.58141318431983
8.8 9.58778525229247
8.81 9.59412106290204
8.82 9.60042022532588
8.83 9.606682351002
8.84 9.61290705365298
8.85 9.61909394930984
8.86 9.6252426563357
8.87 9.63135279544938
8.88 9.63742398974869
8.89 9.64345586473378
8.9 9.64944804833019
8.91 9.65540017091179
8.92 9.66131186532365
8.93 9.6671827669046
8.94 9.67301251350977
8.95 9.67880074553294
8.96 9.68454710592869
8.97 9.69025124023444
8.98 9.69591279659231
8.99 9.70153142577086
9 9.70710678118655
9.01 9.7126385189252
9.02 9.71812629776319
9.03 9.72356977918845
9.04 9.72896862742141
9.05 9.73432250943569
9.06 9.73963109497861
9.07 9.74489405659162
9.08 9.75011106963046
9.09 9.75528181228518
9.1 9.76040596560003
9.11 9.76548321349309
9.12 9.77051324277579
9.13 9.77549574317224
9.14 9.78043040733833
9.15 9.78531693088075
9.16 9.79015501237569
9.17 9.79494435338751
9.18 9.79968465848709
9.19 9.80437563527008
9.2 9.80901699437495
9.21 9.81360844950079
9.22 9.81814971742502
9.23 9.82264051802086
9.24 9.82708057427456
9.25 9.83146961230254
9.26 9.83580736136827
9.27 9.84009355389894
9.28 9.84432792550201
9.29 9.84851021498151
9.3 9.85264016435409
9.31 9.85671751886505
9.32 9.86074202700394
9.33 9.86471344052016
9.34 9.86863151443819
9.35 9.8724960070728
9.36 9.87630668004386
9.37 9.88006329829113
9.38 9.88376563008869
9.39 9.88741344705928
9.4 9.89100652418837
9.41 9.89454463983802
9.42 9.89802757576062
9.43 9.90145511711225
9.44 9.90482705246602
9.45 9.90814317382508
9.46 9.91140327663545
9.47 9.91460715979861
9.48 9.91775462568398
9.49 9.92084548014103
9.5 9.92387953251129
9.51 9.92685659564012
9.52 9.92977648588825
9.53 9.93263902314309
9.54 9.93544403082987
9.55 9.93819133592248
9.56 9.94088076895423
9.57 9.94351216402819
9.58 9.94608535882755
9.59 9.94860019462551
9.6 9.95105651629515
9.61 9.953454172319
9.62 9.95579301479833
9.63 9.95807289946232
9.64 9.96029368567694
9.65 9.96245523645365
9.66 9.9645574184578
9.67 9.96660010201691
9.68 9.96858316112863
9.69 9.97050647346854
9.7 9.97236992039768
9.71 9.97417338696985
9.72 9.97591676193875
9.73 9.97759993776479
9.74 9.97922281062177
9.75 9.98078528040323
9.76 9.98228725072869
9.77 9.98372862894954
9.78 9.98510932615477
9.79 9.9864292571765
9.8 9.98768834059514
9.81 9.98888649874451
9.82 9.99002365771656
9.83 9.99109974736598
9.84 9.99211470131448
9.85 9.99306845695493
9.86 9.99396095545518
9.87 9.99479214176173
9.88 9.99556196460308
9.89 9.99627037649294
9.9 9.99691733373313
9.91 9.99750279641627
9.92 9.99802672842827
9.93 9.99848909745054
9.94 9.99888987496197
9.95 9.99922903624072
9.96 9.99950656036573
9.97 9.999722430218
9.98 9.99987663248166
9.99 9.99996915764479
10 10
10.01 10
10.02 10
10.03 10
10.04 10
10.05 10
10.06 10
10.07 10
10.08 10
10.09 10
10.1 10
10.11 10
10.12 10
10.13 10
10.14 10
10.15 10
10.16 10
10.17 10
10.18 10
10.19 10
10.2 10
10.21 10
10.22 10
10.23 10
10.24 10
10.25 10
10.26 10
10.27 10
10.28 10
10.29 10
10.3 10
10.31 10
10.32 10
10.33 10
10.34 10
10.35 10
10.36 10
10.37 10
10.38 10
10.39 10
10.4 10
10.41 10
10.42 10
10.43 10
10.44 10
10.45 10
10.46 10
10.47 10
10.48 10
10.49 10
10.5 10
10.51 10
10.52 10
10.53 10
10.54 10
10.55 10
10.56 10
10.57 10
10.58 10
10.59 10
10.6 10
10.61 10
10.62 10
10.63 10
10.64 10
10.65 10
10.66 10
10.67 10
10.68 10
10.69 10
10.7 10
10.71 10
10.72 10
10.73 10
10.74 10
10.75 10
10.76 10
10.77 10
10.78 10
10.79 10
10.8 10
10.81 10
10.82 10
10.83 10
10.84 10
10.85 10
10.86 10
10.87 10
10.88 10
10.89 10
10.9 10
10.91 10
10.92 10
10.93 10
10.94 10
10.95 10
10.96 10
10.97 10
10.98 10
10.99 10
11 10
11.01 10
11.02 10
11.03 10
11.04 10
11.05 10
11.06 10
11.07 10
11.08 10
11.09 10
11.1 10
11.11 10
11.12 10
11.13 10
11.14 10
11.15 10
11.16 10
11.17 10
11.18 10
11.19 10
11.2 10
11.21 10
11.22 10
11.23 10
11.24 10
11.25 10
11.26 10
11.27 10
11.28 10
11.29 10
11.3 10
11.31 10
11.32 10
11.33 10
11.34 10
11.35 10
11.36 10
11.37 10
11.38 10
11.39 10
11.4 10
11.41 10
11.42 10
11.43 10
11.44 10
11.45 10
11.46 10
11.47 10
11.48 10
11.49 10
11.5 10
11.51 10
11.52 10
11.53 10
11.54 10
11.55 10
11.56 10
11.57 10
11.58 10
11.59 10
11.6 10
11.61 10
11.62 10
11.63 10
11.64 10
11.65 10
11.66 10
11.67 10
11.68 10
11.69 10
11.7 10
11.71 10
11.72 10
11.73 10
11.74 10
11.75 10
11.76 10
11.77 10
11.78 10
11.79 10
11.8 10
11.81 10
11.82 10
11.83 10
11.84 10
11.85 10
11.86 10
11.87 10
11.88 10
11.89 10
11.9 10
11.91 10
11.92 10
11.93 10
11.94 10
11.95 10
11.96 10
11.97 10
11.98 10
11.99 10
12 10
};
\addplot [very thick, black, dashed]
table {%
0 20
0.01 20
0.02 20
0.03 20
0.04 20
0.05 20
0.06 20
0.07 20
0.08 20
0.09 20
0.1 20
0.11 20
0.12 20
0.13 20
0.14 20
0.15 20
0.16 20
0.17 20
0.18 20
0.19 20
0.2 20
0.21 20
0.22 20
0.23 20
0.24 20
0.25 20
0.26 20
0.27 20
0.28 20
0.29 20
0.3 20
0.31 20
0.32 20
0.33 20
0.34 20
0.35 20
0.36 20
0.37 20
0.38 20
0.39 20
0.4 20
0.41 20
0.42 20
0.43 20
0.44 20
0.45 20
0.46 20
0.47 20
0.48 20
0.49 20
0.5 20
0.51 20
0.52 20
0.53 20
0.54 20
0.55 20
0.56 20
0.57 20
0.58 20
0.59 20
0.6 20
0.61 20
0.62 20
0.63 20
0.64 20
0.65 20
0.66 20
0.67 20
0.68 20
0.69 20
0.7 20
0.71 20
0.72 20
0.73 20
0.74 20
0.75 20
0.76 20
0.77 20
0.78 20
0.79 20
0.8 20
0.81 20
0.82 20
0.83 20
0.84 20
0.85 20
0.86 20
0.87 20
0.88 20
0.89 20
0.9 20
0.91 20
0.92 20
0.93 20
0.94 20
0.95 20
0.96 20
0.97 20
0.98 20
0.99 20
1 20
1.01 20
1.02 20
1.03 20
1.04 20
1.05 20
1.06 20
1.07 20
1.08 20
1.09 20
1.1 20
1.11 20
1.12 20
1.13 20
1.14 20
1.15 20
1.16 20
1.17 20
1.18 20
1.19 20
1.2 20
1.21 20
1.22 20
1.23 20
1.24 20
1.25 20
1.26 20
1.27 20
1.28 20
1.29 20
1.3 20
1.31 20
1.32 20
1.33 20
1.34 20
1.35 20
1.36 20
1.37 20
1.38 20
1.39 20
1.4 20
1.41 20
1.42 20
1.43 20
1.44 20
1.45 20
1.46 20
1.47 20
1.48 20
1.49 20
1.5 20
1.51 20
1.52 20
1.53 20
1.54 20
1.55 20
1.56 20
1.57 20
1.58 20
1.59 20
1.6 20
1.61 20
1.62 20
1.63 20
1.64 20
1.65 20
1.66 20
1.67 20
1.68 20
1.69 20
1.7 20
1.71 20
1.72 20
1.73 20
1.74 20
1.75 20
1.76 20
1.77 20
1.78 20
1.79 20
1.8 20
1.81 20
1.82 20
1.83 20
1.84 20
1.85 20
1.86 20
1.87 20
1.88 20
1.89 20
1.9 20
1.91 20
1.92 20
1.93 20
1.94 20
1.95 20
1.96 20
1.97 20
1.98 20
1.99 20
2 20
2.01 20
2.02 20
2.03 20
2.04 20
2.05 20
2.06 20
2.07 20
2.08 20
2.09 20
2.1 20
2.11 20
2.12 20
2.13 20
2.14 20
2.15 20
2.16 20
2.17 20
2.18 20
2.19 20
2.2 20
2.21 20
2.22 20
2.23 20
2.24 20
2.25 20
2.26 20
2.27 20
2.28 20
2.29 20
2.3 20
2.31 20
2.32 20
2.33 20
2.34 20
2.35 20
2.36 20
2.37 20
2.38 20
2.39 20
2.4 20
2.41 20
2.42 20
2.43 20
2.44 20
2.45 20
2.46 20
2.47 20
2.48 20
2.49 20
2.5 20
2.51 20
2.52 20
2.53 20
2.54 20
2.55 20
2.56 20
2.57 20
2.58 20
2.59 20
2.6 20
2.61 20
2.62 20
2.63 20
2.64 20
2.65 20
2.66 20
2.67 20
2.68 20
2.69 20
2.7 20
2.71 20
2.72 20
2.73 20
2.74 20
2.75 20
2.76 20
2.77 20
2.78 20
2.79 20
2.8 20
2.81 20
2.82 20
2.83 20
2.84 20
2.85 20
2.86 20
2.87 20
2.88 20
2.89 20
2.9 20
2.91 20
2.92 20
2.93 20
2.94 20
2.95 20
2.96 20
2.97 20
2.98 20
2.99 20
3 20
3.01 19.9997779355447
3.02 19.999111761904
3.03 19.9980015382513
3.04 19.9964473632029
3.05 19.9944493748099
3.06 19.9920077505449
3.07 19.9891227072872
3.08 19.9857945013031
3.09 19.982023428223
3.1 19.9778098230154
3.11 19.9731540599572
3.12 19.9680565526
3.13 19.9625177537341
3.14 19.9565381553475
3.15 19.9501182885828
3.16 19.9432587236896
3.17 19.9359600699741
3.18 19.928222975745
3.19 19.9200481282557
3.2 19.9114362536434
3.21 19.9023881168647
3.22 19.8929045216274
3.23 19.8829863103191
3.24 19.8726343639329
3.25 19.8618496019884
3.26 19.8506329824505
3.27 19.8389855016443
3.28 19.8269081941664
3.29 19.8144021327929
3.3 19.8014684283847
3.31 19.7881082297881
3.32 19.7743227237332
3.33 19.7601131347284
3.34 19.7454807249515
3.35 19.7304267941377
3.36 19.7149526794643
3.37 19.6990597554316
3.38 19.682749433741
3.39 19.6660231631695
3.4 19.6488824294413
3.41 19.6313287550953
3.42 19.6133636993506
3.43 19.5949888579671
3.44 19.5762058631046
3.45 19.5570163831772
3.46 19.5374221227056
3.47 19.5174248221652
3.48 19.4970262578319
3.49 19.4762282416241
3.5 19.4550326209418
3.51 19.4334412785028
3.52 19.4114561321748
3.53 19.3890791348056
3.54 19.3663122740496
3.55 19.343157572191
3.56 19.3196170859642
3.57 19.2956929063714
3.58 19.2713871584965
3.59 19.2467020013166
3.6 19.2216396275101
3.61 19.1962022632619
3.62 19.1703921680659
3.63 19.1442116345238
3.64 19.1176629881421
3.65 19.0907485871251
3.66 19.0634708221655
3.67 19.035832116232
3.68 19.0078349243544
3.69 18.9794817334051
3.7 18.9507750618785
3.71 18.9217174596671
3.72 18.8923115078351
3.73 18.8625598183893
3.74 18.8324650340467
3.75 18.8020298280002
3.76 18.7712569036805
3.77 18.7401489945169
3.78 18.7087088636937
3.79 18.6769393039051
3.8 18.6448431371071
3.81 18.6124232142667
3.82 18.5796824151092
3.83 18.5466236478614
3.84 18.5132498489942
3.85 18.4795639829616
3.86 18.4455690419367
3.87 18.411268045547
3.88 18.3766640406051
3.89 18.341760100839
3.9 18.3065593266183
3.91 18.2710648446793
3.92 18.2352798078472
3.93 18.1992073947559
3.94 18.1628508095656
3.95 18.1262132816785
3.96 18.0892980654517
3.97 18.052108439908
3.98 18.0146477084451
3.99 17.9769191985418
4 17.9389262614624
4.01 17.9006722719592
4.02 17.862160627973
4.03 17.8233947503304
4.04 17.7843780824409
4.05 17.7451140899907
4.06 17.7056062606344
4.07 17.6658581036859
4.08 17.6258731498065
4.09 17.5856549506908
4.1 17.5452070787519
4.11 17.5045331268035
4.12 17.4636367077415
4.13 17.422521454222
4.14 17.3811910183397
4.15 17.3396490713029
4.16 17.2978993031074
4.17 17.2559454222092
4.18 17.2137911551945
4.19 17.171440246449
4.2 17.1288964578254
4.21 17.0861635683088
4.22 17.0432453736817
4.23 17.0001456861863
4.24 16.956868334186
4.25 16.9134171618254
4.26 16.869796028689
4.27 16.8260088094579
4.28 16.7820593935663
4.29 16.7379516848552
4.3 16.6936896012265
4.31 16.6492770742943
4.32 16.604718049036
4.33 16.5600164834421
4.34 16.5151763481639
4.35 16.4702016261615
4.36 16.4250963123499
4.37 16.3798644132437
4.38 16.3345099466019
4.39 16.2890369410703
4.4 16.2434494358243
4.41 16.1977514802096
4.42 16.151947133383
4.43 16.1060404639512
4.44 16.0600355496103
4.45 16.0139364767826
4.46 15.9677473402543
4.47 15.9214722428117
4.48 15.8751152948764
4.49 15.8286806141406
4.5 15.7821723252012
4.51 15.7355945591932
4.52 15.6889514534232
4.53 15.6422471510015
4.54 15.5954858004743
4.55 15.5486715554552
4.56 15.5018085742561
4.57 15.4549010195178
4.58 15.4079530578408
4.59 15.3609688594143
4.6 15.3139525976466
4.61 15.2669084487938
4.62 15.2198405915893
4.63 15.1727532068724
4.64 15.1256504772167
4.65 15.0785365865591
4.66 15.0314157198278
4.67 14.9842920625706
4.68 14.9371698005832
4.69 14.8900531195375
4.7 14.8429462046094
4.71 14.7958532401075
4.72 14.7487784091012
4.73 14.7017258930491
4.74 14.654699871428
4.75 14.6077045213608
4.76 14.5607440172463
4.77 14.513822530388
4.78 14.4669442286237
4.79 14.4201132759552
4.8 14.3733338321785
4.81 14.3266100525142
4.82 14.2799460872387
4.83 14.2333460813152
4.84 14.1868141740256
4.85 14.1403544986029
4.86 14.0939711818643
4.87 14.0476683438441
4.88 14.001450097428
4.89 13.9553205479879
4.9 13.9092837930173
4.91 13.8633439217668
4.92 13.8175050148814
4.93 13.7717711440379
4.94 13.7261463715831
4.95 13.6806347501731
4.96 13.6352403224134
4.97 13.5899671204994
4.98 13.5448191658586
4.99 13.4998004687936
5 13.4549150281253
5.01 13.4101668308379
5.02 13.3655598517253
5.03 13.3210980530371
5.04 13.2767853841274
5.05 13.2326257811037
5.06 13.1886231664773
5.07 13.1447814488147
5.08 13.101104522391
5.09 13.0575962668432
5.1 13.0142605468261
5.11 12.971101211669
5.12 12.9281220950336
5.13 12.8853270145735
5.14 12.8427197715952
5.15 12.8003041507204
5.16 12.7580839195498
5.17 12.7160628283285
5.18 12.6742446096127
5.19 12.6326329779384
5.2 12.5912316294914
5.21 12.5500442417788
5.22 12.5090744733025
5.23 12.4683259632343
5.24 12.4278023310925
5.25 12.3875071764203
5.26 12.3474440784663
5.27 12.3076165958667
5.28 12.2680282663287
5.29 12.2286826063165
5.3 12.1895831107393
5.31 12.1507332526404
5.32 12.1121364828887
5.33 12.0737962298724
5.34 12.0357158991947
5.35 11.9978988733706
5.36 11.960348511527
5.37 11.9230681491041
5.38 11.8860610975594
5.39 11.8493306440732
5.4 11.8128800512565
5.41 11.776712556862
5.42 11.7408313734956
5.43 11.7052396883314
5.44 11.6699406628287
5.45 11.6349374324511
5.46 11.6002331063879
5.47 11.565830767278
5.48 11.531733470936
5.49 11.497944246081
5.5 11.4644660940673
5.51 11.4313019886179
5.52 11.3984548755605
5.53 11.3659276725655
5.54 11.3337232688872
5.55 11.301844525107
5.56 11.2702942728791
5.57 11.2390753146794
5.58 11.2081904235564
5.59 11.1776423428845
5.6 11.1474337861211
5.61 11.1175674365646
5.62 11.0880459471171
5.63 11.0588719400478
5.64 11.0300480067608
5.65 11.0015767075645
5.66 10.9734605714444
5.67 10.9457020958381
5.68 10.9183037464141
5.69 10.891267956852
5.7 10.8645971286272
5.71 10.8382936307967
5.72 10.8123597997893
5.73 10.7867979391978
5.74 10.7616103195746
5.75 10.7367991782295
5.76 10.7123667190317
5.77 10.6883151122135
5.78 10.6646464941775
5.79 10.6413629673075
5.8 10.6184665997807
5.81 10.595959425385
5.82 10.5738434433377
5.83 10.5521206181083
5.84 10.5307928792437
5.85 10.5098621211969
5.86 10.489330203159
5.87 10.4691989488936
5.88 10.449470146575
5.89 10.4301455486297
5.9 10.4112268715801
5.91 10.3927157958925
5.92 10.3746139658277
5.93 10.3569229892949
5.94 10.3396444377089
5.95 10.3227798458507
5.96 10.3063307117306
5.97 10.290298496456
5.98 10.274684624101
5.99 10.2594904815798
6 10.2447174185242
6.01 10.230366747163
6.02 10.2164397422058
6.03 10.2029376407298
6.04 10.1898616420696
6.05 10.177212907711
6.06 10.1649925611878
6.07 10.1532016879819
6.08 10.1418413354266
6.09 10.1309125126144
6.1 10.1204161903063
6.11 10.1103533008464
6.12 10.1007247380788
6.13 10.091531357268
6.14 10.0827739750234
6.15 10.0744533692261
6.16 10.0665702789607
6.17 10.0591254044486
6.18 10.0521194069867
6.19 10.0455529088883
6.2 10.0394264934276
6.21 10.0337407047883
6.22 10.028496048015
6.23 10.0236929889685
6.24 10.0193319542841
6.25 10.0154133313344
6.26 10.0119374681939
6.27 10.0089046736089
6.28 10.0063152169699
6.29 10.0041693282873
6.3 10.0024671981713
6.31 10.0012089778151
6.32 10.0003947789809
6.33 10.0000246739907
6.34 10
6.35 10
6.36 10
6.37 10
6.38 10
6.39 10
6.4 10
6.41 10
6.42 10
6.43 10
6.44 10
6.45 10
6.46 10
6.47 10
6.48 10
6.49 10
6.5 10
6.51 10
6.52 10
6.53 10
6.54 10
6.55 10
6.56 10
6.57 10
6.58 10
6.59 10
6.6 10
6.61 10
6.62 10
6.63 10
6.64 10
6.65 10
6.66 10
6.67 10
6.68 10
6.69 10
6.7 10
6.71 10
6.72 10
6.73 10
6.74 10
6.75 10
6.76 10
6.77 10
6.78 10
6.79 10
6.8 10
6.81 10
6.82 10
6.83 10
6.84 10
6.85 10
6.86 10
6.87 10
6.88 10
6.89 10
6.9 10
6.91 10
6.92 10
6.93 10
6.94 10
6.95 10
6.96 10
6.97 10
6.98 10
6.99 10
7 10
7.01 10
7.02 10
7.03 10
7.04 10
7.05 10
7.06 10
7.07 10
7.08 10
7.09 10
7.1 10
7.11 10
7.12 10
7.13 10
7.14 10
7.15 10
7.16 10
7.17 10
7.18 10
7.19 10
7.2 10
7.21 10
7.22 10
7.23 10
7.24 10
7.25 10
7.26 10
7.27 10
7.28 10
7.29 10
7.3 10
7.31 10
7.32 10
7.33 10
7.34 10
7.35 10
7.36 10
7.37 10
7.38 10
7.39 10
7.4 10
7.41 10
7.42 10
7.43 10
7.44 10
7.45 10
7.46 10
7.47 10
7.48 10
7.49 10
7.5 10
7.51 10
7.52 10
7.53 10
7.54 10
7.55 10
7.56 10
7.57 10
7.58 10
7.59 10
7.6 10
7.61 10
7.62 10
7.63 10
7.64 10
7.65 10
7.66 10
7.67 10
7.68 10
7.69 10
7.7 10
7.71 10
7.72 10
7.73 10
7.74 10
7.75 10
7.76 10
7.77 10
7.78 10
7.79 10
7.8 10
7.81 10
7.82 10
7.83 10
7.84 10
7.85 10
7.86 10
7.87 10
7.88 10
7.89 10
7.9 10
7.91 10
7.92 10
7.93 10
7.94 10
7.95 10
7.96 10
7.97 10
7.98 10
7.99 10
8 10
8.01 10
8.02 10
8.03 10
8.04 10
8.05 10
8.06 10
8.07 10
8.08 10
8.09 10
8.1 10
8.11 10
8.12 10
8.13 10
8.14 10
8.15 10
8.16 10
8.17 10
8.18 10
8.19 10
8.2 10
8.21 10
8.22 10
8.23 10
8.24 10
8.25 10
8.26 10
8.27 10
8.28 10
8.29 10
8.3 10
8.31 10
8.32 10
8.33 10
8.34 10
8.35 10
8.36 10
8.37 10
8.38 10
8.39 10
8.4 10
8.41 10
8.42 10
8.43 10
8.44 10
8.45 10
8.46 10
8.47 10
8.48 10
8.49 10
8.5 10
8.51 10
8.52 10
8.53 10
8.54 10
8.55 10
8.56 10
8.57 10
8.58 10
8.59 10
8.6 10
8.61 10
8.62 10
8.63 10
8.64 10
8.65 10
8.66 10
8.67 10
8.68 10
8.69 10
8.7 10
8.71 10
8.72 10
8.73 10
8.74 10
8.75 10
8.76 10
8.77 10
8.78 10
8.79 10
8.8 10
8.81 10
8.82 10
8.83 10
8.84 10
8.85 10
8.86 10
8.87 10
8.88 10
8.89 10
8.9 10
8.91 10
8.92 10
8.93 10
8.94 10
8.95 10
8.96 10
8.97 10
8.98 10
8.99 10
9 10
9.01 10
9.02 10
9.03 10
9.04 10
9.05 10
9.06 10
9.07 10
9.08 10
9.09 10
9.1 10
9.11 10
9.12 10
9.13 10
9.14 10
9.15 10
9.16 10
9.17 10
9.18 10
9.19 10
9.2 10
9.21 10
9.22 10
9.23 10
9.24 10
9.25 10
9.26 10
9.27 10
9.28 10
9.29 10
9.3 10
9.31 10
9.32 10
9.33 10
9.34 10
9.35 10
9.36 10
9.37 10
9.38 10
9.39 10
9.4 10
9.41 10
9.42 10
9.43 10
9.44 10
9.45 10
9.46 10
9.47 10
9.48 10
9.49 10
9.5 10
9.51 10
9.52 10
9.53 10
9.54 10
9.55 10
9.56 10
9.57 10
9.58 10
9.59 10
9.6 10
9.61 10
9.62 10
9.63 10
9.64 10
9.65 10
9.66 10
9.67 10
9.68 10
9.69 10
9.7 10
9.71 10
9.72 10
9.73 10
9.74 10
9.75 10
9.76 10
9.77 10
9.78 10
9.79 10
9.8 10
9.81 10
9.82 10
9.83 10
9.84 10
9.85 10
9.86 10
9.87 10
9.88 10
9.89 10
9.9 10
9.91 10
9.92 10
9.93 10
9.94 10
9.95 10
9.96 10
9.97 10
9.98 10
9.99 10
10 10
10.01 10
10.02 10
10.03 10
10.04 10
10.05 10
10.06 10
10.07 10
10.08 10
10.09 10
10.1 10
10.11 10
10.12 10
10.13 10
10.14 10
10.15 10
10.16 10
10.17 10
10.18 10
10.19 10
10.2 10
10.21 10
10.22 10
10.23 10
10.24 10
10.25 10
10.26 10
10.27 10
10.28 10
10.29 10
10.3 10
10.31 10
10.32 10
10.33 10
10.34 10
10.35 10
10.36 10
10.37 10
10.38 10
10.39 10
10.4 10
10.41 10
10.42 10
10.43 10
10.44 10
10.45 10
10.46 10
10.47 10
10.48 10
10.49 10
10.5 10
10.51 10
10.52 10
10.53 10
10.54 10
10.55 10
10.56 10
10.57 10
10.58 10
10.59 10
10.6 10
10.61 10
10.62 10
10.63 10
10.64 10
10.65 10
10.66 10
10.67 10
10.68 10
10.69 10
10.7 10
10.71 10
10.72 10
10.73 10
10.74 10
10.75 10
10.76 10
10.77 10
10.78 10
10.79 10
10.8 10
10.81 10
10.82 10
10.83 10
10.84 10
10.85 10
10.86 10
10.87 10
10.88 10
10.89 10
10.9 10
10.91 10
10.92 10
10.93 10
10.94 10
10.95 10
10.96 10
10.97 10
10.98 10
10.99 10
11 10
11.01 10
11.02 10
11.03 10
11.04 10
11.05 10
11.06 10
11.07 10
11.08 10
11.09 10
11.1 10
11.11 10
11.12 10
11.13 10
11.14 10
11.15 10
11.16 10
11.17 10
11.18 10
11.19 10
11.2 10
11.21 10
11.22 10
11.23 10
11.24 10
11.25 10
11.26 10
11.27 10
11.28 10
11.29 10
11.3 10
11.31 10
11.32 10
11.33 10
11.34 10
11.35 10
11.36 10
11.37 10
11.38 10
11.39 10
11.4 10
11.41 10
11.42 10
11.43 10
11.44 10
11.45 10
11.46 10
11.47 10
11.48 10
11.49 10
11.5 10
11.51 10
11.52 10
11.53 10
11.54 10
11.55 10
11.56 10
11.57 10
11.58 10
11.59 10
11.6 10
11.61 10
11.62 10
11.63 10
11.64 10
11.65 10
11.66 10
11.67 10
11.68 10
11.69 10
11.7 10
11.71 10
11.72 10
11.73 10
11.74 10
11.75 10
11.76 10
11.77 10
11.78 10
11.79 10
11.8 10
11.81 10
11.82 10
11.83 10
11.84 10
11.85 10
11.86 10
11.87 10
11.88 10
11.89 10
11.9 10
11.91 10
11.92 10
11.93 10
11.94 10
11.95 10
11.96 10
11.97 10
11.98 10
11.99 10
12 10
};
\end{axis}

\begin{axis}[
axis y line=right,
height=\figureheight,
scaled y ticks=false,
tick align=outside,
width=\figurewidth,
x grid style={white!69.0196078431373!black},
xmin=0, xmax=12,
xtick pos=left,
xtick style={color=black},
xticklabel style={align=center},
y grid style={white!69.0196078431373!black},
ylabel={Distance [\si{\meter}]},
ymin=27.2, ymax=44.8,
ytick pos=right,
ytick style={color=black},
yticklabel style={/pgf/number format/fixed,/pgf/number format/precision=3},
yticklabel style={anchor=west}
]
\addplot [very thick, gray, dotted]
table {%
0 39.96
0.01 39.92
0.02 39.88
0.03 39.84
0.04 39.8
0.05 39.76
0.06 39.72
0.07 39.68
0.08 39.64
0.09 39.6
0.1 39.56
0.11 39.52
0.12 39.48
0.13 39.44
0.14 39.4
0.15 39.36
0.16 39.32
0.17 39.28
0.18 39.24
0.19 39.2
0.2 39.16
0.21 39.12
0.22 39.08
0.23 39.04
0.24 39
0.25 38.96
0.26 38.92
0.27 38.88
0.28 38.84
0.29 38.8
0.3 38.76
0.31 38.72
0.32 38.68
0.33 38.64
0.34 38.6
0.35 38.56
0.36 38.52
0.37 38.48
0.38 38.44
0.39 38.4
0.4 38.36
0.41 38.32
0.42 38.28
0.43 38.24
0.44 38.2
0.45 38.16
0.46 38.12
0.47 38.08
0.48 38.04
0.49 38
0.5 37.96
0.51 37.92
0.52 37.88
0.53 37.84
0.54 37.8
0.55 37.76
0.56 37.72
0.57 37.68
0.58 37.64
0.59 37.6
0.6 37.56
0.61 37.52
0.62 37.48
0.63 37.44
0.64 37.4
0.65 37.36
0.66 37.32
0.67 37.28
0.68 37.24
0.69 37.2
0.7 37.16
0.71 37.12
0.72 37.08
0.73 37.04
0.74 37
0.75 36.96
0.76 36.92
0.77 36.88
0.78 36.84
0.79 36.8
0.8 36.76
0.81 36.72
0.82 36.68
0.83 36.64
0.84 36.6
0.85 36.56
0.86 36.52
0.87 36.48
0.88 36.44
0.89 36.4
0.9 36.36
0.91 36.32
0.92 36.28
0.93 36.24
0.94 36.2
0.95 36.16
0.96 36.12
0.97 36.08
0.98 36.04
0.99 36
1 35.96
1.01 35.92
1.02 35.88
1.03 35.84
1.04 35.8
1.05 35.76
1.06 35.72
1.07 35.68
1.08 35.64
1.09 35.6
1.1 35.56
1.11 35.52
1.12 35.48
1.13 35.44
1.14 35.4
1.15 35.36
1.16 35.32
1.17 35.28
1.18 35.24
1.19 35.2
1.2 35.16
1.21 35.12
1.22 35.08
1.23 35.04
1.24 35
1.25 34.96
1.26 34.92
1.27 34.88
1.28 34.84
1.29 34.8
1.3 34.76
1.31 34.72
1.32 34.68
1.33 34.64
1.34 34.6
1.35 34.56
1.36 34.52
1.37 34.48
1.38 34.44
1.39 34.4
1.4 34.36
1.41 34.32
1.42 34.28
1.43 34.24
1.44 34.2
1.45 34.16
1.46 34.12
1.47 34.08
1.48 34.04
1.49 34
1.5 33.96
1.51 33.92
1.52 33.88
1.53 33.84
1.54 33.8
1.55 33.76
1.56 33.72
1.57 33.68
1.58 33.64
1.59 33.6
1.6 33.56
1.61 33.52
1.62 33.48
1.63 33.44
1.64 33.4
1.65 33.36
1.66 33.32
1.67 33.28
1.68 33.24
1.69 33.2
1.7 33.16
1.71 33.12
1.72 33.08
1.73 33.04
1.74 33
1.75 32.96
1.76 32.92
1.77 32.88
1.78 32.84
1.79 32.8
1.8 32.76
1.81 32.72
1.82 32.68
1.83 32.64
1.84 32.6
1.85 32.56
1.86 32.52
1.87 32.48
1.88 32.44
1.89 32.4
1.9 32.36
1.91 32.32
1.92 32.28
1.93 32.24
1.94 32.2
1.95 32.16
1.96 32.12
1.97 32.08
1.98 32.04
1.99 32
2 31.96
2.01 31.9200024673884
2.02 31.8800123367899
2.03 31.8400345423724
2.04 31.8000740175432
2.05 31.7601356946439
2.06 31.720224504647
2.07 31.6803453768509
2.08 31.6405032385767
2.09 31.6007030148634
2.1 31.5609496281647
2.11 31.5212479980453
2.12 31.481603040877
2.13 31.4420196695361
2.14 31.4025027930997
2.15 31.3630573165433
2.16 31.3236881404381
2.17 31.2844001606489
2.18 31.2451982680315
2.19 31.206087348132
2.2 31.1670722808844
2.21 31.1281579403102
2.22 31.0893491942178
2.23 31.0506509039019
2.24 31.0120679238436
2.25 30.9736051014113
2.26 30.9352672765616
2.27 30.8970592815404
2.28 30.8589859405853
2.29 30.8210520696277
2.3 30.7832624759959
2.31 30.7456219581184
2.32 30.7081353052281
2.33 30.6708072970668
2.34 30.6336427035902
2.35 30.5966462846739
2.36 30.5598227898197
2.37 30.5231769578627
2.38 30.4867135166789
2.39 30.4504371828933
2.4 30.4143526615897
2.41 30.3784646460197
2.42 30.3427778173135
2.43 30.3072968441912
2.44 30.2720263826749
2.45 30.2369710758011
2.46 30.2021355533347
2.47 30.1675244314833
2.48 30.1331423126122
2.49 30.098993784961
2.5 30.0650834223601
2.51 30.0314157839488
2.52 29.9979954138941
2.53 29.9648268411102
2.54 29.9319145789794
2.55 29.8992631250733
2.56 29.8668769608761
2.57 29.8347605515071
2.58 29.8029183454462
2.59 29.7713547742592
2.6 29.7400742523241
2.61 29.7090811765594
2.62 29.6783799261523
2.63 29.647974862289
2.64 29.6178703278855
2.65 29.5880706473197
2.66 29.5585801261646
2.67 29.529403050923
2.68 29.5005436887627
2.69 29.4720062872535
2.7 29.4437950741051
2.71 29.4159142569066
2.72 29.3883680228665
2.73 29.3611605385545
2.74 29.3342959496451
2.75 29.3077783806609
2.76 29.2816119347189
2.77 29.2558006932773
2.78 29.2303487158832
2.79 29.2052600399232
2.8 29.1805386803732
2.81 29.1561886295516
2.82 29.1322138568726
2.83 29.1086183086016
2.84 29.0854059076116
2.85 29.0625805531411
2.86 29.040146120554
2.87 29.0181064611003
2.88 28.9964654016782
2.89 28.9752267445987
2.9 28.9543942673507
2.91 28.9339717223679
2.92 28.9139628367975
2.93 28.8943713122702
2.94 28.8752008246719
2.95 28.856455023917
2.96 28.8381375337233
2.97 28.8202519513882
2.98 28.8028018475672
2.99 28.7857907660532
3 28.7692222235582
3.01 28.753097488852
3.02 28.7374155827437
3.03 28.7221754989074
3.04 28.7073762040652
3.05 28.6930166381706
3.06 28.6790957145953
3.07 28.6656123203158
3.08 28.6525653161029
3.09 28.6399535367122
3.1 28.627775791076
3.11 28.6160308624968
3.12 28.6047175088429
3.13 28.5938344627443
3.14 28.5833804317909
3.15 28.573354098732
3.16 28.5637541216766
3.17 28.5545791342962
3.18 28.5458277460276
3.19 28.537498542278
3.2 28.529590084631
3.21 28.5221009110541
3.22 28.5150295361069
3.23 28.5083744511514
3.24 28.5021341245625
3.25 28.4963070019408
3.26 28.4908915063255
3.27 28.4858860384089
3.28 28.4812889767522
3.29 28.4770986780022
3.3 28.4733134771088
3.31 28.4699316875439
3.32 28.4669516015212
3.33 28.464371490217
3.34 28.4621896039919
3.35 28.4604041726135
3.36 28.45901340548
3.37 28.4580154918447
3.38 28.4574086010413
3.39 28.4571908827101
3.4 28.4573604670254
3.41 28.4579154649226
3.42 28.4588539683277
3.43 28.4601740503857
3.44 28.4618737656915
3.45 28.4639511505203
3.46 28.466404223059
3.47 28.4692309836383
3.48 28.4724294149659
3.49 28.475997482359
3.5 28.4799331339793
3.51 28.4842343010663
3.52 28.4888988981732
3.53 28.4939248234023
3.54 28.4993099586404
3.55 28.5050521697961
3.56 28.5111493070362
3.57 28.5175992050229
3.58 28.524399683152
3.59 28.5315485457906
3.6 28.5390435825157
3.61 28.5468825683528
3.62 28.5550632640149
3.63 28.5635834161419
3.64 28.5724407575402
3.65 28.5816330074222
3.66 28.5911578716466
3.67 28.6010130429583
3.68 28.6111962012287
3.69 28.6217050136961
3.7 28.6325371352064
3.71 28.6436902084535
3.72 28.6551618642201
3.73 28.6669497216183
3.74 28.6790513883303
3.75 28.691464460849
3.76 28.7041865247189
3.77 28.7172151547767
3.78 28.7305479153912
3.79 28.7441823607048
3.8 28.7581160348726
3.81 28.7723464723035
3.82 28.7868711978996
3.83 28.8016877272961
3.84 28.8167935671009
3.85 28.8321862151339
3.86 28.847863160666
3.87 28.8638218846579
3.88 28.8800598599984
3.89 28.896574551743
3.9 28.9133634173509
3.91 28.9304239069232
3.92 28.9477534634394
3.93 28.9653495229939
3.94 28.9832095150328
3.95 29.0013308625889
3.96 29.0197109825171
3.97 29.0383472857295
3.98 29.057237177429
3.99 29.0763780573434
4 29.095767319958
4.01 29.1154023547487
4.02 29.1352805464133
4.03 29.1553992751033
4.04 29.175755916654
4.05 29.1963478428146
4.06 29.2171724214777
4.07 29.2382270169076
4.08 29.259508989968
4.09 29.2810156983494
4.1 29.3027444967952
4.11 29.324692737327
4.12 29.3468577694699
4.13 29.3692369404757
4.14 29.3918275955464
4.15 29.4146270780561
4.16 29.4376327297723
4.17 29.4608418910765
4.18 29.4842519011835
4.19 29.5078600983597
4.2 29.5316638201412
4.21 29.5556604035498
4.22 29.5798471853089
4.23 29.6042215020583
4.24 29.628780690567
4.25 29.6535220879465
4.26 29.6784430318619
4.27 29.7035408607422
4.28 29.7288129139896
4.29 29.7542565321877
4.3 29.7798690573084
4.31 29.805647832918
4.32 29.8315902043816
4.33 29.8576935190666
4.34 29.8839551265455
4.35 29.9103723787963
4.36 29.9369426304029
4.37 29.9636632387536
4.38 29.9905315642382
4.39 30.0175449704445
4.4 30.0447008243527
4.41 30.0719964965293
4.42 30.099429361319
4.43 30.1269967970355
4.44 30.1546961861513
4.45 30.1825249154853
4.46 30.2104803763902
4.47 30.2385599649373
4.48 30.2667610821008
4.49 30.2950811339403
4.5 30.3235175317815
4.51 30.3520676923965
4.52 30.3807290381815
4.53 30.4094989973338
4.54 30.438375004027
4.55 30.4673544985845
4.56 30.4964349276523
4.57 30.5256137443691
4.58 30.554888408536
4.59 30.5842563867838
4.6 30.6137151527394
4.61 30.6432621871902
4.62 30.6728949782469
4.63 30.7026110215053
4.64 30.7324078202056
4.65 30.762282885391
4.66 30.7922337360638
4.67 30.822257899341
4.68 30.8523529106069
4.69 30.882516313665
4.7 30.9127456608884
4.71 30.9430385133674
4.72 30.9733924410568
4.73 31.0038050229203
4.74 31.0342738470744
4.75 31.0647965109296
4.76 31.0953706213302
4.77 31.1259937946929
4.78 31.1566636571426
4.79 31.1873778446477
4.8 31.2181340031529
4.81 31.2489297887102
4.82 31.2797628676086
4.83 31.3106309165019
4.84 31.3415316225344
4.85 31.3724626834653
4.86 31.4034218077908
4.87 31.4344067148652
4.88 31.4654151350193
4.89 31.4964448096779
4.9 31.5274934914745
4.91 31.5585589443651
4.92 31.5896389437398
4.93 31.6207312765326
4.94 31.6518337413292
4.95 31.6829441484735
4.96 31.714060320172
4.97 31.7451800905957
4.98 31.7763013059817
4.99 31.8074218247313
5 31.8385395175075
5.01 31.8696522673299
5.02 31.9007579696682
5.03 31.9318545325336
5.04 31.9629398765686
5.05 31.9940119351345
5.06 32.0250686543976
5.07 32.056107993413
5.08 32.0871279242074
5.09 32.1181264318586
5.1 32.1491015145749
5.11 32.180051183771
5.12 32.2109734641434
5.13 32.2418663937429
5.14 32.272728024046
5.15 32.3035564200236
5.16 32.3343496602092
5.17 32.3651058367635
5.18 32.3958230555386
5.19 32.4264994361396
5.2 32.4571331119845
5.21 32.4877222303623
5.22 32.5182649524893
5.23 32.5487594535633
5.24 32.5792039228162
5.25 32.6095965635647
5.26 32.6399355932588
5.27 32.6702192435294
5.28 32.7004457602328
5.29 32.7306134034945
5.3 32.7607204477502
5.31 32.7907651817858
5.32 32.820745908775
5.33 32.8506609463154
5.34 32.8805086264624
5.35 32.9102872957619
5.36 32.9399953152807
5.37 32.969631060635
5.38 32.9991929220177
5.39 33.0286793042232
5.4 33.0580886266708
5.41 33.0874193234265
5.42 33.1166698432223
5.43 33.1458386494746
5.44 33.1749242203001
5.45 33.2039250485306
5.46 33.2328396417254
5.47 33.261666522182
5.48 33.2904042269461
5.49 33.3190513078182
5.5 33.3476063313598
5.51 33.3760678788972
5.52 33.4044345465238
5.53 33.4327049451009
5.54 33.4608777002562
5.55 33.4889514523811
5.56 33.5169248566262
5.57 33.5447965828952
5.58 33.572565315837
5.59 33.6002297548359
5.6 33.6277886140007
5.61 33.6552406221519
5.62 33.6825845228069
5.63 33.7098190741644
5.64 33.7369430490861
5.65 33.7639552350781
5.66 33.7908544342691
5.67 33.8176394633889
5.68 33.8443091537433
5.69 33.8708623511893
5.7 33.8972979161074
5.71 33.923614723373
5.72 33.9498116623259
5.73 33.9758876367391
5.74 34.0018415647846
5.75 34.0276723789992
5.76 34.0533790262478
5.77 34.0789604676859
5.78 34.10441567872
5.79 34.1297436489672
5.8 34.1549433822126
5.81 34.180013896366
5.82 34.2049542234167
5.83 34.2297634093871
5.84 34.2544405142847
5.85 34.2789846120531
5.86 34.3033947905211
5.87 34.3276701513509
5.88 34.3518098099849
5.89 34.3758128955907
5.9 34.3996785510051
5.91 34.4234059326773
5.92 34.4469942106099
5.93 34.4704425682989
5.94 34.4937502026729
5.95 34.5169163240307
5.96 34.5399401559773
5.97 34.5628209353593
5.98 34.5855579121988
5.99 34.6081503496262
6 34.6305975238114
6.01 34.6529008828595
6.02 34.6750640466064
6.03 34.6970906473159
6.04 34.7189843293402
6.05 34.7407487487797
6.06 34.7623875731412
6.07 34.7839044809955
6.08 34.8053031616341
6.09 34.8265873147244
6.1 34.8477606499648
6.11 34.8688268867382
6.12 34.889789753765
6.13 34.9106529887553
6.14 34.9314203380601
6.15 34.9520955563219
6.16 34.9726824061247
6.17 34.9931846576428
6.18 35.0136060882898
6.19 35.0339504823662
6.2 35.0542216307064
6.21 35.074423330326
6.22 35.0945593840677
6.23 35.1146336002469
6.24 35.134649792297
6.25 35.1546117784144
6.26 35.1745233812026
6.27 35.1943884273163
6.28 35.2142107471054
6.29 35.233994174258
6.3 35.2537425454437
6.31 35.2734596999565
6.32 35.2931494793576
6.33 35.3128157271177
6.34 35.3324613013022
6.35 35.3520858536668
6.36 35.3716887905235
6.37 35.3912695195182
6.38 35.4108274496662
6.39 35.4303619913893
6.4 35.4498725565523
6.41 35.4693585584986
6.42 35.4888194120868
6.43 35.5082545337271
6.44 35.5276633414167
6.45 35.5470452547759
6.46 35.5663996950842
6.47 35.5857260853156
6.48 35.6050238501745
6.49 35.6242924161309
6.5 35.643531211456
6.51 35.6627396662574
6.52 35.6819172125143
6.53 35.7010632841122
6.54 35.7201773168786
6.55 35.7392587486168
6.56 35.7583070191415
6.57 35.7773215703126
6.58 35.7963018460702
6.59 35.8152472924686
6.6 35.8341573577105
6.61 35.8530314921811
6.62 35.871869148482
6.63 35.8906697814649
6.64 35.9094328482653
6.65 35.9281578083361
6.66 35.9468441234804
6.67 35.9654912578856
6.68 35.9840986781557
6.69 36.0026658533443
6.7 36.0211922549879
6.71 36.0396773571377
6.72 36.0581206363927
6.73 36.0765215719317
6.74 36.0948796455454
6.75 36.1131943416684
6.76 36.1314651474111
6.77 36.1496915525914
6.78 36.1678730497656
6.79 36.1860091342606
6.8 36.2040993042044
6.81 36.2221430605571
6.82 36.2401399071419
6.83 36.2580893506758
6.84 36.2759909007996
6.85 36.2938440701084
6.86 36.3116483741818
6.87 36.3294033316135
6.88 36.3471084640412
6.89 36.3647632961762
6.9 36.3823673558322
6.91 36.399920173955
6.92 36.4174212846513
6.93 36.4348702252172
6.94 36.452266536167
6.95 36.4696097612614
6.96 36.4868994475356
6.97 36.5041351453275
6.98 36.5213164083051
6.99 36.5384427934944
7 36.5555138613062
7.01 36.5725291755639
7.02 36.5894883035299
7.03 36.6063908159322
7.04 36.6232362869915
7.05 36.6400242944468
7.06 36.6567544195819
7.07 36.673426247251
7.08 36.6900393659042
7.09 36.7065933676133
7.1 36.7230878480966
7.11 36.739522406744
7.12 36.7558966466415
7.13 36.7722101745959
7.14 36.7884626011593
7.15 36.8046535406524
7.16 36.8207826111889
7.17 36.836849434699
7.18 36.8528536369522
7.19 36.8687948475812
7.2 36.8846727001042
7.21 36.9004868319474
7.22 36.9162368844678
7.23 36.9319225029751
7.24 36.9475433367537
7.25 36.9630990390839
7.26 36.9785892672638
7.27 36.9940136826305
7.28 37.0093719505803
7.29 37.02466374059
7.3 37.0398887262372
7.31 37.05504658522
7.32 37.0701369993775
7.33 37.0851596547089
7.34 37.1001142413933
7.35 37.1150004538082
7.36 37.1298179905492
7.37 37.144566554448
7.38 37.1592458525906
7.39 37.1738555963359
7.4 37.1883955013333
7.41 37.20286528754
7.42 37.2172646792386
7.43 37.2315934050538
7.44 37.2458511979694
7.45 37.2600377953448
7.46 37.2741529389309
7.47 37.2881963748861
7.48 37.3021678537925
7.49 37.3160671306704
7.5 37.329893964994
7.51 37.3436481207063
7.52 37.3573293662331
7.53 37.370937474498
7.54 37.3844722229358
7.55 37.3979333935066
7.56 37.411320772709
7.57 37.4246341515937
7.58 37.4378733257756
7.59 37.4510380954475
7.6 37.4641282653912
7.61 37.4771436449906
7.62 37.490084048243
7.63 37.5029492937702
7.64 37.5157392048306
7.65 37.5284536093293
7.66 37.5410923398289
7.67 37.5536552335603
7.68 37.5661421324319
7.69 37.5785528830403
7.7 37.5908873366788
7.71 37.6031453493475
7.72 37.6153267817615
7.73 37.6274314993597
7.74 37.6394593723132
7.75 37.6514102755334
7.76 37.6632840886793
7.77 37.6750806961652
7.78 37.686799987168
7.79 37.6984418556337
7.8 37.7100062002841
7.81 37.721492924623
7.82 37.7329019369424
7.83 37.7442331503277
7.84 37.7554864826633
7.85 37.7666618566379
7.86 37.7777591997488
7.87 37.7887784443068
7.88 37.79971952744
7.89 37.8105823910979
7.9 37.8213669820552
7.91 37.8320732519145
7.92 37.8427011571098
7.93 37.8532506589089
7.94 37.863721723416
7.95 37.8741143215736
7.96 37.8844284291644
7.97 37.8946640268127
7.98 37.9048210999859
7.99 37.9148996389948
8 37.9248996389948
8.01 37.9348210999859
8.02 37.9446640268128
8.03 37.9544284291644
8.04 37.9641143215736
8.05 37.973721723416
8.06 37.9832506589089
8.07 37.9927011571098
8.08 38.0020732519145
8.09 38.0113669820552
8.1 38.0205823910979
8.11 38.02971952744
8.12 38.0387784443068
8.13 38.0477591997488
8.14 38.0566618566379
8.15 38.0654864826633
8.16 38.0742331503277
8.17 38.0829019369424
8.18 38.0914929246231
8.19 38.1000062002841
8.2 38.1084418556337
8.21 38.116799987168
8.22 38.1250806961652
8.23 38.1332840886793
8.24 38.1414102755334
8.25 38.1494593723132
8.26 38.1574314993597
8.27 38.1653267817615
8.28 38.1731453493475
8.29 38.1808873366788
8.3 38.1885528830403
8.31 38.1961421324319
8.32 38.2036552335603
8.33 38.2110923398289
8.34 38.2184536093293
8.35 38.2257392048306
8.36 38.2329492937702
8.37 38.240084048243
8.38 38.2471436449906
8.39 38.2541282653912
8.4 38.2610380954475
8.41 38.2678733257756
8.42 38.2746341515937
8.43 38.281320772709
8.44 38.2879333935066
8.45 38.2944722229358
8.46 38.300937474498
8.47 38.3073293662331
8.48 38.3136481207063
8.49 38.319893964994
8.5 38.3260671306704
8.51 38.3321678537925
8.52 38.3381963748862
8.53 38.3441529389309
8.54 38.3500377953448
8.55 38.3558511979694
8.56 38.3615934050538
8.57 38.3672646792386
8.58 38.37286528754
8.59 38.3783955013333
8.6 38.3838555963359
8.61 38.3892458525906
8.62 38.394566554448
8.63 38.3998179905493
8.64 38.4050004538082
8.65 38.4101142413933
8.66 38.4151596547089
8.67 38.4201369993775
8.68 38.42504658522
8.69 38.4298887262372
8.7 38.43466374059
8.71 38.4393719505803
8.72 38.4440136826305
8.73 38.4485892672638
8.74 38.4530990390838
8.75 38.4575433367537
8.76 38.4619225029751
8.77 38.4662368844678
8.78 38.4704868319474
8.79 38.4746727001042
8.8 38.4787948475812
8.81 38.4828536369522
8.82 38.486849434699
8.83 38.4907826111889
8.84 38.4946535406524
8.85 38.4984626011593
8.86 38.5022101745959
8.87 38.5058966466415
8.88 38.509522406744
8.89 38.5130878480966
8.9 38.5165933676133
8.91 38.5200393659042
8.92 38.523426247251
8.93 38.5267544195819
8.94 38.5300242944468
8.95 38.5332362869915
8.96 38.5363908159322
8.97 38.5394883035299
8.98 38.5425291755639
8.99 38.5455138613062
9 38.5484427934944
9.01 38.5513164083051
9.02 38.5541351453275
9.03 38.5568994475356
9.04 38.5596097612614
9.05 38.562266536167
9.06 38.5648702252172
9.07 38.5674212846513
9.08 38.569920173955
9.09 38.5723673558322
9.1 38.5747632961762
9.11 38.5771084640412
9.12 38.5794033316135
9.13 38.5816483741818
9.14 38.5838440701084
9.15 38.5859909007996
9.16 38.5880893506758
9.17 38.5901399071419
9.18 38.5921430605571
9.19 38.5940993042044
9.2 38.5960091342606
9.21 38.5978730497656
9.22 38.5996915525914
9.23 38.6014651474112
9.24 38.6031943416684
9.25 38.6048796455454
9.26 38.6065215719317
9.27 38.6081206363927
9.28 38.6096773571377
9.29 38.6111922549879
9.3 38.6126658533443
9.31 38.6140986781557
9.32 38.6154912578856
9.33 38.6168441234804
9.34 38.6181578083361
9.35 38.6194328482653
9.36 38.6206697814649
9.37 38.621869148482
9.38 38.6230314921811
9.39 38.6241573577105
9.4 38.6252472924686
9.41 38.6263018460702
9.42 38.6273215703126
9.43 38.6283070191415
9.44 38.6292587486168
9.45 38.6301773168786
9.46 38.6310632841122
9.47 38.6319172125143
9.48 38.6327396662574
9.49 38.633531211456
9.5 38.6342924161309
9.51 38.6350238501745
9.52 38.6357260853156
9.53 38.6363996950842
9.54 38.6370452547759
9.55 38.6376633414167
9.56 38.6382545337271
9.57 38.6388194120868
9.58 38.6393585584986
9.59 38.6398725565523
9.6 38.6403619913893
9.61 38.6408274496662
9.62 38.6412695195182
9.63 38.6416887905236
9.64 38.6420858536668
9.65 38.6424613013022
9.66 38.6428157271177
9.67 38.6431497260975
9.68 38.6434638944862
9.69 38.6437588297515
9.7 38.6440351305476
9.71 38.6442933966779
9.72 38.6445342290585
9.73 38.6447582296808
9.74 38.6449660015746
9.75 38.6451581487706
9.76 38.6453352762633
9.77 38.6454979899738
9.78 38.6456468967122
9.79 38.6457826041405
9.8 38.6459057207345
9.81 38.6460168557471
9.82 38.6461166191699
9.83 38.6462056216963
9.84 38.6462844746831
9.85 38.6463537901136
9.86 38.646414180559
9.87 38.6464662591414
9.88 38.6465106394954
9.89 38.6465479357304
9.9 38.6465787623931
9.91 38.6466037344289
9.92 38.6466234671447
9.93 38.6466385761701
9.94 38.6466496774205
9.95 38.6466573870581
9.96 38.6466623214545
9.97 38.6466650971523
9.98 38.6466663308275
9.99 38.646666639251
10 38.646666639251
10.01 38.646666639251
10.02 38.646666639251
10.03 38.646666639251
10.04 38.646666639251
10.05 38.646666639251
10.06 38.646666639251
10.07 38.646666639251
10.08 38.646666639251
10.09 38.646666639251
10.1 38.646666639251
10.11 38.646666639251
10.12 38.646666639251
10.13 38.646666639251
10.14 38.646666639251
10.15 38.646666639251
10.16 38.646666639251
10.17 38.646666639251
10.18 38.646666639251
10.19 38.646666639251
10.2 38.646666639251
10.21 38.646666639251
10.22 38.646666639251
10.23 38.646666639251
10.24 38.646666639251
10.25 38.646666639251
10.26 38.646666639251
10.27 38.646666639251
10.28 38.646666639251
10.29 38.646666639251
10.3 38.646666639251
10.31 38.646666639251
10.32 38.646666639251
10.33 38.646666639251
10.34 38.646666639251
10.35 38.646666639251
10.36 38.646666639251
10.37 38.646666639251
10.38 38.646666639251
10.39 38.646666639251
10.4 38.646666639251
10.41 38.646666639251
10.42 38.646666639251
10.43 38.646666639251
10.44 38.646666639251
10.45 38.646666639251
10.46 38.646666639251
10.47 38.646666639251
10.48 38.646666639251
10.49 38.646666639251
10.5 38.646666639251
10.51 38.646666639251
10.52 38.646666639251
10.53 38.646666639251
10.54 38.646666639251
10.55 38.646666639251
10.56 38.646666639251
10.57 38.646666639251
10.58 38.646666639251
10.59 38.646666639251
10.6 38.646666639251
10.61 38.646666639251
10.62 38.646666639251
10.63 38.646666639251
10.64 38.646666639251
10.65 38.646666639251
10.66 38.646666639251
10.67 38.646666639251
10.68 38.646666639251
10.69 38.646666639251
10.7 38.646666639251
10.71 38.646666639251
10.72 38.646666639251
10.73 38.646666639251
10.74 38.646666639251
10.75 38.646666639251
10.76 38.646666639251
10.77 38.646666639251
10.78 38.646666639251
10.79 38.646666639251
10.8 38.646666639251
10.81 38.646666639251
10.82 38.646666639251
10.83 38.646666639251
10.84 38.646666639251
10.85 38.646666639251
10.86 38.646666639251
10.87 38.646666639251
10.88 38.646666639251
10.89 38.646666639251
10.9 38.646666639251
10.91 38.646666639251
10.92 38.646666639251
10.93 38.646666639251
10.94 38.646666639251
10.95 38.646666639251
10.96 38.646666639251
10.97 38.646666639251
10.98 38.646666639251
10.99 38.646666639251
11 38.646666639251
11.01 38.646666639251
11.02 38.646666639251
11.03 38.646666639251
11.04 38.646666639251
11.05 38.646666639251
11.06 38.646666639251
11.07 38.646666639251
11.08 38.646666639251
11.09 38.646666639251
11.1 38.646666639251
11.11 38.646666639251
11.12 38.646666639251
11.13 38.646666639251
11.14 38.646666639251
11.15 38.646666639251
11.16 38.646666639251
11.17 38.646666639251
11.18 38.646666639251
11.19 38.646666639251
11.2 38.646666639251
11.21 38.646666639251
11.22 38.646666639251
11.23 38.646666639251
11.24 38.646666639251
11.25 38.646666639251
11.26 38.646666639251
11.27 38.646666639251
11.28 38.646666639251
11.29 38.646666639251
11.3 38.646666639251
11.31 38.646666639251
11.32 38.646666639251
11.33 38.646666639251
11.34 38.646666639251
11.35 38.646666639251
11.36 38.646666639251
11.37 38.646666639251
11.38 38.646666639251
11.39 38.646666639251
11.4 38.646666639251
11.41 38.646666639251
11.42 38.646666639251
11.43 38.646666639251
11.44 38.646666639251
11.45 38.646666639251
11.46 38.646666639251
11.47 38.646666639251
11.48 38.646666639251
11.49 38.646666639251
11.5 38.646666639251
11.51 38.646666639251
11.52 38.646666639251
11.53 38.646666639251
11.54 38.646666639251
11.55 38.646666639251
11.56 38.646666639251
11.57 38.646666639251
11.58 38.646666639251
11.59 38.646666639251
11.6 38.646666639251
11.61 38.646666639251
11.62 38.646666639251
11.63 38.646666639251
11.64 38.646666639251
11.65 38.646666639251
11.66 38.646666639251
11.67 38.646666639251
11.68 38.646666639251
11.69 38.646666639251
11.7 38.646666639251
11.71 38.646666639251
11.72 38.646666639251
11.73 38.646666639251
11.74 38.646666639251
11.75 38.646666639251
11.76 38.646666639251
11.77 38.646666639251
11.78 38.646666639251
11.79 38.646666639251
11.8 38.646666639251
11.81 38.646666639251
11.82 38.646666639251
11.83 38.646666639251
11.84 38.646666639251
11.85 38.646666639251
11.86 38.646666639251
11.87 38.646666639251
11.88 38.646666639251
11.89 38.646666639251
11.9 38.646666639251
11.91 38.646666639251
11.92 38.646666639251
11.93 38.646666639251
11.94 38.646666639251
11.95 38.646666639251
11.96 38.646666639251
11.97 38.646666639251
11.98 38.646666639251
11.99 38.646666639251
12 38.646666639251
};
\end{axis}

\end{tikzpicture}

	% This file was created by tikzplotlib v0.9.8.
\begin{tikzpicture}

\begin{axis}[
height=\figureheight,
scaled y ticks=false,
tick align=outside,
tick pos=left,
width=\figurewidth,
x grid style={white!69.0196078431373!black},
xlabel={Time [\si{\second}]},
xmajorgrids,
xmin=0, xmax=12,
xtick style={color=black},
xticklabel style={align=center},
y grid style={white!69.0196078431373!black},
ylabel={Probability of collision},
ymajorgrids,
ymin=0, ymax=1,
ytick style={color=black},
yticklabel style={/pgf/number format/fixed,/pgf/number format/precision=3}
]
\addplot [very thick, gray]
table {%
0 4.44089209850063e-16
0.01 4.44089209850063e-16
0.02 4.44089209850063e-16
0.03 4.44089209850063e-16
0.04 4.44089209850063e-16
0.05 4.44089209850063e-16
0.06 4.44089209850063e-16
0.07 4.44089209850063e-16
0.08 4.44089209850063e-16
0.09 4.44089209850063e-16
0.1 4.44089209850063e-16
0.11 4.44089209850063e-16
0.12 4.44089209850063e-16
0.13 4.44089209850063e-16
0.14 4.44089209850063e-16
0.15 4.44089209850063e-16
0.16 4.44089209850063e-16
0.17 4.44089209850063e-16
0.18 4.44089209850063e-16
0.19 4.44089209850063e-16
0.2 4.44089209850063e-16
0.21 4.44089209850063e-16
0.22 4.44089209850063e-16
0.23 5.55111512312578e-16
0.24 5.55111512312578e-16
0.25 5.55111512312578e-16
0.26 5.55111512312578e-16
0.27 7.7715611723761e-16
0.28 7.7715611723761e-16
0.29 7.7715611723761e-16
0.3 7.7715611723761e-16
0.31 7.7715611723761e-16
0.32 7.7715611723761e-16
0.33 7.7715611723761e-16
0.34 7.7715611723761e-16
0.35 7.7715611723761e-16
0.36 7.7715611723761e-16
0.37 8.88178419700125e-16
0.38 8.88178419700125e-16
0.39 8.88178419700125e-16
0.4 9.99200722162641e-16
0.41 9.99200722162641e-16
0.42 9.99200722162641e-16
0.43 9.99200722162641e-16
0.44 1.22124532708767e-15
0.45 1.22124532708767e-15
0.46 1.22124532708767e-15
0.47 1.22124532708767e-15
0.48 1.22124532708767e-15
0.49 1.33226762955019e-15
0.5 1.33226762955019e-15
0.51 1.33226762955019e-15
0.52 1.33226762955019e-15
0.53 1.4432899320127e-15
0.54 1.4432899320127e-15
0.55 1.4432899320127e-15
0.56 1.66533453693773e-15
0.57 1.66533453693773e-15
0.58 1.66533453693773e-15
0.59 1.66533453693773e-15
0.6 1.66533453693773e-15
0.61 1.88737914186277e-15
0.62 1.88737914186277e-15
0.63 1.88737914186277e-15
0.64 1.88737914186277e-15
0.65 1.88737914186277e-15
0.66 2.22044604925031e-15
0.67 2.22044604925031e-15
0.68 2.22044604925031e-15
0.69 2.22044604925031e-15
0.7 2.22044604925031e-15
0.71 2.55351295663786e-15
0.72 2.55351295663786e-15
0.73 2.66453525910038e-15
0.74 2.88657986402541e-15
0.75 2.88657986402541e-15
0.76 2.99760216648792e-15
0.77 2.99760216648792e-15
0.78 3.10862446895044e-15
0.79 3.10862446895044e-15
0.8 3.33066907387547e-15
0.81 3.33066907387547e-15
0.82 3.44169137633799e-15
0.83 3.5527136788005e-15
0.84 3.77475828372553e-15
0.85 3.88578058618805e-15
0.86 3.99680288865056e-15
0.87 3.99680288865056e-15
0.88 4.32986979603811e-15
0.89 4.32986979603811e-15
0.9 4.32986979603811e-15
0.91 4.66293670342566e-15
0.92 4.66293670342566e-15
0.93 4.9960036108132e-15
0.94 4.9960036108132e-15
0.95 5.10702591327572e-15
0.96 5.21804821573824e-15
0.97 5.55111512312578e-15
0.98 5.6621374255883e-15
0.99 5.88418203051333e-15
1 6.10622663543836e-15
1.01 6.10622663543836e-15
1.02 6.32827124036339e-15
1.03 6.43929354282591e-15
1.04 6.77236045021345e-15
1.05 6.99440505513849e-15
1.06 7.21644966006352e-15
1.07 7.32747196252603e-15
1.08 7.66053886991358e-15
1.09 7.99360577730113e-15
1.1 8.21565038222616e-15
1.11 8.43769498715119e-15
1.12 8.65973959207622e-15
1.13 8.99280649946377e-15
1.14 9.32587340685131e-15
1.15 9.43689570931383e-15
1.16 9.76996261670138e-15
1.17 1.02140518265514e-14
1.18 1.0547118733939e-14
1.19 1.0769163338864e-14
1.2 1.11022302462516e-14
1.21 1.14352971536391e-14
1.22 1.16573417585641e-14
1.23 1.21014309684142e-14
1.24 1.24344978758018e-14
1.25 1.28785870856518e-14
1.26 1.33226762955019e-14
1.27 1.37667655053519e-14
1.28 1.4210854715202e-14
1.29 1.45439216225896e-14
1.3 1.50990331349021e-14
1.31 1.56541446472147e-14
1.32 1.60982338570648e-14
1.33 1.66533453693773e-14
1.34 1.70974345792274e-14
1.35 1.75415237890775e-14
1.36 1.80966353013901e-14
1.37 1.87627691161651e-14
1.38 1.93178806284777e-14
1.39 1.99840144432528e-14
1.4 2.05391259555654e-14
1.41 2.12052597703405e-14
1.42 2.18713935851156e-14
1.43 2.25375273998907e-14
1.44 2.32036612146658e-14
1.45 2.39808173319034e-14
1.46 2.4757973449141e-14
1.47 2.56461518688411e-14
1.48 2.64233079860787e-14
1.49 2.73114864057789e-14
1.5 2.80886425230165e-14
1.51 2.89768209427166e-14
1.52 2.98649993624167e-14
1.53 3.08642000845794e-14
1.54 3.19744231092045e-14
1.55 3.28626015289046e-14
1.56 3.38618022510673e-14
1.57 3.49720252756924e-14
1.58 3.61932706027801e-14
1.59 3.74145159298678e-14
1.6 3.85247389544929e-14
1.61 3.96349619791181e-14
1.62 4.09672296086683e-14
1.63 4.21884749357559e-14
1.64 4.35207425653061e-14
1.65 4.50750547997814e-14
1.66 4.65183447317941e-14
1.67 4.78506123613442e-14
1.68 4.9515946898282e-14
1.69 5.11812814352197e-14
1.7 5.26245713672324e-14
1.71 5.42899059041702e-14
1.72 5.61772850460329e-14
1.73 5.77315972805081e-14
1.74 5.97299987248334e-14
1.75 6.18394224716212e-14
1.76 6.38378239159465e-14
1.77 6.57252030578093e-14
1.78 6.79456491070596e-14
1.79 7.00550728538474e-14
1.8 7.24975635080227e-14
1.81 7.46069872548105e-14
1.82 7.71605002114484e-14
1.83 7.97140131680862e-14
1.84 8.20454815197991e-14
1.85 8.49320613838245e-14
1.86 8.75965966429249e-14
1.87 9.04831765069503e-14
1.88 9.34807786734382e-14
1.89 9.64783808399261e-14
1.9 9.96980276113391e-14
1.91 1.02917674382752e-13
1.92 1.06248343456627e-13
1.93 1.09690034832965e-13
1.94 1.13353770814228e-13
1.95 1.16906484493029e-13
1.96 1.21014309684142e-13
1.97 1.2490009027033e-13
1.98 1.29118937763906e-13
1.99 1.33226762955019e-13
2 1.37889699658444e-13
2.01 1.41997524849558e-13
2.02 1.45994327738208e-13
2.03 1.49658063719471e-13
2.04 1.52655665885959e-13
2.05 1.55320201145059e-13
2.06 1.57651669496772e-13
2.07 1.59428026336172e-13
2.08 1.6064927166326e-13
2.09 1.61315405478035e-13
2.1 1.61648472385423e-13
2.11 1.61204383175573e-13
2.12 1.60427227058335e-13
2.13 1.58983937126322e-13
2.14 1.56874513379535e-13
2.15 1.54432022725359e-13
2.16 1.51545442861334e-13
2.17 1.48103751484996e-13
2.18 1.44440015503733e-13
2.19 1.40221168010157e-13
2.2 1.35780275911657e-13
2.21 1.30673249998381e-13
2.22 1.25455201782643e-13
2.23 1.20237153566904e-13
2.24 1.14686038443779e-13
2.25 1.08801856413265e-13
2.26 1.02917674382752e-13
2.27 9.72555369571637e-14
2.28 9.13713549266504e-14
2.29 8.53761505936745e-14
2.3 7.97140131680862e-14
2.31 7.39408534400354e-14
2.32 6.86117829218347e-14
2.33 6.31716901011714e-14
2.34 5.77315972805081e-14
2.35 5.28466159721575e-14
2.36 4.80726569662693e-14
2.37 4.35207425653061e-14
2.38 3.93018950717305e-14
2.39 3.530509218308e-14
2.4 3.15303338993544e-14
2.41 2.80886425230165e-14
2.42 2.48689957516035e-14
2.43 2.18713935851156e-14
2.44 1.92068583260152e-14
2.45 1.67643676718399e-14
2.46 1.45439216225896e-14
2.47 1.24344978758018e-14
2.48 1.0769163338864e-14
2.49 9.10382880192628e-15
2.5 7.99360577730113e-15
2.51 6.43929354282591e-15
2.52 5.55111512312578e-15
2.53 4.66293670342566e-15
2.54 3.99680288865056e-15
2.55 3.10862446895044e-15
2.56 2.55351295663786e-15
2.57 1.88737914186277e-15
2.58 1.66533453693773e-15
2.59 1.33226762955019e-15
2.6 9.99200722162641e-16
2.61 7.7715611723761e-16
2.62 5.55111512312578e-16
2.63 4.44089209850063e-16
2.64 4.44089209850063e-16
2.65 4.44089209850063e-16
2.66 1.11022302462516e-16
2.67 1.11022302462516e-16
2.68 1.11022302462516e-16
2.69 1.11022302462516e-16
2.7 0
2.71 0
2.72 0
2.73 0
2.74 0
2.75 0
2.76 0
2.77 0
2.78 0
2.79 0
2.8 0
2.81 0
2.82 0
2.83 0
2.84 0
2.85 0
2.86 0
2.87 0
2.88 0
2.89 0
2.9 0
2.91 0
2.92 0
2.93 0
2.94 0
2.95 0
2.96 0
2.97 0
2.98 0
2.99 0
3 0
3.01 0
3.02 0
3.03 0
3.04 0
3.05 0
3.06 0
3.07 0
3.08 0
3.09 0
3.1 0
3.11 0
3.12 0
3.13 0
3.14 0
3.15 0
3.16 0
3.17 0
3.18 0
3.19 0
3.2 0
3.21 0
3.22 0
3.23 0
3.24 0
3.25 0
3.26 0
3.27 0
3.28 0
3.29 0
3.3 0
3.31 0
3.32 0
3.33 0
3.34 0
3.35 0
3.36 0
3.37 0
3.38 0
3.39 0
3.4 0
3.41 0
3.42 0
3.43 0
3.44 0
3.45 0
3.46 0
3.47 0
3.48 0
3.49 0
3.5 0
3.51 0
3.52 0
3.53 0
3.54 0
3.55 0
3.56 0
3.57 0
3.58 0
3.59 0
3.6 0
3.61 0
3.62 0
3.63 0
3.64 0
3.65 0
3.66 0
3.67 0
3.68 0
3.69 0
3.7 0
3.71 0
3.72 0
3.73 0
3.74 0
3.75 0
3.76 0
3.77 0
3.78 0
3.79 0
3.8 0
3.81 0
3.82 0
3.83 0
3.84 0
3.85 0
3.86 0
3.87 0
3.88 0
3.89 0
3.9 0
3.91 0
3.92 0
3.93 0
3.94 0
3.95 0
3.96 0
3.97 0
3.98 0
3.99 0
4 0
4.01 0
4.02 0
4.03 0
4.04 0
4.05 0
4.06 0
4.07 0
4.08 0
4.09 0
4.1 0
4.11 0
4.12 0
4.13 0
4.14 0
4.15 0
4.16 0
4.17 0
4.18 0
4.19 0
4.2 0
4.21 0
4.22 0
4.23 0
4.24 0
4.25 0
4.26 0
4.27 0
4.28 0
4.29 0
4.3 0
4.31 0
4.32 0
4.33 0
4.34 0
4.35 0
4.36 0
4.37 0
4.38 0
4.39 0
4.4 0
4.41 0
4.42 0
4.43 0
4.44 0
4.45 0
4.46 0
4.47 0
4.48 0
4.49 0
4.5 0
4.51 0
4.52 0
4.53 0
4.54 0
4.55 0
4.56 0
4.57 0
4.58 0
4.59 0
4.6 0
4.61 0
4.62 0
4.63 0
4.64 0
4.65 0
4.66 0
4.67 0
4.68 0
4.69 0
4.7 0
4.71 0
4.72 0
4.73 0
4.74 0
4.75 0
4.76 0
4.77 0
4.78 0
4.79 0
4.8 0
4.81 0
4.82 0
4.83 0
4.84 0
4.85 0
4.86 0
4.87 0
4.88 0
4.89 0
4.9 0
4.91 0
4.92 0
4.93 0
4.94 0
4.95 0
4.96 0
4.97 0
4.98 0
4.99 0
5 0
5.01 0
5.02 0
5.03 0
5.04 0
5.05 0
5.06 0
5.07 0
5.08 0
5.09 0
5.1 0
5.11 0
5.12 0
5.13 0
5.14 0
5.15 0
5.16 0
5.17 0
5.18 0
5.19 0
5.2 0
5.21 0
5.22 0
5.23 0
5.24 0
5.25 0
5.26 0
5.27 0
5.28 0
5.29 0
5.3 0
5.31 0
5.32 0
5.33 0
5.34 0
5.35 0
5.36 0
5.37 0
5.38 0
5.39 0
5.4 0
5.41 0
5.42 0
5.43 0
5.44 0
5.45 0
5.46 0
5.47 0
5.48 0
5.49 0
5.5 0
5.51 0
5.52 0
5.53 0
5.54 0
5.55 0
5.56 0
5.57 0
5.58 0
5.59 0
5.6 0
5.61 0
5.62 0
5.63 0
5.64 0
5.65 0
5.66 0
5.67 0
5.68 0
5.69 0
5.7 0
5.71 0
5.72 0
5.73 0
5.74 0
5.75 0
5.76 0
5.77 0
5.78 0
5.79 0
5.8 0
5.81 0
5.82 0
5.83 0
5.84 0
5.85 0
5.86 0
5.87 0
5.88 0
5.89 0
5.9 0
5.91 0
5.92 0
5.93 0
5.94 0
5.95 0
5.96 0
5.97 0
5.98 0
5.99 0
6 0
6.01 0
6.02 0
6.03 0
6.04 0
6.05 0
6.06 0
6.07 0
6.08 0
6.09 0
6.1 0
6.11 0
6.12 0
6.13 0
6.14 0
6.15 0
6.16 0
6.17 0
6.18 0
6.19 0
6.2 0
6.21 0
6.22 0
6.23 0
6.24 0
6.25 0
6.26 0
6.27 0
6.28 0
6.29 0
6.3 0
6.31 0
6.32 0
6.33 0
6.34 0
6.35 0
6.36 0
6.37 0
6.38 0
6.39 0
6.4 0
6.41 0
6.42 0
6.43 0
6.44 0
6.45 0
6.46 0
6.47 0
6.48 0
6.49 0
6.5 0
6.51 0
6.52 0
6.53 0
6.54 0
6.55 0
6.56 0
6.57 0
6.58 0
6.59 0
6.6 0
6.61 0
6.62 0
6.63 0
6.64 0
6.65 0
6.66 0
6.67 0
6.68 0
6.69 0
6.7 0
6.71 0
6.72 0
6.73 0
6.74 0
6.75 0
6.76 0
6.77 0
6.78 0
6.79 0
6.8 0
6.81 0
6.82 0
6.83 0
6.84 0
6.85 0
6.86 0
6.87 0
6.88 0
6.89 0
6.9 0
6.91 0
6.92 0
6.93 0
6.94 0
6.95 0
6.96 0
6.97 0
6.98 0
6.99 0
7 0
7.01 0
7.02 0
7.03 0
7.04 0
7.05 0
7.06 0
7.07 0
7.08 0
7.09 0
7.1 0
7.11 0
7.12 0
7.13 0
7.14 0
7.15 0
7.16 0
7.17 0
7.18 0
7.19 0
7.2 0
7.21 0
7.22 0
7.23 0
7.24 0
7.25 0
7.26 0
7.27 0
7.28 0
7.29 0
7.3 0
7.31 0
7.32 0
7.33 0
7.34 0
7.35 0
7.36 0
7.37 0
7.38 0
7.39 0
7.4 0
7.41 0
7.42 0
7.43 0
7.44 0
7.45 0
7.46 0
7.47 0
7.48 0
7.49 0
7.5 0
7.51 0
7.52 0
7.53 0
7.54 0
7.55 0
7.56 0
7.57 0
7.58 0
7.59 0
7.6 0
7.61 0
7.62 0
7.63 0
7.64 0
7.65 0
7.66 0
7.67 0
7.68 0
7.69 0
7.7 0
7.71 0
7.72 0
7.73 0
7.74 0
7.75 0
7.76 0
7.77 0
7.78 0
7.79 0
7.8 0
7.81 0
7.82 0
7.83 0
7.84 0
7.85 0
7.86 0
7.87 0
7.88 0
7.89 0
7.9 0
7.91 0
7.92 0
7.93 0
7.94 0
7.95 0
7.96 0
7.97 0
7.98 0
7.99 0
8 0
8.01 0
8.02 0
8.03 0
8.04 0
8.05 0
8.06 0
8.07 0
8.08 0
8.09 0
8.1 0
8.11 0
8.12 0
8.13 0
8.14 0
8.15 0
8.16 0
8.17 0
8.18 0
8.19 0
8.2 0
8.21 0
8.22 0
8.23 0
8.24 0
8.25 0
8.26 0
8.27 0
8.28 0
8.29 0
8.3 0
8.31 0
8.32 0
8.33 0
8.34 0
8.35 0
8.36 0
8.37 0
8.38 0
8.39 0
8.4 0
8.41 0
8.42 0
8.43 0
8.44 0
8.45 0
8.46 0
8.47 0
8.48 0
8.49 0
8.5 0
8.51 0
8.52 0
8.53 0
8.54 0
8.55 0
8.56 0
8.57 0
8.58 0
8.59 0
8.6 0
8.61 0
8.62 0
8.63 0
8.64 0
8.65 0
8.66 0
8.67 0
8.68 0
8.69 0
8.7 0
8.71 0
8.72 0
8.73 0
8.74 0
8.75 0
8.76 0
8.77 0
8.78 0
8.79 0
8.8 0
8.81 0
8.82 0
8.83 0
8.84 0
8.85 0
8.86 0
8.87 0
8.88 0
8.89 0
8.9 0
8.91 0
8.92 0
8.93 0
8.94 0
8.95 0
8.96 0
8.97 0
8.98 0
8.99 0
9 0
9.01 0
9.02 0
9.03 0
9.04 0
9.05 0
9.06 0
9.07 0
9.08 0
9.09 0
9.1 0
9.11 0
9.12 0
9.13 0
9.14 0
9.15 0
9.16 0
9.17 0
9.18 0
9.19 0
9.2 0
9.21 0
9.22 0
9.23 0
9.24 0
9.25 0
9.26 0
9.27 0
9.28 0
9.29 0
9.3 0
9.31 0
9.32 0
9.33 0
9.34 0
9.35 0
9.36 0
9.37 0
9.38 0
9.39 0
9.4 0
9.41 0
9.42 0
9.43 0
9.44 0
9.45 0
9.46 0
9.47 0
9.48 0
9.49 0
9.5 0
9.51 0
9.52 0
9.53 0
9.54 0
9.55 0
9.56 0
9.57 0
9.58 0
9.59 0
9.6 0
9.61 0
9.62 0
9.63 0
9.64 0
9.65 0
9.66 0
9.67 0
9.68 0
9.69 0
9.7 0
9.71 0
9.72 0
9.73 0
9.74 0
9.75 0
9.76 0
9.77 0
9.78 0
9.79 0
9.8 0
9.81 0
9.82 0
9.83 0
9.84 0
9.85 0
9.86 0
9.87 0
9.88 0
9.89 0
9.9 0
9.91 0
9.92 0
9.93 0
9.94 0
9.95 0
9.96 0
9.97 0
9.98 0
9.99 0
10 0
10.01 0
10.02 0
10.03 0
10.04 0
10.05 0
10.06 0
10.07 0
10.08 0
10.09 0
10.1 0
10.11 0
10.12 0
10.13 0
10.14 0
10.15 0
10.16 0
10.17 0
10.18 0
10.19 0
10.2 0
10.21 0
10.22 0
10.23 0
10.24 0
10.25 0
10.26 0
10.27 0
10.28 0
10.29 0
10.3 0
10.31 0
10.32 0
10.33 0
10.34 0
10.35 0
10.36 0
10.37 0
10.38 0
10.39 0
10.4 0
10.41 0
10.42 0
10.43 0
10.44 0
10.45 0
10.46 0
10.47 0
10.48 0
10.49 0
10.5 0
10.51 0
10.52 0
10.53 0
10.54 0
10.55 0
10.56 0
10.57 0
10.58 0
10.59 0
10.6 0
10.61 0
10.62 0
10.63 0
10.64 0
10.65 0
10.66 0
10.67 0
10.68 0
10.69 0
10.7 0
10.71 0
10.72 0
10.73 0
10.74 0
10.75 0
10.76 0
10.77 0
10.78 0
10.79 0
10.8 0
10.81 0
10.82 0
10.83 0
10.84 0
10.85 0
10.86 0
10.87 0
10.88 0
10.89 0
10.9 0
10.91 0
10.92 0
10.93 0
10.94 0
10.95 0
10.96 0
10.97 0
10.98 0
10.99 0
11 0
11.01 0
11.02 0
11.03 0
11.04 0
11.05 0
11.06 0
11.07 0
11.08 0
11.09 0
11.1 0
11.11 0
11.12 0
11.13 0
11.14 0
11.15 0
11.16 0
11.17 0
11.18 0
11.19 0
11.2 0
11.21 0
11.22 0
11.23 0
11.24 0
11.25 0
11.26 0
11.27 0
11.28 0
11.29 0
11.3 0
11.31 0
11.32 0
11.33 0
11.34 0
11.35 0
11.36 0
11.37 0
11.38 0
11.39 0
11.4 0
11.41 0
11.42 0
11.43 0
11.44 0
11.45 0
11.46 0
11.47 0
11.48 0
11.49 0
11.5 0
11.51 0
11.52 0
11.53 0
11.54 0
11.55 0
11.56 0
11.57 0
11.58 0
11.59 0
11.6 0
11.61 0
11.62 0
11.63 0
11.64 0
11.65 0
11.66 0
11.67 0
11.68 0
11.69 0
11.7 0
11.71 0
11.72 0
11.73 0
11.74 0
11.75 0
11.76 0
11.77 0
11.78 0
11.79 0
11.8 0
11.81 0
11.82 0
11.83 0
11.84 0
11.85 0
11.86 0
11.87 0
11.88 0
11.89 0
11.9 0
11.91 0
11.92 0
11.93 0
11.94 0
11.95 0
11.96 0
11.97 0
11.98 0
11.99 0
12 0
};
\addplot [very thick, black]
table {%
0 0.000284273914537635
0.01 0.000285063792955339
0.02 0.000285857140338363
0.03 0.000286653975428697
0.04 0.000287454317089295
0.05 0.000288258184304977
0.06 0.000289065596183343
0.07 0.00028987657195569
0.08 0.000290691130977941
0.09 0.000291509292731579
0.1 0.000292331076824586
0.11 0.000293156502992403
0.12 0.000293985591098879
0.13 0.000294818361137246
0.14 0.000295654833231089
0.15 0.00029649502763533
0.16 0.000297338964737226
0.17 0.000298186665057363
0.18 0.000299038149250667
0.19 0.000299893438107424
0.2 0.000300752552554306
0.21 0.000301615513655403
0.22 0.000302482342613273
0.23 0.000303353060769989
0.24 0.000304227689608209
0.25 0.00030510625075224
0.26 0.000305988765969124
0.27 0.000306875257169725
0.28 0.000307765746409833
0.29 0.000308660255891268
0.3 0.000309558807963001
0.31 0.000310461425122285
0.32 0.000311368130015788
0.33 0.000312278945440747
0.34 0.000313193894346123
0.35 0.000314112999833769
0.36 0.00031503628515961
0.37 0.000315963773734832
0.38 0.000316895489127078
0.39 0.000317831455061662
0.4 0.000318771695422789
0.41 0.000319716234254783
0.42 0.00032066509576333
0.43 0.000321618304316732
0.44 0.00032257588444717
0.45 0.00032353786085198
0.46 0.000324504258394934
0.47 0.000325475102107548
0.48 0.000326450417190379
0.49 0.000327430229014355
0.5 0.000328414563122101
0.51 0.000329403445229284
0.52 0.000330396901225976
0.53 0.000331394957178012
0.54 0.000332397639328375
0.55 0.000333404974098588
0.56 0.00033441698809012
0.57 0.000335433708085802
0.58 0.000336455161051257
0.59 0.000337481374136342
0.6 0.000338512374676607
0.61 0.000339548190194761
0.62 0.000340588848402153
0.63 0.00034163437720027
0.64 0.000342684804682243
0.65 0.000343740159134373
0.66 0.000344800469037662
0.67 0.000345865763069366
0.68 0.000346936070104558
0.69 0.000348011419217704
0.7 0.000349091839684262
0.71 0.000350177360982277
0.72 0.000351268012794012
0.73 0.000352363825007577
0.74 0.000353464827718587
0.75 0.000354571051231818
0.76 0.000355682526062898
0.77 0.000356799282939995
0.78 0.000357921352805531
0.79 0.000359048766817911
0.8 0.000360181556353265
0.81 0.000361319753007204
0.82 0.000362463388596597
0.83 0.000363612495161364
0.84 0.000364767104966282
0.85 0.000365927250502808
0.86 0.000367092964490919
0.87 0.000368264279880976
0.88 0.00036944122985559
0.89 0.000370623847831523
0.9 0.000371812167461592
0.91 0.000373006222636599
0.92 0.000374206047487274
0.93 0.000375411676386241
0.94 0.000376623143949996
0.95 0.000377840485040913
0.96 0.000379063734769252
0.97 0.000380292928495207
0.98 0.000381528101830958
0.99 0.00038276929064274
1 0.000384016531052951
1.01 0.000385269859442253
1.02 0.000386529312451713
1.03 0.000387794926984957
1.04 0.000389066740210339
1.05 0.000390344789563143
1.06 0.000391629112747791
1.07 0.000392919747740084
1.08 0.000394216732789455
1.09 0.000395520106421252
1.1 0.000396829907439032
1.11 0.000398146174926888
1.12 0.000399468948251787
1.13 0.00040079826706594
1.14 0.000402134171309186
1.15 0.000403476701211408
1.16 0.000404825897294961
1.17 0.000406181800377129
1.18 0.00040754445157261
1.19 0.000408913892296014
1.2 0.000410290164264392
1.21 0.000411673309499791
1.22 0.000413063370331823
1.23 0.000414460389400272
1.24 0.000415864409657716
1.25 0.000417275474372179
1.26 0.000418693627129807
1.27 0.000420118911837568
1.28 0.000421551372725982
1.29 0.000422991054351874
1.3 0.000424438001601154
1.31 0.000425892259691625
1.32 0.000427353874175818
1.33 0.000428822890943854
1.34 0.000430299356226331
1.35 0.000431783316597246
1.36 0.000433274818976937
1.37 0.000434773910635059
1.38 0.000436280639193586
1.39 0.000437795052629849
1.4 0.000439317199279593
1.41 0.000440847127840068
1.42 0.000442384887373157
1.43 0.000443930527308526
1.44 0.000445484097446803
1.45 0.000447045647962806
1.46 0.000448615229408773
1.47 0.000450192892717652
1.48 0.000451778689206406
1.49 0.000453372670579363
1.5 0.000454974888931583
1.51 0.000456585396752271
1.52 0.000458204246928227
1.53 0.000459831492747312
1.54 0.000461467187901968
1.55 0.000463111386492757
1.56 0.000464764143031953
1.57 0.000466425512447151
1.58 0.000468095550084922
1.59 0.000469774311714501
1.6 0.000471461853531522
1.61 0.000473158232161769
1.62 0.000474863504664986
1.63 0.000476577728538712
1.64 0.000478300961722163
1.65 0.000480033262600142
1.66 0.000481774690007001
1.67 0.000483525303230636
1.68 0.000485285162016515
1.69 0.000487054326571765
1.7 0.000488832857569286
1.71 0.000490620816151903
1.72 0.00049241826393658
1.73 0.000494225263018654
1.74 0.000496041875976122
1.75 0.000497868165873979
1.76 0.000499704196268582
1.77 0.000501550031212077
1.78 0.000503405735256856
1.79 0.000505271373460073
1.8 0.000507147011388189
1.81 0.000509032715121587
1.82 0.000510928551259207
1.83 0.00051283458692325
1.84 0.000514750889763925
1.85 0.000516677527964232
1.86 0.000518614570244814
1.87 0.000520562085868844
1.88 0.000522520144646967
1.89 0.000524488816942299
1.9 0.000526468173675466
1.91 0.000528458286329702
1.92 0.000530459226956001
1.93 0.000532471068178321
1.94 0.000534493883198837
1.95 0.000536527745803256
1.96 0.000538572730366183
1.97 0.000540628911856546
1.98 0.000542696365843069
1.99 0.000544775168499816
2 0.000546865396611783
2.01 0.000548963990749615
2.02 0.000551067774121616
2.03 0.000553176562746675
2.04 0.000555290144865684
2.05 0.000557408280445531
2.06 0.000559530700676292
2.07 0.000561657107461414
2.08 0.000563787172900702
2.09 0.000565920538765924
2.1 0.000568056815968838
2.11 0.000570195584021483
2.12 0.000572336390488545
2.13 0.00057447875043169
2.14 0.000576622145845688
2.15 0.000578766025086287
2.16 0.000580909802289748
2.17 0.000583052856784023
2.18 0.00058519453249163
2.19 0.000587334137324307
2.2 0.000589470942569613
2.21 0.000591604182269729
2.22 0.000593733052592783
2.23 0.000595856711197156
2.24 0.000597974276589316
2.25 0.000600084827475866
2.26 0.000602187402110653
2.27 0.000604280997637909
2.28 0.000606364569432601
2.29 0.000608437030439345
2.3 0.000610497250511458
2.31 0.000612544055751914
2.32 0.000614576227858282
2.33 0.000616592503473897
2.34 0.00061859157354788
2.35 0.000620572082706878
2.36 0.000622532628641716
2.37 0.000624471761512508
2.38 0.000626387983376136
2.39 0.000628279747640349
2.4 0.000630145458549181
2.41 0.000631983470704749
2.42 0.000633792088630962
2.43 0.000635569566385081
2.44 0.000637314107223577
2.45 0.000639023863329128
2.46 0.000640696935606171
2.47 0.000642331373552826
2.48 0.000643925175217565
2.49 0.000645476287249447
2.5 0.000646982605051263
2.51 0.000648441973045398
2.52 0.000649852185062704
2.53 0.00065121098486514
2.54 0.000652516066813336
2.55 0.000653765076690708
2.56 0.000654955612696064
2.57 0.000656085226617029
2.58 0.000657151425196855
2.59 0.000658151671707459
2.6 0.000659083387741678
2.61 0.000659943955237808
2.62 0.000660730718749566
2.63 0.000661440987974443
2.64 0.000662072040553374
2.65 0.000662621125154206
2.66 0.000663085464851176
2.67 0.000663462260812016
2.68 0.000663748696303607
2.69 0.000663941941026341
2.7 0.000664039155786322
2.71 0.000664037497513423
2.72 0.000663934124631873
2.73 0.000663726202788583
2.74 0.000663410910942688
2.75 0.000662985447817955
2.76 0.000662447038717602
2.77 0.000661792942698826
2.78 0.000661020460101852
2.79 0.000660126940425677
2.8 0.00065910979053985
2.81 0.000657966483218537
2.82 0.000656694565980005
2.83 0.000655291670211221
2.84 0.000653755520553795
2.85 0.000652083944523838
2.86 0.000650274882334605
2.87 0.000648326396886898
2.88 0.000646236683888456
2.89 0.000644004082059513
2.9 0.000641627083378027
2.91 0.000639104343314084
2.92 0.000636434690999435
2.93 0.000633617139274434
2.94 0.000630650894551387
2.95 0.000627535366430093
2.96 0.000624270176998642
2.97 0.000620855169749979
2.98 0.000617290418042728
2.99 0.000613576233033177
3 0.000609713171004221
3.01 0.000590488706120663
3.02 0.000573767438200827
3.03 0.000560304778740804
3.04 0.000550947856898581
3.05 0.000546613076044169
3.06 0.000548242395839171
3.07 0.000556741431688576
3.08 0.000572905796842464
3.09 0.000597344354717719
3.1 0.000630408831116962
3.11 0.000672138666034329
3.12 0.000722228508609588
3.13 0.000780023830105855
3.14 0.000844547909475144
3.15 0.000914560752091291
3.16 0.000988647066320185
3.17 0.00106532629572895
3.18 0.00114317352741971
3.19 0.00122093700484841
3.2 0.00129763712741583
3.21 0.00137263380755387
3.22 0.00144565353602031
3.23 0.00151677332060236
3.24 0.00158636437100976
3.25 0.00165500289958874
3.26 0.00172335829133357
3.27 0.00179207041446253
3.28 0.00186162851572893
3.29 0.0019322643157049
3.3 0.00200387147165438
3.31 0.00207596196680136
3.32 0.0021476665937883
3.33 0.00221778132886987
3.34 0.00228485465119809
3.35 0.00234730416821382
3.36 0.00240354603176653
3.37 0.00245211895976972
3.38 0.00249178663231585
3.39 0.0025216071270163
3.4 0.00254096448813312
3.41 0.00254956395950295
3.42 0.00254739768611409
3.43 0.00253469117135158
3.44 0.00251184228168893
3.45 0.00247936421376434
3.46 0.00243784184646307
3.47 0.00238790769405302
3.48 0.0023302397708099
3.49 0.00226557964438031
3.5 0.00219476533516501
3.51 0.00211877095253241
3.52 0.00203874336478292
3.53 0.00195602599174314
3.54 0.00187216110010259
3.55 0.0017888647456376
3.56 0.00170797249786261
3.57 0.00163135877354189
3.58 0.00156083720854439
3.59 0.00149805311308804
3.6 0.00144438091164307
3.61 0.00140083915930207
3.62 0.00136803333681477
3.63 0.00134613269155345
3.64 0.00133488271225747
3.65 0.0013336502368794
3.66 0.00134149438207042
3.67 0.00135725390060047
3.68 0.00137964042834931
3.69 0.00140732737405953
3.7 0.00143902575456427
3.71 0.00147354076143899
3.72 0.00150980583556905
3.73 0.00154689406372277
3.74 0.00158400938537628
3.75 0.00162046212024518
3.76 0.00165563456510048
3.77 0.00168894287374708
3.78 0.00171980123280419
3.79 0.0017475936205151
3.8 0.00177165731916191
3.81 0.00179128094803889
3.82 0.00180571817516524
3.83 0.00181421653265717
3.84 0.00181605900561361
3.85 0.0018106144268573
3.86 0.00179739136652855
3.87 0.00177608935499413
3.88 0.00174664110942316
3.89 0.00170924008403363
3.9 0.00166434915934998
3.91 0.00161268850097557
3.92 0.00155520325662021
3.93 0.00149301438724546
3.94 0.00142735806235706
3.95 0.00135952028647677
3.96 0.00129077355387039
3.97 0.00122232139683778
3.98 0.00115525497746371
3.99 0.00109052379066237
4 0.00102892052728298
4.01 0.000971078523097738
4.02 0.000917479170536371
4.03 0.000868466216517731
4.04 0.000824263917382394
4.05 0.000784996417997137
4.06 0.000750706307010783
4.07 0.000721370941542124
4.08 0.000696915741131374
4.09 0.00067722417169035
4.1 0.000662144555227238
4.11 0.00065149414874765
4.12 0.000645061142641682
4.13 0.000642605343205984
4.14 0.000643858332542307
4.15 0.000648523847994327
4.16 0.00065627900048392
4.17 0.000666776769458568
4.18 0.000679649991347653
4.19 0.000694516824561044
4.2 0.000710987456710537
4.21 0.000728671646976934
4.22 0.000747186589542443
4.23 0.000766164552755318
4.24 0.000785259790339695
4.25 0.000804154321171264
4.26 0.000822562310964357
4.27 0.000840232938230198
4.28 0.00085695176604151
4.29 0.000872540754323666
4.3 0.000886857125834973
4.31 0.000899791341034328
4.32 0.000911264446492803
4.33 0.000921225045557095
4.34 0.000929646107032313
4.35 0.000936521785752222
4.36 0.00094186438467984
4.37 0.00094570154643274
4.38 0.00094807372583104
4.39 0.000949031965645341
4.4 0.000948635975436532
4.41 0.000946952497722253
4.42 0.000944053935742185
4.43 0.000940017211723421
4.44 0.000934922822670849
4.45 0.000928854061336933
4.46 0.000921896372329134
4.47 0.00091413681662976
4.48 0.000905663621630585
4.49 0.000896565797759782
4.5 0.000886932806651627
4.51 0.000876854269418381
4.52 0.000866419706833331
4.53 0.000855718306076408
4.54 0.000844838711115292
4.55 0.000833868835803502
4.56 0.000822895700395757
4.57 0.000812005293441556
4.58 0.000801282461957194
4.59 0.000790810833433411
4.6 0.000780672773650341
4.61 0.000770949384482569
4.62 0.000761720545923051
4.63 0.000753065006471861
4.64 0.000745060525858322
4.65 0.000737784073825323
4.66 0.000731312088431972
4.67 0.000725720797051952
4.68 0.000721086602984189
4.69 0.000717486540370473
4.7 0.000714998799949652
4.71 0.000713703328084225
4.72 0.000713682501484307
4.73 0.000715021880133453
4.74 0.00071781104109532
4.75 0.000722144496150053
4.76 0.000728122696571237
4.77 0.00073585312880098
4.78 0.000745451505299993
4.79 0.000757043055424825
4.8 0.000770763921792917
4.81 0.000786762668209487
4.82 0.000805201905812149
4.83 0.000826260044595025
4.84 0.000850133177849413
4.85 0.000877037107236107
4.86 0.000907209516104759
4.87 0.000940912298201236
4.88 0.000978434047938363
4.89 0.0010200927168111
4.9 0.00106623843815043
4.91 0.0011172565190408
4.92 0.00117357059365367
4.93 0.00123564592622396
4.94 0.00130399284413634
4.95 0.00137917027178965
4.96 0.00146178932374634
4.97 0.00155251690082244
4.98 0.00165207921492391
4.99 0.00176126514732653
5 0.00188092932056095
5.01 0.00201199473608363
5.02 0.00215545479869324
5.03 0.00231237451470086
5.04 0.00248389061510236
5.05 0.00267121031885949
5.06 0.00287560841690107
5.07 0.00309842232729391
5.08 0.00334104474954728
5.09 0.00360491353508837
5.1 0.00389149839574797
5.11 0.00420228409659926
5.12 0.004538749826787
5.13 0.004902344513321
5.14 0.00529445793651467
5.15 0.00571638761615484
5.16 0.00616930155424139
5.17 0.00665419702841516
5.18 0.00717185571254539
5.19 0.00772279544146463
5.2 0.00830721892828699
5.21 0.00892495969662976
5.22 0.00957542544842019
5.23 0.0102575391340384
5.24 0.0109696782538828
5.25 0.0117096135635259
5.26 0.0124744495478109
5.27 0.0132605708890724
5.28 0.0140636016629982
5.29 0.0148783869098188
5.3 0.0156990090166834
5.31 0.0165188531953496
5.32 0.017330736274678
5.33 0.0181271101718302
5.34 0.0189003453181945
5.35 0.0196430903084162
5.36 0.0203486933179917
5.37 0.0210116603276552
5.38 0.021628117077604
5.39 0.0221962377217758
5.4 0.022716604160836
5.41 0.0231924655954791
5.42 0.0236298765416473
5.43 0.0240377014806325
5.44 0.0244274836942503
5.45 0.0248131834424294
5.46 0.0252107959924916
5.47 0.0256378632478096
5.48 0.0261128943740674
5.49 0.0266547115531708
5.5 0.0272817374689034
5.51 0.0280112419425349
5.52 0.0288585668159648
5.53 0.029836351143736
5.54 0.0309537832110393
5.55 0.0322159116621363
5.56 0.0336230543409215
5.57 0.0351703488597574
5.58 0.0368474913652402
5.59 0.0386387071571899
5.6 0.0405229868306324
5.61 0.0424746037204768
5.62 0.0444639037656969
5.63 0.0464583307209036
5.64 0.0484236228718176
5.65 0.0503250975698885
5.66 0.0521289315077877
5.67 0.0538033497580443
5.68 0.0553196539506603
5.69 0.056653045250162
5.7 0.0577832248110144
5.71 0.0586947768780885
5.72 0.0593773531453127
5.73 0.05982567992252
5.74 0.0600394040717446
5.75 0.060022784418516
5.76 0.0597842286874864
5.77 0.059335677567574
5.78 0.0586918501027403
5.79 0.057869386777016
5.8 0.0568859525703367
5.81 0.0557593832608817
5.82 0.0545069658111638
5.83 0.0531449325083848
5.84 0.0516882185910284
5.85 0.0501504896840105
5.86 0.0485443974784055
5.87 0.046881979944141
5.88 0.0451750951039447
5.89 0.0434357714844308
5.9 0.04167637648072
5.91 0.0399095440948723
5.92 0.0381478588043746
5.93 0.036403351054061
5.94 0.0346869079836476
5.95 0.0330077277928362
5.96 0.0313729404146852
5.97 0.0297874819255074
5.98 0.0282542543878507
5.99 0.0267745406504487
6 0.0253485897909247
6.01 0.0239761337833722
6.02 0.022657093462913
6.03 0.0213919963036447
6.04 0.0201819832741492
6.05 0.0190287352412632
6.06 0.0179341976203449
6.07 0.016900183552411
6.08 0.0159279539525255
6.09 0.0150178657687997
6.1 0.0141691523319777
6.11 0.0133798616811691
6.12 0.0126469415864357
6.13 0.0119664327129467
6.14 0.0113337186133343
6.15 0.0107437830380001
6.16 0.010191438028406
6.17 0.00967150513681187
6.18 0.00917895116027204
6.19 0.00870899396129996
6.2 0.00825719994497401
6.21 0.00781959162332641
6.22 0.00739277299363878
6.23 0.00697406577924117
6.24 0.00656163556797297
6.25 0.00615457807800035
6.26 0.00575293553361408
6.27 0.00535762263015899
6.28 0.00497025913786537
6.29 0.0045929273000679
6.3 0.00422789049887701
6.31 0.00387731922595202
6.32 0.00354306798413658
6.33 0.00322653325497949
6.34 0.00312711536790539
6.35 0.00313014989055684
6.36 0.00313324590388453
6.37 0.00313640335987682
6.38 0.00313962220801551
6.39 0.00314290239523008
6.4 0.00314624386585131
6.41 0.00314964656156428
6.42 0.00315311042136085
6.43 0.00315663538149146
6.44 0.00316022137541657
6.45 0.00316386833375742
6.46 0.00316757618424635
6.47 0.00317134485167672
6.48 0.00317517425785223
6.49 0.00317906432153592
6.5 0.00318301495839875
6.51 0.00318702608096771
6.52 0.00319109759857369
6.53 0.00319522941729893
6.54 0.00319942143992421
6.55 0.00320367356587571
6.56 0.00320798569117164
6.57 0.00321235770836869
6.58 0.00321678950650818
6.59 0.00322128097106212
6.6 0.0032258319838791
6.61 0.00323044242313008
6.62 0.00323511216325406
6.63 0.00323984107490378
6.64 0.00324462902489129
6.65 0.00324947587613369
6.66 0.00325438148759874
6.67 0.00325934571425068
6.68 0.00326436840699614
6.69 0.00326944941263015
6.7 0.0032745885737824
6.71 0.00327978572886367
6.72 0.00328504071201253
6.73 0.00329035335304239
6.74 0.00329572347738877
6.75 0.00330115090605704
6.76 0.00330663545557048
6.77 0.00331217693791888
6.78 0.0033177751605075
6.79 0.00332342992610669
6.8 0.00332914103280193
6.81 0.00333490827394459
6.82 0.00334073143810323
6.83 0.00334661030901564
6.84 0.00335254466554159
6.85 0.00335853428161632
6.86 0.00336457892620485
6.87 0.0033706783632571
6.88 0.00337683235166396
6.89 0.00338304064521418
6.9 0.0033893029925523
6.91 0.00339561913713762
6.92 0.00340198881720408
6.93 0.00340841176572136
6.94 0.00341488771035704
6.95 0.00342141637343997
6.96 0.0034279974719248
6.97 0.00343463071735783
6.98 0.00344131581584409
6.99 0.00344805246801575
7 0.003454840369002
7.01 0.0034616792084002
7.02 0.00346856867024861
7.03 0.00347550843300052
7.04 0.00348249816949995
7.05 0.00348953754695894
7.06 0.00349662622693638
7.07 0.00350376386531857
7.08 0.00351095011230138
7.09 0.00351818461237416
7.1 0.00352546700430543
7.11 0.00353279692113031
7.12 0.00354017399013979
7.13 0.00354759783287187
7.14 0.00355506806510456
7.15 0.00356258429685079
7.16 0.00357014613235535
7.17 0.00357775317009367
7.18 0.00358540500277276
7.19 0.00359310121733406
7.2 0.00360084139495842
7.21 0.0036086251110732
7.22 0.00361645193536135
7.23 0.00362432143177281
7.24 0.00363223315853788
7.25 0.00364018666818295
7.26 0.00364818150754829
7.27 0.00365621721780813
7.28 0.00366429333449297
7.29 0.0036724093875141
7.3 0.00368056490119044
7.31 0.00368875939427763
7.32 0.00369699237999937
7.33 0.0037052633660812
7.34 0.00371357185478636
7.35 0.00372191734295422
7.36 0.00373029932204081
7.37 0.0037387172781618
7.38 0.00374717069213777
7.39 0.00375565903954183
7.4 0.00376418179074948
7.41 0.00377273841099088
7.42 0.00378132836040543
7.43 0.00378995109409856
7.44 0.00379860606220094
7.45 0.00380729270992992
7.46 0.00381601047765322
7.47 0.00382475880095497
7.48 0.00383353711070384
7.49 0.00384234483312363
7.5 0.00385118138986581
7.51 0.00386004619808444
7.52 0.00386893867051318
7.53 0.0038778582155444
7.54 0.0038868042373105
7.55 0.0038957761357672
7.56 0.00390477330677897
7.57 0.00391379514220642
7.58 0.00392284102999573
7.59 0.00393191035426995
7.6 0.00394100249542227
7.61 0.00395011683021116
7.62 0.00395925273185731
7.63 0.00396840957014234
7.64 0.00397758671150931
7.65 0.00398678351916484
7.66 0.00399599935318299
7.67 0.00400523357061057
7.68 0.00401448552557419
7.69 0.00402375456938859
7.7 0.00403304005066659
7.71 0.00404234131543028
7.72 0.00405165770722363
7.73 0.0040609885672263
7.74 0.00407033323436863
7.75 0.00407969104544789
7.76 0.00408906133524544
7.77 0.00409844343664505
7.78 0.00410783668075204
7.79 0.00411724039701343
7.8 0.00412665391333877
7.81 0.00413607655622181
7.82 0.00414550765086283
7.83 0.00415494652129151
7.84 0.00416439249049044
7.85 0.00417384488051898
7.86 0.0041833030126376
7.87 0.00419276620743248
7.88 0.0042022337849404
7.89 0.00421170506477372
7.9 0.00422117936624559
7.91 0.00423065600849502
7.92 0.00424013431061202
7.93 0.00424961359176257
7.94 0.00425909317131335
7.95 0.00426857236895625
7.96 0.00427805050483251
7.97 0.00428752689965639
7.98 0.00429700087483837
7.99 0.00430647175260783
8 0.00431593885613492
8.01 0.00432540150965191
8.02 0.00433485903857356
8.03 0.00434431076961674
8.04 0.00435375603091904
8.05 0.00436319415215644
8.06 0.00437262446465978
8.07 0.00438204630153026
8.08 0.00439145899775348
8.09 0.00440086189031244
8.1 0.00441025431829902
8.11 0.00441963562302413
8.12 0.0044290051481263
8.13 0.00443836223967883
8.14 0.00444770624629526
8.15 0.00445703651923322
8.16 0.00446635241249651
8.17 0.00447565328293548
8.18 0.00448493849034551
8.19 0.00449420739756365
8.2 0.00450345937056331
8.21 0.00451269377854699
8.22 0.0045219099940369
8.23 0.00453110739296366
8.24 0.00454028535475272
8.25 0.0045494432624087
8.26 0.00455858050259757
8.27 0.00456769646572646
8.28 0.00457679054602137
8.29 0.00458586214160246
8.3 0.004594910654557
8.31 0.00460393549101008
8.32 0.00461293606119278
8.33 0.004621911779508
8.34 0.00463086206459391
8.35 0.00463978633938481
8.36 0.00464868403116969
8.37 0.00465755457164813
8.38 0.00466639739698385
8.39 0.00467521194785567
8.4 0.004683997669506
8.41 0.00469275401178679
8.42 0.00470148042920293
8.43 0.00471017638095324
8.44 0.00471884133096878
8.45 0.0047274747479488
8.46 0.00473607610539408
8.47 0.0047446448816378
8.48 0.00475318055987396
8.49 0.00476168262818329
8.5 0.00477015057955668
8.51 0.00477858391191625
8.52 0.00478698212813392
8.53 0.00479534473604765
8.54 0.00480367124847525
8.55 0.00481196118322589
8.56 0.00482021406310929
8.57 0.00482842941594261
8.58 0.00483660677455509
8.59 0.00484474567679049
8.6 0.00485284566550735
8.61 0.00486090628857708
8.62 0.00486892709887998
8.63 0.0048769076542992
8.64 0.00488484751771267
8.65 0.00489274625698306
8.66 0.00490060344494582
8.67 0.00490841865939538
8.68 0.00491619148306944
8.69 0.00492392150363159
8.7 0.00493160831365213
8.71 0.00493925151058724
8.72 0.00494685069675655
8.73 0.00495440547931915
8.74 0.00496191547024809
8.75 0.00496938028630341
8.76 0.00497679954900386
8.77 0.00498417288459719
8.78 0.00499149992402925
8.79 0.00499878030291185
8.8 0.00500601366148947
8.81 0.00501319964460493
8.82 0.00502033790166394
8.83 0.00502742808659883
8.84 0.0050344698578312
8.85 0.00504146287823395
8.86 0.00504840681509231
8.87 0.00505530134006436
8.88 0.00506214612914074
8.89 0.0050689408626039
8.9 0.00507568522498673
8.91 0.00508237890503079
8.92 0.00508902159564412
8.93 0.00509561299385869
8.94 0.00510215280078762
8.95 0.00510864072158212
8.96 0.00511507646538833
8.97 0.00512145974530402
8.98 0.00512779027833526
8.99 0.00513406778535312
9 0.00514029199105043
9.01 0.00514646262389865
9.02 0.00515257941610494
9.03 0.00515864210356949
9.04 0.00516465042584309
9.05 0.00517060412608507
9.06 0.00517650295102163
9.07 0.00518234665090456
9.08 0.00518813497947053
9.09 0.00519386769390085
9.1 0.00519954455478181
9.11 0.00520516532606566
9.12 0.00521072977503226
9.13 0.00521623767225138
9.14 0.00522168879154578
9.15 0.00522708290995511
9.16 0.00523241980770052
9.17 0.00523769926815019
9.18 0.00524292107778574
9.19 0.00524808502616952
9.2 0.00525319090591286
9.21 0.00525823851264528
9.22 0.00526322764498468
9.23 0.00526815810450858
9.24 0.00527302969572637
9.25 0.00527784222605263
9.26 0.00528259550578154
9.27 0.00528728934806231
9.28 0.00529192356887587
9.29 0.00529649798701253
9.3 0.00530101242405082
9.31 0.00530546670433757
9.32 0.00530986065496899
9.33 0.00531419410577301
9.34 0.00531846688929274
9.35 0.00532267884077114
9.36 0.0053268297981368
9.37 0.00533091960199091
9.38 0.00533494809559544
9.39 0.00533891512486238
9.4 0.00534282053834424
9.41 0.00534666418722565
9.42 0.00535044592531608
9.43 0.00535416560904373
9.44 0.00535782309745049
9.45 0.00536141825218812
9.46 0.00536495093751531
9.47 0.00536842102029606
9.48 0.00537182836999893
9.49 0.00537517285869741
9.5 0.00537845436107133
9.51 0.00538167275440919
9.52 0.00538482791861154
9.53 0.00538791973619528
9.54 0.00539094809229889
9.55 0.00539391287468856
9.56 0.00539681397376524
9.57 0.00539965128257244
9.58 0.00540242469680495
9.59 0.00540513411481834
9.6 0.00540777943763911
9.61 0.00541036056897576
9.62 0.00541287741523037
9.63 0.00541532988551097
9.64 0.00541771789164453
9.65 0.00542004134819044
9.66 0.00542230017245477
9.67 0.00542449428450482
9.68 0.00542662360718435
9.69 0.0054286880661292
9.7 0.00543068758978333
9.71 0.00543262210941531
9.72 0.00543449155913509
9.73 0.00543629587591113
9.74 0.00543803499958785
9.75 0.00543970887290327
9.76 0.00544131744150692
9.77 0.00544286065397787
9.78 0.00544433846184295
9.79 0.00544575081959507
9.8 0.0054470976847116
9.81 0.00544837901767281
9.82 0.00544959478198028
9.83 0.00545074494417532
9.84 0.00545182947385732
9.85 0.00545284834370196
9.86 0.00545380152947938
9.87 0.00545468901007205
9.88 0.00545551076749263
9.89 0.00545626678690148
9.9 0.00545695705662392
9.91 0.00545758156816727
9.92 0.00545814031623755
9.93 0.00545863329875585
9.94 0.00545906051687432
9.95 0.00545942197499178
9.96 0.00545971768076892
9.97 0.00545994764514307
9.98 0.00546011188234251
9.99 0.00546021040990023
10 0.0054602432486673
10.01 0.0054602432486673
10.02 0.0054602432486673
10.03 0.0054602432486673
10.04 0.0054602432486673
10.05 0.0054602432486673
10.06 0.0054602432486673
10.07 0.0054602432486673
10.08 0.0054602432486673
10.09 0.0054602432486673
10.1 0.0054602432486673
10.11 0.0054602432486673
10.12 0.0054602432486673
10.13 0.0054602432486673
10.14 0.0054602432486673
10.15 0.0054602432486673
10.16 0.0054602432486673
10.17 0.0054602432486673
10.18 0.0054602432486673
10.19 0.0054602432486673
10.2 0.0054602432486673
10.21 0.0054602432486673
10.22 0.0054602432486673
10.23 0.0054602432486673
10.24 0.0054602432486673
10.25 0.0054602432486673
10.26 0.0054602432486673
10.27 0.0054602432486673
10.28 0.0054602432486673
10.29 0.0054602432486673
10.3 0.0054602432486673
10.31 0.0054602432486673
10.32 0.0054602432486673
10.33 0.0054602432486673
10.34 0.0054602432486673
10.35 0.0054602432486673
10.36 0.0054602432486673
10.37 0.0054602432486673
10.38 0.0054602432486673
10.39 0.0054602432486673
10.4 0.0054602432486673
10.41 0.0054602432486673
10.42 0.0054602432486673
10.43 0.0054602432486673
10.44 0.0054602432486673
10.45 0.0054602432486673
10.46 0.0054602432486673
10.47 0.0054602432486673
10.48 0.0054602432486673
10.49 0.0054602432486673
10.5 0.0054602432486673
10.51 0.0054602432486673
10.52 0.0054602432486673
10.53 0.0054602432486673
10.54 0.0054602432486673
10.55 0.0054602432486673
10.56 0.0054602432486673
10.57 0.0054602432486673
10.58 0.0054602432486673
10.59 0.0054602432486673
10.6 0.0054602432486673
10.61 0.0054602432486673
10.62 0.0054602432486673
10.63 0.0054602432486673
10.64 0.0054602432486673
10.65 0.0054602432486673
10.66 0.0054602432486673
10.67 0.0054602432486673
10.68 0.0054602432486673
10.69 0.0054602432486673
10.7 0.0054602432486673
10.71 0.0054602432486673
10.72 0.0054602432486673
10.73 0.0054602432486673
10.74 0.0054602432486673
10.75 0.0054602432486673
10.76 0.0054602432486673
10.77 0.0054602432486673
10.78 0.0054602432486673
10.79 0.0054602432486673
10.8 0.0054602432486673
10.81 0.0054602432486673
10.82 0.0054602432486673
10.83 0.0054602432486673
10.84 0.0054602432486673
10.85 0.0054602432486673
10.86 0.0054602432486673
10.87 0.0054602432486673
10.88 0.0054602432486673
10.89 0.0054602432486673
10.9 0.0054602432486673
10.91 0.0054602432486673
10.92 0.0054602432486673
10.93 0.0054602432486673
10.94 0.0054602432486673
10.95 0.0054602432486673
10.96 0.0054602432486673
10.97 0.0054602432486673
10.98 0.0054602432486673
10.99 0.0054602432486673
11 0.0054602432486673
11.01 0.0054602432486673
11.02 0.0054602432486673
11.03 0.0054602432486673
11.04 0.0054602432486673
11.05 0.0054602432486673
11.06 0.0054602432486673
11.07 0.0054602432486673
11.08 0.0054602432486673
11.09 0.0054602432486673
11.1 0.0054602432486673
11.11 0.0054602432486673
11.12 0.0054602432486673
11.13 0.0054602432486673
11.14 0.0054602432486673
11.15 0.0054602432486673
11.16 0.0054602432486673
11.17 0.0054602432486673
11.18 0.0054602432486673
11.19 0.0054602432486673
11.2 0.0054602432486673
11.21 0.0054602432486673
11.22 0.0054602432486673
11.23 0.0054602432486673
11.24 0.0054602432486673
11.25 0.0054602432486673
11.26 0.0054602432486673
11.27 0.0054602432486673
11.28 0.0054602432486673
11.29 0.0054602432486673
11.3 0.0054602432486673
11.31 0.0054602432486673
11.32 0.0054602432486673
11.33 0.0054602432486673
11.34 0.0054602432486673
11.35 0.0054602432486673
11.36 0.0054602432486673
11.37 0.0054602432486673
11.38 0.0054602432486673
11.39 0.0054602432486673
11.4 0.0054602432486673
11.41 0.0054602432486673
11.42 0.0054602432486673
11.43 0.0054602432486673
11.44 0.0054602432486673
11.45 0.0054602432486673
11.46 0.0054602432486673
11.47 0.0054602432486673
11.48 0.0054602432486673
11.49 0.0054602432486673
11.5 0.0054602432486673
11.51 0.0054602432486673
11.52 0.0054602432486673
11.53 0.0054602432486673
11.54 0.0054602432486673
11.55 0.0054602432486673
11.56 0.0054602432486673
11.57 0.0054602432486673
11.58 0.0054602432486673
11.59 0.0054602432486673
11.6 0.0054602432486673
11.61 0.0054602432486673
11.62 0.0054602432486673
11.63 0.0054602432486673
11.64 0.0054602432486673
11.65 0.0054602432486673
11.66 0.0054602432486673
11.67 0.0054602432486673
11.68 0.0054602432486673
11.69 0.0054602432486673
11.7 0.0054602432486673
11.71 0.0054602432486673
11.72 0.0054602432486673
11.73 0.0054602432486673
11.74 0.0054602432486673
11.75 0.0054602432486673
11.76 0.0054602432486673
11.77 0.0054602432486673
11.78 0.0054602432486673
11.79 0.0054602432486673
11.8 0.0054602432486673
11.81 0.0054602432486673
11.82 0.0054602432486673
11.83 0.0054602432486673
11.84 0.0054602432486673
11.85 0.0054602432486673
11.86 0.0054602432486673
11.87 0.0054602432486673
11.88 0.0054602432486673
11.89 0.0054602432486673
11.9 0.0054602432486673
11.91 0.0054602432486673
11.92 0.0054602432486673
11.93 0.0054602432486673
11.94 0.0054602432486673
11.95 0.0054602432486673
11.96 0.0054602432486673
11.97 0.0054602432486673
11.98 0.0054602432486673
11.99 0.0054602432486673
12 0.0054602432486673
};
\end{axis}

\end{tikzpicture}
\\
	% This file was created by tikzplotlib v0.9.8.
\begin{tikzpicture}

\begin{axis}[
height=\figureheight,
scaled y ticks=false,
tick align=outside,
tick pos=left,
width=\figurewidth,
x grid style={white!69.0196078431373!black},
xlabel={Time [\si{\second}]},
xmajorgrids,
xmin=0, xmax=12,
xtick style={color=black},
xticklabel style={align=center},
y grid style={white!69.0196078431373!black},
ylabel={Speed [\si{\meter\per\second}]},
ymajorgrids,
ymin=7.2, ymax=24.8,
ytick style={color=black},
yticklabel style={/pgf/number format/fixed,/pgf/number format/precision=3}
]
\addplot [very thick, black]
table {%
0 24
0.01 24
0.02 24
0.03 24
0.04 24
0.05 24
0.06 24
0.07 24
0.08 24
0.09 24
0.1 24
0.11 24
0.12 24
0.13 24
0.14 24
0.15 24
0.16 24
0.17 24
0.18 24
0.19 24
0.2 24
0.21 24
0.22 24
0.23 24
0.24 24
0.25 24
0.26 24
0.27 24
0.28 24
0.29 24
0.3 24
0.31 24
0.32 24
0.33 24
0.34 24
0.35 24
0.36 24
0.37 24
0.38 24
0.39 24
0.4 24
0.41 24
0.42 24
0.43 24
0.44 24
0.45 24
0.46 24
0.47 24
0.48 24
0.49 24
0.5 24
0.51 24
0.52 24
0.53 24
0.54 24
0.55 24
0.56 24
0.57 24
0.58 24
0.59 24
0.6 24
0.61 24
0.62 24
0.63 24
0.64 24
0.65 24
0.66 24
0.67 24
0.68 24
0.69 24
0.7 24
0.71 24
0.72 24
0.73 24
0.74 24
0.75 24
0.76 24
0.77 24
0.78 24
0.79 24
0.8 24
0.81 24
0.82 24
0.83 24
0.84 24
0.85 24
0.86 24
0.87 24
0.88 24
0.89 24
0.9 24
0.91 24
0.92 24
0.93 24
0.94 24
0.95 24
0.96 24
0.97 24
0.98 24
0.99 24
1 24
1.01 24
1.02 24
1.03 24
1.04 24
1.05 24
1.06 24
1.07 24
1.08 24
1.09 24
1.1 24
1.11 24
1.12 24
1.13 24
1.14 24
1.15 24
1.16 24
1.17 24
1.18 24
1.19 24
1.2 24
1.21 24
1.22 24
1.23 24
1.24 24
1.25 24
1.26 24
1.27 24
1.28 24
1.29 24
1.3 24
1.31 24
1.32 24
1.33 24
1.34 24
1.35 24
1.36 24
1.37 24
1.38 24
1.39 24
1.4 24
1.41 24
1.42 24
1.43 24
1.44 24
1.45 24
1.46 24
1.47 24
1.48 24
1.49 24
1.5 24
1.51 24
1.52 24
1.53 24
1.54 24
1.55 24
1.56 24
1.57 24
1.58 24
1.59 24
1.6 24
1.61 24
1.62 24
1.63 24
1.64 24
1.65 24
1.66 24
1.67 24
1.68 24
1.69 24
1.7 24
1.71 24
1.72 24
1.73 24
1.74 24
1.75 24
1.76 24
1.77 24
1.78 24
1.79 24
1.8 24
1.81 24
1.82 24
1.83 24
1.84 24
1.85 24
1.86 24
1.87 24
1.88 24
1.89 24
1.9 24
1.91 24
1.92 24
1.93 24
1.94 24
1.95 24
1.96 24
1.97 24
1.98 24
1.99 24
2 24
2.01 24
2.02 24
2.03 24
2.04 24
2.05 24
2.06 24
2.07 24
2.08 24
2.09 24
2.1 24
2.11 24
2.12 24
2.13 24
2.14 24
2.15 24
2.16 24
2.17 24
2.18 24
2.19 24
2.2 24
2.21 24
2.22 24
2.23 24
2.24 24
2.25 24
2.26 24
2.27 24
2.28 24
2.29 24
2.3 24
2.31 24
2.32 24
2.33 24
2.34 24
2.35 24
2.36 24
2.37 24
2.38 24
2.39 24
2.4 24
2.41 24
2.42 24
2.43 24
2.44 24
2.45 24
2.46 24
2.47 24
2.48 24
2.49 24
2.5 24
2.51 24
2.52 24
2.53 24
2.54 24
2.55 24
2.56 24
2.57 24
2.58 24
2.59 24
2.6 24
2.61 24
2.62 24
2.63 24
2.64 24
2.65 24
2.66 24
2.67 24
2.68 24
2.69 24
2.7 24
2.71 24
2.72 24
2.73 24
2.74 24
2.75 24
2.76 24
2.77 24
2.78 24
2.79 24
2.8 24
2.81 24
2.82 24
2.83 24
2.84 24
2.85 24
2.86 24
2.87 24
2.88 24
2.89 24
2.9 24
2.91 24
2.92 24
2.93 24
2.94 24
2.95 24
2.96 24
2.97 24
2.98 24
2.99 24
3 24
3.01 24
3.02 24
3.03 24
3.04 24
3.05 24
3.06 24
3.07 24
3.08 24
3.09 24
3.1 24
3.11 24
3.12 24
3.13 24
3.14 24
3.15 24
3.16 24
3.17 24
3.18 24
3.19 24
3.2 24
3.21 24
3.22 24
3.23 24
3.24 24
3.25 24
3.26 24
3.27 24
3.28 24
3.29 24
3.3 24
3.31 24
3.32 24
3.33 24
3.34 24
3.35 24
3.36 24
3.37 24
3.38 24
3.39 24
3.4 24
3.41 24
3.42 24
3.43 24
3.44 24
3.45 24
3.46 24
3.47 24
3.48 24
3.49 24
3.5 24
3.51 24
3.52 24
3.53 24
3.54 24
3.55 24
3.56 24
3.57 24
3.58 24
3.59 24
3.6 24
3.61 24
3.62 24
3.63 24
3.64 24
3.65 24
3.66 24
3.67 24
3.68 24
3.69 24
3.7 24
3.71 24
3.72 24
3.73 24
3.74 24
3.75 24
3.76 24
3.77 24
3.78 24
3.79 24
3.8 24
3.81 24
3.82 24
3.83 24
3.84 24
3.85 24
3.86 24
3.87 24
3.88 24
3.89 24
3.9 24
3.91 24
3.92 24
3.93 24
3.94 24
3.95 24
3.96 24
3.97 24
3.98 24
3.99 24
4 24
4.01 23.9997532611583
4.02 23.9990130598533
4.03 23.997779441744
4.04 23.9960524829259
4.05 23.9938322899258
4.06 23.9911189996958
4.07 23.9879127796043
4.08 23.9842138274262
4.09 23.9800223713302
4.1 23.975338669865
4.11 23.9701630119435
4.12 23.9644957168246
4.13 23.9583371340938
4.14 23.9516876436414
4.15 23.9445476556394
4.16 23.9369176105158
4.17 23.9287979789278
4.18 23.9201892617325
4.19 23.911091989956
4.2 23.9015067247611
4.21 23.891434057412
4.22 23.8808746092382
4.23 23.8698290315963
4.24 23.8582980058295
4.25 23.8462822432258
4.26 23.8337824849741
4.27 23.8207995021183
4.28 23.80733409551
4.29 23.7933870957588
4.3 23.7789593631814
4.31 23.7640517877483
4.32 23.748665289029
4.33 23.7328008161353
4.34 23.7164593476624
4.35 23.6996418916292
4.36 23.6823494854155
4.37 23.6645831956986
4.38 23.6463441183866
4.39 23.627633378552
4.4 23.6084521303612
4.41 23.588801557004
4.42 23.5686828706204
4.43 23.5480973122256
4.44 23.5270461516338
4.45 23.5055306873799
4.46 23.4835522466389
4.47 23.4611121851448
4.48 23.438211887106
4.49 23.414852765121
4.5 23.3910362600903
4.51 23.3667638411282
4.52 23.3420370054718
4.53 23.3168572783889
4.54 23.2912262130836
4.55 23.2651453906007
4.56 23.2386164197282
4.57 23.211640936898
4.58 23.1842206060849
4.59 23.1563571187042
4.6 23.1280521935069
4.61 23.0993075764743
4.62 23.0701250407095
4.63 23.0405063863291
4.64 23.0104534403509
4.65 22.9799680565824
4.66 22.9490521155055
4.67 22.9177075241612
4.68 22.8859362160315
4.69 22.8537401509204
4.7 22.8211213148327
4.71 22.788081719852
4.72 22.7546234040161
4.73 22.7207484311915
4.74 22.6864588909462
4.75 22.6517568984204
4.76 22.6166445941965
4.77 22.5811241441669
4.78 22.5451977394002
4.79 22.5088675960063
4.8 22.4721359549996
4.81 22.4350050821607
4.82 22.3974772678967
4.83 22.3595548271001
4.84 22.3212400990055
4.85 22.282535447046
4.86 22.2434432587066
4.87 22.2039659453779
4.88 22.1641059422063
4.89 22.1238657079447
4.9 22.0832477248002
4.91 22.0422544982815
4.92 22.0008885570437
4.93 21.959152452733
4.94 21.9170487598289
4.95 21.8745800754855
4.96 21.8317490193713
4.97 21.7885582335076
4.98 21.7450103821055
4.99 21.7011081514016
5 21.6568542494924
5.01 21.6122514061668
5.02 21.5673023727385
5.03 21.5220099218755
5.04 21.4763768474295
5.05 21.4304059642635
5.06 21.3841001080782
5.07 21.3374621352368
5.08 21.2904949225892
5.09 21.2432013672944
5.1 21.1955843866415
5.11 21.1476469178702
5.12 21.0993919179895
5.13 21.050822363595
5.14 21.0019412506856
5.15 20.9527515944787
5.16 20.9032564292238
5.17 20.853458808016
5.18 20.8033618026071
5.19 20.7529685032163
5.2 20.7022820183398
5.21 20.6513054745586
5.22 20.6000420163462
5.23 20.5484948058741
5.24 20.496667022817
5.25 20.4445618641568
5.26 20.3921825439851
5.27 20.3395322933049
5.28 20.286614359832
5.29 20.2334320077935
5.3 20.1799885177276
5.31 20.1262871862804
5.32 20.072331326003
5.33 20.018124265147
5.34 19.9636693474593
5.35 19.9089699319756
5.36 19.8540293928137
5.37 19.7988511189648
5.38 19.7434385140846
5.39 19.6877949962837
5.4 19.6319239979164
5.41 19.575828965369
5.42 19.5195133588473
5.43 19.4629806521633
5.44 19.4062343325206
5.45 19.3492779002994
5.46 19.2921148688409
5.47 19.23474876423
5.48 19.1771831250782
5.49 19.1194215023055
5.5 19.0614674589207
5.51 19.0033245698023
5.52 18.9449964214774
5.53 18.8864866119011
5.54 18.8277987502341
5.55 18.7689364566199
5.56 18.7099033619623
5.57 18.6507031077006
5.58 18.5913393455852
5.59 18.5318157374527
5.6 18.4721359549996
5.61 18.412303679556
5.62 18.3523226018584
5.63 18.2921964218224
5.64 18.2319288483138
5.65 18.1715235989206
5.66 18.110984399723
5.67 18.050314985064
5.68 17.9895190973188
5.69 17.9286004866643
5.7 17.8675629108472
5.71 17.8064101349528
5.72 17.7451459311723
5.73 17.6837740785704
5.74 17.6222983628521
5.75 17.560722576129
5.76 17.4990505166858
5.77 17.4372859887455
5.78 17.3754328022353
5.79 17.3134947725509
5.8 17.2514757203218
5.81 17.1893794711754
5.82 17.1272098555007
5.83 17.0649707082124
5.84 17.0026658685144
5.85 16.9402991796627
5.86 16.8778744887284
5.87 16.8153956463604
5.88 16.7528665065481
5.89 16.6902909263834
5.9 16.6276727658228
5.91 16.5650158874493
5.92 16.5023241562345
5.93 16.4396014392996
5.94 16.3768516056771
5.95 16.3140785260725
5.96 16.251286072625
5.97 16.1884781186689
5.98 16.1256585384946
5.99 16.0628312071097
6 16
6.01 15.9371687928903
6.02 15.8743414615054
6.03 15.8115218813311
6.04 15.748713927375
6.05 15.6859214739275
6.06 15.6231483943229
6.07 15.5603985607004
6.08 15.4976758437655
6.09 15.4349841125507
6.1 15.3723272341772
6.11 15.3097090736166
6.12 15.2471334934519
6.13 15.1846043536396
6.14 15.1221255112716
6.15 15.0597008203373
6.16 14.9973341314856
6.17 14.9350292917876
6.18 14.8727901444993
6.19 14.8106205288246
6.2 14.7485242796782
6.21 14.6865052274491
6.22 14.6245671977647
6.23 14.5627140112545
6.24 14.5009494833142
6.25 14.439277423871
6.26 14.3777016371479
6.27 14.3162259214296
6.28 14.2548540688277
6.29 14.1935898650472
6.3 14.1324370891528
6.31 14.0713995133357
6.32 14.0104809026812
6.33 13.949685014936
6.34 13.889015600277
6.35 13.8284764010794
6.36 13.7680711516862
6.37 13.7078035781776
6.38 13.6476773981416
6.39 13.587696320444
6.4 13.5278640450004
6.41 13.4681842625473
6.42 13.4086606544148
6.43 13.3492968922994
6.44 13.2900966380377
6.45 13.2310635433801
6.46 13.1722012497659
6.47 13.1135133880989
6.48 13.0550035785226
6.49 12.9966754301977
6.5 12.9385325410793
6.51 12.8805784976945
6.52 12.8228168749218
6.53 12.76525123577
6.54 12.7078851311591
6.55 12.6507220997006
6.56 12.5937656674794
6.57 12.5370193478367
6.58 12.4804866411527
6.59 12.424171034631
6.6 12.3680760020836
6.61 12.3122050037163
6.62 12.2565614859154
6.63 12.2011488810352
6.64 12.1459706071863
6.65 12.0910300680244
6.66 12.0363306525407
6.67 11.981875734853
6.68 11.927668673997
6.69 11.8737128137196
6.7 11.8200114822724
6.71 11.7665679922065
6.72 11.713385640168
6.73 11.660467706695
6.74 11.6078174560149
6.75 11.5554381358432
6.76 11.503332977183
6.77 11.4515051941259
6.78 11.3999579836538
6.79 11.3486945254414
6.8 11.2977179816602
6.81 11.2470314967837
6.82 11.1966381973929
6.83 11.146541191984
6.84 11.0967435707762
6.85 11.0472484055213
6.86 10.9980587493144
6.87 10.949177636405
6.88 10.9006080820105
6.89 10.8523530821298
6.9 10.8044156133585
6.91 10.7567986327056
6.92 10.7095050774108
6.93 10.6625378647632
6.94 10.6158998919218
6.95 10.5695940357365
6.96 10.5236231525705
6.97 10.4779900781245
6.98 10.4326976272615
6.99 10.3877485938332
7 10.3431457505076
7.01 10.2988918485984
7.02 10.2549896178945
7.03 10.2114417664924
7.04 10.1682509806287
7.05 10.1254199245145
7.06 10.0829512401711
7.07 10.040847547267
7.08 9.99911144295632
7.09 9.95774550171853
7.1 9.91675227519975
7.11 9.87613429205529
7.12 9.83589405779369
7.13 9.79603405462213
7.14 9.75655674129336
7.15 9.71746455295404
7.16 9.67875990099448
7.17 9.64044517289992
7.18 9.60252273210328
7.19 9.56499491783932
7.2 9.52786404500042
7.21 9.4911324039937
7.22 9.45480226059981
7.23 9.41887585583312
7.24 9.3833554058035
7.25 9.34824310157964
7.26 9.31354110905384
7.27 9.27925156880846
7.28 9.24537659598388
7.29 9.21191828014797
7.3 9.17887868516726
7.31 9.1462598490796
7.32 9.11406378396845
7.33 9.08229247583876
7.34 9.05094788449447
7.35 9.02003194341762
7.36 8.98954655964909
7.37 8.95949361367094
7.38 8.92987495929045
7.39 8.90069242352573
7.4 8.87194780649306
7.41 8.8436428812958
7.42 8.81577939391508
7.43 8.78835906310204
7.44 8.76138358027184
7.45 8.73485460939935
7.46 8.70877378691644
7.47 8.68314272161109
7.48 8.65796299452815
7.49 8.63323615887179
7.5 8.60896373990971
7.51 8.58514723487903
7.52 8.56178811289399
7.53 8.53888781485525
7.54 8.51644775336106
7.55 8.49446931262013
7.56 8.4729538483662
7.57 8.45190268777445
7.58 8.43131712937964
7.59 8.41119844299596
7.6 8.39154786963877
7.61 8.37236662144799
7.62 8.35365588161336
7.63 8.33541680430145
7.64 8.31765051458446
7.65 8.30035810837082
7.66 8.28354065233762
7.67 8.26719918386474
7.68 8.25133471097095
7.69 8.23594821225166
7.7 8.22104063681859
7.71 8.2066129042412
7.72 8.19266590449002
7.73 8.17920049788167
7.74 8.16621751502587
7.75 8.15371775677416
7.76 8.14170199417049
7.77 8.13017096840371
7.78 8.11912539076181
7.79 8.10856594258804
7.8 8.0984932752389
7.81 8.08890801004396
7.82 8.07981073826754
7.83 8.0712020210722
7.84 8.06308238948418
7.85 8.05545234436059
7.86 8.04831235635856
7.87 8.04166286590619
7.88 8.03550428317536
7.89 8.02983698805647
7.9 8.02466133013498
7.91 8.01997762866984
7.92 8.01578617257383
7.93 8.0120872203957
7.94 8.00888100030424
7.95 8.00616771007422
7.96 8.00394751707415
7.97 8.002220558256
7.98 8.00098694014672
7.99 8.00024673884168
8 8
8.01 8.00003084235521
8.02 8.00012336751834
8.03 8.000277569782
8.04 8.00049343963427
8.05 8.00077096375928
8.06 8.00111012503803
8.07 8.00151090254946
8.08 8.00197327157173
8.09 8.00249720358373
8.1 8.00308266626687
8.11 8.00372962350706
8.12 8.00443803539692
8.13 8.00520785823827
8.14 8.00603904454482
8.15 8.00693154304507
8.16 8.00788529868552
8.17 8.00890025263402
8.18 8.00997634228344
8.19 8.01111350125549
8.2 8.01231165940486
8.21 8.0135707428235
8.22 8.01489067384523
8.23 8.01627137105046
8.24 8.01771274927131
8.25 8.01921471959677
8.26 8.02077718937823
8.27 8.02240006223521
8.28 8.02408323806125
8.29 8.02582661303015
8.3 8.02763007960232
8.31 8.02949352653146
8.32 8.03141683887137
8.33 8.03339989798309
8.34 8.0354425815422
8.35 8.03754476354635
8.36 8.03970631432306
8.37 8.04192710053768
8.38 8.04420698520167
8.39 8.046545827681
8.4 8.04894348370485
8.41 8.05139980537449
8.42 8.05391464117245
8.43 8.05648783597181
8.44 8.05911923104577
8.45 8.06180866407751
8.46 8.06455596917013
8.47 8.06736097685691
8.48 8.07022351411175
8.49 8.07314340435988
8.5 8.07612046748871
8.51 8.07915451985897
8.52 8.08224537431602
8.53 8.08539284020139
8.54 8.08859672336456
8.55 8.09185682617492
8.56 8.09517294753398
8.57 8.09854488288775
8.58 8.10197242423938
8.59 8.10545536016198
8.6 8.10899347581163
8.61 8.11258655294072
8.62 8.11623436991131
8.63 8.11993670170887
8.64 8.12369331995614
8.65 8.1275039929272
8.66 8.13136848556181
8.67 8.13528655947984
8.68 8.13925797299606
8.69 8.14328248113495
8.7 8.14735983564591
8.71 8.1514897850185
8.72 8.15567207449799
8.73 8.15990644610106
8.74 8.16419263863173
8.75 8.16853038769746
8.76 8.17291942572544
8.77 8.17735948197914
8.78 8.18185028257498
8.79 8.18639155049921
8.8 8.19098300562505
8.81 8.19562436472992
8.82 8.20031534151291
8.83 8.20505564661249
8.84 8.20984498762431
8.85 8.21468306911925
8.86 8.21956959266167
8.87 8.22450425682777
8.88 8.22948675722421
8.89 8.23451678650691
8.9 8.23959403439997
8.91 8.24471818771482
8.92 8.24988893036954
8.93 8.25510594340838
8.94 8.26036890502139
8.95 8.26567749056431
8.96 8.27103137257859
8.97 8.27643022081155
8.98 8.28187370223681
8.99 8.28736148107479
9 8.29289321881345
9.01 8.29846857422914
9.02 8.30408720340769
9.03 8.30974875976556
9.04 8.31545289407131
9.05 8.32119925446706
9.06 8.32698748649023
9.07 8.3328172330954
9.08 8.33868813467635
9.09 8.34459982908821
9.1 8.35055195166981
9.11 8.35654413526622
9.12 8.36257601025131
9.13 8.36864720455062
9.14 8.3747573436643
9.15 8.38090605069016
9.16 8.38709294634702
9.17 8.393317648998
9.18 8.39957977467412
9.19 8.40587893709796
9.2 8.41221474770753
9.21 8.41858681568017
9.22 8.42499474795672
9.23 8.43143814926574
9.24 8.43791662214787
9.25 8.4444297669804
9.26 8.45097718200187
9.27 8.45755846333688
9.28 8.464173205021
9.29 8.47082099902581
9.3 8.47750143528405
9.31 8.48421410171495
9.32 8.49095858424963
9.33 8.49773446685663
9.34 8.50454133156759
9.35 8.51137875850305
9.36 8.51824632589829
9.37 8.5251436101294
9.38 8.53207018573943
9.39 8.53902562546454
9.4 8.54600950026045
9.41 8.55302137932888
9.42 8.56006083014409
9.43 8.56712741847959
9.44 8.57422070843493
9.45 8.58134026246257
9.46 8.58848564139489
9.47 8.59565640447126
9.48 8.60285210936522
9.49 8.61007231221181
9.5 8.61731656763491
9.51 8.62458442877472
9.52 8.63187544731532
9.53 8.63918917351236
9.54 8.64652515622074
9.55 8.65388294292251
9.56 8.66126207975471
9.57 8.66866211153743
9.58 8.67608258180185
9.59 8.68352303281841
9.6 8.69098300562505
9.61 8.6984620400555
9.62 8.7059596747677
9.63 8.7134754472722
9.64 8.72100889396077
9.65 8.72855955013492
9.66 8.73612695003463
9.67 8.743710626867
9.68 8.75131011283515
9.69 8.75892493916696
9.7 8.7665546361441
9.71 8.7741987331309
9.72 8.78185675860346
9.73 8.78952824017869
9.74 8.79721270464349
9.75 8.80490967798387
9.76 8.81261868541428
9.77 8.82033925140681
9.78 8.82807089972059
9.79 8.83581315343114
9.8 8.84356553495977
9.81 8.85132756610308
9.82 8.85909876806242
9.83 8.86687866147345
9.84 8.87466676643569
9.85 8.88246260254216
9.86 8.89026568890895
9.87 8.89807554420495
9.88 8.90589168668149
9.89 8.91371363420208
9.9 8.92154090427216
9.91 8.92937301406883
9.92 8.93720948047068
9.93 8.94504982008755
9.94 8.95289354929036
9.95 8.96074018424093
9.96 8.96858924092187
9.97 8.97644023516639
9.98 8.98429268268818
9.99 8.99214609911129
10 9
10.01 9.00785390088871
10.02 9.01570731731182
10.03 9.02355976483361
10.04 9.03141075907813
10.05 9.03925981575907
10.06 9.04710645070964
10.07 9.05495017991245
10.08 9.06279051952932
10.09 9.07062698593117
10.1 9.07845909572784
10.11 9.08628636579792
10.12 9.09410831331851
10.13 9.10192445579505
10.14 9.10973431109105
10.15 9.11753739745784
10.16 9.12533323356431
10.17 9.13312133852655
10.18 9.14090123193758
10.19 9.14867243389692
10.2 9.15643446504023
10.21 9.16418684656886
10.22 9.17192910027941
10.23 9.17966074859319
10.24 9.18738131458572
10.25 9.19509032201613
10.26 9.20278729535651
10.27 9.21047175982131
10.28 9.21814324139654
10.29 9.22580126686911
10.3 9.23344536385591
10.31 9.24107506083304
10.32 9.24868988716485
10.33 9.256289373133
10.34 9.26387304996537
10.35 9.27144044986508
10.36 9.27899110603923
10.37 9.2865245527278
10.38 9.29404032523231
10.39 9.3015379599445
10.4 9.30901699437495
10.41 9.31647696718159
10.42 9.32391741819815
10.43 9.33133788846257
10.44 9.33873792024529
10.45 9.34611705707749
10.46 9.35347484377926
10.47 9.36081082648764
10.48 9.36812455268468
10.49 9.37541557122528
10.5 9.38268343236509
10.51 9.38992768778819
10.52 9.39714789063478
10.53 9.40434359552874
10.54 9.41151435860511
10.55 9.41865973753743
10.56 9.42577929156507
10.57 9.43287258152041
10.58 9.43993916985591
10.59 9.44697862067112
10.6 9.45399049973955
10.61 9.46097437453546
10.62 9.46792981426058
10.63 9.4748563898706
10.64 9.48175367410171
10.65 9.48862124149695
10.66 9.49545866843241
10.67 9.50226553314337
10.68 9.50904141575037
10.69 9.51578589828505
10.7 9.52249856471595
10.71 9.52917900097419
10.72 9.535826794979
10.73 9.54244153666312
10.74 9.54902281799813
10.75 9.5555702330196
10.76 9.56208337785213
10.77 9.56856185073426
10.78 9.57500525204328
10.79 9.58141318431983
10.8 9.58778525229247
10.81 9.59412106290204
10.82 9.60042022532588
10.83 9.606682351002
10.84 9.61290705365298
10.85 9.61909394930984
10.86 9.6252426563357
10.87 9.63135279544938
10.88 9.63742398974869
10.89 9.64345586473378
10.9 9.64944804833019
10.91 9.65540017091179
10.92 9.66131186532365
10.93 9.6671827669046
10.94 9.67301251350977
10.95 9.67880074553294
10.96 9.68454710592869
10.97 9.69025124023444
10.98 9.69591279659231
10.99 9.70153142577086
11 9.70710678118655
11.01 9.7126385189252
11.02 9.71812629776319
11.03 9.72356977918845
11.04 9.72896862742141
11.05 9.73432250943569
11.06 9.73963109497861
11.07 9.74489405659162
11.08 9.75011106963046
11.09 9.75528181228518
11.1 9.76040596560003
11.11 9.76548321349309
11.12 9.77051324277579
11.13 9.77549574317224
11.14 9.78043040733833
11.15 9.78531693088075
11.16 9.79015501237569
11.17 9.79494435338751
11.18 9.79968465848709
11.19 9.80437563527008
11.2 9.80901699437495
11.21 9.81360844950079
11.22 9.81814971742502
11.23 9.82264051802086
11.24 9.82708057427456
11.25 9.83146961230254
11.26 9.83580736136827
11.27 9.84009355389894
11.28 9.84432792550201
11.29 9.84851021498151
11.3 9.85264016435409
11.31 9.85671751886505
11.32 9.86074202700394
11.33 9.86471344052016
11.34 9.86863151443819
11.35 9.8724960070728
11.36 9.87630668004386
11.37 9.88006329829113
11.38 9.88376563008869
11.39 9.88741344705928
11.4 9.89100652418837
11.41 9.89454463983802
11.42 9.89802757576062
11.43 9.90145511711225
11.44 9.90482705246602
11.45 9.90814317382508
11.46 9.91140327663545
11.47 9.91460715979861
11.48 9.91775462568398
11.49 9.92084548014103
11.5 9.92387953251129
11.51 9.92685659564012
11.52 9.92977648588825
11.53 9.93263902314309
11.54 9.93544403082987
11.55 9.93819133592248
11.56 9.94088076895423
11.57 9.94351216402819
11.58 9.94608535882755
11.59 9.94860019462551
11.6 9.95105651629515
11.61 9.953454172319
11.62 9.95579301479833
11.63 9.95807289946232
11.64 9.96029368567694
11.65 9.96245523645365
11.66 9.9645574184578
11.67 9.96660010201691
11.68 9.96858316112863
11.69 9.97050647346854
11.7 9.97236992039768
11.71 9.97417338696985
11.72 9.97591676193875
11.73 9.97759993776479
11.74 9.97922281062177
11.75 9.98078528040323
11.76 9.98228725072869
11.77 9.98372862894954
11.78 9.98510932615477
11.79 9.9864292571765
11.8 9.98768834059514
11.81 9.98888649874451
11.82 9.99002365771656
11.83 9.99109974736598
11.84 9.99211470131448
11.85 9.99306845695493
11.86 9.99396095545518
11.87 9.99479214176173
11.88 9.99556196460308
11.89 9.99627037649294
11.9 9.99691733373313
11.91 9.99750279641627
11.92 9.99802672842827
11.93 9.99848909745054
11.94 9.99888987496197
11.95 9.99922903624072
11.96 9.99950656036573
11.97 9.999722430218
11.98 9.99987663248166
11.99 9.99996915764479
12 10
};
\addplot [very thick, black, dashed]
table {%
0 20
0.01 20
0.02 20
0.03 20
0.04 20
0.05 20
0.06 20
0.07 20
0.08 20
0.09 20
0.1 20
0.11 20
0.12 20
0.13 20
0.14 20
0.15 20
0.16 20
0.17 20
0.18 20
0.19 20
0.2 20
0.21 20
0.22 20
0.23 20
0.24 20
0.25 20
0.26 20
0.27 20
0.28 20
0.29 20
0.3 20
0.31 20
0.32 20
0.33 20
0.34 20
0.35 20
0.36 20
0.37 20
0.38 20
0.39 20
0.4 20
0.41 20
0.42 20
0.43 20
0.44 20
0.45 20
0.46 20
0.47 20
0.48 20
0.49 20
0.5 20
0.51 20
0.52 20
0.53 20
0.54 20
0.55 20
0.56 20
0.57 20
0.58 20
0.59 20
0.6 20
0.61 20
0.62 20
0.63 20
0.64 20
0.65 20
0.66 20
0.67 20
0.68 20
0.69 20
0.7 20
0.71 20
0.72 20
0.73 20
0.74 20
0.75 20
0.76 20
0.77 20
0.78 20
0.79 20
0.8 20
0.81 20
0.82 20
0.83 20
0.84 20
0.85 20
0.86 20
0.87 20
0.88 20
0.89 20
0.9 20
0.91 20
0.92 20
0.93 20
0.94 20
0.95 20
0.96 20
0.97 20
0.98 20
0.99 20
1 20
1.01 20
1.02 20
1.03 20
1.04 20
1.05 20
1.06 20
1.07 20
1.08 20
1.09 20
1.1 20
1.11 20
1.12 20
1.13 20
1.14 20
1.15 20
1.16 20
1.17 20
1.18 20
1.19 20
1.2 20
1.21 20
1.22 20
1.23 20
1.24 20
1.25 20
1.26 20
1.27 20
1.28 20
1.29 20
1.3 20
1.31 20
1.32 20
1.33 20
1.34 20
1.35 20
1.36 20
1.37 20
1.38 20
1.39 20
1.4 20
1.41 20
1.42 20
1.43 20
1.44 20
1.45 20
1.46 20
1.47 20
1.48 20
1.49 20
1.5 20
1.51 20
1.52 20
1.53 20
1.54 20
1.55 20
1.56 20
1.57 20
1.58 20
1.59 20
1.6 20
1.61 20
1.62 20
1.63 20
1.64 20
1.65 20
1.66 20
1.67 20
1.68 20
1.69 20
1.7 20
1.71 20
1.72 20
1.73 20
1.74 20
1.75 20
1.76 20
1.77 20
1.78 20
1.79 20
1.8 20
1.81 20
1.82 20
1.83 20
1.84 20
1.85 20
1.86 20
1.87 20
1.88 20
1.89 20
1.9 20
1.91 20
1.92 20
1.93 20
1.94 20
1.95 20
1.96 20
1.97 20
1.98 20
1.99 20
2 20
2.01 20
2.02 20
2.03 20
2.04 20
2.05 20
2.06 20
2.07 20
2.08 20
2.09 20
2.1 20
2.11 20
2.12 20
2.13 20
2.14 20
2.15 20
2.16 20
2.17 20
2.18 20
2.19 20
2.2 20
2.21 20
2.22 20
2.23 20
2.24 20
2.25 20
2.26 20
2.27 20
2.28 20
2.29 20
2.3 20
2.31 20
2.32 20
2.33 20
2.34 20
2.35 20
2.36 20
2.37 20
2.38 20
2.39 20
2.4 20
2.41 20
2.42 20
2.43 20
2.44 20
2.45 20
2.46 20
2.47 20
2.48 20
2.49 20
2.5 20
2.51 20
2.52 20
2.53 20
2.54 20
2.55 20
2.56 20
2.57 20
2.58 20
2.59 20
2.6 20
2.61 20
2.62 20
2.63 20
2.64 20
2.65 20
2.66 20
2.67 20
2.68 20
2.69 20
2.7 20
2.71 20
2.72 20
2.73 20
2.74 20
2.75 20
2.76 20
2.77 20
2.78 20
2.79 20
2.8 20
2.81 20
2.82 20
2.83 20
2.84 20
2.85 20
2.86 20
2.87 20
2.88 20
2.89 20
2.9 20
2.91 20
2.92 20
2.93 20
2.94 20
2.95 20
2.96 20
2.97 20
2.98 20
2.99 20
3 20
3.01 19.9997779355447
3.02 19.999111761904
3.03 19.9980015382513
3.04 19.9964473632029
3.05 19.9944493748099
3.06 19.9920077505449
3.07 19.9891227072872
3.08 19.9857945013031
3.09 19.982023428223
3.1 19.9778098230154
3.11 19.9731540599572
3.12 19.9680565526
3.13 19.9625177537341
3.14 19.9565381553475
3.15 19.9501182885828
3.16 19.9432587236896
3.17 19.9359600699741
3.18 19.928222975745
3.19 19.9200481282557
3.2 19.9114362536434
3.21 19.9023881168647
3.22 19.8929045216274
3.23 19.8829863103191
3.24 19.8726343639329
3.25 19.8618496019884
3.26 19.8506329824505
3.27 19.8389855016443
3.28 19.8269081941664
3.29 19.8144021327929
3.3 19.8014684283847
3.31 19.7881082297881
3.32 19.7743227237332
3.33 19.7601131347284
3.34 19.7454807249515
3.35 19.7304267941377
3.36 19.7149526794643
3.37 19.6990597554316
3.38 19.682749433741
3.39 19.6660231631695
3.4 19.6488824294413
3.41 19.6313287550953
3.42 19.6133636993506
3.43 19.5949888579671
3.44 19.5762058631046
3.45 19.5570163831772
3.46 19.5374221227056
3.47 19.5174248221652
3.48 19.4970262578319
3.49 19.4762282416241
3.5 19.4550326209418
3.51 19.4334412785028
3.52 19.4114561321748
3.53 19.3890791348056
3.54 19.3663122740496
3.55 19.343157572191
3.56 19.3196170859642
3.57 19.2956929063714
3.58 19.2713871584965
3.59 19.2467020013166
3.6 19.2216396275101
3.61 19.1962022632619
3.62 19.1703921680659
3.63 19.1442116345238
3.64 19.1176629881421
3.65 19.0907485871251
3.66 19.0634708221655
3.67 19.035832116232
3.68 19.0078349243544
3.69 18.9794817334051
3.7 18.9507750618785
3.71 18.9217174596671
3.72 18.8923115078351
3.73 18.8625598183893
3.74 18.8324650340467
3.75 18.8020298280002
3.76 18.7712569036805
3.77 18.7401489945169
3.78 18.7087088636937
3.79 18.6769393039051
3.8 18.6448431371071
3.81 18.6124232142667
3.82 18.5796824151092
3.83 18.5466236478614
3.84 18.5132498489942
3.85 18.4795639829616
3.86 18.4455690419367
3.87 18.411268045547
3.88 18.3766640406051
3.89 18.341760100839
3.9 18.3065593266183
3.91 18.2710648446793
3.92 18.2352798078472
3.93 18.1992073947559
3.94 18.1628508095656
3.95 18.1262132816785
3.96 18.0892980654517
3.97 18.052108439908
3.98 18.0146477084451
3.99 17.9769191985418
4 17.9389262614624
4.01 17.9006722719592
4.02 17.862160627973
4.03 17.8233947503304
4.04 17.7843780824409
4.05 17.7451140899907
4.06 17.7056062606344
4.07 17.6658581036859
4.08 17.6258731498065
4.09 17.5856549506908
4.1 17.5452070787519
4.11 17.5045331268035
4.12 17.4636367077415
4.13 17.422521454222
4.14 17.3811910183397
4.15 17.3396490713029
4.16 17.2978993031074
4.17 17.2559454222092
4.18 17.2137911551945
4.19 17.171440246449
4.2 17.1288964578254
4.21 17.0861635683088
4.22 17.0432453736817
4.23 17.0001456861863
4.24 16.956868334186
4.25 16.9134171618254
4.26 16.869796028689
4.27 16.8260088094579
4.28 16.7820593935663
4.29 16.7379516848552
4.3 16.6936896012265
4.31 16.6492770742943
4.32 16.604718049036
4.33 16.5600164834421
4.34 16.5151763481639
4.35 16.4702016261615
4.36 16.4250963123499
4.37 16.3798644132437
4.38 16.3345099466019
4.39 16.2890369410703
4.4 16.2434494358243
4.41 16.1977514802096
4.42 16.151947133383
4.43 16.1060404639512
4.44 16.0600355496103
4.45 16.0139364767826
4.46 15.9677473402543
4.47 15.9214722428117
4.48 15.8751152948764
4.49 15.8286806141406
4.5 15.7821723252012
4.51 15.7355945591932
4.52 15.6889514534232
4.53 15.6422471510015
4.54 15.5954858004743
4.55 15.5486715554552
4.56 15.5018085742561
4.57 15.4549010195178
4.58 15.4079530578408
4.59 15.3609688594143
4.6 15.3139525976466
4.61 15.2669084487938
4.62 15.2198405915893
4.63 15.1727532068724
4.64 15.1256504772167
4.65 15.0785365865591
4.66 15.0314157198278
4.67 14.9842920625706
4.68 14.9371698005832
4.69 14.8900531195375
4.7 14.8429462046094
4.71 14.7958532401075
4.72 14.7487784091012
4.73 14.7017258930491
4.74 14.654699871428
4.75 14.6077045213608
4.76 14.5607440172463
4.77 14.513822530388
4.78 14.4669442286237
4.79 14.4201132759552
4.8 14.3733338321785
4.81 14.3266100525142
4.82 14.2799460872387
4.83 14.2333460813152
4.84 14.1868141740256
4.85 14.1403544986029
4.86 14.0939711818643
4.87 14.0476683438441
4.88 14.001450097428
4.89 13.9553205479879
4.9 13.9092837930173
4.91 13.8633439217668
4.92 13.8175050148814
4.93 13.7717711440379
4.94 13.7261463715831
4.95 13.6806347501731
4.96 13.6352403224134
4.97 13.5899671204994
4.98 13.5448191658586
4.99 13.4998004687936
5 13.4549150281253
5.01 13.4101668308379
5.02 13.3655598517253
5.03 13.3210980530371
5.04 13.2767853841274
5.05 13.2326257811037
5.06 13.1886231664773
5.07 13.1447814488147
5.08 13.101104522391
5.09 13.0575962668432
5.1 13.0142605468261
5.11 12.971101211669
5.12 12.9281220950336
5.13 12.8853270145735
5.14 12.8427197715952
5.15 12.8003041507204
5.16 12.7580839195498
5.17 12.7160628283285
5.18 12.6742446096127
5.19 12.6326329779384
5.2 12.5912316294914
5.21 12.5500442417788
5.22 12.5090744733025
5.23 12.4683259632343
5.24 12.4278023310925
5.25 12.3875071764203
5.26 12.3474440784663
5.27 12.3076165958667
5.28 12.2680282663287
5.29 12.2286826063165
5.3 12.1895831107393
5.31 12.1507332526404
5.32 12.1121364828887
5.33 12.0737962298724
5.34 12.0357158991947
5.35 11.9978988733706
5.36 11.960348511527
5.37 11.9230681491041
5.38 11.8860610975594
5.39 11.8493306440732
5.4 11.8128800512565
5.41 11.776712556862
5.42 11.7408313734956
5.43 11.7052396883314
5.44 11.6699406628287
5.45 11.6349374324511
5.46 11.6002331063879
5.47 11.565830767278
5.48 11.531733470936
5.49 11.497944246081
5.5 11.4644660940673
5.51 11.4313019886179
5.52 11.3984548755605
5.53 11.3659276725655
5.54 11.3337232688872
5.55 11.301844525107
5.56 11.2702942728791
5.57 11.2390753146794
5.58 11.2081904235564
5.59 11.1776423428845
5.6 11.1474337861211
5.61 11.1175674365646
5.62 11.0880459471171
5.63 11.0588719400478
5.64 11.0300480067608
5.65 11.0015767075645
5.66 10.9734605714444
5.67 10.9457020958381
5.68 10.9183037464141
5.69 10.891267956852
5.7 10.8645971286272
5.71 10.8382936307967
5.72 10.8123597997893
5.73 10.7867979391978
5.74 10.7616103195746
5.75 10.7367991782295
5.76 10.7123667190317
5.77 10.6883151122135
5.78 10.6646464941775
5.79 10.6413629673075
5.8 10.6184665997807
5.81 10.595959425385
5.82 10.5738434433377
5.83 10.5521206181083
5.84 10.5307928792437
5.85 10.5098621211969
5.86 10.489330203159
5.87 10.4691989488936
5.88 10.449470146575
5.89 10.4301455486297
5.9 10.4112268715801
5.91 10.3927157958925
5.92 10.3746139658277
5.93 10.3569229892949
5.94 10.3396444377089
5.95 10.3227798458507
5.96 10.3063307117306
5.97 10.290298496456
5.98 10.274684624101
5.99 10.2594904815798
6 10.2447174185242
6.01 10.230366747163
6.02 10.2164397422058
6.03 10.2029376407298
6.04 10.1898616420696
6.05 10.177212907711
6.06 10.1649925611878
6.07 10.1532016879819
6.08 10.1418413354266
6.09 10.1309125126144
6.1 10.1204161903063
6.11 10.1103533008464
6.12 10.1007247380788
6.13 10.091531357268
6.14 10.0827739750234
6.15 10.0744533692261
6.16 10.0665702789607
6.17 10.0591254044486
6.18 10.0521194069867
6.19 10.0455529088883
6.2 10.0394264934276
6.21 10.0337407047883
6.22 10.028496048015
6.23 10.0236929889685
6.24 10.0193319542841
6.25 10.0154133313344
6.26 10.0119374681939
6.27 10.0089046736089
6.28 10.0063152169699
6.29 10.0041693282873
6.3 10.0024671981713
6.31 10.0012089778151
6.32 10.0003947789809
6.33 10.0000246739907
6.34 10
6.35 10
6.36 10
6.37 10
6.38 10
6.39 10
6.4 10
6.41 10
6.42 10
6.43 10
6.44 10
6.45 10
6.46 10
6.47 10
6.48 10
6.49 10
6.5 10
6.51 10
6.52 10
6.53 10
6.54 10
6.55 10
6.56 10
6.57 10
6.58 10
6.59 10
6.6 10
6.61 10
6.62 10
6.63 10
6.64 10
6.65 10
6.66 10
6.67 10
6.68 10
6.69 10
6.7 10
6.71 10
6.72 10
6.73 10
6.74 10
6.75 10
6.76 10
6.77 10
6.78 10
6.79 10
6.8 10
6.81 10
6.82 10
6.83 10
6.84 10
6.85 10
6.86 10
6.87 10
6.88 10
6.89 10
6.9 10
6.91 10
6.92 10
6.93 10
6.94 10
6.95 10
6.96 10
6.97 10
6.98 10
6.99 10
7 10
7.01 10
7.02 10
7.03 10
7.04 10
7.05 10
7.06 10
7.07 10
7.08 10
7.09 10
7.1 10
7.11 10
7.12 10
7.13 10
7.14 10
7.15 10
7.16 10
7.17 10
7.18 10
7.19 10
7.2 10
7.21 10
7.22 10
7.23 10
7.24 10
7.25 10
7.26 10
7.27 10
7.28 10
7.29 10
7.3 10
7.31 10
7.32 10
7.33 10
7.34 10
7.35 10
7.36 10
7.37 10
7.38 10
7.39 10
7.4 10
7.41 10
7.42 10
7.43 10
7.44 10
7.45 10
7.46 10
7.47 10
7.48 10
7.49 10
7.5 10
7.51 10
7.52 10
7.53 10
7.54 10
7.55 10
7.56 10
7.57 10
7.58 10
7.59 10
7.6 10
7.61 10
7.62 10
7.63 10
7.64 10
7.65 10
7.66 10
7.67 10
7.68 10
7.69 10
7.7 10
7.71 10
7.72 10
7.73 10
7.74 10
7.75 10
7.76 10
7.77 10
7.78 10
7.79 10
7.8 10
7.81 10
7.82 10
7.83 10
7.84 10
7.85 10
7.86 10
7.87 10
7.88 10
7.89 10
7.9 10
7.91 10
7.92 10
7.93 10
7.94 10
7.95 10
7.96 10
7.97 10
7.98 10
7.99 10
8 10
8.01 10
8.02 10
8.03 10
8.04 10
8.05 10
8.06 10
8.07 10
8.08 10
8.09 10
8.1 10
8.11 10
8.12 10
8.13 10
8.14 10
8.15 10
8.16 10
8.17 10
8.18 10
8.19 10
8.2 10
8.21 10
8.22 10
8.23 10
8.24 10
8.25 10
8.26 10
8.27 10
8.28 10
8.29 10
8.3 10
8.31 10
8.32 10
8.33 10
8.34 10
8.35 10
8.36 10
8.37 10
8.38 10
8.39 10
8.4 10
8.41 10
8.42 10
8.43 10
8.44 10
8.45 10
8.46 10
8.47 10
8.48 10
8.49 10
8.5 10
8.51 10
8.52 10
8.53 10
8.54 10
8.55 10
8.56 10
8.57 10
8.58 10
8.59 10
8.6 10
8.61 10
8.62 10
8.63 10
8.64 10
8.65 10
8.66 10
8.67 10
8.68 10
8.69 10
8.7 10
8.71 10
8.72 10
8.73 10
8.74 10
8.75 10
8.76 10
8.77 10
8.78 10
8.79 10
8.8 10
8.81 10
8.82 10
8.83 10
8.84 10
8.85 10
8.86 10
8.87 10
8.88 10
8.89 10
8.9 10
8.91 10
8.92 10
8.93 10
8.94 10
8.95 10
8.96 10
8.97 10
8.98 10
8.99 10
9 10
9.01 10
9.02 10
9.03 10
9.04 10
9.05 10
9.06 10
9.07 10
9.08 10
9.09 10
9.1 10
9.11 10
9.12 10
9.13 10
9.14 10
9.15 10
9.16 10
9.17 10
9.18 10
9.19 10
9.2 10
9.21 10
9.22 10
9.23 10
9.24 10
9.25 10
9.26 10
9.27 10
9.28 10
9.29 10
9.3 10
9.31 10
9.32 10
9.33 10
9.34 10
9.35 10
9.36 10
9.37 10
9.38 10
9.39 10
9.4 10
9.41 10
9.42 10
9.43 10
9.44 10
9.45 10
9.46 10
9.47 10
9.48 10
9.49 10
9.5 10
9.51 10
9.52 10
9.53 10
9.54 10
9.55 10
9.56 10
9.57 10
9.58 10
9.59 10
9.6 10
9.61 10
9.62 10
9.63 10
9.64 10
9.65 10
9.66 10
9.67 10
9.68 10
9.69 10
9.7 10
9.71 10
9.72 10
9.73 10
9.74 10
9.75 10
9.76 10
9.77 10
9.78 10
9.79 10
9.8 10
9.81 10
9.82 10
9.83 10
9.84 10
9.85 10
9.86 10
9.87 10
9.88 10
9.89 10
9.9 10
9.91 10
9.92 10
9.93 10
9.94 10
9.95 10
9.96 10
9.97 10
9.98 10
9.99 10
10 10
10.01 10
10.02 10
10.03 10
10.04 10
10.05 10
10.06 10
10.07 10
10.08 10
10.09 10
10.1 10
10.11 10
10.12 10
10.13 10
10.14 10
10.15 10
10.16 10
10.17 10
10.18 10
10.19 10
10.2 10
10.21 10
10.22 10
10.23 10
10.24 10
10.25 10
10.26 10
10.27 10
10.28 10
10.29 10
10.3 10
10.31 10
10.32 10
10.33 10
10.34 10
10.35 10
10.36 10
10.37 10
10.38 10
10.39 10
10.4 10
10.41 10
10.42 10
10.43 10
10.44 10
10.45 10
10.46 10
10.47 10
10.48 10
10.49 10
10.5 10
10.51 10
10.52 10
10.53 10
10.54 10
10.55 10
10.56 10
10.57 10
10.58 10
10.59 10
10.6 10
10.61 10
10.62 10
10.63 10
10.64 10
10.65 10
10.66 10
10.67 10
10.68 10
10.69 10
10.7 10
10.71 10
10.72 10
10.73 10
10.74 10
10.75 10
10.76 10
10.77 10
10.78 10
10.79 10
10.8 10
10.81 10
10.82 10
10.83 10
10.84 10
10.85 10
10.86 10
10.87 10
10.88 10
10.89 10
10.9 10
10.91 10
10.92 10
10.93 10
10.94 10
10.95 10
10.96 10
10.97 10
10.98 10
10.99 10
11 10
11.01 10
11.02 10
11.03 10
11.04 10
11.05 10
11.06 10
11.07 10
11.08 10
11.09 10
11.1 10
11.11 10
11.12 10
11.13 10
11.14 10
11.15 10
11.16 10
11.17 10
11.18 10
11.19 10
11.2 10
11.21 10
11.22 10
11.23 10
11.24 10
11.25 10
11.26 10
11.27 10
11.28 10
11.29 10
11.3 10
11.31 10
11.32 10
11.33 10
11.34 10
11.35 10
11.36 10
11.37 10
11.38 10
11.39 10
11.4 10
11.41 10
11.42 10
11.43 10
11.44 10
11.45 10
11.46 10
11.47 10
11.48 10
11.49 10
11.5 10
11.51 10
11.52 10
11.53 10
11.54 10
11.55 10
11.56 10
11.57 10
11.58 10
11.59 10
11.6 10
11.61 10
11.62 10
11.63 10
11.64 10
11.65 10
11.66 10
11.67 10
11.68 10
11.69 10
11.7 10
11.71 10
11.72 10
11.73 10
11.74 10
11.75 10
11.76 10
11.77 10
11.78 10
11.79 10
11.8 10
11.81 10
11.82 10
11.83 10
11.84 10
11.85 10
11.86 10
11.87 10
11.88 10
11.89 10
11.9 10
11.91 10
11.92 10
11.93 10
11.94 10
11.95 10
11.96 10
11.97 10
11.98 10
11.99 10
12 10
};
\end{axis}

\begin{axis}[
axis y line=right,
height=\figureheight,
scaled y ticks=false,
tick align=outside,
width=\figurewidth,
x grid style={white!69.0196078431373!black},
xmin=0, xmax=12,
xtick pos=left,
xtick style={color=black},
xticklabel style={align=center},
y grid style={white!69.0196078431373!black},
ylabel={Distance [\si{\meter}]},
ymin=-11.2, ymax=59.2,
ytick pos=right,
ytick style={color=black},
yticklabel style={/pgf/number format/fixed,/pgf/number format/precision=3},
yticklabel style={anchor=west}
]
\addplot [very thick, gray, dotted]
table {%
0 39.96
0.01 39.92
0.02 39.88
0.03 39.84
0.04 39.8
0.05 39.76
0.06 39.72
0.07 39.68
0.08 39.64
0.09 39.6
0.1 39.56
0.11 39.52
0.12 39.48
0.13 39.44
0.14 39.4
0.15 39.36
0.16 39.32
0.17 39.28
0.18 39.24
0.19 39.2
0.2 39.16
0.21 39.12
0.22 39.08
0.23 39.04
0.24 39
0.25 38.96
0.26 38.92
0.27 38.88
0.28 38.84
0.29 38.8
0.3 38.76
0.31 38.72
0.32 38.68
0.33 38.64
0.34 38.6
0.35 38.56
0.36 38.52
0.37 38.48
0.38 38.44
0.39 38.4
0.4 38.36
0.41 38.32
0.42 38.28
0.43 38.24
0.44 38.2
0.45 38.16
0.46 38.12
0.47 38.08
0.48 38.04
0.49 38
0.5 37.96
0.51 37.92
0.52 37.88
0.53 37.84
0.54 37.8
0.55 37.76
0.56 37.72
0.57 37.68
0.58 37.64
0.59 37.6
0.6 37.56
0.61 37.52
0.62 37.48
0.63 37.44
0.64 37.4
0.65 37.36
0.66 37.32
0.67 37.28
0.68 37.24
0.69 37.2
0.7 37.16
0.71 37.12
0.72 37.08
0.73 37.04
0.74 37
0.75 36.96
0.76 36.92
0.77 36.88
0.78 36.84
0.79 36.8
0.8 36.76
0.81 36.72
0.82 36.68
0.83 36.64
0.84 36.6
0.85 36.56
0.86 36.52
0.87 36.48
0.88 36.44
0.89 36.4
0.9 36.36
0.91 36.32
0.92 36.28
0.93 36.24
0.94 36.2
0.95 36.16
0.96 36.12
0.97 36.08
0.98 36.04
0.99 36
1 35.96
1.01 35.92
1.02 35.88
1.03 35.84
1.04 35.8
1.05 35.76
1.06 35.72
1.07 35.68
1.08 35.64
1.09 35.6
1.1 35.56
1.11 35.52
1.12 35.48
1.13 35.44
1.14 35.4
1.15 35.36
1.16 35.32
1.17 35.28
1.18 35.24
1.19 35.2
1.2 35.16
1.21 35.12
1.22 35.08
1.23 35.04
1.24 35
1.25 34.96
1.26 34.92
1.27 34.88
1.28 34.84
1.29 34.8
1.3 34.76
1.31 34.72
1.32 34.68
1.33 34.64
1.34 34.6
1.35 34.56
1.36 34.52
1.37 34.48
1.38 34.44
1.39 34.4
1.4 34.36
1.41 34.32
1.42 34.28
1.43 34.24
1.44 34.2
1.45 34.16
1.46 34.12
1.47 34.08
1.48 34.04
1.49 34
1.5 33.96
1.51 33.92
1.52 33.88
1.53 33.84
1.54 33.8
1.55 33.76
1.56 33.72
1.57 33.68
1.58 33.64
1.59 33.6
1.6 33.56
1.61 33.52
1.62 33.48
1.63 33.44
1.64 33.4
1.65 33.36
1.66 33.32
1.67 33.28
1.68 33.24
1.69 33.2
1.7 33.16
1.71 33.12
1.72 33.08
1.73 33.04
1.74 33
1.75 32.96
1.76 32.92
1.77 32.88
1.78 32.84
1.79 32.8
1.8 32.76
1.81 32.72
1.82 32.68
1.83 32.64
1.84 32.6
1.85 32.56
1.86 32.52
1.87 32.48
1.88 32.44
1.89 32.4
1.9 32.36
1.91 32.32
1.92 32.28
1.93 32.24
1.94 32.2
1.95 32.16
1.96 32.12
1.97 32.08
1.98 32.04
1.99 32
2 31.96
2.01 31.92
2.02 31.88
2.03 31.84
2.04 31.8
2.05 31.76
2.06 31.72
2.07 31.68
2.08 31.64
2.09 31.6
2.1 31.56
2.11 31.52
2.12 31.48
2.13 31.44
2.14 31.4
2.15 31.36
2.16 31.32
2.17 31.28
2.18 31.24
2.19 31.2
2.2 31.16
2.21 31.12
2.22 31.08
2.23 31.04
2.24 31
2.25 30.96
2.26 30.92
2.27 30.88
2.28 30.84
2.29 30.8
2.3 30.76
2.31 30.72
2.32 30.68
2.33 30.64
2.34 30.6
2.35 30.56
2.36 30.52
2.37 30.48
2.38 30.44
2.39 30.4
2.4 30.36
2.41 30.32
2.42 30.28
2.43 30.24
2.44 30.2
2.45 30.16
2.46 30.12
2.47 30.08
2.48 30.04
2.49 30
2.5 29.96
2.51 29.92
2.52 29.88
2.53 29.84
2.54 29.8
2.55 29.76
2.56 29.72
2.57 29.68
2.58 29.64
2.59 29.6
2.6 29.56
2.61 29.52
2.62 29.48
2.63 29.44
2.64 29.4
2.65 29.36
2.66 29.32
2.67 29.28
2.68 29.24
2.69 29.2
2.7 29.16
2.71 29.12
2.72 29.08
2.73 29.04
2.74 29
2.75 28.96
2.76 28.92
2.77 28.88
2.78 28.84
2.79 28.8
2.8 28.76
2.81 28.72
2.82 28.68
2.83 28.64
2.84 28.6
2.85 28.56
2.86 28.52
2.87 28.48
2.88 28.44
2.89 28.4
2.9 28.36
2.91 28.32
2.92 28.28
2.93 28.24
2.94 28.2
2.95 28.16
2.96 28.12
2.97 28.08
2.98 28.04
2.99 28
3 27.96
3.01 27.9199977793554
3.02 27.8799888969745
3.03 27.839968912357
3.04 27.799933385989
3.05 27.7598778797371
3.06 27.7197979572426
3.07 27.6796891843154
3.08 27.6395471293285
3.09 27.5993673636107
3.1 27.5591454618409
3.11 27.5188770024404
3.12 27.4785575679664
3.13 27.4381827455038
3.14 27.3977481270572
3.15 27.3572493099431
3.16 27.31668189718
3.17 27.2760414978797
3.18 27.2353237276372
3.19 27.1945242089197
3.2 27.1536385714562
3.21 27.1126624526248
3.22 27.0715914978411
3.23 27.0304213609443
3.24 26.9891477045836
3.25 26.9477662006035
3.26 26.906272530428
3.27 26.8646623854444
3.28 26.8229314673861
3.29 26.781075488714
3.3 26.7390901729979
3.31 26.6969712552958
3.32 26.6547144825331
3.33 26.6123156138804
3.34 26.5697704211299
3.35 26.5270746890713
3.36 26.4842242158659
3.37 26.4412148134202
3.38 26.3980423077576
3.39 26.3547025393893
3.4 26.3111913636837
3.41 26.2675046512347
3.42 26.2236382882282
3.43 26.1795881768079
3.44 26.1353502354389
3.45 26.0909203992707
3.46 26.0462946204977
3.47 26.0014688687194
3.48 25.9564391312977
3.49 25.9112014137139
3.5 25.8657517399234
3.51 25.8200861527084
3.52 25.7742007140301
3.53 25.7280915053782
3.54 25.6817546281187
3.55 25.6351862038406
3.56 25.5883823747002
3.57 25.541339303764
3.58 25.4940531753489
3.59 25.4465201953621
3.6 25.3987365916372
3.61 25.3506986142698
3.62 25.3024025359505
3.63 25.2538446522957
3.64 25.2050212821771
3.65 25.1559287680484
3.66 25.10656347627
3.67 25.0569217974324
3.68 25.0070001466759
3.69 24.95679496401
3.7 24.9063027146287
3.71 24.8555198892254
3.72 24.8044430043038
3.73 24.7530686024876
3.74 24.7013932528281
3.75 24.6494135511081
3.76 24.5971261201449
3.77 24.5445276100901
3.78 24.491614698727
3.79 24.4383840917661
3.8 24.3848325231371
3.81 24.3309567552798
3.82 24.2767535794309
3.83 24.2222198159095
3.84 24.1673523143995
3.85 24.1121479542291
3.86 24.0566036446484
3.87 24.0007163251039
3.88 23.94448296551
3.89 23.8879005665184
3.9 23.8309661597845
3.91 23.7736768082313
3.92 23.7160296063098
3.93 23.6580216802574
3.94 23.599650188353
3.95 23.5409123211698
3.96 23.4818053018243
3.97 23.4223263862234
3.98 23.3624728633079
3.99 23.3022420552933
4 23.2416313179079
4.01 23.1806405080159
4.02 23.1192719836971
4.03 23.057528136783
4.04 22.9954113927781
4.05 22.9329242107788
4.06 22.8700690833881
4.07 22.806848536629
4.08 22.7432651298528
4.09 22.6793214556464
4.1 22.6150201397352
4.11 22.5503638408838
4.12 22.485355250793
4.13 22.4199970939943
4.14 22.3542921277413
4.15 22.2882431418979
4.16 22.2218529588238
4.17 22.1551244332566
4.18 22.0880604521913
4.19 22.0206639347562
4.2 21.9529378320868
4.21 21.8848851271958
4.22 21.8165088348402
4.23 21.7478120013861
4.24 21.6787977046697
4.25 21.6094690538557
4.26 21.5398291892928
4.27 21.4698812823662
4.28 21.3996285353468
4.29 21.3290741812378
4.3 21.2582214836182
4.31 21.1870737364837
4.32 21.1156342640837
4.33 21.0439064207568
4.34 20.9718935907618
4.35 20.8995991881071
4.36 20.8270266563765
4.37 20.7541794685519
4.38 20.6810611268341
4.39 20.6076751624593
4.4 20.5340251355139
4.41 20.460114634746
4.42 20.3859472773736
4.43 20.3115267088909
4.44 20.2368566028706
4.45 20.1619406607646
4.46 20.0867826117008
4.47 20.0113862122775
4.48 19.9357552463552
4.49 19.8598935248454
4.5 19.7838048854965
4.51 19.7074931926771
4.52 19.6309623371566
4.53 19.5542162358828
4.54 19.4772588317567
4.55 19.4000940934052
4.56 19.3227260149505
4.57 19.2451586157767
4.58 19.1673959402943
4.59 19.0894420577014
4.6 19.0113010617428
4.61 18.9329770704659
4.62 18.8544742259747
4.63 18.7757966941802
4.64 18.6969486645488
4.65 18.6179343498486
4.66 18.5387579858918
4.67 18.4594238312759
4.68 18.3799361671214
4.69 18.3002992968076
4.7 18.2205175457054
4.71 18.1405952609079
4.72 18.0605368109588
4.73 17.9803465855774
4.74 17.9000289953822
4.75 17.8195884716116
4.76 17.7390294658421
4.77 17.6583564497043
4.78 17.5775739145965
4.79 17.496686371396
4.8 17.4156983501678
4.81 17.3346143998713
4.82 17.2534390880648
4.83 17.1721770006069
4.84 17.0908327413571
4.85 17.0094109318727
4.86 16.9279162111043
4.87 16.8463532350889
4.88 16.7647266766411
4.89 16.6830412250416
4.9 16.6013015857237
4.91 16.5195124799586
4.92 16.437678644537
4.93 16.35580483145
4.94 16.2738958075676
4.95 16.1919563543144
4.96 16.1099912673449
4.97 16.0280053562148
4.98 15.9460034440523
4.99 15.8639903672262
5 15.7819709750126
5.01 15.6999501292593
5.02 15.6179327040491
5.03 15.5359235853608
5.04 15.4539276707277
5.05 15.3719498688961
5.06 15.2899950994801
5.07 15.2080682926159
5.08 15.1261743886139
5.09 15.0443183376094
5.1 14.9625050992113
5.11 14.8807396421492
5.12 14.7990269439197
5.13 14.7173719904295
5.14 14.6357797756386
5.15 14.554255301201
5.16 14.4728035761042
5.17 14.3914296163074
5.18 14.3101384443774
5.19 14.2289350891246
5.2 14.1478245852362
5.21 14.0668119729084
5.22 13.9859022974779
5.23 13.9051006090515
5.24 13.8244119621343
5.25 13.7438414152569
5.26 13.6633940306017
5.27 13.5830748736273
5.28 13.5028890126923
5.29 13.4228415186775
5.3 13.3429374646077
5.31 13.2631819252713
5.32 13.1835799768401
5.33 13.1041366964874
5.34 13.0248571620047
5.35 12.9457464514187
5.36 12.8668096426058
5.37 12.7880518129072
5.38 12.7094780387419
5.39 12.6310933952198
5.4 12.5529029557532
5.41 12.4749117916682
5.42 12.3971249718147
5.43 12.3195475621763
5.44 12.2421846254794
5.45 12.1650412208009
5.46 12.0881224031764
5.47 12.0114332232069
5.48 11.9349787266655
5.49 11.8587639541032
5.5 11.7827939404547
5.51 11.7070737146428
5.52 11.6316082991837
5.53 11.5564027097903
5.54 11.4814619549768
5.55 11.4067910356617
5.56 11.3323949447709
5.57 11.2582786668407
5.58 11.1844471776204
5.59 11.1109054436747
5.6 11.0376584219859
5.61 10.964711059556
5.62 10.8920682930086
5.63 10.8197350481908
5.64 10.7477162397753
5.65 10.6760167708618
5.66 10.604641532579
5.67 10.5335954036867
5.68 10.4628832501777
5.69 10.3925099248795
5.7 10.3224802670573
5.71 10.2527991020158
5.72 10.1834712407019
5.73 10.1145014793082
5.74 10.0458945988754
5.75 9.97765536489645
5.76 9.90978852691991
5.77 9.84229881815459
5.78 9.77519095507401
5.79 9.70846963702158
5.8 9.64213954581617
5.81 9.57620534535826
5.82 9.51067168123663
5.83 9.44554318033559
5.84 9.38082445044288
5.85 9.31652007985822
5.86 9.25263463700253
5.87 9.18917267002786
5.88 9.12613870642813
5.89 9.06353725265059
5.9 9.00137279370816
5.91 8.9396497927926
5.92 8.87837269088853
5.93 8.81754590638848
5.94 8.7571738347088
5.95 8.69726084790658
5.96 8.63781129429764
5.97 8.57882949807551
5.98 8.52031975893157
5.99 8.46228635167628
6 8.40473352586152
6.01 8.34766550540424
6.02 8.29108648821125
6.03 8.23500064580523
6.04 8.17941212295218
6.05 8.12432503729002
6.06 8.06974347895867
6.07 8.01567151023148
6.08 7.9621131651481
6.09 7.90907244914873
6.1 7.85655333871001
6.11 7.80455978098232
6.12 7.75309569342858
6.13 7.70216496346487
6.14 7.65177144810239
6.15 7.60191897359127
6.16 7.55261133506603
6.17 7.50385229619263
6.18 7.45564558881751
6.19 7.40799491261814
6.2 7.36090393475564
6.21 7.31437628952904
6.22 7.26841557803154
6.23 7.22302536780868
6.24 7.17820919251838
6.25 7.13397055159302
6.26 7.09031290990347
6.27 7.04723969742527
6.28 7.00475430890669
6.29 6.96286010353909
6.3 6.92156040462928
6.31 6.88085849927407
6.32 6.84075763803707
6.33 6.80126103462761
6.34 6.76237087862485
6.35 6.72408611461405
6.36 6.68640540309719
6.37 6.64932736731541
6.38 6.612850593334
6.39 6.57697363012956
6.4 6.54169498967956
6.41 6.50701314705408
6.42 6.47292654050993
6.43 6.43943357158694
6.44 6.40653260520656
6.45 6.37422196977276
6.46 6.34249995727511
6.47 6.31136482339412
6.48 6.28081478760889
6.49 6.25084803330692
6.5 6.22146270789612
6.51 6.19265692291918
6.52 6.16442875416995
6.53 6.13677624181226
6.54 6.10969739050067
6.55 6.08319016950367
6.56 6.05725251282887
6.57 6.0318823193505
6.58 6.00707745293897
6.59 5.98283574259266
6.6 5.95915498257183
6.61 5.93603293253467
6.62 5.91346731767551
6.63 5.89145582886516
6.64 5.8699961227933
6.65 5.84908582211305
6.66 5.82872251558764
6.67 5.80890375823912
6.68 5.78962707149914
6.69 5.77088994336195
6.7 5.75268982853922
6.71 5.73502414861716
6.72 5.71789029221548
6.73 5.70128561514853
6.74 5.68520744058838
6.75 5.66965305922994
6.76 5.65461972945811
6.77 5.64010467751685
6.78 5.62610509768032
6.79 5.61261815242591
6.8 5.5996409726093
6.81 5.58717065764147
6.82 5.57520427566754
6.83 5.56373886374769
6.84 5.55277142803994
6.85 5.54229894398473
6.86 5.53231835649158
6.87 5.52282658012754
6.88 5.51382049930743
6.89 5.50529696848613
6.9 5.49725281235254
6.91 5.48968482602549
6.92 5.48258977525138
6.93 5.47596439660374
6.94 5.46980539768452
6.95 5.46410945732716
6.96 5.45887322580145
6.97 5.45409332502021
6.98 5.44976634874759
6.99 5.44588886280926
7 5.44245740530419
7.01 5.43946848681821
7.02 5.43691859063926
7.03 5.43480417297434
7.04 5.43312166316804
7.05 5.43186746392291
7.06 5.43103795152119
7.07 5.43062947604852
7.08 5.43063836161896
7.09 5.43106090660178
7.1 5.43189338384978
7.11 5.43313204092923
7.12 5.43477310035129
7.13 5.43681275980507
7.14 5.43924719239214
7.15 5.4420725468626
7.16 5.44528494785266
7.17 5.44888049612366
7.18 5.45285526880262
7.19 5.45720531962423
7.2 5.46192667917423
7.21 5.46701535513429
7.22 5.47246733252829
7.23 5.47827857396996
7.24 5.48444501991192
7.25 5.49096258889613
7.26 5.49782717780559
7.27 5.50503466211751
7.28 5.51258089615767
7.29 5.52046171335619
7.3 5.52867292650452
7.31 5.53721032801372
7.32 5.54606969017404
7.33 5.55524676541565
7.34 5.5647372865707
7.35 5.57453696713652
7.36 5.58464150154003
7.37 5.59504656540333
7.38 5.60574781581042
7.39 5.61674089157516
7.4 5.62802141351023
7.41 5.63958498469727
7.42 5.65142719075813
7.43 5.66354360012711
7.44 5.67592976432439
7.45 5.68858121823039
7.46 5.70149348036123
7.47 5.71466205314511
7.48 5.72808242319984
7.49 5.74175006161112
7.5 5.75566042421202
7.51 5.76980895186323
7.52 5.78419107073429
7.53 5.79880219258574
7.54 5.81363771505212
7.55 5.82869302192593
7.56 5.84396348344227
7.57 5.85944445656452
7.58 5.87513128527073
7.59 5.89101930084077
7.6 5.90710382214438
7.61 5.92338015592991
7.62 5.93984359711376
7.63 5.95648942907075
7.64 5.97331292392491
7.65 5.9903093428412
7.66 6.00747393631782
7.67 6.02480194447918
7.68 6.04228859736946
7.69 6.05992911524695
7.7 6.07771870887876
7.71 6.09565257983635
7.72 6.11372592079145
7.73 6.13193391581263
7.74 6.15027174066238
7.75 6.16873456309463
7.76 6.18731754315293
7.77 6.20601583346889
7.78 6.22482457956127
7.79 6.24373892013539
7.8 6.262753987383
7.81 6.28186490728256
7.82 6.30106679989989
7.83 6.32035477968916
7.84 6.33972395579433
7.85 6.35916943235072
7.86 6.37868630878713
7.87 6.39826968012807
7.88 6.41791463729631
7.89 6.43761626741575
7.9 6.4573696541144
7.91 6.4771698778277
7.92 6.49701201610196
7.93 6.51689114389801
7.94 6.53680233389496
7.95 6.55674065679422
7.96 6.57670118162348
7.97 6.59667897604091
7.98 6.61666910663945
7.99 6.63666663925103
8 6.65666663925104
8.01 6.67666633082748
8.02 6.6966650971523
8.03 6.71666232145448
8.04 6.73665738705814
8.05 6.75664967742054
8.06 6.77663857617016
8.07 6.79662346714467
8.08 6.81660373442895
8.09 6.83657876239312
8.1 6.85654793573045
8.11 6.87651063949538
8.12 6.89646625914141
8.13 6.91641418055902
8.14 6.93635379011358
8.15 6.95628447468312
8.16 6.97620562169627
8.17 6.99611661916993
8.18 7.01601685574709
8.19 7.03590572073454
8.2 7.05578260414049
8.21 7.07564689671226
8.22 7.09549798997381
8.23 7.1153352762633
8.24 7.13515814877059
8.25 7.15496600157462
8.26 7.17475822968084
8.27 7.19453422905848
8.28 7.21429339667787
8.29 7.23403513054757
8.3 7.25375882975155
8.31 7.27346389448623
8.32 7.29314972609752
8.33 7.31281572711769
8.34 7.33246130130227
8.35 7.3520858536668
8.36 7.37168879052357
8.37 7.3912695195182
8.38 7.41082744966618
8.39 7.43036199138938
8.4 7.44987255655232
8.41 7.46935855849858
8.42 7.48881941208685
8.43 7.50825453372713
8.44 7.52766334141668
8.45 7.5470452547759
8.46 7.5663996950842
8.47 7.58572608531563
8.48 7.60502385017452
8.49 7.62429241613091
8.5 7.64353121145603
8.51 7.66273966625744
8.52 7.68191721251428
8.53 7.70106328411227
8.54 7.72017731687862
8.55 7.73925874861687
8.56 7.75830701914153
8.57 7.77732157031265
8.58 7.79630184607026
8.59 7.81524729246864
8.6 7.83415735771052
8.61 7.85303149218112
8.62 7.87186914848201
8.63 7.89066978146491
8.64 7.90943284826535
8.65 7.92815780833608
8.66 7.94684412348046
8.67 7.96549125788567
8.68 7.9840986781557
8.69 8.00266585334435
8.7 8.02119225498789
8.71 8.03967735713771
8.72 8.05812063639273
8.73 8.07652157193172
8.74 8.0948796455454
8.75 8.11319434166843
8.76 8.13146514741117
8.77 8.14969155259138
8.78 8.16787304976563
8.79 8.18600913426064
8.8 8.20409930420439
8.81 8.22214306055709
8.82 8.24013990714196
8.83 8.25808935067583
8.84 8.27599090079959
8.85 8.29384407010839
8.86 8.31164837418178
8.87 8.3294033316135
8.88 8.34710846404126
8.89 8.36476329617619
8.9 8.38236735583219
8.91 8.39992017395504
8.92 8.41742128465134
8.93 8.43487022521726
8.94 8.45226653616704
8.95 8.4696097612614
8.96 8.48689944753562
8.97 8.5041351453275
8.98 8.52131640830513
8.99 8.53844279349439
9 8.55551386130625
9.01 8.57252917556396
9.02 8.58948830352989
9.03 8.60639081593223
9.04 8.62323628699152
9.05 8.64002429444685
9.06 8.65675441958194
9.07 8.67342624725099
9.08 8.69003936590423
9.09 8.70659336761334
9.1 8.72308784809665
9.11 8.73952240674398
9.12 8.75589664664147
9.13 8.77221017459597
9.14 8.78846260115932
9.15 8.80465354065242
9.16 8.82078261118895
9.17 8.83684943469897
9.18 8.85285363695223
9.19 8.86879484758125
9.2 8.88467270010417
9.21 8.90048683194737
9.22 8.91623688446781
9.23 8.93192250297515
9.24 8.94754333675367
9.25 8.96309903908387
9.26 8.97858926726385
9.27 8.99401368263048
9.28 9.00937195058027
9.29 9.02466374059002
9.3 9.03988872623718
9.31 9.05504658522003
9.32 9.07013699937753
9.33 9.08515965470897
9.34 9.10011424139329
9.35 9.11500045380826
9.36 9.12981799054928
9.37 9.14456655444798
9.38 9.15924585259059
9.39 9.17385559633594
9.4 9.18839550133334
9.41 9.20286528754005
9.42 9.21726467923861
9.43 9.23159340505381
9.44 9.24585119796946
9.45 9.26003779534483
9.46 9.27415293893088
9.47 9.28819637488617
9.48 9.30216785379251
9.49 9.3160671306704
9.5 9.32989396499405
9.51 9.3436481207063
9.52 9.35732936623315
9.53 9.37093747449802
9.54 9.38447222293581
9.55 9.39793339350659
9.56 9.41132077270905
9.57 9.42463415159367
9.58 9.43787332577566
9.59 9.45103809544747
9.6 9.46412826539122
9.61 9.47714364499067
9.62 9.49008404824299
9.63 9.50294929377026
9.64 9.51573920483066
9.65 9.52845360932931
9.66 9.54109233982896
9.67 9.55365523356029
9.68 9.56614213243193
9.69 9.57855288304026
9.7 9.59088733667883
9.71 9.60314534934751
9.72 9.61532678176148
9.73 9.62743149935969
9.74 9.63945937231325
9.75 9.65141027553342
9.76 9.66328408867927
9.77 9.6750806961652
9.78 9.686799987168
9.79 9.69844185563369
9.8 9.71000620028409
9.81 9.72149292462306
9.82 9.73290193694244
9.83 9.74423315032771
9.84 9.75548648266334
9.85 9.76666185663792
9.86 9.77775919974883
9.87 9.78877844430679
9.88 9.79971952743997
9.89 9.81058239109795
9.9 9.82136698205523
9.91 9.83207325191454
9.92 9.84270115710983
9.93 9.85325065890896
9.94 9.86372172341605
9.95 9.87411432157364
9.96 9.88442842916442
9.97 9.89466402681276
9.98 9.90482109998587
9.99 9.91489963899476
10 9.92489963899476
10.01 9.93482109998588
10.02 9.94466402681276
10.03 9.95442842916442
10.04 9.96411432157364
10.05 9.97372172341605
10.06 9.98325065890895
10.07 9.99270115710983
10.08 10.0020732519145
10.09 10.0113669820552
10.1 10.0205823910979
10.11 10.02971952744
10.12 10.0387784443068
10.13 10.0477591997488
10.14 10.0566618566379
10.15 10.0654864826633
10.16 10.0742331503277
10.17 10.0829019369424
10.18 10.0914929246231
10.19 10.1000062002841
10.2 10.1084418556337
10.21 10.116799987168
10.22 10.1250806961652
10.23 10.1332840886793
10.24 10.1414102755334
10.25 10.1494593723133
10.26 10.1574314993597
10.27 10.1653267817615
10.28 10.1731453493475
10.29 10.1808873366788
10.3 10.1885528830403
10.31 10.1961421324319
10.32 10.2036552335603
10.33 10.211092339829
10.34 10.2184536093293
10.35 10.2257392048307
10.36 10.2329492937703
10.37 10.240084048243
10.38 10.2471436449907
10.39 10.2541282653912
10.4 10.2610380954475
10.41 10.2678733257757
10.42 10.2746341515937
10.43 10.281320772709
10.44 10.2879333935066
10.45 10.2944722229358
10.46 10.300937474498
10.47 10.3073293662331
10.48 10.3136481207063
10.49 10.319893964994
10.5 10.3260671306704
10.51 10.3321678537925
10.52 10.3381963748862
10.53 10.3441529389309
10.54 10.3500377953448
10.55 10.3558511979695
10.56 10.3615934050538
10.57 10.3672646792386
10.58 10.37286528754
10.59 10.3783955013333
10.6 10.3838555963359
10.61 10.3892458525906
10.62 10.394566554448
10.63 10.3998179905493
10.64 10.4050004538083
10.65 10.4101142413933
10.66 10.415159654709
10.67 10.4201369993775
10.68 10.42504658522
10.69 10.4298887262372
10.7 10.43466374059
10.71 10.4393719505803
10.72 10.4440136826305
10.73 10.4485892672639
10.74 10.4530990390839
10.75 10.4575433367537
10.76 10.4619225029752
10.77 10.4662368844678
10.78 10.4704868319474
10.79 10.4746727001042
10.8 10.4787948475813
10.81 10.4828536369522
10.82 10.486849434699
10.83 10.490782611189
10.84 10.4946535406524
10.85 10.4984626011593
10.86 10.502210174596
10.87 10.5058966466415
10.88 10.509522406744
10.89 10.5130878480967
10.9 10.5165933676133
10.91 10.5200393659042
10.92 10.523426247251
10.93 10.5267544195819
10.94 10.5300242944469
10.95 10.5332362869915
10.96 10.5363908159322
10.97 10.5394883035299
10.98 10.542529175564
10.99 10.5455138613063
11 10.5484427934944
11.01 10.5513164083051
11.02 10.5541351453275
11.03 10.5568994475356
11.04 10.5596097612614
11.05 10.562266536167
11.06 10.5648702252173
11.07 10.5674212846513
11.08 10.569920173955
11.09 10.5723673558322
11.1 10.5747632961762
11.11 10.5771084640413
11.12 10.5794033316135
11.13 10.5816483741818
11.14 10.5838440701084
11.15 10.5859909007996
11.16 10.5880893506758
11.17 10.590139907142
11.18 10.5921430605571
11.19 10.5940993042044
11.2 10.5960091342606
11.21 10.5978730497656
11.22 10.5996915525914
11.23 10.6014651474112
11.24 10.6031943416684
11.25 10.6048796455454
11.26 10.6065215719317
11.27 10.6081206363927
11.28 10.6096773571377
11.29 10.6111922549879
11.3 10.6126658533444
11.31 10.6140986781557
11.32 10.6154912578857
11.33 10.6168441234805
11.34 10.6181578083361
11.35 10.6194328482654
11.36 10.6206697814649
11.37 10.621869148482
11.38 10.6230314921811
11.39 10.6241573577105
11.4 10.6252472924686
11.41 10.6263018460703
11.42 10.6273215703127
11.43 10.6283070191415
11.44 10.6292587486169
11.45 10.6301773168786
11.46 10.6310632841123
11.47 10.6319172125143
11.48 10.6327396662574
11.49 10.633531211456
11.5 10.6342924161309
11.51 10.6350238501745
11.52 10.6357260853156
11.53 10.6363996950842
11.54 10.6370452547759
11.55 10.6376633414167
11.56 10.6382545337271
11.57 10.6388194120869
11.58 10.6393585584986
11.59 10.6398725565523
11.6 10.6403619913894
11.61 10.6408274496662
11.62 10.6412695195182
11.63 10.6416887905236
11.64 10.6420858536668
11.65 10.6424613013023
11.66 10.6428157271177
11.67 10.6431497260975
11.68 10.6434638944862
11.69 10.6437588297516
11.7 10.6440351305476
11.71 10.6442933966779
11.72 10.6445342290585
11.73 10.6447582296808
11.74 10.6449660015746
11.75 10.6451581487706
11.76 10.6453352762633
11.77 10.6454979899738
11.78 10.6456468967123
11.79 10.6457826041405
11.8 10.6459057207345
11.81 10.6460168557471
11.82 10.6461166191699
11.83 10.6462056216963
11.84 10.6462844746831
11.85 10.6463537901136
11.86 10.646414180559
11.87 10.6464662591414
11.88 10.6465106394954
11.89 10.6465479357304
11.9 10.6465787623931
11.91 10.646603734429
11.92 10.6466234671447
11.93 10.6466385761702
11.94 10.6466496774205
11.95 10.6466573870581
11.96 10.6466623214545
11.97 10.6466650971523
11.98 10.6466663308275
11.99 10.646666639251
12 10.646666639251
};
\end{axis}

\end{tikzpicture}

	% This file was created by tikzplotlib v0.9.8.
\begin{tikzpicture}

\begin{axis}[
height=\figureheight,
scaled y ticks=false,
tick align=outside,
tick pos=left,
width=\figurewidth,
x grid style={white!69.0196078431373!black},
xlabel={Time [\si{\second}]},
xmajorgrids,
xmin=0, xmax=12,
xtick style={color=black},
xticklabel style={align=center},
y grid style={white!69.0196078431373!black},
ylabel={Probability of collision},
ymajorgrids,
ymin=0, ymax=1,
ytick style={color=black},
yticklabel style={/pgf/number format/fixed,/pgf/number format/precision=3}
]
\addplot [very thick, gray]
table {%
0 4.44089209850063e-16
0.01 4.44089209850063e-16
0.02 4.44089209850063e-16
0.03 4.44089209850063e-16
0.04 4.44089209850063e-16
0.05 4.44089209850063e-16
0.06 4.44089209850063e-16
0.07 4.44089209850063e-16
0.08 4.44089209850063e-16
0.09 4.44089209850063e-16
0.1 4.44089209850063e-16
0.11 4.44089209850063e-16
0.12 4.44089209850063e-16
0.13 4.44089209850063e-16
0.14 4.44089209850063e-16
0.15 4.44089209850063e-16
0.16 4.44089209850063e-16
0.17 4.44089209850063e-16
0.18 4.44089209850063e-16
0.19 4.44089209850063e-16
0.2 4.44089209850063e-16
0.21 4.44089209850063e-16
0.22 4.44089209850063e-16
0.23 5.55111512312578e-16
0.24 5.55111512312578e-16
0.25 5.55111512312578e-16
0.26 5.55111512312578e-16
0.27 7.7715611723761e-16
0.28 7.7715611723761e-16
0.29 7.7715611723761e-16
0.3 7.7715611723761e-16
0.31 7.7715611723761e-16
0.32 7.7715611723761e-16
0.33 7.7715611723761e-16
0.34 7.7715611723761e-16
0.35 7.7715611723761e-16
0.36 7.7715611723761e-16
0.37 8.88178419700125e-16
0.38 8.88178419700125e-16
0.39 8.88178419700125e-16
0.4 9.99200722162641e-16
0.41 9.99200722162641e-16
0.42 9.99200722162641e-16
0.43 9.99200722162641e-16
0.44 1.22124532708767e-15
0.45 1.22124532708767e-15
0.46 1.22124532708767e-15
0.47 1.22124532708767e-15
0.48 1.22124532708767e-15
0.49 1.33226762955019e-15
0.5 1.33226762955019e-15
0.51 1.33226762955019e-15
0.52 1.33226762955019e-15
0.53 1.4432899320127e-15
0.54 1.4432899320127e-15
0.55 1.4432899320127e-15
0.56 1.66533453693773e-15
0.57 1.66533453693773e-15
0.58 1.66533453693773e-15
0.59 1.66533453693773e-15
0.6 1.66533453693773e-15
0.61 1.88737914186277e-15
0.62 1.88737914186277e-15
0.63 1.88737914186277e-15
0.64 1.88737914186277e-15
0.65 1.88737914186277e-15
0.66 2.22044604925031e-15
0.67 2.22044604925031e-15
0.68 2.22044604925031e-15
0.69 2.22044604925031e-15
0.7 2.22044604925031e-15
0.71 2.55351295663786e-15
0.72 2.55351295663786e-15
0.73 2.66453525910038e-15
0.74 2.88657986402541e-15
0.75 2.88657986402541e-15
0.76 2.99760216648792e-15
0.77 2.99760216648792e-15
0.78 3.10862446895044e-15
0.79 3.10862446895044e-15
0.8 3.33066907387547e-15
0.81 3.33066907387547e-15
0.82 3.44169137633799e-15
0.83 3.5527136788005e-15
0.84 3.77475828372553e-15
0.85 3.88578058618805e-15
0.86 3.99680288865056e-15
0.87 3.99680288865056e-15
0.88 4.32986979603811e-15
0.89 4.32986979603811e-15
0.9 4.32986979603811e-15
0.91 4.66293670342566e-15
0.92 4.66293670342566e-15
0.93 4.9960036108132e-15
0.94 4.9960036108132e-15
0.95 5.10702591327572e-15
0.96 5.21804821573824e-15
0.97 5.55111512312578e-15
0.98 5.6621374255883e-15
0.99 5.88418203051333e-15
1 6.10622663543836e-15
1.01 6.10622663543836e-15
1.02 6.32827124036339e-15
1.03 6.43929354282591e-15
1.04 6.77236045021345e-15
1.05 6.99440505513849e-15
1.06 7.21644966006352e-15
1.07 7.32747196252603e-15
1.08 7.66053886991358e-15
1.09 7.99360577730113e-15
1.1 8.21565038222616e-15
1.11 8.43769498715119e-15
1.12 8.65973959207622e-15
1.13 8.99280649946377e-15
1.14 9.32587340685131e-15
1.15 9.43689570931383e-15
1.16 9.76996261670138e-15
1.17 1.02140518265514e-14
1.18 1.0547118733939e-14
1.19 1.0769163338864e-14
1.2 1.11022302462516e-14
1.21 1.14352971536391e-14
1.22 1.16573417585641e-14
1.23 1.21014309684142e-14
1.24 1.24344978758018e-14
1.25 1.28785870856518e-14
1.26 1.33226762955019e-14
1.27 1.37667655053519e-14
1.28 1.4210854715202e-14
1.29 1.45439216225896e-14
1.3 1.50990331349021e-14
1.31 1.56541446472147e-14
1.32 1.60982338570648e-14
1.33 1.66533453693773e-14
1.34 1.70974345792274e-14
1.35 1.75415237890775e-14
1.36 1.80966353013901e-14
1.37 1.87627691161651e-14
1.38 1.93178806284777e-14
1.39 1.99840144432528e-14
1.4 2.05391259555654e-14
1.41 2.12052597703405e-14
1.42 2.18713935851156e-14
1.43 2.25375273998907e-14
1.44 2.32036612146658e-14
1.45 2.39808173319034e-14
1.46 2.4757973449141e-14
1.47 2.56461518688411e-14
1.48 2.64233079860787e-14
1.49 2.73114864057789e-14
1.5 2.80886425230165e-14
1.51 2.89768209427166e-14
1.52 2.98649993624167e-14
1.53 3.08642000845794e-14
1.54 3.19744231092045e-14
1.55 3.28626015289046e-14
1.56 3.38618022510673e-14
1.57 3.49720252756924e-14
1.58 3.61932706027801e-14
1.59 3.74145159298678e-14
1.6 3.85247389544929e-14
1.61 3.96349619791181e-14
1.62 4.09672296086683e-14
1.63 4.21884749357559e-14
1.64 4.35207425653061e-14
1.65 4.50750547997814e-14
1.66 4.65183447317941e-14
1.67 4.78506123613442e-14
1.68 4.9515946898282e-14
1.69 5.11812814352197e-14
1.7 5.26245713672324e-14
1.71 5.42899059041702e-14
1.72 5.61772850460329e-14
1.73 5.77315972805081e-14
1.74 5.97299987248334e-14
1.75 6.18394224716212e-14
1.76 6.38378239159465e-14
1.77 6.57252030578093e-14
1.78 6.79456491070596e-14
1.79 7.00550728538474e-14
1.8 7.24975635080227e-14
1.81 7.46069872548105e-14
1.82 7.71605002114484e-14
1.83 7.97140131680862e-14
1.84 8.20454815197991e-14
1.85 8.49320613838245e-14
1.86 8.75965966429249e-14
1.87 9.04831765069503e-14
1.88 9.34807786734382e-14
1.89 9.64783808399261e-14
1.9 9.96980276113391e-14
1.91 1.02917674382752e-13
1.92 1.06248343456627e-13
1.93 1.09690034832965e-13
1.94 1.13353770814228e-13
1.95 1.16906484493029e-13
1.96 1.21014309684142e-13
1.97 1.2490009027033e-13
1.98 1.29118937763906e-13
1.99 1.33226762955019e-13
2 1.37889699658444e-13
2.01 1.4210854715202e-13
2.02 1.46993528460371e-13
2.03 1.51767487466259e-13
2.04 1.56874513379535e-13
2.05 1.6198153929281e-13
2.06 1.67199587508549e-13
2.07 1.72750702631674e-13
2.08 1.783018177548e-13
2.09 1.84297022087776e-13
2.1 1.90514271025677e-13
2.11 1.96620497661115e-13
2.12 2.03059791203941e-13
2.13 2.09832151654155e-13
2.14 2.16937579011756e-13
2.15 2.24043006369357e-13
2.16 2.31481500634345e-13
2.17 2.39142039504259e-13
2.18 2.47246667584022e-13
2.19 2.55351295663786e-13
2.2 2.639000129534e-13
2.21 2.72559752545476e-13
2.22 2.81774603649865e-13
2.23 2.90989454754254e-13
2.24 3.00648395068492e-13
2.25 3.10862446895044e-13
2.26 3.21187521024058e-13
2.27 3.31845662060459e-13
2.28 3.42947892306711e-13
2.29 3.54494211762812e-13
2.3 3.66484620428764e-13
2.31 3.78697073699641e-13
2.32 3.9146463848283e-13
2.33 4.0467629247587e-13
2.34 4.18221013376296e-13
2.35 4.32320845789036e-13
2.36 4.466427228067e-13
2.37 4.61963800546528e-13
2.38 4.7750692289128e-13
2.39 4.93494134445882e-13
2.4 5.10369524420184e-13
2.41 5.27466958999412e-13
2.42 5.45341549695877e-13
2.43 5.63771251904654e-13
2.44 5.8297811023067e-13
2.45 6.02518035464072e-13
2.46 6.22946139117175e-13
2.47 6.44262421189978e-13
2.48 6.66466881682481e-13
2.49 6.89004409082372e-13
2.5 7.123190925995e-13
2.51 7.36855021443716e-13
2.52 7.61946061800245e-13
2.53 7.87925280576474e-13
2.54 8.15014722377327e-13
2.55 8.42659275690494e-13
2.56 8.71525074330748e-13
2.57 9.01390073693165e-13
2.58 9.32143251475281e-13
2.59 9.64117674584486e-13
2.6 9.97313343020778e-13
2.61 1.03128616757431e-12
2.62 1.06681330436231e-12
2.63 1.10333964187248e-12
2.64 1.14142029161712e-12
2.65 1.18072218668885e-12
2.66 1.22135634939013e-12
2.67 1.26365584662835e-12
2.68 1.3070655668912e-12
2.69 1.35258471090083e-12
2.7 1.39921407793508e-12
2.71 1.4477308241112e-12
2.72 1.49769086021934e-12
2.73 1.54964929777179e-12
2.74 1.60338409216365e-12
2.75 1.65889524339491e-12
2.76 1.71684888528034e-12
2.77 1.77624581709779e-12
2.78 1.83808523956941e-12
2.79 1.90192306348536e-12
2.8 1.96831440035794e-12
2.81 2.03703720558224e-12
2.82 2.10798045685578e-12
2.83 2.1816992656909e-12
2.84 2.25808260978511e-12
2.85 2.33713048913842e-12
2.86 2.41873188144837e-12
2.87 2.50344189822727e-12
2.88 2.59137156177758e-12
2.89 2.68218780519192e-12
2.9 2.77644573998259e-12
2.91 2.87403434384714e-12
2.92 2.97517566139049e-12
2.93 3.07998071491511e-12
2.94 3.1881164375136e-12
2.95 3.30069305221059e-12
2.96 3.41726646979623e-12
2.97 3.53794771257299e-12
2.98 3.66295882514578e-12
2.99 3.79218878521215e-12
3 3.92696986040164e-12
3.01 4.07152089820784e-12
3.02 4.23328039289572e-12
3.03 4.41469083511947e-12
3.04 4.61619631408894e-12
3.05 4.84079443197061e-12
3.06 5.09037256790634e-12
3.07 5.36792832406263e-12
3.08 5.67623725800104e-12
3.09 6.01907412800529e-12
3.1 6.40032471466156e-12
3.11 6.82409684316099e-12
3.12 7.29571958402175e-12
3.13 7.82163223078669e-12
3.14 8.40749692088139e-12
3.15 9.06164032699053e-12
3.16 9.79250014410127e-12
3.17 1.06099573571328e-11
3.18 1.15255582855411e-11
3.19 1.25522925387145e-11
3.2 1.37058142613e-11
3.21 1.50026657763647e-11
3.22 1.64631641652591e-11
3.23 1.81108461561053e-11
3.24 1.99711358561672e-11
3.25 2.20753415547392e-11
3.26 2.44594344778193e-11
3.27 2.71638267435037e-11
3.28 3.02367020310612e-11
3.29 3.37332384248157e-11
3.3 3.77178288601954e-11
3.31 4.226530236906e-11
3.32 4.74631445257501e-11
3.33 5.34124966478089e-11
3.34 6.02322636211738e-11
3.35 6.80611123016206e-11
3.36 7.70604691169297e-11
3.37 8.74205152712193e-11
3.38 9.93629623025072e-11
3.39 1.13148157510068e-10
3.4 1.29081856314883e-10
3.41 1.47522438709302e-10
3.42 1.68891789442682e-10
3.43 1.93686511273938e-10
3.44 2.22490248447116e-10
3.45 2.5599078412597e-10
3.46 2.95000579519922e-10
3.47 3.40475980742383e-10
3.48 3.93547305854725e-10
3.49 4.55550153155571e-10
3.5 5.28062815696728e-10
3.51 6.12952133494105e-10
3.52 7.12428005478216e-10
3.53 8.29106450161987e-10
3.54 9.66087432274776e-10
3.55 1.12704412469355e-09
3.56 1.31633393074537e-09
3.57 1.53912393940914e-09
3.58 1.80154768880669e-09
3.59 2.11088657664504e-09
3.6 2.4757883521076e-09
3.61 2.9065252427074e-09
3.62 3.41530459468942e-09
3.63 4.01663546867326e-09
3.64 4.72776828797095e-09
3.65 5.56921497807394e-09
3.66 6.56537091359155e-09
3.67 7.74525521496372e-09
3.68 9.14338904589584e-09
3.69 1.08008456622954e-08
3.7 1.27664956384166e-08
3.71 1.5098489458687e-08
3.72 1.78660226612948e-08
3.73 2.11514317172146e-08
3.74 2.50526887013436e-08
3.75 2.96863625903754e-08
3.76 3.51911377816094e-08
3.77 4.17319971912633e-08
3.78 4.95051811766345e-08
3.79 5.87440757149338e-08
3.8 6.97261978155339e-08
3.81 8.27814764514656e-08
3.82 9.83020659317546e-08
3.83 1.16753973489203e-07
3.84 1.38690823048293e-07
3.85 1.64770147192961e-07
3.86 1.95772653088788e-07
3.87 2.32625001039821e-07
3.88 2.76426717182865e-07
3.89 3.28481965627958e-07
3.9 3.90337049127609e-07
3.91 4.63824631147247e-07
3.92 5.5111585328671e-07
3.93 6.54781722175812e-07
3.94 7.77865358347896e-07
3.95 9.23966973931378e-07
3.96 1.09734374775794e-06
3.97 1.30302712131325e-06
3.98 1.54696045440161e-06
3.99 1.83616044258095e-06
4 2.1789062505384e-06
4.01 2.5829004660638e-06
4.02 3.05616027829192e-06
4.03 3.60956097178455e-06
4.04 4.2555198398464e-06
4.05 5.0081893924192e-06
4.06 5.883671143736e-06
4.07 6.90025170024811e-06
4.08 8.07866294261217e-06
4.09 9.44236818134492e-06
4.1 1.10178762287028e-05
4.11 1.28350854139425e-05
4.12 1.49276596113079e-05
4.13 1.7333438416145e-05
4.14 2.00948836313053e-05
4.15 2.32595642551958e-05
4.16 2.68806821861478e-05
4.17 3.10176408225837e-05
4.18 3.57366587453445e-05
4.19 4.1111430594043e-05
4.2 4.72238371914679e-05
4.21 5.41647068675877e-05
4.22 6.20346298042529e-05
4.23 7.09448270697077e-05
4.24 8.10180758109569e-05
4.25 9.23896918545442e-05
4.26 0.000105208570694737
4.27 0.000119638287540025
4.28 0.000135858256759769
4.29 0.000154064950656507
4.3 0.00017447317708863
4.31 0.000197317414968312
4.32 0.000222853206145457
4.33 0.000251358601628837
4.34 0.000283135659473799
4.35 0.000318511991022841
4.36 0.000357842351482063
4.37 0.000401510270092942
4.38 0.000449929714378516
4.39 0.000503546782152031
4.4 0.000562841414139981
4.41 0.000628329119226612
4.42 0.000700562703452423
4.43 0.000780133993034116
4.44 0.000867675540781376
4.45 0.000963862304426577
4.46 0.00106941328452415
4.47 0.00118509310874249
4.48 0.00131171354857984
4.49 0.00145013495379187
4.5 0.0016012675891236
4.51 0.00176607285731922
4.52 0.00194556439184723
4.53 0.00214080900233582
4.54 0.00235292745536364
4.55 0.00258309507304544
4.56 0.00283254213173845
4.57 0.00310255404325921
4.58 0.00339447130117987
4.59 0.00370968917513514
4.6 0.00404965713658056
4.61 0.00441587800014298
4.62 0.00480990676556181
4.63 0.00523334914627516
4.64 0.00568785977194264
4.65 0.00617514005358943
4.66 0.00669693570168151
4.67 0.00725503388917981
4.68 0.00785126005358772
4.69 0.00848747433408037
4.7 0.00916556764206033
4.71 0.00988745736586549
4.72 0.0106550827128441
4.73 0.0114703996946089
4.74 0.012335375763962
4.75 0.0132519841147132
4.76 0.0142221976583699
4.77 0.0152479826944675
4.78 0.0163312922940431
4.79 0.0174740594184802
4.8 0.0186781897985745
4.81 0.0199455546012081
4.82 0.0212779829134193
4.83 0.0226772540759033
4.84 0.024145089900025
4.85 0.025683146804298
4.86 0.0272930079078776
4.87 0.0289761751199916
4.88 0.0307340612653179
4.89 0.0325679822861064
4.9 0.0344791495623454
4.91 0.0364686623914376
4.92 0.0385375006687052
4.93 0.0406865178095693
4.94 0.0429164339534408
4.95 0.0452278294882306
4.96 0.0476211389329382
4.97 0.0500966452140202
4.98 0.0526544743691878
4.99 0.0552945907099499
5 0.0580167924716313
5.01 0.0608207079767704
5.02 0.0637057923347547
5.03 0.066671324697358
5.04 0.0697164060864325
5.05 0.072839957806543
5.06 0.0760407204517319
5.07 0.0793172535119391
5.08 0.0826679355809529
5.09 0.0860909651640722
5.1 0.0895843620800477
5.11 0.0931459694482947
5.12 0.0967734562489213
5.13 0.100464320439768
5.14 0.104215892611489
5.15 0.108025340158703
5.16 0.111889671942453
5.17 0.115805743416618
5.18 0.119770262188658
5.19 0.123779793982896
5.2 0.127830768972844
5.21 0.131919488447478
5.22 0.136042131775158
5.23 0.140194763627928
5.24 0.144373341428261
5.25 0.148573722979944
5.26 0.152791674244704
5.27 0.157022877226352
5.28 0.161262937924691
5.29 0.165507394322133
5.3 0.169751724366941
5.31 0.173991353918199
5.32 0.178221664619033
5.33 0.182438001666214
5.34 0.186635681446083
5.35 0.19080999900869
5.36 0.194956235354198
5.37 0.199069664507799
5.38 0.203145560361811
5.39 0.207179203266033
5.4 0.211165886350042
5.41 0.215100921563626
5.42 0.218979645424298
5.43 0.222797424463419
5.44 0.226549660365184
5.45 0.230231794795456
5.46 0.233839313920009
5.47 0.237367752614484
5.48 0.240812698370936
5.49 0.244169794908437
5.5 0.247434745497688
5.51 0.25060331601211
5.52 0.253671337720216
5.53 0.256634709836438
5.54 0.259489401849794
5.55 0.262231455651979
5.56 0.264856987488534
5.57 0.267362189758759
5.58 0.269743332691946
5.59 0.271996765929361
5.6 0.27411892004315
5.61 0.276106308024957
5.62 0.277955526778669
5.63 0.27966325865314
5.64 0.281226273052106
5.65 0.282641428159811
5.66 0.283905672821974
5.67 0.285016048622822
5.68 0.28596969219979
5.69 0.286763837838283
5.7 0.287395820389519
5.71 0.287863078554898
5.72 0.288163158580633
5.73 0.288293718406399
5.74 0.288252532311518
5.75 0.28803749610177
5.76 0.287646632879053
5.77 0.287078099434975
5.78 0.286330193307899
5.79 0.285401360540897
5.8 0.284290204175574
5.81 0.282995493513536
5.82 0.281516174173609
5.83 0.279851378968324
5.84 0.278000439618028
5.85 0.275962899314802
5.86 0.273738526141268
5.87 0.271327327341345
5.88 0.268729564430703
5.89 0.26594576912435
5.9 0.262976760047023
5.91 0.259823660179123
5.92 0.256487914976444
5.93 0.252971311086098
5.94 0.249275995563704
5.95 0.245404495477941
5.96 0.241359737768304
5.97 0.237145069199983
5.98 0.232764276236589
5.99 0.228221604626942
6 0.223521778476522
6.01 0.218670018547647
6.02 0.213672059505297
6.03 0.208534165798074
6.04 0.203263145836437
6.05 0.197866364103743
6.06 0.192351750810082
6.07 0.186727808675377
6.08 0.181003616407212
6.09 0.175188828421563
6.1 0.169293670341529
6.11 0.163328929801908
6.12 0.157305942086607
6.13 0.151236570133068
6.14 0.145133178454126
6.15 0.139008600554386
6.16 0.132876099456574
6.17 0.126749321004695
6.18 0.120642239676328
6.19 0.114569096717139
6.2 0.108544330507501
6.21 0.10258249918467
6.22 0.0966981956744979
6.23 0.0909059554340359
6.24 0.0852201573700694
6.25 0.0796549185772172
6.26 0.0742239837310331
6.27 0.0689406101736778
6.28 0.0638174499388042
6.29 0.0588664301737338
6.3 0.0540986336254947
6.31 0.049524181056309
6.32 0.0451521176364074
6.33 0.0409903055192761
6.34 0.0370545001496383
6.35 0.0333754835525986
6.36 0.0299507010216375
6.37 0.026774189653558
6.38 0.0238390883597051
6.39 0.0211376994361735
6.4 0.0186615586518468
6.41 0.0164015131490332
6.42 0.0143478062915026
6.43 0.012490168441462
6.44 0.0108179125048566
6.45 0.00932003295780759
6.46 0.00798530696062605
6.47 0.00680239608406519
6.48 0.00575994711949646
6.49 0.00484669042422392
6.5 0.00405153426815508
6.51 0.00336365370072478
6.52 0.0027725725483092
6.53 0.00226823728225145
6.54 0.001841081664405
6.55 0.00148208127775318
6.56 0.00118279727954151
6.57 0.00093540896741684
6.58 0.000732735017786923
6.59 0.000568243531391954
6.6 0.000436051294375961
6.61 0.000330912923979132
6.62 0.000248200806277077
6.63 0.000183876939617855
6.64 0.000134457963007373
6.65 9.69747666652587e-05
6.66 6.89281474024472e-05
6.67 4.82419818607083e-05
6.68 3.32153463107332e-05
6.69 2.24749157278215e-05
6.7 1.49288332450537e-05
6.71 9.72306224100716e-06
6.72 6.20102774384623e-06
6.73 3.86713341127276e-06
6.74 2.3545174098194e-06
6.75 1.39719722502818e-06
6.76 8.06560555899338e-07
6.77 4.51995867623012e-07
6.78 2.45328088066543e-07
6.79 1.28635452623094e-07
6.8 6.4972758728743e-08
6.81 3.15113576343506e-08
6.82 1.46218844809809e-08
6.83 6.46507503088145e-09
6.84 2.71127975626229e-09
6.85 1.07280828665068e-09
6.86 3.98107102839163e-10
6.87 1.37591160687123e-10
6.88 4.39331904189544e-11
6.89 1.28387300790678e-11
6.9 3.3957281431185e-12
6.91 8.02247157594138e-13
6.92 1.66755498298699e-13
6.93 2.97539770599542e-14
6.94 4.32986979603811e-15
6.95 4.44089209850063e-16
6.96 0
6.97 0
6.98 0
6.99 0
7 0
7.01 0
7.02 0
7.03 0
7.04 0
7.05 0
7.06 0
7.07 0
7.08 0
7.09 0
7.1 0
7.11 0
7.12 0
7.13 0
7.14 0
7.15 0
7.16 0
7.17 0
7.18 0
7.19 0
7.2 0
7.21 0
7.22 0
7.23 0
7.24 0
7.25 0
7.26 0
7.27 0
7.28 0
7.29 0
7.3 0
7.31 0
7.32 0
7.33 0
7.34 0
7.35 0
7.36 0
7.37 0
7.38 0
7.39 0
7.4 0
7.41 0
7.42 0
7.43 0
7.44 0
7.45 0
7.46 0
7.47 0
7.48 0
7.49 0
7.5 0
7.51 0
7.52 0
7.53 0
7.54 0
7.55 0
7.56 0
7.57 0
7.58 0
7.59 0
7.6 0
7.61 0
7.62 0
7.63 0
7.64 0
7.65 0
7.66 0
7.67 0
7.68 0
7.69 0
7.7 0
7.71 0
7.72 0
7.73 0
7.74 0
7.75 0
7.76 0
7.77 0
7.78 0
7.79 0
7.8 0
7.81 0
7.82 0
7.83 0
7.84 0
7.85 0
7.86 0
7.87 0
7.88 0
7.89 0
7.9 0
7.91 0
7.92 0
7.93 0
7.94 0
7.95 0
7.96 0
7.97 0
7.98 0
7.99 0
8 0
8.01 0
8.02 0
8.03 0
8.04 0
8.05 0
8.06 0
8.07 0
8.08 0
8.09 0
8.1 0
8.11 0
8.12 0
8.13 0
8.14 0
8.15 0
8.16 0
8.17 0
8.18 0
8.19 0
8.2 0
8.21 0
8.22 0
8.23 0
8.24 0
8.25 0
8.26 0
8.27 0
8.28 0
8.29 0
8.3 0
8.31 0
8.32 0
8.33 0
8.34 0
8.35 0
8.36 0
8.37 0
8.38 0
8.39 0
8.4 0
8.41 0
8.42 0
8.43 0
8.44 0
8.45 0
8.46 0
8.47 0
8.48 0
8.49 0
8.5 0
8.51 0
8.52 0
8.53 0
8.54 0
8.55 0
8.56 0
8.57 0
8.58 0
8.59 0
8.6 0
8.61 0
8.62 0
8.63 0
8.64 0
8.65 0
8.66 0
8.67 0
8.68 0
8.69 0
8.7 0
8.71 0
8.72 0
8.73 0
8.74 0
8.75 0
8.76 0
8.77 0
8.78 0
8.79 0
8.8 0
8.81 0
8.82 0
8.83 0
8.84 0
8.85 0
8.86 0
8.87 0
8.88 0
8.89 0
8.9 0
8.91 0
8.92 0
8.93 0
8.94 0
8.95 0
8.96 0
8.97 0
8.98 0
8.99 0
9 0
9.01 0
9.02 0
9.03 0
9.04 0
9.05 0
9.06 0
9.07 0
9.08 0
9.09 0
9.1 0
9.11 0
9.12 0
9.13 0
9.14 0
9.15 0
9.16 0
9.17 0
9.18 0
9.19 0
9.2 0
9.21 0
9.22 0
9.23 0
9.24 0
9.25 0
9.26 0
9.27 0
9.28 0
9.29 0
9.3 0
9.31 0
9.32 0
9.33 0
9.34 0
9.35 0
9.36 0
9.37 0
9.38 0
9.39 0
9.4 0
9.41 0
9.42 0
9.43 0
9.44 0
9.45 0
9.46 0
9.47 0
9.48 0
9.49 0
9.5 0
9.51 0
9.52 0
9.53 0
9.54 0
9.55 0
9.56 0
9.57 0
9.58 0
9.59 0
9.6 0
9.61 0
9.62 0
9.63 0
9.64 0
9.65 0
9.66 0
9.67 0
9.68 0
9.69 0
9.7 0
9.71 0
9.72 0
9.73 0
9.74 0
9.75 0
9.76 0
9.77 0
9.78 0
9.79 0
9.8 0
9.81 0
9.82 0
9.83 0
9.84 0
9.85 0
9.86 0
9.87 0
9.88 0
9.89 0
9.9 0
9.91 0
9.92 0
9.93 0
9.94 0
9.95 0
9.96 0
9.97 0
9.98 0
9.99 0
10 0
10.01 0
10.02 0
10.03 0
10.04 0
10.05 0
10.06 0
10.07 0
10.08 0
10.09 0
10.1 0
10.11 0
10.12 0
10.13 0
10.14 0
10.15 0
10.16 0
10.17 0
10.18 0
10.19 0
10.2 0
10.21 0
10.22 0
10.23 0
10.24 0
10.25 0
10.26 0
10.27 0
10.28 0
10.29 0
10.3 0
10.31 0
10.32 0
10.33 0
10.34 0
10.35 0
10.36 0
10.37 0
10.38 0
10.39 0
10.4 0
10.41 0
10.42 0
10.43 0
10.44 0
10.45 0
10.46 0
10.47 0
10.48 0
10.49 0
10.5 0
10.51 0
10.52 0
10.53 0
10.54 0
10.55 0
10.56 0
10.57 0
10.58 0
10.59 0
10.6 0
10.61 0
10.62 0
10.63 0
10.64 0
10.65 0
10.66 0
10.67 0
10.68 0
10.69 0
10.7 0
10.71 0
10.72 0
10.73 0
10.74 0
10.75 0
10.76 0
10.77 0
10.78 0
10.79 0
10.8 0
10.81 0
10.82 0
10.83 0
10.84 0
10.85 0
10.86 0
10.87 0
10.88 0
10.89 0
10.9 0
10.91 0
10.92 0
10.93 0
10.94 0
10.95 0
10.96 0
10.97 0
10.98 0
10.99 0
11 0
11.01 0
11.02 0
11.03 0
11.04 0
11.05 0
11.06 0
11.07 0
11.08 0
11.09 0
11.1 0
11.11 0
11.12 0
11.13 0
11.14 0
11.15 0
11.16 0
11.17 0
11.18 0
11.19 0
11.2 0
11.21 0
11.22 0
11.23 0
11.24 0
11.25 0
11.26 0
11.27 0
11.28 0
11.29 0
11.3 0
11.31 0
11.32 0
11.33 0
11.34 0
11.35 0
11.36 0
11.37 0
11.38 0
11.39 0
11.4 0
11.41 0
11.42 0
11.43 0
11.44 0
11.45 0
11.46 0
11.47 0
11.48 0
11.49 0
11.5 0
11.51 0
11.52 0
11.53 0
11.54 0
11.55 0
11.56 0
11.57 0
11.58 0
11.59 0
11.6 0
11.61 0
11.62 0
11.63 0
11.64 0
11.65 0
11.66 0
11.67 0
11.68 0
11.69 0
11.7 0
11.71 0
11.72 0
11.73 0
11.74 0
11.75 0
11.76 0
11.77 0
11.78 0
11.79 0
11.8 0
11.81 0
11.82 0
11.83 0
11.84 0
11.85 0
11.86 0
11.87 0
11.88 0
11.89 0
11.9 0
11.91 0
11.92 0
11.93 0
11.94 0
11.95 0
11.96 0
11.97 0
11.98 0
11.99 0
12 0
};
\addplot [very thick, black]
table {%
0 5.98595094998751e-05
0.01 5.98438876235546e-05
0.02 5.98288670967295e-05
0.03 5.98144533597339e-05
0.04 5.9800651922724e-05
0.05 5.97874683663011e-05
0.06 5.97749083421389e-05
0.07 5.97629775736145e-05
0.08 5.97516818564437e-05
0.09 5.97410270593199e-05
0.1 5.97310191245579e-05
0.11 5.972166406874e-05
0.12 5.97129679833682e-05
0.13 5.97049370355184e-05
0.14 5.96975774685002e-05
0.15 5.96908956025195e-05
0.16 5.96848978353455e-05
0.17 5.96795906429821e-05
0.18 5.96749805803424e-05
0.19 5.96710742819275e-05
0.2 5.96678784625096e-05
0.21 5.96653999178186e-05
0.22 5.96636455252325e-05
0.23 5.96626222444721e-05
0.24 5.96623371182991e-05
0.25 5.96627972732185e-05
0.26 5.96640099201848e-05
0.27 5.96659823553116e-05
0.28 5.96687219605853e-05
0.29 5.96722362045828e-05
0.3 5.96765326431927e-05
0.31 5.96816189203403e-05
0.32 5.96875027687161e-05
0.33 5.96941920105085e-05
0.34 5.97016945581399e-05
0.35 5.97100184150066e-05
0.36 5.97191716762216e-05
0.37 5.97291625293624e-05
0.38 5.97399992552211e-05
0.39 5.97516902285585e-05
0.4 5.97642439188621e-05
0.41 5.97776688911066e-05
0.42 5.97919738065192e-05
0.43 5.98071674233468e-05
0.44 5.9823258597628e-05
0.45 5.98402562839672e-05
0.46 5.98581695363128e-05
0.47 5.98770075087387e-05
0.48 5.98967794562283e-05
0.49 5.9917494735462e-05
0.5 5.99391628056086e-05
0.51 5.99617932291186e-05
0.52 5.99853956725208e-05
0.53 6.00099799072229e-05
0.54 6.00355558103135e-05
0.55 6.00621333653681e-05
0.56 6.00897226632577e-05
0.57 6.01183339029596e-05
0.58 6.01479773923718e-05
0.59 6.01786635491292e-05
0.6 6.02104029014234e-05
0.61 6.02432060888237e-05
0.62 6.02770838631021e-05
0.63 6.03120470890591e-05
0.64 6.03481067453534e-05
0.65 6.0385273925333e-05
0.66 6.04235598378684e-05
0.67 6.04629758081881e-05
0.68 6.05035332787169e-05
0.69 6.05452438099149e-05
0.7 6.05881190811194e-05
0.71 6.06321708913879e-05
0.72 6.06774111603438e-05
0.73 6.07238519290227e-05
0.74 6.0771505360721e-05
0.75 6.08203837418456e-05
0.76 6.08704994827657e-05
0.77 6.09218651186645e-05
0.78 6.09744933103939e-05
0.79 6.1028396845329e-05
0.8 6.10835886382242e-05
0.81 6.11400817320701e-05
0.82 6.11978892989513e-05
0.83 6.12570246409052e-05
0.84 6.13175011907806e-05
0.85 6.13793325130976e-05
0.86 6.14425323049082e-05
0.87 6.15071143966558e-05
0.88 6.15730927530365e-05
0.89 6.16404814738594e-05
0.9 6.17092947949079e-05
0.91 6.1779547088799e-05
0.92 6.18512528658449e-05
0.93 6.1924426774912e-05
0.94 6.19990836042802e-05
0.95 6.20752382825022e-05
0.96 6.21529058792608e-05
0.97 6.22321016062261e-05
0.98 6.23128408179116e-05
0.99 6.23951390125287e-05
1 6.24790118328405e-05
1.01 6.25644750670135e-05
1.02 6.26515446494682e-05
1.03 6.27402366617277e-05
1.04 6.28305673332648e-05
1.05 6.29225530423461e-05
1.06 6.30162103168754e-05
1.07 6.31115558352331e-05
1.08 6.32086064271148e-05
1.09 6.33073790743656e-05
1.1 6.34078909118129e-05
1.11 6.35101592280956e-05
1.12 6.36142014664901e-05
1.13 6.37200352257334e-05
1.14 6.38276782608421e-05
1.15 6.39371484839289e-05
1.16 6.40484639650136e-05
1.17 6.41616429328319e-05
1.18 6.42767037756383e-05
1.19 6.43936650420068e-05
1.2 6.45125454416242e-05
1.21 6.46333638460818e-05
1.22 6.47561392896595e-05
1.23 6.48808909701071e-05
1.24 6.50076382494183e-05
1.25 6.51364006546008e-05
1.26 6.52671978784399e-05
1.27 6.54000497802565e-05
1.28 6.55349763866591e-05
1.29 6.56719978922897e-05
1.3 6.58111346605622e-05
1.31 6.59524072243961e-05
1.32 6.60958362869417e-05
1.33 6.62414427222987e-05
1.34 6.63892475762278e-05
1.35 6.65392720668547e-05
1.36 6.66915375853663e-05
1.37 6.6846065696699e-05
1.38 6.70028781402189e-05
1.39 6.71619968303943e-05
1.4 6.73234438574581e-05
1.41 6.74872414880636e-05
1.42 6.7653412165929e-05
1.43 6.7821978512475e-05
1.44 6.7992963327451e-05
1.45 6.81663895895528e-05
1.46 6.83422804570304e-05
1.47 6.8520659268285e-05
1.48 6.87015495424564e-05
1.49 6.88849749799999e-05
1.5 6.90709594632517e-05
1.51 6.92595270569839e-05
1.52 6.94507020089484e-05
1.53 6.96445087504087e-05
1.54 6.98409718966608e-05
1.55 7.00401162475416e-05
1.56 7.02419667879256e-05
1.57 7.0446548688209e-05
1.58 7.06538873047811e-05
1.59 7.08640081804834e-05
1.6 7.10769370450551e-05
1.61 7.12926998155653e-05
1.62 7.15113225968329e-05
1.63 7.17328316818305e-05
1.64 7.19572535520768e-05
1.65 7.21846148780133e-05
1.66 7.24149425193671e-05
1.67 7.26482635254991e-05
1.68 7.28846051357369e-05
1.69 7.31239947796937e-05
1.7 7.33664600775712e-05
1.71 7.3612028840446e-05
1.72 7.38607290705432e-05
1.73 7.41125889614908e-05
1.74 7.43676368985601e-05
1.75 7.46259014588892e-05
1.76 7.48874114116898e-05
1.77 7.51521957184372e-05
1.78 7.54202835330437e-05
1.79 7.56917042020152e-05
1.8 7.5966487264589e-05
1.81 7.62446624528564e-05
1.82 7.65262596918651e-05
1.83 7.68113090997058e-05
1.84 7.70998409875796e-05
1.85 7.73918858598477e-05
1.86 7.76874744140625e-05
1.87 7.79866375409798e-05
1.88 7.82894063245532e-05
1.89 7.85958120419094e-05
1.9 7.89058861633032e-05
1.91 7.92196603520554e-05
1.92 7.95371664644702e-05
1.93 7.98584365497332e-05
1.94 8.01835028497905e-05
1.95 8.05123977992074e-05
1.96 8.08451540250081e-05
1.97 8.11818043464955e-05
1.98 8.15223817750498e-05
1.99 8.18669195139095e-05
2 8.22154509579293e-05
2.01 8.25680096933211e-05
2.02 8.29246294973721e-05
2.03 8.32853443381437e-05
2.04 8.36501883741504e-05
2.05 8.40191959540185e-05
2.06 8.43924016161228e-05
2.07 8.47698400882053e-05
2.08 8.51515462869719e-05
2.09 8.55375553176693e-05
2.1 8.59279024736413e-05
2.11 8.6322623235866e-05
2.12 8.67217532724704e-05
2.13 8.7125328438228e-05
2.14 8.75333847740327e-05
2.15 8.79459585063563e-05
2.16 8.83630860466836e-05
2.17 8.87848039909284e-05
2.18 8.92111491188301e-05
2.19 8.96421583933311e-05
2.2 9.00778689599337e-05
2.21 9.05183181460393e-05
2.22 9.0963543460267e-05
2.23 9.14135825917553e-05
2.24 9.18684734094436e-05
2.25 9.23282539613366e-05
2.26 9.27929624737495e-05
2.27 9.32626373505372e-05
2.28 9.37373171723039e-05
2.29 9.42170406955972e-05
2.3 9.47018468520851e-05
2.31 9.51917747477163e-05
2.32 9.56868636618639e-05
2.33 9.61871530464552e-05
2.34 9.6692682525084e-05
2.35 9.72034918921106e-05
2.36 9.77196211117448e-05
2.37 9.82411103171177e-05
2.38 9.87679998093386e-05
2.39 9.93003300565393e-05
2.4 9.98381416929067e-05
2.41 0.000100381475517704
2.42 0.00010093037249428
2.43 0.000101484873749068
2.44 0.000102045020570577
2.45 0.000102610854408372
2.46 0.000103182416872047
2.47 0.000103759749730187
2.48 0.000104342894909332
2.49 0.000104931894492924
2.5 0.00010552679072026
2.51 0.000106127625985429
2.52 0.000106734442836257
2.53 0.000107347283973242
2.54 0.000107966192248492
2.55 0.00010859121066466
2.56 0.000109222382373882
2.57 0.000109859750676712
2.58 0.000110503359021067
2.59 0.000111153251001171
2.6 0.000111809470356504
2.61 0.000112472060970762
2.62 0.000113141066870819
2.63 0.0001138165322257
2.64 0.000114498501345567
2.65 0.000115187018680714
2.66 0.000115882128820574
2.67 0.000116583876492741
2.68 0.000117292306562014
2.69 0.000118007464029448
2.7 0.000118729394031431
2.71 0.000119458141838785
2.72 0.000120193752855877
2.73 0.000120936272619769
2.74 0.00012168574679938
2.75 0.000122442221194689
2.76 0.000123205741735957
2.77 0.000123976354482987
2.78 0.000124754105624409
2.79 0.000125539041477015
2.8 0.000126331208485114
2.81 0.00012713065321994
2.82 0.000127937422379095
2.83 0.000128751562786037
2.84 0.000129573121389616
2.85 0.000130402145263653
2.86 0.000131238681606579
2.87 0.000132082777741117
2.88 0.000132934481114026
2.89 0.000133793839295901
2.9 0.000134660899981035
2.91 0.000135535710987342
2.92 0.000136418320256346
2.93 0.000137308775853242
2.94 0.000138207125967027
2.95 0.000139113418910701
2.96 0.000140027703121554
2.97 0.000140950027161523
2.98 0.000141880439717645
2.99 0.00014281898960258
3 0.000143765725755242
3.01 0.000161215588096444
3.02 0.000180905287858888
3.03 0.000203024023694681
3.04 0.000227745403025943
3.05 0.000255220027649786
3.06 0.000285567825224322
3.07 0.000318870412133891
3.08 0.000355163796988573
3.09 0.000394431741311103
3.1 0.000436600083804872
3.11 0.000481532307536623
3.12 0.000529026587514908
3.13 0.00057881450297275
3.14 0.000630561538653531
3.15 0.000683869437556791
3.16 0.00073828040888707
3.17 0.000793283143705182
3.18 0.00084832055021086
3.19 0.000902799092359993
3.2 0.000956099599564176
3.21 0.0010075894097141
3.22 0.00105663570937653
3.23 0.001102619939345
3.24 0.0011449531359791
3.25 0.00118309207454433
3.26 0.00121655606692175
3.27 0.00124494424155599
3.28 0.00126795310009213
3.29 0.00128539410770973
3.3 0.00129721104063468
3.31 0.00130349679496305
3.32 0.00130450936694636
3.33 0.00130068675628914
3.34 0.00129266062716471
3.35 0.00128126868671422
3.36 0.00126756589966387
3.37 0.00125283483310611
3.38 0.00123859559169308
3.39 0.00122661592877763
3.4 0.00121892216991063
3.41 0.00121781153149386
3.42 0.00122586623802104
3.43 0.00124596952832214
3.44 0.00128132320311237
3.45 0.00133546582918755
3.46 0.00141229012326803
3.47 0.00151605744963869
3.48 0.00165140685177996
3.49 0.00182335567924205
3.5 0.00203728875049744
3.51 0.00229893318937724
3.52 0.00261431665049409
3.53 0.0029897076415826
3.54 0.00343153804560889
3.55 0.00394630966898522
3.56 0.00454048854954826
3.57 0.00522039263585868
3.58 0.00599208003494416
3.59 0.00686124604321077
3.6 0.00783313738756301
3.61 0.00891249136699604
3.62 0.0101035058937562
3.63 0.0114098439410124
3.64 0.0128346729085506
3.65 0.0143807363120455
3.66 0.0160504523992019
3.67 0.0178460321537323
3.68 0.0197696078995359
3.69 0.0218233634398003
3.7 0.0240096572822832
3.71 0.0263311318159665
3.72 0.0287908030517559
3.73 0.031392127446264
3.74 0.0341390441540292
3.75 0.0370359926256406
3.76 0.0400879066880841
3.77 0.0433001870797616
3.78 0.0466786548911352
3.79 0.0502294885431888
3.8 0.0539591468973641
3.81 0.0578742809120228
3.82 0.0619816360138146
3.83 0.0662879470959218
3.84 0.0707998278317159
3.85 0.07552365582878
3.86 0.0804654550574665
3.87 0.0856307769711052
3.88 0.0910245817833043
3.89 0.0966511214662716
3.9 0.10251382616302
3.91 0.10861519584385
3.92 0.114956699161475
3.93 0.121538681548888
3.94 0.128360284641714
3.95 0.135419379078243
3.96 0.142712512625819
3.97 0.150234875396946
3.98 0.157980283652065
3.99 0.165941183343153
4 0.174108674141974
4.01 0.182473362978829
4.02 0.191024950945079
4.03 0.199751295451676
4.04 0.208639173480638
4.05 0.217674388979815
4.06 0.226841895380215
4.07 0.236125931049904
4.08 0.245510165159336
4.09 0.254977851173182
4.1 0.264511985013855
4.11 0.274095464868162
4.12 0.283711249632203
4.13 0.293342513107302
4.14 0.302972791263637
4.15 0.31258612016671
4.16 0.322167162500405
4.17 0.331701321002625
4.18 0.341174837537968
4.19 0.350574876949556
4.2 0.359889595242417
4.21 0.369108192039724
4.22 0.378220947608318
4.23 0.387219245062259
4.24 0.396095578616128
4.25 0.404843548969916
4.26 0.413457847063691
4.27 0.421934227544292
4.28 0.430269473341307
4.29 0.438461352760502
4.3 0.4465085704753
4.31 0.454410713737543
4.32 0.462168195043848
4.33 0.469782192390263
4.34 0.477254588131307
4.35 0.484587907335779
4.36 0.491785256405361
4.37 0.498850262597481
4.38 0.505787014974065
4.39 0.512600007185631
4.4 0.51929408239716
4.41 0.525874380569861
4.42 0.532346288231828
4.43 0.538715390801016
4.44 0.544987427465766
4.45 0.551168248580938
4.46 0.557263775500805
4.47 0.563279962742629
4.48 0.569222762356166
4.49 0.575098090363495
4.5 0.580911795129402
4.51 0.586669627524206
4.52 0.592377212747355
4.53 0.598040023690523
4.54 0.60366335573232
4.55 0.609252302872347
4.56 0.614811735129343
4.57 0.620346277145915
4.58 0.625860287960081
4.59 0.631357841921034
4.6 0.636842710742511
4.61 0.642318346701591
4.62 0.647787867002937
4.63 0.653254039338392
4.64 0.658719268678896
4.65 0.664185585339768
4.66 0.669654634361441
4.67 0.67512766624554
4.68 0.680605529080956
4.69 0.686088662086319
4.7 0.69157709058436
4.71 0.697070422410325
4.72 0.702567845741434
4.73 0.70806812831779
4.74 0.713569618008003
4.75 0.719070244655578
4.76 0.724567523125981
4.77 0.730058557459967
4.78 0.735540046027318
4.79 0.741008287567704
4.8 0.746459188003064
4.81 0.751888267909845
4.82 0.757290670551027
4.83 0.762661170388097
4.84 0.76799418202352
4.85 0.773283769565745
4.86 0.778523656462711
4.87 0.783707235917018
4.88 0.788827582077201
4.89 0.793877462295362
4.9 0.798849350851637
4.91 0.803735444669876
4.92 0.808527681684902
4.93 0.813217762666763
4.94 0.81779717745722
4.95 0.822257236721797
4.96 0.826589110458394
4.97 0.830783874619152
4.98 0.834832567281468
4.99 0.838726255828904
5 0.842456116552377
5.01 0.846013527932789
5.02 0.84939017859322
5.03 0.852578190486882
5.04 0.855570257294411
5.05 0.858359797224968
5.06 0.860941118445388
5.07 0.86330959421129
5.08 0.865461843475953
5.09 0.86739591136479
5.1 0.86911144251208
5.11 0.870609838977624
5.12 0.871894393434266
5.13 0.872970387697853
5.14 0.873845146615097
5.15 0.874528037967403
5.16 0.875030410483317
5.17 0.87536546430523
5.18 0.875548051269385
5.19 0.875594405978996
5.2 0.875521812634362
5.21 0.875348216614548
5.22 0.87509179352739
5.23 0.874770491508415
5.24 0.874401564659227
5.25 0.874001116473703
5.26 0.873583671834955
5.27 0.873161794744232
5.28 0.872745766556753
5.29 0.8723433364342
5.3 0.871959552310998
5.31 0.871596677237461
5.32 0.871254192778257
5.33 0.870928888387513
5.34 0.870615033415143
5.35 0.870304626566748
5.36 0.86998771608074
5.37 0.869652782363953
5.38 0.869287173069036
5.39 0.868877578347292
5.4 0.868410531085584
5.41 0.867872913282359
5.42 0.867252445466594
5.43 0.866538131569473
5.44 0.86572062751368
5.45 0.86479249879093
5.46 0.863748331385721
5.47 0.862584662494044
5.48 0.861299703307279
5.49 0.859892836010737
5.5 0.858363880833173
5.51 0.856712145559771
5.52 0.85493528784478
5.53 0.853028037972565
5.54 0.850980844386562
5.55 0.848778514646153
5.56 0.846398929513101
5.57 0.843811907593507
5.58 0.840978293306929
5.59 0.837849333582961
5.6 0.834366400559291
5.61 0.830461110425644
5.62 0.826055883486459
5.63 0.821064987523449
5.64 0.815396104453506
5.65 0.80895245678514
5.66 0.801635522290198
5.67 0.793348349017244
5.68 0.783999454874658
5.69 0.773507254088244
5.7 0.761804896351871
5.71 0.748845335680069
5.72 0.73460637057549
5.73 0.719095324642976
5.74 0.702352979981899
5.75 0.684456349078514
5.76 0.66551988815971
5.77 0.645694825503473
5.78 0.625166403836106
5.79 0.604149008392547
5.8 0.582879352654469
5.81 0.561608095106542
5.82 0.540590431990552
5.83 0.520076325295057
5.84 0.500301063455793
5.85 0.481476809057643
5.86 0.463785672009183
5.87 0.447374678699696
5.88 0.43235281473443
5.89 0.418790128898394
5.9 0.406718722431357
5.91 0.396135326043168
5.92 0.387005093955633
5.93 0.379266218208265
5.94 0.372834980264422
5.95 0.367610900146843
5.96 0.363481704641889
5.97 0.360327905221285
5.98 0.358026844927747
5.99 0.356456135658986
6 0.355496459511574
6.01 0.355033748556862
6.02 0.354960786584037
6.03 0.355178294968611
6.04 0.355595574527845
6.05 0.356130777894604
6.06 0.356710884479783
6.07 0.357271444247254
6.08 0.357756148785507
6.09 0.358116279715163
6.1 0.358310076198451
6.11 0.358302055799703
6.12 0.3580623165103
6.13 0.357565842508751
6.14 0.356791832122905
6.15 0.355723063327154
6.16 0.354345309705536
6.17 0.352646817879421
6.18 0.35061785568432
6.19 0.348250338671571
6.2 0.345537540650276
6.21 0.342473891879694
6.22 0.339054866140675
6.23 0.335276955277274
6.24 0.331137726965647
6.25 0.326635958519577
6.26 0.321771836574611
6.27 0.316547209603515
6.28 0.310965877504207
6.29 0.305033900073789
6.3 0.298759904160369
6.31 0.292155367813875
6.32 0.28523485901419
6.33 0.278016206742754
6.34 0.273268740551743
6.35 0.269833998231318
6.36 0.266381183568511
6.37 0.26291275291206
6.38 0.259431123669872
6.39 0.255938669712185
6.4 0.252437717074211
6.41 0.248930539983831
6.42 0.245419357236684
6.43 0.241906328937397
6.44 0.238393553621791
6.45 0.234883065770771
6.46 0.231376833722362
6.47 0.227876757984019
6.48 0.224384669943012
6.49 0.220902330968523
6.5 0.217431431895002
6.51 0.213973592872579
6.52 0.210530363566815
6.53 0.207103223686929
6.54 0.203693583818929
6.55 0.200302786537748
6.56 0.19693210777064
6.57 0.193582758382729
6.58 0.190255885954692
6.59 0.18695257672211
6.6 0.183673857646049
6.61 0.180420698584874
6.62 0.177194014538102
6.63 0.173994667934361
6.64 0.170823470936957
6.65 0.167681187742396
6.66 0.164568536849202
6.67 0.161486193276544
6.68 0.15843479071455
6.69 0.155414923590538
6.7 0.152427149037899
6.71 0.149471988756753
6.72 0.146549930757943
6.73 0.143661430984246
6.74 0.140806914804905
6.75 0.137986778381659
6.76 0.13520138990644
6.77 0.132451090712615
6.78 0.129736196263291
6.79 0.12705699702157
6.8 0.124413759208895
6.81 0.121806725458576
6.82 0.119236115372498
6.83 0.116702125989567
6.84 0.114204932174986
6.85 0.111744686939698
6.86 0.109321521699508
6.87 0.106935546483374
6.88 0.104586850100231
6.89 0.102275500273474
6.9 0.10000154375187
6.91 0.0977650064052365
6.92 0.0955658933127275
6.93 0.0934041888509957
6.94 0.091279856788904
6.95 0.0891928403948055
6.96 0.0871430625617621
6.97 0.0851304259553911
6.98 0.0831548131883632
6.99 0.0812160870249068
7 0.0793140906180168
7.01 0.0774486477814545
7.02 0.0756195632980011
7.03 0.0738266232648768
7.04 0.0720695954766899
7.05 0.070348229845788
7.06 0.0686622588594347
7.07 0.0670113980728031
7.08 0.0653953466364269
7.09 0.0638137878564032
7.1 0.0622663897853669
7.11 0.0607528058420024
7.12 0.0592726754566602
7.13 0.0578256247404753
7.14 0.0564112671752589
7.15 0.0550292043213395
7.16 0.0536790265404682
7.17 0.0523603137308767
7.18 0.0510726360715691
7.19 0.049815554772959
7.2 0.0485886228310084
7.21 0.047391385782091
7.22 0.0462233824558974
7.23 0.0450841457237977
7.24 0.0439732032401957
7.25 0.0428900781745429
7.26 0.0418342899318158
7.27 0.0408053548594099
7.28 0.0398027869385603
7.29 0.038826098458547
7.3 0.0378748006721204
7.31 0.0369484044307241
7.32 0.036046420798272
7.33 0.0351683616423841
7.34 0.0343137402021497
7.35 0.0334820716316382
7.36 0.0326728735185265
7.37 0.0318856663773586
7.38 0.0311199741170896
7.39 0.0303753244826974
7.4 0.0296512494707732
7.41 0.0289472857191173
7.42 0.0282629748704767
7.43 0.0275978639106661
7.44 0.026951505481404
7.45 0.0263234581682879
7.46 0.0257132867644049
7.47 0.0251205625101529
7.48 0.0245448633098998
7.49 0.0239857739261788
7.5 0.0234428861521486
7.51 0.0229157989631059
7.52 0.0224041186478586
7.53 0.0219074589208054
7.54 0.0214254410155829
7.55 0.0209576937611621
7.56 0.020503853641285
7.57 0.0200635648381359
7.58 0.0196364792611467
7.59 0.0192222565618281
7.6 0.0188205641355146
7.61 0.0184310771108958
7.62 0.018053478328198
7.63 0.0176874583068562
7.64 0.0173327152035033
7.65 0.0169889547610759
7.66 0.0166558902498151
7.67 0.0163332424009158
7.68 0.0160207393335482
7.69 0.0157181164759492
7.7 0.0154251164812552
7.71 0.0151414891387112
7.72 0.0148669912808716
7.73 0.0146013866873692
7.74 0.0143444459858058
7.75 0.0140959465502842
7.76 0.0138556723980748
7.77 0.0136234140848822
7.78 0.0133989685991447
7.79 0.0131821392557793
7.8 0.0129727355897529
7.81 0.0127705732498375
7.82 0.0125754738928832
7.83 0.0123872650789178
7.84 0.0122057801673611
7.85 0.0120308582146196
7.86 0.0118623438733084
7.87 0.0117000872933273
7.88 0.0115439440250001
7.89 0.0113937749244713
7.9 0.0112494460615365
7.91 0.01111082863007
7.92 0.0109777988611992
7.93 0.0108502379393643
7.94 0.0107280319213893
7.95 0.0106110716586806
7.96 0.010499252722662
7.97 0.0103924753335445
7.98 0.0102906442925237
7.99 0.0101936689174919
8 0.0101014629823448
8.01 0.0100123288002276
8.02 0.00992461567770499
8.03 0.00983830642366608
8.04 0.00975338396522381
8.05 0.0096698313502775
8.06 0.00958763174999226
8.07 0.009506768461194
8.08 0.00942722490867764
8.09 0.00934898464742761
8.1 0.00927203136474914
8.11 0.00919634888230885
8.12 0.00912192115808483
8.13 0.00904873228822412
8.14 0.00897676650880845
8.15 0.00890600819752676
8.16 0.00883644187525517
8.17 0.00876805220754406
8.18 0.00870082400601228
8.19 0.00863474222964904
8.2 0.00856979198602403
8.21 0.00850595853240581
8.22 0.00844322727678992
8.23 0.00838158377883683
8.24 0.00832101375072119
8.25 0.00826150305789318
8.26 0.00820303771975326
8.27 0.0081456039102412
8.28 0.0080891879583412
8.29 0.0080337763485044
8.3 0.00797935572098977
8.31 0.00792591287212559
8.32 0.00787343475449291
8.33 0.00782190847703239
8.34 0.00777132130507649
8.35 0.00772166066030893
8.36 0.00767291412065277
8.37 0.00762506942008956
8.38 0.00757811444841069
8.39 0.00753203725090355
8.4 0.00748682602797404
8.41 0.00744246913470735
8.42 0.00739895508036919
8.43 0.00735627252784927
8.44 0.00731441029304891
8.45 0.00727335734421516
8.46 0.0072331028012228
8.47 0.00719363593480658
8.48 0.00715494616574577
8.49 0.00711702306400256
8.5 0.00707985634781651
8.51 0.00704343588275699
8.52 0.00700775168073535
8.53 0.00697279389897888
8.54 0.0069385528389683
8.55 0.00690501894534054
8.56 0.00687218280475899
8.57 0.00684003514475242
8.58 0.00680856683252494
8.59 0.00677776887373821
8.6 0.00674763241126792
8.61 0.00671814872393602
8.62 0.0066893092252205
8.63 0.00666110546194406
8.64 0.00663352911294336
8.65 0.00660657198772065
8.66 0.00658022602507847
8.67 0.00655448329173989
8.68 0.00652933598095479
8.69 0.00650477641109413
8.7 0.00648079702423328
8.71 0.0064573903847257
8.72 0.00643454917776833
8.73 0.00641226620795981
8.74 0.00639053439785273
8.75 0.00636934678650101
8.76 0.00634869652800348
8.77 0.00632857689004495
8.78 0.0063089812524353
8.79 0.00628990310564801
8.8 0.00627133604935902
8.81 0.00625327379098634
8.82 0.00623571014423202
8.83 0.00621863902762669
8.84 0.00620205446307774
8.85 0.00618595057442179
8.86 0.00617032158598229
8.87 0.00615516182113268
8.88 0.0061404657008661
8.89 0.00612622774237196
8.9 0.00611244255762008
8.91 0.00609910485195288
8.92 0.00608620942268628
8.93 0.00607375115771956
8.94 0.00606172503415486
8.95 0.00605012611692653
8.96 0.00603894955744087
8.97 0.00602819059222665
8.98 0.00601784454159641
8.99 0.00600790680831938
9 0.00599837287630595
9.01 0.00598923830930371
9.02 0.00598049874960609
9.03 0.00597214991677281
9.04 0.0059641876063632
9.05 0.00595660768868174
9.06 0.00594940610753684
9.07 0.00594257887901199
9.08 0.00593612209025028
9.09 0.00593003189825175
9.1 0.00592430452868381
9.11 0.00591893627470495
9.12 0.00591392349580127
9.13 0.00590926261663655
9.14 0.00590495012591502
9.15 0.00590098257525747
9.16 0.00589735657809034
9.17 0.00589406880854756
9.18 0.00589111600038563
9.19 0.00588849494591106
9.2 0.0058862024949209
9.21 0.00588423555365543
9.22 0.00588259108376365
9.23 0.00588126610128072
9.24 0.00588025767561777
9.25 0.00587956292856362
9.26 0.00587917903329828
9.27 0.00587910321341807
9.28 0.00587933274197225
9.29 0.00587986494051083
9.3 0.00588069717814341
9.31 0.00588182687060907
9.32 0.00588325147935655
9.33 0.00588496851063519
9.34 0.00588697551459582
9.35 0.00588927008440175
9.36 0.00589184985534939
9.37 0.00589471250399854
9.38 0.00589785574731184
9.39 0.00590127734180314
9.4 0.00590497508269496
9.41 0.00590894680308426
9.42 0.00591319037311649
9.43 0.00591770369916797
9.44 0.00592248472303575
9.45 0.00592753142113537
9.46 0.00593284180370565
9.47 0.00593841391402077
9.48 0.00594424582760917
9.49 0.00595033565147897
9.5 0.00595668152334984
9.51 0.00596328161089099
9.52 0.00597013411096498
9.53 0.00597723724887731
9.54 0.00598458927763137
9.55 0.00599218847718861
9.56 0.00600003315373377
9.57 0.00600812163894475
9.58 0.00601645228926716
9.59 0.00602502348519317
9.6 0.00603383363054448
9.61 0.00604288115175922
9.62 0.00605216449718277
9.63 0.0060616821363618
9.64 0.00607143255934203
9.65 0.00608141427596891
9.66 0.00609162581519152
9.67 0.00610206572436914
9.68 0.00611273256858063
9.69 0.00612362492993624
9.7 0.00613474140689194
9.71 0.00614608061356573
9.72 0.00615764117905631
9.73 0.00616942174676358
9.74 0.00618142097371095
9.75 0.00619363752986936
9.76 0.0062060700974831
9.77 0.00621871737039684
9.78 0.00623157805338424
9.79 0.00624465086147791
9.8 0.00625793451930039
9.81 0.00627142776039658
9.82 0.00628512932656696
9.83 0.00629903796720205
9.84 0.00631315243861775
9.85 0.0063274715033915
9.86 0.00634199392969962
9.87 0.00635671849065513
9.88 0.00637164396364665
9.89 0.00638676912967799
9.9 0.00640209277270862
9.91 0.00641761367899475
9.92 0.00643333063643147
9.93 0.00644924243389547
9.94 0.00646534786058864
9.95 0.00648164570538262
9.96 0.00649813475616408
9.97 0.00651481379918105
9.98 0.00653168161839006
9.99 0.00654873699480456
10 0.00656597870584408
10.01 0.00658340552468484
10.02 0.00660101621961142
10.03 0.00661880955336974
10.04 0.0066367842825216
10.05 0.0066549391568004
10.06 0.00667327291846871
10.07 0.00669178430167753
10.08 0.00671047203182712
10.09 0.00672933482492997
10.1 0.00674837138697593
10.11 0.00676758041329914
10.12 0.00678696058794767
10.13 0.00680651058305545
10.14 0.00682622905821686
10.15 0.00684611465986407
10.16 0.00686616602064738
10.17 0.00688638175881861
10.18 0.00690676047761779
10.19 0.00692730076466334
10.2 0.00694800119134581
10.21 0.00696886031222547
10.22 0.00698987666443404
10.23 0.00701104876708043
10.24 0.00703237512066105
10.25 0.0070538542064747
10.26 0.00707548448604227
10.27 0.0070972644005315
10.28 0.00711919237018707
10.29 0.00714126679376604
10.3 0.00716348604797917
10.31 0.00718584848693804
10.32 0.00720835244160845
10.33 0.00723099621927022
10.34 0.00725377810298353
10.35 0.00727669635106243
10.36 0.00729974919655505
10.37 0.00732293484673176
10.38 0.00734625148258048
10.39 0.00736969725831015
10.4 0.00739327030086225
10.41 0.0074169687094309
10.42 0.00744079055499128
10.43 0.00746473387983738
10.44 0.00748879669712863
10.45 0.00751297699044622
10.46 0.00753727271335886
10.47 0.00756168178899894
10.48 0.00758620210964843
10.49 0.00761083153633584
10.5 0.00763556789844348
10.51 0.0076604089933261
10.52 0.00768535258594079
10.53 0.0077103964084884
10.54 0.00773553816006689
10.55 0.00776077550633674
10.56 0.0077861060791989
10.57 0.00781152747648534
10.58 0.0078370372616624
10.59 0.00786263296354756
10.6 0.00788831207603956
10.61 0.00791407205786224
10.62 0.00793991033232243
10.63 0.00796582428708196
10.64 0.00799181127394436
10.65 0.00801786860865611
10.66 0.00804399357072316
10.67 0.00807018340324233
10.68 0.00809643531274864
10.69 0.0081227464690781
10.7 0.0081491140052466
10.71 0.00817553501734496
10.72 0.00820200656445061
10.73 0.00822852566855568
10.74 0.00825508931451237
10.75 0.0082816944499951
10.76 0.00830833798548028
10.77 0.00833501679424351
10.78 0.00836172771237468
10.79 0.00838846753881087
10.8 0.00841523303538764
10.81 0.00844202092690861
10.82 0.00846882790123354
10.83 0.00849565060938534
10.84 0.00852248566567567
10.85 0.00854932964785017
10.86 0.00857617909725226
10.87 0.00860303051900709
10.88 0.00862988038222444
10.89 0.00865672512022171
10.9 0.00868356113076664
10.91 0.00871038477633996
10.92 0.00873719238441831
10.93 0.00876398024777723
10.94 0.00879074462481439
10.95 0.00881748173989368
10.96 0.00884418778370921
10.97 0.00887085891367043
10.98 0.00889749125430775
10.99 0.00892408089769879
11 0.00895062390391561
11.01 0.00897711630149275
11.02 0.00900355408791608
11.03 0.00902993323013275
11.04 0.00905624966508194
11.05 0.00908249930024661
11.06 0.00910867801422638
11.07 0.00913478165733102
11.08 0.00916080605219532
11.09 0.00918674699441447
11.1 0.00921260025320065
11.11 0.0092383615720603
11.12 0.00926402666949208
11.13 0.00928959123970581
11.14 0.00931505095336175
11.15 0.00934040145833064
11.16 0.00936563838047406
11.17 0.00939075732444516
11.18 0.00941575387450964
11.19 0.00944062359538667
11.2 0.00946536203310998
11.21 0.00948996471590853
11.22 0.00951442715510672
11.23 0.00953874484604438
11.24 0.00956291326901555
11.25 0.00958692789022677
11.26 0.00961078416277373
11.27 0.00963447752763685
11.28 0.0096580034146951
11.29 0.00968135724375797
11.3 0.00970453442561516
11.31 0.00972753036310421
11.32 0.00975034045219492
11.33 0.00977296008309121
11.34 0.00979538464134944
11.35 0.00981760950901323
11.36 0.00983963006576418
11.37 0.00986144169008854
11.38 0.00988303976045888
11.39 0.00990441965653115
11.4 0.00992557676035593
11.41 0.00994650645760433
11.42 0.00996720413880727
11.43 0.00998766520060854
11.44 0.0100078850470306
11.45 0.010027859090753
11.46 0.0100475827544031
11.47 0.0100670514718579
11.48 0.0100862606895577
11.49 0.0101052058678301
11.5 0.0101238824822242
11.51 0.0101422860248547
11.52 0.0101604120057551
11.53 0.0101782559542392
11.54 0.0101958134202719
11.55 0.0102130799758461
11.56 0.010230051216368
11.57 0.0102467227620476
11.58 0.010263090259297
11.59 0.0102791493821317
11.6 0.010294895833579
11.61 0.0103103253470886
11.62 0.0103254336879483
11.63 0.010340216654702
11.64 0.0103546700805698
11.65 0.0103687898348704
11.66 0.0103825718244444
11.67 0.0103960119950778
11.68 0.0104091063329257
11.69 0.0104218508659349
11.7 0.0104342416652651
11.71 0.0104462748467081
11.72 0.0104579465721036
11.73 0.0104692530507521
11.74 0.0104801905408234
11.75 0.0104907553507604
11.76 0.0105009438406766
11.77 0.0105107524237486
11.78 0.0105201775676005
11.79 0.010529215795681
11.8 0.0105378636886327
11.81 0.0105461178856518
11.82 0.0105539750858378
11.83 0.0105614320495336
11.84 0.0105684855996534
11.85 0.0105751326229998
11.86 0.0105813700715675
11.87 0.0105871949638337
11.88 0.0105926043860355
11.89 0.0105975954934313
11.9 0.0106021655115475
11.91 0.0106063117374092
11.92 0.0106100315407534
11.93 0.0106133223652258
11.94 0.0106161817295584
11.95 0.0106186072287284
11.96 0.0106205965350981
11.97 0.010622147399534
11.98 0.0106232576525052
11.99 0.0106239252051599
12 0.0106241480503801
};
\end{axis}

\end{tikzpicture}
\\
	% This file was created by tikzplotlib v0.9.8.
\begin{tikzpicture}

\begin{axis}[
height=\figureheight,
scaled y ticks=false,
tick align=outside,
tick pos=left,
width=\figurewidth,
x grid style={white!69.0196078431373!black},
xlabel={Time [\si{\second}]},
xmajorgrids,
xmin=0, xmax=6,
xtick style={color=black},
xticklabel style={align=center},
y grid style={white!69.0196078431373!black},
ylabel={Speed [\si{\meter\per\second}]},
ymajorgrids,
ymin=9.55695328945044, ymax=24.6877641290738,
ytick style={color=black},
yticklabel style={/pgf/number format/fixed,/pgf/number format/precision=3}
]
\addplot [very thick, black]
table {%
0 24
0.01 24
0.02 24
0.03 24
0.04 24
0.05 24
0.06 24
0.07 24
0.08 24
0.09 24
0.1 24
0.11 24
0.12 24
0.13 24
0.14 24
0.15 24
0.16 24
0.17 24
0.18 24
0.19 24
0.2 24
0.21 24
0.22 24
0.23 24
0.24 24
0.25 24
0.26 24
0.27 24
0.28 24
0.29 24
0.3 24
0.31 24
0.32 24
0.33 24
0.34 24
0.35 24
0.36 24
0.37 24
0.38 24
0.39 24
0.4 24
0.41 24
0.42 24
0.43 24
0.44 24
0.45 24
0.46 24
0.47 24
0.48 24
0.49 24
0.5 24
0.51 24
0.52 24
0.53 24
0.54 24
0.55 24
0.56 24
0.57 24
0.58 24
0.59 24
0.6 24
0.61 24
0.62 24
0.63 24
0.64 24
0.65 24
0.66 24
0.67 24
0.68 24
0.69 24
0.7 24
0.71 24
0.72 24
0.73 24
0.74 24
0.75 24
0.76 24
0.77 24
0.78 24
0.79 24
0.8 24
0.81 24
0.82 24
0.83 24
0.84 24
0.85 24
0.86 24
0.87 24
0.88 24
0.89 24
0.9 24
0.91 24
0.92 24
0.93 24
0.94 24
0.95 24
0.96 24
0.97 24
0.98 24
0.99 24
1 24
1.01 24
1.02 24
1.03 24
1.04 24
1.05 24
1.06 24
1.07 24
1.08 24
1.09 24
1.1 24
1.11 24
1.12 24
1.13 24
1.14 24
1.15 24
1.16 24
1.17 24
1.18 24
1.19 24
1.2 24
1.21 24
1.22 24
1.23 24
1.24 24
1.25 24
1.26 24
1.27 24
1.28 24
1.29 24
1.3 24
1.31 24
1.32 24
1.33 24
1.34 24
1.35 24
1.36 24
1.37 24
1.38 24
1.39 24
1.4 24
1.41 24
1.42 24
1.43 24
1.44 24
1.45 24
1.46 24
1.47 24
1.48 24
1.49 24
1.5 24
1.51 24
1.52 24
1.53 24
1.54 24
1.55 24
1.56 24
1.57 24
1.58 24
1.59 24
1.6 24
1.61 24
1.62 24
1.63 24
1.64 24
1.65 24
1.66 24
1.67 24
1.68 24
1.69 24
1.7 24
1.71 24
1.72 24
1.73 24
1.74 24
1.75 24
1.76 24
1.77 24
1.78 24
1.79 24
1.8 24
1.81 24
1.82 24
1.83 24
1.84 24
1.85 24
1.86 24
1.87 24
1.88 24
1.89 24
1.9 24
1.91 24
1.92 24
1.93 24
1.94 24
1.95 24
1.96 24
1.97 24
1.98 24
1.99 24
2 24
2.01 24
2.02 24
2.03 24
2.04 24
2.05 24
2.06 24
2.07 24
2.08 24
2.09 24
2.1 24
2.11 24
2.12 24
2.13 24
2.14 24
2.15 24
2.16 24
2.17 24
2.18 24
2.19 24
2.2 24
2.21 24
2.22 24
2.23 24
2.24 24
2.25 24
2.26 24
2.27 24
2.28 24
2.29 24
2.3 24
2.31 24
2.32 24
2.33 24
2.34 24
2.35 24
2.36 24
2.37 24
2.38 24
2.39 24
2.4 24
2.41 24
2.42 24
2.43 24
2.44 24
2.45 24
2.46 24
2.47 24
2.48 24
2.49 24
2.5 24
2.51 24
2.52 24
2.53 24
2.54 24
2.55 24
2.56 24
2.57 24
2.58 24
2.59 24
2.6 24
2.61 24
2.62 24
2.63 24
2.64 24
2.65 24
2.66 24
2.67 24
2.68 24
2.69 24
2.7 24
2.71 24
2.72 24
2.73 24
2.74 24
2.75 24
2.76 24
2.77 24
2.78 24
2.79 24
2.8 24
2.81 24
2.82 24
2.83 24
2.84 24
2.85 24
2.86 24
2.87 24
2.88 24
2.89 24
2.9 24
2.91 24
2.92 24
2.93 24
2.94 24
2.95 24
2.96 24
2.97 24
2.98 24
2.99 24
3 24
3.01 24
3.02 24
3.03 24
3.04 24
3.05 24
3.06 24
3.07 24
3.08 24
3.09 24
3.1 24
3.11 24
3.12 24
3.13 24
3.14 24
3.15 24
3.16 24
3.17 24
3.18 24
3.19 24
3.2 24
3.21 24
3.22 24
3.23 24
3.24 24
3.25 24
3.26 24
3.27 24
3.28 24
3.29 24
3.3 24
3.31 24
3.32 24
3.33 24
3.34 24
3.35 24
3.36 24
3.37 24
3.38 24
3.39 24
3.4 24
3.41 24
3.42 24
3.43 24
3.44 24
3.45 24
3.46 24
3.47 24
3.48 24
3.49 24
3.5 24
3.51 24
3.52 24
3.53 24
3.54 24
3.55 24
3.56 24
3.57 24
3.58 24
3.59 24
3.6 24
3.61 24
3.62 24
3.63 24
3.64 24
3.65 24
3.66 24
3.67 24
3.68 24
3.69 24
3.7 24
3.71 24
3.72 24
3.73 24
3.74 24
3.75 24
3.76 24
3.77 24
3.78 24
3.79 24
3.8 24
3.81 24
3.82 24
3.83 24
3.84 24
3.85 24
3.86 24
3.87 24
3.88 24
3.89 24
3.9 24
3.91 24
3.92 24
3.93 24
3.94 24
3.95 24
3.96 24
3.97 24
3.98 24
3.99 24
4 24
4.01 23.9997532611583
4.02 23.9990130598533
4.03 23.997779441744
4.04 23.9960524829259
4.05 23.9938322899258
4.06 23.9911189996958
4.07 23.9879127796043
4.08 23.9842138274262
4.09 23.9800223713302
4.1 23.975338669865
4.11 23.9701630119435
4.12 23.9644957168246
4.13 23.9583371340938
4.14 23.9516876436414
4.15 23.9445476556394
4.16 23.9369176105158
4.17 23.9287979789278
4.18 23.9201892617325
4.19 23.911091989956
4.2 23.9015067247611
4.21 23.891434057412
4.22 23.8808746092382
4.23 23.8698290315963
4.24 23.8582980058295
4.25 23.8462822432258
4.26 23.8337824849741
4.27 23.8207995021183
4.28 23.80733409551
4.29 23.7933870957588
4.3 23.7789593631814
4.31 23.7640517877483
4.32 23.748665289029
4.33 23.7328008161353
4.34 23.7164593476624
4.35 23.6996418916292
4.36 23.6823494854155
4.37 23.6645831956986
4.38 23.6463441183866
4.39 23.627633378552
4.4 23.6084521303612
4.41 23.588801557004
4.42 23.5686828706204
4.43 23.5480973122256
4.44 23.5270461516338
4.45 23.5055306873799
4.46 23.4835522466389
4.47 23.4611121851448
4.48 23.438211887106
4.49 23.414852765121
4.5 23.3910362600903
4.51 23.3667638411282
4.52 23.3420370054718
4.53 23.3168572783889
4.54 23.2912262130836
4.55 23.2651453906007
4.56 23.2386164197282
4.57 23.211640936898
4.58 23.1842206060849
4.59 23.1563571187042
4.6 23.1280521935069
4.61 23.0993075764743
4.62 23.0701250407095
4.63 23.0405063863291
4.64 23.0104534403509
4.65 22.9799680565824
4.66 22.9490521155055
4.67 22.9177075241612
4.68 22.8859362160315
4.69 22.8537401509204
4.7 22.8211213148327
4.71 22.788081719852
4.72 22.7546234040161
4.73 22.7207484311915
4.74 22.6864588909462
4.75 22.6517568984204
4.76 22.6166445941965
4.77 22.5811241441669
4.78 22.5451977394002
4.79 22.5088675960063
4.8 22.4721359549996
4.81 22.4350050821607
4.82 22.3974772678967
4.83 22.3595548271001
4.84 22.3212400990055
4.85 22.282535447046
4.86 22.2434432587066
4.87 22.2039659453779
4.88 22.1641059422063
4.89 22.1238657079447
4.9 22.0832477248002
4.91 22.0422544982815
4.92 22.0008885570437
4.93 21.959152452733
4.94 21.9170487598289
4.95 21.8745800754855
4.96 21.8317490193713
4.97 21.7885582335076
4.98 21.7450103821055
4.99 21.7011081514016
5 21.6568542494924
5.01 21.6122514061668
5.02 21.5673023727385
5.03 21.5220099218755
5.04 21.4763768474295
5.05 21.4304059642635
5.06 21.3841001080782
5.07 21.3374621352368
5.08 21.2904949225892
5.09 21.2432013672944
5.1 21.1955843866415
5.11 21.1476469178702
5.12 21.0993919179895
5.13 21.050822363595
5.14 21.0019412506856
5.15 20.9527515944787
5.16 20.9032564292238
5.17 20.853458808016
5.18 20.8033618026071
5.19 20.7529685032163
5.2 20.7022820183398
5.21 20.6513054745586
5.22 20.6000420163462
5.23 20.5484948058741
5.24 20.496667022817
5.25 20.4445618641568
5.26 20.3921825439851
5.27 20.3395322933049
5.28 20.286614359832
5.29 20.2334320077935
5.3 20.1799885177276
5.31 20.1262871862804
5.32 20.072331326003
5.33 20.018124265147
5.34 19.9636693474593
5.35 19.9089699319756
5.36 19.8540293928137
5.37 19.7988511189648
5.38 19.7434385140846
5.39 19.6877949962837
5.4 19.6319239979164
5.41 19.575828965369
5.42 19.5195133588473
5.43 19.4629806521633
5.44 19.4062343325206
5.45 19.3492779002994
5.46 19.2921148688409
5.47 19.23474876423
5.48 19.1771831250782
5.49 19.1194215023055
5.5 19.0614674589207
5.51 19.0033245698023
5.52 18.9449964214774
5.53 18.8864866119011
5.54 18.8277987502341
5.55 18.7689364566199
5.56 18.7099033619623
5.57 18.6507031077006
5.58 18.5913393455852
5.59 18.5318157374527
5.6 18.4721359549996
5.61 18.412303679556
5.62 18.3523226018584
5.63 18.2921964218224
5.64 18.2319288483138
5.65 18.1715235989206
5.66 18.110984399723
5.67 18.050314985064
5.68 17.9895190973188
5.69 17.9286004866643
5.7 17.8675629108472
5.71 17.8064101349528
5.72 17.7451459311723
5.73 17.6837740785704
5.74 17.6222983628521
5.75 17.560722576129
5.76 17.4990505166858
5.77 17.4372859887455
5.78 17.3754328022353
5.79 17.3134947725509
5.8 17.2514757203218
5.81 17.1893794711754
5.82 17.1272098555007
5.83 17.0649707082124
5.84 17.0026658685144
5.85 16.9402991796627
5.86 16.8778744887284
5.87 16.8153956463604
5.88 16.7528665065481
5.89 16.6902909263834
5.9 16.6276727658228
5.91 16.5650158874493
5.92 16.5023241562345
5.93 16.4396014392996
5.94 16.3768516056771
5.95 16.3140785260725
5.96 16.251286072625
5.97 16.1884781186689
5.98 16.1256585384946
5.99 16.0628312071097
6 16
};
\addplot [very thick, black, dashed]
table {%
0 20
0.01 20
0.02 20
0.03 20
0.04 20
0.05 20
0.06 20
0.07 20
0.08 20
0.09 20
0.1 20
0.11 20
0.12 20
0.13 20
0.14 20
0.15 20
0.16 20
0.17 20
0.18 20
0.19 20
0.2 20
0.21 20
0.22 20
0.23 20
0.24 20
0.25 20
0.26 20
0.27 20
0.28 20
0.29 20
0.3 20
0.31 20
0.32 20
0.33 20
0.34 20
0.35 20
0.36 20
0.37 20
0.38 20
0.39 20
0.4 20
0.41 20
0.42 20
0.43 20
0.44 20
0.45 20
0.46 20
0.47 20
0.48 20
0.49 20
0.5 20
0.51 20
0.52 20
0.53 20
0.54 20
0.55 20
0.56 20
0.57 20
0.58 20
0.59 20
0.6 20
0.61 20
0.62 20
0.63 20
0.64 20
0.65 20
0.66 20
0.67 20
0.68 20
0.69 20
0.7 20
0.71 20
0.72 20
0.73 20
0.74 20
0.75 20
0.76 20
0.77 20
0.78 20
0.79 20
0.8 20
0.81 20
0.82 20
0.83 20
0.84 20
0.85 20
0.86 20
0.87 20
0.88 20
0.89 20
0.9 20
0.91 20
0.92 20
0.93 20
0.94 20
0.95 20
0.96 20
0.97 20
0.98 20
0.99 20
1 20
1.01 20
1.02 20
1.03 20
1.04 20
1.05 20
1.06 20
1.07 20
1.08 20
1.09 20
1.1 20
1.11 20
1.12 20
1.13 20
1.14 20
1.15 20
1.16 20
1.17 20
1.18 20
1.19 20
1.2 20
1.21 20
1.22 20
1.23 20
1.24 20
1.25 20
1.26 20
1.27 20
1.28 20
1.29 20
1.3 20
1.31 20
1.32 20
1.33 20
1.34 20
1.35 20
1.36 20
1.37 20
1.38 20
1.39 20
1.4 20
1.41 20
1.42 20
1.43 20
1.44 20
1.45 20
1.46 20
1.47 20
1.48 20
1.49 20
1.5 20
1.51 20
1.52 20
1.53 20
1.54 20
1.55 20
1.56 20
1.57 20
1.58 20
1.59 20
1.6 20
1.61 20
1.62 20
1.63 20
1.64 20
1.65 20
1.66 20
1.67 20
1.68 20
1.69 20
1.7 20
1.71 20
1.72 20
1.73 20
1.74 20
1.75 20
1.76 20
1.77 20
1.78 20
1.79 20
1.8 20
1.81 20
1.82 20
1.83 20
1.84 20
1.85 20
1.86 20
1.87 20
1.88 20
1.89 20
1.9 20
1.91 20
1.92 20
1.93 20
1.94 20
1.95 20
1.96 20
1.97 20
1.98 20
1.99 20
2 20
2.01 20
2.02 20
2.03 20
2.04 20
2.05 20
2.06 20
2.07 20
2.08 20
2.09 20
2.1 20
2.11 20
2.12 20
2.13 20
2.14 20
2.15 20
2.16 20
2.17 20
2.18 20
2.19 20
2.2 20
2.21 20
2.22 20
2.23 20
2.24 20
2.25 20
2.26 20
2.27 20
2.28 20
2.29 20
2.3 20
2.31 20
2.32 20
2.33 20
2.34 20
2.35 20
2.36 20
2.37 20
2.38 20
2.39 20
2.4 20
2.41 20
2.42 20
2.43 20
2.44 20
2.45 20
2.46 20
2.47 20
2.48 20
2.49 20
2.5 20
2.51 20
2.52 20
2.53 20
2.54 20
2.55 20
2.56 20
2.57 20
2.58 20
2.59 20
2.6 20
2.61 20
2.62 20
2.63 20
2.64 20
2.65 20
2.66 20
2.67 20
2.68 20
2.69 20
2.7 20
2.71 20
2.72 20
2.73 20
2.74 20
2.75 20
2.76 20
2.77 20
2.78 20
2.79 20
2.8 20
2.81 20
2.82 20
2.83 20
2.84 20
2.85 20
2.86 20
2.87 20
2.88 20
2.89 20
2.9 20
2.91 20
2.92 20
2.93 20
2.94 20
2.95 20
2.96 20
2.97 20
2.98 20
2.99 20
3 20
3.01 19.9997779355447
3.02 19.999111761904
3.03 19.9980015382513
3.04 19.9964473632029
3.05 19.9944493748099
3.06 19.9920077505449
3.07 19.9891227072872
3.08 19.9857945013031
3.09 19.982023428223
3.1 19.9778098230154
3.11 19.9731540599572
3.12 19.9680565526
3.13 19.9625177537341
3.14 19.9565381553475
3.15 19.9501182885828
3.16 19.9432587236896
3.17 19.9359600699741
3.18 19.928222975745
3.19 19.9200481282557
3.2 19.9114362536434
3.21 19.9023881168647
3.22 19.8929045216274
3.23 19.8829863103191
3.24 19.8726343639329
3.25 19.8618496019884
3.26 19.8506329824505
3.27 19.8389855016443
3.28 19.8269081941664
3.29 19.8144021327929
3.3 19.8014684283847
3.31 19.7881082297881
3.32 19.7743227237332
3.33 19.7601131347284
3.34 19.7454807249515
3.35 19.7304267941377
3.36 19.7149526794643
3.37 19.6990597554316
3.38 19.682749433741
3.39 19.6660231631695
3.4 19.6488824294413
3.41 19.6313287550953
3.42 19.6133636993506
3.43 19.5949888579671
3.44 19.5762058631046
3.45 19.5570163831772
3.46 19.5374221227056
3.47 19.5174248221652
3.48 19.4970262578319
3.49 19.4762282416241
3.5 19.4550326209418
3.51 19.4334412785028
3.52 19.4114561321748
3.53 19.3890791348056
3.54 19.3663122740496
3.55 19.343157572191
3.56 19.3196170859642
3.57 19.2956929063714
3.58 19.2713871584965
3.59 19.2467020013166
3.6 19.2216396275101
3.61 19.1962022632619
3.62 19.1703921680659
3.63 19.1442116345238
3.64 19.1176629881421
3.65 19.0907485871251
3.66 19.0634708221655
3.67 19.035832116232
3.68 19.0078349243544
3.69 18.9794817334051
3.7 18.9507750618785
3.71 18.9217174596671
3.72 18.8923115078351
3.73 18.8625598183893
3.74 18.8324650340467
3.75 18.8020298280002
3.76 18.7712569036805
3.77 18.7401489945169
3.78 18.7087088636937
3.79 18.6769393039051
3.8 18.6448431371071
3.81 18.6124232142667
3.82 18.5796824151092
3.83 18.5466236478614
3.84 18.5132498489942
3.85 18.4795639829616
3.86 18.4455690419367
3.87 18.411268045547
3.88 18.3766640406051
3.89 18.341760100839
3.9 18.3065593266183
3.91 18.2710648446793
3.92 18.2352798078472
3.93 18.1992073947559
3.94 18.1628508095656
3.95 18.1262132816785
3.96 18.0892980654517
3.97 18.052108439908
3.98 18.0146477084451
3.99 17.9769191985418
4 17.9389262614624
4.01 17.9006722719592
4.02 17.862160627973
4.03 17.8233947503304
4.04 17.7843780824409
4.05 17.7451140899907
4.06 17.7056062606344
4.07 17.6658581036859
4.08 17.6258731498065
4.09 17.5856549506908
4.1 17.5452070787519
4.11 17.5045331268035
4.12 17.4636367077415
4.13 17.422521454222
4.14 17.3811910183397
4.15 17.3396490713029
4.16 17.2978993031074
4.17 17.2559454222092
4.18 17.2137911551945
4.19 17.171440246449
4.2 17.1288964578254
4.21 17.0861635683088
4.22 17.0432453736817
4.23 17.0001456861863
4.24 16.956868334186
4.25 16.9134171618254
4.26 16.869796028689
4.27 16.8260088094579
4.28 16.7820593935663
4.29 16.7379516848552
4.3 16.6936896012265
4.31 16.6492770742943
4.32 16.604718049036
4.33 16.5600164834421
4.34 16.5151763481639
4.35 16.4702016261615
4.36 16.4250963123499
4.37 16.3798644132437
4.38 16.3345099466019
4.39 16.2890369410703
4.4 16.2434494358243
4.41 16.1977514802096
4.42 16.151947133383
4.43 16.1060404639512
4.44 16.0600355496103
4.45 16.0139364767826
4.46 15.9677473402543
4.47 15.9214722428117
4.48 15.8751152948764
4.49 15.8286806141406
4.5 15.7821723252012
4.51 15.7355945591932
4.52 15.6889514534232
4.53 15.6422471510015
4.54 15.5954858004743
4.55 15.5486715554552
4.56 15.5018085742561
4.57 15.4549010195178
4.58 15.4079530578408
4.59 15.3609688594143
4.6 15.3139525976466
4.61 15.2669084487938
4.62 15.2198405915893
4.63 15.1727532068724
4.64 15.1256504772167
4.65 15.0785365865591
4.66 15.0314157198278
4.67 14.9842920625706
4.68 14.9371698005832
4.69 14.8900531195375
4.7 14.8429462046094
4.71 14.7958532401075
4.72 14.7487784091012
4.73 14.7017258930491
4.74 14.654699871428
4.75 14.6077045213608
4.76 14.5607440172463
4.77 14.513822530388
4.78 14.4669442286237
4.79 14.4201132759552
4.8 14.3733338321785
4.81 14.3266100525142
4.82 14.2799460872387
4.83 14.2333460813152
4.84 14.1868141740256
4.85 14.1403544986029
4.86 14.0939711818643
4.87 14.0476683438441
4.88 14.001450097428
4.89 13.9553205479879
4.9 13.9092837930173
4.91 13.8633439217668
4.92 13.8175050148814
4.93 13.7717711440379
4.94 13.7261463715831
4.95 13.6806347501731
4.96 13.6352403224134
4.97 13.5899671204994
4.98 13.5448191658586
4.99 13.4998004687936
5 13.4549150281253
5.01 13.4101668308379
5.02 13.3655598517253
5.03 13.3210980530371
5.04 13.2767853841274
5.05 13.2326257811037
5.06 13.1886231664773
5.07 13.1447814488147
5.08 13.101104522391
5.09 13.0575962668432
5.1 13.0142605468261
5.11 12.971101211669
5.12 12.9281220950336
5.13 12.8853270145735
5.14 12.8427197715952
5.15 12.8003041507204
5.16 12.7580839195498
5.17 12.7160628283285
5.18 12.6742446096127
5.19 12.6326329779384
5.2 12.5912316294914
5.21 12.5500442417788
5.22 12.5090744733025
5.23 12.4683259632343
5.24 12.4278023310925
5.25 12.3875071764203
5.26 12.3474440784663
5.27 12.3076165958667
5.28 12.2680282663287
5.29 12.2286826063165
5.3 12.1895831107393
5.31 12.1507332526404
5.32 12.1121364828887
5.33 12.0737962298724
5.34 12.0357158991947
5.35 11.9978988733706
5.36 11.960348511527
5.37 11.9230681491041
5.38 11.8860610975594
5.39 11.8493306440732
5.4 11.8128800512565
5.41 11.776712556862
5.42 11.7408313734956
5.43 11.7052396883314
5.44 11.6699406628287
5.45 11.6349374324511
5.46 11.6002331063879
5.47 11.565830767278
5.48 11.531733470936
5.49 11.497944246081
5.5 11.4644660940673
5.51 11.4313019886179
5.52 11.3984548755605
5.53 11.3659276725655
5.54 11.3337232688872
5.55 11.301844525107
5.56 11.2702942728791
5.57 11.2390753146794
5.58 11.2081904235564
5.59 11.1776423428845
5.6 11.1474337861211
5.61 11.1175674365646
5.62 11.0880459471171
5.63 11.0588719400478
5.64 11.0300480067608
5.65 11.0015767075645
5.66 10.9734605714444
5.67 10.9457020958381
5.68 10.9183037464141
5.69 10.891267956852
5.7 10.8645971286272
5.71 10.8382936307967
5.72 10.8123597997893
5.73 10.7867979391978
5.74 10.7616103195746
5.75 10.7367991782295
5.76 10.7123667190317
5.77 10.6883151122135
5.78 10.6646464941775
5.79 10.6413629673075
5.8 10.6184665997807
5.81 10.595959425385
5.82 10.5738434433377
5.83 10.5521206181083
5.84 10.5307928792437
5.85 10.5098621211969
5.86 10.489330203159
5.87 10.4691989488936
5.88 10.449470146575
5.89 10.4301455486297
5.9 10.4112268715801
5.91 10.3927157958925
5.92 10.3746139658277
5.93 10.3569229892949
5.94 10.3396444377089
5.95 10.3227798458507
5.96 10.3063307117306
5.97 10.290298496456
5.98 10.274684624101
5.99 10.2594904815798
6 10.2447174185242
};
\end{axis}

\begin{axis}[
axis y line=right,
height=\figureheight,
scaled y ticks=false,
tick align=outside,
width=\figurewidth,
x grid style={white!69.0196078431373!black},
xmin=0, xmax=6,
xtick pos=left,
xtick style={color=black},
xticklabel style={align=center},
y grid style={white!69.0196078431373!black},
ylabel={Distance [\si{\meter}]},
ymin=-1.77218684219822, ymax=58.7510565162952,
ytick pos=right,
ytick style={color=black},
yticklabel style={/pgf/number format/fixed,/pgf/number format/precision=3},
yticklabel style={anchor=west}
]
\addplot [very thick, gray, dotted]
table {%
0 31.5553
0.01 31.5153
0.02 31.4753
0.03 31.4353
0.04 31.3953
0.05 31.3553
0.06 31.3153
0.07 31.2753
0.08 31.2353
0.09 31.1953
0.1 31.1553
0.11 31.1153
0.12 31.0753
0.13 31.0353
0.14 30.9953
0.15 30.9553
0.16 30.9153
0.17 30.8753
0.18 30.8353
0.19 30.7953
0.2 30.7553
0.21 30.7153
0.22 30.6753
0.23 30.6353
0.24 30.5953
0.25 30.5553
0.26 30.5153
0.27 30.4753
0.28 30.4353
0.29 30.3953
0.3 30.3553
0.31 30.3153
0.32 30.2753
0.33 30.2353
0.34 30.1953
0.35 30.1553
0.36 30.1153
0.37 30.0753
0.38 30.0353
0.39 29.9953
0.4 29.9553
0.41 29.9153
0.42 29.8753
0.43 29.8353
0.44 29.7953
0.45 29.7553
0.46 29.7153
0.47 29.6753
0.48 29.6353
0.49 29.5953
0.5 29.5553
0.51 29.5153
0.52 29.4753
0.53 29.4353
0.54 29.3953
0.55 29.3553
0.56 29.3153
0.57 29.2753
0.58 29.2353
0.59 29.1953
0.6 29.1553
0.61 29.1153
0.62 29.0753
0.63 29.0353
0.64 28.9953
0.65 28.9553
0.66 28.9153
0.67 28.8753
0.68 28.8353
0.69 28.7953
0.7 28.7553
0.71 28.7153
0.72 28.6753
0.73 28.6353
0.74 28.5953
0.75 28.5553
0.76 28.5153
0.77 28.4753
0.78 28.4353
0.79 28.3953
0.8 28.3553
0.81 28.3153
0.82 28.2753
0.83 28.2353
0.84 28.1953
0.85 28.1553
0.86 28.1153
0.87 28.0753
0.88 28.0353
0.89 27.9953
0.9 27.9553
0.91 27.9153
0.92 27.8753
0.93 27.8353
0.94 27.7953
0.95 27.7553
0.96 27.7153
0.97 27.6753
0.98 27.6353
0.99 27.5953
1 27.5553
1.01 27.5153
1.02 27.4753
1.03 27.4353
1.04 27.3953
1.05 27.3553
1.06 27.3153
1.07 27.2753
1.08 27.2353
1.09 27.1953
1.1 27.1553
1.11 27.1153
1.12 27.0753
1.13 27.0353
1.14 26.9953
1.15 26.9553
1.16 26.9153
1.17 26.8753
1.18 26.8353
1.19 26.7953
1.2 26.7553
1.21 26.7153
1.22 26.6753
1.23 26.6353
1.24 26.5953
1.25 26.5553
1.26 26.5153
1.27 26.4753
1.28 26.4353
1.29 26.3953
1.3 26.3553
1.31 26.3153
1.32 26.2753
1.33 26.2353
1.34 26.1953
1.35 26.1553
1.36 26.1153
1.37 26.0753
1.38 26.0353
1.39 25.9953
1.4 25.9553
1.41 25.9153
1.42 25.8753
1.43 25.8353
1.44 25.7953
1.45 25.7553
1.46 25.7153
1.47 25.6753
1.48 25.6353
1.49 25.5953
1.5 25.5553
1.51 25.5153
1.52 25.4753
1.53 25.4353
1.54 25.3953
1.55 25.3553
1.56 25.3153
1.57 25.2753
1.58 25.2353
1.59 25.1953
1.6 25.1553
1.61 25.1153
1.62 25.0753
1.63 25.0353
1.64 24.9953
1.65 24.9553
1.66 24.9153
1.67 24.8753
1.68 24.8353
1.69 24.7953
1.7 24.7553
1.71 24.7153
1.72 24.6753
1.73 24.6353
1.74 24.5953
1.75 24.5553
1.76 24.5153
1.77 24.4753
1.78 24.4353
1.79 24.3953
1.8 24.3553
1.81 24.3153
1.82 24.2753
1.83 24.2353
1.84 24.1953
1.85 24.1553
1.86 24.1153
1.87 24.0753
1.88 24.0353
1.89 23.9953
1.9 23.9553
1.91 23.9153
1.92 23.8753
1.93 23.8353
1.94 23.7953
1.95 23.7553
1.96 23.7153
1.97 23.6753
1.98 23.6353
1.99 23.5953
2 23.5553
2.01 23.5153
2.02 23.4753
2.03 23.4353
2.04 23.3953
2.05 23.3553
2.06 23.3153
2.07 23.2753
2.08 23.2353
2.09 23.1953
2.1 23.1553
2.11 23.1153
2.12 23.0753
2.13 23.0353
2.14 22.9953
2.15 22.9553
2.16 22.9153
2.17 22.8753
2.18 22.8353
2.19 22.7953
2.2 22.7553
2.21 22.7153
2.22 22.6753
2.23 22.6353
2.24 22.5953
2.25 22.5553
2.26 22.5153
2.27 22.4753
2.28 22.4353
2.29 22.3953
2.3 22.3553
2.31 22.3153
2.32 22.2753
2.33 22.2353
2.34 22.1953
2.35 22.1553
2.36 22.1153
2.37 22.0753
2.38 22.0353
2.39 21.9953
2.4 21.9553
2.41 21.9153
2.42 21.8753
2.43 21.8353
2.44 21.7953
2.45 21.7553
2.46 21.7153
2.47 21.6753
2.48 21.6353
2.49 21.5953
2.5 21.5553
2.51 21.5153
2.52 21.4753
2.53 21.4353
2.54 21.3953
2.55 21.3553
2.56 21.3153
2.57 21.2753
2.58 21.2353
2.59 21.1953
2.6 21.1553
2.61 21.1153
2.62 21.0753
2.63 21.0353
2.64 20.9953
2.65 20.9553
2.66 20.9153
2.67 20.8753
2.68 20.8353
2.69 20.7953
2.7 20.7553
2.71 20.7153
2.72 20.6753
2.73 20.6353
2.74 20.5953
2.75 20.5553
2.76 20.5153
2.77 20.4753
2.78 20.4353
2.79 20.3953
2.8 20.3553
2.81 20.3153
2.82 20.2753
2.83 20.2353
2.84 20.1953
2.85 20.1553
2.86 20.1153
2.87 20.0753
2.88 20.0353
2.89 19.9953
2.9 19.9553
2.91 19.9153
2.92 19.8753
2.93 19.8353
2.94 19.7953
2.95 19.7553
2.96 19.7153
2.97 19.6753
2.98 19.6353
2.99 19.5953
3 19.5553
3.01 19.5152977793554
3.02 19.4752888969745
3.03 19.435268912357
3.04 19.395233385989
3.05 19.3551778797371
3.06 19.3150979572426
3.07 19.2749891843155
3.08 19.2348471293285
3.09 19.1946673636107
3.1 19.1544454618409
3.11 19.1141770024404
3.12 19.0738575679664
3.13 19.0334827455038
3.14 18.9930481270572
3.15 18.9525493099431
3.16 18.91198189718
3.17 18.8713414978797
3.18 18.8306237276372
3.19 18.7898242089197
3.2 18.7489385714562
3.21 18.7079624526248
3.22 18.6668914978411
3.23 18.6257213609443
3.24 18.5844477045836
3.25 18.5430662006035
3.26 18.501572530428
3.27 18.4599623854444
3.28 18.4182314673861
3.29 18.376375488714
3.3 18.3343901729979
3.31 18.2922712552958
3.32 18.2500144825331
3.33 18.2076156138804
3.34 18.1650704211299
3.35 18.1223746890713
3.36 18.0795242158659
3.37 18.0365148134202
3.38 17.9933423077576
3.39 17.9500025393893
3.4 17.9064913636837
3.41 17.8628046512347
3.42 17.8189382882282
3.43 17.7748881768079
3.44 17.7306502354389
3.45 17.6862203992707
3.46 17.6415946204977
3.47 17.5967688687194
3.48 17.5517391312977
3.49 17.5065014137139
3.5 17.4610517399234
3.51 17.4153861527084
3.52 17.3695007140301
3.53 17.3233915053782
3.54 17.2770546281187
3.55 17.2304862038406
3.56 17.1836823747002
3.57 17.136639303764
3.58 17.0893531753489
3.59 17.0418201953621
3.6 16.9940365916372
3.61 16.9459986142698
3.62 16.8977025359505
3.63 16.8491446522957
3.64 16.8003212821771
3.65 16.7512287680484
3.66 16.70186347627
3.67 16.6522217974324
3.68 16.6023001466759
3.69 16.55209496401
3.7 16.5016027146287
3.71 16.4508198892254
3.72 16.3997430043038
3.73 16.3483686024876
3.74 16.2966932528281
3.75 16.2447135511081
3.76 16.1924261201449
3.77 16.1398276100901
3.78 16.086914698727
3.79 16.0336840917661
3.8 15.9801325231372
3.81 15.9262567552798
3.82 15.8720535794309
3.83 15.8175198159095
3.84 15.7626523143995
3.85 15.7074479542291
3.86 15.6519036446484
3.87 15.5960163251039
3.88 15.53978296551
3.89 15.4832005665184
3.9 15.4262661597845
3.91 15.3689768082313
3.92 15.3113296063098
3.93 15.2533216802574
3.94 15.194950188353
3.95 15.1362123211698
3.96 15.0771053018243
3.97 15.0176263862234
3.98 14.9577728633079
3.99 14.8975420552933
4 14.8369313179079
4.01 14.7759405080159
4.02 14.7145719836971
4.03 14.652828136783
4.04 14.5907113927781
4.05 14.5282242107788
4.06 14.4653690833882
4.07 14.402148536629
4.08 14.3385651298528
4.09 14.2746214556464
4.1 14.2103201397352
4.11 14.1456638408838
4.12 14.080655250793
4.13 14.0152970939943
4.14 13.9495921277413
4.15 13.8835431418979
4.16 13.8171529588238
4.17 13.7504244332566
4.18 13.6833604521913
4.19 13.6159639347562
4.2 13.5482378320868
4.21 13.4801851271958
4.22 13.4118088348402
4.23 13.3431120013861
4.24 13.2740977046697
4.25 13.2047690538557
4.26 13.1351291892928
4.27 13.0651812823662
4.28 12.9949285353468
4.29 12.9243741812378
4.3 12.8535214836182
4.31 12.7823737364837
4.32 12.7109342640837
4.33 12.6392064207568
4.34 12.5671935907618
4.35 12.4948991881072
4.36 12.4223266563765
4.37 12.3494794685519
4.38 12.2763611268341
4.39 12.2029751624593
4.4 12.1293251355139
4.41 12.055414634746
4.42 11.9812472773736
4.43 11.9068267088909
4.44 11.8321566028706
4.45 11.7572406607646
4.46 11.6820826117008
4.47 11.6066862122775
4.48 11.5310552463552
4.49 11.4551935248454
4.5 11.3791048854965
4.51 11.3027931926771
4.52 11.2262623371566
4.53 11.1495162358828
4.54 11.0725588317567
4.55 10.9953940934052
4.56 10.9180260149505
4.57 10.8404586157767
4.58 10.7626959402943
4.59 10.6847420577014
4.6 10.6066010617428
4.61 10.528277070466
4.62 10.4497742259747
4.63 10.3710966941802
4.64 10.2922486645488
4.65 10.2132343498486
4.66 10.1340579858918
4.67 10.0547238312759
4.68 9.97523616712144
4.69 9.89559929680761
4.7 9.81581754570538
4.71 9.73589526090793
4.72 9.65583681095878
4.73 9.57564658557736
4.74 9.49532899538217
4.75 9.41488847161158
4.76 9.33432946584208
4.77 9.25365644970429
4.78 9.17287391459653
4.79 9.09198637139601
4.8 9.01099835016781
4.81 8.92991439987134
4.82 8.84873908806476
4.83 8.76747700060691
4.84 8.68613274135711
4.85 8.60471093187268
4.86 8.52321621110426
4.87 8.44165323508892
4.88 8.36002667664114
4.89 8.27834122504157
4.9 8.19660158572374
4.91 8.11481247995859
4.92 8.03297864453697
4.93 7.95110483145002
4.94 7.86919580756756
4.95 7.78725635431444
4.96 7.70529126734486
4.97 7.62330535621478
4.98 7.54130344405231
4.99 7.45929036722623
5 7.37727097501256
5.01 7.29525012925927
5.02 7.21323270404914
5.03 7.13122358536075
5.04 7.04922767072773
5.05 6.96724986889613
5.06 6.88529509948012
5.07 6.8033682926159
5.08 6.72147438861392
5.09 6.63961833760941
5.1 6.55780509921125
5.11 6.47603964214924
5.12 6.39432694391969
5.13 6.31267199042947
5.14 6.23107977563857
5.15 6.14955530120098
5.16 6.06810357610424
5.17 5.98672961630737
5.18 5.90543844437743
5.19 5.82423508912465
5.2 5.74312458523616
5.21 5.66211197290836
5.22 5.58120229747793
5.23 5.50040060905153
5.24 5.41971196213428
5.25 5.33914141525692
5.26 5.25869403060173
5.27 5.17837487362735
5.28 5.09818901269231
5.29 5.01814151867754
5.3 4.93823746460766
5.31 4.85848192527126
5.32 4.77887997684012
5.33 4.69943669648737
5.34 4.62015716200473
5.35 4.54104645141868
5.36 4.46210964260581
5.37 4.3833518129072
5.38 4.30477803874195
5.39 4.22639339521985
5.4 4.14820295575325
5.41 4.07021179166818
5.42 3.99242497181466
5.43 3.91484756217634
5.44 3.83748462547942
5.45 3.76034122080094
5.46 3.68342240317642
5.47 3.60673322320689
5.48 3.53027872666547
5.49 3.45406395410323
5.5 3.37809394045469
5.51 3.30237371464285
5.52 3.22690829918368
5.53 3.15170270979032
5.54 3.07676195497685
5.55 3.00209103566172
5.56 2.92769494477089
5.57 2.85357866684068
5.58 2.77974717762039
5.59 2.70620544367471
5.6 2.63295842198592
5.61 2.56001105955601
5.62 2.48736829300859
5.63 2.41503504819085
5.64 2.34301623977532
5.65 2.27131677086176
5.66 2.19994153257897
5.67 2.12889540368672
5.68 2.05818325017767
5.69 1.98780992487954
5.7 1.91778026705734
5.71 1.84809910201578
5.72 1.77877124070195
5.73 1.70980147930822
5.74 1.64119459887544
5.75 1.57295536489645
5.76 1.50508852691991
5.77 1.43759881815459
5.78 1.37049095507401
5.79 1.30376963702158
5.8 1.23743954581617
5.81 1.17150534535826
5.82 1.10597168123663
5.83 1.04084318033559
5.84 0.976124450442885
5.85 0.911820079858224
5.86 0.847934637002531
5.87 0.78447267002786
5.88 0.721438706428128
5.89 0.658837252650592
5.9 0.596672793708166
5.91 0.534949792792599
5.92 0.473672690888527
5.93 0.412845906388483
5.94 0.352473834708803
5.95 0.292560847906582
5.96 0.23311129429764
5.97 0.174129498075512
5.98 0.115619758931576
5.99 0.057586351676278
6 3.35258615180578e-05
};
\end{axis}

\end{tikzpicture}

	% This file was created by tikzplotlib v0.9.8.
\begin{tikzpicture}

\begin{axis}[
height=\figureheight,
scaled y ticks=false,
tick align=outside,
tick pos=left,
width=\figurewidth,
x grid style={white!69.0196078431373!black},
xlabel={Time [\si{\second}]},
xmajorgrids,
xmin=0, xmax=6,
xtick style={color=black},
xticklabel style={align=center},
y grid style={white!69.0196078431373!black},
ylabel={Probability of collision},
ymajorgrids,
ymin=0, ymax=1,
ytick style={color=black},
yticklabel style={/pgf/number format/fixed,/pgf/number format/precision=3}
]
\addplot [very thick, gray]
table {%
0 1.91069382537989e-13
0.01 1.97397653778353e-13
0.02 2.03947969623641e-13
0.03 2.10720330073855e-13
0.04 2.17603712826531e-13
0.05 2.24820162486594e-13
0.06 2.3214763444912e-13
0.07 2.40030217923959e-13
0.08 2.47912801398797e-13
0.09 2.56239474083486e-13
0.1 2.64899213675562e-13
0.11 2.73669975570101e-13
0.12 2.82662782069565e-13
0.13 2.92099677778879e-13
0.14 3.0186964039558e-13
0.15 3.12083692222132e-13
0.16 3.22408766351145e-13
0.17 3.33066907387547e-13
0.18 3.44280159936261e-13
0.19 3.55826479392363e-13
0.2 3.68038932663239e-13
0.21 3.80362408236579e-13
0.22 3.92907928414843e-13
0.23 4.06119582407882e-13
0.24 4.19775325610772e-13
0.25 4.34097202628436e-13
0.26 4.48641124251026e-13
0.27 4.6385117968839e-13
0.28 4.7928327973068e-13
0.29 4.95492535890207e-13
0.3 5.12145881259585e-13
0.31 5.295763827462e-13
0.32 5.47561995745127e-13
0.33 5.65991697953905e-13
0.34 5.85087533977457e-13
0.35 6.04849503815785e-13
0.36 6.25610674376276e-13
0.37 6.46815934146616e-13
0.38 6.69131416941582e-13
0.39 6.91668944341473e-13
0.4 7.15094650161063e-13
0.41 7.39519556702817e-13
0.42 7.64943663966733e-13
0.43 7.90922882742962e-13
0.44 8.18012324543815e-13
0.45 8.45989944764369e-13
0.46 8.75077788009548e-13
0.47 9.04942787371965e-13
0.48 9.35918009759007e-13
0.49 9.67892432868211e-13
0.5 1.00119912360697e-12
0.51 1.03539399276542e-12
0.52 1.07092112955343e-12
0.53 1.10778053397098e-12
0.54 1.14586118371562e-12
0.55 1.18538512339228e-12
0.56 1.22613030839602e-12
0.57 1.26865185023917e-12
0.58 1.31250565971186e-12
0.59 1.35791378141903e-12
0.6 1.40498723766314e-12
0.61 1.45339296153679e-12
0.62 1.50368606455231e-12
0.63 1.55575552440723e-12
0.64 1.60971236340401e-12
0.65 1.66577862614758e-12
0.66 1.72362124573056e-12
0.67 1.78346226675785e-12
0.68 1.84530168922947e-12
0.69 1.90958360235527e-12
0.7 1.9764190284377e-12
0.71 2.045141833662e-12
0.72 2.1165291741454e-12
0.73 2.1905810498879e-12
0.74 2.26729746088949e-12
0.75 2.34656738484773e-12
0.76 2.42850184406507e-12
0.77 2.51343390544889e-12
0.78 2.60180765820905e-12
0.79 2.69306799083324e-12
0.8 2.78754797022884e-12
0.81 2.88569168560571e-12
0.82 2.98727709235891e-12
0.83 3.09219316818599e-12
0.84 3.20121706920418e-12
0.85 3.31412675080855e-12
0.86 3.43114425760405e-12
0.87 3.55226958959065e-12
0.88 3.67794683597822e-12
0.89 3.80795395216182e-12
0.9 3.94284604965378e-12
0.91 4.08251210615163e-12
0.92 4.22717416626028e-12
0.93 4.37738734149207e-12
0.94 4.53281856493959e-12
0.95 4.69368988120777e-12
0.96 4.86088946871632e-12
0.97 5.033640171348e-12
0.98 5.21316323442989e-12
0.99 5.3991255910546e-12
1 5.5916382635246e-12
1.01 5.79136738565467e-12
1.02 5.99842397974726e-12
1.03 6.21291906810484e-12
1.04 6.43529673993726e-12
1.05 6.66600108445436e-12
1.06 6.90503210165616e-12
1.07 7.1528338807525e-12
1.08 7.40996153325568e-12
1.09 7.67608199225833e-12
1.1 7.95230548078507e-12
1.11 8.2386319988359e-12
1.12 8.5353946133182e-12
1.13 8.8431484357443e-12
1.14 9.16255959992895e-12
1.15 9.49307299435986e-12
1.16 9.83646497587642e-12
1.17 1.01925134998737e-11
1.18 1.05616626555616e-11
1.19 1.09441344875449e-11
1.2 1.13412612634534e-11
1.21 1.17528209386819e-11
1.22 1.21797016916503e-11
1.23 1.26225696561733e-11
1.24 1.30818689214607e-11
1.25 1.35583766436298e-11
1.26 1.4052425889588e-11
1.27 1.45651268823599e-11
1.28 1.50967016665504e-11
1.29 1.56482604651842e-11
1.3 1.62205804343785e-11
1.31 1.6814327707948e-11
1.32 1.743027944201e-11
1.33 1.80693238149843e-11
1.34 1.87321269606855e-11
1.35 1.94199101244408e-11
1.36 2.01337835292748e-11
1.37 2.08741912643973e-11
1.38 2.16427986643453e-11
1.39 2.24399387960261e-11
1.4 2.32677210831866e-11
1.41 2.41263675704317e-11
1.42 2.50173215476934e-11
1.43 2.59421373272062e-11
1.44 2.69019251319946e-11
1.45 2.78981282519908e-11
1.46 2.89315238433119e-11
1.47 3.00049984858219e-11
1.48 3.11186632018234e-11
1.49 3.22749604819705e-11
1.5 3.34750005492879e-11
1.51 3.47206707829173e-11
1.52 3.60138585620007e-11
1.53 3.73563402433774e-11
1.54 3.87502252507943e-11
1.55 4.01970678964858e-11
1.56 4.16994216934086e-11
1.57 4.32589519760995e-11
1.58 4.48785453244227e-11
1.59 4.6559867072915e-11
1.6 4.83060258460455e-11
1.61 5.0118686978351e-11
1.62 5.20014031835103e-11
1.63 5.39565059298752e-11
1.64 5.59864377080999e-11
1.65 5.80946402095606e-11
1.66 6.02846661479362e-11
1.67 6.25586249469734e-11
1.68 6.49201803426536e-11
1.69 6.73737732270752e-11
1.7 6.99217350685899e-11
1.71 7.25683957369938e-11
1.72 7.53177520351755e-11
1.73 7.81734676991164e-11
1.74 8.11404277101246e-11
1.75 8.42221847818792e-11
1.76 8.74239569625956e-11
1.77 9.07502961666751e-11
1.78 9.42058653308209e-11
1.79 9.77965486370636e-11
1.8 1.01527009022107e-10
1.81 1.05403241690283e-10
1.82 1.09430908779018e-10
1.83 1.13616338559552e-10
1.84 1.17965304191614e-10
1.85 1.22484578035653e-10
1.86 1.2718137654133e-10
1.87 1.32062361046792e-10
1.88 1.37135414135514e-10
1.89 1.42407419190249e-10
1.9 1.47887369017496e-10
1.91 1.53582591089219e-10
1.92 1.59502633323427e-10
1.93 1.6565637750432e-10
1.94 1.72052816438395e-10
1.95 1.78702275199782e-10
1.96 1.85614967840309e-10
1.97 1.92800997389497e-10
1.98 2.00271799144502e-10
1.99 2.0803891942478e-10
2 2.16114348638996e-10
2.01 2.24510743329631e-10
2.02 2.33240871061469e-10
2.03 2.42318498600014e-10
2.04 2.51757947822284e-10
2.05 2.61573540605298e-10
2.06 2.71781153138306e-10
2.07 2.82396217521352e-10
2.08 2.93435720166713e-10
2.09 3.04917091575874e-10
2.1 3.16857984294927e-10
2.11 3.29277605182199e-10
2.12 3.4219538314062e-10
2.13 3.55632190363053e-10
2.14 3.69608788020059e-10
2.15 3.84148046705945e-10
2.16 3.99272059858902e-10
2.17 4.15005918519284e-10
2.18 4.31374380660543e-10
2.19 4.48403425501454e-10
2.2 4.66120808617632e-10
2.21 4.84554396606995e-10
2.22 5.03734831625025e-10
2.23 5.23691645604174e-10
2.24 5.4445803421288e-10
2.25 5.66067193119579e-10
2.26 5.88554316394152e-10
2.27 6.11955708329504e-10
2.28 6.36309560597681e-10
2.29 6.61655397138361e-10
2.3 6.88034296203455e-10
2.31 7.15490000580132e-10
2.32 7.44067141233984e-10
2.33 7.73812347532044e-10
2.34 8.04774580309697e-10
2.35 8.3700502084838e-10
2.36 8.7055673780867e-10
2.37 9.0548490927489e-10
2.38 9.4184726684432e-10
2.39 9.79704761761013e-10
2.4 1.01911934446974e-09
2.41 1.06015718426278e-09
2.42 1.10288655985613e-09
2.43 1.14737830347877e-09
2.44 1.19370702211796e-09
2.45 1.24195076445233e-09
2.46 1.29219013267345e-09
2.47 1.34450950373122e-09
2.48 1.39899680728917e-09
2.49 1.45574441390295e-09
2.5 1.51484813581959e-09
2.51 1.57640811515591e-09
2.52 1.64052826878702e-09
2.53 1.70731806470314e-09
2.54 1.77689063463049e-09
2.55 1.8493652165219e-09
2.56 1.92486537820002e-09
2.57 2.00352012758032e-09
2.58 2.08546435676027e-09
2.59 2.17083917508631e-09
2.6 2.2597909099531e-09
2.61 2.35247343827183e-09
2.62 2.44904618806885e-09
2.63 2.54967669199857e-09
2.64 2.65453958814277e-09
2.65 2.76381617592136e-09
2.66 2.87769696960538e-09
2.67 2.99638047707163e-09
2.68 3.12007331082498e-09
2.69 3.24899263048906e-09
2.7 3.38336358929325e-09
2.71 3.52342222065261e-09
2.72 3.66941532714549e-09
2.73 3.82159881517907e-09
2.74 3.98024246894835e-09
2.75 4.14562617567782e-09
2.76 4.31804247913448e-09
2.77 4.49779835598463e-09
2.78 4.68521266228095e-09
2.79 4.88061890902003e-09
2.8 5.08436626134312e-09
2.81 5.29681853933539e-09
2.82 5.51835588336047e-09
2.83 5.74937608632808e-09
2.84 5.99029437164944e-09
2.85 6.24154428141566e-09
2.86 6.50357923070999e-09
2.87 6.77687228556323e-09
2.88 7.06191827237745e-09
2.89 7.35923466610444e-09
2.9 7.66936092411186e-09
2.91 7.99286203889693e-09
2.92 8.33032764990804e-09
2.93 8.68237459705767e-09
2.94 9.04964692072241e-09
2.95 9.43281830423359e-09
2.96 9.83259240694423e-09
2.97 1.02497051956973e-08
2.98 1.06849258330044e-08
2.99 1.11390573431791e-08
3 1.16129417193633e-08
3.01 1.21218818316393e-08
3.02 1.26837492731013e-08
3.03 1.33037196814456e-08
3.04 1.39876189519939e-08
3.05 1.47420078366878e-08
3.06 1.55742773122469e-08
3.07 1.64927586032704e-08
3.08 1.75068489705055e-08
3.09 1.86271553737072e-08
3.1 1.98656593397573e-08
3.11 2.12359080320468e-08
3.12 2.27532284124976e-08
3.13 2.44349798173005e-08
3.14 2.63008399503661e-08
3.15 2.83731349526306e-08
3.16 3.06772205416195e-08
3.17 3.32419193282973e-08
3.18 3.61000261905886e-08
3.19 3.92888915845546e-08
3.2 4.2851094117502e-08
3.21 4.68352178151221e-08
3.22 5.12967504029405e-08
3.23 5.62991245844913e-08
3.24 6.19149197467195e-08
3.25 6.82272548457874e-08
3.26 7.53314060020216e-08
3.27 8.33366761154863e-08
3.28 9.23685682385766e-08
3.29 1.02571306781485e-07
3.3 1.14110764615205e-07
3.31 1.27177863351591e-07
3.32 1.41992527735724e-07
3.33 1.58808282191281e-07
3.34 1.77917594434973e-07
3.35 1.99658101052158e-07
3.36 2.24419846817092e-07
3.37 2.52653719834406e-07
3.38 2.84881271461757e-07
3.39 3.21706163708413e-07
3.4 3.63827501348446e-07
3.41 4.12055381926457e-07
3.42 4.67329013820184e-07
3.43 5.3073785022395e-07
3.44 6.03546231214835e-07
3.45 6.87222124984288e-07
3.46 7.83470648579865e-07
3.47 8.94273171403448e-07
3.48 1.02193292506048e-06
3.49 1.16912820569137e-06
3.5 1.33897443010955e-06
3.51 1.53509651112937e-06
3.52 1.76171326526919e-06
3.53 2.0237358355768e-06
3.54 2.32688244161672e-06
3.55 2.67781213647567e-06
3.56 3.08428069151212e-06
3.57 3.55532221629762e-06
3.58 4.10146070983775e-06
3.59 4.73495639108368e-06
3.6 5.47009243823116e-06
3.61 6.3235086272817e-06
3.62 7.31458939384755e-06
3.63 8.46591498471216e-06
3.64 9.80378571513452e-06
3.65 1.1358830861008e-05
3.66 1.31667154652515e-05
3.67 1.52689603271083e-05
3.68 1.77138927127674e-05
3.69 2.05577479029939e-05
3.7 2.3865944629109e-05
3.71 2.7714560769776e-05
3.72 3.21920394372732e-05
3.73 3.74011598283097e-05
3.74 4.34613119862926e-05
3.75 5.05111200003183e-05
3.76 5.87114641653619e-05
3.77 6.82489593673008e-05
3.78 7.93399544332107e-05
3.79 9.22351255062681e-05
3.8 0.00010722474571323
3.81 0.00012464472353424
3.82 0.000144883513427785
3.83 0.000168390014443465
3.84 0.000195682585788171
3.85 0.000227359322637621
3.86 0.000264109750873986
3.87 0.000306728115830324
3.88 0.000356128457517091
3.89 0.000413361682989888
3.9 0.00047963486539726
3.91 0.000556333018528088
3.92 0.000645043615060126
3.93 0.000747584135776247
3.94 0.000866032955219032
3.95 0.00100276388596909
3.96 0.00116048471811059
3.97 0.00134228010154602
3.98 0.00155165912544108
3.99 0.00179260794986369
4 0.0020696478379757
4.01 0.00238649481828279
4.02 0.00274676668478646
4.03 0.00315563061068247
4.04 0.00361875533975808
4.05 0.00414234439385675
4.06 0.0047331693822471
4.07 0.00539860306839768
4.08 0.00614665180102958
4.09 0.00698598686605634
4.1 0.00792597426473152
4.11 0.0089767023717614
4.12 0.0101490068761113
4.13 0.0114544923578082
4.14 0.0129055498072311
4.15 0.01451536935056
4.16 0.0162979474074806
4.17 0.0182680874764841
4.18 0.0204413937206824
4.19 0.0228342565146906
4.2 0.025463829112441
4.21 0.0283479946085802
4.22 0.0315053223939978
4.23 0.0349550133506814
4.24 0.0387168330939993
4.25 0.0428110326529867
4.26 0.0472582560823044
4.27 0.0520794346240945
4.28 0.0572956671842783
4.29 0.0629280870559913
4.3 0.0689977150121731
4.31 0.0755252990988629
4.32 0.0825311416886092
4.33 0.0900349145974279
4.34 0.0980554633256663
4.35 0.106610601749483
4.36 0.115716898860835
4.37 0.125389459425023
4.38 0.135641700690364
4.39 0.146485127538395
4.4 0.157929108698798
4.41 0.169980656864468
4.42 0.182644215722012
4.43 0.195921457055308
4.44 0.209811091178148
4.45 0.224308694001166
4.46 0.239406554033255
4.47 0.255093542554798
4.48 0.271355010076479
4.49 0.288172712011892
4.5 0.305524766244427
4.51 0.32338564496032
4.52 0.341726202753041
4.53 0.360513742583357
4.54 0.379712120709685
4.55 0.399281891191138
4.56 0.41918049001813
4.57 0.439362458350646
4.58 0.459779703750763
4.59 0.480381797692776
4.6 0.50111630703065
4.61 0.521929156508064
4.62 0.542765018820938
4.63 0.563567728196078
4.64 0.584280712943182
4.65 0.604847441981431
4.66 0.625211879947781
4.67 0.645318945173023
4.68 0.665114964574969
4.69 0.684548119376687
4.7 0.703568875521019
4.71 0.722130392728757
4.72 0.740188906342167
4.73 0.757704076410359
4.74 0.77463929890691
4.75 0.790961974517684
4.76 0.806643731088653
4.77 0.821660596566527
4.78 0.835993120083135
4.79 0.84962643970897
4.8 0.862550296312246
4.81 0.874758993886236
4.82 0.886251307628441
4.83 0.897030341949654
4.84 0.907103341438935
4.85 0.916481458593117
4.86 0.92517948281879
4.87 0.93321553581525
4.88 0.940610738934955
4.89 0.947388858483157
4.9 0.953575935152977
4.91 0.959199903892137
4.92 0.964290210462557
4.93 0.968877430787
4.94 0.972992898884614
4.95 0.976668348789555
4.96 0.979935575336666
4.97 0.982826118101051
4.98 0.985370972111808
4.99 0.987600328243274
5 0.989543345439849
5.01 0.991227956173067
5.02 0.992680705781922
5.03 0.993926625628396
5.04 0.994989139327039
5.05 0.99589000069553
5.06 0.996649261535148
5.07 0.997285266896024
5.08 0.997814675118746
5.09 0.998252499675233
5.1 0.998612169658351
5.11 0.998905605689342
5.12 0.999143308019559
5.13 0.999334453690932
5.14 0.99948699977842
5.15 0.999607789956462
5.16 0.999702661898007
5.17 0.999776553316405
5.18 0.999833604784629
5.19 0.999877257800695
5.2 0.999910346901267
5.21 0.999935184947176
5.22 0.999953641006128
5.23 0.99996721053219
5.24 0.999977077783561
5.25 0.99998417062607
5.26 0.999989208038286
5.27 0.99999274076487
5.28 0.999995185659358
5.29 0.999996854318382
5.3 0.99999797664004
5.31 0.999998719943823
5.32 0.999999204272681
5.33 0.99999951446426
5.34 0.999999709532442
5.35 0.999999829846435
5.36 0.999999902536504
5.37 0.999999945496268
5.38 0.999999970293883
5.39 0.999999984250428
5.4 0.999999991894758
5.41 0.99999999596078
5.42 0.999999998056007
5.43 0.999999999099177
5.44 0.999999999599465
5.45 0.999999999829784
5.46 0.999999999931169
5.47 0.999999999973651
5.48 0.999999999990507
5.49 0.999999999996803
5.5 0.999999999999002
5.51 0.999999999999714
5.52 0.999999999999925
5.53 0.999999999999983
5.54 0.999999999999996
5.55 0.999999999999999
5.56 1
5.57 1
5.58 1
5.59 1
5.6 1
5.61 1
5.62 1
5.63 1
5.64 1
5.65 1
5.66 1
5.67 1
5.68 1
5.69 1
5.7 1
5.71 1
5.72 1
5.73 1
5.74 1
5.75 1
5.76 1
5.77 1
5.78 1
5.79 1
5.8 1
5.81 1
5.82 1
5.83 1
5.84 1
5.85 1
5.86 1
5.87 1
5.88 1
5.89 1
5.9 1
5.91 1
5.92 1
5.93 1
5.94 1
5.95 1
5.96 1
5.97 1
5.98 1
5.99 1
6 1
};
\addplot [very thick, black]
table {%
0 8.59740547182281e-05
0.01 8.63692917173019e-05
0.02 8.67689421940246e-05
0.03 8.71730420250659e-05
0.04 8.75816272731538e-05
0.05 8.7994734186529e-05
0.06 8.8412399198377e-05
0.07 8.88346589262429e-05
0.08 8.92615501714254e-05
0.09 8.9693109918351e-05
0.1 9.012937533393e-05
0.11 9.05703837668925e-05
0.12 9.10161727471052e-05
0.13 9.14667799848708e-05
0.14 9.19222433702075e-05
0.15 9.23826009721106e-05
0.16 9.28478910377968e-05
0.17 9.33181519919305e-05
0.18 9.37934224358309e-05
0.19 9.42737411466657e-05
0.2 9.47591470766246e-05
0.21 9.5249679352078e-05
0.22 9.57453772727203e-05
0.23 9.62462803106952e-05
0.24 9.6752428109709e-05
0.25 9.72638604841265e-05
0.26 9.77806174180549e-05
0.27 9.83027390644115e-05
0.28 9.88302657439804e-05
0.29 9.93632379444555e-05
0.3 9.99016963194711e-05
0.31 0.000100445681687622
0.32 0.000100995235031471
0.33 0.000101550397496549
0.34 0.000102111210390342
0.35 0.00010267771518127
0.36 0.000103249953497664
0.37 0.000103827967126722
0.38 0.000104411798013475
0.39 0.000105001488259733
0.4 0.000105597080123034
0.41 0.000106198616015588
0.42 0.000106806138503214
0.43 0.00010741969030428
0.44 0.000108039314288638
0.45 0.000108665053476561
0.46 0.000109296951037678
0.47 0.000109935050289913
0.48 0.000110579394698429
0.49 0.000111230027874572
0.5 0.000111886993574822
0.51 0.000112550335699751
0.52 0.000113220098292989
0.53 0.000113896325540197
0.54 0.000114579061768055
0.55 0.000115268351443256
0.56 0.000115964239171517
0.57 0.000116666769696604
0.58 0.000117375987899375
0.59 0.000118091938796833
0.6 0.00011881466754121
0.61 0.000119544219419064
0.62 0.000120280639850396
0.63 0.000121023974387805
0.64 0.000121774268715652
0.65 0.000122531568649263
0.66 0.000123295920134157
0.67 0.00012406736924531
0.68 0.000124845962186447
0.69 0.000125631745289375
0.7 0.000126424765013346
0.71 0.000127225067944471
0.72 0.000128032700795165
0.73 0.000128847710403643
0.74 0.00012967014373346
0.75 0.000130500047873098
0.76 0.000131337470035605
0.77 0.000132182457558292
0.78 0.000133035057902478
0.79 0.000133895318653299
0.8 0.000134763287519574
0.81 0.000135639012333742
0.82 0.000136522541051854
0.83 0.000137413921753643
0.84 0.000138313202642664
0.85 0.000139220432046505
0.86 0.000140135658417083
0.87 0.000141058930331009
0.88 0.00014199029649005
0.89 0.000142929805721668
0.9 0.000143877506979655
0.91 0.000144833449344858
0.92 0.000145797682025997
0.93 0.000146770254360595
0.94 0.000147751215815998
0.95 0.000148740615990508
0.96 0.000149738504614629
0.97 0.000150744931552418
0.98 0.000151759946802965
0.99 0.000152783600501989
1 0.000153815942923555
1.01 0.000154857024481929
1.02 0.000155906895733561
1.03 0.000156965607379207
1.04 0.000158033210266191
1.05 0.000159109755390812
1.06 0.000160195293900908
1.07 0.000161289877098565
1.08 0.000162393556442989
1.09 0.000163506383553543
1.1 0.000164628410212947
1.11 0.000165759688370657
1.12 0.000166900270146414
1.13 0.000168050207833983
1.14 0.000169209553905071
1.15 0.000170378361013443
1.16 0.000171556681999233
1.17 0.000172744569893454
1.18 0.000173942077922724
1.19 0.000175149259514192
1.2 0.000176366168300695
1.21 0.000177592858126131
1.22 0.000178829383051063
1.23 0.000180075797358556
1.24 0.00018133215556026
1.25 0.00018259851240273
1.26 0.000183874922874013
1.27 0.000185161442210475
1.28 0.000186458125903907
1.29 0.000187765029708895
1.3 0.000189082209650468
1.31 0.000190409722032028
1.32 0.000191747623443576
1.33 0.000193095970770222
1.34 0.000194454821201013
1.35 0.000195824232238056
1.36 0.000197204261705968
1.37 0.000198594967761648
1.38 0.000199996408904372
1.39 0.000201408643986236
1.4 0.000202831732222935
1.41 0.000204265733204902
1.42 0.000205710706908801
1.43 0.000207166713709388
1.44 0.000208633814391751
1.45 0.000210112070163926
1.46 0.000211601542669913
1.47 0.000213102294003078
1.48 0.000214614386719973
1.49 0.000216137883854565
1.5 0.000217672848932892
1.51 0.000219219345988142
1.52 0.000220777439576183
1.53 0.000222347194791534
1.54 0.000223928677283798
1.55 0.000225521953274562
1.56 0.000227127089574762
1.57 0.000228744153602547
1.58 0.00023037321340163
1.59 0.000232014337660139
1.6 0.000233667595729988
1.61 0.000235333057646758
1.62 0.00023701079415013
1.63 0.000238700876704836
1.64 0.000240403377522182
1.65 0.000242118369582117
1.66 0.000243845926655889
1.67 0.000245586123329272
1.68 0.000247339035026388
1.69 0.00024910473803414
1.7 0.000250883309527253
1.71 0.000252674827593938
1.72 0.000254479371262203
1.73 0.000256297020526802
1.74 0.000258127856376844
1.75 0.000259971960824085
1.76 0.000261829416931882
1.77 0.000263700308844859
1.78 0.000265584721819272
1.79 0.00026748274225409
1.8 0.000269394457722815
1.81 0.000271319957006038
1.82 0.00027325933012475
1.83 0.000275212668374432
1.84 0.000277180064359913
1.85 0.000279161612031036
1.86 0.00028115740671912
1.87 0.000283167545174255
1.88 0.000285192125603425
1.89 0.000287231247709482
1.9 0.000289285012730986
1.91 0.00029135352348292
1.92 0.0002934368843983
1.93 0.000295535201570686
1.94 0.000297648582797624
1.95 0.000299777137625017
1.96 0.000301920977392448
1.97 0.000304080215279481
1.98 0.000306254966352929
1.99 0.000308445347615136
2 0.000310651478053264
2.01 0.000312873478689619
2.02 0.000315111472633024
2.03 0.000317365585131247
2.04 0.000319635943624531
2.05 0.000321922677800206
2.06 0.000324225919648425
2.07 0.000326545803519035
2.08 0.000328882466179605
2.09 0.000331236046874615
2.1 0.000333606687385846
2.11 0.000335994532093975
2.12 0.000338399728041393
2.13 0.000340822424996282
2.14 0.000343262775517947
2.15 0.000345720935023446
2.16 0.000348197061855523
2.17 0.000350691317351866
2.18 0.000353203865915724
2.19 0.000355734875087884
2.2 0.000358284515620049
2.21 0.000360852961549628
2.22 0.000363440390275956
2.23 0.000366046982637984
2.24 0.000368672922993447
2.25 0.000371318399299535
2.26 0.000373983603195091
2.27 0.000376668730084377
2.28 0.000379373979222399
2.29 0.000382099553801847
2.3 0.000384845661041658
2.31 0.00038761251227724
2.32 0.000390400323052362
2.33 0.000393209313212764
2.34 0.000396039707001496
2.35 0.000398891733156011
2.36 0.000401765625007059
2.37 0.000404661620579388
2.38 0.000407579962694291
2.39 0.000410520899074037
2.4 0.000413484682448196
2.41 0.000416471570661902
2.42 0.000419481826786087
2.43 0.000422515719229703
2.44 0.000425573521853984
2.45 0.000428655514088752
2.46 0.000431761981050839
2.47 0.000434893213664615
2.48 0.000438049508784694
2.49 0.000441231169320828
2.5 0.000444438504365032
2.51 0.000447671829320979
2.52 0.000450931466035703
2.53 0.00045421774293363
2.54 0.000457530995153005
2.55 0.000460871564684721
2.56 0.000464239800513619
2.57 0.000467636058762268
2.58 0.000471060702837302
2.59 0.000474514103578317
2.6 0.000477996639409402
2.61 0.000481508696493329
2.62 0.000485050668888449
2.63 0.000488622958708345
2.64 0.000492225976284278
2.65 0.000495860140330481
2.66 0.000499525878112341
2.67 0.000503223625617525
2.68 0.000506953827730087
2.69 0.00051071693840762
2.7 0.000514513420861495
2.71 0.000518343747740242
2.72 0.000522208401316121
2.73 0.00052610787367495
2.74 0.000530042666909228
2.75 0.000534013293314619
2.76 0.000538020275589861
2.77 0.000542064147040141
2.78 0.000546145451784006
2.79 0.000550264744963882
2.8 0.000554422592960231
2.81 0.000558619573609455
2.82 0.000562856276425567
2.83 0.000567133302825724
2.84 0.000571451266359681
2.85 0.000575810792943237
2.86 0.000580212521095734
2.87 0.000584657102181701
2.88 0.000589145200656701
2.89 0.000593677494317451
2.9 0.000598254674556318
2.91 0.000602877446620246
2.92 0.000607546529874204
2.93 0.000612262658069229
2.94 0.000617026579615174
2.95 0.000621839057858212
2.96 0.00062670087136321
2.97 0.000631612814201054
2.98 0.000636575696241026
2.99 0.00064159034344831
3 0.000646657598186747
3.01 0.00073133674047593
3.02 0.000828186756563205
3.03 0.000938389394223737
3.04 0.00106307861658873
3.05 0.00120330257524957
3.06 0.00135998297818618
3.07 0.00153387322985644
3.08 0.00172551692172933
3.09 0.00193520836483488
3.1 0.00216295686514889
3.11 0.0024084563411399
3.12 0.00267106167575593
3.13 0.00294977290039891
3.14 0.00324322795579003
3.15 0.00354970440313954
3.16 0.00386713011296147
3.17 0.004193102682457
3.18 0.00452491716421726
3.19 0.00485960165664501
3.2 0.00519396042209619
3.21 0.00552462445590972
3.22 0.00584810980185734
3.23 0.00616088435095859
3.24 0.00645944430706257
3.25 0.00674040187599665
3.26 0.007000585949153
3.27 0.00723715752133059
3.28 0.00744774123289848
3.29 0.00763057371137681
3.3 0.00778466830458106
3.31 0.00790999440317479
3.32 0.00800766797067674
3.33 0.00808014832649008
3.34 0.00813143490210386
3.35 0.00816725685816097
3.36 0.00819524830443915
3.37 0.00822510248316871
3.38 0.00826869956647692
3.39 0.00834020440015712
3.4 0.00845613215820742
3.41 0.00863538093898381
3.42 0.008899230364105
3.43 0.00927130394679366
3.44 0.00977749038315343
3.45 0.0104458153498188
3.46 0.0113062515750381
3.47 0.0123904518707563
3.48 0.0137313885919819
3.49 0.0153628847079156
3.5 0.017319027160362
3.51 0.0196334628125579
3.52 0.0223385907388826
3.53 0.0254646807191304
3.54 0.0290389645501856
3.55 0.0330847614159272
3.56 0.0376207079626926
3.57 0.0426601650753769
3.58 0.0482108647892453
3.59 0.054274842098312
3.6 0.0608486694227399
3.61 0.0679239798342015
3.62 0.0754882336837749
3.63 0.0835256570823818
3.64 0.09201826384531
3.65 0.100946867291218
3.66 0.110291994734222
3.67 0.120034633673651
3.68 0.130156761277898
3.69 0.140641634011638
3.7 0.151473838731687
3.71 0.162639127703037
3.72 0.174124076259163
3.73 0.185915612728473
3.74 0.198000475967962
3.75 0.210364656979888
3.76 0.222992878356324
3.77 0.2358681593723
3.78 0.248971505999467
3.79 0.262281754439344
3.8 0.275775584498054
3.81 0.289427705874971
3.82 0.303211207018871
3.83 0.317098043584019
3.84 0.331059632750532
3.85 0.345067511763948
3.86 0.359094014790019
3.87 0.373112921997085
3.88 0.387100038629879
3.89 0.401033669203041
3.9 0.414894961889079
3.91 0.428668109511276
3.92 0.442340405009955
3.93 0.455902159676284
3.94 0.46934650094836
3.95 0.482669072595009
3.96 0.495867663493848
3.97 0.508941792085047
3.98 0.521892272333877
3.99 0.534720784186211
4 0.547429467618496
4.01 0.560020479505602
4.02 0.572495740578143
4.03 0.584856634854609
4.04 0.597103795429219
4.05 0.609236957770709
4.06 0.621254867884791
4.07 0.633155240712118
4.08 0.644934763002362
4.09 0.656589134260577
4.1 0.668113139113845
4.11 0.679500744506604
4.12 0.690745215423807
4.13 0.701839243294725
4.14 0.712775081791548
4.15 0.723544685361509
4.16 0.734139846484266
4.17 0.7445523283011
4.18 0.754773989898842
4.19 0.764796902134568
4.2 0.774613452445941
4.21 0.784216437599571
4.22 0.793599143780852
4.23 0.802755413821298
4.24 0.811679701692727
4.25 0.820367114672972
4.26 0.828813443807419
4.27 0.83701518345829
4.28 0.844969540853381
4.29 0.852674436623075
4.3 0.860128497354089
4.31 0.867331041196213
4.32 0.874282057539585
4.33 0.8809821817403
4.34 0.887432665816076
4.35 0.893635345966103
4.36 0.89959260769377
4.37 0.905307349231528
4.38 0.910782943886357
4.39 0.916023201844692
4.4 0.92103233189912
4.41 0.925814903486996
4.42 0.9303758093645
4.43 0.934720229179
4.44 0.938853594148473
4.45 0.942781553008931
4.46 0.946509939349565
4.47 0.950044740419913
4.48 0.95339206746388
4.49 0.956558127610955
4.5 0.959549197335342
4.51 0.962371597478119
4.52 0.965031669815654
4.53 0.967535755148603
4.54 0.969890172879539
4.55 0.972101202043075
4.56 0.974175063749787
4.57 0.976117905003975
4.58 0.977935783854969
4.59 0.979634655841906
4.6 0.981220361692544
4.61 0.982698616237439
4.62 0.984074998501486
4.63 0.985354942935441
4.64 0.986543731750277
4.65 0.987646488317242
4.66 0.988668171596101
4.67 0.989613571553315
4.68 0.99048730553085
4.69 0.991293815525012
4.7 0.992037366333079
4.71 0.992722044523812
4.72 0.993351758186115
4.73 0.993930237408288
4.74 0.994461035438617
4.75 0.994947530476476
4.76 0.995392928041831
4.77 0.995800263870042
4.78 0.996172407278308
4.79 0.996512064949955
4.8 0.996821785083224
4.81 0.99710396185214
4.82 0.99736084012868
4.83 0.997594520417658
4.84 0.997806963958653
4.85 0.997999997952856
4.86 0.998175320876886
4.87 0.998334507850425
4.88 0.998479016029783
4.89 0.99861019000519
4.9 0.998729267185447
4.91 0.998837383159369
4.92 0.99893557702897
4.93 0.99902479671407
4.94 0.999105904231755
4.95 0.999179680956383
4.96 0.99924683286623
4.97 0.999307995781277
4.98 0.999363740592723
4.99 0.999414578478813
5 0.999460966093685
5.01 0.999503310706807
5.02 0.999541975260934
5.03 0.999577283307446
5.04 0.999609523770432
5.05 0.99963895548603
5.06 0.999665811462143
5.07 0.999690302806161
5.08 0.999712622274826
5.09 0.999732947410387
5.1 0.999751443239855
5.11 0.999768264528225
5.12 0.999783557590646
5.13 0.999797461681321
5.14 0.999810109987267
5.15 0.99982163026225
5.16 0.999832145139924
5.17 0.999841772165772
5.18 0.99985062358546
5.19 0.999858805924036
5.2 0.999866419387126
5.21 0.999873557113693
5.22 0.999880304311081
5.23 0.99988673730821
5.24 0.999892922571998
5.25 0.999898915744954
5.26 0.999904760776443
5.27 0.999910489233576
5.28 0.999916119885795
5.29 0.999921658655751
5.3 0.999927099013877
5.31 0.999932422863081
5.32 0.999937601913883
5.33 0.999942599493574
5.34 0.999947372673483
5.35 0.999951874546377
5.36 0.999956056451393
5.37 0.999959869934267
5.38 0.99996326824866
5.39 0.999966207247302
5.4 0.999968645572438
5.41 0.999970544123641
5.42 0.999971864848664
5.43 0.999972568963184
5.44 0.999972614756341
5.45 0.999971955183697
5.46 0.999970535493365
5.47 0.999968291181618
5.48 0.99996514663551
5.49 0.999961014889765
5.5 0.999955798989716
5.51 0.999949395481579
5.52 0.999941700497508
5.53 0.999932618703574
5.54 0.999922074974741
5.55 0.999910028028706
5.56 0.999896484447582
5.57 0.999881510717595
5.58 0.999865240407541
5.59 0.999847873698191
5.6 0.999829667346054
5.61 0.999810914699788
5.62 0.999791917125206
5.63 0.999772949477789
5.64 0.999754222537789
5.65 0.999735844431671
5.66 0.999717781333169
5.67 0.99969981585604
5.68 0.999681500331866
5.69 0.999662102358155
5.7 0.999640542180043
5.71 0.999615325893877
5.72 0.999584484721394
5.73 0.999545536986222
5.74 0.999495492291519
5.75 0.999430911303629
5.76 0.999348014081106
5.77 0.99924279371328
5.78 0.999111048004202
5.79 0.998948210130392
5.8 0.998748868661699
5.81 0.998505944492928
5.82 0.998209645526154
5.83 0.997846534913049
5.84 0.997399307467165
5.85 0.996848166086441
5.86 0.996174914944736
5.87 0.995370423862174
5.88 0.994443740797203
5.89 0.993427900708041
5.9 0.992383848455069
5.91 0.991432652231838
5.92 0.990859456329501
5.93 0.991182968561103
5.94 0.992765739098719
5.95 0.995091869867067
5.96 0.997030908132967
5.97 0.997714718001839
5.98 0.998271210886885
5.99 0.999220060403955
6 1
};
\end{axis}

\end{tikzpicture}

	\caption{Estimated probability of collision for 3 hypothetical scenarios. 
		The left plots show the speeds of the ego vehicle (solid black line) and lead vehicle (dashed black line) and the distance between the ego vehicle and the lead vehicle (dotted gray line, scale on the right of the plot).
		The right plots show the estimated probability of collision corresponding to the 3 scenarios according to the metric explained in \cref{sec:ngsim metric} that is based on our proposed method (black lines) and, for comparison, the metric of \textcite{wang2014evaluation} explained in \cref{sec:wang stamatiadis explanation} (gray lines).}
	\label{fig:scenarios}		
\end{figure}

The first scenario in \cref{fig:scenarios} (top row) shows a scenario in which the lead vehicle initially drives with a speed of \SI{20}{\meter\per\second}.
The lead vehicle starts to decelerate after \SI{3}{\second} toward a speed of \SI{10}{\meter\per\second} with an average deceleration of \SI{3}{\meter\per\second\squared}.
The ego vehicle initially drives with a speed of \SI{24}{\meter\per\second} at a distance of \SI{40}{\meter} from the lead vehicle.
The ego vehicle starts decelerating after \SI{2}{\second} toward a speed of \SI{8}{\meter\per\second} within \SI{4}{\second}.
It takes \SI{4}{\second} more to reach the speed of the lead vehicle.
Because the ego vehicle always maintains a relatively large distance toward the lead vehicle, both surrogate safety metrics do not qualify this scenario as risky, considering the estimated probability of collisions that stays below 0.1.

The second scenario in \cref{fig:scenarios} (center row) differs from the first scenario in that the ego vehicle starts to decelerate \SI{2}{\second} later.
As a result, the ego vehicle approaches the lead vehicle up to a distance of \SI{5.4}{\meter}.
According to $\probabilityestcond{\collision}{\situationcurrent}$ from \cref{sec:ngsim metric} (black line in the right plot of \cref{fig:scenarios}), the probability of collision reaches almost 1, indicating that around that time, the risk of a collision is high.
The local minimum of $\probabilityestcond{\collision}{\situationcurrent}$ at around \SI{6}{\second} illustrates the effect of the numerical approximation of $\probabilitycond{\collision}{\situationcurrent}$.
Because we have used $\simulationthreshold=0.1>0$, the resulting estimation may have an error. 
When lowering the threshold $\simulationthreshold$, the resulting $\probabilityestcond{\collision}{\situationcurrent}$ in the center right plot in \cref{fig:scenarios} will be smoother. 
This goes, however, at the cost of having to do more simulations.
Alternatively, the bandwidth matrix $\bandwidthnw$ may be increased. 
On the one hand, this will lower the variance of the error, but, on the other hand, it will increase the bias of the result.
We refer the interested reader to \autocite{chen2017tutorial} for more details on the effect of $\bandwidthnw$.

The third scenario in \cref{fig:scenarios} (bottom row) differs from the second scenario in that the initial distance between ego vehicle and the lead vehicle is \SI{31.5}{\meter} instead of \SI{40}{\meter}. 
As a result, the ego vehicle collides with the lead vehicle after \SI{6}{\second}.
As expected, the surrogate safety metrics in \cref{fig:scenarios} indicate a collision probability of 1.
The difference between $\probabilityestcond{\collision}{\situationcurrent}$ and $\wangstamatiadis$ is that $\probabilityestcond{\collision}{\situationcurrent}$ earlier increases. 



\subsection{Benchmarking surrogate safety metric with expected causal tendencies}
\label{sec:tendencies}

In this section, we demonstrate an approach for benchmarking a surrogate safety metric.
The approach is based on the so-called \emph{expected causal tendencies} of \textcite{mullakkal2017comparative}.
\textcite{mullakkal2017comparative} argue that the risk increases if the approaching speed of the ego vehicle toward the leading vehicle increases.
Also, the risk increases with a higher ego vehicle speed \autocite{aarts2006driving} or a higher driver reaction time \autocite{klauer2006impact}.
On the other hand, the risk decreases with a higher road friction \autocite{wallman2001friction} or a larger intervehicle spacing \autocite{mullakkal2017comparative}.

To check whether the developed surrogate safety metric follows these 5 expected causal tendencies\footnote{In \autocite{mullakkal2017comparative}, a sixth expected causal tendency is mentioned based on \autocite{evans1994driver}, namely the vehicle mass. 
Our interpretation of \autocite{evans1994driver}, however, is that the ratio of masses of two colliding vehicles influences the safety risk and that one cannot argue that a higher mass of the ego vehicle necessarily increases the safety risk. 
Therefore, we exclude the ego vehicle mass from our analysis.}, we evaluate the partial derivatives of the metric of \cref{eq:nadaraya watson}.
The intuition is as follows: If the expected causal tendency for an input X (e.g., the ego vehicle speed) is that the risk increases as X increases, then we expect that the partial derivative of our surrogate safety metric with respect to X to be positive.
Furthermore, if we evaluate the partial derivative at many points, we expect that at least the majority of these evaluated partial derivatives are positive.
Analogously, if we expect that the risk metric decreases with increasing X, then we expect that at least the majority of the evaluated partial derivatives are negative.

To illustrate the approach for benchmarking a surrogate safety metric, we use the surrogate safety metric based on the \ac{ngsim} data set as explained in \cref{sec:ngsim metric} with a few adaptations.
Because we have not described an expected tendency regarding $\accelerationleadsymbol$, we simply use $\accelerationleadsymbol=0$.
Also, the relative speed, i.e., $\speedegosymbol-\speedleadsymbol$, is used instead of $\speedleadsymbol$.
Instead of assuming a random reaction time $\timereact$ and \ac{madr} $\accelerationmax$, these are now considered as input to our metric. 
Finally, instead of using the log of the gap between the ego vehicle and the lead vehicle, we use the gap as a direct input.
Thus, we have:
\begin{equation}
	\label{eq:input partial derivatives}
	\situationcurrent^T = \begin{bmatrix}
		\speedegosymbol-\speedleadsymbol & \speedegosymbol & \timereact & \gapsymbol & \accelerationmax
	\end{bmatrix}.
\end{equation}

We computed $\probabilityestcond{\collision}{\situationcurrent}$ using \cref{eq:nadaraya watson} where the points $\situationcurrentinstance{\situationindexdesign}, \situationindexdesign\in\{1,\ldots,\numberofdesignpoints\}$ are taken from a grid.
A threshold $\simulationthreshold=0.02$ was used.
For each input variable, 10 different values at equal distance were used, resulting in $\numberofdesignpoints=10^5$.
Here, $\speedegosymbol-\speedleadsymbol$ ranged from \SI{0}{\meter\per\second} to \SI{20}{\meter\per\second}, $\speedegosymbol$ ranged from \SI{10}{\meter\per\second} to \SI{30}{\meter\per\second}, $\timereact$ ranged from \SI{0.5}{\second} to \SI{1.5}{\second}, $\gapsymbol$ ranged from \SI{5}{\meter} to \SI{30}{\meter}, and $\accelerationmax$ ranged from \SI{4}{\meter\per\second\squared} to \SI{10}{\meter\per\second\squared}.
For the bandwidth matrix $\bandwidthnw$, we used a diagonal matrix with 0.20, 0.13, 0.20, 2.25, and 81 on the diagonal.
For each input variable listed in \cref{eq:input partial derivatives}, we evaluated the partial derivative of \cref{eq:nadaraya watson} at each $\situationcurrentinstance{\situationindexdesign}, \situationindexdesign\in\{1,\ldots,\numberofdesignpoints\}$.

\Cref{tab:tendencies} shows the result of the benchmarking. 
It shows that the surrogate safety metric follows the expected causal tendencies mostly. 
E.g., in more than \SI{99}{\%} of the cases, the partial derivative of the relative speed ($\speedegosymbol-\speedleadsymbol$) is positive.
For the remaining \SI{1}{\%}, the partial derivative is negative, albeit only slightly. 

\begin{table}
	\centering
	\caption{Percentiles of the partial derivatives of the surrogate safety metric and the corresponding expected causal tendencies.}
	\label{tab:tendencies}
	\begin{tabular}{lrrrrr}
		\toprule
		& $\speedegosymbol-\speedleadsymbol$ & $\speedegosymbol$ & $\timereact$ & $\gapsymbol$ & $\accelerationmax$ \\
		\otoprule
		Expected tendency & Increase & Increase & Increase & Decrease & Decrease \\
		Maximum         &  0.1629 &  0.1555 &  1.3136 &  0.0010 &  0.0037 \\
		99th percentile &  0.1585 &  0.1162 &  0.8968 &  0.0002 &  0.0002 \\
		95th percentile &  0.1495 &  0.0524 &  0.6765 &  0.0000 & -0.0000 \\
		90th percentile &  0.1346 &  0.0151 &  0.5351 & -0.0000 & -0.0000 \\
		75th percentile &  0.0746 &  0.0012 &  0.2917 & -0.0002 & -0.0003 \\
		50th percentile &  0.0114 &  0.0001 &  0.0605 & -0.0070 & -0.0054 \\
		25th percentile &  0.0004 &  0.0000 &  0.0022 & -0.0320 & -0.0290 \\
		10th percentile &  0.0000 & -0.0000 &  0.0000 & -0.0545 & -0.0654 \\
		 5th percentile &  0.0000 & -0.0002 &  0.0000 & -0.0645 & -0.0880 \\
		 1st percentile &  0.0000 & -0.0007 & -0.0020 & -0.0781 & -0.1337 \\
		Minimum         & -0.0035 & -0.0030 & -0.0180 & -0.1076 & -0.2030 \\
		\bottomrule
	\end{tabular}
\end{table}

