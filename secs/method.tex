\section{\acl{ourmethod} method}
\label{sec:method}

In this section, we propose the \cstarta\ac{ourmethod} method: \cenda a method for deriving a measure that quantifies the risk of a certain event, such as a collision, in a particular situation in which a vehicle - hereafter, the \textit{ego vehicle} - is in and that is applicable for real-time use.
\cstarta The \ac{ourmethod} method \cenda consists of four steps. 
The first step is the parameterization of the ``current situation'' and the possible ``future situations''.
Second, based on the current situation, we estimate the probability (density) for the possible future situations. 
The third step includes determining the probability of the specified event based on the current and the future situations.
Finally, local regression is used to speed up the calculations and to make it possible to use the \ac{ssm} in real time. 
These four steps are described in the following subsections.

In the remainder of this article, the following notation is used. 
To denote a probability, $\probability{\cdot}$ is used. 
\Iac{pdf} is denoted by $\density{\cdot}$. 
The probability of $\dummyvara$ given $\dummyvarb$ is denoted by $\probabilitycond{\dummyvara}{\dummyvarb}$.
Similarly, a conditional \ac{pdf} is denoted by $\densitycond{\cdot}{\cdot}$. 
To denote the estimation of any of the aforementioned functions, a circumflex is used, e.g, $\probabilityest{\dummyvara}$ denotes the estimated probability of $\dummyvara$.



\subsection{Parameterize current and future situations}
\label{sec:parametrization}

The first step is to parameterize the current situation the ego vehicle is in. 
In other words, the current situation needs to be described using $\situationcurrentdim$ numbers that are stacked into one vector $\situationcurrent \in \situationcurrentspace \subseteq \realnumbers^{\situationcurrentdim}$. 
\cstartb This vector contains all relevant aspects for determining the risk. \cendb
As an example, $\situationcurrent$ could contain the speed of the ego vehicle and the distance toward its preceding vehicle. 
In \cref{sec:case study}, we will consider more examples.

Next to describing the current situation, the future situation is described using $\situationfuturedim$ numbers stacked into one vector $\situationfuture \in \situationfuturespace \subseteq \realnumbers^{\situationfuturedim}$. 
Together with $\situationcurrent$, $\situationfuture$ contains enough information to describe how the relevant future, e.g., the next 5 seconds, around the ego vehicle develops over time. 
As an example, $\situationfuture$ could contain the speed for the next 5 seconds of the leading vehicle (if any) that is in front of the ego vehicle.
\cstartb In \cref{sec:case study}, we will consider more examples. \cendb

Let $\collision$ denote an event, e.g., a collision, such that the probability of this event is $\probability{\collision}$.
The goal of our \ac{ssm} is to estimate the probability of the event $\collision$ given a particular situation $\situationcurrent$, i.e., $\probabilitycond{\collision}{\situationcurrent}$.
We do this by considering all future situations, $\situationfuturespace$, and calculating the probability of a collision given each possible value of $\situationfuture$. 
Using integration, we obtain $\probabilitycond{\collision}{\situationcurrent}$:
\begin{equation}
	\label{eq:probability collision expectation}
	\probabilitycond{\collision}{\situationcurrent} 
	= \int_{\situationfuturespace} 
	\probabilitycond{\collision}{\situationcurrent, \situationfuture} 
	\densitycond{\situationfuture}{\situationcurrent} 
	\ud \situationfuture.
\end{equation}
%In \cref{sec:estimate future}, we propose a method to estimate $\densitycond{\situationfuture}{\situationcurrent}$ and in \cref{sec:estimate collision}, we propose a method to estimate $\probabilitycond{\collision}{\situationcurrent, \situationfuture}$.



\subsection{Estimate $\densitycond{\situationfuture}{\situationcurrent}$}
\label{sec:estimate future}

In this section, we propose a method to estimate $\densitycond{\situationfuture}{\situationcurrent}$, i.e., the \ac{pdf} of $\situationfuture$ given $\situationcurrent$.
Using the product rule for probability, we can write:
\begin{equation}
	\densitycond{\situationfuture}{\situationcurrent} 
	= \frac{\density{\situationcurrent, \situationfuture}}{\density{\situationcurrent}}
	= \frac{\density{\situationcurrent, \situationfuture}}{
		\int_{\situationfuturespace} \density{\situationcurrent, \situationfuture} \ud\situationfuture
	}.
\end{equation}
Thus, it suffices to estimate $\density{\situationcurrent, \situationfuture}$. 

Our proposal is to estimate $\density{\situationcurrent, \situationfuture}$ in a data-driven manner. 
A data-driven approach brings several benefits.
First, the estimate automatically adapts to local driving styles and behaviors, which can change from region to region, provided that the data are obtained from the same local traffic.
Second, assumptions such as a constant speed of other vehicles, are not needed.
For our data-driven approach, let us assume that we have obtained $\situationnumberof$ situations, denoted by $\situationcurrentinstance{\situationindex}\in\situationcurrentspace, \situationindex\in\{1,\ldots,\situationnumberof\}$, and their corresponding future situations described by $\situationfutureinstance{\situationindex}\in\situationfuturespace$.
\cstartb In the remainder of this subsection, we will explain several methods for estimating $\density{\situationcurrent, \situationfuture}$ using $(\situationcurrentinstance{\situationindex},\situationfutureinstance{\situationindex}), \situationindex\in\{1,\ldots,\situationnumberof\}$ considering different assumptions. \cendb



\cstartb\subsubsection{Kernel density estimation}\cendb
\label{sec:one kde}

We first explain how to estimate $\density{\situationcurrent, \situationfuture}$ if we assume that all $\situationcurrentdim+\situationfuturedim$ parameters depend on each other. 
The shape of the \ac{pdf} $\density{\situationcurrent, \situationfuture}$ is unknown beforehand. 
Furthermore, the shape of the estimated \ac{pdf} might change as more data are acquired. 
Assuming a functional form of the \ac{pdf} and fitting the parameters of the \ac{pdf} to the data may therefore lead to inaccurate fits unless extensive manual tuning is applied.
We employ a non-parametric approach using \ac{kde} \autocite{rosenblatt1956remarks, parzen1962estimation} because the shape of the \ac{pdf} is then automatically computed and \ac{kde} is highly flexible regarding the shape of the \ac{pdf}. 
Using the \ac{kde}, the estimated \ac{pdf} becomes:
\begin{equation}
	\label{eq:kde estimate}
	\densityest{\situationcurrent,\situationfuture}
	= \frac{1}{\situationnumberof} \sum_{\situationindex=1}^{\situationnumberof}
	\kernelfuncnormalized{\bandwidthmatrix}{
		\begin{bmatrix}
			\situationcurrent \\
			\situationfuture
		\end{bmatrix} -
		\begin{bmatrix}
			\situationcurrentinstance{\situationindex} \\
			\situationfutureinstance{\situationindex}
		\end{bmatrix}
	},
\end{equation}
where $\kernelfuncnormalized{\bandwidthmatrix}{\cdot}$ is an appropriate kernel function with \cstartb a $(\situationcurrentdim+\situationfuturedim)$-by-$(\situationcurrentdim+\situationfuturedim)$ symmetric \cendb positive definite \emph{bandwidth} \cstartb or \emph{smoothing} \cendb matrix $\bandwidthmatrix$. 
The choice of the kernel $\kernelfuncnormalized{\bandwidthmatrix}{\cdot}$ is not as important as the choice of the bandwidth matrix $\bandwidthmatrix$ \autocite{turlach1993bandwidthselection}.
For example, the Gaussian kernel is given by \autocite{duong2007ks}
\begin{equation}
	\label{eq:kernel current future}
	\kernelfuncnormalized{\bandwidthmatrix}{\dummyvarkernel}
	= \frac{1}{\left( 2 \pi \right)^{\left( \situationcurrentdim + \situationfuturedim \right) / 2} 
	\left|\bandwidthmatrix\right|^{1/2} }
	\e{ -\frac{1}{2} \dummyvarkernel\transpose \bandwidthmatrix^{-1} \dummyvarkernel }.
\end{equation}

The bandwidth matrix $\bandwidthmatrix$ controls the width of the kernel, or, in other words, the influence of each data point \cstartb (i.e., $\begin{bmatrix}\situationcurrentinstance{\situationindex}\transpose & \situationfutureinstance{\situationindex}\transpose\end{bmatrix}\transpose$) \cendb on nearby regions \cstartb (see \autocite{wand1994multivariate} for a more extensive explanation of the bandwidth matrix)\cendb. 
There are many different ways of estimating the bandwidth matrix, ranging from simple reference rules like, e.g., Silverman's rule of thumb \autocite{silverman1986density} to more elaborate methods; see \autocite{turlach1993bandwidthselection, chiu1996comparative, jones1996brief, bashtannyk2001bandwidth, zambom2013review} for reviews of different bandwidth selection methods.

Drawing samples from the estimated \ac{pdf} in \cref{eq:kde estimate} is straightforward: two random numbers are drawn, one to choose a random generator kernel out of the $\situationnumberof$ kernels that are used to construct the \ac{kde}, and one random number from that kernel.
Sampling from $\densityestcond{\situationfuture}{\situationcurrent}$ works similarly, but instead of using an equal probability for each random generator kernel to be selected, different probabilities are used based on $\situationcurrent$.
For more information on sampling from a conditional \ac{pdf} obtained using \ac{kde}, see \autocite{holmes2012fast, degelder2021conditional}.



\cstartb\subsubsection{Assuming independence}\cendb
\label{sec:no special case}

Due to the curse of dimensionality \autocite{scott2015multivariate}, estimating $\density{\situationcurrent, \situationfuture}$ with one \ac{kde} according to \cref{eq:kde estimate} becomes inaccurate if $\situationcurrentdim + \situationfuturedim$ becomes large.
There are a few ways to avoid this curse of dimensionality.
Without going into much detail, we list a few options.

One option is to assume that one or more parameters are independent of the other parameters. 
E.g., suppose that $\situationfuture\transpose=\begin{bmatrix}\situationfutureparta\transpose & \situationfuturepartb\transpose\end{bmatrix}$, such that $\situationfuturepartb$ is independent of $\situationcurrent$ and $\situationfutureparta$.
Then we can write
\begin{equation}
	\density{\situationcurrent, \situationfuture}
	= \density{\situationcurrent, \situationfutureparta, \situationfuturepartb}
	= \density{\situationcurrent, \situationfutureparta} \cdot \density{\situationfuturepartb}.
\end{equation}
In this case, we would need to estimate $\density{\situationcurrent, \situationfutureparta}$ and $\density{\situationfuturepartb}$, which can be done in a similar manner as presented in \cref{sec:one kde}.
Because these two \acp{pdf} have fewer variables than $\density{\situationcurrent, \situationfuture}$, the two estimated \acp{pdf} will suffer less from the curse of dimensionality \autocite{scott2015multivariate}.

Another option is to model $\densitycond{\situationfuture}{\situationcurrent}$ as a cascade of conditional probabilities. 
For example, using the partitioning $\situationfuture\transpose=\begin{bmatrix}\situationfutureparta\transpose & \situationfuturepartb\transpose\end{bmatrix}$, $\densitycond{\situationcurrent}{\situationfuture}$ can be approximated using two conditional densities:
\begin{equation}
	\densitycond{\situationfuture}{\situationcurrent}
	= \densitycond{\situationfutureparta, \situationfuturepartb}{\situationcurrent}
	= \densitycond{\situationfutureparta}{\situationfuturepartb, \situationcurrent} \cdot \densitycond{\situationfuturepartb}{\situationcurrent}
	\approx \densitycond{\situationfutureparta}{\situationfuturepartb} \cdot \densitycond{\situationfuturepartb}{\situationcurrent}.
\end{equation}
The same partitioning can be applied to $\densitycond{\situationfutureparta}{\situationfuturepartb}$ and $\densitycond{\situationfuturepartb}{\situationcurrent}$ until only two-dimensional \acp{pdf} need to be estimated.
\cstarta Although this will lead to larger approximation errors, the lower-dimensional \acp{pdf} can be better estimated. \cenda
For more information on this approach, we refer the reader to \autocite{aas2009paircopula, nagler2016evading}.



\subsubsection{Reduce number of parameters using singular value decomposition}
\label{sec:parameter reduction}

Another way to avoid the curse of dimensionality is to use \iac{svd} \autocite{golub2013matrix} to reduce the number of parameters.
With \iac{svd}, the parameters $\situationcurrent$ and $\situationfuture$ are transformed into a lower-dimensional vector of parameters in such a way that the reduced vector of parameters describes as much of the variation as possible.
To do this, \iac{svd} is made of the matrix that contains all $\situationnumberof$ observed situations:
\begin{equation}
	\begin{bmatrix}
		\situationcurrentinstance{1}-\situationcurrentmean & \cdots & \situationcurrentinstance{\situationnumberof}-\situationcurrentmean \\
		\situationfutureinstance{1}-\situationfuturemean & \cdots & \situationfutureinstance{\situationnumberof}-\situationfuturemean
	\end{bmatrix} = \svdu \svds \svdv\transpose.
\end{equation}
Here, $\situationcurrentmean=\frac{1}{\situationnumberof}\sum_{\situationindex=1}^{\situationnumberof}\situationcurrentinstance{\situationindex}$ and $\situationfuturemean=\frac{1}{\situationnumberof}\sum_{\situationindex=1}^{\situationnumberof}\situationfutureinstance{\situationindex}$.
The matrices $\svdu \in \realnumbers^{\left(\situationcurrentdim+\situationfuturedim\right)\times\left(\situationcurrentdim+\situationfuturedim\right)}$ and $\svdv \in \realnumbers^{\situationnumberof \times \situationnumberof}$ are orthonormal, i.e., $\svdu^{-1}=\svdu\transpose$ and $\svdv^{-1}=\svdv\transpose$.
Moreover, $\svds\in\realnumbers^{\left(\situationcurrentdim+\situationfuturedim\right)\times\situationnumberof}$ has only zeros except at the diagonal: the $(\svdindex,\svdindex)$-th element is $\svdsv{\svdindex}$, $\svdindex\in\{1,\ldots,\svdrank\}$ with  $\svdrank=\min(\situationcurrentdim+\situationfuturedim, \situationnumberof)$, such that
\begin{equation}
	\svdsv{1} \geq \svdsv{2} \geq \ldots \geq \svdsv{\svdrank} \geq 0.
\end{equation}
Because these so-called singular values are in decreasing order, we can approximate $\situationcurrent$ and $\situationfuture$ by setting $\svdsv{\svdindex}=0$ for $\svdindex > \dimension$ with $\situationcurrentdim < \dimension < \situationcurrentdim+\situationfuturedim$\footnote{\cstartb We have $\dimension < \situationcurrentdim+\situationfuturedim$, such that the dimension is reduced and we have $\dimension>\situationcurrentdim$, such that the number of linear constraints in \cref{eq:linear constraint} ($\situationcurrentdim$) is smaller than the number of variables ($\dimension$).\cendb}:
\begin{equation}
	\label{eq:svd approximation}
	\begin{bmatrix}
		\situationcurrentinstance{\situationindex} - \situationcurrentmean \\
		\situationfutureinstance{\situationindex} - \situationfuturemean
	\end{bmatrix}
	= \sum_{\svdindex=1}^{\svdrank} \svdsv{\svdindex} \svdventry{\situationindex}{\svdindex} \svduvec{\svdindex}
	\approx \sum_{\svdindex=1}^{\dimension} \svdsv{\svdindex} \svdventry{\situationindex}{\svdindex} \svduvec{\svdindex},
	= \begin{bmatrix} \svduupperleft \\ \svdulowerleft \end{bmatrix} \svdsupperleft \svdvvecd{\situationindex},
\end{equation}
where $\svdventry{\situationindex}{\svdindex}$ is the $(\situationindex,\svdindex)$-th element of $\svdv$ and $\svduvec{\svdindex}$ is the $\svdindex$-th column of $\svdu$.
Moreover, $\svduupperleft$ is the $\situationcurrentdim$-by-$\dimension$ upper left submatrix of $\svdu$, $\svdulowerleft$ is the $\situationfuturedim$-by-$\dimension$ lower left submatrix $\svdu$, $\svdsupperleft\in\realnumbers^{\dimension\times\dimension}$ is the diagonal matrix with the first $\dimension$ singular values on its diagonal and $\svdvvecd{\situationindex}\transpose = \begin{bmatrix} \svdventry{\situationindex}{1} & \cdots & \svdventry{\situationindex}{\dimension} \end{bmatrix}$.
\cstartb Thus, with $\situationcurrentmean$, $\situationfuturemean$, $\svduupperleft$, $\svdulowerleft$, and $\svdsupperleft$, the $(\situationcurrentdim+\situationfuturedim)$-dimensional vector $\begin{bmatrix}\situationcurrentinstance{\situationindex}\transpose & \situationfutureinstance{\situationindex}\transpose\end{bmatrix}\transpose$ is approximated using the $\dimension$-dimensional vector $\svdvvecd{\situationindex}$. \cendb

Instead of estimating the \ac{pdf} of $\begin{bmatrix}\situationcurrentinstance{\situationindex}\transpose & \situationfutureinstance{\situationindex}\transpose\end{bmatrix}\transpose$, we now estimate the \ac{pdf} of $\svdvvecd{\situationindex}$ using \ac{kde} as described in \cref{sec:one kde}.
To sample from $\densityestcond{\situationfuture}{\situationcurrent}$, we can sample from the estimated distribution of $\svdvvecd{\situationindex}$.
Because \cref{eq:svd approximation} is a linear mapping, the sample $\svdvvecsymbol$ that is drawn from the estimated distribution of $\svdvvecd{\situationindex}$ is subject to a linear constraint:
\begin{equation}
	\label{eq:linear constraint}
	\svduupperleft \svdsupperleft \svdvvecsymbol = \situationcurrent - \situationcurrentmean.
\end{equation}
In \autocite{degelder2021conditional}, an algorithm is provided for sampling from \iac{kde} with a Gaussian kernel of \cref{eq:kernel current future} such that the resulting samples are subject to a linear constraint like \cref{eq:linear constraint}.



\subsection{Estimate $\probabilitycond{\collision}{\situationcurrent}$ using a Monte Carlo simulation}
\label{sec:estimate collision}

\cstarta Monte Carlo simulations are used to estimate $\probabilitycond{\collision}{\situationcurrent}$, i.e., the probability of an event $\collision$ given the current situation $\situationcurrent$. \cenda
The details of the simulation depends on the actual application. 
For example, if the goal of our \ac{ssm} is to evaluate the risk that a human-driven vehicle collides, the simulation should involve human driving behavior models. 
On the other hand, if the goal is to evaluate the risk of collision when \iac{ads} is controlling the vehicle, the simulation should include the model of this \ac{ads}.

A straightforward way to compute $\probabilitycond{\collision}{\situationcurrent}$ is to repeat a certain number of simulations with the same $\situationcurrent$ and count the number of simulations that result in the event $\collision$.
If $\numberofsimulations$ denotes the number of simulations and $\numberofcollisions$ is the number of events $\collision$, then $\probabilitycond{\collision}{\situationcurrent}$ could be estimated using
\begin{equation}
	\label{eq:binomial estimation}
	\probabilityestcond{\collision}{\situationcurrent}
	= \frac{\numberofcollisions}{\numberofsimulations}.
\end{equation}

An important choice for estimating $\probabilitycond{\collision}{\situationcurrent}$ is the number of simulations, $\numberofsimulations$.
One approach is to keep increasing $\numberofsimulations$ until there is enough confidence in the estimation of \cref{eq:binomial estimation}.
E.g., the Clopper-Pearson interval \autocite{clopper1934use} or the Wilson score interval \autocite{wilson1927probable} can be used to determine the confidence of the estimation of \cref{eq:binomial estimation}.
A disadvantage of this approach is that only the fact whether the event $\collision$ occurred or not is used, while the simulation provides more information. 
Therefore, we provide an alternative approach to estimate $\probabilitycond{\collision}{\situationcurrent}$.

\cstarta For the alternative approach, let us assume that one simulation run provides more information that just the fact that the event $\collision$ occurred or not.
Let $\simulationresult \in \realnumbers^{\dimsimulationresult}$ be a continuous variable representing the result of a simulation run and let $\spacecollision$ denote the \cenda\cstartb known \cendb\cstarta set of possible simulation results in which the event $\collision$ occurred. 
Thus, \cenda $\simulationresult \in \spacecollision$ if and only if the simulation results in the event $\collision$.
Therefore, we have
\begin{equation}
	\probabilitycond{\collision}{\situationcurrent}
	= \probabilitycond{\simulationresult \in \spacecollision}{\situationcurrent}
	= \int_{\spacecollision} \densitycond{\simulationresult}{\situationcurrent} \ud \simulationresult.
\end{equation}
Similar as with the estimation of $\density{\situationcurrent, \situationfuture}$ in \cref{sec:estimate future}, we employ \ac{kde} to estimate $\densitycond{\simulationresult}{\situationcurrent}$:
\begin{equation}
	\label{eq:kde simulation result}
	\densityestcond{\simulationresult}{\situationcurrent}
	= \frac{1}{\numberofsimulations} 
	\sum_{\simulationindex=1}^{\numberofsimulations} \kernelfuncnormalized{\simulationbandwidth}{\simulationinstance{\simulationindex} - \simulationresult},
\end{equation}
where $\simulationinstance{\simulationindex}$ denotes the result of the $\simulationindex$-th simulation and $\simulationbandwidth$ denotes an appropriate bandwidth matrix.
The kernel function $\kernelfuncnormalized{\simulationbandwidth}{\cdot}$ is similarly defined as \cref{eq:kernel current future}.
We can now estimate $\probabilitycond{\collision}{\situationcurrent}$ by substituting $\densityestcond{\simulationresult}{\situationcurrent}$ of \cref{eq:kde simulation result} for $\densitycond{\simulationresult}{\situationcurrent}$:
\begin{equation}
	\label{eq:estimate probability of collision}
	\probabilityestcond{\collision}{\situationcurrent}
	= \probabilityestcond{\simulationresult \in \spacecollision}{\situationcurrent}
	= \int_{\spacecollision} \densityestcond{\simulationresult}{\situationcurrent} \ud \simulationresult
	=\frac{1}{\numberofsimulations}
	\sum_{\simulationindex=1}^{\numberofsimulations} \int_{\spacecollision}
	\kernelfuncnormalized{\simulationbandwidth}{\simulationinstance{\simulationindex} - \simulationresult} \ud \simulationresult.
\end{equation}

Similar as with \cref{eq:binomial estimation}, we need to choose the number of simulations $\numberofsimulations$.
Our proposal is to keep increasing $\numberofsimulations$ until the variance of $\probabilityestcond{\simulationresult \in \spacecollision}{\situationcurrent}$ is below a threshold $\simulationthreshold$.
The variance follows from \autocite{nadaraya1964some}:
\begin{equation}
	\label{eq:variance estimation}
	\variance{\probabilityestcond{\simulationresult \in \spacecollision}{\situationcurrent}}
	= \frac{\probabilitycond{\simulationresult \in \spacecollision}{\situationcurrent}
		\left( 1-\probabilitycond{\simulationresult \in \spacecollision}{\situationcurrent} \right)}{\numberofsimulations}.
\end{equation}
Because $\probabilitycond{\simulationresult \in \spacecollision}{\situationcurrent}$ is unknown, we use the estimated counterpart of \cref{eq:estimate probability of collision}.
Thus, $\numberofsimulations$ is increased until the following condition is met:
\begin{equation}
	\label{eq:condition stop simulations}
	\frac{\probabilityestcond{\simulationresult \in \spacecollision}{\situationcurrent}
		\left( 1-\probabilityestcond{\simulationresult \in \spacecollision}{\situationcurrent} \right)}{\numberofsimulations}
	< \simulationthreshold.
\end{equation}



\subsection{Regression for real-time estimation of $\probabilitycond{\collision}{\situationcurrent}$}
\label{sec:final metric calculation}

To evaluate the risk measure during real-time operation of the ego vehicle, the expression of \cref{eq:estimate probability of collision} is problematic, because it would require multiple $\numberofsimulations$ simulation.
Even if the calculation is accelerated using a technique like importance sampling, it might take too long.
Therefore, we propose to evaluate \cref{eq:estimate probability of collision} only for some fixed $\situationcurrentinstance{\situationindexdesign}$, $\situationindexdesign\in\{1,\ldots,\numberofdesignpoints\}$.
\cstartb Next, regression is used to estimate \cref{eq:estimate probability of collision}.
To not rely on a fixed shape for which certain parameters are fitted (e.g., polynomial or logistic regression), non-parametric \cendb regression is used to estimate \cref{eq:estimate probability of collision}.
More specifically, we use the \ac{nw} kernel estimator \autocite{wasserman2006nonparametric}, \cstartb such that the approximation is guaranteed to give a number between 0 and 1\cendb:
\begin{equation}
	\label{eq:nadaraya watson}
	\probabilityestcond{\collision}{\situationcurrent}
	\approx \frac{ \sum_{\situationindex=1}^{\numberofdesignpoints}
		\kernelfuncnormalized{\bandwidthnw}{\situationcurrent - \situationcurrentinstance{\situationindexdesign}}
		\probabilityestcond{\collision}{\situationcurrentinstance{\situationindexdesign}}
	}{\sum_{\situationindex=1}^{\numberofdesignpoints}
		\kernelfuncnormalized{\bandwidthnw}{\situationcurrent - \situationcurrentinstance{\situationindexdesign}}}.
\end{equation}
Here, $\probabilityestcond{\collision}{\situationcurrentinstance{\situationindexdesign}}$ is based \cref{eq:estimate probability of collision} and $\kernelfuncnormalized{\bandwidthnw}{\cdot}$ represents the Gaussian kernel given by \cref{eq:kernel current future}.
Two important choices have to be made: The choice of the $\situationcurrentinstance{\situationindexdesign}$, $\situationindexdesign\in\{1,\ldots,\numberofdesignpoints\}$ for which to evaluate \cref{eq:estimate probability of collision} and the choice of the bandwidth matrix $\bandwidthnw$.
We suggest to base the design points $\situationcurrentinstance{\situationindexdesign}$, $\situationindexdesign\in\{1,\ldots,\numberofdesignpoints\}$ on the data that is used to estimate $\densitycond{\situationfuture}{\situationcurrent}$ in \cref{sec:estimate future}, i.e., $\situationcurrentinstance{\situationindex}$, $\situationindex \in \{1,\ldots,\situationnumberof\}$, such that all $\situationcurrentinstance{\situationindex}$ have at least one design point $\situationcurrentinstance{\situationindexdesign}$  nearby.
In other words, $\situationcurrentinstance{\situationindexdesign}$, $\situationindexdesign\in\{1,\ldots,\numberofdesignpoints\}$ are chosen such that
\begin{equation}
	\label{eq:design points distance}
	\min_{\situationindexdesign} 
	\left( \situationcurrentinstance{\situationindex} - \situationcurrentinstance{\situationindexdesign} \right)\transpose
	\weightmatrix 
	\left( \situationcurrentinstance{\situationindex} - \situationcurrentinstance{\situationindexdesign} \right)
	\leq \distancedesignpoints^2,
	\quad \forall \situationindex \in \{1, \ldots, \situationnumberof\},
\end{equation}
where $\weightmatrix$ denotes a weighting matrix and $\distancedesignpoints$ denotes the maximum ``distance''. 
Note that if $\weightmatrix$ is the identity matrix, then \cref{eq:design points distance} calculates the minimum squared Euclidean distance.
\cstarta Choosing $\weightmatrix$ is a trade-off: If $\weightmatrix$ is too large, then too many details are lost in the approximation of \cref{eq:nadaraya watson}.
If $\weightmatrix$ is too small, it takes too long to evaluate \cref{eq:estimate probability of collision} $\numberofdesignpoints$ times, as $\numberofdesignpoints$ increases with decreasing $\weightmatrix$. \cenda
The bandwidth matrix $\bandwidthnw$ might be based on $\weightmatrix$, e.g., $\bandwidthnw=\weightmatrix^{-1}$.
Alternatively, $\bandwidthnw$ might be based on the measurement uncertainty of $\situationcurrent$, where a larger $\bandwidthnw$ applies in case of a larger measurement uncertainty of $\situationcurrent$.


