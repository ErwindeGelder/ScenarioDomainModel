\section{Introduction}
\label{sec:introduction}

% Safety is important
Road safety is an important key performance indicator in transportation. 
Due to the enormous societal and economy losses incurred in road accidents, road safety research is an important research topic.
For example, in 2018\cstarta\footnote{\cstarta At the time of writing, more recent results were not yet available.\cenda}\cenda, there were over 6.7 million accidents in the U.S.A.\ \autocite{nhtsa2020summary}, which is about 1.3 accidents per 1 million vehicle kilometers driven.
These accidents in 2018 led to 2.7 million injured people and 37 thousand fatalities \autocite{nhtsa2020summary}.
Furthermore, apart from these societal losses, the economic costs of all accidents in the U.S.A.\ in 2018 was 242 billion dollars \autocite{nhtsa2020summary}.
Similarly, the \textcite{eu2020roadsafety} reported over 22 thousand fatalities in 2019.

% Quantify safety with surrogate metrics
Road safety can be expressed in terms of injuries, fatalities, or crashes per kilometer of driving, but \cstartb ``that is a slow, reactive process'' \autocite{arun2021systematic}.
Furthermore, ``crashes are rare events and historical crash data does not capture near crashes that are also critical for improving safety'' \autocite{wang2021review}.
An alternative for expressing road safety \cendb that does not rely on historical crash data is the use of safety indicators that directly measure the safety risk in traffic conflicts \cstartb \autocite{arun2021systematic, wang2021review, tarko2018surrogate}\cendb.
%Particularly challenging is that ``there is no yet consensus on what `driving safely' means'' \autocite{tejada2020safe}.
Traffic conflicts are far more frequent than traffic accidents and the frequency of traffic conflicts can be used to predict the frequency of crashes \cstartb \autocite{tarko2018estimating, davis2011outline}\cendb.
\cstartb To define traffic conflicts, thresholds on so-called \acp{ssm} are used, where \acp{ssm} characterize the risk of a collision or harm given an initial condition \autocite{arun2021systematic}. \cendb
% Examples
\acp{ssm} vary from measures that estimate the remaining time until a collision, such as the well-known \ac{ttc} \autocite{hayward1972near}, to metrics that estimate the probability that a human driver cannot avoid a collision, see, e.g., \autocite{wang2014evaluation}.

% Common approach
\cstartb \Acp{ssm} typically rely on assumptions of what the driver(s) or system(s) controlling the vehicle(s) of interest are capable of doing and how their future trajectories --- given an initial condition --- will develop. 
For example, \ac{ttc} \autocite{hayward1972near}, the ratio of the distance toward and the speed difference with an approaching object, is computed by assuming a constant relative velocity. 
As a result of these assumptions, \acp{ssm} are only applicable in certain types of scenarios.
For example, \ac{ttc} is only applicable when approaching an object. \cendb
More complex \acp{ssm} consider, e.g., a human model that can react to a risky situation by braking \autocite{wang2014evaluation} or the uncertainty over the future ambient traffic state \autocite{mullakkal2020probabilistic}.
Regardless of the complexity of these models, however, these \acp{ssm} consider neither the specific capabilities of the driver or of the system controlling the vehicle, nor the local context for predicting the future of the vehicle's environment.  

% Our proposal
\cstarta In this paper, we present the \ac{ourmethod} method: A novel data-driven approach for deriving \acp{ssm} that are not limited to certain types of scenarios.
Because our method does not rely on a predetermined \cenda\cstartb assumptions about driver behavior\cendb\cstarta, the derived \acp{ssm} can be adapted to the situations in which they are applied. 
In addition, to avoid relying on predetermined assumptions on how the ambient traffic evolves over time, the \ac{ourmethod} method includes a data-driven approach for modeling the variations. 
Monte Carlo simulations are employed to accurately predict the safety risk given these variations.
To enable the real-time evaluation of the derived \acp{ssm}, we use the \ac{nw} kernel estimator \autocite{wasserman2006nonparametric} for local regression.
% Advantages of our method
The \ac{ourmethod} method provides the following benefits: \cenda
\begin{itemize}
	\item The derived \acp{ssm} give a probability that, e.g., a collision will happen in the near future. 
	\cstartb Since a traffic conflict can be defined as the probability of unsuccessful evasion in a traffic interaction (according to \textcite{davis2011outline}), a probability is easier to interpret than, e.g., a value ranging from 0 to infinity. \cendb
	
	\item \cstarta Next to deriving new \acp{ssm}, \cenda it is possible to reproduce already existing measures that provide a probability. 
	Therefore, our approach can be seen as a generalization for deriving such existing \acp{ssm}.
	
	\item Any driver behavior model can be used.
	It is also possible to use a model of \iac{ads}, such that the derived \ac{ssm} estimates the safety risk if this \ac{ads} controls the vehicle.
	
	\item Because we use a data-driven approach, our \ac{ssm} adapts to the recorded data. 
	In this way, it is possible to \cstarta adapt the \ac{ssm} to, e.g., the local traffic behavior\cenda.
	
	\item Our approach can be applied to various scenarios.
\end{itemize}

% Explain case study.
\cstartb We illustrate the \ac{ourmethod} method and its benefits, by means of a case study. 
The case study demonstrates that when using our method with the assumptions of the \ac{ssm} of \textcite{wang2014evaluation}, both the derived \ac{ssm} and the latter yield the same result.
The case study continues with evaluating the collision risk of three longitudinal (in lane) traffic conflicts which are a priori known to be, respectively, safe (i.e., no collision possible), moderately safe and unsafe (i.e., collision occurs), based on vehicle kinematics. 
We evaluate the risk of each of the scenarios using the \ac{ssm} by \textcite{wang2014evaluation} and \iac{ssm} derived by the \ac{ourmethod} method, based on data from the \ac{ngsim}. 
Moreover, since comparison between these measures is not directly possible in general scenarios, we use the case study to also introduce a method to benchmark \acp{ssm} using expected risk tendencies. \cendb

% Structure
This article is organized as follows.
\cref{sec:literature review} provides an overview of \acp{ssm} described in the literature.
The proposed \ac{ourmethod} method is presented in \cref{sec:method}.
In \cref{sec:case study}, we illustrate the method in a case study.
The article is concluded in \cref{sec:conclusions}.
