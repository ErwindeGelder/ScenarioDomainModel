\documentclass[10pt,final,a4paper,oneside,onecolumn]{article}

%%==========================================================================
%% Packages
%%==========================================================================
\usepackage[a4paper,left=3.5cm,right=3.5cm,top=3cm,bottom=3cm]{geometry} %% change page layout; remove for IEEE paper format
\usepackage[T1]{fontenc}                        %% output font encoding for international characters (e.g., accented)
\usepackage[cmex10]{amsmath}                    %% math typesetting; consider using the [cmex10] option
\usepackage{amssymb}                            %% special (symbol) fonts for math typesetting
\usepackage{amsthm}                             %% theorem styles
\usepackage{dsfont}                             %% double stroke roman fonts: the real numbers R: $\mathds{R}$
\usepackage{mathrsfs}                           %% formal script fonts: the Laplace transform L: $\mathscr{L}$
\usepackage[pdftex]{graphicx}                   %% graphics control; use dvips for TeXify; use pdftex for PDFTeXify
\usepackage{array}                              %% array functionality (array, tabular)
\usepackage{upgreek}                            %% upright Greek letters; add the prefix 'up', e.g. \upphi
\usepackage{stfloats}                           %% improved handling of floats
\usepackage{multirow}                           %% cells spanning multiple rows in tables
%\usepackage{subfigure}                         %% subfigures and corresponding captions (for use with IEEEconf.cls)
\usepackage{subfig}                             %% subfigures (IEEEtran.cls: set caption=false)
\usepackage{fancyhdr}                           %% page headers and footers
\usepackage[official,left]{eurosym}             %% the euro symbol; command: \euro
\usepackage{appendix}                           %% appendix layout
\usepackage{xspace}                             %% add space after macro depending on context
\usepackage{verbatim}                           %% provides the comment environment
\usepackage[dutch,USenglish]{babel}             %% language support
\usepackage{wrapfig}                            %% wrapping text around figures
\usepackage{longtable}                          %% tables spanning multiple pages
\usepackage{pgfplots}                           %% support for TikZ figures (Matlab/Python)
\pgfplotsset{compat=1.14}						%% Run in backwards compatibility mode
\usepackage[breaklinks=true,hidelinks,          %% implement hyperlinks (dvips yields minor problems with breaklinks;
bookmarksnumbered=true]{hyperref}   %% IEEEtran: set bookmarks=false)
%\usepackage[hyphenbreaks]{breakurl}            %% allow line breaks in URLs (don't use with PDFTeX)
\usepackage[final]{pdfpages}                    %% Include other pdfs
\usepackage[capitalize]{cleveref}				%% Referensing to figures, equations, etc.
\usepackage{units}								%% Appropriate behavior of units


\pagestyle{fancy}                                       %% set page style
\fancyhf{}                                              %% clear all header & footer fields
\renewcommand*{\headrulewidth}{0pt}                  	%% No line in this case

\newcommand{\expectation}[1]{\textup{E} \left[ #1 \right]}
\newcommand{\mise}[2]{\textup{MISE}_{#1}\left( #2 \right)}
\newcommand{\amise}[2]{\textup{AMISE}_{#1} \left( #2 \right)}
\newcommand{\measure}[2]{J_{#1} \left( #2 \right)}
\newcommand*{\ud}{\mathrm{\,d}}

\usepackage{titlesec}
\titlespacing*{\paragraph}{0ex}{1ex}{1ex}
\newcommand{\toauthor}{\paragraph*{Comment to author:} \itshape}
\newcommand{\fromauthor}{\paragraph*{Reply:} \normalfont}
\newcommand{\additionend}[1]{\color{black}[#1]\color{red}}
\newcommand{\addition}[1]{\additionend{#1}\ }

\usepackage{changebar}
\newcommand{\cstart}{\cbstart\color{red}}
\newcommand{\cend}{\cbend\color{black}}

\begin{document}
	
\section*{Cover letter}

This cover letter describes how we have addressed the reviewers' comments for the paper title ``Have we collected enough field data''. Any additions to the paper are shown in \color{red}red\color{black}. Furthermore, a gray bar is shown next to the updated parts of the paper.
	
\section*{Review 1}

\toauthor The issue is relevant to a broad audience but the article is addressing a limited group because of the necessary and complicated maths. Consider to shorten parts of section IV on examples in favor of the discussion. 

\fromauthor The discussion is expanded, see the following comments for details. Furthermore, some text is added to the method section to help the reader interpreting the maths. At the end of Section~III-B, we added:

\cstart
In summary, the measure Eq.\ (8) \addition{equation of the measure of completeness} is an estimation of the MISE of Eq.\ (1) \addition{equation of the Mean Integrated Squared Error} given that the pdf is estimated using the KDE of Eq.\ (2) \additionend{equation of the Kernel Density Estimation}. Because the MISE cannot be directly evaluated, the asymptotic MISE is used with the estimated pdf substituted for the real pdf. 
\cend

In the first paragraph of Section~III-C, we added the following sentence to explain the consequence of assuming that parameters are independent:

\cstart
In that case, the joint distribution is not modeled using only one multivariate KDE, but using a combinations of KDEs.
\cend

We did not shorten the parts of section~IV, because we do not have an indication that the paper is too long.

\toauthor For instance, braking seems a relatively simple activity but the method does not allow for full stops.

\fromauthor A note is added to the first paragraph of Section IV-B:

\cstart
Note, however, that the method can be applied separately for the full stops. In fact, the analysis for full stops will be simpler, because a full stop activity can be parametrized using only two parameters because the end speed always equals zero. 
\cend

\toauthor chosen parameters are (clearly) dependent and thus the outcome shows that for an accurate completeness many samples are needed. 

\fromauthor In fact, the average deceleration is practically independent of the other parameters. This is seen by comparing the one-leave-out cross-validation scores when using one multivariate KDE and two KDEs. When two KDEs are used, the one-leave-out cross-validation score is better. We think it is a bit too technical to mention this in the paper, as we need to explain the cross-validation score that is used for the bandwidth selection. Furthermore, for illustrating the two methods, it is not really relevant. Hence, we addressed this by mentioning: 

\cstart
Note that the correlation between the average deceleration and the other parameters is fairly low, so this justifies this choice.
\cend

\toauthor But what sort of threshold is needed? Why 0.003? 

\fromauthor This is addressed in the second paragraph of the discussion:

\cstart 
Furthermore, the threshold depends on the number of parameters for one activity, denoted by $d$ in Section~III. Based on experience with the dataset used in Section~IV, assuming that the dataset is normalized such that the standard deviation equals one, a threshold between 0.01 and 0.001 gives good results. When a threshold of 0.01 is reached, a reasonable reliable pdf can be constructed to analyze nominal driving behavior, whereas a threshold of around 0.001 is required to also accurately analyze the edge cases.
\cend

It is also mentioned as future work in the conclusion:

\cstart \addition{...} and we will investigate the appropriate thresholds for the measure to quantify completeness in different applications. \cend

\toauthor Any idea (rough estimate) how many hours ND is enough for a `naturalistic' driver behavior model that will improve micro-simulations drastically or what sort of activities are likely to be modeled (i.e.\ complete) accurately because they happen a lot (like braking), don't need many (or complex) parameters and which activities are rare, diverse and therefore need large ND mileages? Maybe it is far too early to tell, but that would be an answer as well. 

\fromauthor Some text is added to the discussion's last paragraph to elaborate on this:

\cstart 
The advantage of using the activities for determining the completeness is that there is only a limited number of types of activities. As a result, for each type of activity, it is expected that there is no need for an extremely large dataset to obtain a fair number of similar activities. On the other hand, however, it is not known how much data is required to obtain the desired threshold, because, e.g., this depends on the parametrization that is chosen. 
\cend 

\section*{Review 2}

\toauthor This is an interesting paper on statistical methodology. If I understand it correctly, it is similar to the usual sample size formulae necessary for given precision of parameter estimates or power of statistical tests -- with the extension to a desired description of the 'complete' probability distribution. Typically, extensions of methodology to more complicated situations (e.g., multivariate vs. univariate) are justified by showing that the general formulae simplify to the more familiar form in special cases. Is that the case for this methodology? Suppose the pdf was Bernoulli? 

\fromauthor In the current paper, it is assumed that the probability density function (pdf) is estimated using kernel density estimation (KDE), regardless of the underlying distribution. Hence, even if the underlying distribution is as simple as, for example, a uniform distribution (let us assume a continuous distribution), the same formulas apply. The only way to simplify the measure for completeness is by assuming a univariate distribution. In that case, 
\begin{equation}
	\measure{f}{n} = \frac{h^4}{4} \sigma_K^4 \int_{\mathbb{R}^d} \left( \nabla^2 \hat{f}(x;n) \right)^2 \ud x + \frac{\mu_K}{nh^d}
\end{equation}
simplifies to
\begin{equation}
	\measure{f}{n} = \frac{h^4}{4} \sigma_K^4 \int_{-\infty}^{\infty} \left( \frac{\ud^2 f(x)}{\ud x^2} \right)^2 \ud x + \frac{\mu_K}{nh}.
\end{equation}
As this is not considered as a major simplification, this is omitted from the paper.

\toauthor The title of the paper is a bit misleading, as you state in the discussion that the desired threshold for completeness will require more application-specific research. A better title would be `How do we determine if we've collected enough field data?'

\fromauthor We changed the title as suggested while adding some context information. The new title is:

\cstart
Safety assessment of automated vehicles: how to determine whether we have collected enough field data?
\cend

\section*{Review 3}

\toauthor It's a very interesting concept. You are correct in that the notion of ``enough'' data relies very heavily upon the goal of the use of the data. For example, the crash scenarios/typology may be sufficient for proving safety equivalence or passing a minimum set of regulations, but when designing systems to consider edge cases, is there ever really ``enough'' data? 

\fromauthor A note is added to the second paragraph of the discussion (Section~V) to explain that only infinite data results in the measure for completeness to approach zero:

\cstart Only in case of an infinite set of data, the measure for completeness approaches zero, so this threshold needs to be larger than zero. \cend

\section*{Review 4}

\toauthor In the paper, the author proposed a method to decide whether enough field data is collected. The topic is very important, and the paper was well constructed. The paper can be further improved by considering the following:

1. The basis of the paper is built on the ``scenario''. It will make the paper more complete if the concept and the way to obtain scenario is first discussed - either defined directly by experts or extracted through unsupervised learning.

\fromauthor Few examples with references are given when introducing the concept of scenario (Section~II):

\cstart 
Extracting scenarios from data received significant attention and the applied methods are very diverse. For example, Elrofai~et~al.\ (2016) use a model-based approach to detect scenarios in which the ego vehicle is changing lane, whereas Kasper~et~al.\ (2011) use Bayesian networks to detect scenarios with lane changes of other vehicles around the ego vehicle. Xie~et~al.\ (2018) use a random forest classifier for extracting various scenarios and Paardekooper~et~al.\ (2019) employ rule-based algorithms for scenario extraction. 
\cend

\toauthor 2. How to justify the example given in this paper is a good representation of the proposed method? How sensitive is the conclusion to different criteria? More statistical analysis is appreciated.

\fromauthor A paragraph is added to the discussion to elaborate on some considerations when employing our measure for completeness:

\cstart
When using our measure for completeness, the following might be considered. As explained in Section~III, the measure for completeness is based on the AMISE. It is also mentioned that the AMISE only differs from the MISE by higher-order terms under some mild conditions\footnote{The pdf $f(\cdot)$ needs to comply with the regularity conditions, $K(u) \geq 0, \forall u$, $\int_{\mathbb{R}^d} K(u) \ud u = 1$ and $\sigma_K=\int_{\mathbb{R}^d} \|u\|^2 K(u) \ud u$ from Eq.~(5) is not infinite.}. This requires that the real pdf is smooth, i.e., without large spikes (Marron and Wand 1992). Marron and Wand (1992) also state that the AMISE is strictly higher than the MISE under some mild conditions\footnote{The Laplacian of $f(\cdot)$ needs to be continuous and square-integrable and $K(u) \geq 0, \forall u$.}. As a result, it is likely that the measure for completeness, which is an approximation of the AMISE, is higher than the MISE. This could lead to an overestimation of the number of required samples.
\cend


\end{document}