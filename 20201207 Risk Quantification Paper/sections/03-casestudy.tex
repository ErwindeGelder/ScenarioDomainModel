\section{Case study}
\label{sec:case study}



\subsection{Driver model}
\label{sec:driver model}

For the driver model, the \ac{idmplus} from \autocite{schakel2010effects} is used. 
\ac{idmplus} itself is based on the \ac{idm} from \autocite{treiber2000congested}.
To mimic the ``measurement noise'' of human drivers, the \ac{hdm} of \autocite{treiber2006delays} is used.
To model the longitudinal behavior, the \ac{acc} of \autocite{xiao2017realistic} is used. 
The driver will take over after \iac{fcw} is issued.
The \ac{fcw} is modelled according to \autocite{kiefer2005developing}.
As proposed by \autocite{xiao2017realistic}, the driver also takes over control if the approaching speed difference of the subject vehicle towards its leading vehicle is larger than \SI{15}{\meter\per\second} while the leading vehicle is within its perception range of \SI{150}{\meter}.



\subsection{Scenarios}
\label{sec:scenarios}

The following scenario categories are considered:
\begin{itemize}
    \item lead vehicle decelerating;
    \item approaching slower vehicle;
    \item cut-in; and
    \item cut-out.
\end{itemize}

\todo{Explain how the scenarios are parameterized.}

To obtain data, the data set described in \autocite{paardekooper2019dataset6000km} is used. We observed 1310 lead-vehicle-decelerating scenarios, 655 approaching-slower-vehicle scenaios, 298 cut-in scenarios, and XXX cut-out scenarios.
