%%%%%%%%%%%%%%%%%%%%%%%%%%%%%%%%%%%%%%%%%%%%%%%%%%%%%%%%%%%%%%%%%%%%%%%%%%%%%%%%
\section{Introduction}
\label{sec:introduction}
%%%%%%%%%%%%%%%%%%%%%%%%%%%%%%%%%%%%%%%%%%%%%%%%%%%%%%%%%%%%%%%%%%%%%%%%%%%%%%%%

%%%%%%%%%%%%%%%%%%%%
% Future Vision
%%%%%%%%%%%%%%%%%%%%

AVs have a very important role to play in future mobility systems. There are plenty of very important reasons for that. Here are some. Highly automated AVs will be fundamentally different to current existing vehicles as they are a combination of two distinct entities: a vehicle in the traditional sense and an artificial driver.\\

%%%%%%%%%%%%%%%%%%%%
% Problem
%%%%%%%%%%%%%%%%%%%%

To successfully introduce AVs on the roads, car manufactures will have to produce evidence to the public and regulators that proves that (1) the vehicle technology is safe in the current sense (functional safety and SOTIF) and (2) that the artificial driver drives safely in the context where the vehicle operates. Moreover, in contrast with current type approval practice, where vehicles are assessed once before deployment, future highly automated AVs will have to produce evidence of their safety in a periodic or even continuous fashion, owning to changes on their software and algorithms over their life-cycle.\\

The continuous monitoring and re-certification of a vehicle's safe driving behavior, a concept called Monitored Deployment, will rely on clear, measurable definitions of safe driving behavior that can be both used as requirements for the development of self-driving functionalities and to monitor deployed AVs.

%%%%%%%%%%%%%%%%%%%%
% Focus
%%%%%%%%%%%%%%%%%%%%

There is not yet a consensus interpretation of what driving safely means, especially in the context of AVs. The dominant emphasis in the literature on traffic safety is on the so-called severe traffic conflicts, situations in traffic that if not corrected lead to collisions and even casualties. The emphasis on critical moments in traffic is driven by the need to understand their causes and develop measures to prevent them. \\

However, statistically speaking the vast majority of driving interactions (by human drivers) are not critical (see evidence). That is, they carry medium to low severity and do not end up in collisions. Consequently, it is to be expected that AVs would also be exposed mostly to such non-critical interactions. Moreover, it is also reasonable to expect that if AVs interact in traffic with other road users in a predictable and recognizable way, the current levels of traffic interaction severity would be preserved.\\

Hence, in this paper we propose to use human driving as a reference from which to define "safe driving behavior" and create a method to ascertain whether an AV drives safely in support of the Monitored Deployment concept. 


%%%%%%%%%%%%%%%%%%%%
% Approach
%%%%%%%%%%%%%%%%%%%%

Our approach relies on the  hypothesis that current human driving carries low to medium severity and is, therefore, safe. Hence we describe here a method to characterize the "most typical" human driving interactions from naturalistic data sets. Based on the concept of traffic interactions introduced by Hyden and Svensson,  we use a ML approach to model typical longitudinal driving interaction in highway settings. The resulting ML algorithm is able to determine the degree "typicality" of any new driving interaction applied to it. We then ascribe a level of severity to each interaction based on a new  data-driven severity metric derived from WS. Our result show that typical driving indeed carries lower severity than non-typical one, providing support for our hypothesis. Moreover, by choosing an appropriate threshold on the level of "typicality" the ML algorithm can be used in the setting of monitored deployment to assess whether a particular AV drives safely.

%%%%%%%%%%%%%%%%%%%%
% Paper Organisation
%%%%%%%%%%%%%%%%%%%%

\begin{figure}
\includegraphics[width=1.0\linewidth]{figs/interactions}
\caption{caption}
\label{fig:interactions}
\end{figure}

Figure \ref{fig:interactions}, Tejada et al. \cite{tejada2020safe}

\begin{figure}
\begin{tikzpicture}[scale=1.0]
\draw(0,0) -- (10,0);
\end{tikzpicture}
\caption{caption}
\end{figure}